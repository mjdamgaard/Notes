\documentclass{report}
%\usepackage{style}
\usepackage[utf8]{inputenc}


\usepackage{amssymb}
\usepackage{comment}
\usepackage{amsmath}
\usepackage{graphicx}
\usepackage{braket}
\usepackage{listings}
\lstset{
	basicstyle=\ttfamily\footnotesize,
	literate={~} {$\sim$}{1}
}

\usepackage{latexsym}

\usepackage{siunitx}
\usepackage{textcomp}

% \usepackage{mathabx}

\usepackage{lipsum}



\usepackage[margin=1.4in]{geometry}
%\usepackage[margin=1.5in]{geometry}

\title{Note collection (2021--2022)} %Slutdato: 20/02-22
\author{Mads Juul Damgaard}
\date{\today}

\begin{document}
\maketitle



{\small\slshape%\itshape %\centering
Jeg skriver for det meste på dansk, men skriver på engelsk, når jeg vil skrive noget mere formelt. Jeg tænker at de engelske sektioner gerne skal være opsamlende og velstrukturerede nok til, at man kan læse dem for sig. Her vil jeg sandsynligvis også rette sektionerne mere til, men generelt vil jeg ikke bekymre mig så meget om at skrive korrekt og rette til osv. *(Nej, sådan bliver det ikke helt alligevel. Så forskellen er i stedet bare, at jeg for nogen sektioner vil foretrække at skrive på engelsk, og det er så især sektioner, hvor jeg føler behov for at prøve at formulere mig lidt mere præcist (hvor mine danske noter nok som regel vil være lidt mere løst for hoften --- dog uden at sige, at det engelske ikke også godt kan være ret meget fra hoften og uden gennemrettelser). Det ender nok stadigt ud med, at de engelske sektioner bliver en anelse mere ``vigtige,'' dvs. genlæsningsværdige, end de danske, men nu må vi se.)

Jeg vil give datoer i begyndelsen af hver sektion, og hvis en paragraf indeholder en ny idé eller indsigt vil jeg også skrive datoen. Paragrafer uden en dato og uden en stjerne foran hører til foregående paragraf. Jeg kan godt gøre ophold og så færdiggøre en række paragrafer en af de efterfølgende dage uden at notere dette med en ny dato, hvis opholdet ikke skyldes en tænke-pause, og det derfor ikke betyder noget ift.\ hvornår jeg fandt på tingene. Langt de fleste ting jeg skriver her vil nok være omkring gamle idéer, men jeg vil alligevel notere den nuværende data (for at lette på (u-)overskueligheden). En stjerne foran en paragraf eller en indskudt sætning/tekst betyder og det er en senere tilføjelse. For disse vil jeg nok som regel undlade at skrive datoen for rettelsen/indsættelsen. 

Jeg holder backups, så det kan godt være, at jeg sletter eller udkommenterer sektioner, hvis jeg ikke længere føler at de er særligt belysende.
}
{\\\centering\noindent
	\vspace{-\baselineskip}
	\hspace{-2.5em}
	{$|$\hspace{\linewidth}\hspace{2.5em}$|$}
}





\part{General notes}
\chapter{General notes}

\section{Summary/brainstorm of all the different subjects I need to write about. (17.03.21--20.10.21)}
Since I just intend to make a collective ``summary/brainstorm'' (which will be much more of a summary than a brainstorm, but which will be written quickly; almost sort of like a brainstorm) here over basically all my ideas (except possibly a few that I forget), I will just write them here under `General notes.' When I want to write more, an perhaps more formally, about the same subjects again in their designated chapters/sections, I can just use this text as a jumping-off point, and potentially make references to it as well. Since these are meant as quickly written notes, I will write them mainly in Danish (since this tends to give me the quickest writing flow, naturally).


*(18.10.21) Start ikke nødvendigvis med at læse dette notesæt fra ende til anden, men se først, hvad jeg har skrevet i opsummeringssektionen (næsten) nederst i sættet (hvilket i øvrigt er en kopi fra min ``Opsummering af idéer \ldots''-sektion fra midten af notesættet, så man kan altså også starte der).

*\textit{(22.02.22) Jeg afslutter hele dokumentet i dag. De to ``opfølgende'' sektioner efter denne ``summary''-sektion er også værd at læse. (Og det er rigtigt meget i denne sektion bestemt også; pointen var bare, at man nok med fordel kunne starte i ``Opsummering af de idéer, \ldots''-undersektionen i stedet for bare at prøve at læse det fra ende til anden.\,:)) I den efterfølgende ``Draft to first \ldots''-sektion kan man også læse rigtigt mange vigtige (synes jeg) ting, bl.a.\ også hvad mine planer er fremadrettet, nu jeg afslutter dette dokument, men desværre står mange af disse ting godt nok i kommentarerne (i tex-kildeteksten), for jeg har ikke haft tid til at skrive dem ind her i den renderede tekst. Jeg kan forresten lige bemærke, hvis ikke jeg har gjort det før, at når jeg har en sektion/paragraf, jeg egentligt synes er vigtig nok, så har jeg gerne sørget for at skrive ``ude i kommentarerne'' som en slags nøglefrase, man kan søge på i dokumentet. Men ellers kan jeg måske også anbefale bare at søge på datoanmærkningerne (inkl.\ i kommentarerne) og så læse lidt fra modsat ende af (fra de nyeste af), hvis man gerne vil danne sig et fuldt billede af, hvor jeg står nu i skrivende stund med alle mine idéer.}

\subsection{Min ITP-idé}
Jeg har allerede (pt.) skrevet en del om min ITP idé nedenfor\footnote{Se sektion \ref{ITP_feb-maj}.}, og har som nævnt i overskriften af dette notesæt lige kommet frem til en ny idé herom, som bl.a.\ går ud på at undlade at forsøge at gøre IL-sproget sikkert og simpelt, således at det kun opererer på sikre abstraktioner over filer og mapper i stedet for fysiske filer og mapper på maskinen. Det var ellers, hvad jeg i udgangspunktet lagde op til med disse noter. Men i stedet er jeg nu gået over til at mene, at man bør gøre IL-sproget så kraftfuldt, at det både er i stand til at beskrive og analysere selve ITP'en selv, og at det også ligefrem kan ``se'' og lave operationer på selve mappen, hvori ITP-programmet og dets tilhørende filer ligger (samt selvfølgelig også alle andre mapper som ITP'en råder over). Pointen er så som nævnt i noterne (idet jeg påklistrede mine nye idéer på disse noter (uden at gøre dem til en del af den resterende brødtekst, som jeg så bare afsluttede, før den var helt færdig)), at man så udformer nogle `meta-propositioner' som man så erklærer som antagelser for ITP-mapperne. Idet man kan vise, at programmer ikke vil bryde den invariant, der ligger i disse antagelser, kan man så få lov til at tilføje dette program til hovedprogrammet. Hovedprogrammet er så samtidigt ikke længere bare det ITP-program man starter med, men er i stedet for bare en slags launcher over de godkendte programmer, inklusiv det første ITP-program. Selvfølgelig bør hovedprogrammet for god ordens skyld have en reference til listen over (de grundlæggende) meta-antagelser, men dette program skal ikke selv kunne ``tolke'' dem eller; det kan sagtens være en ``dum'' launcher-applikation. Samtidigt behøver det heller ikke være en del af meta-antagelserne, at brugeren ikke må launche og køre de gyldige programmer uden om hovedprogrammet. Det ville tvært imod være at foretrække at det fremgår af meta-antagelserne, at brugeren godt selv til hver en tid må køre de gyldige programmer, og så skal der bare antages et godt system, så brugeren nemt kan se hvilke programmer er gyldige at køre selv, så vedkommende ikke kommer til at bryde sine egne meta-antagelser. Man kunne f.eks. have med i antagelserne, at alle programmer, der kan eksekveres fra en vis mappe, godt må dette. Meta-antagelserne beskriver nemlig ikke bare nogle restriktioner for hvad de gyldige programmer må gøre, men er også en kontrakt for, hvad brugeren må foretage af ændringer i mapperne, inklusiv hvilke programmer, der må køres fra disse. Grunden til at jeg lavede dette skift, kan jeg lige nævne, var i øvrigt, at jeg alligevel kom frem til, at jeg skulle benytte sådanne meta-antagelser for min ITP-løsning, og så åbnede dette alternativ til løsningen sig ligesom op naturligt derfra (for hvis man alligevel skal have disse meta-antagelser omkring de fysiske ITP-mapper, så bør man også helst kunne beskrive og analysere disse i selve ITP'en, og hvis man alligevel for denne magt, så er det jo et lille skridt til denne nye idé, hvor ITP'erne kan beskrive og analysere hinanden og dermed bruge hinandens resultater). Hvor ligger så ansvaret for at ``tolke'' og tjekke meta-antagelserne, hvis det ikke ligger hos hovedprogrammet selv? Jo, det ligger i udgangspunktet hos den første ITP, som hovedprogrammet starter med (hvis den altså ikke starter med flere end én). Så en del af de første meta-antagelser bør dermed være, at en vis ITP er et gyldigt program, og denne ITP skal så være kraftig nok (og skal selvfølgelig også være sound (dvs.\ baseret på et fornuftigt deduktivt system med en klar (standard) semantik og uden modstrid)) til at analysere meta-antagelserne og skal så altså sørge for selv at overholde dem. Pointen er så, at med dette behøver vi ikke længere at tænke særligt meget over, hvad det grundlæggende fundament (i.e.\ deduktive system) skal være for ITP'en. For enhver ITP, der kan analysere sig selv såvel som alle andre tilsvarende ITP'er, kan jo på en måde ``se sig selv'' som den oprindelige ITP, også selvom den i virkeligheden ikke er det, men er initialiseret af enten den første ITP, brugeren erklærede gyldig i meta-antagelserne, eller en af dennes børne-ITP'er. Med dette begreb, `børne-ITP'er,' mener jeg de ITP-programmer/-applikationer, som er vist gyldige via det, vi så kan kalde forælder-ITP'en. Og fordi børne-ITP'er ikke nødvendigvis får lov at råde over alle ITP-mapperne, men måske kun nogle undermapper, så er det ikke sikkert at børne-ITP'erne kan ``se,'' at de er børne-ITP'er selv. Men kan nemlig endda også sagtens lægge et program ind, der svarer til hovedprogrammet, komplet med et nyt sæt af meta-antagelser, som også skal overholdes. Og på denne måde kan ingen af disse ITP-program-systemer nogensinde ``vide,'' om de er det ydre ITP-system, eller om de er en del af et yderliggende system. Dette gør, at hvis man nu udvikler et mere simpelt og/eller et mere effektivt ITP-fundament, så kan man meget nemt rulle dette ud, uden at det kommer til at forstyrre de brugere, der af en eller anden grund holder sig til det gamle fundament. For hvis brugere af det nye fundament kan bare bruge et bevis for gyldigheden af det gamle fundament til at køre en sådan ITP på beviser, udformet med ITP'er af det gamle fundament. Så for at rulle et nyt fundament ud skal man altså bare vise gyldigheden af det gamle fundament i det nye. Og hvis brugere af det gamle system gerne vil bruge beviser udformet med det nye (hvis semantikken og refleksionsmulighederne er kraftfulde nok til at beskrive/benytte det nye system), kan de tilsvarende også bare vise korrektheden af det nye fundament i det gamle. Så på den måde bliver det nemt for brugerne at beholde deres egne præferencer, når de konstruerer beviser, uden at det forhindre dem i at dele beviser med hinanden og køre disse. Jeg bør lige nævne, at det bør være en central antagelse, at ITP-programmerne altid gemmer de bevisopskrifter --- det jeg har kaldt `proof scripts' indtil videre i mine noter --- som brugeren udformer. Formatet for disse opskrifter kan så være helt frit for, og hvis man så vil dele sine beviser med en anden bruger, der bruger en anderledes ITP som sin foretrukne, kan man så enten bare vedlægge sit eget bevis for sin egen ITP (givet at man enten bruger samme fundament eller har beviser for korrektheden af hinandens), eller også kan man simpelthen, hvis der lige er gjort lidt mere forarbejde, bare bruge at program til konvertere bevisopskrifterne (hvor hver brugere så bare kan sørge for at have konversionsprogrammer mellem deres foretrukne formater og et vist konventionelt standardformat). *Nå ja, og angående at rulle nye fundamenter ud: Hvis disse stadig kan beskrives af de gamle fundamenter, så kan brugere selvfølgelig også bare adoptere de nye systemer ved at oprette dem som børne-ITP-systemer. De behøver altså ikke at vende om på strukturen i deres ITP-mappe-system, så det nye fundament ligger yderst, men kan i stedet placere det i en undermappe og bruge det derfra (efter at de påkrævede gyldighedsbeviser er kørt).

I tillæg til `børne-' og `forælder-' ITP'er så er det også værd at nævne, at man sagtens kan have flere ITP'er i brug, som ens hovedprogram (bl.a.) kan launche, som hver især kan beskrive og analysere korrektheden af hinanden. Det er derfor at jeg snakker om en `første ITP' og ikke en `kerne-ITP.' Man kan nemlig sagtens generere børne-ITP'er (hvis vi så overhovedet skal kalde dem det i det tilfælde), som råder over samme mapper som forælder-ITP'en, og som er på hel lige fod med denne. Sådanne ITP'er kan så på en måde ``glemme,'' hvilken en af dem, der var den originale. Man kan således omfortolke sådanne ITP'er og tænke på dem som `søskende-ITP'er.' 

Hvor avanceret eller hvor simpel skal den første ITP så være? Jamen den kan så netop være ligeså avanceret/simpel, som man har lyst til. Det kunne måske være en idé både at sarte med en simpel og en avanceret ITP, så man kan vise korrektheden af sidstnævnte med førstnævnte, men på den anden side kan man bare inkludere flere forskellige bevisverifikationsmoduler i sidstnævnte, navnligt et simpelt et også. Pointen med et simpelt verifikationsmodul er nemlig bare, at så er der færre potentielle fejlkilder ved verifikationerne. Men hvis man så bare adskiller verifikationsmodulerne i en avanceret ITP, så vil man også nemt kunne vise, at de fejlkilder heller ikke er til stede der. Nå, men hvad skal en ``avanceret'' ITP så kunne, tænker jeg? Jo, for det første skal en god ITP ikke bare kunne vise gyldige programmer, som nævnt, som kan launches og køre selvstændigt i ITP-systemet, men den skal også kunne lave DLL-biblioteker til sig selv. (Nå ja, og hvis dette er lidt kompliceret, kunne man også bare nøjes med en interpreter i tidlige versioner af en sådan ITP.) Jeg har så skrevet om, her i noterne i sektion \ref{ITP_feb-maj}, hvordan dette så kan bruges til at oprette hurtige omskrivningsregler til at manipulere matematiske objekter (ved at man sørger for at have dem som binære repræsentationer). Jeg gik så endda videre til at foreslå, at man kunne implementere børne-ITP'er via sådanne underprocedurer (altså i form af disse `refleksionsprogrammer,' som jeg kaldte dem i de noter). Nu er dette så ikke længere nødvendigt, fordi man nu, som nævnt, skal kunne vise korrektheden af og benytte selvstændige programmer, men derfor kan det jo stadig sagtens beholdes som en mulighed. Det kræver nemlig bare, at man kobler et renderingsmodul til, og sørger for at refleksionsprogrammerne kan være interaktive, hvad man bør gøre alligevel. I øvrigt, når det kommer til renderingsmoduler, og når det kommer til I/O generelt, så behøver man ikke at beskrive semantikken omkring dette fuldt ud; man behøver f.eks.\ ikke beskrive, hvordan ting skal se ud på den fysiske skærm, eller hvilke nogle fysiske knapper, brugeren skal trykke på. Man kan selvfølgelig bare koncentrere sig om interfaces'ne ud til den virkelige verden (så som et vist markup sprog til rendering, eller en vist binært format, når det kommer til f.eks.\ keyboardinput), og så glemme hvad der sker på den anden side af de interfaces (i den virkelige verden). Hvis man så f.eks. har lyst til at vise noget om renderingen som eksempel, så kan man bare selv konstruere en model for, hvordan man antager, at output fører til renderede objekter på en skærm, og så vise de pågældende ting, man ønsker, under denne antagelse. Nå, det var lige en hurtig sidenote. Men ja, når det kommer til et forslag til en faktisk ITP, så tror jeg stadig at den type ITP, jeg var ved at beskrive i nævnte noter, langt hen ad vejen ville være en, jeg ville holde fast i at foreslå. Jeg synes også det med løbende omskrivninger af `proof scriptet' (f.eks. til lige efter, man har kørt et interaktivt program) var en smart ting, og selvfølgelig var gendannelsesprocedurerne også vigtige. Nu går jeg som nævnt sidst i samme noter dog ind for, at programmerne bare skal ses som ikke-deterministiske på en måde, som en god måde bl.a.\ at få I/O og interaktivitet ind i billedet (og hvad var det nu mere?.. gendannelse og error handling var det vist..). Bemærk som en vigtig ting, at jeg også med mine refleksionregler, som jeg har tænkt dem, også går ind for, at verifikationsmodulet ligeså vel kan køre de samme refleksionsprogrammer, som brugeren kører under beviskonstruktionen. Nu kan det egentligt godt være, at dette slet ikke er kontroversielt på nogen måde alligevel, for mon ikke også det er sådan med gængse ITP'er? Det skal jeg måske lige læse op på, men uanset hvad, er der gode grunde til det. Jeg har diskuteret dette i sektion \ref{ITP_feb-maj}-noterne, men det er egentligt et ret simpelt spørgsmål, for der er ikke rigtigt nogen gode modargumenter. For hvis man gerne vil have sit verifikationsmodul til at bruge længere til på at verificere et bevis, imod at man så kan stole på det mere, så kan man sagtens bare gøre dette. Det kræver bare lige, at man kan omskrive beviserne, men det kan man bare bygge en transpiler til. Og med den måde, jeg har i sinde at bruge certifikater, så kan man udforme specielle certifikater, som kun gives, hvis beviser er kørt med et simpelt verifikationsmodul. Så dermed gør det ikke noget, at brugere kan benytte forskellige verifikationsmoduler med forskellige grader af sikkerhed i deres vante arbejde: når det endeligt kommer til at stole på andre brugeres beviser, så er det meningen at man skal benytte en meget formel proces (hvor man bruger disse `certifikater'), og så kan man sagtens implementere en formel standard, hvor sætninger vægtes som mere pålidelige, hvis de er blevet verificerede med visse v-moduler frem for andre.

Mine tanker om at bruge `certifikater' handler bare om i korte træk, at brugerfælleskabet kan udforme certifikater, f.eks.\ fra troværdige grupper af brugere, som fortæller at vise propositioner er beviste, så at brugeren ikke selv behøver at downloade og køre de samme beviser. Brugeren kan så i stedet bare inkludere det som en meta-antagelse, og hvis der forekommer filer i visse af ITP-mapperne (f.eks.\ dem alle sammen), som indeholder et propositionsbibliotek samt et certifikat af en given type, der godkender disse propositioner til et vist niveau, jamen så kan disse propositioner noteres som sande, eller muligvis som `sandsynligvis sande,' hvorved de så kan bruges med forbehold. Angående at bruge propositioner med forbehold, så kunne det f.eks.\ indebære, at man ikke giver sig selv lov til at bruge dem til, at vise korrekthed af programmer. Naturligvis bør disse forbehold så implementeres ved, at alle de viste sætninger, der er vist med propositioner, der har påklistret visse forbeholdsflag, skal selv få passende forbeholdsflag klistret på sig. Man kan i øvrigt se disse forbeholdsflag som en slags (meta-)antecedenter, hvilket er grunden til, at jeg har kaldt dem fodnote-antecedenter i nævnte tidligere noter (nemlig fordi man kan se dem som (meta-)antecedenter, men evt.\ bare kan rendere dem som flag/fodnoter). Tanken er så også, at disse propositioner med forbehold også kan inkludere propositioner eller metoder, som man ikke helt har bevist korrektheden af endnu. I mit visions-afsnit i sektion \ref{ITP_feb-maj} har jeg beskrevet, hvor vigtigt jeg tror dette er for at løfte ITP'er op på niveau med de konventionelle metoder, som matematikere benytter, og muligvis højere endnu, fordi man så får digitale værktøjer til hjælp for selv svær bevisførsel, men også særligt fordi man gør hele processen lettere at blive en del af. Og dette er en generel fordel, jeg ser ved at formalisere selv abstrakte processer, i videnskabssamfund og i vores samfund generelt, som jeg ser/forudsiger det. Dette, og mere, har jeg også skrevet om i nævnte visionsafsnit. Angående hvad jeg mener med `sandsynligvis sande,' så kan der både være tale om faktiske sandsynlighedsvurderinger af propositioner eller også bare om et vilkårligt pointsystem, hvor propositioner kan få forskellige point sammensætninger (og hvor point kan være både diskrete aller kontinuerte). Og når det så kommer til at benytte dette pointsystem, så handler det først og fremmest om at cleare propositioner til brug i sensitive beviser så som til at initialisere programmer eller refleksionsregler eller til at omskrive filer m.m. Dertil kan man også selv opfinde måder at reducere disse point-(/fodnote-)antecedenter på, ved at lave nogle brede point-kategorier (så man kan lave point-antecedenterne mindre verbose), hvor man så altså omdanner antecedenten til: ``hvis alle antecedenter med pointsammensætninger inden for pointkategori $x$ er sande, så \ldots'' Når det kommer til point, der repræsenterer sandsynligheder, så skal man selvfølgelig lige passe på, at man ikke har korrelationer til stede i, hvad man kan se som akserne i sit sandsynlighedsrum. Så det skal man altså lige tages højde for, når man formulerer de pågældende meta-antagelser. Korrelationer man således skal passe på, er f.eks., hvis samme bruger/instans har givet svar, der giver anledning til forskellige point, men hvor disse point så naturligvis vil være korreleret uafhængigt af det faktiske svar på spørgsmålet. Det kan også være, hvis to tilsyneladende forskellige spørgsmål i virkeligheden er korrelerede (så at folk giver korrelerede svar på de to spørgsmål uafhængigt sandheden omkring de underliggende forhold), eller det kan være, hvis man kun primært for spurgt folk med den samme bias, således at deres svar i denne gruppe vil være korrelerede uafhængigt af sandheden omkring de underliggende parametre. 

Jeg skal også nævne, at måden ITP'erne (i brugerens ITP-system) gemmer sætninger på, simpelthen også bare er beskrevet i meta-antagelserne. Man kan så gøre sine meta-antagelser åbne for, at ITP'erne også kan erklære deres egne måder at gemme sætninger på. Og dette kan man også sige om andre ting angående meta-antagelserne; på mange måder kan de godt være åbne for, at man definerer formater, certifikater, programmeringssprog, osv., ligesom at de jo også som nævnt bør være åbne for, at man kan erklære helt nye ITP'er. Jeg tror, det ville være en rigtig god idé, hvis man fokuserer nye biblioteksformater omkring, det `opslagsprogram,' der skal hive dem frem, og jeg tror så, det ville være rigtigt smart, hvis disse opslagsprogrammer for det første er query-baserede, og at bevisopskrifter ikke behøver at fejle nødvendigvis, hvis opslagsprogrammet returnerer `ikke bevist.' Det vil nemlig forøge letheden, hvormed brugere kan dele beviser; hvis en bruger ikke lige har den pågældende proposition, så kan bevisverifikationen stadig køre færdig --- hvis altså ikke propositionerne bruges til noget vitalt så som dynamisk at initialisere programmer (hvis man tillader dette for beviset) --- og de manglende propositioner kan bare hægtes på til slut som antecedenter. Pointen er så, at brugere så bare kan slutte sig til internettet bagefter, og query'e propositionen der, downloade et bevis for det (som muligvis bare kan bestå af et certifikat-tjek), og bruge de på at fjerne de pågældende (påhægtede) antecedenter fra de gemte sætninger. Dermed selvfølgelig ikke sagt, at ITP-programmerne ikke skal have mulighed for at query'e internettet selv undervejs i programeksekveringen, det må de godt kunne. Dette kan så bare ses som en særlig form for I/O, der bare skal være lidt mere beskyttet, så brugere ikke kommer til at køre andre brugeres beviser, der så pludselig kommunikerer med internettet på en ikke-hensigtsmæssig måde fra brugerens synspunkt (og dette kunne slev være noget så banalt som at afsløre, via de efterspurgte sætninger, hvad brugeren er i gang med at arbejde på, hvis denne gerne vil holde dette privat). 

Og lad mig lige understrege, at meta-antagelser ikke skal ses som aksiom-sættet for ITP'en. Det skal de nemlig bestemt ikke. Det er f.eks.\ ikke sådan, at to brugere behøver at prøve at matche deres meta-antagelser helt, før at de kan bruge hinandens sætninger. Meta-antagelserne er i stedet bare antagelser om, hvad der gælder for det pågældende fil-system. Og hvis én bruger har brugt en sund og fornuftig logik til at udføre et bevis, og hvis den anden brugers eget logiske fundament bare er stærk nok (ved f.eks.\ at inkludere refleksions-regler/-aksiomer) til at verificere denne metode, jamen så kan anden bruger stadig importere første brugers bevis, også selvom deres meta-antagelser måske ser ret forskellige ud.


Nå ja, og så har jeg da også lige gjort nogle tanker omkring, hvad en god ITP til at arbejde med matematik kunne indeholde af muligheder, som er værd at nævne. Hvis man læser mine sektion \ref{main_12}-noter, som beskriver et overordnet bud på et indre ITP-miljø, så er mine tanker også beskrevet deri (samt mange uddaterede tanker dog også). For det første er det nok smart indføre typer, som brugere også selv kan erklære nye af. Så selv om man starter med f.eks.\ mængdelære som grundlag, så kan man jo hurtigt omdanne miljøet (via en barne-ITP) til et HOL-agtigt, typet deduktivt system. Tja, dette giver vel egentligt lidt sig selv. Men derudover vil jeg gerne nævne, at brugere helt sikkert også skal have mulighed for at erklære nye funktioner/operationer og så samtidigt kunne definere hvordan disse skal renderes. \ldots Tja, dette er jo egentligt også rimeligt trivielt, er det ikke? Tjo, men det er måske værd at nævne så, at denne rendering så skal kunne tage input, der selv er sammensatte termer, f.eks.\ tupler og matricer, og så overtrumfe disse termers normale rendering og simpelthen få lov at udplukke alle de nestede inputtermer og sætte dem op på en ny måde i renderingen af det samlede udtryk. Og sidst men ikke mindst kan jeg også lige nævne mine tanker om muligheden for at erklære nye regler via sætninger. Tanken er altså, at man kan bruge sætninger, der beskriver en omskrivningsregel, til at erklære en ny regel til listen (over de regler man kan bruge til at manipulere termerne og sætningerne i sit miljø). Her kan man så endda bruge lokale variable, som introduceres via kvantorer, man kan lave if-statements med hendholdsvis en disjuktion af to konjunktioner (som en antecedent), hvor første udtryk i hver konjunktion er conditional-udtrykket med hver deres logiske fortegn, og loops ved at bruge forall-kvantorer. Man kan endda også køre underrutiner, hvis man for visse funktioner og/eller prædikater sætter faste reduktions-regler. Så allerede har vi her et helt programmeringssprog, men altså udformet i logiske formularudtryk. Jeg tror dette kunne være rigtigt gavnligt, for når man alligevel arbejder med formularer, så er det for det første sikkert rart, at man ikke skal omforme sine viste udsagn til programmer i et mere konventionelt programmeringssprog, men bare kan udplukke regelprogrammerne direkte fra de viste sætninger. Og for brugere der ikke i forvejen kender til programmeringssprog, men som bare har lært (eller er i gang med at lære) at arbejde med matematiske formularer, jamen så er det sikkert rart for disse, at det ikke så er afhængige af at skulle lære et eller andet ekstern programmeringssprog, før de får mulighed for selv at oprette nye regler. 
I øvrigt har jeg også en idé til, hvad jeg vist har kaldt `farve-regler' (eller noget i den stil) i mine sektion \ref{main_12}-noter. Disse handler i bund og grund bare om, at man også kan bryde lidt ud af det billede, hvor man fortolker alle termer nøjagtigt ud fra deres output værdier, men også kan se på deres formular struktur og lave omskrivningsregler, der udfører beregninger over disse strukturer. Et godt eksempel kunne være at indføre en regel om at regel om at bruge en sætning, $A \to B$, til at omskrive et $A$, som er nested inde i en større logisk formular. Her kunne man så lave en formularknude-farvnings-algoritme til at farve alle ydre formularknuder fra udtrykket alt efter om de har et lige eller et ulige antal negationer foran sig, hvor en implikation så tæller som en negation, når det kommer til antecedenten. Og hvis så $A$-udtrykket i sidste ende kan farves med en farve, der repræsenterer et lige antal negationer, så vil man så kunne udskifte det med $B$. Et andet simpelt eksempel kunne være en aritmetisk algoritme a la dem, vi alle har lært i skolen, når man f.eks.\ skal lægge tal sammen eller gange dem sammen. Her kunne en bruger så udforme en farve-algoritme til at farve cifrene med parameter-farver, der holder styr på, hvad der er i mente, når man så skal finde næste resultat-ciffer, hvilket i en sådan algoritme også kunne implementeres ved at farve resultat-placeholder-knuder med parameter-farver. Og når alle resultat-placeholder-knuderne således er farvet, så kunne reglen så bare bestå i, at indsætte farve-parametrene på de pågældende pladser. Bemærk at sådanne regelalgoritmer, der laver computerberegninger på selve strukturen af udtrykket, netop kan lade sig gøre i sådan nogle ITP-systemer som disse, fordi de ligesom er ``klar over'', at alle de matematiske udtryk i sidste ende bunder i binære repræsentationer, man laver computeroperationer på, og man har derfor altid mulighed for indføre refleksionsregler, når disse bare altid resulterer i gyldige operationer på de termer, man arbejder med. Og særligt kan man så også lave højere-ordens-refleksionsregler, så som en regel som denne, der giver brugeren mulighed for at oprette regler løbende, i dette tilfælde disse farve-regler. Nå, men hvad er så idéen med sådanne farve-regler i stedet for bare at have mulighed for at skrive regelprogrammer i et mere gængst programmeringssprog, især når sådanne programmer sikkert ville blive mere effektive? Jo, den er simpelthen bare den, at jeg tror det ville være en nem måde at komme i gang med at oprette regelalgoritmer selv for mange brugere, særligt brugere på et lidt lavt niveau, som heller ikke kender så meget til programmering i forvejen, og så tror jeg i øvrigt også sådanne farve-algoritmer vil være nemme og hurtige både at konstruere og bevise korrektheden af. For som man måske har regnet ud fra mine eksempler, så tænker jeg farve-algoritmerne bare skal være rekursivt definerede som udgangspunkt, og dermed skal man altså bare lige vise nogle rekursionssætninger, og så kan man komme i gang med sin nye regel. Og en yderligere fordel ved sådanne farveregler er så også, som gør sig gældende især for brugere, der skal lære algoritmerne at kende (studerende m.m.), fordi de er ret intuitive og er lette at følge. Man kunne således tilføje funktionalitet, så algoritmerne kan afspilles lidt langsomt (og med pausemuligheder m.m.), så mennesker kan følge med. Det var en lidt lang forklaring om noget, der er en ret specifik lille ting, men jeg synes, det var værd lige at få det med her.
%Dette kan lade sig gøre, fordi... (tjek)
%Pointen med så specifikt at bruge sådanne "farve-regler" er så, at det er matematisk... og at brugeren så dermed let kan indføre sine egne komplekse regler uden at skulle til at skrive dem i et programmeringssprog... (Og det er i øvrigt nemt fordi det i grunden er rekursivt).. (tjek)

%Disse handler om, at man kan køre underrutiner, ikke bare til at omskrive termer, men til at stille spørgsmål til dem. Her kan man endda så også bruge refleksionsaksiomerne (som jeg har beskrevet i sektion \ref{ITP_feb-maj}, og som i øvrigt også (med fordel) kan implementeres via regler og ikke aksiomer), som gør, at vi altid kan skifte mellem at se et udtryk som en faktisk sætning og som et matematisk objekt, hvor den indre struktur har betydning, og hvor man derfor kan spørge til selve opbygningen er udtrykket. For hvis man har disse refleksionsaksiomer, kan man nemlig hurtigt omskrive frem og tilbage mellem de to. Dette betyder at man kan lave meta-agtige regler, der stiller spørgsmål direkte til selve udtrykkenes struktur, og her tænker jeg så at mine farve-regler kunne være særligt smarte. Som eksempel kunne det være, at man gerne ville bruge en $A \to B$ sætning, til at omskrive et $A$, som er nested inde i et større logisk udtryk. Her...%Hm, er det her overhovedet særligt relevant, når man også bare kan lave lav-niveau-programmer direkte? (Som man jo så også kan bruge direkte på termet; for vi er jo inde i en barne-ITP, så alle udtrykkene er bare omfortolkede binare strukturer..) ..Jo, de er værd at nævne, som så kan det også godt siges kortere.
%Hm, og jeg brugte jo også her lidt det med, at man kan gå frem og tilbage mellem ordner..

Og nu jeg lige nævner nogle specifikke idéer, så kunne man også lige nævne det med, at lærere gerne må kunne have det sådan, at de bare kan fjerne lidt fra et bevis, og bum, så har de en opgave eleverne kan løse. Tja, jeg vil egentligt ikke sige så meget her, for det giver lidt sig selv, og desuden bør man alligevel læse sektion \ref{ITP_feb-maj}-noterne for at kunne forstå disse noter ordentligt, og der her jeg vist også forklaret det fint nok.



Okay. Hvis jeg lige kort skal opsummere, hvad de vigtigste ting er ved denne idé her som afslutning, så vil jeg bare sige, at kerne-målet med ITP'en for det første er at få et godt system til at formalisere selv abstrakte bevismetoder via certifikaterne. Jeg kan igen igen bare lige referere til visionsafsnittet i sektion \ref{ITP_feb-maj}-noterne for en forklaring på, hvorfor jeg mener dette. Desuden synes jeg at konceptet om at bruge meta-antagelser er rigtigt godt, for det er jo også bare en måde, bl.a., at formalisere antagelser, man alligevel gør sig implicit med enhver anden ITP. Og når man så først for antagelserne skrevet formelt op, så er det også smart, at man så kan gøre dem vilkårligt avancerede, så man virkeligt kan få nogle effektive ITP-systemer. Jeg ser det også som en vigtig idé, at fokusere refleksionsprincippet i ITP'erne på lav-niveau IL-sprog, der er tættere på maskinekode ift.\ et funktionelt sprog, som andre ITP'er ellers typisk vil gøre brug af. Og grunden til at man kan tillade sig det her, er at man selv får lov at vise korrektheden af programmerne. Man kan så f.eks.\ sagtens bruge et høj-niveau-sprog som et mellemtrin til at vise korrektheden af programmer (ved at vise korrektheden af en compiler (der kompilerer til IL-sproget)), men kan så altså også gøre meget mere end dette (f.eks.\ lave rigtigt effektive programmer). Og med mine seneste idéer, der gør at ITP-systemet er åbnet helt op for ændringer, i sådan en grad så at to brugere endda sagtens kan starte med ITP'er, der bruger helt forskellige underliggende deduktive systemer, så tror jeg, at der virkeligt kan blive god samling på hele ITP-fælleskabet. For alle kan så få deres egne præferencer, men uden at det behøver at besværliggøre kommunikationen af beviser sætninger på tværs af hele ITP-brugerfællesskabet. For jeg tror nemlig, at det så bliver ret trivielt at holde, hvad der svarer til transpilere imellem alle de forskellige systemer (effektivt set). Herved tror jeg aldrig brugere behøver at bekymre sig for, at deres arbejde delvist kunne gå tabt, hvis nu deres ITP blev forældet, eller bliver for niche. Tja.. Om ikke andet er det bare rart med et så åbent fundament, for hvorfor skulle man så nogensinde begrænse sig til en ITP, der ikke bygger på sådan et dejligt åbent fundament? (Jeg synes nemlig også, der er noget filosofisk behageligt over, at fundamentet er helt åbent; ingen aksiomer eller antagelser end dem, man selv sætter, og heller ingen implicitte antagelser, for man sørger nemlig bare for altid at sætte dem eksplicit.)

% Hm, jeg kunne også lige nævne noget med, at mere intuitiv bevisførsel bliver muligt, fordi man kan gå frem og tilbage mellem ordner..


% "Uh, husk!: Jeg kom lige til at tænke på: Det er vigtigt at pointere, at meta-antagelserne ikke skal ses helt som et aksiomsæt, for to brugere der har forskellige antagelser, skal alligevel sagtens kunne dele beviser. Så det handler nemlig derfor om ikke at betragte dem som matematiske aksiomer, men som antagelser om, hvordan ens maskine kan bygge og verificere beviser, og sørge for at gemme sætninger korrekt osv. Ja, vigtigt at pointere."



%Har jeg nævnt nok om "refleksionsaksiomerne"?
%Kunne jeg også lige understrege, at man jo med dette ITP-system godt kan tillade sig at bruge programmer (eller dvs. compilere), også selvom semantikken ikke er 100 \% defineret for hver kørsel?
%Burde man også sige noget om intuitive actions og om at den indbyggede mulighed for HOL-frihed er smart her.? Og at ét oplagt område for sådanne bevistyper kunne være diskret matematik, og bare dette i sig selv er jo vigtigt nok.?
%Vær sikker på, at det er tydeligt nok nævnt, hvordan "åbne" meta-antagelser kan "føde" "barne-antagelser." *Tja, nu har jeg jo godt nok udkommenteret den tekst og brugt nogle lidt andre temrer og formuleringer..

\subsection{Formel matematisk programmering (og semantisk web m.m.)}
Jeg synes, jeg var meget heldig, da jeg skrev følgende sektion, taget fra mine appendiks-noter (se sektion \ref{main_12}) nedenfor (som jeg altså lige kopiere ind igen her):\\

{\slshape
By having the built-in (lower-level) program(s) powerful enough to be able to run interactive programs with graphical interfaces and all, there is then no limit for how the users can basically extend the application. 
This is of course true for any programming language but in this instance, it might be even more useful, since there is a difference between updating and making extensions for the application itself, which requires testing and peer-reviews, and then just proving some theorems about a certain program. This is one of the big possible benefits from a new mathematical (proof-based) programming language paradigm if it can work, namely the fact that one does not need to test and review updates. If we look at object-oriented programming, we already have a good way of implementing a program in a top-down way and divide it up into modules. These modules can then be documented for their intended function and consequently implemented and tested. But what if the documentation is instead written mathematically, so that the implementation can be proved to have the correct function? At a first glance, all this seem to do is to shift the work of programming the module to the documentation. But perhaps writing a documentation mathematically and writing a rigorous documentation normally might not be too different once the paradigm has developed enough. And when I think about it, I think it would often be easier to describe a module rigorously instead of programming it in the imperative. This might differ from task to task as well as from person to person, but this is why it is a good idea to have several programming paradigms in the first place. Furthermore, \iffalse This is a good way of putting it: Good with several paradigms, and furthermore...\fi it would be possible to compile some of these mathematical documentations to imperative programs, and once the paradigm gets developed, even quite abstract documentation might be compiled this way, at least partially. The IDE could then give suggestions for different design patterns (prepared by other users by having things proved about them) to pursue, where potential gaps to solve, i.e. when a part of the solution is dependent on outside factors that therefore needs a specially tailored proof by the user dependent on these outside factors, are left open. This is why even the implementation itself might get to be a more top-down process with such a mathematical programming paradigm. Imagine writing a rather abstract documentation of what a certain module needs to do and then just choosing between a couple of implementation patterns and then to have the IDE write out the whole template for the implementation as well as the new mathematical documentation for each submodule that needs further implementation if there are any (and where you might even be able to use the same procedure for some of these submodules). I think this can save a lot of work. And since this is all just mathematics, it is very easy to just import any new and useful implementation patterns as these are all just mathematical propositions (so no need for any rigorous peer-reviewing). And then once you are done with the implementation you are done! No need for further testing or for using external theorem provers. This fact alone could save a lot of work even at early stages of the paradigm where it might still be more work to make mathematical implementations rather than imperative ones. Furthermore, when you are done, the product also gets more value since other parties that depend on your program does not need to trust the correctness of your tests or your documentation as the can just run the proofs themselves. As an additional point, a mathematical paradigm might be easier to work with in big projects with a lot of different programmers where a lot of rigorous intercommunication is needed. Such project obviously need rigorous planning and documentation anyway, so why not make this mathematical in order to get rid of any ambiguity. Also, while it is possible to make mistakes in the documentation and for instance forget certain eventualities, a lot of the documentations of a big project will be documentation of submodules and the correctness of the more overall documentation of the program will thus be dependent on these. If it is made sure, that it is first proven that the overall (mathematical) documentation holds if the documentation of the submodules hold (in a top-down fashion), then there is never the risk that individual programmers work on a perceived task that is different from what was intended (or at least that risk is reduced I should say). 
Oh, and there is another point as well. In an open source community, having mathematical documentations might also greatly increase the value of such programs in that community, since it might be possible to then make it easier to search for a certain solution to a problem at hand. It might de easier for servers to serve the right solutions to the clients once everything is mathematically described, as the servers then might just need to do a little bit of math on their own order to find a good solution. A bit like Wolfram Alpha, I guess, but more rigorous and for programs (submodules and subroutines) as well. I will expand more on the possibilities I see for an open source community when programs have mathematical documentations in the next section/chapter. Oh, and yet another point is of course how it makes updating applications much more easy. With normal paradigms each update to a unit of a program has to be rigorously tested both individually, and should also preferably be tested in the whole context. But once a program has a mathematical backbone all the way through, each unit can then be updated as many times as desired and all that is needed to do before exporting is just to prove that it still fulfills the already defined documentation. So I would imagine that updating could be a much simpler task overall this way. 

So to sum some things up about the paradigm, it means that programmers are intended to use theorem proving rather than tests (but does not necessarily force them to do so, as documentations can be relaxed to conclude a probability of working as intended instead of a yes/no, and then one can make use of assumptions about have testing yields probabilities, which would also help make testing more rigorous in a way, but this is still not really what is intended for the paradigm) and to make these proofs in a top-down way. I know that with only these describing attributes, one could just choose to use this same approach with existing paradigm (for instance with the functional paradigm), but this would still not be exactly the same. The point is that in this mathematical paradigm, the theorem proving becomes part of the regular programming and people can then share lemmas which proves that a certain implementation design pattern yields a certain functionality given some restrictions of some variable submodules with each other and then these sort of lemmas can furthermore be used by IDEs to come up with possible solution routes for implementing a module and then print out any such chosen template so that it is ready to be completed. To make theorem proving the main part of the actual programming itself thus should open up to new way of programming (collaboratively) in my view and might thus constitute a valuable new programming paradigm.\\
}


Ja, så pointen er altså for det første, at det i nogle tilfælde kan være nemmere at formulere matematiske invarianter for eksempelvis funktioner (eller for objekter eller for et samlet system, hvor objekter interagerer). Dette i sig selv er vel grund nok til at udvikle et godt matematisk programmeringsparadigme, for så vil der jo være visse projekter, hvor dette paradigme vil være smart (altså projekter med mange sådanne forhold, hvor invarianterne er lettest at beskrive frem for at programmere dem imperativt). 
Hm, men fordelen ved dette kommer vel meget an på, hvor let det så er at bevise sig frem til, om funktioner m.m.\ overholder visse restriktioner\ldots\ Og hvor let det er at bygge kompileringsprocedurer for matematiske restriktioner\ldots\ Tjo, men angående det første så er det vel bare et spørgsmål om, hvor hurtigt og let det vil gå med at danne gode design patterns i brugerfællesskabet til at løse, eller indgå som en del af løsningen til, forskellige problemer. Ja, og min mening er så, at dette med tiden vil komme. Og pointen er jo så netop også, at man i mellemtiden kan lappe hullerne med certifikater (m.m.) i stedet, på helt samme måde som at man kan lappe hullerne midlertidigt med certifikater (m.m.), når det kommer til beviser i ITP-fælleskabet generelt. 

Ja, så ved at beskrive sine programmeringsobjekter matematisk, kan man altså med tiden få det nemmere, fordi man kan kompilere formuleringer til, potentielt åbne, implementeringer, og så kan man samtidigt også sammensætte diverse -undermoduler/-systemer mere formelt, når de er semantisk dokumenteret --- også selv hvis man ikke kan bevise sig frem til sammensætningen, fordi man så bare kan underskrive de propositioner, man ikke har tid til at vise, i stedet. Og når vi snakker om `potentielt åbne' implementeringer, så handler det for det første om, at IDE'en kan komme med forskellige løsningsforslag ved at bruge forskellige design patterns --- hvilke den kan søge efter på på internettet via semantisk søgning --- og hvis der er frie parametre i restriktionerne, så kan IDE'en prøve at finde løsningsforslag, hvor disse parametre er åbne, eller kan prøve at forslå forskellige løsninger, der hver især har disse parametre (som altså ikke nødvendigvis er konkrete parametre; jeg taler bare om generelle, abstrakte parametre) specificerede. Og lige for at præcisere så kan `løsningsforslag' bare være præcis dette, altså forslag til at løse problemet men uden noget bevis tilknyttet for, at dette forslag kan løse det, eller det kan være faktiske skabeloner for en løsning, som er vist kan løse problemet i visse tilfælde. I første tilfælde kan løsningsforslagene altså bare gives som en slags hints, eventuelt med bevis- og/eller program-skitseringer vedlagt, og i andet tilfælde kan der altså være tale om skabelonsobjekter, der allerede har propositioner vist om sig (som man kan downloade og stole på, hvis de bare har passande certifikater). Nu skriver jeg `skabelonsobjekter'\ldots\ Der kunne faktisk godt være tale om matematiske objekter, man downloader i til en god arbejds-ITP, som er beregnet til programmering, og hvor man så kan specificere (udfylde, om man vil) skabelonen simpelthen ved at lave en række (gyldige pr.\ skabelonspropositionen) matematiske manipulationer. Ja, i øvrigt lidt en interessant tanke, denne her med at programmere ved at operere på programmeringsobjekter med gyldige manipulationer/operationer. Så det matematiske programmeringsparadigme kunne også samtidigt lægge grundlag for et nyt paradigme, når det kommer til programmeringsinterfaces (hvor altså programmering ikke bare handler om at skrive tekst og så indimellem vælge skabeloner og godkende autocompletions, men hvor kan tage meget mere alsidige former for actions for at bygge på sit programmeringsprojekt). Ja, og det kunne i øvrigt også føre endnu videre til en meget mere visuel form for programmering, hvor man renderer sit programmeringsprojekt som en visuel model, man kan manipulere direkte, i stedet for bare at have en række tekstdokumenter. Nå ja, og i en lidt senere fremtid kunne man endda forvente, at manipulationerne kunne blive NL-input, som computeren processerer og fortolker til semantiske udsagn (bygget i matematik-sproget), og som så kan formulere en manipulation til programmeringsprojekt-objektet. Ja, spændende og helt klart værd at nævne. 


En anden vigtig pointe er, at projekter, der har propositioner vist om sig (med tilhørende bevis-scripts) er lettere at hente ind og bruge i et andet matematisk programmeringsprojekt. Det bliver således nemmere at hente og bruge andres løsninger uden først at skulle læse en hel masse selv for at være sikker på, at man forstår semantikken nøjagtigt, og uden at skulle lave ligeså mange tests af de hentede skabeloner og moduler. Det er faktisk en ret vigtig pointe: man slipper for selv at have hele ansvaret for at forstå semantikken af underprocedurer og moduler; nu kan computeren tage en del af dette ansvar. Det bliver også lettere, mener jeg, at opdaterer og vedligeholde projekter, fordi man så bare skal sørge for, at de samme restriktioner er overholdt, og ultimativt at beviset for propositionerne om hele projektet stadig kan køre, og når de så kan det, jamen så er der så stor sandsynlighed for, at opdateringen er gyldig (og korrekthedsbevarende, om man vil). Tilmed mener jeg også, at det vil blive nemmere at kommunikerer målene og programmeringsspecifikationerne (requirements) på tværs af et projekt, og mindske arbejdet med at sørge for, at et formål bliver kommunikeret rigtigt. For én ting er, at man så bare kan sørge for at formulere dem semantisk præcist med et matematisk sprog, som en ledene designer, og så bliver programørerne så nødt til før eller siden at forstå disse specifikationer, for at kunne udføre beviserne i deres ende, men en anden ting er, at man på sin egen ende, design-enden, også lettere kan fange mangler i ens designspecifikationer, for man skal jo selv bruge de samme formelle specifikationer som lemmaer i beviset for de overordnede propositioner om projektet. Hermed kan man altså også bruge dette som en klar hjælp til at få sat de rigtige underspecifikationer som designer, hvis man så bare hele tiden sørger for at konkludere noget på baggrund af hver eneste underspecifikation. For hvis man gør dette, så sørger man bedre for, at ingen af de givne specifikationer er forkert formuleret. Og da man så selvfølgelig også samtidigt bør formulere sine mål for det overordnede projekt ret præcist, så kan designerne altså også bedre sikre sig, at de har givet nok (og korrekte) underspecifikationer til at disse mål bliver opfyldt i det samlede system. Bemærk i øvrigt, at disse mål sagtens kan indeholde bløde udsagn, som så kan bevises via abstrakte, men formelle, procedurer. Dette kunne f.eks.\ være i form af udsagn om, hvordan brugerfladen leve op til abstrakte designprincipper, eller hvordan man tester dette, samt hvordan man tester hele projektet generelt (og testkrav kan således også være en del af underspecifikationerne). Selv sådan noget som at udføre user tests kan formaliseres. Så handler det nemlig bare om, at deltagerne i procedurerne løbende underskriver udsagn om, hvad de har foretaget sig, samt giver underskrifter for, at diverse ting er forløbet korrekt og/eller opfylder de rette krav. Og når vi taler om testing, så kan man også benytte samme princip, som når man henter program-moduler og -skabeloner fra internettet, nemlig at man ikke behøver at nærlæse semantikken fra start ligeså grundigt, fordi de ikke er ligeså slemt, hvis man misforstår noget. Hvis man misforstår noget, mister man nok allerhøjst nogle tidsresurser, for man vil altid opdage det i tide --- så længe man bare sørger for så altid at have nærlæst og forstået de kontrakter, man løbende skal underskrive\ldots\ Ja, så der vil der jo altid være en potentiel fejlkilde, men det vil stadig være lettere at styre, når tingene således er mere skåret ud i pap. Man kunne endda muligvis sikre en sådan procedure yderligere, ved at deltagere også skal underskrive udsagn om, hvor grundigt de mere fundamentale udsagn, der skal underskrives, er blevet læst og hvornår (så der også er sørget for, at det ikke er kun læst for lang tid siden og muligvis derfor ikke kan huskes korrekt af proceduredeltager). 


Og som jeg har skrevet, så forbedrer det også projektet, at det selv kan tillægges formelle propositioner, for det gør det for det første lettere at bruge i andre formelle sammenhænge (inkl.\ hvis man vil bygge videre på projektet og/eller bruge det som modul i andre projekter) --- og det gør også at brugerne/klienterne/køberne i sidste ende kan opnå større tillid til projektløsningen. Og samtidigt bliver det også lettere at søge på, for hvis man ønsker en metode eller et modul, der har visse kvaliteter, så bør man jo så bare kunne formulere disse kvaliteter matematisk, i form af en variabel proposition, og så søge på, om der findes viste propositioner, der matcher (altså matcher den propositionsskabelon, man søger på). Og når andre brugere uploader løsninger til nettet (til passende lokationer), så vil de jo netop prøve at vise de mest brugbare propositioner omkring deres løsninger. Og hvis serveren endda får mulighed og resurser til at regne korollarer ud, så gør det endda ikke noget, hvis brugeren formulerer efterspørgslen en anelse anderledes. Klienter og servere kunne i øvrigt også hjælpe hinanden ved at vedtage standard-former, så server sørger for gemme korollarer og klient sørger for at søge på korollarer, der er på en standard-form af den hhv.\ uploadede og den efterspurgte sætning. Og desuden kan brugere selvfølgelig også hjælpe hinanden med at uploade flere nyttige sætninger omkring hinandens uploadede løsninger. 

Hm, jeg har vist ikke fået nævnt det(?), men man bør vel også lige komme ind på, at man med matematisk programmering ikke behøver at bekymre om felter osv.; man kan programmere kun med proporties i hele udviklingsfasen, inden man så til sidst kan finde den mest effektive måde lige at implementere det på\ldots\ \ldots Tjo, på den anden side giver det vel lidt sig selv; det må være næsten en velkendt fordel ved et matematisk programmeringssprog i forvejen.





\subsubsection{Semantisk web m.m.}
Jeg mener i øvrigt, at dette kunne blive en god start på det semantiske web, som man kalder det, nemlig ved at disse metoder på et tidspunkt vil, tror jeg, brede sig ud til andre områder, som ikke bare omhandler matematik og programmering. I første omgang vil det selvfølgelig brede sig til andre tekniske felter, så som fysik- og ingeniør-felter såvel som andre naturvidenskabelige felter. Men det behøver jo ikke at stoppe der. Jeg tror endda det kunne brede sig helt ud til alle de ting vi kunne finde på at søge på internettet: pop-kultur, opskrifter, underholdning, brugbare informationer så som lukketider og ledige parkeringspladser\ldots You name it. Jeg kunne blive ved. For selv sådan nogle virkeligt virkelige ting (om man vil) som de sidstnævnte kan modelleres, og så handler det altså bare om at finde frem til nogle gode standarder i brugerfællesskabet for, hvordan man modellerer sådanne real-world-systemer, og så kan folk så bare søge på sætninger om sådanne modeller. Og ved så at bruge standarder for certifikater og ved at bygge et helt system af autoritære stemmer --- f.eks.\ kilder til hvad lukketider er osv.\ --- så ville man dermed nemt kunne søge på ting, man aldrig har søgt på før, men stadig vide, hvordan man skulle formulere sine efterspørgselssætninger. Og på et tidspunkt --- og det kan også godt være, at det begynder at være en realitet, før den nævnte teknologi breder sig ud til resten af webbet --- så vil man også kunne gå mere og mere væk fra, at klienterne/brugerne skal formulere deres internet-queries som formular-skabeloner, men kan begynde at bruge naturlige sprog mere og mere. Dette vil så kræve, at brugerfællesskabet vil udvikle semantisk veldefinerede undersprog af naturlige sprog. Det vil sige, at man starter med at inkludere semantisk simple sætninger (og i første omgang starter et sæt grundlæggende semantiske atomer, som alt semantik videre bliver bygget op af (og der eksisterer vist allerede teori om, hvilke sprog-atomer, man kunne starte med, for semantisk veldefinerede sprog)) og så bygger videre derfra. Og når man kommer til sætninger, der er tvetydige i det pågældende naturlige sprog, så bør man så faktisk begynde at fastsætte udvidelser af det naturlige sprog, så semantikken kan defineres. Min tanke af så, at man bør kunne se det som at sætte semantiske flag/fodnoter på sine sætninger, så semantikken bliver entydig igen. Pointen er så her, at man stadig bevarer det oprindelige sprog nedenunder, sådan at man altid kan oversætte det semantiske sprog til det naturlige sprog i princippet bare ved at fjerne flagene/fodnoterne fra det. Og, ikke mindst, hvis man vil oversætte fra naturligt sprog til semantisk veldefineret sprog, hvilket man jo skal have gjort hver gang, man skal søge på noget i det semantiske web, så handler det bare om at få sat de pågældende flag fra brugerens side. Så når teknologien først bliver godt udviklet, så forestiller jeg mig altså, at brugeren skriver, eller taler, sætninger ind i et tekstfelt, hvor interfacet så løbende prøver at oversætte det til det formelle, semantisk veldefinerede sprog, og hver gang man så når til en tvetydighed, kan denne søgningsinterface-applikation så bare give feedback, så brugeren kan sikre sig, at de rette semantiske flag bliver valgt for søgningen. Så det er sådan, at det semantiske web (for vi skal nok nå dertil på et tidspunkt, er jeg overbevist om) kommer til at fungere i praksis set fra brugeren, og jeg tror så altså at det kunne blive en rigtig god indgang til denne teknologi, at man først begynder at bruge matematiske formularer mere og mere, når man søger. For når man først for øjnene mere op for, hvad man kan med semantisk præcise søgninger, så vil der også blive rigeligt skub på udviklingen til så at gøre sådanne søgninger mere og mere brugervenlige. Og hvordan går man så fra at have matematiske model-standarder, som kan bruges til at modellere real-world-systemer på en fast måde, således at brugere kan søge information om emner, de ikke er vant til at formulere formular-sætninger om, og så til at bruge sprog-modeller, der beskriver den virkelige verden gennem de atomare semantik-fortolkninger, sproget bygger på? Jamen hvorfor ikke bare sørge for fra starten af, at de standarder, man sætter i brugerfællesskabet for sine real-world-modeller, også simpelthen bare bygger på, at man tillægger semantiske fortolkninger til et sæt af atomare byggesten? \ldots Tja, nu hvor jeg tænker over det, så er der vil ikke rigtigt noget alternativ til dette alligevel? *(Tjo, det er der måske, men det handler så bare om at bygge det, så selve sproget og de certifikater, der bruges til at besvare spørgsmål omkring sproget, i to separate lag, så det underlæggende sprog ikke afhænger af certifikaterne, og så man dermed f.eks.\ løbende kan udskifte certifikat-metoderne til at vurdere sandheder omkring sproget rimeligt frit.)\,\ldots\ Men ja, uanset hvad, så er tanken så bare, at man sagtens kan starte med mere tekniske sprog-atomer, og også gerne med langt flere af dem (og hvor man løbende kan tilføje flere), end man ville, hvis man skulle starte fra et lille sæt af meget alsidige semantiske atomer. Et sådant atomsæt er nemlig, hvad man sikkert ville bevæge sig mod for et semantisk sprog, men man behøver altså ikke at starte her. Fordi vi bygger modellerne matematisk kan man nemlig altid bare danne oversættelser mellem atomsæt. Derfor kan man altså sagtens starte med et meget teknisk semantisk sprog til at søge på, hvorfra man så hurtigt kan bygge termer og prædikater, der handler om det relevante emne, uden først at skulle igennem en masse lingvistisk arbejde først, og så kan man altid oversætte dette løbende til et mere og mere semantisk ``rent'' atomsæt. Bemærk at brugere uanset hvad løbende så kan danne nye termer og prædikater til brug i disse modeller, og dermed til at søge med osv. Brugere der så følger med i denne sproglige udvikling kan dermed så blive bedre og bedre til hurtigt at lave præcise søgninger ved at tage nyudviklede termer og prædikater (og relationer selvfølgelig) i brug. Dette svarer jo så helt til, at en person i almindelighed bliver bedre til at søge (og læse) om et teknisk emne, jo flere tekniske termer denne lærer. Men det smarte ved at have et semantisk formelt sprog bygget over matematik, som er værd at nævne, er at det så særligt også bliver nemmere at søge information om tekniske felter, \emph{uden} at man kender de tekniske termer på forhånd. For selvom de brugere, der har uploaded information, har benyttet sig af de teknisk avancerede termer m.m.\ så vil der jo her findes en indbygget oversættelse til alternative formuleringer, der \emph{ikke} bruger disse termer. Dette bliver en af de helt store fordele ved et semantisk web, som jeg ser det (selvom der dog er mange store fordele at vælge imellem). Hvis man nu f.eks.\ ville søge på, hvad gængse ITP'er har af muligheder for at udføre regnestykker og udregne funktionskald, men ikke i forvejen kender til begrebet `reflection principle,' hvilket, så vidt jeg husker, hjælper denne søgning gevaldigt, jamen så kan man jo bare søge semantisk på konceptet, og så bør man altså på et veludviklet semantisk web alligevel hurtigt komme frem til det samme (og hvor man så hurtigt kan lære de tekniske begreber i samme ombæring). Jeg kunne også finde på mange eksempler fra programmering, hvor man selv forstår, hvad man leder efter, men ikke i udgangspunktet kender de tekniske termer, der ellers ville lette søgning helt vildt. Men jeg er sikker på, at læsere selv vil kunne finde deres egne eksempler, også inde for alle mulige andre emner. Men med et godt semantisk web vil alle sådan nogle søgninger bare blive meget nemmere. Det vil altså både medføre, at det kommer til at blive meget nemmere at søge på ting, hvor vi ikke lige kender de rette nøgleord at søge på fra starten, og i det hele taget at vi kan præcisere vores søgninger meget mere nøjagtigt --- også i tilfælde hvor der slet ikke findes nogle "magiske" nøgleord, der lige kan lette den søgning helt vildt, endda selvom det man søger efter, egentligt kan beskrives helt kort og præcist i virkeligheden (hvilket jeg også tit synes, er tilfældet).
%... %reduktionsprincip? transpiler? designprincipper? ... (tjek)
%Giv eksempel med tranpiler eller noget, hvor man måske ikke ved, hvad det hedder, men godt ved, hvad man søger efter, og forklar, at fordi det er baseret i matematisk, så vil man så kunne søge semantisk efter de termer, der så kan udvide ens søge-repetiore for fremtiden. (tjek)
%Skulle man nævne noget omkring, at almindelig tekst på nettet så også løbende bliver mere og mere semantisk?.. Og skulle man undertrege noget mere omkring at man bare kan benytte sig af URL-antagelser til sine programmer, så man også kan søge informantion på nettet, uden at lokationen er tiltænkt til at være semantiske..?.. (måske er det ikke så vigtigt...)



Og nu vi snakker om det semantiske web, så tror jeg også, vi vil se et skift, når det kommer til, hvordan vi bruger internettet, hvor vi går fra at webbet er meget centreret om forskellige, adskilte hjemmesider, hvori man kan tilgå indhold og foretage handlinger (så som at uploade ting, købe ting osv.), og så til at centrere sig mere omkring selve indholdsobjekterne og handlingsinterface-objekterne (og ikke så meget \emph{hvor} på webbet, de kommer fra). Og samtidigt tror jeg, at webbrugerfladerne vil blive mere åbne, sådan at hjemmesiderne måske bare giver nogle forslag til, hvordan brugerfladen kunne være, men hvor ansvaret i sidste ende ligger mere hos brugeren selv og dennes browser. Brugeren vil så, som jeg forestiller mig det, kunne justere sine egne præferencer for, hvordan diverse objekter skal opsættes. Jeg tror nemlig også, at en stor ting, som matematisk/semantisk programmering kommer til at bringe med sig, er, at vi får kategoriseret designprincipper og designmønstre meget mere. Jeg tror således, at vi kan danne en hel ontologi over alverdens former for brugerflader, hvor man så ved at specificere og justere på parametre (som definerer en sti ned gennem ontologien samt valg af de parametre, der eventuelt hører til hver skridt) kan opnå en hvilken som helst type brugerflade, man ønsker. Eller dvs.\ det man opnår, er så rammen for en brugerflade, som så tager form alt efter de indholdsobjekter, man så fylder ind i den. (Og her giver det så også god mening at tænke de forskellige typer indholdsobjekter ind i brugerflade-ontologien; man behøver altså ikke at forvente, at brugerflade-rammerne selv skal kunne analysere typerne af indholdsobjekterne, men kan godt bare benytte nogle klassifikationsstandarder, evt.\ ved at blive enige om nogle header-formater til diverse indholdsobjekter, som ontologien så også kan benytte). Dette i sig selv, gør så ikke så meget, for hvad er forskellen på at bygge en brugerflade i et markup-sprog eller ved at bygge den ved at specificere parametre i en ontologi? Nå jo, dette kunne jo faktisk netop også være en del nemmere, fordi det så svarer til at have et alsidigt bibliotek af skabeloner, hvor man samtidigt med det samme kan se, hvor godt de forskellige løsninger overholder diverse designprincipper osv., således at det bliver rigtigt let for selv utrænede individer hurtigt at bygge solide brugerflader. Ja, så det er en god pointe i sig selv. Men derudover er forskellen så også, at det så også bliver meget nemmere at gøre det muligt for brugere at justere på dette design, så at deres egne præferencer kommer mere ind over det. Dette kunne som eksempel være præferencer om, hvor stor teksten skal være, eller efter hvilke principper bjælkemenuen (eller andre menuer) skal være indrettet, hvordan de foretrækker at notifikationer bliver vist, hvad de foretrækker angående cookies og angående reklamer (hvor og hvordan skal reklamer vises), hvad skal farvetemaet være, hvor højt afspillede lyde må peake, og hvordan lydstyrken generelt bør være justeret for forskellige lyde, hvor mange forstyrrende elementer må der være, bør visse links kræve dobbeltklik i stedet for enkeltklik, så man ikke så nemt kommer til at følge dem, hvorhenne på fladen skal kontaktinformation helst stå, hvor skal man trykke for at se en oversigt over mulighederne på siden, skal brugerfladen være begrænset af hensyn f.eks.\ rettet mod, at børn skal kunne bruge den (skal der være købs- og indholds-begrænsninger), osv.\ osv.\ osv. 
Samtidigt kan brugerpræferencer også inkludere, hvilket nogle overlays man har på grænsefladen. Det kunne f.eks.\ være overlays, der forbinder til eksterne hjemmesider så som sociale medier osv., men når vi nu snakker om det semantiske web, så ville et oplagt overlay være et, hvor man kan skrive (mere eller mindre semantisk præcise) kommentarer til objekter, og se hvad andre har givet af kommentare i brugernetværket. Det kunne også være et overlay, der sørger for at analysere semantikken omkring objektet (inkl.\ af indholdet, hvilket især bliver brugbart, når man begynder bedre at kunne processere tekst generelt) og komme med relevante links, f.eks.\ til at forklare indhold eller til at give bud på relaterede emner. 
Man kan også forestille sig mange flere muligheder for overlays, og i det hele taget kan man forestille sig, at brugere sagtens kunne få lyst til at blande forskellige hjemmesider, så man kan tilgå flere ting på én gang i den samme brugerflade. 
Og det er af disse grunde, at jeg tror på, at fremtidens web vil blive mere objektorienteret (selvom det udtryk er taget (men jeg har ikke fundet på, hvad jeg ellers skal kalde det)), altså hvor vi taler om indholdsobjekter og delbrugerflade-/handlings-objekter, og hvor brugerflade design så vil bestemmes i højere grad i brugernes ende. 

Når det kommer til, hvilken browser man så skal bruge til dette mere semantiske web, så kunne man jo passende bruge et program, som indgår i ens ITP-system, så man derved har direkte mulighed for selv at bevise ting omkring de downloadede objekter --- og selvfølgelig også bede sin computer om at eftervise ting. Og så kan man jo netop selv holde de ontologier, man gør brug af, så som en brugerflade-ontologi, i sit ITP-system og bruge det til at rendere brugerfladen og til at finde ud af hvilke objekter, der skal efterspørges. Så jeg forestiller mig altså, at brugernetværket bygger et (alsidigt) browserprogram som brugere kan bruge som en integreret del af deres ITP-systemer. Jeg forestiller mig så, at kernen til dette browserprogram så skal være et sæt af meta-antagelser, og disse må meget gerne føjes til listen af faktiske meta-antagelser, sådan at brugeren ikke skal hente den samme ting to gange, hvis denne gerne vil bruge det i et bevis\ldots\ Tja, det ville måske egentligt ikke gøre så meget, så det må man vel bare selv om. Men ved altså så at bruge en liste af meta-antagelser som kerne, så får man jo for det første en browser, der nemt kan opdateres, fordi man ligesom med sine ITP'er bare kan lave nogle lidt åbne meta-antagelser, der så definerer en opdateringsprotokol. Og browser-programmet kan så på en måde selv sørge for at holde lidt øje med, hvor på internettet man kan søge diverse informationer/objekter. Med andre ord kan man få lidt en selvlærende browser, som altså selv kan sørge for at holde sig ajour med webbet/nettet og også kan opdatere sig selv ifølge brugerens præferencer. Dette bliver nok særligt vigtigt at have, i takt med at webbet bliver mere og mere semantisk og orienteret mod objekter frem for hjemmesider, som jeg jo tror det vil. For når vi jo som nævnt et punkt, hvor brugere ikke behøver at bekymre sig om \emph{hvor} på nettet/webbet ting er, men bare om \emph{hvad} denne ønsker at tilgå. Og så vil det jo være smart med en browser, der bare selv sørger for at holde sig ajour med, hvor den skal søge efter diverse objekter og hvordan. Og jeg mener altså så. at kernen til sådan en slags browser snildt bare kunne være en række meta-antagelser (og sammenkoblet med et ITP-system, så browser-programmet kan lave beviser og udledninger på baggrund af brugeren). I øvrigt, når vi nu taler om et mere objektorienteret (eller hvad vi skal kalde det) web, så mener jeg også, at et matematisk programmeringsparadigme kunne hjælpe denne udvikling. Man kunne nemlig bruge det til at give større fleksibilitet på internettet, hvor servere så selv kan have en række meta-antagelser for, hvilke nogle serverprogrammer, de gerne vil tilbyde at køre. Dermed kan vi så opnå et servernetværk (i.e.\ et del-internet), som er i stand til at ændre sig selv dynamisk og fleksibelt, alt efter hvad brugerbasen/klientbasen har brug for. Hvis en anseelig mængde brugere så går over til at ønske, at serveren kører nogle andre programmer (enten opdaterede versioner eller helt andre), end de kører til det givne tidspunkt, så kan serveren så i første omgang tjekke, om denne efterspørgsel er stor nok, og så derefter eftervise, at de uploadede programmer er sikre, og at de i det hele taget overholder serverens egne parametre for, hvilke programmer kan godkendes (som godt kan være ens slags cost-benefit-analyse, der så også afhænger af efterspørgslen). Jeg skal så også lige nævne, at når internettet bliver orienteret mod indholdsobjekterne frem for hjemmesider, så er det vigtigt allerførst at udforme en god krypteringsprotokol, så en bruger kan være sikker på, at andre folk ikke kan snage i, hvilke nogle objekter denne tilgår. Man kunne måske godt bygge dette over HTTPS-protokollen, hvor URL'erne så bare er skjulte for brugeren, men man får nok brug for noget mere end dette, især hvis serverne også bliver mere fleksible selv\ldots\ Tja, svaret er vel egentligt bare, at man sørger for at oprette en krypteret forbindelse til alle de servere (a la HTTPS-protokollen), man nu end opretter forbindelse med, så browseren kan downloade relevante objekter, og så først derefter begynder at query'e specifikke objekter fra de servere. Ja, så det vist ikke nogen svær overgang. 
% Hvordan nettet så også nok vil ændre karrakter, så servere hjælper hinanden mere.. Og det leder til at beskrive, hvordan man kunne bruge ITP-programmering til at udvikle nettet.. Men så skal jeg jo også måske helst have nævnt ITP-browser-principper først..? (tjek)
%Husk objekt-https (sikkerhed) (tjek). 
%Og overvej noget om offline-reklamer.. ..Tja, sådan noget kan man gøre, når vi når til, at brugere kan overvåge sig selv på en ikke-kompromiterende/-pinlig/-afslørende og så-godt-som-anonym måde. Og dette kunne bare kræve, at der er en overvågningsmekanisme, der slår til en gang imellem, hvor dem, der skal tjekke det, så kan bede om klip (video-, lyd-, bevægelse-, keyboardtryk- m.m.) fra tilfældigt udvalgte tidpunkter, og hvor brugeren så lige bør gennemse dem og sikre sig, at ingen af de adspugte klip er kompromiterende, og så får lov at veto'e et lille antal af dem, uden at de medfører en bøde. (Men dette kunne man jo betragte som "øvrige noter.") (mangler, men i 'øvrige noter')


\subsubsection{Flere noter om dette matematiske/semantiske programmerings-/web-paradigme generelt}
%Nu hvor vi så har snakket om det semantiske web, så lad mig lige vende tilbage til det matematiske (og semantiske) programmeringsparadigme generelt og nævne nogle flere punkter om dette. ...
Og det er ikke kun direkte i forbindelse med internettet, at mere matematisk programmering og flere semantiske objekter kan forbedre teknologien. Desktop-applikationer (og mobile-applikationer osv.) som vi kender dem kunne også begynde at bruge denne teknologi mere og mere. Ligesom med servernetværket kunne applikationer således også gøre sig selv mere åbne for brugerdeltagelse, når det kommer til applikationsdesignet og mulighederne for at justere applikationen til forskellige brugeres præferencer. De kunne således f.eks.\ også begynde at åbne op for, at brugere er med til at bestemme og justere brugerfladen efter deres behov, ligesom jeg tænker det med hjemmesiderne, og kunne også gøre tilsvarende for alle mulige andre moduler og features, der indgår i applikationen. På den måde kan applikationsdesignerne således også nærmest ``udlicitere'' arbejde til sin brugerbase og spare dette arbejde selv. 


Jeg vil også genre nævne, at jeg tror, det bliver sådan, at brugere får stor mulighed for at deltage i udviklingen af disse semantiske teknologier på mange fronter, også selvom de ikke har så meget teknisk færdighed hvad angår programmering og/eller matematik. Særligt udviklingen af de formelle sprog (altså dem der bygger over gængse NL-sprog) er et virkeligt vigtigt punkt, hvor nærmest alle kan være med til at deltage. Et godt eksempel kunne være til video- eller tekstoversættelser, hvor enhver bruger, der opdager en fejl eller mangel, så vælge at tilføje semantisk data til tekstudsnittet, som retter oversættelsen. Og et endnu mere centralt eksempel er, når man oversætter fra NL til formelt sprog. Her tror jeg virkeligt, det bliver vigtigt at kunne trække på den brede brugerskare til at hjælpe med at udvikle de formelle sprog. En bruger der opdager, at der er problemer med en oversættelse/processering af et NL-tekstudsnit, eksempelvis fordi der bruges en vending, der ikke er registreret helt, eller fordi, der konteksten gør at ellers tvetydige udtryk bliver ret entydige, så kan så hurtigt lige tilføje semantisk data, der beskriver disse forhold. Dette semantiske data kunne så være en formel sætning, der f.eks.\ siger at ``vendingen $r(x, y)$ kan oversættes med $s(x, y)$, når der gælder $p$ om den givne kontekst.'' Det sværre i denne process bliver nok så at danne en god måde, hvor brugere nemt kan sige noget om konteksten\ldots Et eksempel på $p$ her kunne i øvrigt være, ``at der ikke tidligere er nævnt noget om en specifik ko, der har haft mulighed for at gå ud på noget is.'' På den anden side bør sådanne forhold mere være noget, man generelt skal tage højde for, når sætninger med mulige vendinger i skal processeres. Så jeg tænker, at det i stedet nok bliver noget med at definere en god mængde semantik-flag, som brugeren kan lære sig, og så kan sætte omkring konteksten. Dette kunne så være flag så som et flag, der signalerer, at teksten godt kan indeholde slang af en vis art (altså fra en vis befolkningsgruppe og/eller tidsperiode osv.), samt flag, der signalerer emnet og temaet omkring den overordnede tekst --- er det f.eks.\ en videnskabelig tekst, og i så fald inden for hvilket område? Jeg tror sådanne kontekstflag muligvis kunne blive en god måde at sørge for, at brugere ret let kan komme godt i gang med at tilføje semantisk data til oversættelser frem og tilbage mellem et NL og det tilsvarende formelle sprog. Og når brugere kan generere en masse sådant data, så kan man så begynde at forfine NLP-algoritmer, så de bliver ret effektive (kunne jeg forestille mig). Og NL-til/fra-formelt-sprog-oversættelser er selvfølgelig rigtigt centrale for det semantiske web. Bemærk i øvrigt, at sådanne ``kontekst-flag'' også ville kunne bruges  i modsatte retning som det, jeg har kaldt `semantik-flag,' nemlig som nogle indledende, overordnede semantik-flag, der gør at man kan slippe for ligeså meget arbejde med at udvælge semantik-flag for enkelte tekstudsnit, når man udformer en samlet tekst i et formelt sprog. For tanken er, at hvis man starter med at sætte nogle gode kontekst-flag for sin tekst, så vil det hjælpe med at bestemme nogle tvetydigheder undervejs og vil derfor øge sandsynligheden for, at den pågældende NL-til-formelt-sprogs-oversættelsesapplikation vil gætte korrekt i første omgang. Nå ja, og man kan i øvrigt også, hvad der svarer til de individuelle semantik-flag, når man korregerer en oversættelse af en faktisk tekst på nettet/webbet. For brugerkorrektionerne til oversættelser er ikke bare gavnlige, fordi de tilføjer semantisk data at bruge af til at forfine NLP-algoritmer, men de er også gavnlige, simpelthen fordi de korregerer en faktisk tekst, som andre brugere også gerne vil læse og bruge. Og her kan det så være gavnligt, når brugere bare kan lave disse korrektioner simpelthen ved bare at sætte et semantik-flag for teksten; dette vil lette korrekturarbejdet en hel del, især når man bare er interesseret i at oversætte denne specifikke tekst, og ikke har tid til at give sig i kast med at prøve at finde frem til, hvorfor samme semantik-flag allerede var underforstået fra konteksten --- hvis den overhovedet er det; det kan jo også bare være, at teksten er tvetydig i bund og grund. I dette tilfælde må man så bare opgive de mulige semantik-flag, så efterfølgende læsere har adgang til alle de mulige fortolkninger af teksten. 

Selvom semantiske oversættelser selvfølgelig er ret centralt for det semantiske web, er der også mange andre områder, hvor den brede brugerskare kan bidrage rigtigt meget, også selv hvis de ikke har nogen specielle matematik- eller programmeringsfærdigheder. Vi snakker altså bl.a.\ kommentarer til og ratings af diverse indholdsobjekter, samt spørgsmål og forklaringer til eksisterende tekst i dette indhold. Vi snakker også brugerdata så som brugerflade-præferencer og lignende, som nogle brugere måske gerne vil dele med resten af fællesskabet for at give inspirationskilder. Det kan også være idéer til designmuligheder for applikationer og for webbet, hvor brugere så kan uploade (formelle) kommentarer på, hvad de synes kunne være gode tiltag. Og her indebærer `gode tiltag' selvfølgelig altid, at tiltagene er på en form, hvor det i sidste ende er brugerne selv, der vælger, hvilke forks af designet, de vil benytte. I øvrigt, når nu jeg nævner noget som bruger-ratings, så kan der jo være mange forskellige former for rating-systemer. Men det gode ved et semantisk web som dette, der grunder i matematik (og hvor brugere bruger matematiske browsere, som beskrevet), er jo netop, at brugere bare kan udforme deres egne rating-systemer, som kan udvikles (og forkes) løbende, således at hver bruger for mulighed for at vælge det rating-system, der passer til lige denne. På tilsvarende vis kan brugere også vælge deres egne filtre for, hvilke kommentarer, de vil se, osv.\ osv. 
%Og også bidrage med brugerdata generelt; ratings, kommentarer, forklaringer, spørgsmål til eksisterende tekster, bidrag med nye tekster og nyt indhold, idéer til designs og til udviklingen generelt, og deling af personlige præferencer til inspiration hos andre. (tjek)

Når det nu kommer til alle sådanne nogle tiltag, hvor brugere benytter hinanden til inspirationskilder og til at udvælge indhold osv., så vil jeg i øvrigt også gerne lige nævne, at jeg tror det bliver en virkeligt vigtig ting for udviklingen og brugbarheden af det semantiske web, at brugere også laver gode ontologier (dvs.\ hver bruger kan jo have deres egen ontologi, men kan bruge af alt arbejdet og al dataet, som andre brugere uploader) over selve brugernetværket, som prøver på at omfatte og klassificere så mange brugere som muligt. Ved så at have gode ontologier til at klassificere andre brugere kan folk som nævnt bedre filtrere kommentarer og brugervurderinger, så de er mest relevante for vedkommende som muligt. Og samtidigt, og dette er grunden til, at jeg lige ville understrege det, så er handler hele opbygningen af det semantiske web jo om at bruge andre brugeres data (til at opbygge relevante ontologier m.m.). Og fordi denne udvikling har stor gavn af, at så mange brugere som muligt deltager, så er det altså en rigtig central ting (ligesom at sprog-ontologien er det), at få bygget gode brugernetværk-ontologier i fælleskab (som hver enkel bruger så kan justere privat). Man kan i øvrigt også se det lidt fra den her vinkel: Tænk på, hvor mange mennesker skriver en tekst eller tilføjer andet indhold til webbet, som det er nu, hvor det nærmest bare forvinder ud i æteren; det bliver f.eks.\ måske bare lige læst af en håndfuld tilfældige mennesker. Og det kan være på trods af, at man oprigtigt føler, at man har noget værdifuldt at sige, der kan være brugbart for mange andre mennesker, og særligt måske den gruppe af mennesker, der deler interesser og behov (eller andre ting, alt efter hvad det drejer sig om) med en selv. Dette tilfælde kan mange nok forholde sig til. Men med et semantisk web kan man i første omgang bedre sørge for, at kommentarer og andre uploads ikke går til spilde (fordi de f.eks.\ var uploadet til en hjemmeside, hvor der ikke er så mange besøgende; det problem er der ikke længer i et semantisk web), og særligt når dette web så oven i købet har gode muligheder for at klassificere brugere ud fra alle mulige forskellige kvaliteter, jamen så kan man pludselig være meget mere sikker på, at nå ud til de brugere, der kunne finde ens upload relevant. (Og i et semantisk web, hvor kommentarer og andet indhold kan processeres semantisk, kan brugere i øvrigt også lettere få et overblik over relevante uploads, så selv hvis de ikke læser teksten direkte selv, så kan de stadig få gavn af den, hvilket altså også øger sandsynligheden for, at andre brugere får gavn af ens uploads.)
%Nævn lidt om brugerklassifikation. (tjek)
%Og måske at man altid kan bidrage; med f.eks. sprog-udvikling, og dette kunne især ske i forbindelse med undertekst- (og oversættelse-) korrektioner. (tjek)

Jeg har nogle flere ting, jeg gerne vil nævne i dette notesæt om matematisk/semantisk programmering/web. Jeg vil bl.a.\ gerne nævne et lille eksempel, hvor et matematisk/semantisk programmeringsparadigme kunne være gavnligt. Jeg har allerede nævnt nogle ting i flæng ovenfor for forskellige områder, hvor brugerpræferencer kan variere meget, og hvor man derfor har gavn af, at gøre brugerfladerne åbne for justeringer. Dette eksempel er lidt anderledes, da det omhandler valg af et specifikt design, hvor det ikke (som udgangspunkt\ldots) er meningen at brugerne skal ændre det igen, men hvor programdesigneren/erne har flere muligheder at vælge imellem. Jeg tænker på et eksempel, hvor en spildesigner skal vælge, hvordan spillerkarakteren skal bevæge sig. Selvom man godt kunne finde på mange andre eksempler også, så tror jeg bare, jeg vil nøjes med dette her. I et matematisk programmeringsparadigme kan spildesigneren så f.eks.\ skrive i dokumentationen, at ``spillerkarakteren skal kunne hoppe,'' og man kunne så eventuelt præcisere dette (hvis det ikke allerede er underforstået) ved at følge det med ``på en måde, som det generelt er kendt fra platformsgenren.'' Fint, men nu kræver det så, før denne dokumentation bliver opfyldt, at man vælger et specifikt bevægelsesmønster, for der er mange forskellige at vælge imellem (det ved enhver, der har spillet forskellige typer platformsspil). Min pointe er så bl.a.\ den, at en spildesigner, der måske meget har haft én løsning i tankerne --- eller måske ikke har givet det så mange tanker --- så får mulighed for med det samme, når dokumentationen er skrevet, at søge på løsningsforslag på nettet og få et overblik over, hvor mange forskellige løsninger (mere eller mindre konventionelle), der allerede eksisterer til problemet. Ydermere kan de forskellige løsningsforslag så have brugervurderinger m.m.\ tilknyttet sig, så man f.eks.\ som spiludvikler allerede med det samme kunne se detaljer om, hvilke nogle løsninger der passer bedst i forskellige sammenhænge; er det f.eks.\ et høj-præcisions platformspil, hvor brugeren gerne må føle en høj grad af kontrol med karakteren, og hvor det i stedet er omgivelserne der skal supplere udfordringerne, eller skal hoppet i sig selv være udfordrende at mestre (og skal det så være hurtigt-at-lære-svært-at-mestre-udfordrende (såsom Super Meat Boy), eller skal de bare være udfordrende generelt (såsom jeg anser Super Mario 1 for at være), eller kræver spillet ikke særlig høj præcision, så at bevægelsen derfor bare skal være glat og flot og se på (og med en glat følelse af det også)? Så pointen (også med at nævne alle disse muligheder) er så, at en designer så med et godt semantisk/matematisk web/programmeringsfællesskab lynhurtigt kan få indblik i alle disse overvejelser og lynhurtigt kan teste forskellige muligheder. Dette er i modsætning til hvis man har et normalt web, hvor alle sådan nogle løsninger ikke er tilgængelige (fordi dette ville kræve at nogen i første omgang uploadede deres løsninger til en hjemmeside, og så derefter at gængse søgemaskiner ville kunne finde frem til det (hvilket de jo ikke kan, og derfor er der heller ingen, der gider at uploade sådan nogle specifikke ting, uden at der først er en klar forespørgsel efter det)), og hvor man derfor skal programmere alle løsningerne selv. Så handler det om først at programmere én løsning, evaluere, finde ud af, at det ikke duede, som man havde regnet med, programmere en lidt anden løsning, evaluere, ændre lidt, overskide tidsplanen, kassere dette igen og starte forfra med en ny løsning\ldots\ Dette er en lidt pessimistisk, men ikke en usandsynlig forudsigelse (efter hvad jeg kan forestille mig --- især hvis vi lige ser bort fra at dette eksempel måske ikke er den mest tidskrævende proces til at begynde med (men ellers kunne man bare finde et andet, bedre eksempel, og alle tidstagende processer tæller jo op, så\ldots)) for, hvordan en typisk designproces kunne foregå. Jeg mener altså, at fremtidige programdesignere vil kunne spare en hel del tid vil at benytte det semantiske web, og at tidshorisonter dermed nemmere kan overholdes. 
%Nævn gerne mario-hop.. (tjek)
%Og nævn semantisk (tale-)input. (tja, der er vel egentligt ikke så meget at sige, og jeg har jo nævnt det allerede.. så er det ikke fint?)

Jeg vil også gerne nævne lidt om at bruge semantik-redskaber til at udforme og redigere dokumenter. For det første så kan et understøttende formelt sprog jo hjælpe med at sikre, at de sætninger, man formulerer, er tydelige, og at de kommunikerer det, man ønsker. Så allerede der kunne teknologien understøtte og forbedre processen, når man skal udforme forklarende og/eller opsummerende tekster m.m. Og dette kunne endda tages endnu videre, hvis man fik udviklet en måde at bruge de semantiske processeringsmuligheder til at hjælpe med at strukturere dokumenter. Man kunne således forstille sig, at man ville kunne skabe et bedre overblik over sammenhængen mellem de enkelte sætninger, både ved at holde styr på, hvilke emner (og underemner) de forklarer om, men også særligt hvilke nogle tidligere antagelser og (del-)konklusioner, de bygger over --- med andre ord hvilke nogle tidligere udsagn man skal forstå og/eller være enig med for at kunne forstå og/eller være enig med de pågældende udsagn. Hvis/når dette bliver muligt, tror jeg således, at man ville kunne opnå gode muligheder for at få overblik over, hvad man har forklaret --- og hvor tydeligt --- i sit dokument, og hvad man mangler at forklare. Ja, man kunne endda modellere sit dokument så at sige som en ontologi over de forskellige emner, teksten berører, og som så indeholder alle de udsagn, man argumenterer for eller antager (eller bare nævner). Og når det så kommer til udsagn, som man regner med at læseren allerede kender til og forstår, og muligvis er enige med, så kunne man jo også holde en slags grund-ontologi (over det relevante område), som beskriver, hvad læserne antages allerede at vide om emnet (og over hvilke antagelser, de forventes at være klar på at gøre automatisk, uden at forfatteren behøver at komme med yderligere argumenter herfor). Ja, så det kunne eventuelt være en gavnlig måde at gøre tingene på. Jeg forestiller mig endda også, at vi måske kan nå til et punkt, hvor tekstredigeringsprogrammet selv kan hjælpe med at stykke dokumentet sammen, og altså endda hjælpe med at finde frem til en god, overskuelig disposition. I første omgang ville dette nok have form af, at man kunne bede sit redigeringsprogram om at komme med forslag til omstruktureringer, men på sigt kunne man måske opnå mere end dette: Måske kunne man opnå, at man bare som forfatter starter med at udbygge en tekst-ontologi over, hvad man gerne vil fortælle/forklare, hvor man så bare kan tilføje udsagn i mere eller mindre vilkårlig rækkefølge, der beskriver disse ting. Redigeringsprogrammet kan så løbende give bud på tekstudkast, som man så kan udbygge og justere enten ved at redigere direkte i teksten eller ved at tilføje flere formelle udsagn til ontologien. Før at vi kan nå så lagt at et sådant programm ligefrem kan hjælpe brugere med at strukturere deres dokumenter automatisk, så kræver det godt nok, at man først udvikler en måde, hvor ordnen og overskueligheden kan analyseres og vurderes automatisk. Så det kan godt være, at der går noget tid før den del af teknologien bliver en mulighed. På den anden side kan det også være, at der findes en nem måde at gøre det på; en nem måde at parametrisere ordnen og læseligheden af dokumenter. Det må tiden vel vise. Men hvis dette lykkes, og hvis denne teknologi om at kunne hjælpe med at ordne dokumenter og sørge for, at meningen i dem er tydelig, så kunne det jo i øvrigt blive endnu en ting, der kunne accelerere udbredelsen af det semantiske web m.m., da dette så vil føre til flere semantisk formelle dokumenter på webbet, hvilket vil sige mere brugbar data, og vil så også skabe yderligere efterspørgsel efter teknologi til at processere udsagn semantisk. 
%Automatisk strukturering. (Overskuelig og let at følge, i modsætning til rodede tekster --- ordenspoint?...) (tjek)
%At tilføje enkelte udsagn lidt hulter til bulter, og så stykke / få stykket dem sammen bagefter. (tjek)
%Kunne nævne at sådanne programmer ville føre til mere semantisk data, som vil hjælpe udviklingen. (tejk)
 
%Mon ikke sem-doks bliver 'øvrige noter?' (Hm..? Nej, fint her!) - Ja, jeg kan starte med at sige, at det hjælper sætningsformuleringen, og så at det kan hjælpe til at give semantisk oversigt, og at man i sidste ende så også kan automatisere forslag (m.m.) til at ændre (eller danne) strukturen i en tekst (måske bare ud fra individuelle udsagn, der så stykkes sammen). (tjek)



%*(Nu fik jeg faktisk opklaret det, jeg ville komme ind på i denne sektion, i den efterfølgende (ikke-udkommenterede) sektion:)
%Til slut her vil jeg også lige prøve at understrege, hvorfor jeg mener at et matematisk grundlag er rigtigt gavnligt for det semantiske web. For mig handler det meget om, at ...
%%Hm...? %Ja, vigtigt også særligt det med NL...
%%Nævn modelérvoksmetafor. - Vigtigt med matematik, pga. det kan omformes, og ikke mindst fordi konventionerne er selvbeskrævne. Til det sidste kan man så bare bruge konventionelle renderinger af modellerne, til at give mening til det. Og noget andet vigtigt udover matematik, som også bidrager gevaldigt til dette princip at det er selvbeskrevet: Det at man (nok) for bygget et formelt sprog over NL: Det gør semantikken jo virkeligt "selvbeskrevet" effektivt set.



Til slut her kan jeg se, at jeg også lige bør forklare mere om ``ontologiernes'' opbygning. Jeg har jo fokuseret meget på det semantiske sprog i det ovenstående og ikke så meget på de matematiske strukturer, der ligger bag dem. Når man definere en formelt version af et naturligt sprog (eller en undermængde af et, hvad man jo vil gøre først), så kræver det jo en underlæggende formel (matematisk, som jeg tænker det) model af den verden, der danner ramme for alle de ting, man vil kunne beskrive med sproget. Der kan jo være mange tilgange til, hvordan man kan opbygge sådanne modeller/ontologier. Når vi tænker på det semantiske web, så handler teorien bare om basalt set at erklære så mange relationsvariable, som man har brug for, hvis navne så skal illustrere deres semantik, og så benytte disse til at opbygge ontologierne. Dette er jo en meget god start, for det er så bredt som det kan være; så kan relationsnavnene med deres tilhørende underforstået semantik jo udgøre de sprog-atomer, man har brug for, hvilke man jo så altid kan oversætte til og fra andre sprog-atom-sæt, når man finder og udvikler sådanne alternativer. Uanset hvilket atom-sæt man starter med, vil det så være en god idé så hurtigt som muligt, at få defineret nogle brugbare (og faktisk gerne ret brede/vage) relationer, som så kan benyttes til at bygge modeller op med samme lethed som OOP-miljøer eller UML-diagrammer. Det vil bl.a.\ sige, at man definerer ret brede relationer så som `indeholder'/`holder'/`har,' og hvad man ellers har brug for. På den måde kan gør man det let hurtigt at opbygge overordnede skitser af den verden, man vil beskrive (eller rettere som man vil bruge som ramme til at beskrive ting i), som man så sidenhen kan uddybe med mere præcise relationer i stedet (så at man altså præcisere de mere vage relationer med nogle mere specifikke). Ved så altid at bruge et velkendt naturligt sprog som mål for, hvad man gerne vil have ens sprog-model til at kunne beskrive, så opnår man derved også i samme omgang (altså udover alle de mere indlysende fordele ved at gøre dette), at det bliver en nem proces at smelte to sprog-modeller/-ontologier sammen. For hvis begge ontologier får formaliseret en delmængde af det samme sprog, så handler det jo bare om at finde fællesmængden af disse to delmængder, og så kan man udlede en oversættelses procedure direkte fra dette (hvilket jeg altså så tror vil kunne gøres nærmest automatisk, når først denne fællesmængde er identificeret). 
Ja, så det er ligesom det generelle indenfor, hvordan sådanne modeller/ontologier kan opbygges på et teknisk plan. Og så har jeg jo nævnt, hvor vigtigt det også er med modeller til at klassificere brugerne, så man bedre kan gøre brug af alle mulige bidrag her på tværs af hele det brede fællesskab. Og dette gælder selvfølgelig også her, så det er også noget af det første, man bør arbejde på. For det er jo et omfattende projekt, så jo flere bidragere man kan inddrage, jo bedre er det. Men hvis man så bare inddrager alle med lige meget at skulle have sagt, så vil der blive al for meget støj til at man kan komme nogen vegne. Og hvem skal så bestemme, hvem der får noget at skulle have sagt? Jamen, det er jo det smarte ved at bruge et så åbent browser-system som de ovenfor beskrevne ITP-browsere kan implementere, nemlig at folk selv kan sætte deres egene præferencer og dermed deres egne filtre for, hvad der benyttes eller ej. Dette gør også, at hvis ens bidrag nu i første omgang er blevet nedstemt på en måde, så det ikke generelt kommer med i de ontologier, som andre folk ser, men man ved, at det alligevel var et værdifuldt bidrag, så behøver man ikke at bekymre sig helt så meget. For så længe man kan blive ved med at gøre sine bidrag tilgængelige for resten af internettet, så vil folk jo med tiden opdage værdien af ens bidrag, begynde at adoptere det og i øvrigt derfra så også med al sandsynlighed begynde at analysere dine andre bidrag med større vægt/interesse. Og når nu vi taler om, hvordan ITP-browser-systemet kan være gavnligt for det semantiske web, hvilket jo er et meget passende emne at slutte sektionen af på, så er det jo ikke bare selve sprog-ontologierne, der er værd at udvikle, når det kommer til at opbygge grundlaget for det semantiske web, men også de platforme og brugerflader, man kan gøre det med. Så i takt med at brugere i fællesskabet for udviklet de omtalte UML-agtige meta-modeller til videre udvikling af ontologierne, så er det jo også værd at udvikle brugerflade-miljøer til at arbejde med disse meta-modeller. Brugerflader hvor man f.eks.\ kan få renderet sine (UML- og/eller OOP-agtige) modeller, ville jo være en indlysende fordel. Og her kan man så ikke bare benytte det samme princip, hvor man klassificerer brugerne, så hver enkle bruger får mulighed for at ``filtrere'' sig frem  til den mest brugbare brugerflade for vedkommende; man kan så også benytte refleksionsmulighederne i sit ITP-system til at verificere og køre disse brugerflade-løsninger. 
%(Jeg overvejede lige, om man lige skulle præcisere, at et brugerklassifikationssystem og alt det der selvfølgelig ikke kun er brugbart, når man skal opbygge modeller inden for allerede udformede meta-modeller, men også bør bruges til alle andre led i processen, inklusiv at udforme meta-modellerne, men dette er nok for indlysende i forvejen, til at det er værd at nævne. Så nu skriver jeg det altså bare lige som denne kommentar.)




%Sektion om ITP-browser (og gentag net-antagelser). (tjek)
%Og videre: Hvordan man bedre kan have servere til at køre brugerlavede programmer.. (tjek) Og hvordan dette så kan benyttes til at skifte fokuset på nettet, så der er tale mere objekter man får serveret frem for bare hjemmesider. (tjek) Og så dette med at danne sine egen interface-præferencer (tjek) og det med at benytte brugerklassifikationer til at hjælpe med at bruge (semantiske) kommentarer og anbefalinger (og om o.s.-søge-algoritmer generelt). (mangler) Jeg bør også lige fokusere særligt på designprincipper omkring interfaces og sådan.. (jeg har nævnt nogle ting i flæng..) Og her kunne jeg måske tangere lidt og nævne det med mario-hop osv.. (mangler) Og husk at nævn, at objekter og data stadig godt kan være mere eller mindre closed og muligvis bag betalingsmure. (behøver vist ikke her alligevel) Her kunne man så også passende komme ind på, at dette internet system også kan bruge formelle protokoller, og at dette kan bruges til at give større frihed for brugerne, og få dem mere aktive derved --- også fordi man sikkert kan belønne dem bedre (muligvis) som firma/forening. (bliver nok udskudt til "øvrige noter") Og sådanne mere åbne firmaer vil så dermed sikkert endda kunne konkurrere med \sout{andre store firmaer} konventionelle løsninger. (bliver nok udskudt til "øvrige noter")



% Husk:
	% - Fra papir-noter: "Designe via sprog-kommandoer." Har jeg dækket det? (tja, jeg har vist nævnt at semantisk input kunne være tale-input... Ja, det må være fint nok; det er en ret triviel vision..)
		%(
			% - Fra mine papir-noter er også et punkt, der lidt kortere kan skrives: "Semantiske (og dynamisk opdatérbare) moduler og interfaces til konventionelle applikationer også."
		%)
		% - Ontologier her? ..Ja, det tror jeg faktisk; så for jeg da nok at skrive om.. *Tja, eller i en lille sektion for sig..
		% - Husk platforms-(hoppe-)eksempel! ... Ja og inkludér for resten bare så meget som muligt fra "omni-side"-halløjet som muligt (og som giver mening); hvis det relaterer sig meget til programmering, er det bedre at have det i denne sektion. *Tja, måske, eller ikke... Jo, måske faktisk...
		% - Har jeg nævnt noget om semantisk programmering, hvor man bare taler sine inputs? Nej (tror ikke), det må jeg lige gøre.
		% - sem-doks.
		% - Anonymitet/sikkerhed, når man efterspørger objekter på tværs af udbydere.
		% - Uh, husk!: Jeg kom lige til at tænke på: Det er vigtigt at pointere, at meta-antagelserne ikke skal ses helt som et aksiomsæt, for to brugere der har forskellige antagelser, skal alligevel sagtens kunne dele beviser. Så det handler nemlig derfor om ikke at betragte dem som matematiske aksiomer, men som antagelser om, hvordan ens maskine kan bygge og verificere beviser, og sørge for at gemme sætninger korrekt osv. Ja, vigtigt at pointere. 

%Faktisk brainstorm (om matematisk programmering):
%Ja, så en pointe er, at matematiske invarianter kan være lettere at formulere end imperative programmer. ..Hvilket er rigtigt smart, når man så kan kompilere matematiske formuleringer.. Og IDE'en kunne give forslag til forskellige design patterns.. Hm, og autogenererede templates.. Og hvor brugeren så bare skal udfylde forskellige specifikationer og udforme eventuelle beviser, der ikke kan udformes helt automatisk (men hvor template'et måske kan give en generel bevis-guide..).. Easy to import new templates and design patterns.. Easy testing and peer-reviewing processes --- which also makes for a more valuable ``final'' product.. *(Also increases value because solutions can be semantically categorized and thus be made easier to search for.. *And also because the ``final'' product is easier to update and maintain (as I think it will be)..) More precise semantic communication and requirement specification across large (or smaller) projects. (Because the overall proof needs it, holes in the specifications will be easier to detect.).. Ja, og jeg kunne også tilføje til denne liste, at mere formaliserede procedurer/protokoller også gør det nemmere at tjekke, at man har gjort tingene ordenligt (når man f.eks. tester), og gør det så også nemmere at advicere det til klienter/brugere/købere (ved at kunne fremvise et certifikat eller ved at have underskrevet en (konventionel) procedure-rapport selv). I øvrigt kan man så som ledere af et projekt i det hele taget nemmere gøre programmerings-/testing-procedurer til en eksplicit del af requirement specifikations'ne.. 


\subsubsection{Kryptovaluta}
*[(16.05.21) Jeg er ikke længere helt så optaget af / spændt på udviklingen af KV-systemer nu, som jeg var, da jeg skrev denne sektion. Så sektionen her er nok ikke så vigtig, som jeg tænkte, da jeg skrev den (medmindre jeg tager fejl nu og vil skifte mening igen på et tidspunkt).] *[(03.06.21) Jo, nu har lidt skiftet mening igen; KV-systemer er lidt interessante. Jeg er nemlig kommet frem til, at det muligvis kunne være interessant med muligheden om at bruge et KV-system som en investeringsfond. (Se til sidst i sektionen med \textbf{Noter omkring muligheder for det fremtidige marked generelt} nedenfor i dette notesæt (samt de ting, der refereres tilbage til, inklusiv ting nedenfor i denne sektion).)] *[(23.06.21) Og nu har jeg skiftet mening igen. Jeg tror ikke rigtigt på vigtigheden af mine idéer til KV-løsninger. Jeg tør dog ikke at udkommentere denne sektion, for jeg er ret sikker på, at jeg refererer tilbage til den flere steder (og altså i forbindelse med områder jeg stadig mener er brugbare).]

Og når vi nu taler om det fremtidige net, så kan jeg jo passende nævne nogle tanker, jeg har omkring den fremtidige udvikling af blockchain og kryptovalutaer. Her tror jeg nemlig også, man kan gøre fremskridt ved at bruge matematikken i højere grad. Jeg forestiller mig således, at man må kunne basere en kryptovaluta i en matematisk model, hvor man således bare erklærer en liste over, hvad der svarer til meta-antagelser, bare hvor disse er antagelser omkring hele den samlede mængde data på internettet i stedet for bare omkring et enkelt mappesystem for en bruger. Tja, eller nu hvor jeg skriver det, så kan en ITP-bruger jo også indirekte antage forhold omkring internettet ved at antage, hvad der kan forekomme i download-mappen, så forskellen ligger nok mere bare i, at kryptovaluta- (KV-)antagelserne gøres i fælleskab for det nye KV-system. Og så er yderligere forskel også, at det måske kan være gavnligt for KV-antagelserne at indføre en tidsparameter, sådan at semantikken bag modellen er, at hvis man vil vide hvad der gælder for KV-systemet på et givet tidspunkt i virkeligheden, så skal man formelt set formulere sine spørgsmål som propositioner og så se, om de er propositioner af modellen (der defineres af antagelserne), når tidsparameteren er sat til det pågældende tidspunkt. På denne måde slipper man for at skulle finde på nogen tricks til at måle tidspunktet i KV-antagelserne (ved at bruge certifikater og pålidelige kilder osv.), men gør det bare til en grundlæggende del af fortolkningen af KV-systemet. 

Kryptovaluta-antagelserne skal så i grunden primært gøre to ting, som jeg forestiller mig det. De skal for det første definere en fordelingsfunktion (eller relation; det er ikke så vigtigt, hvordan man modellere det), der til ethvert tidspunkt kan afgøre, hvor mange penge der er på hver konto (som jo bare hver især har et ID i form af en binær streng), ved at tage den samlede mængde offentliggjorte data på internettet som input. Hvad der præcis betegner denne mængde til et givent tidspunkt er heldigvis ikke så vigtigt, og det kan defineres på flere måder. Og man kan måske endda fint slippe afsted med at definere det lidt løst. For eftersom der ikke vil være nogen stake-holdere, der er interesserede i at lade KV-systemet udvikle sig, så en sådan tvetydighed bliver betydende, så vil det heller ikke forekomme. Denne pengefordelingsfunktion kommer altså så til, så længe denne funktion er gældende, at beskrive protokollen for, hvordan transaktioner foregår i KV-systemet. Man kunne f.eks.\ have en pengefordelingsfunktion, der i starten bare beskriver en blockchain-protokol tilsvarende en af dem, vi allerede kender. Den anden store ting, som KV-modellen/KV-antagelserne skal gøre, er at definere en protokol for, hvordan man løbende kan opdatere KV-modellen. Måden man kan implementere dette på er selvfølgelig så at gøre KV-meta-antagelserne åbne lidt på samme måde som for ITP-systemerne, således at de definerer måder, hvorpå nye propositioner kan tilføjes modellen, og i dette tilfælde snakker vi så særligt her om propositioner, der definerer KV-systemet til det pågældende tidspunkt. Hvis jeg lige skulle give et hurtigt bud på, hvordan man kunne udforme sine protokoller for at opdatere KV-systemet, så ville det være noget i retning af at inddele, hvad der kan stemmes om, i forskellige klasser, således at forslag til at ændre, hvordan transaktion-protokollen er, udgør én klasse, hvad der betales til miners udgør en anden, hvordan man ændre afstemnings- og vedtagelsesprotokollerne for at ændre protokollerne for disse nedre klasser, som i øvrigt begge beskrives via fordelingsfunktionen, udgør en tredje klasse (højerestående, om man vil), og i sidste ende, foruden eventuelle andre klasser, som man selv kan tænke sig til, kan der så muligvis være en overordnet klasse, der udgøres af, hvordan man stemmer og vedtager ændringer for de KV-antagelser, der definerer hele det praktiske system, inklusiv de antagelser om, hvordan man kan ændre selve disse helt grundlæggende antagelser. Så ved at slutte rækken af med sådanne praktisk set selvrefererende antagelser, så opnår man, at der ingen øvre grænse for, hvor meget systemet kan opdateres (på formel vis), når størstedelen af aktive brugere går med på det. Det kan også være, at dette hurlumhej ikke er nødvendigt, men det var bare et forslag --- ja, det var det hele jo sådan set bare. Når jeg mener, at en sådan ``selvreference'' ikke bliver problematisk på et logisk plan, så er det fordi, det jo bare svarer til at have en grundlov, der kan fornyes gang på gang, hvor man for hver ny grundlov så bare sørger for, at denne også selv inkludere en ny protokol for at stemme om at forny den efter en vis tid. Og så handler det jo bare om, at stole på at folk ligesom fastholder denne tradition, og der er altså ingen logisk modstrid ved sådan en selvfornyende proces. Nå, anyway\ldots\ Tanken med disse forskellige klasser er jo så selvfølgelig, at de skal have forskellige grader af, hvor ofte der kan stemmes, og hvor meget der skal til for at vedtage nye forslag, hvor mange underskrifter skal til for at sætte gang i en større afstemningsproces osv.\ osv. Her er det så bare vigtigt, at man altid kun kigger på ``aktive brugere,'' så kæden ikke forgår, hvis en for stor andel af den oprindelige pengemængde går tabt. Og når det kommer til, hvordan stemmeretten fordeles, så ville jeg foreslå den mulighed, at den bare fordeles proportionelt med kontoernes pengebeholdning. For når man investerer i sådan et system, der kan ændres ud fra andre deltageres stemmer, så giver det jo god mening at vælge et stemmesystem, så at indstemte ændringer i reglen altid vil være nogle, som folk mener, vil forøge kryptovalutaens værdi. Og dette opnår man jo nok med meget stor sandsynlighed, hvis stemmeretten er proportionel med, hvor meget stake man har i systemet. Man kunne så til gengæld spørge, om ikke sådan et system kunne føre til at majoritetsvælde, og det er selvfølgelig altid værd at have i tankerne. Men her tror jeg dog bare, at så snart velhavende deltagere begynder at rotte sig sammen og lave unfair beslutninger, som tilgodeser dem selv, vil være det tidspunkt, hvorefter valutaen lynhurtigt vil begynde at gå i sænk, og folk vil migrere til andre systemer. Men dette er jo selvfølgelig også noget i sig selv, man gerne vil undgå, og derfor er det også smartest, tror jeg, fra stake-holdernes side at sørge for allerede på et tidligt tidspunkt at forsøge at sætte restriktioner op, der forhindre sådanne typer tiltag. For hvis systemet indeholder restriktioner, der gør det sværere at indføre korrupte ændringer (f.eks.\ hvis man vedtager, at der aldrig --- eller i det mindste for en lang periode --- må indføres mekanismer til omfordeling af penge, det kunne være én ting, der ville være meget god at indstemme/fastlægge på et tidligt tidspunkt), så gør man jo KV-systemet så meget desto mere attraktivt for nye købere. Og bemærk i øvrigt nemlig, at bare fordi man baserer KV-systemet i nogle ret åbne meta-antagelser i starten, så kan man sagtens sørge for i samme antagelser, at denne åbenhed kan lukkes mere og mere som systemet udvikler sig, enten for en midlertidig periode, som skal fornys indimellem, eller sågar permanent, hvis man virkeligt ønsker dette. 

Om denne idé til et KV-system implementeres som en helt ny KV eller som en state channel på et eksisterende KV-system (hvis dette er i stand til at bære sådan en state channel på en effektiv måde) er sådan set underordnet; det kommer bare an på, hvad brugerne helst vil. Nogle vil måske prøve at bruge idéerne til at sætte fut i helt nye systemer, og andre vil måske bruge dem til at udbygge og forbedre mulighederne for eksisterende KV-systemer; jeg kan forestille mig begge ting. Og når det kommer til de mulige fordele ved sådanne typer KV-systemer, så er det jo i første omgang, at det nok bliver nemt at opdatere systemet løbende i henhold protokoller a la dem, jeg lige har beskrevet og givet nogle hurtige, overordnede bud på. Og det gode ved denne form for opdateringsprotokol er, at den er beskrevet formelt i selve systemet, og er altså således en helt central del i systemet. For i princippet kan deltagere i ethvert KV-system jo sagtens mødes eller skrives sammen i virkeligheden og beslutte sig for først en stemmeproces og dernæst et sæt af ændringer, som skal implementeres i systemet, inklusiv eventuelle ændringer i fortolkning, hvordan man bør fortolke systemet. Men dette ville jo være en svær proces, hvis man skulle gøre det spontant, især i et fælleskab, hvor sådanne typer processer ikke på forhånd indgår. Men når processen derimod allerede fra start er formelt vedtaget i systemet, klart beskrevet som en del af KV-antagelserne, så skriver alle deltagere jo implicit under på disse processer, og det vil således ikke skabe hverken splid eller forvirring, hvis en afstemningsproces bliver udløst af visse forhold, selv nogle der er uforudsete. For man har jo allerede givet sit samtykke til denne proces ved at investere i den.

Og når vi taler om fordele ved at bruge matematik i et KV-system, så vil jeg også gerne nævne, at jeg også tror, matematisk formulerede smart-kontrakter kunne blive en rigtig vigtig del af den fremtidige KV-teknologi. Bemærk at for den type KV-system jeg har beskrevet her, der er transaktionsprotokollerne allerede beskrevet matematisk på et grundlæggende plan, så her ville det jo være oplagt også at bruge matematik som grundlag for smart-kontrakterne. Og når man gør dette, så kommer man jo til, hvis mine forestillinger holder i høj nok grad, at kunne opnå mange af de samme fordele, som jeg har beskrevet ved det semantiske web ovenfor. Her tænker jeg særligt på, at man løbende kan vise mere effektive metoder til at bruge refleksionsprogrammer til at bestemme propositioner, og endnu vigtigere at man kan udbygge kontraktsprog-ontologien mere og mere, og endda samtidigt forsøge at forene den mere og mere med naturlige sprog (ligesom at jeg tænker, det kunne ske for det semantiske web), så flere brugere kan være med til at formulere nye former for kontrakter frit efter behov. For én ting er at KV-systemet jo gerne skal udbygge en datastruktur over transaktionshistorikken, men der er jo ingen ting til hinder for, at man også udbygger andre datastrukturer i fællesskabet, eksempelvis ontologier med tilhørende prædikat-/relationsdefinitioner, så man på den måde kan udbygge ordforrådet i systemet, hvormed bl.a.\ kontrakter kan formuleres. Jeg tror således, at dette kan komme til at betyde alverden for brugbarheden af KV-systemerne; hvis først man for dannet et semantisk KV-system, så tror jeg ikke, der er nogen, der vil se sig tilbage derfra.

%(02.04.21) Uh, jeg kome lige i tanke om (eller den idé, men det kan vist godt være, at jeg har haft en lignende idé før..), at NL-aksiomer godt i princippet kunne være en del af grund-semantikken bag fortolkningen af KV-systemet. Det må jeg bestemt også nævne. Og det passer fint at nævne det nu her i den følgende sektion, som jeg alligevel ville skrive omkring det at bruge ekstern real-world semantik via dommergrupper m.m. i til kontrakterne også. 
%(03.04.21) Nej, jeg har vist ikke haft helt den idé før. Jeg tænkte lidt videre over det i går aftes og kom frem til et par ting, bl.a. dette. For det er jo derfor, jeg har været så optaget i lang tid om at se på, hvordan man kunne danne en proces, hvor man bliver ved med at bevæge sig mod bedre og bedre sandhedsontologi. Og det er nemlig først, da jeg kom på min nye idé til den økonomiske bevægelse, hvor man jo også bruger faktisk, virkelig sandhed i stedet for at prøve at danne en proces, man kan regne med altid vil bevæge sig imod denne i sidste ende. Så ja, dette er en helt ny blockchain-/KV-system-idé. Og så har jeg nemlig også tænkt lidt videre, og måske er det faktisk en god idé.. For jeg tænkte ellers lidt at, nå ja, man ville jo alligevel skulle finde en digital proces i praksis, der kan vurdere sandheden; ja, men forskellen er, at denne proces så ikke er endegyldig. Den kan erstattes, så snart man finder en fejl. Og alle vil i princippet have gavn af, at erstatte den, hvis samme alle godt ved, hvad klokken egentligt er slået, for hvad man betragter som virkelig sandhed, vil jo bare konvergerer mod virkelig sandhed. Dette er selvfølgelig medmindre, man simpelthen mister data, men så var det jo, jeg kom på i går (selvom denne idé jo også svarer lidt til tidligere idéer, men nu er der altså bare lidt et nyt spin på dem), at man bare fra start vedtager, at når værdi er splittet imellem et udsagn, som kun kan tillægges en sandsynlighed, jamen så splittes værdien bare i henhold til denne sandsynlighed. Dette giver så mening, for hvis man så ikke ved, om der kan dukke data op senere, som kan hjælpe med at bestemme spørgsmålet yderligere, eller hvis man ved at det sker, men ikke kender dataet endnu, jamen så er det jo helt samme billede; det naturlige er alligevel så, at værdien så splittes al efter sandsynlighed. Og så kan man jo i øvrigt bare forsikringsfirmaer, når man gerne vil sælge en risiko forbundet med en kontrakt m.m. Så ja... Det er faktisk en ret god idé det her..! I tror jeg så lige jeg vil starte med at tænke lidt mere over betydningen af denne idé (skal jeg så separere den fra sem-web-idéen igen, eller hvad?), og så vil jeg komme tilbage til skriveriet.
%Okay, jeg tror faktisk ikke, det kommer til at ændre helt vildt meget. Jeg kan nok egentligt godt bare skrive om begge løsninger som planlagt her i slutningen disse sem-web-m.m.-noter. Der er lige det ved det, at denne nye form for KV ikke kan bæres af eksisterende kæder i en state channel. Men netop af denne grund er det jo nok ret godt om ikke andet lige at nævne det. For det gør nemlig stadig ikke, at jeg behøver at sælge denne KV-idé mere end højst nødvendigt. Jeg kan sagtens alligevel bare nævne det hele ret kortfattet.

Og nu skrev jeg ``forene det mere og mere med naturlige sprog'' lige her ovenfor, men hvor meget kan man egentligt forene den tekniske ontologi, der jo udspringer af meta-antagelserne og den fortolkning, der ligger bag disse, med et naturligt sprog? %Jo, som jeg lige kom i tanke om, mens jeg skrev dette, så kan det jo faktisk i princippet godt lade sig gøre at forene det helt, hvis man simpelthen starter med KV-antagelser samt tilhørende semantisk fortolkning af dem, der er avanceret nok. Man kunne f.eks.\ i princippet starte med intet mindre end antagelser, der er formuleret i et naturligt sprog, og hvor man så allerede fra start af benytter en konsensus på tværs af fællesskabet om, at sætninger i pågældende naturlige sprog, som er formuleret, så de er helt entydige og forståelige for enhver, der gør et oprigtigt forsøg på at tolke dem (og som har almindelige eller bedre mental kapacitet ift.\ gennemsnittet) kan indgå i propositioner, hvorved en sandhedsbestemmelse af sådanne propositioner i henhold til KV-modellen (til det givne tidspunkt), så afhænger af den faktiske sandhed af disse NL-sætninger. Dette er altså ikke en umulighed. Spørgsmålet er dog, om det er specielt gavnligt for sådan et KV-system, for der findes også andre løsninger på, hvordan man kan vurdere sandheden af NL-sætninger som en del af kontrakter. Lad mig beskrive en sådan anden løsning her, hvad jeg også oprindeligt havde tænkt mig at gøre i denne paragraf (men nu bliver det lige i følgende paragraf i stedet), og så kan jeg lige vende tilbage til den førstnævnte løsningsstrategi til sidst og sammenligne de to lidt.\footnote{
%	Jeg har i skrivende stund lige haft lille dags pause (en aften og en morgen rettere) til at tænke lidt, og det er faktisk en ret god idé, den her med at basere KV-systemfortolkningen (hvilken fortolkningen af pengefordelingen grunder i) i en entydig undermængde af et naturligt sprog, så kontrakters udfald kan afhænge (hvis idéen altså holder) af den virkelige sandhed af sådanne sætninger. Det er i hvert fald værd lige at nævne. (Og jeg har også en idé til, hvad man gør, når ellers entydige sætninger bare ikke kan afgøres.) Så kan en sådan KV bare ikke længere bæres i af noget eksisterende KV-system (i en såkaldt state channel), men systemet skal grundlægges som et selvstændigt system. Men dette faktum gør på den anden side kun idéen ekstra interessant at overveje mulighederne af, så jeg synes altså helt klart, den er værd at nævne (hvad jeg derfor vil gøre senere i denne sektion).
%}
%
%Hvis man har et KV-system, hvor der er mulighed for at skrive kontrakter via matematiske formularer, og hvor der bag dette formelle sprog i det mindste er mulighed for at besvare alle spørgsmål, der kan bestemmes matematisk givet det tilgængelige data, så kan man stadig godt nå videre til et punkt, hvor man også effektivt set kan bestemme andre spørgsmål, der omhandler verden generelt og ikke bare den data, der er tilgængelig på internettet. Der er ikke noget skelsættende ved dette; det handler bare om at lave en organisation eller flere, der kan hyres til på upartisk vis at dømme sådanne kontrakter. Dette princip er allerede velkendt for eksisterende teknologier, nemlig at bruge tredjeparter til at bedømme spørgsmål, der ikke ellers kan bedømmes. Den eneste forskel her er så %...Hm, det er at man kan udforme alsidige sætninger, men er det ikke også nærmest for trivielt til at nævne? Jeg ville jo gerne nævne det, så jeg lige kunne bemærke også, at aktierne for sådanne dommerfirmaer kunne indgå i systemet, men... Hm, måske hvis jeg slår ned på om pointe om.. en pointe om at værdien af systemet så også hænger sammen med aktierne af sådanne organisationer, der holder til på systemet...? 
Tja, da jeg begyndte at skrive denne sektion tænkte jeg bare, at man til at besvare de spørgsmål, der ligger udenfor den semantiske rækkevidde af meta-antagelserne, bare kunne benytte sig af tredjepartsorganisationer/-firmaer/-foreninger til at bedømme en mere alsidige mængde spørgsmål, der så kan formuleres som NL-sætninger. To parter kunne således skrive en kontrakt med en NL-formulering i, hvor man så har hyret en tredjepart til at bedømme dette, når kontrakten skal aktiveres, og hvor man så selvfølgelig benytter et særligt certifikat fra tredjeparten. Dette er der jo ikke noget nyt i; princippet bliver også brugt i gængse KV-systemer (så vidt jeg ved). Dog tænker jeg, at det vil hjælpe denne proces, at man har et formaliseret version af NL-sprog, så man kan udforme kontrakterne præcis. Samtidigt kan det også godt være, at man kan spare noget fremtidig bedømmelsesarbejde ved at sætningerne er formuleret matematisk, for man kan jo tænke sig, at en tredjepart-dommerorganisation også sørger for at udvikle og vedligeholde en (måske ML/AI-agtig) ontologi, der kan hjælpe med at besvare spørgsmål, der ikke er alt for komplicerede. Dette ville jo ikke være helt dumt, i hvert fald ikke som jeg kan forestille mig det, og dermed har vi endnu et punkt, hvor matematisk formalisme kan være gavnlig for KV-teknologier. Inden jeg beskriver min alternative idé i næste paragraf, så vil jeg også lige foreslå muligheden for, at man også kunne have et (ret centraliseret) KV-system, der udspringer af en bestemt tredjepartsdommer-organisation, hvor organisationen som udgangspunkt altid står for at afvikle og bedømme kontrakter, og hvor værdien i valutaen så ligger i, hvor brugbare organisationen formår at gøre disse kontraktmuligheder, samt selvfølgelig hvor lille de kan gøre risikoen for, at systemet kollapser (og organisationen går konkurs), inden kontrakterne kan afvikles og veksles. Jeg ved godt, at denne version af idéen basalt set bare beskriver en bank, der så bruger KV-teknologi til at opbevare penge og afvikle transaktioner og kontrakter, men jeg synes alligevel muligheden var værd at nævne. I øvrigt kunne en sådan organisation så også muligvis koble sine aktier til selve KV-systemets valuta, så de kan handles digitalt herved, da det jo ikke kan skade at understøtte en valuta ved at koble værdier til den. I øvrigt kunne en mindre centraliseret version af dette system være, hvor organisationen erstattes af det samlede fælleskab på en kæde, der så vedtager en måde at styre organisationen på via afstamninger osv., og som så fra start fastlægger en kurs for, hvordan man skal gå i gang med at udbygge de (ML/AI-agtige) %(Jeg mener jo 'prædiktive ontologier,' men jeg har ikke lyst til at introducere det begreb)
ontologier, der skal være grundlag for at vurdere kontrakter m.m.\ i systemet. %Bemærk at det smarte ved at aftale sådan en kurs i KV-systemet, er at det kan give brugerne mulighed for, at udforme 
Denne version er altså inden for rammerene af det KV-system, jeg startede med at beskrive, men bare hvor at fællesskabet så selv sørger for også at tage ansvar for at oprette og styre de tredjepartsdommer-foreninger, som alligevel er nødvendige for at systemet kan opnås sit fulde potentiale.

Okay, men jeg har også lige en idé til et alternativt KV-system, som jeg fik nu her imens jeg skrev ovenstående i denne sektion.\footnote{
	Idéen udspringer godt nok ret naturligt fra andre idéer, jeg har fået og arbejdet med på det seneste, så det er altså ikke fordi, det var sådan en lynnedslagsagtig idé, jeg fik her. Men det er selvfølgelig altid dejligt lige at få vendt nogle idéer, så de passer på et nyt område også.
} 
For man kunne jo faktisk også godt forsøge at forene semantikken bag meta-antagelserne helt med naturlige sprog, forstået på den måde at selve den grundlæggende fortolkning af hele KV-systemet, inkl.\ pengefordelingen, kan komme til at afhænge af fortolkningen af faktiske NL-sætninger og, ikke mindst, af vurderinger af disse. Hvis vi sammenligner med det andet KV-system, jeg har beskrevet overfor, eller med gængse kryptovalutaer for den sags skyld, så afhænger disse jo alle af en konsensus om, hvordan pengene fordeler sig i systemet. For gængse KV-systemer er der en meget simpel fortolkning, der ikke er til at tage fejl af, og for det overfor beskrevne KV-system kan pengefordelingen potentielt godt bunde i nogle mere komplicerede forhold, men tilgengæld vil der være en klar matematisk model over denne udregning, der heller ikke er til at tage fejl af. For det system jeg beskriver nu, handler det ligeledes også bare om at opnå en fortolkningskonsensus af pengefordelingen. Forskellen er så bare, at man tillader at denne konsensus godt kan %... Hm, men det er da ikke alle spørgsmål man kan besvare alligevel, også selv hvis man bruger princip om at bruge sandsynlighedsvurderingerne til at splitte værdien, når det ser ud til, at man aldrig får data til at besvare spørgsmålet (hvilket nemlig reversibel handling, hvis man så finder data alligevel, hvilket ikke gør noget, for man kan altid sælge risici fra sig, så disse fordeles jævnt og kan kancelere hinanden, om man vil), for det kan vel komme til at afhænge af, hvilke nogle grundaksiomer, man bruger for verden?.. Ja, for tænkt det som priors: Folk kan jo være uenige om universets prior-sandsynligheder. Hm, hvad dælen gør man så lige... ..He, man kunne jo bare blive enige om at antage CUH.;)... Hm, hvorfor egentligt ikke?..!.. Hm, ja okay, så man behøver ikke at bekymre sig så meget om univers-priors, men kan der være andre... Nej. Der kan ikke være uenighed omkring andre grundlæggende verdensforståelses-aksiomer.. nej, rigtigt nok.. for selvom folk godt kan være uenige om p-ontologier i lang tid, så er det jo kun så længe, de ikke har teknologien til at gøre det mere fundamentalt (for selv vores psykologi kan jo også i princippet kortlægges, sammen med alt muligt andet), og samtidigt undgår vi jo også netop, at folks forskellige fortolkninger af begreber, og af virkeligheden i sig selv, bliver et problem, for dette kan man jo altsammen rede ud ved at bruge et formelt, præcist sprog... Okay, man kan jo godt nok ikke præcisere det formelle sprog helt, så der vil jo altid være nogen tvetydigheder, man ikke har opdaget og/eller fået bugt med, når man udformer kontrakter. Hm, og så har vi det problem, at det ikke betyder så meget for brugere, hvis der viser sig, at være semantiske huller i en kontrakt, men en ophobning af sådanne ubesvarede spørgsmål kan blive problematisk for systemet. Hm... Uh, nå ja, men på den anden side kan man jo altid så bare lave en vurdering af, hvad man tror, meningen var i den oprindelige kontrakt. Jeg så man skal lige have den vurdering over, når man i sidste ende skal lave en vurdering. Så samlede regnestykke bliver basalt set bare, antaget CUH (eller en anden fornuftig univers-prior, hvis man vil), hvad er sandsynlighedsrummet for de forskellige oprindelige betydninger af kontrakterne (altså hvad var den oprindelige hensigt), og for hvert af disse, hvad er sandsynligheden for, at kontraktsætningen var sand eller falsk, og så skal man så bare til hver en tid fortolke pengene fordelt, så de er splittet i henhold til dette sandsynlighedsrum. Og så kunne man så også med fordel lige tilføje en formel måde at ophæve risikoen ved en kontrakt, så at ingen nogensinde efterfølgende behøver at genoverveje kontrakten efter det punkt, både praktisk set, eftersom de ikke har betydning derefter på pengefordelingen, og også formalt set (så man ikke formelt er tvunget til at gøre dette alligevel). Den sikre måde at to parter kan afvikle en kontrakt endeligt er jo selvfølgelig bare, at udstede en modpartskontrakt, der benytter helt samme udsagn, men hvor man så sørger for at alle udsving i sandhedsvurderingerne bare kancellere alle tilsvanrende udsving i den oprindelige kontrakt, så de resulterende udsving i pengefordelingen forsvinder. Nummer to kontrakt kan altså således neutralisere den første kontrakt, så den samlede transaktion altid vil være konstant for eftertiden. 
afhænge af mere abstrakte forhold, nemlig af virkelig sandhed omkring verden. Jeg forestiller mig således et system, hvor brugere kan udforme kontrakter, men hvor sandhedsbestemmelsen af udsagnene, der udgør kontrakterne, ikke bestemmes i en formel proces, hvorefter man så vedtager en transaktion på baggrund af dem, men hvor hver kontrakt i stedet forvandles til to værdipapirer, som så løbende kan ændre værdi ud fra, hvad man finder ud af omkring udsagnet i det. Det vil med andre ord sige, at man i stedet for at finde en proces til at afgøre spørgsmål, der kan være uenighed om, så i stedet lader sådanne mulige uenigheder være velkommen og bare betragter dem som en naturlig del af systemet. For i den virkelige verden kan man jo også sagtens have uenigheder om, hvor meget visse værdipapirer er værd, uden at det ødelægger det økonomiske system. Der sker bare det, at det tilføjer en smule risiko. Og deltagere, der ejer værdipapirer med en risiko på sig, kan jo altid gå sammen om at forsikre hinanden imod sådanne risici. Idéen er så, at man i KV-systemet til hver en tid anser den grundlæggende værdi af et kontrakt-værdipapir (også selvom man ikke er sikker på, at man kan regne denne værdi ud korrekt på pågældende tidspunkt) for at være lig den gennemsnitlige statistiske værdi, hvis man kendte al eksisterende data, der kan bruges til at besvare spørgsmålet til det pågældende tidspunkt og havde al den fornødne fornuft og regnekraft til at udregne sandsynlighedsrummet, der bruges til denne udregning. Man kunne også variere denne vedtægt til at være: ``hvis man selv kendte al data, der er kendt af andre samlet set,'' eller endda til bare: ``med al den data, der er tilgængelig for en (på det pågældende tidspunkt).'' Pointen er, at fra den enkelte deltagers subjektive synspunkt vil alle disse variationer give de samme økonomiske valg for dette vedkommende. De to første variationer er muligvis at fortrække, fordi de afhænger af den objektive virkelighed og ikke af den subjektive\ldots Tja, på den anden side skal der nok være mennesker, der vil foretrække den subjektive definition, så alle tre variationer er fine at have med (og nærmest vedtage på én gang). Med denne definition gør det således ikke noget, at man f.eks.\ irreversibelt mister data til at besvare et spørgsmål, eller hvis man bare aldrig har mulighed for at finde det svar i dette univers. Og for spørgsmål, der slet ikke kan besvares eller tildeles en sandsynlighed, enten fordi de indeholder et paradoks, eller fordi de baserer sig på spørgsmål, hvor det ikke er muligt, at finde en konvergent måde at tillægge dem en sandsynlighed (jeg er ikke sikker, men jeg kunne forestille mig, at sidstnævnte godt kunne eksisterer), så må man bare enten erklære sandsynligheden for 50-50 \%, eller erklære kontrakten for ugyldig (eller noget andet tilsvarende simpelt). Og, endnu vigtigere, hvis en kontrakt er formuleret på et tidspunkt, hvor der fandtes en tvetydighed, der gør, at man ikke kan være %... Hm, hvad hvis de to partier så havde forskellige tolkninger og intentioner??... Ah, der kan bare bruge en vedtægt om, at man så ser på den gennemsnitlige person til daværende tid og ser på... Hm, nej for det fungerer kun hvis kontrakthaverne har været dumme; man har ikke lyst til at straffe nogen for at være skarpere end gennemsnittet, hvis nu gennemsnittet ville have tilbøjelighed til at misforstå noget, som vedkommende ikke misforstod... Hm, det gør det jo lidt sværere... Tja, og når det kommer til stykket, så har man vel heller ikke lyst til at straffe personer for ikke at være skarp og/eller at nærlæse teksten grundigt nok, for det ville jo bare bakke op om svindlere. Så hvad gør man lige?... Hm, måsker \emph{er} det faktisk lidt af en knude, for det handler jo bl.a. om, hvem der skal have ansvaret for, hvis kontrakter ikke har været fortået helt af begge parter, er det vedkomne der har skrevet kontrakten, og det vedkomne, der har læst og godkendt kontrakten, og hvad hvis kontrakten er udformet i fælleskab? Så måske er det faktisk et svært spørgsmål.. Hm, kunne man på en eller anden måde gøre selve, hvor meget parterne tror på, at de har forstået teksten rigtigt, til en del af kontrakten...? Hm, gad vide, hvad gængs jura gør.. (Tja, de bruger vel bare dommerbestemmelser i sidste ende, og så er det det.. Men det kan man jo ikke rigtigt her..) Hm... ...Ah jo, man kan faktisk godt bare gøre noget med at sige, at man går udfra samtidens gennemsnitlige fortolkning, eller rettere for sandsynlighedsfordelingen over fortolkninger, når man skal bedømme tvetydige kontrakter, for så må "skarpe partier" jo bare sørge for, at tingene er formuleret mere præcist, hvis de opfatter det på en anden måde end gennemsnittet, og partier der ikke anser sig selv for skarpe, skal så bare nøjes med at forstå kontraktens tvetydige formuleringer på gennemsnitlig vis. Og det er nemlig også fair, at kræve af parterne, at de skal formulere kontrakten, ikke bare så de selv forstår den, men så alle kan forstå den bedst muligt, for det er jo en central del, at man også skal kunne sælge kontrakt-værdipapirene videre. Og når det så kommer til folk, der prøver at udnytte andre ved at der er tvetydig betydning for en sætning, hvor man så udnytter at det andet parti slet ikke ser den ene mulighed, så.. ja, så må man så bare betragte dette som en del af systemet, og så selv som fælleskab finde på diverse løsninger til bedst muligt at undgå sådanne situationer (eksempelvis ved at udbrede en tradition om, at der gerne skal flere forskellige øjne på en kontrakttekst, før den vedtages, så tvetydigheder bedre kan opfanges..). 
helt sikre på, hvad de undertegnede partier mente med teksten hver især på det pågældende tidspunkt, så mener jeg, at man bare bør kunne vedtage et princip om altid at gå efter, hvad man vurderer er sandsynlighedsrummet for, hvordan personer, der taler pågældende sprog, på det tidspunkt statistisk set ville fortolke denne sætning. For man kan nemlig alligevel ikke rigtigt kræve, at fortolkningen skal tage udgangspunkt i parternes individuelle forståelse, for det er nemlig en vigtig del af idéen, at kontrakt-værdipapirer kan handles med, og så duer det ikke rigtigt, at fortolkningen afhænger specifikt af de oprindelige kontraktindehaveres fortolkning. På denne måde kommer der altså to sandsynlighedsrum (eller parameterrum/udfaldsrum, om man vel) i spil, når værdipapirene skal vægtes, nemlig udfaldsrummet for, hvad man vurderer er sandt i denne verden, og et udfaldsrum for, hvordan man vurderer, at personer generelt på udstedelsestidspunktet har tilbøjelighed til at tolke eventuelle tvetydige sætninger. Der er dog også et tredje parameterrum, men dette skal man dog gerne tage højde for fra starten af. Man skal således også fra start af beslutte en konsensus om, hvordan universets prior-sandsynligheder vurderes, hvis vi skal være helt krakilske, for dette kan nemlig i princippet godt føre til uenigheder, man ikke kan afgøre ellers. Men dette behøver ikke at blive et problem i praksis, for her snakker vi jo kun om spørgsmål, man ikke kan besvare empirisk, og hvor tit har man lige brug for at få sådanne besvaret i kontrakter? Men derfor er det stadig nok klogt lige at antage en simpel univers-prior. Personligt er jeg tilhænger af \emph{Computable Universe Hypothesis} (CUH),\footnote{
	Dette, efter hvordan jeg har forstået begrebet, tilsvarer \emph{Mathematical Universe Hypothesis} (MUH), hvis man antager en %konstruktivistisk grundteori --- og nu skal jeg vist også passe på, for måske bruger jeg også `konstruktivistisk' forkert (eller måske er det tvetydigt). Med `konstruktivistisk' mener jeg her: modeller a la Gödels konstruerbare 
	konstruerbar grundteori for multiverset a la Gödels kontruerbare univers. Jeg har i øvrigt i sinde at skrive nogle noter, nedenfor eller i et andet notesæt, omkring mine tanker herom.
}
og jeg tror dette kunne være en fin simpel teori til at give en univers-prior, man kan arbejde med. Og det handler nemlig ikke om at vide den rigtige univers-prior i den samlede eksistens / det samlede multivers; det handler bare om at sørge for, at folk ikke kan slippe udenom kontrakter ved at hævde en langt ude univers-prior. Et eksempel på en sådan kunne være: ``jeg tror på, at der kun findes universer, hvor folk der går under navnet $x$ altid\ldots\ vil tabe til folk med navnet $y$'' eller noget i den stil --- eller sågar noget mere fornuftigt, måske religiøst, der stadig kan bruges til at slippe uden om en kontrakt. Og igen: det handler ikke om at afgøre, hvad der er rigtigt og forkert her; det handler bare om, at vedtage en fælles konsensus om baggrunden for at besvare alle spørgsmål. Og hvis f.eks.\ religiøse spørgsmål kan have indflydelse på udfaldet af en kontrakt, så er det bare underskriveres ansvar at sørge for, at disse spørgsmål er taget højde for som en del af kontrakten (hvor man måske eksempelvis kan antage et religiøst forhold, hvilket man f.eks.\ sagtens kan gøre med CUH, da det er en virkeligt fri teori). Når det så er på plads, hvordan man fortolker kontrakter under alle disse tre kilder til uenighed og usikkerhed, så har man så fundamentet til systemet. Jeg skal lige præcisere så, at kontrakter så dannes ved, at brugere veksler de penge, de maksimalt kan ende med at betale til kontrakten, med et nyt værdi-papir, hvor der nu altså er en risiko. Dette fremhæver et andet spørgsmål, nemlig hvad så når man vil danne en ny kontrakt, men kun ejer værdipapirer med risiko på. Her tænker jeg så bare, at der skal være formelle måde, at veksle sådanne værdipapirer tilbage til normale penge igen, bl.a.\ ved at danne modkontrakter, som jeg beskrev ovenfor *[Nej, det har jeg vist ikke, men det handler bare om at vædde om det modsatte, så at sige, så de to ``væddemål'' går ud med hinanden ift.\ risikoen og giver en konstant værdifordeling.], plus hvad man ellers kan finde på. Jeg kan ikke forestille mig, at det vil blive et problem med uafsluttede kontrakter, der ophober sig på kæden, for så snart folk har brug for rede penge, så kan de jo enten bare sælge deres risiko, eller sørge for at kontrakten bliver afviklet som beskrevet. 

Okay, det blev en lidt lang forklaring, men det korte af det lange er bare, at man måske kunne lave et KV-system, hvor fortolkningen af pengefordelingen ikke afgøres løbende i diskrete skridt, så alle med jævne mellemrum bliver enige om en fælles transaktionshistorik, men hvor pengefordelingen kan afhænge af ubesvarede spørgsmål formuleret i naturlige sprog (eller i formaliserede versioner af dem), og så er det op til brugerne om, vurdere hvad værdipapirer er værd. Og tanken er så, at selvom brugere så kan være uenige om værdien af et værdipapir, så kan de alligevel altid arbejde ud fra, at der findes en ``rigtig værdi,'' som så bare er givet ved, hvad fremtidige generationer beslutter er sandt eller falskt. Noget som så både er et plus og et minus ved denne version af idéen om et mere matematisk/semantisk KV-system, er, at det så ikke kan bæres af noget eksisterende KV-system. Det kan således eksempelvis ikke implementeres som en state channel i en eksisterende blockchain, for dette kan kun lade sig gøre, hvis man i sidste ende har en protokol til at oversætte pengefordelingen til et eller andet digitalt underskrevet skema, som kan overføres til hovedkæden. Dette vil således kræve en eller anden central enhed til at bedømme sandheden omkring udfaldet af kontrakter, og det går jo imod hele princippet i systemet. Grunden til at jeg så kalder det både et plus og et minus, er, at det jo er et minus for folk, der leder efter løsninger til at forbedre teknologien for eksisterende systemer, men er et muligt plus for folk, der leder efter teknologier til at opstarte et helt nyt KV-system. 
%
%To partier, der har indgået en kontrakt kan endda nemt aftale at afvikle kontrakten, så den bliver 100 \% risikofri, ved at indgå en modkontrakt, der sørger for at modvirke enhver senere opdagelse, der får  

%Hm, og her ville jeg have en paragraf omkring det med et løfte-baseret system generelt --- at det måske ikke kollapser på samme måde, men...? Holder det nu helt?.. Giver det mening?.. Tja, det gør det måske ikke, nej..? Tja, problemet er vel, at det kunne giver mening i en sammenhæng, hvor man nærmest baserer en slags stat (med politi) på det, så nej: nej, det giver ikke mening.


		%Semantik i kontrakter (ved brug af samme teknologier som semantisk web --- og så kan alle brugere jo være med og danne nye kontrakter osv.). (tjek)
		%Reflektion i smart-kontrakter. (tjek)
		%Dommergruppe-aktier kan logges på. *Og ultra-kort om, hvordan disse kan udbygge sandhedsontologier. (tjek-tjek)
%Nævne noget om blok-rotation og om denne potentielle sikkerhedsbrist i gængse BCs?.. (altså nemlig den, at man kan uploade ulovligt indhold, som også er en velkendt protentiel brist).?
	%Uh, jeg kan også lige nævne noget om løfte-baserede systemer (hvor man benytter, at vi har et indbygget (eller hvad det er) æresprincip som mennesker (i det mindste i de fleste kulturer..) om at holde ord, også når dette ord er givet på skrift, og dermed kan man opnå stabilitet i et system; ord er jo ord, og løfter kan ikke ``kollapse'' på samme måde, som en kæde kan det). (hm, giver dette overhovedet rigtig mening..?)
	%Nævn mine tanker om, hvad jeg lidt er kommet frem til, ift. hvorfor kryptovalutaer har den værdi, de har (altså når nu man ellers kunne tænke, at de må bunde i ingenting (hvad de jo lidt også gør, men så alligevel netop ikke..)).. *(Eller måske er dette lidt overflødigt i virkeligheden...) *(Nu, eftermiddag (03.04.21), har jeg faktisk lige ændret min teori for, hvad der, i teorien altså, bør give værdi til et valuta-system så som et KV-system. Og dette bliver alligevel lidt for økonomsik (også selvom det er ret simpelt), så det hører nu nok mere til den kommende sektion om de mere økonomisk orienterede (og forretningsorienterede) idéer og tanker omkring sem-web m.m.)
%Mangler: "To partier, der har indgået en kontrakt kan endda nemt aftale at afvikle kontrakten, så den bliver 100 \% risikofri, ved at indgå en modkontrakt, der sørger for at modvirke enhver senere opdagelse, der får."











%\subsection{Blockchain}
%
%Jeg har en idé til en ny form for blockchain. Idéen er at have en blockchain, der på en måde beskriver sin egen protokol for, hvordan den skal opdateres, inklusiv hvordan pengetransaktionshistorikken bliver opdateret, og hvor stake-holderne (altså dem der ejer en delmængde af krypto-pengene) via en tilhørende stemmeret kan vælge, hvordan blockchain-protokollen skal ændres løbende. I princippet behøver blockchain-protokollen ingen gang handle om blokkæder, men kan godt ende med at bruge andre krypto-protokoller, så derfor bliver `blockchain' egentligt lidt en misnomer. Så lad mig referere til det i stedet som et kryptovaluta-system (KV-system / KVS). Idéen er så grundlæggende den at starte med en matematisk model, der for det første beskriver en protokol for at opdatere transaktionshistorikken, hvor denne defineres ud fra en pengefordelingsfunktion (PFF). %Jeg vender tilbage til, hvad dette indebærer. 
%Og udover at indeholde en definition af denne PFF, skal modellen også indeholde en protokol for, hvordan man opdaterer selve modellen (inklusiv begge disse protokoller). Man kan altså således både opdatere PFF'en herved samt opdatere protokollen for, hvordan PFF'en kan opdateres i fremtiden. Pengefordelingsfunktionen er muligvis en partiel funktion (det kunne den meget vel være), der tager en mængde data samt et (potentielt) konto-id og returnerer, hvor mange penge der er på denne konto. Førstnævnte input kan være i form af en mængde af binære tupler, der repræsenterer filer og/eller hukommelsesstrukturer, og sidstnævnte kan så bare være i form af en enkelt binær tupel. Uh, og man kunne faktisk også vedtage, at funktionen tager et ekstra (numerisk) input, som repræsenterer et tidspunktet for, hvornår den givne kontobeholdning evalueres, og så kan man altid se, om man bliver ved med, at have behov for dette input eller ej. Funktionen er nemlig alligevel bare teoretisk, og der er derfor ingen grund til at bekymre sig om effektiviteten af en faktisk implementation af den. Og selvom funktionen som nævnt bør udskiftes løbende, så bør fortolkningen af den alligevel helst være helt fastlagt, så derfor er det godt lige at sørge for fra starten af, at dens domæne indeholder alle de input, den skal bruge. Jeg ville dog mene, at hvis man medtager tidspunktet som input, så skal det ikke være for at kunne evaluere fortidige pengefordelinger; der behøver nemlig ikke at være nogen garanti for, at man kan genfinde disse. Det skal bare være, så at der er en mulighed for, at pengefordelingen kan være tidsafhængig, også selv hvis der er stilstand i datamængden i netværket, hvis man nu ønsker dette. En central egenskab for fortolkningen af denne PFF er så et princip om, at mere tilført data til datamængden vil give et mere korrekt svar, når det kommer til pengefordelingen. Dette skal forstås på den måde, at hvis et parti (inkl.\ enkeltpersoner) hævder, at en række kontoer har et vist beløb i sig, og bruger en datamængde $A$ som begrundelse (ved at give denne til PFF'en og iagttage outputtet for kontoerne), og et andet parti så offentliggør en datamængde $A \cup B$, der giver en anderledes pengefordeling, så må første parti altså, pr.\ den korrekte fortolkning af KVS'et, bøje sig, i det mindste indtil der kan findes en datamængde $A\cup B\cup C$, der igen kan give deres påstand ret. En anden løs definition kunne være, at den faktiske pengefordeling i princippet altid er givet ud fra PFF'en taget på al offentlig data til det givne tidspunkt. Jeg kunne forestille mig et par forskellige måder, hvorpå man kunne prøve at definere dette princip mere præcist, især hvis man først fandt en god måde at definere et præcist begreb for, hvad ``offentligt tilgængeligt data'' indebærer, men egentligt er det ikke vildt vigtigt. For i bund og grund baserer det sig jo alligevel bare på en fortolkningskonsensus iblandt alle deltagerne, og i sidste ende handler det jo bare om, at der ligesom for gængse blockchains skal være en måde, hvorpå en bruger kan iagttage en fortidig pengetransaktion, og få vished om, at effekten af denne transaktion vil komme til udtryk i den fremtidige pengefordeling og altså ikke blive glemt (eller negligeret) af fællesskabet. Og når det bare er dette, vi ønsker at opnå, så er det altså en svært at finde frem til en brugbar tolkning af PFF'en. 
%
%Nå, lad os så gå videre til, hvordan PFF'en opdateres. Som sagt er den defineret i en central matematisk model, som er indeholdt i KVS'et. Og denne model definerer så også en protokol for, hvordan denne model kan ændres over tid. Okay, hvordan definerer vi så denne protokol? %; gør vi det også med en form for funktion? Tja, det er egentligt ikke så vigtigt; bare man finder en måde, der holder. 
%%Men det ville dog nok være en god idé i det mindste at have en form for et fast skema over modellens struktur, så semantikken bag modellen altid er entydig; så fortolkningen ikke ligesom viskes ud over tid. 
%%Jeg kan ikke lade være med at tænke på mit ITP-system der, med dets meta-antagelser\ldots\ Men KVS'et kræver så mere end bare en liste over meta-antagelser; det kræver tidsafhængige meta-antagelser\ldots\ Men kunne man ikke bare inkludere en tidsparameter i deres formulering, som jo så skal have en simpel fortolkning tillagt sig, nemlig at denne skal sættes til den rette værdi, når antagelserne skal analyseres\ldots\ Hm, men er dette overhovedet så vigtigt?\ldots\ Ah ja, det er det måske\ldots
%Tja, jeg har før bare tænkt, at man ligesom skulle have en model, der kunne beskrive sig selv, og så have en måde at initiere en opdatering, hvor en udvalgt indre model (en modeleret model) så kommer ud og bliver den nye model, men nu hvor jeg har fået min nye ITP-idé, så kan jeg se en mere simpel tilgæng til det. Man kunne definere KVS'et via, hvad der svarer til meta-antagelser, bare hvor disse ikke refererer til, hvad der forekommer i en individuel brugeres mapper, men i stedet refererer til, hvad der forekommer på det samlede netværk mellem KVS-deltagerne (altså hvad der forekommer på ``det offentlige web,'' kunne man også sige). Og så skal disse meta-antagelser (hvad vi jo så stadig godt kan se dem som) for KVS'et bare beskrive en model, hvor man kan forbinde input i form af evalueringstidspunktet samt en datamængde med en pengefordeling. Og ud fra dette resultat kan man så finde ud af hvad en given konto indeholder af penge, betinget at det opgivne evalueringstidspunktet er korrekt sat og at datamængden er tilstrækkelig omfattende ift.\ den eksisterende data på det offentlige netværk (i.e.\ på internettet), så at man ikke er kommet til at overse nogen transaktioner. Og så er der bare lige det ved det, at man aldrig i princippet kan vide 100 \%, om der nogen nye transaktioner eller noget spritnyt data, man har overset, men løsningen er så bare, at man altid tager de pågældende resultater med et tilpas gran salt, indtil nok tid er gået sidenhen, sådan at man er sikker på, at der ikke var nogen nye transaktioner, man havde overset. Og i så fald kan man så formelt konkludere, at resultatet var sandt på daværende tidspunkt. Så ja, det må være den formelle måde at tolke KVS'et på, og så kan man i praksis selv vælge alle mulige smutveje til at nå de samme resultater, hvis man finder nogen gode nogen. Og når KVS-meta-antagelserne på denne måde kan være tidsafhængige, jamen så kan man jo bare implementere opdatérbarheden ved at gøre meta-antagelserne åbne og så (muligvis) bruge tidsparameteren til at definere, hvilke nogle barne-mata-antagelser de kan føde. Husk nemlig, hvordan meta-antagelserne for ITP-systemet med fordel kunne gøres åbne, så man løbende bl.a.\ kunne bruge certifikater fra troværdige kilder til basalt set at føje nye underantagelser til. Man kan så altså se disse ``åbne antagelser'' som en salgs højere-niveau-antagelser, der ligesom kan føde nye antagelser. Og min tanke er så, at man bare kan gøre helt det samme for KVS'et, så at opdateringer af systemet (og dermed af PFF'en) kommer i form af de nye antagelser, man løbende kan føde ud fra de gamle. Og når tiden ovenikøbet er med som en parameter, så kan man så endda videre sørge for, at barne-antagelser kan fødes ikke bare pga., at der pludselig er den fornødne data til det, men fødslerne kan også hænge sammen med, at man er nået et givent tidspunkt. Og med tidsparameteren i spil, kan der i øvrigt også potentielt sættes en udløbsdatoer på nye barne-antagelser, så de måske ligesom kan give plads til nye.
%
%Okay, hvad opnår vi så ved at have et åbent defineret kryptovaluta-system, og hvis systemet er så åbent over for ændringer, hvad har deltagerne så af vished for, at deres penge ikke forsvinder eller låses på en eller anden måde i fremtidige opdateringer. Lad mig lige diskutere det sidstnævnte først. Det er klart, at det er essentielt for et godt KVS, at folk kan være sikre på, at deres penge ikke bliver taget fra dem som følge af en eller anden uforudset opdatering. Derfor nytter det ikke noget for deltagere med mange krypto-penge, at prøve at rotte sig sammen om at tage fra de fattige og give til de rige i systemet, for så mister systemet sin integritet og vil højst sandsynligt gå til grunde på bekostning af andre KVS'er, måske nogle tilsvarende nogen, der dog har bevaret integriteten (for ingen har jo patent på teknologien, så der kan sagtens dannes flere af denne type KVS'er, som så kan konkurrere om popularitet). Så med dette på sinde tror jeg faktisk, at man ville kunne nå rigtig langt bare med et KVS, hvor meta-antagelserne definere en opdateringsproces, hvor folk/partier bare har stemmeret efter, hvor mange krypto-penge de har på deres konti.\footnote{
%	Givet at visse andre faktorer også holder, som jeg vil komme ind på senere.
%} 
%Og selvom det som sagt kan være smart at have ret åbne meta-antagelser til at starte med, så kan man jo stadig sagtens sørge for, at det er muligt at begrænse denne frihed løbende, og sætte nogle invarianter/restriktioner på KVS'et, som ikke bare lige sådan kan brydes igen. En af de helt tidlige invarianter man kunne sætte, sandsynligvis helt fra starten, kunne være den at pengemængden altid skal være konstant\ldots\ Tja, eller måske kunne man faktisk også bare nøjes med at lade denne invariant kræve meget bred konsensus samt en varig formel proces, før den kan brydes. En anden oplagt omstændelig-at-bryde invariant er f.eks.\ den, at stemmeretten er proportionelt med pengebeholdningen. 
%
%Det man opnår, er altså så et KVS med opdatérbare protokoller, så man ikke bare investerer i en hel bestemt protokol, når man køber kryptopengene, men i stedet investerer i teknologien overordnet set, samt en tillid til, at stake-holderne, eller rettere stemmeret-holderne, vil sørge for at bevæge udviklingen af det specifikke KVS i en fornuftig retning, der øger dens værdi. (Og at sætte stemmeret proportionelt med stake kunne så være en god mulighed som en måde at sikre dette.) Et af de store problemer med f.eks.\ Bitcoin er jo som bekendt, at det ikke bare sådan kan skaleres op på stor skala pga.\ de resurser, det kræver, men sådanne forhold er man slet ikke bundet af her; her kan man bare løbende justere protokollen, så resurserne er brugt på en effektiv måde. Når det så kommer til rene PoS-blockchains, så er det lidt sværere at argumentere for, at disse matematiske KVS-systemer vil slå sådanne blockchains... %Hm, at man ikke behøver at bekymre sig om integriteten af en main-kæde.. og at man faktisk kan mere med matematiske kontrakter, kan man ikke, for så kan man vel... bare referere til en fremtidig model... Hm.... 
%
%%Og selv når det kommer til rene PoS-blockchains har denne type KVS en fordel. For systemet behøver som nævnt ingen gang at have form som en\ldots\ Hm, men det gør PoS-kæder jo heller ikke\ldots?  
%
%
%%Brain: Okay, man kan vel egentligt opnå det samme i princippet med en PoS-kæde?... Behøver man at bekymre sig om main-kæden der?... Tja, nej, det er vel ikke så svært at lave en PoS-kæde, hvor main-kæden godt kan fryse... Nej, det er det vel ikke.. Eneste er det, at folk skal gå og holde styr på main-kodeordene... Tja, ok, men er idéen så ikke bare en idé om, at kan lave og/eller konvertere kæder til matematiske kæder..? Jo, for det er jo så bare under antagelse af, at de pågældende eksisterende kæder tillader en sådan udvikling, uden at det bliver besværligt, end hvis man bare startede med en matematisk kæde. Og værdien ved en matematisk kæde er så, at man kan danne en ret alsidig dommerenhed... Og at man kan opdatere den nemt efter behov...
%%Hm, det kan være, at jeg faktisk kan gøre det helt kort og bare nævne det under sem-web- og ITP-noterne. For det er jo bare: Man kunne lave matematiske kæder og/eller state channels til eksisterende kæder (der kan rumme det), og her kunne man så oprette dommerforeninger til at vurdere alsidige matematiske/semantiske kontrakter (og gerne arbejde på at udbygge en sandhedsmodel også), hvis aktier selvfølgelig godt kunne logges på kæden også, eller muligvis bruges som den centrale mønt, for det er jo kun gavnligt med mere værdi logget på kæden. Og så hører det jo til, at en sådan matematisk kæde kunne gøre ret åben over for opdatering (og så er matematiske kontrakter sikkert også rare at bruge).  
%
%
%
%%Matematiske kontrakter!
%%Semantiske KVS-antagelser.
%%Mining i starten..?
%%Videre at tale om: "Okay, hvad opnår vi så ved at have et åbent defineret kryptovaluta-system?" Vi opnår, at brugerne bare altid kan skifte tilbage til PoW fra PoS, hvis at PoS får det til at gå i stå... Hm.. Og så er det jo generelt rart med at system, der kan gøres mere effektivt løbende.. ..eksempelvis ved brug af refleksion.. Og det er særligt dejligt, at systemet ikke behøver at være én lang kæde, men kan spredes ud.. Hm, er der egentligt noget at brokke sig over, at systemet nok i sin natur ikke vil være ligeså decentraliseret, fordi den alt andet end lige kræver PoS..?...
%
%
%
%%Ikke til disse noter men: MDF og CCS virker vist i øvrigt også helt fint på engelsk; eneste jeg lige kan komme i tanke om, er, at CCS ligger lidt tæt på CSS.. 




%Faktisk brainstorm:
%Tja, jeg kan jo godt skrive om matematiske blockchains, hvor protokollerne kan opdeteres og PoS-brugere, og hvor kontrakter skrives i matematik. Men det er så lidt åbent, hvordan brugere skulle logge sig på forskællige kæde protokoller, således at der kan ske mining. Tja, men kunne man ikke bare veksle mønter for kontrakt-mønter, hvor værdien så kan afhænge af, hvad andre brugere gør, og kan give afkast til minere... ..Så man ligesom kan opdatere kæden ved at underskive kontrakter, der kun træder i kraft, hvis en vis mængde andre brugere også underskriver.. Ja, det lyder faktisk fint nok.

%Hm, og en anden version er så, hvor man ikke har nogen penge-beholdning, men hvor man bruger kæden til at underskrive semantiske (NL-kraftfulde) kontrakter for, hvad kontrakthaver skyldes af visse andre RW-entities, inkl. brugere med officielt ID. Og så håber man så på, at eftertiden altid vil tage disse kontrakter seriøst, ligesom hvis man havde underskrevet fysiske kontrakter (og hvis kontrakten i sig selv er lovlig, i.e. hvis handlen er lovlig). Tja, det er da en idé.. (19.03.21)
%*Tja, og en blanding, hvor coin-værdien baserer sig på kontraktlødigheden... Ja, det er vel bare en tilføjelse til første version..

%Og så kommer vi vel til blokkæder, der har aktie-, og muligvis stemmerets-, coins... ..Hvor man så kunne bygge kæden over et open source-agtigt firma, hvor der gælder en eller anden nyskabende ting (altså en eller anden (eller flere) af de ting, jeg har tænkt på..)..

%Indskudt: Hm, i øvrigt gælder der for de matematiske blokchains, at de kunne lægge op til forskellige protokoller for at "mine sandhed."

% Men tilbage til aktie-/stemmerets-BCs... Tja, det kræver, at firmaer kan købe sig på kæden, hvor disse firmaers forretningsmodel så skal have et udgangspunkt om at følge vedtægter på kæden. Men jeg har vist ikke helt fundet nogen holdbar mellemvej mellem normale firmaer (hvor aktier giver stemmeret om foretag) og Bevægelsen, som jo handler om meget mere end bare en kæde.. Hm... Oh, well. Jeg kan bare vende tilbage til det. ... Hm, hvad med firmaer, der betaler aktie-reducerende afkast til deres brugere, og som så har nogle anti-poitivt-feedback-foranstaltninger, og hvor man dermed så... Hm.. Tja.. og noget med at brugere kan vinde aktier, ved at stemme gode forslag igennem... Hm, tjo.. Hm, ikke sikkert, at jeg kan finde en god og simpel (og flashy, for jeg tror jo selv mest på bevægelsesidéen) mellemvej..

%Tja, jeg tror egentligt bare jeg vil holde mig til matematiske PoS-coin-BCs i første omgang, hvor coin-ejere så kan logge deres mønter på forskellige protokol-model-forks og kan vedtage skatter for brugere, der sent hopper med dem. .. Og i anden omgang kan jeg så nævne lidt om et nætværk, der kigger hinanden over skuldrene, til at mine god og lødig data til p-ontologier, som kontrakter så kan refere til, hvor den fremtidige udvikling af ontologien kan tages med. Og jeg kan også nævne et nætværk, der så bruger dette til at lave bagudbelønningsfirmaer (med princip om skalering).

%Uh, der kan faktisk godt være noget om det med, at coins'ne får værdi ud fra efterspørgslen, for man kan jo bare betragte den samlede coin-mængde på tværs af alle kæder som én samlet valutta, men bare med varierene vekselrater indbyrdes, og så får disse så hver især værdi efter den samlede efterspørgsel og så i øvrigt vægtningen grundet de individuelle mønters eftertragtethed pga. deres brand (og også individuelle brugbarhed). Og hvis man så glemmer den individuelle vægtning, så kan man altså bare se det som en samlet mængde penge, hvis (samlede) værdi, foruden mulig hype osv., afhænger af efterspørgslen. Nice..! (28.03.21)










%\subsection{Mere om ontologier}




%Faktisk brainstorm om (p-)ontologier *(og "omni-side"):
%Tja, princippet omkring ontologierne er jo vel ret simpelt.. P-ontologier handler jo bare om, at have en liste af antagelser samt noget (internet-)data, som så fører til ontologi-modeller, der så igen kan bruges til at give sandsynligheds svar om forskellige spørgsmål. Og så er det så meningen, at man som bruger af et funktionelt ontologi-system prøver at udvælge nogle ikke-korrolerede antagelser fra listen, helst så vidtrækkende som muligt, hvis det har indflydelse på svarene, og så giver disse visse sandsynligheder hver især, for så at opnå en resulterende svar-ontologi.. Og for andre former for ontologier, så kan man jo bare erstatte 'sandsynligheder' med vægtningspoint af en eller anden art.. Og hvordan arbejder brugerfælleskabet/erne så på at forbedre ontologi-system-data m.m.?.. Tja, det handler vel egentligt bare lidt om at formulere diverse antagelser (ligesom at meta-antagelserne også er helt centrale for ITP-systemet i den idé).. Og selvfølgelig at tilføje data m.m. til fælleskabet.. Ja, så simpelt er det vel egentligt i bund og grund.. Og dette bliver vel også grundlaget/fundamentet for en "omni-side"... Ja, bare hvor man så fortæller en omni-side-applikation i sit ITP-system, hvordan den skal søge data til brugerens queryede omni-side-objekter (/internet-objekter) og så også, hvordan de skal stilles op. Ja, så det er på en måde bare at lave en browser-applikation i sit ITP-system, og det kan så være fundamentet til det, jeg har set som en omni-side (en browser er jo en slags omni-side).. Hm, og nu hvor jeg taler om dette, hvad var det så mine tanker var omkring denne form for web-løsning (som jeg altså kaldte 'omni-side' i mine noter)?.. Tja, det er jo for det første bare at gå væk fra web-paradigmet, hvor man har en masse selvstændige hjemmesider, men deres egne design og muligheder, til et paradigme, hvor der mest er fokus på 'mulighederne' og objekterne (som så er beskrevet semantisk), men hvor brugere bare for serveret de ting i sin egen "hjemmeside" (om man vil), hvor denne så kan have sine egne design-præferencer for, hvordan de forskellige objekter bliver serveret. ..Ja, og at alle disse åbne brugerflade designs så er open source og matematisk beskrevet, så det er let at blande forskellige løsninger.. Og så er det også meningen, at servere skal være mere alsidige og køre mere open source applikationer.. Tja, og en vigtig ting omkring det er jo så også, at benytte semantisk data, som brugere har uploadet omkring objekter, og at selv URI-instanserne også kan være åbne, så man kan tilpasse dem med forskellige præferencer (og bruge andre brugeres anbefalinger til ens specifikationer).. Og ja, det er så bl.a. vigtigt med veludviklede måder at klassificere medbrugere på, så man bedre kan benytte sig af deres råd (hvorved man så også ligesom skal klassificere sig selv..).. Derudover er bl.a. sprogontologier selvsagt vigtige, samt gode interfaces, hvor der er feedback på at formulere semantisk præcise sætninger (ud fra hvor avanceret den samtidige sprog-ontologi er).. 
%Og angående selve ontologierne, så kan det altid betale sig at inkludere andre brugergruppers del-ontologier på en eller anden måde i sin ontologi, for man kan jo bare give dem en lav prior-snadsynlighed (eller tilsvarende point-parameter)..
%Det her med at få gode prædiktive ontologier som vedligeholdes og udbygges af store fælleskaber, tror jeg virkeligt, kan ændre meget og betyde helt vildt meget for vores teknologiske og samfundsmæssige udvikling. \emph{Virkeligt} meget. Men det er også virkeligt simpelt, så jeg tror faktisk ikke jeg vil påpege disse udsigter i første omgang, når jeg løbende vil udgive mine noter. Jeg tror det er bedre, at holde det lidt nede på jorden, så folk kan følge bedre med. Jeg har i øvrigt også fundet på en måde at formulere det med de åbne web på, så det ikke associerer så meget til en økonomisk bevægelse ligefrem, men bare en ret åben firmastruktur. Så ved at putte de to ting lidt i baggrunden, nemlig hvad man kan opnå med p-ontologier og helt hvor fladt web-firmaerne kan gøres, så tror jeg, jeg kan fokusere på de ting, der er lidt nemmere at sluge i første omgang, uden at det skal lede til en masse åbne, ubesvarende spørgsmål. Så vil jeg derfor i øvrigt heller ikke nævne så meget om ontologi-/model-udbyggelse for en matematisk BC, hvis jeg tager den idé med i første omgang. Jeg vil i øvrigt også bare holde mig helt til det digitale i den omgang. 















%\subsection{Økonomisk bevægelse}


%(15.04.21) Jeg udkommenterer hermed hele denne sektion, eller dvs., da her kun står kommentarer i forvejen, udkommenterer jeg bare sektionsoverskriften lige her ovenfor. I sidste paragraf i nedenstående kommentarsektion (skrevet d. 15/4-2021) kan man læse, hvorfor jeg ikke længere behøver at gøre et stort nummer ud af tankerne, men i stedet bare kan føje det til samlingen af "øvrige noter" (som sektionen hedder pt.), hvad jeg altså vil gøre. 


%Opsummering og videretænkning over nedenstående ('Ny') brainstorm (start: (08.04.21)):
%Så nu tror jeg efterhånden, jeg er ved at få styr på min idé (jeg troede godt nok, at jeg allerede havde godt styr på den, men der skulle åbenbart en del mere til). Opgaven er nu for det første at få det ridset op og få destileret det, så jeg forhåbentligt kan finde ind til nogle simple koncepter (altid en vigtig proces; at destilere idéen og koncepterne i den, men jeg føler nu, at jeg ikke er langt fra mål). Jeg skal også overveje noget mere, hvordan de tidlige implementationer af idéen/bevægelsen kunne være. Et problem er bl.a. lidt, at man nok kun kan adminitrere bidragshandlinger i starten, og af disse nok også kun de handlinger, der ikke allerede har modtaget løn. Men det er vel i og for sig også fint nok. I starten handler det jo, naturligvis, bare om at opstarte bevægelsen og dets systemer osv., og så kan det jo snildt bare være op til folk, der gerne vil hjælpe til så at notere sine behjælpelige handlinger, så de ikke går i glemmebogen. Og jeg mener jo nemlig, at det er helt fint, hvis bevægelsen bare starter med at fokusere på sig selv, og altså således bare starter med at fokusere på de bidrag, der har med denne at gøre, og så er meningen altså, at den bare langsomt skal brede sig ud til at kunne rumme alle handlinger i samfundet (både aktive og passive, og altså også inkl. modtage-handlinger, passive og aktive). Og hvornår skal denne overgang så ske? Hm.. Tja, den skal vel sådanset bare ske, når bevægelsen får administrative kræfter nok... Okay, så opstår der godt nok lidt et.. problem.. Hm, eller måske ikke.. Hvordan sikre man sig, at aktier, der hjalp bevægelsen om at komme i gang, vil blive værdsat, efter den er kommet op at køre?.. Tja, i princippet bør der vel... Nå ja, det er jo ikke længere den nære fremtid, men den fjerne fremtid, der på en måde bestemmer kursen (eftersom folk jo bare kan holde på aktierne, indtil kursen kommer op og matcher den faktiske, fortjente værdi, og i princippet bør kursen således altid regnes ud fra, hvad man forventer dette at blive, for kursen bør jo sættes ud fra, hvad man i sidste ende forventer aktien at blive værd). Og er dette så nok, eller skal start-bevægelsen lige give sig selv lov, at tilføje nogle bonusser til de første bidragere?.. Og her tænker jeg altså bonusser i form af, at man bryder den udregning (med parameterrummet og alt det), der ellers har for mål at være helt ikke-diskriminerende.. Tja, nej.. I princippet kunne en opstartende bevægelse godt forsøge at gøre dette, men det er nok frarådeligt (efter min mening), for det ville jo betyde, at man gjorde så at lignende entreprenører, evt. bare i en mindre eller større skala, i fremtiden ikke fortjener den samme belønning, som bevægelses-entreprenørerne, på trods af at udregningen jo har med i sig, hvor mange (modtagere) der er berørt af handlingerne. Og vil det ikke være dårlig stil at gå ind for, at man selv for mere end andre, der pr. målsætningen omkring, hvordan udregningen gøres, bør have fortjent det samme ift. lykke-velfærdsberegningen? Jo, det ville det jo nok. Det eneste man dog kunne tage højde for, der kan måske give en forskel, er det her med, at handlinger ikke kan blive ligeså nøjagtigt registreret i begyndelsen.. ..Jo og måske er der også en ting til, nemlig at for første bidrag skal man tage højde for, at de blev gjort inden i et lidt andet økonomisk system.. Men bør dette ikke bare være en del af risiko(tagnings)-vurderingen, når man skal bedømme fortjenesten, hvor man jo altså naturligvis skal (og kommer til at) fremhæve handlinger, der gjordes med en stor risiko for, at der ingen fortjeneste ville være (hvis planerne ikke holdt).? Jo, det bør det jo nok, men hvordan afgør man så denne risiko; det kan jo være svært at bedømme i fremtiden.. Ja, og det er så netop her, at første bidragere måske kan føle sig i fare for at blive snydt, hvis nu eftertiden ender med at nedvurdere, hvor stor risikoen var.. Og her hjælper det så ikke noget, at fremtiden måske ender med at vurderer den højere igen, for det vil folk jo netop i sådan en situation så ikke forudsige. Hm, men risikotagningen har vel mere med, hvad folks generelle vurdering er på det pågældende tidspunkt, frem for den egentlige sandsynlighed.. Nå ja, det giver sig selv; ja det har den. Så man kunne jo måske med fordel sørge for at have motivere et udsnit af folk til at vædde om udfaldende; så ville man jo få en god prejning på, hvad risikoen er. Ja, så måske er det bare det, de første bidragere (f.eks.) skal sørge for, nemlig at lave stikprøver af samtidens risikovurderinger fra folk.. Ja, det må jo være det, der er svaret.. Ja, og at "se på væddemål" kan jo også i stedet bare være at se på aktiekurser, så hvis bevægelse-implementationsforeningen bare udsteder nogle aktier i sig selv også (således at en forening / et firma om at opstarte bevægelsen kunne startes som et privat firma, hvis profitmål så er at tage nogle på forhånd faststatte renter for en kort tid, efter at bevægelsen er kommet på benene, hvor disse renter så går til aktieindehaverne (hvorved aktierne således vil formindskes i værdi i takt med at afkastene bliver betalt)). ... Åh, der er faktisk et problem, som gør at startbidragerne kan have grund til frygt: Jeg nævnte jo, at eventuelle huller, som giver mulighed for at diskriminere tidsligt (eller personligt eller lokationsbestemt eller befolkningsgruppebestemt), kan lappes i trejde lag. Jamen, det går jo ikke, for så har start-bidragerne ingen garanti for, at de selv bliver reddet af disse lapninger.! Så hvad gør man så? ..Okay, jeg skal lige tænke lidt, men i øvrigt: "startbidrag" kommer vil egentligt også bare til at bestå af pengedonationer, friviligt arbejde på at opsætte systemet og så reklame; det er vel rimeligt meget det? Hm... .. Hm, kunne man ikke bare starte med et implementeringsfirma, der har stemmeret-aktier, som bruges til at stemme om ændringer og rettelser i systemet, både for hvordan aktier registreres osv. men særligt også for, hvordan parameterrummet skal konstrueres for at eliminere de integritetsbrydende diskriminationshuller. Og hvis man så har meget stake i bevægelsen, så har man så også incitament til at købe (og evt. gå sammen med andre personer i samme båd om at købe) og formidle stemmeretsaktierne, så man kan hjælpe med at sikre systemet mod sådanne huller. ..Så er der bare lige det, at man gerne vil nå til et stabilt system, og derved vil det være bedst hvis aktierne udfases. Men det er jo lidt ærgerligt for dem, der har taget ansvar og har holdt disse aktier, så det ville jo være rart, hvis aktierne så kunne udfases via et afkast. Men ja, er det så ikke bare det man ligesom skal gøre; starte et bevægelsesopstartsfirma, som efter opstarten kan kræve nogle (begrænsede) renter af bevægelsen, som så kan bruges til at betale aktierne af, og så kan disse aktier tillige bruges til at give stemmeret til, hvordan systemet skal udformes (dog med klare mål sat i starten, så at deltagerne i bevægelsen kan føle sig sikre på, at ikke bare de selv men også størstedelen af alle andre opskrevne deltagere vil gøre oprør, hvis ledelsen skulle indstemme ændringer i systemet, der bryder med disse målsætninger). Lyder umiddelbart meget godt.. Bemærk forresten at disse noter indtil videre har været mere "videretænkning" end "opsummering," men det er jo også helt fint. ... Okay, nogle flere tanker: Der bør jo godt nok være en klar integritetsmålsætning i hele bevægelsen om ikke at snyde tidlige bidragere, hvilket så kun kan gælde helt fra start; alt andet ville jo være hykleri, og da det i sidste ende er forudsigelse af, hvad fremtidens samfund vil blive enige om er fair, så behøver start-bidragerne i princippet altså ikke være særligt bange (også fordi deres belønning jo alligevel afhænger af, at dette system kommer ordenligt op og køre (på en.. integritiv?.. måde)). På den anden side kunne man godt spørge sig selv, hvad er fair, når vi tager højde for, at økonomien er i en overgangsperiode? Skal man så give start-bidragsyderne løn efter samtidens (high-stakes-agtige (and much to win..)) økonomi, eller efter en formentlig mere rolig og muligvis også lidt mere fladt struktureret økonomi? Eller måske en blanding? Hvad er fair?... Ah vent, er det ikke bare noget med at lade parameterrummet være en åben variabel i systemet, hvor man også bare så bare regner med, at fremtiden vil ende med en ikke-hyklerisk model?!:) (08.04.21) Hm, det kan man jo, hvis man tør, men gør man nu også det?:) ..Ja, det mener jeg helt klart sagtens man kan; for ethvert tilsyneladende integritetsbrud og enhver udvikling, der ikke virker særlig ærlig/fair, vil jo bare påvirke hele opstarten negativt. Og man kan sagtens sørge for, at det er stake-holderne, der får lov at varetage systemets udvikling i starten uanset hvad; det kan man jo gøre på mange måder. Og hvis man så bare lige sørger for på en eller anden måde at lave stikprøver, der viser usikkerheden, så eftertiden ikke kan påstå, at der ikke var en særlig høj risikotagning ved at investere i bevægelsen (med "ydelser," som jo kan dække over alle mulige ting), så er man jo dermed rimeligt godt sikret, altså også selvom ens egne belønninger som start-bidragsyder også udgøres blot af handlingsaktier. Så derfor mener jeg altså ikke, der kan være de store farer for, at man som start-bidrager ikke får den anerkendelse og den løn man fortjener, hvis altså systemet først kommer rigtigt op og køre (hvad man jo alligevel satser på med sin investering). Og da det er et helt liberalt system, så kan man jo bare handle med sine aktier, hvis man som invester selv for kolde fødder på en eller anden måde; så kan man bare sælge sine aktier videre til en anden, der ikke har det. 
%Hm, det er altså alligevel som om, det ikke helt holder... Jeg tror næsten på en eller anden måde, at jeg skal bringe det lidt tilbage og prøve at blande det mere sammen med et system, hvor lønninger fastsættes i et liberalt system, som så bliver baggrund for at vurdere belønninger til velgørende handlinger/bidrag.. Hm... Ja.. ja, det bør næsten skulle holdes lidt i to sådanne lag i starten.. Det virkede ellers så godt, men jeg fik vist lidt overset problemet med, at kurserne kan forventes at variere i tiden (hvilket man dog nok muligvis kan finde en vej ud af; jeg føler, jeg har en løsningsvej), og så er der også simpelthen det, at det vil være alt alt for svært at arbejde med i praksis set fra et firmas øjne, hvis man skal regne alle lønningerne i komplicerede og dynamiske udregninger... Tja, tjo.. jo... Jo, sådan er det jo, men det var jeg jo også lidt inde på, hvad angår, at man ikke kan registrere alle handlinger i starten, men derfor mest bare bør fokusere på tydelige (ellers velgørenhedsagtige) bidrag til bevægelsen og til samfundet generelt.. Ja, så i første fase efter opstartningsfasen bør man altså bare fokusere på velgørenhedsagtige bidrag, så systemet bliver væsenligt simplificeret, og hvis man så på et tidspunkt gerne vil indlede en fase, og det kan vel i og for sig være frit for i princippet, om man gør..(?) hvor man begynder at gøre disse bidragsaktier til en mere central og fundamental del af hele økonomien, og altså ikke bare en ting, der ligesom ligger ovenpå (hvor man altså så begynder at erstatte valuta generelt med alsidige handlings-aktier), så.. ja, hvad gør man så i denne overgangsfase? ..Tja, hvis man nu havde det helt åbent, hvad man kan registrere som aktier, sådan at folk med tiden i princippet nærmset kan få alle deres handlinger registreret, og hvis man så bare på en eller anden måde for et godt tidafhængigt system, hvor man dog ikke kan udnytte samme tidsafhængighed... Hm... ..Og nu kom jeg også lige til at tænke på, om det ikke faktisk ville være godt med nogle start-bonus-kurver. For man vil jo f.eks. gerne have generelt, at folk skal oprette deres gerninger som aktier i stedet for at finde konventionel løn og/eller funding for dem, og hertil ville det jo være smart med en bonus, så der kan gøres noget reklame. Og generelt er der måske også bare rart med nogle mere håndfaste gevinstkurver, når man investerer, så man kan få en stærk følelse af, at de første investeringer er særligt værdsat; og hvor der nemlig ikke kan være så meget diskussion om det. Jeg det ville nok være godt at holde sig mere til nogle faste, ikke-så-diskuterbare belønningsudsigter i starten for aktionærene.. ... Hm, og det kan egentligt også være, at jeg generelt ser hele denne idé lidt for meget fra opfinderes synspunkt.. Jeg må lige sørge for at genoverveje det hele noget mere og prøve at finde mere ind til, hvad udsigterne mere præcist er for et samfund, der indfører sådan en type økonomi, og altså ikke bare i den fjerne fremtid, men mere i den nære.. (08.04.21)
%(12.04.21) Jeg har gjort nogle tanker torsdag aften, fredag morgen/middag og så også hen over weekenden. For det første tænkte jeg på, at den nye teknologi, som jeg gerne vil have denne idé til at frembringe, jo må skulle handle om at få mulighed for at belønne mere fair for folk, der bidrager til samfundet men uden at have repræsentation og "lønforhandlingsmagt," kan vi lige kalde det her. Og det må jeg (vist) meget gerne holde godt fast i. Den tanke fik jeg i torsdags, og nu her kan jeg også se, at bevægelsen egentligt meget gerne må fokusere på det semantiske web, open source programmering generelt og også alle former for nye mere open source forretninger. Dette giver også særligt god mening, da man alligevel skal bruge open source-ontologier osv. til at holde styr på lønningsniveauer/kurser osv. Fredag morgen/middag (eller nok i virkeligheden tidlig eftermiddag..) kom jeg også på den vigtige tanke, at lykke-forøgelser pr. handling faktisk er endeligt; eller dvs. medmindre man varigt ændrer kursen for hele civilisationen, men i så fald må man jo ligesom bare indføre et loft, hvis belønninger/kurser skal regnes direkte ud fra lykke-forøgelser.. Tja, og der findes faktisk også mindre ektreme ting, som man kan argumentere for, vil øge lykken varigt for fremtiden, så som at bevare historiske genstande eller historisk information, men det må man jo så bare have med, og.. Tja, man kunne jo fastsætte en belønning, der så bruger kurve-halen i udregningen. Ja, det lyder som en fin idé. Nå men pointen er altså, at selv for vigtige opfindelser kan man jo tillade sig selv at regne med, at pågældende opfindelse alligevel ville blive gjort på et senere tidspunkt, og derved vil lykke-forøgelsen, man så skal prøve at estimere, typsik være endelig. ..Tja, det kommer nok egentligt ikke til at betyde så meget for den endelige idé, for jeg tænker dog, at parameterrummet stadig skal være åbent, men måske er konklusionen er nok lidt, at tror lidt mere på, at bevægelsen godt må sigte mod en lykkeorienteret beregning, så snart det kan lade sig gøre.. Tja, og så fik jeg også en tanke, jeg synes har været ret vigtig, der fredag, men den er vist lidt svær at redegøre for, nu hvor jeg tænker over det.. Men et af de helt store take-aways var, at man nok sagtens kan.. kan have bevægelsen delt op, hvor hver underguppe faktisk kan gøre tingene på sin helt egen måde.. Tja, jeg kan jo vende tilbage til det, når jeg får mere samling på tankerne, hvis det bliver relevant (hvilket det helt sikkert gør, men det er ikke sikkert, det bliver relevant så at referere tilbage til de tanker..).. Ja, jeg skal som sagt tænke noget mere, men slutteligt her i weekenden har jeg så gjort nogle tanker om, hvordan foreningen, som nævnt, godt kan fokusere primært på sem-web og de tilhørende emner til at starte med, og at denne bevægelse dog ikke behøver et centralt firma.. hvilket nok egentligt er en ret vigtig tanke.. For det svarer nemlig bare til, hvis linux-fællesskabet gik sammen om at gøre linux closed source fremover, og begyndte at kræve eksempelvis betalte medlemskaber for at benytte systemet, eller rettere for at benytte de efterfølgende opdateringer af det; det tror jeg sagtens kan lade sig gøre. Og et sem-web-fælleskab kan også sagtens, tænker jeg, regne på sin bidragsgraf og se, hvis der er mange modtagere, der ikke bidrager tilbage (hvad der jo vil blive), hvorved man så kan lave en udregning i fælleskab, der siger, lykke- og fremskridtsmæssigt vil det nok være smartest på nuværende tidspunkt at gøre systemet mere closed, og begynde at putte ting bag betalingsmure fremover, så vi kan begynde allerede at betale af på nogle af de første bidrageres aktier (så at folk kan begynde at se, at bidrags-aktiernes værdi ikke bare er varm luft, hvilket så kan forøge lysten til at bidrage til systemet). Folk der så begynder at bidrage ved at betale for tjenesterne, bidrager jo herved også, og dette bør bestemt også noteres, også selvom betalingen er tvunget, det er lige meget. Jeg tror bestemt ikke, det vil være en dum idé (..Nej, det er jo klart, så det er jeg sikker på), at belønne sine første medlemmer for at skabe interesse og for at finansiere bevægelsen, slet ikke. Nå, men det var lige for at komme lidt up to speed efter weekenden. Nu til lidt mere seriøs idé-udvikling (er planen, om ikke andet). ... Ja okay, så humlen af hele systemet, som idéen er nu, er lidt, at man går sammen om at lave systemer, hvor man kan registrere bidrag og danne ontologier for parameterrummene osv., og særligt hvor der så er en måde at stemme om forskellige tiltag i, hvordan man belønner/donerer til forskellige bidragsydere i fælleskebet, og når man så er aktiv i fælleskabet og stemmer belønningsprincipper igennem, så kan man så i samme ombæring stemme belønninger igennem for denne form for aktivitet, og... Øh... Tja ja, og så er der så nogle principper, der skal gøre, at løfter bliver overholdt på en rimelig vis... Hm, jeg kan mærke, at jeg har svært ved lige at finde ind til noget fast igen, efter den seneste udvikling (men det er jo også lidt svært). En indskudt tanke er, at man måske også kunne komme langt med bare at starte med et princip om en demokratisk forretning, hvor folk stemmer via ontologi-parametre osv. --- det kan man jo selvfølgelig sikkert generelt (men dog lidt lang tid siden, jeg har tænkt i helt de baner..), men måske også ift. denne idé.. Ja, der er flere tanker lige nu, jeg lige skal få styr på osv. Jeg tror jeg vil gå en tænke-tur. Selv tak for det gode vejr.;) ... Nå, det fik jeg vist skrevet lidt for tidligt. Men solen titter da lidt frem indimellem, f.eks. nu her. Er kommet frem til lidt, men fortsætter bare alligevel tænkeriet lidt, så går nok bare ud igen. ^^  
%(13.04.21) Okay, nu tror jeg, jeg er nået til en ny version af idéen, som umiddelbart giver ret god mening.. Det skal, som jeg ser det nu, ikke længere være en central del af det, at man altid får sine registrerede bidrag omdannet til værdipapirer, man kan handle med igen; det kan komme lidt an på den underforening, man tilhører. Bidrag registreres stadig i hver underforening, og således registreres de af den samlede forening/bevægelse også, men det kan så være op til lokale implementationer af bevægelsen, hvordan disse skal belønnes i princippet, dog guidet af visse principper, som jeg vil komme tilbage til. Men ja, så nu er der så slet ikke noget grundlæggende lag i systemet, udover de principper man tror på og følger, som bl.a. er det, at det kan betale sig at være fair og belønne og straffe bagud, og at neglicering af at gøre dette derfor i sig selv bør være værd at straffe, så man derved forebygger imod det i fremtiden. Og et andet helt underlæggende princip er så, at man har en vis tro på, at fremtidens økonomiske system vil, hvis det forbliver bare en anelse liberalistisk (selv bare nok til at "belønninger"/"lønninger" eksisterer), belønne ud fra beregninger om, hvad der giver mest lykke/velfærd, og uanset hvad at det fremtidige system vil sørge for at respektere aftaler, kontrakter og forventninger i høj grad for folk fra en tidligere generation, især når dette system jo har lagt grund til, hvad end man finder på i fremtiden, hvis man nu går væk fra de økonomiske tanker, jeg beskriver her, og over til nogle nye. Dette princip er så ikke længere helt så vigtigt, fordi man ikke længere behøver de her bidragsaktier, der ikke har nogen udløbs-/veksel-dato på sig, men det bliver alligevel betydende, fordi hver generation alligevel skal kunne stole på, at den næste generation vil belønne den forrige på en retfærdig måde (for i dette system kommer man nemlig til at sætte sin lid til dette i en eller anden grad, det er i hvert fald lidt hele pointen: at man kan bidrage uden først at have klare lovmæssige kontrakter, der sikrer en belønning, men bare kan stole på at fælleskabet vil ære ens bidrag med en retfærdig belønning i stedet). Og fordi der så ikke længere er et centralt, grundlæggende system, som underforeningernes implementationer bygger ovenpå, jamen så handler idéen jo bare om beskrive mulige udgaver af underforeninger, der opnår visse fordele ved at følge visse overordnede principper, og så beskrive, hvordan sådanne underforeninger så med tiden kan smelte sammen og/eller indgå i et samlet fælleskab, der også følger nogle overordnede principper. Således kommer min "idé om en økonomisk bevægelse" altså nu, hvis jeg er på rette spor med de seneste ændringer, til bare at handle om, at beskrive nogle muligheder for at organiserer sig på en måde, så bidragere --- og her kommer vi så sandsynligvis særligt til at snakke om sem-web og tilsvarende/vedrørende teknologier --- kan belønnes mere bagud for deres bidrag (og her snakker vi også birdag i egenskab af kunder/investorer eller af planlæggere/bestyrelsesmedlemmer), samt for hvordan sådanne organisationer kan udvikle sig og arbejde og måske smelte sammen i fremtiden. 
%Så lad mig prøve at opsummere det / færdigudvikle det. Idéen er ret simpel nu, så simpel at den nok kommer til at kunne forklares som en tilføjelse til sem-web-noterne og ingen gang behøver sin egen sektion. Tja.. Eller måske skal man stadigvæk lige have lidt tilløb til den, for det er jo stadigvæk relevant for den med det fremtidige (lykkelige) samfundsøkonomi... Nå anyway.. Et godt forslag til en underforening (for at starte et sted) ville være en/et demokratisk forening/fællesskab, hvor man lægger en fælles (variabel) plan for, hvordan lønninger bør gives, som så kan udformes som en ontologi, der så i sidste ende kommer til at beskrive et (variabelt, udskifteligt) parameterrum for, hvad der karakteriserer diverse bidrag i fællesskabet. Og så finder man altså en demokratisk proces for, hvordan man stemmer ændringer i lønningsforholdene og/eller ændringer i ontologien igennem. Denne proces kan så være mere eller mindre PoS-vægtet kontra én-stemme-pr.-næse-agtigt; det kommer an på, hvad man beslutter i det/den pågældende underfælleskab/organisation. Man kan jo så også med fordel beslutte en særskilt stemmeproces for at ændre førstnævnte stemmeproces. Deltageres mål hver især med at fastsætte (be)lønningerne er jo så, at hver deltager har en vis interesse i at tilstrække bidrag til organisationen (dog hver især formentligt med deres egen vægtning på hvilke typer bidrag, de er mest interesserede i), samt en interesse i selv at blive tilgodeset af belønningsplanen. Desuden er der så endnu en vigtig interesse mere (vigtig for hele denne idé især), nemlig at de stemmeberettigede deltagere også er interesserede i ikke at blive straffet for deres afgivne stemmer i fremtiden ved f.eks. slet ikke at blive belønnet for dem. Og hvad er der så, der skulle udløse sådan en straf? Det kan for det første være, hvis stemmen tydeligvis er influeret af grådighed, sådan at vedkommende tydeligvis har haft sine egne økonomiske interesser langt mere på sinde end fællesskabets. Hvorvidt dette bliver straffet eller bare er en naturlig del af den demokratiske proces kan jo så komme meget an på det/den individuelle underfællesskab/organistation. Det ville jo være smart, hvis man ikke behøver det og i stedet kan udvikle et system, hvor der er plads til, at folk bare stemmer efter deres egeninteresser, men ellers så har man altså også denne mulighed at dæmpe selviske stemmer ved at installere en hjemmel for, at man gerne må fratrække belønning fra folk, der tydeligvis har stemt mere selvisk, end hvad godt er, og dermed måske har tynget underfællesskabet. Men ellers er den helt store ting, man bør fygte straf for i en sådan (under-)bevægelse/organistation, hvis man med sin stemmeberettigelse forsøger at indføre en uretfærdig behandling, hvor tidligere bidrag tydeligvis bliver nedprioriteret til fordel for nyere bidrag. Hvis en sådan bevægelse skal virke, hvor bidrag bliver belønnet bagud, så er det jo selvfølgelig super vigtigt, at dette bliver gjort på retfærdig vis, og at en ny generation f.eks. ikke lader den tidligere generation i stikken i den henseende. Og derfor er det vigtigt, at hvis en generation gør dette, vil denne selv blive straffet af den næste (hvor der så selvfølgelig er en naturlig hjemmel i hele fælleskabet om, at sidstnævnte straf af stemmeafgivere fra den forrige generation så er naturlig og derfor ikke i sig selv kan give anledning til, at næste generation så straffer én selv igen for at give den straf til den forrige (medmindre sidstnævnte straf var ude af proportioner pr. den grundlæggende straf-hjemmel i organisationen)). Så det er altså med dette princip, jeg tror på, at en sådan organisation kan fungere. Og samtidigt bør alle disse organisationer så også følge et princip om, at målet i sidste ende er at slå sig sammen for at benytte bedre og bedre planlægningsteknologi til at finde frem til bedre og bedre samfund(søkonomier), hvor man så (som en del af dette mål) også vil basere sine lønningssatser mere og mere ud fra, hvor meget lykke/velfærd de bringer til samtiden og til eftertiden (præcis hvordan denne beregning kommer til at være, behøver man dog ikke at spekulere i; det skal fremtidens samfund nok finde ud af). Målet med enhver organisation af denne art skal så dermed være på et tidspunkt at slås sammen med andre til en større organisation (indtil alle de undervejs dannede organisationer til sidst er slået sammen i en eller anden grad), og så vil der dermed også opstå endnu at mål for de stemmeberettigede deltager i hver forening, nemlig ikke bare at blive tilgodeset for deres bidrag som stemmeafgiver, når en ny generation kommer til at træde til inden for samme organisation, men også at blive tilgodeset, skulle organisationen smelte sammen med andre organisationer i ens levetid, hvorved der jo også dannes en ny effektiv ledelse derved, som deltagerne/medlemmerne hver især så bliver underlagt. Dette gør så, at hvis en forening bliver korrumperet på et højt plan, således at magten med belønningerne kan gå i arv til de samme korrumperede interesser for hver generation, jamen så mister denne organisation muligheden for at merge med andre ikke-korrumperede organisationer, og herved har den korrupte organisation mistet sine fremtidsudsigter, i hvert fald til at tiltrække de samme former for bidrag fra folk, som så stoler på at de bliver belønnet retfærdigt efterfølgende (uden først at skulle indgå en masse lønningskontrakter). Og hvis der opstår mange forskellige af disse organisationer, så vil der altid være en god mængde af dem, der ikke er korrumperede, og disse vil så holde stand og aldrig smelte sammen med korrupte fællesskaber, før korruptionen er redt ud igen. Så ja.. Og ja, princippet med alt dette, som jeg gerne vil nævne, er så, at nysammensmeltede organisationer så også vil have samme interesser i at tilgodese alle de nye medlemmer på en retfærdig måde (så gode bidrag bliver retfærdigt belønnet og negative bidrag, så som alt for selvisk opførsel som stemmeberettiget/medbestyrrende medlem, bliver trukket fra), nemlig fordi en fosømmelse af at opfylde dette, vil føre til en forventet afstraffelse enten fra den næste generation eller fra den næste overgenerations ledelse, skulle organisationen smelte sammen igen med andre til en endnu større organisation. Og da konceptet bag sådan belønnelse og afstraffelse er rimeligt simpelt og rimeligt centralt for at sådanne bagud-belønningsorganisationer kan fungere, så tror jeg altså altid, at medlemmerne kan antage, at de fremtidige versioner af organisationen, hvad end det bare skyldes en udskiftet ledelse eller en sammensmeltning med andre organisationer, vil også følge de samme principper. Cool, så det var altså sådan, at jeg nu forestiller mig, at organisationerne ligesom kan opnå, at de hele tiden holder sig selv og hinanden i ørene, så medlemmerne kan forvente at få deres bidrag belønnet retfærdigt (ud fra, hvad der nogenlunde er blevet lovet). Nå ja, jeg skal også lige sige, at man jo selvfølgelig godt kan have ændringer i belønningsniveauerne, det er jo en del af det, men man skal bare altid sørge for at tidligere medlemmers forventninger ikke pludselig bliver vendt på hovedet; man skal således altid sørge for at lave langsomme ændringer, så ingen føler sig pludselig snydt. Og så er det selvfølgelig også samtidigt vigtigt, at ændringer i belønningssystemet aldrig går unødigt ud over en tidligere generation mere end det gør den nuværende, det klart; en stor del af, hvad "retfærdighed" vil sige her, er til gamle bidrag altid bliver belønnet ligesom, hvis de var gjort i nutiden (hvor der så (på magisk vis) var ligeså stor brug for dem i nutiden, som der var i datiden, hvor de blev gjort). Okay, men nogle principper, der sørger for at organisationerne holder sig selv og hinanden i ørene, er selvfølgelig ikke hele idéen. Jeg skal også forklare, hvorfor jeg tror, sådanne organisationer kan have en økonomisk fordel, der giver grobund til dem, og hvordan de altså kan begynde at tjene penge ind, som de kan begynde at donere til de tidlige bidragere osv. ...
%(15.04.21) Nå, jeg tror faktisk, at jeg kommer til at droppe tanken lidt om at økonomisk system, der både bygger på, at gerninger skal belønnes retfærdigt og med forøgelse af lykke/velfærd på sinde, og som også skal kunne opbygges fra nærmest ingenting af en bevægelse (inden i det nuværende system). Jeg tror nemlig ikke rigtigt længere på, at det holder. Så jeg vil gå lidt tilbage til mine tidligere tanker, som bare handler om, hvordan man kan lave ret åbne organisationer, der kan hjælpe til at "belønne bagud," dvs. kan sørge for at de bidragene medlemmer bliver retfærdigt belønnet og får tilsvarende meget at skulle have sagt i foreningen.. Og nu har jeg så bare nogle tanker oveni, der gør billedet lidt mere simpelt. Jeg har nemlig lige fået et lidt bedre overblik, synes jeg, over hvad et økonomisk system grundlæggende er: vi kan nemlig se det som en (directional (kan ikke lige huske det danske navn)) handlings-graf mellem ydere og modtagere samt en løfte/samtykke-graf. Og et godt økonomisk system er så et, der forøger velfærden via handlingsgrafen samt sørger for, at deltagerne ikke kommer til at føle sig uretfærdigt behandlet, jævnfør løfte/samtykke-grafen. Dette er i sig selv ikke nogen særlig signifikant indsigt, men jeg kom så også, i går, til at tænke på, at folk jo med gode semantiske og matematiske teknologier jo sagtens kan dissekere alt dette og sørge for at gå sammen om den bedste løsning for dem (og deres samvittighed, ikke at forglemme), især fordi denne teknologi også vil gøre det nemt på automatisk vis at nå frem til aftaler (eksempelvis via automatiserede handler, hvis der er behov), der er optimerede for alle deltagere. Så jeg mener altså, vi kommer til at nå til et punkt, hvor det bliver meget nemt for folk at indgå nye aftaler på tværs af store befolkninger, og at styrresystemer og økonomier m.m. dermed hurtigt kan ændre sig (men hvor man jo altid vil respektere løfte-samtykke-grafen til en vis grad, medmindre man gerne vil lægge op til selv at blive ladt i stikken af den næste generation (eller af alle mulige andre foreninger/instanser)). Så det om folk i sidste ende når til en et (lykke-velfærd-)etisk samfund (hvad jeg helt klart mener, de vil), kommer jo bare i sidste ende an på, hvor samvittighedsfuldt mennesket er og rationelt (for hvis man analyserer eksistens-mulighederne med rationelle øjne, så vil man, mener jeg, jo nok komme frem til, at forsøg på maksimering af den samlede lykke-velfærd faktisk også \emph{er} den "selviske" handling oven i købet) kontra, hvor nemt det er, og hvor meget der er at vinde ved, at lave et ulige og uretfærdigt samfund (og altså også hvor villige folk så også er til at prøve at gøre dette). Og ja, jeg tror altså vi er i meget gode hænder på det punkt, altså for os der bekymrer sig om (\emph{bekymrer sig om}, og ikke \emph{er bekymret for}) fremtidens samfund. Og jeg er heller ikke vildt bekymret for, i øvrigt, om folk ikke nok skal kunne danne gode foreninger/organisationer, der kan sørge for, at bidrag til det semantiske web kan findes en måde at tjenes penge på, og at man ikke kan ordne det sådan, at de penge så går til dem, der har fortjent dem, således at man rigtigt kan få gang i sem-web-bidragene. Det tror jeg bestemt man kan, endda også selvom at man som ledelse (som i øvrigt gerne må være ret decentraliseret) i en organisation kun har det incitament, at belønningerne giver sikkerhed til nye brugere til, at de også vil blive belønnet, samt også at man ikke taber de vedrørende gamle bidragere, hvilket nemlig også vil være vigtigt, såvel som at det selvfølgelig også giver et bedre brand til organisationen generelt, at man har ære og integritet. 





%Ny Indledende Brainstorm for lige at samle op på det, inden jeg skriver (så jeg ikke skal skrive så meget om igen, ligesom for blockchain-/KV-sektionen) (start d. (05.04.21)):
%
% Så, det handler vel egentligt om at finde nogle grundprincipper, lidt ligesom for det, jeg lige har skrevet omkring KV, men hvor det ikke er generel sandhed, man vurderer, men generel lykke/velfærd. Og jeg har så nu en idé til, hvordan man kan gøre den subjektive vurdering af, hvad der er lykke/velfærd, til noget objektivt i praksis. Og så tænker jeg jo lidt, at fundamentet af bevægelsen så er, hvordan vurderer og belønner dette, men så kan man så med fordel starte med at lave lidt "urene" implementationer af denne i form af firmaer og foreninger. Og det er her at et godt system, der løngiver efter et "scaling-princip," som jeg jo kalder det, samt mange andre praktiske forslag kan komme ind i billedet. Ja, så hvis jeg bare kan finde et godt fundament, så bør resten ligesom falde lidt på plads derfor (for hvis man så har en god idé til en implementation, så gør det ikke længere så meget, at den ikke holder vand 100 \%, for når den bygger på et sådant solidt fundament, så vil der altid bare være incitament til, at lappe sådanne huller på en god og glat måde). Og jeg har allerede et fundament, der virker fornuftigt nok, til at vurdere sandhed (se kryptovaluta-undersektionen i sem-web sektionen lige ovenfor), og jeg tror også, jeg har et holdbart princip til at vurdere velfærd/lykke ovenpå dette. Så den trejde brik, jeg lige skal overveje igen, er, om jeg også har et godt princip om, hvordan man belønner bidrag ud fra dette --- et princip som heller ikke har selvreference-paradokser.. Hm... Det jeg hidtil har tænkt, er jo bare omkring, at vedtage et system for, hvordan deltagere opretter bidragstokens, som så også inkludere en måde at trække negative bidrag fra, og hvor man bare nøjes med at tælle fra starten af at folk "melder sig ind" i bevægelsen... Men virker dette ikke lidt usolidt..?.. Jo, man bør vel nok vedtage noget mere simpelt, så som at man betragter alle noterede handlinger fra medlemmer som et slags bidrag.. Hm... Hm, ah, kunne man så ikke gøre noget med, at man bare selv kan samle en række bidrag til en token, som derved kan handles med, men hvor der så hele tiden er en slags baggrundstoken over alle de handlinger, men ikke har tokenficeret, som man så i sidste ende så også skal belønnes (og muligvis straffes; det kan vi ikke udelukke på det grundlæggende plan --- selvom jeg nu personligt vil tro, man helt klart kommer længst med ikke at skabe skyldnere, og at forebyggelse i øvrigt for det meste slår korporlig straf) for. Så når man altså udsteder en token, så fjerne man pågældende handlinger fra sin egen tidslinje, nærmest, men så alligevel ikke indtil man sælger token'en videre, for den der så ejer omtalte token, har så ret til alle belønninger for tokenen. *(Nice nok idé, så den får jo lige en datoangivelse: (05.04.21)) Det ville måske være en idé så at antage, at man kun kan handle med de gode sider af ens handlinger, så man altid selv bliver straffet for... Hm... Indskudt note: Systemet er så kun selvopretholdende, så længe man generelt (eller bare at nok personer) vurderer at systemets opretholdelse fører til større velfærd. OK; tilbage til overvejelsen om, hvorvidt man også skal kunne sælge sine dårlige handlinger... Tja, man kan vel ikke altid adskille gode og dårlige handlinger, for nogen kan jo være begge, og andre gange ved man ikke, at en handling er dårlig, før det går op for én, at man eksempelvis skulle have tænkt sig bedre om. Ja, og det løser jo også sig selv, for hvis man bare undlader at sætte nogen sådan restriktion i grundsætningerne af systemet, og så finder ud af, at sådanne restriktioner kunne være meget gode, jamen så kan og bør man jo bare indføre det i et implementationssystem, der bygger ovenpå dette lag. Nice. 
%Okay, men jeg skal så stadig genoverveje, hvornår disse tokens.. indløses.. hey, kunne man mon gøre noget tilsvarende a la det, jeg lige har tænkt omkring min seneste version af et KV-system? Altså kunne man gøre det sådan, at det er op til deltagerne selv at veksle deres værdipapirer til risikofrie værdier (nemlig ved at aftale at udjævne og/eller helt eliminere hinandens risici)? Ja, det virker da som om, det bare er det, der er svaret..! (05.04.21) ..Og man støder ikke ind i problemer med, at andre grupper ikke vil annerkende en anden gruppes papirer, så længe det kan bevises, at handlinger rigtigt nok blev gjort. ..Tja, eller det er vel medmindre, at forskellige grupper har forskellige definitioner om, hvornår man blev "medlem" af bevægelsen, og/eller om hvornår man bør regne handlingerne fra. Men det umiddelbare svar er vel så bare ikke at sætte noget starttidspunkt, således at alle levende væsner, og særligt alle mennesker (og fordi der er frihed til at man godt må starte med "urene" implementationer i opstartsfasen (hvor længe denne end kommer til at vare), så kan man altså, vil jeg mene, sagtens bare starte med kun at medregne mennesker (selvfølgelig så længe man også bare tager dyrevelfærd seriøst)), bare anses for "medlemmer" --- og altså at man dermed slet ikke har et sådant skel. Og hvis man så f.eks. har lavet en god handling, men allerede har fået løn, så er det jo ikke sværere, end at handlingen noteres som ``lavede x stykke arbejde imod løn,'' i stedet for ``lavede x stykke arbejde uden løn for det, men bare for velgørenhedens skyld,'' hvorved eftertiden kan beslutte hvilken.. efterløn.. sådan en slags handling fortjener. Ja, `efterløn' tror jeg, jeg vil holde fast i som begreb her. Nå, men nu kan jeg jo så se, at jeg mangler endnu en ting, nemlig hvordan løn skal vurderes til en given handling.. Tja, det nemme ville jo være, hvis man enten kunne bruge mit skaleringsprincip som noget grundlæggende, eller måske endnu bedre bare kunne bruge et princip om, at aflønningshandlingen i sig selv er en handling.. Hm, men det sidste er jo desværre netop selvrefererende, og det kræver også at aflønningen igen sker som en handling fra en styrende enhed... Tja, måske kan man godt overleve selvreferencen/cirkelslutningen/hvad-man-vil-kalde-det, men alligevel... Hm, og skaleringsprincippet lever på en måde lidt sit eget liv, og er altså ret uafhængigt af det teorigrundlag, jeg her prøver at gøre til fundament... Hm, kunne man.. kunne man bruge et slags princip om at betale folk, hvad man ville anse for mest gavnligt, hvis man skulle spole tiden tilbage..? Hm, det giver jo umiddelbart nogle kuasilitetsproblemer (pga. kaos osv.), men... Tja, men er skaleringsprincippet så egentligt ikke mere fornuftigt? Men får man ikke også en cirkelslutning med dette princip.. jo, det gør man vel... Hm, medmindre måske man sagde noget med, at lønen så regnes efter, hvad man ville beslutte, hvis man skulle regne på det som en styreenhed og udgive lønnen; uagtet om man ren faktisk behøver at gøre dette i praksis... Ah, ja, det virker faktisk som om, det er det, der bliver løsningen.. (05.04.21)
%(06.04.21) Tja, det holder ikke helt, det her, for man kan ikke basere det på, hvad der sker i den fjerne fremtid, for belønninger er jo stadig noget arbitrært bestemt... Tja, eller man kunne måske godt vedtage noget, men... Hm... Nå, det kan jeg lige vende tilbage til, men problemet er altså, at.. ja, at belønningsfortjeneste(-nytte) vs. bidragsnytte kan variere i tid i virkeligheden. Ja, sådan kan man vel se på det.. Så var det jeg tænkte lidt på i aftes, om ikke man kunne gøre noget med at sige, at belønningen skal udregnes relativt til den tid, hvor man først har et godt overblik over konsekvenserne.. Selvfølgelig er det lidt svært, for det ved man jo aldrig helt, hvornår man har.. ...Okay, jeg kan lige sige, at jeg også i gør så tænkte lidt på, om man jo ikke skulle gå tilbage til at have underforeninger, der så bestemmer diverse lønningsniveauer, og nu tror jeg muligvis, jeg øjner, hvad der kunne føre til en løsning.. ... Hm, nej for foreningsdannelse medfører faktisk også nogle problemer... Hm, medmindre man kan finde en god måde, at smelte forskellige foreninger lidt sammen.. Jeg tænker jo på, at deltagere måske kunne stemme om en model for, hvordan belønnings-bidrags-forholdet er (og hvordan skaleringsberegningerne udføres for den sags skyld).. Uh, og man kunne ikke bare gøre noget med, at man som organisation for afstemningshandlingen sat på handlingshistorikken, så at en afstemning, der fører til, at en generation tænker mest på dem selv og ikke så meget på fremtidige generationer, f.eks. så automatisk bliver trukket fra i værdi igen, fordi denne generation så ligesom får givet sig selv en bøde automatisk ved en sådan afstemning..?.. Kunne man ikke gøre noget i denne stil?... Tjo, men så er problemet dog, at fremtidige generationer så bare kan beslutte sig for, at fortidige handlinger ikke var så gode, og at man derfor ikke skylder fortiden noget, og dette vil så ikke helt kunne straffes igen, medmindre altså at et sådan integritetsbrud vil føre til mindre tillid for den nuværende generation. Ja, og dette er jo et ret centralt problem, som man skal sikre sig holder.. ...Hm, og stammer problemet ikke fra, at man nærmest altid.. har mere gavn af, at fortidige bidragsydere tror på, at de bliver belønnet mere end... Hm.. Og selvfølgelig (hvad end problemet helt præcist er) vil vigtigheden af at opretholde integriteten af et godt system kunne redde dette forhold, men det kræver så, at man har et system; det er ikke nødvendigvis nok bare at have overordnede principper, eller er det?? Hm...  På den anden side er målet jo ideelt set at forøje al velfærd, ikke bare fremtidig.. Hm, er der egentligt overhovedet et problem, havde jeg nær sagt.. ..Nå jo, problemet kommer jo også især i og med, at man kan sælge sine bidrags-værdipapirer videre.. Hm.. Tja, men hvis man nu i stedet nøjes med at se på det, som at man kan sælge sin risiko videre, hvad så..?.. ..Og at man "sælger" bidragsaktier ved at oprette lån i dem, kan man det..? Og samtidigt kan man jo godt sikre sig, at dommen sker uden hensynstagen til ejerskabssituationen af aktien.. Hm.. Ja okay, så man kan nok noget her, men vi kan stadig så vende tilbage til problemet om, hvad man skal gøre med tidsvariationen i belønningerne, for her er det netop ikke sikkert, at tingene løser sig selv.. Og i øvrigt kan meningerne ikke bare variere over tid, men også over sted, og dette kan altså også blive et problem... ..Hm, man kunne måske sige noget med, at en generation først skal leve op til at give fortiden bedre belønninger end den ventede, hvis de vil forhøje deres egen, men kan dette så ikke også føre til negative spiraler mon?.. Indskudt: Hm, og det er virkeligt et problem det her, for i en fjærn fremtid bliver det nok sådan, at det fleste ikke rigtigt bidrager med så meget, men at få individers handlinger så kommer til at betyde en hel masse (f.eks. kunst eller.. folk der varetager administrationsarbejde..? Hm.. Åh, det er sgu lidt svært det her..). Tilbage til mulige negative spiraler osv.: Tja, skal jeg lige tænke over.. Problemet er, at man ikke bare kan koble værdi til velfærdsforøgning direkte... Tja, men kunne man ikke også komme langt ved bare at sige, at man bruger skaleringsprincippet ift. samtidens liberale prissætninger.. og så tænker jeg nemlig lidt, at aktie-havere bare selv ansøger om, at få aktierne bedømt og vekslet til noget mere fastlagt, når de synes.. og hvis man så generelt har et forhold, at uforudsete begivenheder, der får bidrag til at stige eller falde i værdi, nærmest altid vurderes, når bidraget er gjort på ærlig vis, så at uforudsete positive ændringer bliver belønnet, men at uforudsete uheldige ting ikke bliver afstraffet. For så har deltagerne nemlig incitament til at holde på aktierne så længe som muligt, og kun veksle dem, når de er i pengebehov.. Ja, hvad med sådan et billede? ...Tja, det er der vel stadig problemer med... Og det fjerner sig jo samtidigt også fra et fundamnet, der ser på lykke-velfærdsforøgelse.. ...Hm, nu fik jeg lige en tanke omkring, om man ikke kunne gøre noget med at tage udgangspunkt i stemmer, og belønne stemmer alt efter, hvordan de er med til at hæve lykke/velfærd?.. Tjo, tja... Tjo, der er da lidt en idé at spore her, er der ikke? Men bliver det ikke stadig umiddelbart lidt selvrefererende..? ... Okay, jeg har tænkt lidt videre over det og prøvet at zoome lidt ud og se på, om man kan skille det i lag, så der ikke bliver en selvreference, og så har jeg jo så selvfølgeligt zoomet helt ud til spørgsmålet om, hvad der et etisk samfund (hvor min holdning jo er, at man bør gå efter at maksimere velfærd/lykke, men hvor man godt kan tillade sig at fjerne fokus lidt fra den enkelte persons handlinger, men se det mere fra et perspektiv om, hvad der er de etisk rigtige beslutninger for samfundet, og ikke fra den enkeltes synspunkt (for man kan jo f.eks. sagtens i princippet have et således etisk samfund, men hvor ingen invivider faktisk går op i at prøve at handle etisk (men hvor alle motivationerne bare kommer fra systemet, de indgår i))), men jeg når jo bare til, at en ledelse af et etisk samfund nødvendigvis må sørge for som en del af arbejdsfunktionen at hele tiden at evaluere og lave bestemmelser, der skal sikre imod egen og fremtidig korruption, således at den etiske kurs bliver holdt og ikke går i vasken (midlertidigt self.) pga. selviske handlinger fra styringens side.. Men ja, det bringer mig bare ikke rigtigt i mål, for idéen er jo at finde et konkret system, der kan udgøre et robust og ikke-nemt-korrumperbart system; ikke beskrive en overordnet etisk kurs for at nå frem til sådan en idé. Men nu er jeg så lige kommet tilbage til nogle tanker, der kan være brugbare: Kan man ikke bare sige noget med, at der ikke \emph{må} (og jeg kan vende tilbage til, hvad 'må' indebærer) diskrimineres ift., hvornår (..og hvor..?.. Hm, det kan jeg komme tilbage til) handlingen blev lavet, når der aktien skal veksles til mere fastlagte penge. Tjo, det giver en vej mod en løsning for ét problem, men hvordan skal man så nå frem til belønningen, og hvad hvis forskellige grupper er uenige herom? Tja, kunne man mon sige noget med... Jeg føler, jeg går lidt i ring nu, men hvis man så kunne gøre, så at belønningsbeslutningen igen bliver bedømt på baggrund af, hvor meget lykke det giver (jep, totalt meget i ring, men føler, det er godt, at tænke lidt på tastaturet).. Og så kommer jeg nu tilbage til, om ikke man så kunne adskille det, så man egentligt i første omgang skal udregne belønningsbesutnings-belønninger ud, og at det andet så kommer i laget lige oven på dette... Men så bliver det næste spørgsmål vel bare, hvad med belønningbelønningsbeslutningsbelønningerne; hvordan skal de belønnes? Og hvem skal beslutte denne belønning, og hvordan skal de så belønnes?... Ja, det bliver en anelse fjollet, hvis man ikke lige kan finde en måde at knække selvreference-kæden på... Hm, og nu tænker jeg bare på noget; og man kunne ikke lave et eller andet med, at definere et bidragsrum ud fra anstrengelser/ulejlighed (inkl. risikotagning), gavn for andre skabninger (også hvor risici er inkluderet i udregningen) og privilegier/dygtighed til at kunne udføre handlingen.?... ..Så man i stedet for at sammenligne med alle mulige komplicerede forhold i samfundet og bruge et omskaleringsprincip, så at man bare fastsætter lønninger ud for et sådant mere abstrakt parameterrum (men dog med nogle i princippet målelige parametre, hvis man finder en god måde at definere dem på..)..? Hm, så man altså prøver at modelere belønningerne, men ja, det giver jo meget god mening uanset hvad. Men hvad kan man så bruge dette til ift. de problemer, jeg slås med her..?.. ..Ah, hvis nu bare man kunne lave en måde, hvor hver person bare kan have deres egne parameter indstillinger, og hvor hver handling så på en eller anden måde kan regnes ud ved at se på personens egne indstillinger samt dem for de personer, handlingen berører (altså på en eller anden fair måde)..!?! Interessant... (06.04.21) Hm, og hvad kunne denne fair måde være?.. Det skal jo helst være baseret på et simpelt princip, men det er måske lidt svært at finde... (Hm, og der er dog lige det, at det ikke umiddelbart fobinder så godt til overordnet lykke, og særligt også at det så kun kan regnes ordenligt for personer og ikke bare for alle skabninger..) Hm.. Nå, nu tænker jeg lidt: Kunne man ikke bare gå tilbage og byde velkommen til "cirkelslutningen" og sige, at bevægelsen bare bør holdes oppe pga. en tro på, at dens beståelse fører til mere lykke, og hvor alle handlingsaktier så bare betales ved at en ny aktie udstedes, som så har værdi, fordi man tror på, at folk senere igen vil betale imod aktier, fordi disse så igen vil tro på... ...? Men jeg har det dog helt klart som om, der lige skal et mere solidt grundlag (i form af et godt princip) under dette system først... Hov forresten, anstrengelser... behøver de at være med i omtalte rum, det gør de vel ikke rigtigt?.. Og så har vi mest bare privilegie/evne/dygtigheds-forhold indover, men kunne man mon så ikke finde en måde, at få sådanne forhold ind som et lag ovenpå det grundlæggende, når man skal give belønning..? For hvis så, så kunne man jo bruge et \emph{rigtigt} simpelt princip.. (der i princippet også må være det naturlige resultat af et, om ikke andet, simplificeret liberalistisk system, hvis man modellerer et sådant).. Men hvordan kan man så tilføje eller trække disse forhold fra i aktierne; kan man det?.. Tja, det ville jo svare ret meget til en skat (især hvis man også bare siger at 'skat' godt kan dække over en negativ skat; altså en slags bonus/ydelse).. Hm, og er det så ikke bare folks privilegier, at opbygge stater m.m. oven på det økonomiske system, hvor man så kan have skat; helt ligesom vores nuværende samfund er det ift., hvis man tænker staterne, som noget der bygger oven på diverse valutasystemer som landet/staten bruger.?.. (Disse tanker virker godt nok bekendte, så det er faktisk muligvis noget (meget muligt endda), jeg har tænkt på før (for jeg har jo netop også haft en ret god mavefornemmelse omkring, hvad jeg nåede frem til, inden jeg begyndte på disse brainstrom-noter)). Hm, men dette beskriver så en økonomi, der så altid kan ændres igen og modificceres af den stat, der benytter den, men hvor belønning regnet ud fra lykke/velfærdsskabelse så er mere oplagt, ligesom at vores nuværende samfund lægger op til, at man erhverer sig de midler, man selv kan købe ved enten at sælge ting man ejer, eller sælge arbejdskraft m.m. (men hvor det så i praksis ikke helt er sådan, bl.a. pga. sociale ydelser og også pga. skatter til den anden side osv.).. Men to spørgsmål er så, kan dette blive til en styreformsbevægelse også (hvor politikkere også på en eller anden måde bliver tvunget/lokket til at følge lykkeberegninger i højere grad), og hvordan kan denne økonomi opstå og vinde frem indeni en eksisterende økonomi; kan den stadigvæk det? ... ...Åh!.. Hvad med at folk "stemmer" på et belønningsparameterrum via hvordan de værdisætter aktierne selv; hvor meget værdi de vælger at anse dem for at have?..! Hm.. Interessant nok.. (06.04.21) ..He, det ville jo lige netop være en simpel måde, hvorpå man kan finde en værdi af en handling, på tværs af de mennesker, den berører..!;):D.. (06.04.21) He, jeg kan godt se nogle mulige problemer osv., men det må jeg jo bare tænke videre over i morgen.:) 
%(07.04.21) Okay, så jeg tænker altså nu på et system, hvor man definere et parameterrum, der muligvis bare kan bestå af en parameter... *(Nej, er det egentligt ikke bedre at medtage en eller flere parametre for privilegie/dygtighed (og så kan man i øvrigt også godt dele gavn-parameteren op i flere), og så bliver det bare lønnen i hele dette parameterrum, man stemmer om.) Hm.. Og så er tanken, at man sørger for ikke at diskriminere, når en aktie veksles. Og så har jeg lige tænkt, at man jo måske bare kunne sige, at fælleskabet kan gå sammen om at rydde op i gamle anliggender, vil at gå med til at veksle en aktie, hvilket så i sig selv er en handling, men at ejeren så altid bare kan modsætte sig dette (sikkert også noteret som en handling, hvorfor ikke), hvis denne synes, at lønnen er for lille.. Hm, men kan dette ikke bare ske helt liberalt ved at man går sammen om at neutralisere risici?.. Jo, hvorfor ikke?.. For gør det noget, hvis aktierne hober sig op? Ja.. For det bringer mig jo tilbage til, hvordan man skal betale for handlinger... Hm, hvis man kunne regne med at det at modtage gavn fra en handling kunne ses som en nagativ handling.. Hm, eller om ikke andet, at man tager 'gavn fra handlinger' med i regnestykket... Nå ja. Det skal det jo klart være, for ellers kan der også nemt komme totalt uballance  i, hvem der modtager gavn fra handlingerne, og dette vil slet ikke være særligt godt. Uh, og så skal man også medregne 'havde mulighed for at drage gavn af handlingen,' sådan at man kan forhindre, at folk ikke benægter sig selv glæder til ingen verdens nytte andet end at så, skylder de ikke ligeså meget væk igen... Tja, eller man kan i hvert fald modvirke sådanne effekter (hvilket så er op til folk med deres usynlige stemmeret). ..Og der er i øvrigt intet til hinder for, at man visse steder bygger systemer op, der hjælper med at fastsætte kurserne, og hvor man således godt kan gøre det til en mere direkte demokratisk proces, hvordan kurserne for de forskellige aktier skal ligge. ..Og så man man næsten have et eller andet princip om, at folk skal fordele de negative aktier ud på de resterende aktier, de ejer, med jævne mellemrum... Hm, og man kunne faktisk også have nogle grundaktier, som svarer til, hvad man har ret til, som person, og dette kunne så være en måde at bringe sociale ydelser / universiel indkomst og skat osv. ind i billedet..! Indskudt: Angående et lykke-etisk system i alt det her: bliver man ikke bare nødt til at sige, at etikken bliver et grundlæggende koncept, man bør prøve at stride imod, og hvor man så kan komme med bud løbende, ligesom det der bliver mit bud her, på systemer til at gå i denne retning? For selvreferencen gør nemlig lidt, at konceptet ikke rigtigt giver mening andet end et etisk koncept man forsøger at følge. For indtil man finder et 100 \% holdtbart system, hvad man jo nok aldrig gør, jamen kan det jo kun være en grundlæggende overbevisning, der holder en på sporet; et system kan nelig selvsagt ikke gøre det. Ja, det giver totalt meget mening, nu jeg skriver det. Og \emph{hvis} man så skulle finde et rigtigt holdbart system, jamen så er det jo bare fint (for så er det jo alligevel en etisk retning, der skal få folk til vælge at implementere dette system); der er bare ingen grund til at regne med dette indtil da. Ok, super lige at få klaring på etik-drømmen; den må bare forblive en etisk retning. Og så er det gode ved dette nemlig, at folk så, mener jeg, godt kan regne med, at et lykkeligt system ikke kollapser på en hård måde, for hvis samfundet allerede fungerer ret godt, så vil der altid være etisk incitament for at lave en blød overgang til det næste. Og dermed kan man altså sagtens arbejde med et system, der anteger, at det altid vil vare ved (når værdien af aktierne skal defineres/vurderes/modeleres), også selvom man ved, at dette præcise system nok bliver erstattet på et tidspunkt. Nå, tilbage igen. For en anden tanke jeg havde, er, om ikke det bliver et problem, når tingene fungerer på et fundament af kurser, at det kan være svært at opnå lokale kurser, der er anderledes fra de globale, og således kommer man ud for, at der bare bliver én politik for den globale bevægelse?.. Nej, måske ikke! Kan man ikke bare danne en graf over alle deltagere og så inkludere denne nærmest rumlige parameter (ja man kunne nærmest forestille sig den som en 3D-vektor i det faktiske, fysiske rum i verden (selvom man nok skal implementere den på anden vis; med en graf f.eks., hvor kanterne representerer tæthed i forhold til, hvor tit handlinger bliver udvekslet)). Og så får folks variationer i vurderinger af, hvad kurserne bør være, i første omgang indflydelse på deres lokale område, og det er jo super. Ja, det må man næsten kunne... Hm, tja.. Så er der bare lige det, at andre kan "spekulere" i et lokalområde og i princippet ændre kurserne fra fjernt.. Hm... Tja, men man kan jo regulere imod det, og især hvis man bare sørger for, at al spekulation også noteres som en handling; så kan man nemlig endda regulere imod det via selve systemet (fordi folk jo så kan vige fra aktier, der inkluderer sådanne handlinger).
%Jeg tror altså umiddelbart, at dette kunne være et godt udgangspunkt til en bevægelse. For selvom det selvfølgelig ville være nemmere, at hvis man bare gjorde noget tilsvarende, men hvor kurserne blev bestemt på demokratisk vis, når altså først man har dannet sådanne foreninger, så vil dette jo basere sig på noget egnetligt ret kompliceret: hvordan definerer man, og hvordan styrer man disse foreninger. Men ved i stedet bare at lade systemet bestå i et system til at danne aktier, og så lade folk selv bestemme kurserne på alle disse aktier på helt liberal vis, så har man jo et rigtigt godt decentraliseret system. Og så kan man netop bare sætte sig for, hvis man synes, at skabe diverse demokratiske foreninger så hurtigt som muligt, som så bare kan bygge ovenpå dette økonomiske fundament. Og en kritik kunne så være, jamen ændrer en sådan bevægelse så ved noget, hvis det bare bunder i, at folk pludselig skal til at værdsætte nye former for værdi-papirer. Ja, for vi ved jo (f.eks. fra blockchain-KV'er), at sådanne strømninger sagtens kan vinde frem ret hurtigt. Og fordi man kan basere hele bevægelsen i nogle grundlæggende etiske tanker, som alle og enhver ved altid vil have en vis opbakning (og en stor en, hvis vi skal være ærlige; selvfølgelig vil fremtidens etiske strøminger rette sig mere og mere mod utilitarisme, især når vi altså kiiger på 'etik' fra systemer/regeringer/samfunds synsvinkler, og ikke fra det enkelte individs synspunkt (det kan man gøre i andre sammenhænge, hvor dette er relevant)). Og med den simple antagelse om, at fremtiden vil med al sandsynlighed vil se meget positivt på tidligere bevægelser, der har først en masse godt med sig, og derfor  sandsynligvis i meget høj grad vil respektere kontrakter gjort i dette system osv., og således vil sørge for, at eventuelle fremtidige overgange til nye systemer alle vil ske på en så blød måde som muligt uden at træde folk på tæerne, jamen så har man jo ret stor vished for, at aktierne i sidste ende vil have den værdi, man tillægger dem i denne bevægelse, så længe dette altså bare gøres på lødig vis. Og bemærk i øvrigt, at denne bevægelse sådan set bare (hvis man ser bort fra den underlæggende etiske retning, man gerne bør basere bevægelsen på) er en effektiv og alsidig måde at gøre folk i stand til bedre at agere polittiske forbrugere, og generelt set bare gør, at folk kan opnå meget større politisk indflydelse på deres samfund (og altså særligt deres lokale samfundet, om alt går vel) via deres forbrug m.m. 
%Og vigtigt lige at pointere så, at en af de store grundsætninger i dette system, der gør, at det ikke ender ud i en cirkelslutning, er, at man i bevægelsen indgår et fælles løfte om, at når aktier veksles/risiko-neutraliseres, så må man aldrig forsøge at konspirere mod at diskriminere aktier ud fra deres ejere eller ud fra tidspunktet for handlingen. Man bør altså således love hinanden altid at slå ned på sådanne konspirationer, gerne ved at få handlingerne kriminaliseret, eftersom det altså er en stor grundpille for bevægelsen, at bidragsydere ikke behøver at frygte at blive diskrimineret på nogen af disse parametre, samt i øvrigt alle andre uærlige parametre, man kan tænke sig til hen ad vejen (og indgå fælles løfter omkring, når de opdages). 
%Så ja, man kan forme bevægelsen lidt i nogle lag, hvor vi på bunden har et en samfund-etisk retning samt et princip om, at folk gerne skal belønnes ud fra, hvor meget gavn deres handlinger giver til samfundet (hvor der så derfra skal være plads til at indføre flere parametre til at beskrive og vurdere forskellige typer af gavn), i hver fald så længe at samfundet på denne måde er bare nogenlunde liberalistisk (og hvorfor skulle man nogensinde prøve at smide alle liberalistiske elementer ud; det tror jeg sagtens man kan regne med ikke sker i nogen nær fremtid), og at man for alle samfundsmæssige overgange, så længe man holder sig til disse simple grundtanker, altid vil bestræbe sig på at lave bløde overgange med høj respekt for de forige systemer og aftaler gjort heri (hvilket er essentielt for at man kan have aktier, der baserer sin værdi i, at fremttiden, og ikke bare samtiden, vil tage denne værdi seriøst). Så de er ligesom de grundlæggende lag, som jeg forestiller mig det nu. Og så kommer det næste lag, hvor man prøver at implementere disse idéer ved f.eks., som jeg vil foreslå det, at danne et parameterum i fælleskab (som vistnok godt kan opdateres lidt løbende, hvis man bare sørger for, at sådanne opdateringer ikke kommer til at træde nogen specielt over tæerne), der beskriver personers udførte handlinger samt hvilken gavn de drager af andres handlinger (og hvordan de drager gavn, som nævnt; bl.a. om det er på en passiv eller mere aktiv måde), og hvor man så også danner et system for at konverere alle disse handlinger m.m. til aktier. [Indskudt: Nå ja, jeg skal også lige overveje lidt mere, hvordan man helt præcist kan neutralisere risiko i aktierne, men det kan jeg vende tilbage til.] Og når dette er fastlagt, og man har et system, der i princippet kan komme til at bære selv en global økonomi (hvorfor ikke?), så er det så som nævnt, at man kan lægger stat-agtige foreninger m.m. overpå i et trejde lag, om man vil, hvor man eksempelvis kan skrue lidt ned for den måske lidt usikre dynamik i et sådant ellers helt liberalistiske system og således kan vedtage forbund og love til at fastsætte aktiekurserne på anden vis, hvilket jo så meget vel kunne være et demokratisk, stemmeret-baseret system. [Nå ja og lige en indskudt eller afsluttende note til mig selv: Jeg skal også lige overveje lidt mere, det her med om man kan have en slags bagrundsaktier, der kan bruges til at uddele ydelser og/eller opkræve skat..]
%Angående det med baggrundsaktier, hvorfor ikke? Hele princippet er jo at bruge sin handlekraft som forbrugere, hvilken i mine øjne bør være ret ultimativ samlet set, til at fastsætte lønninger og skatter/ydelser, og hvis man har nogle bagrundsaktier, så gør det det jo kun nemmere at få til at fungere; så har man en ventil man kan skrue op og ned for ligesom. Og så handler det bare igen om at deinere rummet, således at så vidt muligt prøver at undgå, at der sker integritetsbrydende diskrimination. Angående den tidslige diskrimination så handler det vel bare om... Tja, det handler vel egentligt bare om ikke at overdrive, hvilket man f.eks. kunne gøre ved pludeselig at gøre bagrunden al for stor, sådan at handlinger ikke betyder så meget... Tja, man det vil man jo heller alrdig have lyst rigtigt, så det går nok alt sammen.. En god idé ville dog nok være, hvis man fra start af... Tja nej, man skal nok bare prøve at sørge for at parameterrummet er bredt nok til at kunne rumme alle de forhold, der kan fås brug for, for så kan deltagerne jo altid bare gå sammen om / vedtage at ignorere visse parametre i aktievurderingerne, hvis de bliver overflødige (frem for at prøve at danne et system til at indstemme forandringer i parameterrummet). Ja, det må vel være det, der er svaret. Og at det så meget vel kan være, at der derved kommer huller, hvormed man kan diskriminere på en måde, så vil man jo ikke gøre dette, for det vil bryde den grundlæggende tankegang bag hele systemet. Så man vil derfor i stedet højst sandsynligt bare lappe disse huller ved f.eks. at lave vedtægter i trejde lag (som der godt kan være flere af; der kan godt være flere stater m.m.), så folk mister muligheden (og fristelsen) til at udnytte dem. Og så angående det med at neutralisere risiko i aktierne:... Ah, det er faktisk trivielt nok: Enhver delgraf af deltagere i systemet kan udligne alle de ting, de skylder hinanden, simpelthen ved at nå frem til en pengefordeling imellem dem, kun regnet ud fra alle tidligere handlinger på kanterne af delgrafen, som altså endnu ikke er blevet risikoneutraliseret endnu, og så sørge for at samle alle de pågældende aktier til én divisibel aktie, som derefter kan fordeles ud i delgrafen til alle deltagerene (knuderne i grafen) alt efter den fordeling, man kom frem til. Og således er al skyld imellem disse brugere nu afviklet, og man behøver aldrig derefter at se på pågældende handlinger igen, når det kommer til kanterne på denne delgraf. Fint. 




%Faktisk brainstorm:
%Har lige ((18.03.21) i dag) tænkt lidt over, hvordan afkastene bliver betalt af, og hvem der gør det. Jeg tænker så, at de kan ske ved specielle firmaer, som brugerne kan melde sig til og give bemyndigelse til at samle overskud fra bidragene, og som så skal stå for at betale afkastene. Hm.. Hvad så med bidrag, der går på tværs af firmaer, og hvordan undgår man for meget inerti i disse firmaer...? Tja, min tanke var jo, at firmaerne alle skal gå over til kun at have deres ledelse belønnet via belønningstokens (således at der faktisk bliver lidt "fremtidsception" over firmaerne), men dette kan vel stadig sagtens korrumperes alligevel..? Men tanken er jo netop så, at man kan trække korrupte handler fra i bidragsregnskabet.. Hm, og fordi firma-protokoller og -strukturer bliver mere formaliserede på det tidspunkt, kan man også gøre brug af et princip om at udskifte alle ansatte, på en måde hvor de nye ansatte er valgt tilfældigt og hvor al overførsel af information bliver overvåget. De kunne i hvert fald være en meget god teknik til at gøre det svært for korrumperende kræfter at få tag i firmaet. Så hvad med de tværgående bidrag (og hvordan måler man hvilke bidrag, der hører til hvor, hvis man kan det i første omgang..?), og hvilke nogle faktiske bestemmelser skal man lave..? Tja, firmaerne skal vel have en måde at gå i forbund på, så de kan være fælles om at stå for at betale afkast.. ..Hm, kunne man gøre noget så simpelt som bare at sige, at belønning for at betale afkast er nye (anderledes, for det meste) tokens, og så kan firmaerne bygges i et lag over denne bestemmelse? Er det nok?.. Hm, nej.. for hvem har så magt til at betale disse afkast i første omgang og hvornår? Det skal vel være sådan, at brugeren kan logge sine tokens på (og af og på) forskellige foreninger/bestemmelser (som kunne formuleres matematisk..)..? Kunne man så ikke bare kræve, at alle betalinger af afkast skal veksles for, at betaler selv får dannet sig en token herom, og muligvis at alle skift af foreninger også bliver påført af en del af tokens værdi, så det på den måde er meget indbygget, at disse skift i sig selv kan bedømmes, positivt og negativt.? Ja, og så kunne man også lige inkludere selve det at modtage afkast som en handling, der også kan både belønnes eller "trækkes fra," for på denne måde kan man give brugere motivation for at logge sig af discontinued afkastbetalings-foreninger/-bestemmelser, når det modsatte ikke er særligt konstruktivt for bevægelsen (hvilket vil sige for vedkomnes medmennsker). Er krig et problem i sig selv, udover at det selvfølgelig kan være en farlig kilde til, at man forlader bevægelsen igen.? Det tror jeg faktisk ikke umiddelbart; bevægelsen skal jo alligevel altid efterlade et rum til, at medlemmer kan handle selvisk, og at sådanne gerninger må bedømmes og straffes på ekstern vis.. Tja, men man kunne dog godt lige spørge sig selv, hvor meget skal man tillade selviske handlinger? Hvad f.eks. hvis bevægelsesforeningerne i lang tid udnytter andre grupper, f.eks. den tredje verden?.. Tjo, men pointen er jo bare, at det så er op til medlemmerne m.m. at bringe disse spørgsmål på bane, så de bliver en officiel del af overvejelserne, og på den måde bliver de så i "trække fra"-sanktionérbar ting, hvis foreningsledelser så vælger at ignorere de pågældende problemer, der bliver lagt for dagen herved. Og hvis man f.eks. så beslutter sig for, at man ikke tør at lave en kovending på verdensordnen og begynde at bruge al sin energi på at løfte den tredje verden (og muligvis betale forlig på gamle udnyttelser), jamen så må det bare være op til eftertiden (den dog semi-nære eftertid, men godt nok med "fremtidsception" indover) at bedømme fornuften af disse beslutninger (selvfølgelig hvor bevægelsens egen tilslutning, og dermed fremtid, så også er indregnet i det). Okay, med det her om, hvordan tokens kan logge sig på forskellige betalings-foreninger/-bestemmelser (hvor selve denne aktivitet også noteres som en del af token-værdien), så tror jeg faktisk, jeg er på et meget godt punkt med idéen.. Nu tror jeg, jeg vil gå mig en middagstur i det her fine vejr, som det er nu, og så tænke lidt mere over det. ... Lidt koldt, men dejligt nok, når solen var fremme. Tænkte lidt over, det med (som jeg har nævnt i papirnoter), om man ikke kan lave en blød overgang, så første firmaer godt kan få løn fra overskudet. Men det er der jo også frihed til nu. Hvordan foreningen gør, når det skal dele overskud ud, er jo bare en del af det, der bedømmes via afkast-betalingstokens.. btw, man skal kunne lave én token for betaling af flere tokens, så token-mængden ikke stiger og stiger i antal.. og så ligger det vel bare i næste lag så og er op til medlemmerne at logge sig på foreninger, der har et godt system til at sikre sig, at tokens'ne bliver betalt lødigt og effektivt.? Men hvordan gør man så, at firmaerne ikke bare render med overskudet, og hvad gør man for at sikre sig, at firmaer kan fordele overskudet imellem sig, hvis f.eks. en forening går ned på tilslutning?... Tja, men det må vel bare være nogen medlemmer og firmaer skal indgå kontrakter om i et overliggende lag, så at medlemmerne kan sikre sig imod at firmaerne lige pludselig løber med byttet. Det kan jo heller ikke rigtigt være anderledes.. Men derfor er de jo stadig ret essentielle, de her kontrakter, især i starten.. Nå ja, men en ret simpel version kan jo så netop være, at firmaet kun må give sig selv løn baseret på visse parametre, og disse parametre kan jo så passende lægge op til, at firmaet alt andet end lige skal betale sine ansatte (inkl. ledere særligt) kun med de tokens, som firmaet selv genererer. Og smart hvis firmaer så også generelt går i forbund med hinanden om, at de hver især skylder deres formue til en fællespulje, sådan at de kun har lov at varetage overskudende, så længe de andre i forbundet siger god for dem. Og alt dette kan altså også bare ske i et lag over det grundlæggende ligesom. Så det grundlæggende handler egentligt bare om at logge (altså nedfælde) alt om tokens og de bidrag, de repræsenterer, samt hvilke betalingsbestemmelser, de tilhører, og hvordan disse tilhørsforhold evt. skifter, samt hvorledes afkastene endeligt betales af på tokens (som så igen fører til nogle nye tokens for hver samling afbetalinger). Og så handler det altså om at formulere, hvad et bidrag er, hvordan deres afkast bør bestemmes (i.e. hvordan de bør værdisættes), samt hvilke gerninger man kan trække fra et bidrag. Angående de to første punkter, tror jeg, det skal lyde, at afkast betales efter et scaling-princip, som jeg mangler at forklare (men begynder nu), ud fra hvordan tilsvarende arbejde/bidrag belønnes ved betalingsdatoen, hvor man så evt. må sammenligne med tilsvarende bidrag i nedskalerede systemer, hvor man så bare simpelthen bør skalere værdien op derfra. Det vil sige at det vurderede lykke-/velfærdsbidrag divideret med værdisættelsen (ganget med den procentdel som afkastet skal reducere token'et med) skal holdes konstant. Hvis de så af en eller anden grund gør, at man kommer i klemme, fordi man kommer til at skylde for meget væk til nogle særligt store bidragere, så må man bare som en nødløsning introducere en skat, som kan implementeres ved at bruge det token-tillæg, der genereres, når et token indløses. Men dette bør vist kun være i nødstilfælde, og om ikke andet kun være noget man gør, hvis der er bred enighed i foreningerne om at en sådan skat er gavnlig. Alt andet end lige er det nok meget fornuftigt --- og nok meget sandsynligt at man kan --- at man bare holder sig til en konstant velfærdsstigning-pr-betalingsværdi-skaleringsfaktor, når man ser på (evt. ned- (eller op-)skaleringer fra nutidige lønninger, når det kommer til noget tilsvarende). Det er så også bl.a. sådan at man bør vurdere ting som risikotagnings-belønninger og, omvendt, privilegie-fradrag osv.: Man skal altid bare se på nogle tilsvarende forhold i nutiden, når det kommer til lønninger, eller rettere forhold for, hvordan det (lødigt) estimeres, at lønningerne ville gives. Hvis man således har en fast protokol man følger for nutidslønninger, og man ved at man ville gå ud fra denne i tilfælde af, at der forekom noget lignende, så kan man godt tillade sig at gå ud fra denne i stedet for ud for faktiske forhold, hvis der ikke lige er nogen faktiske forhold at sammenligne med til pågældende tidspunkt, der minder om dem, der skal betales afkast for. Angående hvornår man kan trække gerninger fra, så tror jeg egentligt man ret let kan udforme et godt princip, som bare handler om at se på... Hm... Hm, kunne man mon også bare sige, at medlemmerne så bare logger sig på hjemler for, hvordan ting kan trækkes fra... Men så kan man ikke motivere brugere væk fra udgående foreninger.. Hm... Tja, kunne man ikke bare sige, at alle gerninger potentielt kan trækkes fra i det grundlæggende lag, men at betalingsforeningen godt må have en hjemmel for, hvordan man under visse omstændigheder kan undlade at trække en gerning fra noget sted fra... Hm, men det er vist heller ikke helt holdbart, for medlemmer kan jo handle med tokens... Så derfor skal man nok bruge ret grundlæggende bestemmelser for, hvornår man må trække gerninger fra tokens. Hm, og det er ikke bare at adskille det, så.. alle gerninger, der ikke er noteret som en del af et bidrag, skal "trækkes fra" fra hos gerningsmanden selv..? Hvis man så overhovedet skal have dette som en del af det her grundlæggende lag... Hm, okay, det er faktisk ikke så lige til, det her.. .. Ah, nu ved jeg det. Andre medlemmer skal bare have mulighed for at tilskrive tillægsgerninger til et token, i den (så offentlige) proces, der skal til for at oprette det token. Dette betyder ikke, at bidragene selv skal være offentlige; de må godt være krypterede (hov nej vent... Nå det kan jeg komme tilbage til) til senere eventuelt, men det gør ikke noget, for andre brugeres beskyldninger behøver så ikke at nødvendigvis at have noget med det tokens bidrag at gøre... Men så kan man jo bare sørge for selv at melde slemme gerninger til små tokens... Okay, der er lige et par ting, jeg stadig skal tænke over... Hm, men angående det sidste, er det så ikke bare at sige, at gerninger godt må optræde på flere tokens, men kun kan trækkes fra (og belønnes i øvrigt) én gang samlet set, men hvor man altså godt så må trække én gerning fra ad flere omgange... Tjo, men hvordan ved en token-ejer så, hvor meget der trækkes fra af lige netop den token? Hm, kan man ikke bare sørge for, at de er ordnede i rækkefølge, og at der så trækkes fra fra første token til sidste. Og hvis hele beløbet så endnu ikke er trukket fra, så kan medlemmer bare melde den samme gerning på nye tokens, hvis/når gerningsmand udsteder disse. Og fordi alle gerninger, gode og dårlige, bliver offentligt gjort inden token'et er færdigudstedt, så kan alle se, hvad der købes (og brugere kan så selv gå sammen om at forsikre hinanden, hvis der opstår meget risiko på enkelte tokens). Det bringer mig så dog tilbage til, hvorvidt alle gerninger skal være erklæret i et ikke-krypteret sprog, men det skal de vel, hvis det skal være et fair marked, og det skal det jo.. Eller også kan man bare sige, at der ikke må handles med en token, før alt indhold er offentliggjort. Ja. Okay, men jeg skal faktisk også lige overveje, om det virkeligt er okay, at man bare kan anmelde alle gerninger; hvad så med gerninger i fortiden? Men kunne dette så ikke bare løses ved, lidt som jeg foreslog til noget andet, at den forening som man udsteder token'et hos, har en hjemmel, der beskriver en protokol (som de så skal håndhæve) for, hvor lang tid tilbage at gerningerne skal have været begået, på det givne tidspunkt, hvor gerningen meldes på en token... Hm, men så kan brugere jo bare melde sig til hjemler, hvor den tid er nul... Ja, så det skal jeg lige tænke over. Og samtidigt, så ville det også være smart, hvis bevægelsen var åben for, at medlemmer godt kunne udstede tokens til folk for velfærdsbidrag før bevægelsens start... Men måske man bare kunne lave en donationsforening herom..? Hm... ...Man kunne måske lave et eller andet med at stemme på hjemler... .. Tja, kunne man ikke bare sige, at man simpelthen bare kan tillægge gerningspåstande til folks tokens, fra og med de bliver medlemmer af foreningen? Og så kan alt uden om det, bare ske i et eksternt lag. Virker umiddelbart som en meget god idé.. 






\subsection{Tilføjende bemærkninger omkring ITP}
(20.05.21) Jeg tror nok, jeg vil lægge lidt vægt på idéen som et alternativ til Git (med store fremtidsmuligheder). Forskellen er så, at man som projektleder uploader spørgsmål og åbne løsninger, hvor folk så kan uploade svar til disse i form af proof scripts. *(Jeg tænker faktisk nærmere bestemt, at folk altid skal uploade både sætninger og de tilhørende proof scripts.) Projektledere kan så med fordel også uploade alle potentielt nyttige værktøjer til at bygge disse proof scripts. Særligt kan man jo uploade et miljø, hvor handlingerne er begrænset i henhold til projektets meta-antagelser. Der skal så i øvrigt også lægges op til, at brugere frit kan danne deres egne rating-/certifikat-systemer (og bør gøre dette). Der bør således også lægges op til, at der benyttes udsagn formuleret i naturlige sprog, og ikke kun i formel logik, når projekterne og det undermoduler skal beskrives. 

Jeg vil så efter ret kort tid bringe min kundedrevne organisation på banen og foreslå, at der oprettes sådan en af disse til at give programmørerne løn. Dette betyder så, at programmørerne skal kunne udlicitere deres bidrag til organisationen for en fastsat tid, hvor indholdet så godt må holdes hemmeligt for den generelle omverden, så er organisationen kan sætte betalingsmure op potentielt set. Jeg vil så starte med at foreslå at bruge en demokratisk, dynamisk model, som jeg har beskrevet nedenfor i undersektionen, \textbf{Økonomien omkring det semantiske web og i det hele taget} (næsten til sidst i sektionen), til lønuddelingen og vil så først senere generalisere til modeller for ledelsesdirektioner generelt. 

En hurtig fordel ved dette nye Git-alternativ kan meget vel være, at man nemt kommer til at kunne give detaljer i tillæg til en funktion om præcis \emph{hvor} gennemtestet (og/eller bevist) den er. 

En anden mulig rimelig hurtig fordel kunne være, at man måske ret hurtigt ville opnå en lagt bedre overskuelighed på platformen, hvor man hurtigt kan hoppe på et projekt og begynde og arbejde, og man let kan få overblik over de samlede problemer, der mangler løsninger på platformen --- og når så folk også kan give løn til hinanden og/eller via et samlet demokratisk system, så bliver allerede der rigtigt godt. Ja, det bør jeg forresten også huske at fokusere på; at et system til at udlove og betale dusører for løsninger også skal med.

På lidt længere sigt, men måske stadig ret kort sigt, bliver en fordel også, at man jo nemmere kommer til at kunne søge effektivt på funktioner og andre løsninger, man har brug for.

På lang sigt er der alt for mange fordele til, at det ikke skulle blive en realitet. I den forbindelse kan jeg også nævne, hvad jeg vist ikke har fået gjort, at fremtidig programmering nok bliver meget matematisk deklaration, hvor man bygger et system op ved at erklære udsagn om det --- og i sidste ende bliver det sådan, at man bare kan tale eller skrive disse udsagn i et almindeligt sprog, nemlig når teknologien omkring det semantiske web bliver veludviklet.

Platformen skal selvfølgelig modereres for ulovligt indhold, men her kan man helt sikkert finde en god løsning, hvor man kan få brugerne til at hjælpe med denne proces (med gulerødder og med pisk (sidstnævnte dog nok mest for synderne, dog muligvis inklusiv brugere, der på en eller anden måde har lovet at agere moderatorer, og som så ikke gør det)). 

Fordelene ved at bruge min type ITP-system til platformen er, at alle formater kan ændres og opdateres løbende, uden besvær, og også at serverne kan adoptere brugernes effektive algoritmer med tiden (en proces man dog skal være lidt omhyggelig med, men hvor meget stadig med tiden kan automatiseres ved denne proces (om altså at godkende algoritmer)). Det bliver nok også smart det her med, at man kan guide folk, der skal arbejde på ens projekt, ved at danne et miljø (hvilket på sigt bliver nemt at gøre), hvor programmøren er begrænset til actions, der ikke bryder restriktionerne sat i dette miljø.

Jeg har også nogle gode idéer, synes jeg selv umiddelbart, omkring matematiske ITP-miljøer, inklusiv en ret basal idé om en læringsorienteret matematik-puslespilsspil, hvor spilleren blandt andet løser ligninger ved at tage actions, hvor mængden af mulige actions så primært er begrænset til gyldige operationer (så spilleren bare får teknikkerne i hånden, også selv uden at have lært logikken bag dem nødvendigvis først (og særligt uden at have set beviser for dem osv.)). Men alle sådan nogle idéer og tanker tror jeg altså, må komme i anden række, således at jeg starter med bare at fokusere på programmeringsområdet (og altså også med særligt fokus på internetplatforme omkring det). 

Åh, og noget helt andet er forresten, at man faktisk kommer til at skulle passe ret meget med certifikat-antagelserne, som jeg umiddelbart kan se, for hvordan sikre men sig lige, at de skal have været der fysisk i mappen?\ldots\ Hm, man kunne jo eventuelt lave en mappe, hvor man har en funktion i IL-sproget til at skrive til denne mappe, men ikke har nogen antagelser om denne funktionalitet andet end symbolske antagelser (som man altså ikke kan bruge i et bevis, men kun allerhøjst til (symbolsk) dokumentation). Og så sørger man bare for at ingen andre IL-funktioner kan skrive til denne mappe, men samtidigt at man godt kan læse fra mappen. Og så er det, at man ville kunne oprette sine certifikat-antagelser ved at referere til, hvad der kan findes i den mappe. Ja, dette kunne jo være én løsning i hvert fald. *(Man kunne også antage, at der er et orakel-filter (som der så ikke rent faktisk er i virkeligheden), der tjekker alle certifikater, der gemmes til mappen, og sortere alle ikke-autentiske certifikater fra. Forskellen er så, at man hermed ville kunne beskrive semantikken af programmer lidt grundigere, hvis man antager autenticiteten af eventuelle nye certifikater, men hvor det stadig ville være sikret, at man ikke kan konkludere, at alt hvad der sendes til mappen også bliver gemt i og kan læses fra mappen.)


(23.05.21) Jeg tror faktisk også, det bliver særligt vigtigt det her med, at man kan springe bevisskridt over, for dette er jo særligt brugbart for online dokumenter (kode- og/eller tekst-), hvor der er mange læsere, der selv lige kan give forskellige point-ratings, når de har læst et givent udsnit. Og her er det jo så nærmest helt essentielt at bruge mit system med ``metaantagelser,'' fordi folk jo så kan danne deres egne acceptance levels til at benytte netop dette. Så jeg bør derfor altså fokusere særligt meget på denne del af det. Det er virkeligt skønt, for jeg har ellers gået og følt her i går og i forgårs, at jeg lidt manglede sådan en ting, der forbandt mine tekniske idéer omkring ITP-systemer med idéen om matematiske/semantiske online-platforme/-fællesskabe. Jeg glemte nemlig lidt, at min idé fører mere med sig end bare potentiel effektivitet og gode muligheder for at transformere imellem forskellige formater og sprogsyntakser (og logiske fundamenter); det gør også, at bliver i stand til at lave (mange) abstrakte antagelser undervejs, som man alligevel kan bevise mere og mere på sigt (enten helt logisk eller bare med større og større sandsynlighed), hvorved de så (nemt) kan erstattes i de originale beviser med mere grundlæggende antagelser.

Dette vil så sige, at man faktisk bør tillade selv naturlig tekst som en del af beviser.(!) Det er vist først her i går, at jeg har indset dette. For når idéen nu allerede fra start bør orienteres meget mod det online, så kan man også fra start lægge op til, at bruge hinanden til at tjekke beviser, og hvorfor så ikke bare gå all the way og lægge op til, at folk skal kunne lave bevisskridt, der som udgangspunkt kun kan tjekkes af mennesker, fordi det benytter ikke-formaliseret tekst. Med andre ord behøver vi altså dermed ikke vente på, at vi får bygget en formalisering af de naturlige sprog, før det kan bruges i beviser. Så kan man bare sørge for i stedet, at der allerede tidligt blandt de point, som folk benytter til at rate bevisskridt, findes en pointparameter for, hvor entydig teksten er. Andre brugbare pointparametre kunne så i øvrigt være for korrekthed (hvor sikker pointgiveren er på, at udsagnet er korrekt), læsbarhed, sømmelighed, hvor detaljeret teksten er ift.\ indholdet kontra hvor overordnet, hvor dækkende teksten er for de overordnede punkter, den bør formidle, hvor hurtig den er at læse, sværhedsgraden ift.\ tekniske begreber, humor (hvilket selvfølgelig ikke nødvendigvis er at foretrække), disposition, osv. Man kunne sikkert let finde på flere, eller måske endda finde bedre alternativer til disse, men det var også bare lige for at nævne nogle muligheder. De vigtigste point af disse i praksis vil i starten dog nok være korrekthed og entydighed. Det gode er så også ved alt dette, at så kan denne teknologi så også bruges i ligeså høj grad på dokumentationen som på beviserne for dokumentationen, og man kan derfra også nemt tage det endnu videre, for alle former for tekster (i hvert fald mange tekster men støder på på nettet, som sagtens kan tåle at blive redigeret meget) kunne sådan set have godt af, at forfatterne kan bruge et online fælleskab til at hjælpe med at redigere teksten, så tingene fremstår så tydeligt som muligt --- og så fejltagelser bliver rettet. Som eksempel kunne dette også være diverse videnskabelige artikler og fagtekster, ikke bare inden for datalogi og matematik, men også inden for alle mulige andre fagområder. 

Og når vi snakker tekster, så bør der være tale om hypertekster; der bør inkluderes links til forskellige referencer og/eller videre forklaringer, og det bør også meget gerne være sådan, at hver lille sætning, og term for den sags skyld, skal kunne gives en underlæggende forklarende tekst, som læseren kan klikke på, og som så foldes ud. Her bruger man jo ofte udelukkende bare hyperlinks, men jeg mener altså, det ville være bedre, hvis vi også blev bedre til i stedet at bruge fold-ud-fodnoter (som så eventuelt kan inkludere hyperlinks til dybere forklaringer, men det bliver rart for læseren, hvis denne i høj grad kan blive på samme side, men stadig have adgang til alle mulige små forklarende tekster omkring udsnit, sætninger og termer i hovedteksten, ved at disse bare kan foldes ind og ud og/eller vises i marginen). 

Når nu alle disse dokumenter redigeres løbende online, så kan overgangen også blive ret glidende, når de formelle sprog begynder at blive udviklet, for så kan brugere bare stille og roligt begynde at erstatte tekst rundt omkring, med tilsvarende formelle tekster. Det kan så være, at der bliver en vis periode i overgangen, hvor der bliver en tendens til at simplificere teksterne, hvis man gerne vil have dem formelt formuleret, men ikke hvor det formelle sprog endnu ikke helt er nuanceret nok til at dække mange vendinger. Men hvis man så bare beholder den originale tekst i baggrunden og ikke sletter den helt, så kan man jo så bare løbende gå mere og mere tilbage til nuancerede vendinger, når teknologien tillader dette. Så ville man i øvrigt også kunne gøre det sådan, at læseren bare kan toggle mellem det formelle og det mere nuancerede tekst i denne overgangsperiode (men hvor den formelle tekst så får lov til at definere semantikken, og hvor den nuancerede tekst så bare midlertidigt kun er til pynt). 

Og en anden idé/indsigt, jeg også fik i går aftes/nat (d. (23.05.21) i dag), er, at man godt kunne lade server-algoritmerne være kundestyret via en demokratisk model, hvor brugere får stemmer alt efter deres bidrag til omsætningen --- og muligvis også andre kvaliteter omkring brugerens aktivitet i praksis, men det kan man så lade de betalende kunder om at fastsætte et system, hvor visse andre brugere får større medbestemmelsesmagt. (For man kan jo sagtens indstemme en model, som selv indeholder demokratisk betingede elementer i sig, og hvor der altså så kræves yderligere afstemninger.) Så forskellen er altså, i sammenligning med min anden idé om kundedrevne virksomheder beskrevet nedenfor, at her er det ikke ledelsen generelt men kun server-algoritmerne, der bliver bestemt af kunderne. Dette gør idéen meget simplere at forklare og nemmere at sælge, og uden at det leder tankerne over på en hel masse andet --- og potentielt vækker en endnu muligvis uhensigtsmæssig opsigt. Idéen bruges bare her som en ret nem måde at sikre sig, at server-vedligeholdelsesarbejdet og arbejdet med at udvikle og vedligeholde den digitale platform er i gode hænder (nemlig i høj grad i brugernes egne). Hermed bør server-virksomheds-iværksætterne nemlig bare kunne nøjes med det mere simple arbejde, at definere den demokratiske proces, og så ellers bare bruge tiden mest på at læse og overveje brugernes forslag og gennemgå diverse foreslåede rammer for algoritmers gyldighedsbeviser og sørge for ikke at implementere noget usikkert. I starten kan man selvfølgelig også selv være på banen som iværksætter med forskellige forslag til alle mulige ting, men dette er så bare for at sætte skub i den udvikling, der alligevel nok skal komme med tiden. (Og hvis man så sørger for ikke at være for grådig med ens egen løn, så skal brugerne jo nok vælge at forblive som kunder hos server-virksomheden, selv i det eventuelle tilfælde af, at tingene går lidt langsommere end forventet. Og her kan man jo i øvrigt bare fra starten lade sine egen løn som iværksætter være afhængig af udviklingen (og altså særligt af indtjenesten)). Nå ja, man skal dog også som iværksætter arbejde på at opsætte et godt donationssystem, så brugere (\ldots har jeg egentligt nævnt dette? Tja, måske, måske ikke, men så kommer det i hvert fald her uanset hvad:) kan udlove dusører for løsninger på diverse problemer --- gerne også hvor der kan være mulighed for, at denne dusør kan indløses automatisk, hvis løsningen opfylder visse kriterier (eventuelt ved at have visse certifikat-stempler, men måske også bare ved at være logisk gyldig). Brugere skal også gerne kunne gå sammen om at udlove dusører i fællesskab ved at oprette en pulje, hvor dusøren så kun træder i kraft, hvis mange nok går med til at betale til den pulje (så man ligesom kan motivere hinanden til at donere). Angående motivationen så kan der jo også blive ekstra motivationer i at donere på denne platform, fordi dette så kan få ens brugerprofil til at stige i graderne, og få mere at skulle have sagt (plus donationer giver et klart bevis på, at man har stake i platformen, og dermed er det mere sandsynligt, at man rater ting på lødig vis og ikke bare troller (men det er op til brugerne selv at forme disse brugervurderingssystemer)). *(Hov, men det behøver man forresten ikke at lægge specielt vægt på; at kunderne kan finde på at gå til en konkurrent, og at man ikke behøver at være for grådig. Det korte af det lange er jo bare, at man kan give kunderne mere bestemmelsesmagt til fordel for begge parter, og så behøver den ikke at være længere end det umiddelbart (i salgspitchet).)

Lad mig også lige nævne for mig selv (for jeg har vist lidt haft glemt dette), at min ITP-idé ikke er særligt triviel, og der er faktisk en del tekniske detaljer omkring det (bl.a.\ at der skal bruges refleksionsmodeller (med tilhørende refleksionsregler)) og gode indsigter, så der er altså også kødt teknisk kød på selve idéen alligevel. *Orakel-mappen og orakel-funktionerne er også vigtige idéer, samt at programmerne bør defineres som generelt ikke-deterministiske. Derudover er det vigtigt det her med at have det som en del af meta-antagelserne, at man altid skal gemme et proof script sammen med sine viste sætniger, i stedet for at commit'te sig til et fast proof( script)-format. Det er ligeledes også smart, at man ikke skal commit'te sig til et noget fast sætningsbiblioteksformat. 
%Værd lige at huske på; den er vist ikke helt triviel --- og hvis man kan følge den, så kan man måske allerede se noget smart i den der.



(25.05.21) Angående at sælge/udbrede idéen så er jeg lidt kommet frem til, at man nok kunne fokusere på, hvordan en ny form for stackoverflow (/mathoverflow etc.), som så i øvrigt også kan fungere som en ny form for Git, hvor brugere kan vurdere særligt entydighed og korrekthed af argumenter og udsagn, og på den måde vurdere korrektheden af teksters konklusioner i sidste ende, muligvis kan hjælpe til at sætte godt skub i udviklingen med kurs mod en ny verden af matematisk programmering (og hvad ellers dette må medføre). Jeg kunne så motivere idéen ved at påpege de mulige fordele ved matematisk programmering i fremtiden, fremføre idéen om en mere argumentorienteret stackoverflow-side (men hvor brugerne i sidste ende selv skaber deres egne rating-systemer), og i sidste ende kunne jeg så foreslå min ITP-idé, hvilken jo så bliver ret vigtig (som jeg kan se) for resten af idéen, fordi theorem proving jo så bliver særligt vigtig på sådan en side, så det vil være fuldstændigt oplagt med et ITP-system, som så kan benytte samme teknologi til at opnå muligheder for hurtig og ret automatiseret opdatering. Og effektiviteten af ITP-systemet vil så også blive ret vigtig, når folk så kommer til at bruge det så meget, og dette er jo virkeligt punktet, hvor man kan sige, at min idé må excellere. (Og argumentet er altså: Man vil jo gerne kunne vise korrektheden af effektive programmer på sigt, så hvorfor ikke åbne op for, at disse programmer også potentielt kan bruges til ITP-formål?) Og da hele pointen er at arbejde med pointsystemer til at godkende argumenter, selvom et logisk bevis ikke lige er til stede, så hvorfor ikke åbne op for, at ITP'en kan gøre det samme? Andet ville da ikke rigtigt give mening?\ldots\ Nej, ikke rigtigt; det ville være en underlig begrænsning at give sig selv, især hvis man ved, at andet fint kan lade sig gøre. Allerede så langt så godt. I øvrigt vil det så også blive fordelagtigt, som jeg forudsiger det, at prøve i fællesskabet af formalisere de argumentationsformer så meget meget som muligt, så folk kan gentage kendte argumentationsformer uden at korrektheden skal rate'es helt forfra fra bunden af hver gang, men hvor man kan stole helt eller delvist på korrektheden af denne type argumentation. Og dette kræver altså, at der er en høj grad af logiske muligheder bag systemet, så man er i stand til at definere og genkende disse argumentationsformer. Så dette gør det ekstra vigtigt at logikken omkring at analysere argumenterne bygger på et grundlag så kraftfuldt som formel matematik. Og da brugerne så selv skal kunne bygge og analysere disse værktøjer i deres egen ende, er det altså vigtigt at bygge det hele på formel matematik, hvilket så også vil sige at sørge for, at brugerne kan have en ITP, som selv er i stand til at\ldots\ Hm, giver dette egentligt mening som argument? Jeg lader det lige stå, fordi pointen om at arbejde på at forsøge at formalisere argumentationsformer (svarende til bevis-metoder) jo er vigtig nok, men er dette dog også et argument for lige min ITP-idé?\ldots\ Tja, muligvis ikke, men det er dog et godt argument, hvorfor man helt sikkert bør forsøge at bygge platformen på et fundament af matematik, også selv hvis man ikke tror på den vilde fremtid omkring matematisk programmering, men alligevel tror at en mere argumentorienteret stackoverflow-platform (med et godt rating-system) kunne være gavnligt. 

Jeg bør i øvrigt også huske at nævne det med at projektleder kan guide (ved at begrænse) programmørernes handlinger, hvilket jo også kan blive rigtigt smart. Det er bl.a.\ dette, der gør idéen relateret til Git også, samt at man i det hele taget får større mulighed for at outsource problemer, og selvfølgelig det at opdateringsprocessen så bliver nemmere og mere automatisk --- og man kan nemmere holde gang i mange versioner af projektet på én gang, således at brugerne selv kan vælge, hvad der passer bedst til dem. Men anyway, det førstenævnte relaterer sig så nemlig også særligt til min ITP-idé, fordi det netop benytter et filsystem, hvor man skal bevise korrektheden af nye metoder i henhold til et sæt (meta-)antagelser, før man kan bruge disse metoder. 

Hm, og når nu idéen muligvis skal sælges med meget fokus på internetfællesskabet/-platformen, så ville det jo være rart at kunne vise en speciel idé i den forbindelse, men det kan jeg jo også allerede, hvis jeg tillader mig selv at begynde at snakke lade de mest betalende kunder bestemme algoritmerne (mest) ud fra at vægtet demokratisk system. Og det tror jeg, jeg vil. Jeg kom så også lige på (og det er stadig d. (25.05.21) i dag), at man måske også kunne nævne idéen om også at skabe en organisation, der kan sætte betalingsmure op, og som så også lytter mest til de mest betalende kunder i den forbindelse. Jeg behøver så ikke nødvendigvis at folde hele idéen ud, men kan muligvis bare nævne det, at man kunne danne en organisation, som giver belønninger til de programmører (og lad mig endelig bare holde det til det), der sender løsninger og indhold, inklusiv meningsbidrag og vurderinger, ind, og hvor man så som noget særligt ydermere markedsfører sig på, at kunderne (alt efter hvor meget de betaler) får indflydelse på, hvordan denne belønning skal fordeles. Og så behøver jeg vist ikke være mere specifik; jeg behøver f.eks.\ ikke at fastslå, at kunderne faktisk bør gives den primære magt herved, også over investorerne. At dette er bedst sådan, kan jeg altid nævne senere --- inklusiv de andre ting, jeg har at sige om det. Nice! (25.05.21)


Nå ja, og dertil kommer jo også, at hele det med at have et projekt, der kan opdatere sig selv automatisk, lægger kraftigt op til, at man bruger meta-antagelser og certifikater. 

%Kunne jeg nævne at programmering også bør gøres mere grafisk? (Og ITP'er bør også.) 
%Skal jeg nævne det med at ting bliver lettere at opdatere, eller er det ikke bare en af de mange ret trivielle ting, som matematisk programmering vil føre med sig?..



\subsubsection{Tekniske tilføjelser}
Kopieret fra lige ovenover:\\
{\slshape
	Åh, og noget helt andet er forresten, at man faktisk kommer til at skulle passe ret meget med certifikat-antagelserne, som jeg umiddelbart kan se, for hvordan sikre men sig lige, at de skal have været der fysisk i mappen?\ldots\ Hm, man kunne jo eventuelt lave en mappe, hvor man har en funktion i IL-sproget til at skrive til denne mappe, men ikke har nogen antagelser om denne funktionalitet andet end symbolske antagelser (som man altså ikke kan bruge i et bevis, men kun allerhøjst til (symbolsk) dokumentation). Og så sørger man bare for at ingen andre IL-funktioner kan skrive til denne mappe, men samtidigt at man godt kan læse fra mappen. Og så er det, at man ville kunne oprette sine certifikat-antagelser ved at referere til, hvad der kan findes i den mappe. Ja, dette kunne jo være én løsning i hvert fald. *(Man kunne også antage, at der er et orakel-filter (som der så ikke rent faktisk er i virkeligheden), der tjekker alle certifikater, der gemmes til mappen, og sortere alle ikke-autentiske certifikater fra. Forskellen er så, at man hermed ville kunne beskrive semantikken af programmer lidt grundigere, hvis man antager autenticiteten af eventuelle nye certifikater, men hvor det stadig ville være sikret, at man ikke kan konkludere, at alt hvad der sendes til mappen også bliver gemt i og kan læses fra mappen.)
}

Programmer skal i øvrigt have lov til at tilføje meta-antagelser til listen, hvis det kan vises, at de ikke bryder de tidligere antagelser på listen. Meta-antagelser kan fjernes igen, men kun hvis man også fjerner alle programmer, der\ldots\ er initieret siden\ldots\ Hm\ldots? Ah ja, man kan godt fjerne meta-antagelser og programmer ret frit, for man skal bare sikre sig, når man tilføjer en ny meta-antagelse, at selve mappe-systemet samt de initierede programmer lever op til denne antagelse inden da. Man skal altså bl.a.\ vise, at de eksisterende programmers samlede invariant medfører, at der ikke kan ske brud på den nye invariant, som den tilføjede meta-antagelse vil medføre.

Til certifikat-antagelserne kan man jo starte med at definere en række konstante men ukendte propositioner i sin start-ITP. Og så skal brugeren derfra have lov til at antage, at en endnu ubrugt propositionskonstant af disse, er ækvivalent med et udsagn, som brugeren vægler, som så dog kun må medføre restriktioner på orakel-mappen, som jeg kalder den for nu, som heller ikke strider imod tidligere restriktioner. Med ``orakel-mappen'' mener jeg den mappe, hvor det ikke antages, at hvad der skrives til mappen også bliver gemt i mappen (og man kan så forestille sig, at der er et orakel-filter, der på en måde sorterer alle de kun-hypotetiske filer fra, og kun tillader de filer, som brugeren også rent faktisk kommer til at gemme (men dette er bare én måde at tolke det på; vi kan også bare nøjes med den førstnævnte fortolkning om, at oraklet bare er et ukendt filter)) og er synligt, når man prøver at læse fra mappen bagefter. Så orakel-mappe-antagelserne fungerer altså meget ligesom de generelle meta-antagelser, bortset fra at de umiddelbart også får en kontant proposition tilknyttes sig, således at alle sætninger vist ud fra orakel-mappe-antagelserne får en ukendt proposition som antecedent. Det næste er dog så, at brugeren får mulighed for at\ldots\ Nå nej, never mind alt det med kontante propositioner. Det er nemlig bedre, at brugeren bare selv (i princippet) antager, hvilke nogle antecedenter, der eventuelt kan komme på de sætninger, som læses fra filerne (som så altså i reglen vil have certifikater tilknyttet sig) i orakel-mappen. Hm, men man kunne måske godt bruge en liste af propositioner og prædikater, hvor brugeren kan justere deres sandhedsværdi alt efter, hvilke nogle antagelser brugeren vil bruge, hvis man så bare sørger for aldrig at bestemme denne værdi eksplicit, men bare tillader tilføjelsen af regler, der kan skære antecedenterne fra sætninger (hvor der så lige eventuelt kan spørges om samtykke først)\ldots? Hm, men måske er det lidt meget at gøre bare for en så lille funktionalitet\ldots\ Hm, ja og nej: Hvis der spørges om samtykke, så behøver man ikke løbende at justere sandheden på propositionerne/prædikaterne\ldots\ tjo, eller man kunne så bare stille på samtykke-restriktionerne. Ja, fint. Så man kunne altså give orakel-mappe-antagelserne mulighed for at indeholde samtykke-prompt-prædikatsymboler, som ikke officielt antages at være sande (og heller ikke kan antages hverken sande eller falske), men som kan skæres fra, hvis de står som antecedenter, dog kun hvis brugeren har opnået visse privilegier til dette. Brugerne kan så selv definere systemets privilegie-parameter-sæt og kan definere (alt sammen dog kun med administrator-privilegier), hvilke privilegie-prompt-prædikatsymboler kræver hvilke privilegier, hvilket så også kan afhænge af prædikatsymbolernes input. 

Skal programmer og/eller funktioner også kunne kræve privilegier, og hvordan kan man så sørge for dette? Hm, kunne det ikke bare være at definere endnu et sæt af prædikatsymboler, som ikke kan skæres fra via de samme (samtykke-prompt-)regler, men som tages højde for i reglerne, der initierer nye programmer eller funktioner (og man kunne måske bruge hvert sit sæt til hver af disse), således at passende privilegierestriktioner bliver tilføjet til funktionerne/programmerne, hvis visse prædikat-symboler står som antecedenter. Jeg forestiller mig, at disse privilegierestriktioner så bare afhænger af, hvad var antaget for privilige-prædikatsymbolerne på tidspunktet, hvor programmet/funktionen blev initieret. 

Angående funktioner, hvordan kommer de så til at spille ind? Handler det ikke bare om, at man har funktionsbiblioteker, som programmerne kan bruge, og hvor der så i forbindelse med privilegier kan indsættes et privilegietjek (muligvis med en prompt) i begyndelsen af funktion, hvis funktionen korrektheden vistes med et privilegie-prædikatsymbol som antecedent? Hm, skal privilegie-prædikatsymbolerne antages at returnere sandt, når nye meta-antagelser skal tjekkes, om de allerede gælder, inden de kan antages? Det skal de vel så\ldots\ Ja. Hm, men kan man ikke implementere alt det her på en måde inde i systemet uden at skulle tilføje denne udenom-funktionalitet? 

Indskudt: Det skal i øvrigt være sådan, at programmer aldrig må bryde invarianten fra meta-antagelserne, heller ikke under kørslen. Så man må altså ikke skrive en eneste ugyldig karakter til en fil, også selvom man retter fejlen lige bagefter. Denne restriktion er dog ikke absolut, men stadig sikker meget klog at have i starten. Man kan nok med fordel implementere den bare ved at antage, at programmer (som jo antages at være ikke-deterministiske, hvilket jeg har været inde på tidligere) kan stoppe til hver en tid.

Tilbage til ``kan man ikke implementere alt det her på en måde inde i systemet uden at skulle tilføje denne udenom-funktionalitet?'' Ah, jo man kunne jo bare tilføje nogle privilegie-prompt-funktioner til IL-sproget, som så har udefineret (ikke-deterministisk) (orakel-)adfærd, og så kan man derved opnå det samme i praksis. Nemt nok. 

Og tilbage til spørgsmålet om funktioner og funktionsbiblioteker, så jo, men alt dette kommer jo bare til at være en del af meta-antagelserne omkring program-initiering. 


Der er lidt et spørgsmål om, hvordan man skal forhindre concurrency-fejl, hvis der køres flere programmer på én gang, men dette må man også bare finde ud af efter hånden. Man kunne starte med at sige, at main-programmet (som er en slags launcher), skal holde øje med, hvilke programmer er i gang på hvilke filer/mapper, og at det så ikke tillades at køre to programmer samtidigt, hvis der er et overlap mellem deres filer/mapper. Tja, eller helt i starten kunne man jo bare sige, at der kun kan køres ét program ad gangen. 


(25.05.21) Der bør faktisk være to former for meta-antagelser: Et sæt af antagelser som både kan antages af ITP'en, når den f.eks. skal udtrække sætninger fra filer og som alle nye programmer og meta-antagelser skal overholde, og et sæt hvor det sidstnævnte kun gælder, men hvor ITP'en altså ikke kan antage dem selv. På den måde kan man bedre sikre, at eventuelle integritetsbrud på meta-antagelserne ikke med det samme kan føre til falske sætninger. Det bør så til gengæld næsten være en ret grundlæggende ting, at man kan initialisere særlige programmer til at verificere at meta-antagelserne gælder for mappen, i hvert fald dem der kan tjekkes\ldots\ Hm\ldots\ Ah, jeg har en rettelse/tilføjelse til min idé: Når programmer initieres skal de vises at overholde hver meta-antagelses invariant enkeltvis og ikke kun vises at overholde den samlede invariant. Så hvis der sker brud på én meta-antagelse, så kan man stadig vide med sikkerhed, at programmet ikke vil skabe brud på flere. Dette er (nok) ret vigtigt. Ja, og så ville man så derfra nå langt med at have de to forskellige sæt af antagelser, så ITP'en kun kan antage visse særlige meta-antagelser selv. Og hvis man så derfra vil gøre systemet mere sikkert, så vil det helt sikkert være en god idé at benytte privilegie-orakel-funktionerne (de ikke-deterministiske funktioner, der tjekker om programmet har eller kan gives (ved at prompte brugeren) visse privilegier) og sørge for at alle de basale sætninger, som man har tillid til, og som skal bruges ret ofte, bliver transformeret ind i en fil/en mappe (gerne skrivebeskyttet ellers), muligvis med et grundlæggende certifikat, som kan tjekkes i starten af hver session *(men kun hvis man vil være ekstra sikker, og så skal dette tjek altså stadig ske automatisk og uden at skulle prompte brugeren for særlige privilegier). Så skal der altså spørges om privilegier, når man tilføjer nye sætninger til dette grundlæggende bibliotek, men ellers er idéen så, at der ikke ellers skal specielle privilegier til at at bruge sætninger fra dette bibliotek. For sætninger man har en anelse mindre tillid til, kan man så have dem i biblioteksfiler/-mapper, hvor der skal gives specielle privilegier, før man kan bruge dem. Pointen i relation til meta-antagelserne er så, at man på den måde også kan forhindre, at ITP-systemets programmer kommer til at lede i en ikke 100 \% sikker mappe, og kommer til at finde en modstrid med en meta-antagelse, fordi man så netop bare kan inkludere en privilegie-orakelfunktion i denne antagelse (så at modstriden på den måde formelt set kommer til at afhænge af, hvad denne orakelfunktionen returnerer (hvilket man aldrig kan bevise noget om, for funktionen lever kun i programmet og bliver altså kun skabt, idet man initierer programmet)). Cool, det konkludere næsten denne tankerække; jeg skal bare lige færdiggøre tanken om programmer til at tjekke (og eventuelt rette i øvrigt) meta-antagelserne. Jeg tror det vil være en god idé med sådan nogle programmer som noget grundlæggende også, som man så kan initiere meget ligesom de andre programmer, men som godt må stå et separat sted. Et meta-antagelses-verifikationsprogram skal så vises at det kan finde frem til eventuelle brud på meta-antagelserne (hvorved det så kan være arbitrært verbose omkring, hvor og hvad problemet ligger i) i tillæg til de normale restriktioner som alle programmer skal opfylde (om ikke selv at ødelægge meta-antagelserne), og hvis verifikationsprogrammet også skal have lov til at prøve at rette fejlene, så skal det så også vises, at programmet kun ændrer filsystemet, hvis og kun hvis meta-antagelserne så vil være gældende igen efter ændringen.



\subsubsection{Min idé er opfundet: Milawa theorem prover gør allerede, hvad jeg har tænkt omkring en ITP, der kan bevise korrektheden af og skifte til nye versioner af sig selv (01.06.21)\label{efter_milawa}}

(01.06.21) Nu har jeg, som overskriften antyder altså set, at der allerede er tænkt mange tanker i den retning, som jeg havde forestillet mig. Da jeg begyndte at samle op på min ITP-idé her i starten af året, var jeg jo også egentligt nået frem til, at der er mange (i mange tilfælde lige gode) måder at lave sin ITP på, men at jeg måske kunne bidrage med noget på den front, der handler om, hvordan man kunne formalisere ellers ret uformelle bevisskridt ved brug af certifikater osv. Men så fik jeg jo dag den idé omkring, hvordan man kunne have en meta-antagelse-liste som noget grundlæggende for ITP'en, og på den måde slippe helt kravet om at skulle vægle en konvention til ITP-systemet og muligvis gøre det hele lidt mere frit på den måde, samt ikke mindst få en ITP, der altså ligger godt op til hurtigt at komme i gang med at bruge certifikater osv. Og ja, måske er der stadig en særlig idé i det her med at bruge meta-antagelser (også særligt så man kan gøre mere brug af filer, både til certifikater osv., men også til at gemme sætninger og proof scripts i forskellige (brugerdefinerede) formater). Jeg har endnu ikke gennemgået det så grundigt, men det virker umiddelbart som om, man hidtil har gået uden om programsemantik, der kan manipulere filer. Og mine tanker om ``orakel-mappen'' og orakel-funktioner kan også sagtens være lidt nyskabende, så vidt jeg ved. Men jeg tror dog stadig, hvad angår min strategi for at komme ud med mine idéer, at jeg alligevel nok så skal prøve at fokusere mere på de andre ting omkring idéen, bl.a.\ mine idéer til en ny open source bølge (hvor brugerne kommer til at bestemme mere og mere (bl.a.\ over (be)lønningerne)), og selvfølgelig det om at danne et fælleskab, hvor man kan verificere sætninger, bl.a. om programkorrekthed, på en mere blød måde sammen, som dog med tiden kan transformeres til noget mere og mere stringent (på en ret kontinuer måde). Og i den forbindelse tror jeg så virkeligt, at man skal lægge vægt på det bløde i starten, og slet ikke fokusere så meget på, om maskiner kan eftertjekke det, men bare på at danne gode pointsystemer og gode serveralgoritmer så folk bedre kan få vist lige de ontologi-elementer, der passer til deres behov (hvor vi så i starten taler meget om ontologier over tekniske udsagn og argumenter for samme (inklusiv udsagn og argumenter om programmer og deres korrekthed)). 


Jeg ved ikke helt hvorfor, men jeg føler lige at jeg skal understrege følgende, så det gør jeg lige her: De privilegie-givende orakel-funktioner bruges til særligt at give sikre imod, at man bruger usikre sætninger til at initiere programmer og/eller køre programmer på særlige måder, medmindre brugeren har givet eksplicit ordre med forhøjede rettigheder om dette. Når man så har muligvis ``usikre sætninger,'' så er der altså tale om sætninger, der har specielle brugerdefinerede antecedenter foran sig (det jeg har haft kaldt `fodnote-antecedenter' i mine noter før), som har optaget i sig, hvor mange muligvis ``usikre handlinger'' basalt set, der har været gjort for at opnå sætningen. Dette inkluderer selvfølgelig også de handlinger, hvor brugeren skal spørges om tilladelse først. Selvom mere detaljerede ``fodnote-antecedenter,'' hvis vi forsat lige kan kalde dem det her, på en måde altid er bedre, så kan brugere dog godt vælge at implementere måder at skære detaljer fra, hvis fjernelsen af dem kan ses ikke at komme til at bidrage særlig meget til usikkerhedsheden omkring nogen af de sætninger, der vises af dem efterfølgende, og hvis man dermed føler at det er det værd, at skære dem fra for overskuelighedens skyld (hvorved man dog også altid bare kan transformere beviset tilbage, så de ikke skæres fra, hvis man får brug for dem igen). 


Angående om der så er kød nok på idéen til at komme ud med den, så er jeg lidt kommet frem til (eller næsten; skal lige tænke lidt mere over det), at ja, jeg må give det et skud. I den forbindelse er det i øvrigt værd at nævne, som jeg lige kom i tanke om nu her, at det jo er rigtigt smart at iværksætterne af en på sigt kundedrevet virksomhed starter med at bestemme lønningerne i høj grad, for så kan man dermed give høje lønninger (og særligt inklusiv lønninger i form af juridisk bindende løfter om en del af det fremtidige overskud (som \emph{altid} skal være afhængig af størrelsen af overskuddet, altså procentdelen skal --- og dette gælder hver eneste gang man snakker afkast i form af dele af det samlede overskud; selv de første iværksættere må heller ikke bare sætte en procentsats, som de så ejer; det skal også afhænge af, hvor stor virksomheden vokser sig)) til folk, der er med til at udvikle hele det tekniske grundlag for virksomheden. De første iværksættere kan dermed i princippet bare nøjes med at få styr på alle de nødvendige kontrakter osv, og skal derudover bare have en god forståelse af mulighederne, så de kan fordele lønnen fordelagtigt i starten og herved tiltrække så mange gode programør-iværksættere som muligt, men behøver altså ikke nødvendigvis at kunne løse alle de tekniske problemer til at søsætte virksomheden. Det er dog klart, at jo mindre man gør som iværksætter, uanset i hvilken række man er, så kan man ikke forvente afkast, der er alt for grådigt sat, og jo mindre arbejdsbyrde man selv har, jo mindre kan man slippe afsted med, for hvis brugerne først begynder at vurdere en for grådig, så vil de jo bare tage sig besværet og iværksætte en ny virksomhed selv ud fra samme principper. (Men her skal det så tilføjes, at hvis brugerne er for grådige herved, og lader de oprindelige iværksættere i stikken uden særlig grund, så bliver det muligvis svært for dem, at tiltrække programører nok til at få en god kodebase, der kan opretholdes IP-rettigheder for (især fordi de så kommer til at skulle kopiere en stor del af den originale virksomheds idéer), og i forhold til, hvad der er at sparre ved dette, så skal iværksætterne nok være ret grådige, hvis dette skal kunne betale sig for brugerne.)

Hm, man burde da så også sørge for at udstede aktier med tilsvarende magt til at bestemme lønninger i starten og med overskudsafhængige retter til fremtidigt afkast på sig, så man dermed også hurtigt kommer til rent faktisk at kunne give løn\ldots\ Ja, og hvor hver en aktie så bare også kan indeholde en vis stemmeretsmagt ift.\ at bestemme, hvornår der kan udstede flere aktier (med rettigheder om yderligere afkast, der jo så tilgengæld vil trækkes fra det samlede fremtidige overskud). (Disse tanker hører egentligt til i en følgende sektion, men det går nok.)

Men ja, mine idéer er så i denne forbindelse, omkring en stackoverflow-agtig platform, hvis struktur er mere brugerstyret, særligt pointstrukturen, hvor det så er meningen, at sætninger og tekster i sidste ende gives point særligt for korrekthed, entydighed, forståelighed, samt hvor dækkende teksten er for emnet. En videre idé er så, at lade brugerne bestemme lønfordelingen og brugeralgoritmerne mere og mere, og alt efter hvor meget de bidrager til omsætningen, og jeg har så lige nogle idéer til nogle ITP-løsninger i brugerens ende, så programmerne på stackoverflow-siden let kan tages i brug af brugerne, hvis argumenterne holder, endda på en muligvis ret automatiseret måde. Ja, er det ikke cirka de store træk i det? Og er det ikke så kød nok her, til at det giver mening at prøve at komme ud med den idé? \ldots\ Jeg har i øvrigt min parti-idé, som man måske også kunne komme ud med på egen hånd\ldots\ Hm, men det ville nu være rart, hvis man lige kunne finde en lidt mere konventionel udgave af min ny-open-source-bølge-idé, som dog stadig har kød nok på sig\ldots\ Eller også skal jeg bare satse på at kunne finde de rigtige kontakter\ldots\ Tja, altså der er jo idéen om bare at lade brugerne styre algoritmerne (og så altså måske uden at nævne det med at styre lønningerne --- og i hvert fald uden at nævne, at kunderne/brugerne så også bør (kræve at) tage aktierne og styringen over med tiden)\ldots\ Ja, det lyder faktisk som en virkeligt god idé i hvert fald sørge for ikke at nævne det med kundeovertagelsen (hvad jeg også har tænkt før), og så kunne man måske egentligt godt nævne noget om, at der kan være punkter, hvor brugerne også kan ende med at få medbestemmelse på (freelance-)programmørers løn for bidrag, så disse kan blive tiltrukket mere til at bidrage uden først at skulle lave lønforhandlinger\ldots\ Ja, sgu. Ja, så jeg starter bare med en version af idéen, hvor brugerne (for vi snakker også kun digitale produkter i denne version) bare for lov at bestemme ret meget, og altså ud fra hvor meget de har bidraget til omsætningen i den seneste periode af en vis længde (eller med en aftagende vægtning jo længere tid, der er gået siden hvert køb/abonnementbetaling). 

*Nå nej, det kunne måske være en god idé at have det med fra starten af om at udstede og/eller shorte aktier til de første bidragsydere\ldots\ Hm, på den anden side kan idéen, hvis man ikke tager den del med, der kræver betalingsmure nødvendigvis, jo så tilgengæld bruges som et forslag til eksisterende virksomheder/organisationer også (f.eks.\ Stack Overflow og Git og hvad har vi)\ldots\ Ja, så det er måske faktisk en meget god idé (altså i første omgang at udelade alt det, der lidt kræver, at man starter et helt nyt firma fra bunden)\ldots\ Ja, det kan være min første version. *Nej, jeg vil faktisk nævne de fleste af mine idéer, men så bare i bunden af skitse-papiret til første version af min idé. Jeg vil bare ikke nævne noget om, at man som investor/iværksætter ligefrem lover fremtidige aktier ud på forhånd, således at deres jævnlige afkast ender med at gå mod nul. Jeg vil nævne, at investorerne eventuelt kan lokke brugere og bidragsydere til ved at opsætte et system, hvor disse belønnes (bagud), og eventuelt med aktier, og hvor betalende brugere så kan få medbestemmelse her, men jeg vil altså ikke lægge op til andet, end at investorerne altid bare bevarer alle kortene på hånden (og altid kun aktivt handler aktier væk eller udsteder flere, hvis dette vurderes nyttigt ift.\ at tiltrække flere brugere). \ldots Hm, skal jeg i øvrigt nævne noget om at bruge en dynamisk model med ``stemmekraftfelter?'' Ja, det kan jeg sagtens. Jeg kan så bare nævne det allerede i forbindelse med at brugere kan få lov at bestemme server-algoritmer (hvilket jeg altså kan nævne tidligere i papiret). (02.06.21)

Okay, så nu skriver jeg lige dette notesæt færdigt, og så går jeg ellers i gang med at skrive en lidt mere sammenhængende skitse af den omtalte version af mine idé (til en ny open souce-bølge (med mere formel programmering og testning og med større brugerbestemmelse, og som kan være startskuddet på det semantiske web m.m.)), som jeg kan dele ud af til dem, jeg prøver at forklare om idéen. Umiddelbart efter kan jeg så lige gå i gang med at skrive diverse tilføjelser til idéen samt øvrige idéer, der kunne være interessante så småt at udbrede mig om derefter. (02.06.21)


(03.06.21) Nu har jeg så umiddelbart også en værdifuld KV-idé, nemlig idéen om at lave en investeringsfond i form af et KV-system (se nedenfor til sidst i \textbf{*Yderligere tanker omkring kryptovaluta}%\textbf{Noter omkring muligheder for det fremtidige marked generelt}
-sektionen (ja, entropien i dette notesæt er steget drastisk på det sidste, når det kommer til strukturen, men det sker jo)) med et demokratisk system bag sig, som skal stadig følge principperne beskrevet i min KV-sektion ovenfor, men som også godt kan bruge stemme-kraftfelt-modeller til diverse lidt mindre mindre beslutninger, imellem at der så indimellem stemmes om systemændrende forslag. Jeg tror lige, jeg vil læse kritikpunkterne af PoS-kæder, så jeg lige kan se, om der er noget jeg skal tilføje, men ellers tror jeg umiddelbart faktisk, at jeg har fat i en ret vigtig idé her. Så kan jeg også skrive en lille idé-skitse over denne, og så se, om jeg kan få nogen KV-interesserede med på den. 

*(15.06.21) Som jeg også har indsat i omtalte \textbf{*Yderligere tanker omkring kryptovaluta}-sektion nedenfor: Jeg har lige indset at min idé om at oprette en investeringsfond på baggrund af en kryptovaluta ikke afviger fra at oprette et firma, bare hvor kryptomønterne erstattes med aktier. x) 

*(23.06.21) Jeg tror i det hele taget ikke længere rigtigt på vigtigheden i mine KV-tanker (udover nogle af det mange idéer de har ført med sig selvfølgelig).

Uh, og jeg kan i øvrigt også bare lave en idé-skitse over min QED-teori til at starte med. Jeg behøver jo ikke at udgive en artikel, pre-print eller ej, inden jeg ser, om der er interesse for den; jeg kan godt se om der er interesse først, før jeg udarbejder en hel artikel. :) 

Udover at jeg lige skal læse lidt om status om og kritik af PoS-kæder, så skal jeg også lige huske, at jeg faktisk har noget, jeg nok skal overveje angående den semantiske stackoverflow-side. Jeg skal nemlig lige overveje, hvordan de første server-algoritmer (osv.) kunne struktureres. Det må gerne være et ret åbent system i sig selv allerede, så brugerne ret hurtigt kan gå i gang med at lave folksonomies over forskellige indholdsobjekt-typer og rating-point (hvor både andre brugere og indholdsobjekter kan rates, samt i øvrigt også selve folksonomy-termerne). \ldots Hm, men det må vel så bare være et database-system, hvor der er brugere, indholdsobjekter, typer og point, og hvor brugere så både kan uploade nye objekter, typer og point, samt også ratings af enten brugere, objekter eller typer. Point rates så på en måde i brugeren egen ende, ved at man query'er serverne om en del-ontologi, der giver flest mulige samlede point ud fra brugerens point-vægtning (som altså defineres ud fra et regnestykke, som brugeren opgiver til serveren). Nå ja, og så skal man så også kunne konstruere et helt træ ud fra pointene\ldots\ Ah, men så må vi bare også have objekt-relationer (/-kanter) som noget, brugerne også både kan uploade og rate. Ja, det må næsten bare være noget i den stil, og så skal man bare lige finde frem til en algoritme, der kan sammensætte delontologier ud fra sådan et point-regnestykke. Alt dette skal dog ideelt set stadig bare være oven på et mere åbent lag, hvor brugerne simpelthen selv kan bestemme server-algoritmerne i sidste ende. Det gør dog ikke noget, hvis dette `underliggende lag' først bliver ordentligt implementeret senere; det kan godt lade sig gøre i den rækkefølge. (03.06.21)
\ldots\ Hm, og måske kan man endda bare bruge eksisterende teknologi for relationelle databaser til at servere brugeren alle de objekter, der til sammen kan bringe pointene over en vis (efterspurgt) grænse (hvorefter man så kan udvælge den eller dem med højest pointtal fra denne gruppe som næste led, inden brugeren får den/dem vist); det må man da næsten kunne\ldots\ Det kan jeg jo lige læse på. (03.06.21)


%Jeg må ikke tilføje flere subsubsections her.






\subsection{Øvrige tanker om det semantisk web og den tilhørende teknologi samt generelle tanker og idéer til økonomiske og forretningsmæssige emner}
%Jeg forklarer bare om alle tingene for sig, og hvis noget af det så er brugbart i andre sammenhænge, så må jeg jo bare skille det ad til den lejlighed. 

I denne sektion vil jeg komme ind på nogle flere ting, der for det meste bygger videre på ovenstående sektioner om det semantiske web osv. Mange af disse idéer har med økonomiske og virksomheds- og forretningsmæssige emner at gøre, så jeg vil derfor også bare i det hele taget lade denne sektion inkludere diverse idéer og tanker, jeg har inden for disse emner generelt (altså også dem der ikke handler om det semantiske web). Da det har relevans for resten, vil jeg dog også lige starte med at uddybe nogle ting omkring modelleringsmulighederne med ontologier i det semantiske/matematiske web/net.

\subsubsection{Prædiktive ontologier}
Jeg vil her forklare mere om, hvorfor jeg tror det at have åbne ontologier, som alle kan tilføje bidrag til (hvor hver enkelte brugere så i princippet kan vælge deres egne filtre og algoritmer til at processere denne data), virkeligt bliver en vigtig ting i fremtiden. Der er mange grunde til dette, og jeg vil også komme med nogle flere eksempler senere i denne sektion på nogle muligvis smarte ontologier, men noget jeg ikke rigtigt har talt så meget om indtil videre i disse noter, er hvad jeg går og kalder \emph{prædiktive ontologier} for mig selv. Jeg synes, dette er et meget fint navn, så det vil jeg bare blive ved med at benytte her. Indtil videre har jeg nemlig mest bare snakket om sprog-ontologier/meta-ontologier, som danner ramme for en verden/model, man gerne vil beskrive, og det er indlysende, at man så også skal kunne opbygge specifikke ontologier ud fra disse. Et fint eksempel på en ontologi kunne være en slags leksikon-ontologi, som beskriver, hvad vi der rent faktisk forekommer og gælder om vores verden --- altså det der ikke hersker nogen (reel) uenighed omkring. Men hvad så med spørgsmål, der kan herske uenighed omkring? Jo, her kunne man så lave en prædiktiv ontologi, altså en ontologi, hvor alle spørgsmål man stiller til ontologien, ikke bare resulterer i et enkelt svar, men i flere (mindst to for ja/nej-spørgsmål), der hver især har en sandsynlighedsparameter tilknyttet dem. Hvem bestemmer så, hvordan disse sandsynligheder beregnes? Jamen det gør hver enkelt bruger i princippet selv; de kan helt selv vælge, hvilken metode, de benytter, og kan selv justere og ombygge på eksisterende metoder, hvis de vil. Hver enkelte bruger har jo sine egne browser-antagelser og -definitioner, som de bruger til at definere, hvordan browseren skal hente information og opbygge en informationsstruktur i brugerens ende. Så forskellen er derfor bare, at brugeren så for prædiktive ontologier definerer ontologien, så hvert udsagn (eller måske bare på nær alle helt trivielle udsagn (så som ``en normal krage har vinger'' eller udsagn i den stil), hvis det giver mere mening), der opbygger ontologien, får en sandsynlighed tilknyttet sig. I praksis vil det så betyde, at antagelser altid formuleres som, ``de udsagn, der har det og det certifikat (og altså kommer fra den og den kilde), er sande med den og den sandsynlighed,'' i stedet for at de formuleres med bare ``kan antages at gælde / at være sande.'' Okay, og hvad er så den store fidus ved dette? Jo, jeg tror virkeligt at folk med disse p-ontologier, som vi kan forkorte dem til her, vil få øjnene op for, hvor langt man kan komme med sådanne avancerede formelle analyser (som altså er gjort via modeller, folk i fælleskab kan blive ved og ved med at udbygge). Herved kan man nemlig blive ved med at analysere hvert eneste lille argument og prøve at finde frem til, hvad det videre bygger på. En diskussion kan så ses som et træ, hvor man kan blive ved med at uddybe hvert enkelte blad, indtil man til sidst når ned til de aksiomatiske grundantagelser i modellen (p-modellen/p-ontologien; ja, jeg vil mikse forskellige termer for det samme lidt sammen her, hvad jeg også tit i det hele taget gør). Men, kunne man så spørge, er det ikke et alt for stort arbejde bare for at opklare en enkelt diskussion; vil det ikke blive for meget arbejde i praksis? Jo, måske hvis man kun ser på den enkelte diskussion, men pointen er så, at diskussionerne ikke er målene i sig selv. Det kan sagtens være, at der går lang lang tid, før man kommer frem til svar på den måde. Men når så folk arbejder på forskellige diskussioner/spørgsmål, så vil det klarlægge flere og flere huller i p-modellen, hvilket så kan give inspiration til, hvordan man kan udbygge den endnu mere. Så målet i starten med at kortlægge diskussioner, vil altså ikke være så meget for at nå frem til et svar, i hvert fald ikke nødvendigvis inden for en nært forestående fremtid, men vil mere være for at få gode udgangspunkter til at videreudvikle p-ontologierne i fællesskabet. Og jeg tror altså så, at vi med denne udvikling kommer til at nå helt og aldeles nye højder, når det kommer til at analysere emner og forhold, som vil være langt langt væk fra, hvad vi kan på nuværende tidspunkt. *(Nu kom det alligevel lidt til at lyde som om, p-ontologierne ikke bliver nyttige på nær i en senere fremtid, og at teknologien så kræver, at man gør en masse arbejde uden nogle kortsigtede gevinster. Dette tror jeg ikke er tilfældet. For p-ontologierne kan stadig bruges til hver en tid, og selv for de mere komplekse spørgsmål, der er svære at svare på --- eller små spørgsmål, hvor der endnu ikke har været nok brugere, der har arbejdet på emnet --- vil man altid kunne opnå et svar i form af en sandsynlighed (så snart ontologien bare er kompleks nok til at kunne formulere spørgsmålet). Så gælder det bare, at denne sandsynlighed ikke vil være tæt på enten 100 eller 0 \% for alle spørgsmål til at starte med, men derfor kan det stadig være brugbart at vide, at ens antagelser fører til f.eks.\ 60 \% på, at udsagnet er sandt, eller noget i den stil.)

De grundlæggende aksiomer/antagelser i sådanne modeller, vil så handle om, hvordan man kan analysere data, samt selvfølgelig de matematiske sætninger, så som Bayes' sætning, man kan bruge til dette. Byggeklodserne til diverse p-ontologier vil så i første omgang være de begrebsopbyggende prædikater og relationer, så man danner sig en konceptuel ontologi, og ovenpå dette vil der så selvfølgelig være al den løbende data, man benytter til at besvare spørgsmål med. 
%Men dette er ikke det hele. Jeg mener, at det vil være en rigtig klog idé, at gøre sådanne p-ontologi-modeller åbne over for, at brugere nemt kan komme med bud på nye begreber, som så særligt kan være diverse, potentielt ret abstrakte, koncepter, som kan bruges til at identificere (selv ret abstrakte) tendenser i den verden, som p-ontologien modellerer. Dette kunne f.eks.\ være et koncept om at...
Og i sidste ende er der så de antagelser, som brugerne i princippet selv sætter, og som altså giver alle sandsynlighedsbestemmelserne. Alle disse tre dele af strukturen er selvfølgelig noget, der kan blive ved med at udvikle sig mere og mere, og hvor det derfor også er essentielt at benytte gode brugerklassifikations-ontologier og tilhørende filtre, så ens browser-applikation har en chance for at følge med og opdatere ens p-ontologi, så den kan følge udviklingen. Også selve den konceptuele del af ontologien kan altså blive ved med at udvikles. Folk kan nemlig blive ved med at tilføje begreber, så flere og flere, selv rigtigt abstrakte, koncepter kan benyttes. Pointen jeg vil komme med her, er så, at tilføjelsen af flere og flere brugbare begreber ikke bare gør, at man bedre kan formulere spørgsmål om den verden, p-ontologien modellerer, men også faktisk kan gøre den statistiske analyse mere effektiv. Et udmærket eksempel kunne være `confirmation bias;' et ret abstrakt begreb (i denne sammenhæng), der dog er nødvendigt at forstå og benytte, hvis man skal blive ved med at søge at forbedre sin prædiktive ontologi. Det kunne f.eks.\ være det faktum, at journalister der arbejder for etablerede virksomheder i et vist land, nok vil være mere tilbøjelige til at sætte historier i et positivt lys ift.\ den etablerede del af regeringen samt landets udenrigspolitiske ageren i modsætning til andre lande, og særligt rivaliserende lande. Herved har vi altså en korrelation mellem at tilhøre en vis gruppe mennesker og så mene og/eller rapportere ting på en farvet måde. Man kunne også finde tusindvis af andre eksempler, hvor det at tilhøre en vis gruppe mennesker gør dig farvet i dine holdninger. Confirmation bias spiller så ind, fordi dette kan være med til at forstærke denne splittelse, men der kan også være mange andre ting, der gør at folk har tendens til at udvikle visse synspunkter og holdninger\ldots\ Ja, nu hvor jeg tænker over det, er `confirmation bias' faktisk ikke et så interessant begreb i denne sammenhæng, for det fortæller bare, at der er en vis tendens, og ikke så meget hvorfor. Man kunne sikkert i princippet dissekere begrebet endnu mere og prøve at finde ind til de forskellige bevæggrunde og psykologiske aspekter, folk kan have der giver denne tendens. Men ja, ellers kan vi snakke om, at folk måske gerne vil passe ind i en gruppe og at de ikke selv så har til at have en anden mening end gruppen, fordi dette kunne medføre, at man mister anerkendelse. Dette forhold gælder \emph{bestemt} også i høj grad i intellektuelle kredse og i højklassekulturer såvel som lavklassekulturer. Dertil kan der være alle mulige politiske grunde i gruppen til at have en vis holdning, enten fordi forholdet påvirker gruppemedlemmerne personligt, eller fordi man gerne vil adskille sig fra visse andre grupper. Hvis en p-ontologi så skal være effektiv, så nytter det ikke noget, at man overser alle disse muligheder for korrelationer og bare naivt antager, og hver enkeltperson er kommet frem til sine meninger i et vakuum af indflydelser, og at man derfor bare kan tage et gennemsnit af folks meninger, hvis man vil prøve at nærme sig sandheden; det kan man efter min mening \emph{ofte} slet ikke. Det gode er så, at alle disse forhold, der kan give anledninger til meningskorrelationer osv., selv kan formuleres som koncepter i ontologien, og herved kan man således modellere dem som en kraft, der indvirker på folks holdninger. Og selvom man så ikke kan svare på i starten, hvor stærk denne kraft (eller denne tendens) er, så vil dette ligesom alt andet blive mere og mere klart, jo mere data man får. Et rigtigt vigtigt arbejde omkring p-ontologierne bliver så at identificere sådanne korrelationstendenser i den virkelige verden og modellere dem som koncepter i ontologien. 

Okay, men hvordan skal man så vælge hvilke koncepter skal indgå i ens sæt af aksiomatiske udsagn, der får tildelt prior-sandsynlighederne (dvs.\ de sandsynligheder, som alle andre skal regnes ud fra). Tja, her er målet jo selvfølgelig at finde et sæt, der båder spænder hele ontologien konceptuelt og som (helst) ikke har nogen indbyrdes korrelationer (udsagnene skal altså nok gerne være mest muligt uafhængige af hinanden for udregningens skyld), man kan tænke sig, og derudover vil det jo også sikkert være meget godt, hvis man kan finde et sæt af udsang, som kan tildeles prior-sandsynligheder tæt på enten 0 eller 100 \% --- måske fordi dette så også kunne gøre risikoen for en overset korrelation imellem to grundudsagn mindre betydende (det har jeg ikke lige regnet på, men det lyder umiddelbart meget rigtigt\ldots), men ellers også bare fordi man så ikke behøver at bekymre sig så meget om sin egen usikkerhed på den relevante prior-sandsynlighed (ja, vi taler om en usikkerhed på en usikkerhed; hvem siger, at man vil give nøjagtigt den samme prior-sandsynlighed, hvis man blev spurgt på forskellige tidspunkter af ens liv?). 
Selve ontologien som folk udvikler i fælleskab skal så gerne ellers være ret uafhængig af, hvilket grundudsagnssæt folk vælger, således at folk så vidt muligt er helt frie til at ændre dette sæt løbende. 
Og dette kan altså lade sig gøre, fordi alle de ontologi-opbyggende udsagn selv kan betragtes som en slags meningsudsagn fra de brugere eller instanser, de kommer fra. Så en del af grundudsagnssættet bliver altså også at acceptance level-landskab for, hvilke ontologi-opbyggende udsagn skal tages med eller ej (alt efter hvilke certifikater osv., de har med i ryggen). 
%Journalister fra et land har tendens... Og andet eksempel: journalister har tendens til at mene det samme... Vi har altså korrelationer... Men åh-åhh, her har vi jo også sikkert en korrelation; en korrelation på korrelationer... Så det er et indviklet billede, men matematikken will show us the way... Og det handler så bl.a. om at prøve at finde ind de mest særskilte (ikke-korrelerede) koncepter, som så også er relativt lette at konkludere noget omkring (og som altså stadig medfører alle de andre koncepter, man har defineret)..
%Korrelationer; fjende, men på en måde også en ven, når man kan finde frem til dem.. 
%Man behøver aldrig at skære andre modeller helt fra; der kan jo komme et paradigmeskift-punkt.

Hvis forskellige befolkningsgrupper så ender med at bruge lidt forskellige ontologier og således ikke når til enighed om samme udsagn, så er dette fint. Så længe folk bare er nogenlunde lødige med deres prior-sandsynligheder og ikke skærer afkaster udsagn 100 \%, fordi de ikke lige passer med deres (nuværende) verdensbillede, så vil data jo i sidste ende, på trods af ellers ret hårde prior-sandsynligheder imod det, ende med at pege på, hvad der er det mest rigtige. Det kan således aldrig betale sig for nogen, hvis denne er lødig, at skære andre befolkningsgruppers alternative udgaver helt fra fra sin egen. Man må altid bare antage, at der er en vis lille sandsynlighed for, at de har ret i, hvad end de går og argumenterer for. For selvom man kunne tænke sig, at det muligvis ville blive upraktisk at holde sin ontologi så åben, at alle i princippet kan få indflydelse på den, så behøver dette ikke i praksis rigtigt at skære ned på effektiviteten af at bruge en p-ontologi. For serverene vil nemlig med fordel kunne geares til forskellige typer grundantagelsessæt, så de hurtigt kan servere svar til brugere, der benytter pågældende ontologi-variant, og så kan brugerne bare sørge for at benytte sådanne servere. Disse servere behøver så kun at holde øje en gang imellem med, om nu der skulle være fremkommet data, der gør at de bør begynde at inkludere mere data fra nogen af de andre ontologi-varianter, som de ellers ikke har haft brug for før, hvis de fortsat skal give de bedst mulige svar til pågældende ontologi-variant. Med andre ord bør folk ikke behøve at frygte, at der sker en masse arbejde på dele af den samlede ontologi-data, som kun har meget lille indflydelse på brugeren egne sandsynligheder, for brugeren kan altid bare bede serveren og/eller browseren om til sammen at bruge algoritmer, der primært ser bort fra dette arbejde, og kun en gang imellem afsøger det, for at se om det kan have betydning for brugeren alligevel. 

Og ja, når det kommer til server-algoritmerne, så bør disse jo også bare i sidste ende blive noget som brugerne selv bestemmer. Det skal så gerne blive sådan at brugere på den måde ikke bare bestemmer over deres browseres algoritmer til at finde svar på spørgsmål, men også på dedikerede servere, der kan hjælpe dette arbejde. Heldigvis tror jeg jo også i denne forbindelse, at fremtiden netop vil byde på højere grad af bruger-bestemmelse over server algoritmer, og hvis dette bliver tilfældet, får man således mulighed for at opnå nogle ret effektive midler til at besvare alverdens p-ontologi-spørgsmål i det brede fællesskab.


% P-ontologier, og her (ligesom med certifikater generelt) er det vigtigt at bruge intuition i stor stil. Således bør man også være ihærdig med at prøve at foreslå forskellige pointsystem til diverse ting, som så kan evalueres og omformes til andre systemer efterfølgende (eksempel: Denne tekst har en høj grad af at være forklarende (osv.)).. (tjek; dette får jeg med, når jeg beskriver idéen som et stackoverflow-alternativ)
% - "Bekæmp korrelationer." (Og prøv at danne koncepter, når korrelationer findes (eller forudses)). (tjek, selvom man måske godt kunne understrege vigtigheden noget mere)
% - At prædiktive ontologier aldrig behøver at sortere data helt fra; der kan jo komme et paradigmeskift, når nok data lige pludselig peger i en anden retning. (tjek)







\subsubsection{Økonomien omkring det semantiske web og i det hele taget}
%Jeg så begynde på emnet om økonomien omkring det semantiske web ret udefra og således starte med at beskrive nogle betragtninger om økonomiske systemer generelt. 
I takt med at det semantiske web udvikles, så bliver det også mere og mere vigtigt at sørge for, at alle bidragsyderne til det semantiske web kan belønnes på en god måde, der hjælper med at tiltrække så mange gode bidrag som muligt og kan sætte skub i udviklingen. Hvis nu det semantiske web var et privat foretagende ligesom YouTube, som et godt eksempel, så kunne man jo bare i ledelsen lægge en strategi, der giver belønninger til skaberne/bidragsyderne på en måde, som tiltrækker flest mulige bidrag og maksimerer virksomhedens profit. 
Problemet med dette billede er bare lidt, at en virksomhed, hvor ledelsen er styret primært af et mål om at maksimere profit, ikke altid resulterer den virksomhed, der yder det bedste for deres brugere og bidragsydere (her er YouTube vist også gået hen og blevet et meget godt eksempel bl.a.\ med dets overdrevne brug af reklamer); når målet primært er at skabe gavn for investorene, og målet om at skabe gavn for brugerne og bidragsyderne (hvilket også inkluderer arbejdere selvfølgelig) kun er afledt af dette, jamen så kan den gavn, brugere/arbejdere har, indimellem forsømmes til fordel for investorernes (nemlig hver eneste gang, der er en konflikt imellem de to). Så selvom man helt klart kunne forestille sig nogle gode virksomheder, der tager teten ift.\ at sætte skub i og ride med udviklingen af det semantiske web, hvilket bestemt ville være dejligt, især hvis alternativerne ikke duer, så ville det alligevel være federe, hvis vi på en eller anden måde kunne tænke os til et alternativ, hvor det primære mål for diverse ledende kræfter (omkring det semantiske web og nærliggende teknologier (såvel  for den sags skyld for andre områder, hvor det giver mening)) i langt højere grad er foreneligt med målet om at skabe mest mulig gavn for deltagerne, nemlig brugerne og arbejderne/skaberne (i.e.\ bidragsyderne). Samtidigt vil det semantiske web også bygges på virkeligt mange forskelligartede bidrag, så der kommer også til at ligge en udfordring i at blive enige om, hvordan man skal belønne disse forskelligartede bidrag i hvad end form for organisation, man nu prøver at udforme til at skabe profit på udviklingen og til at belønne bidragsydere retfærdigt. 

Heldigvis er der en løsning på dette, der er ret ligetil, måske endda i sådan en grad, at den vil indfinde sig nærmest lidt af sig selv med tiden (altså uden behov for at planlægge forud i tid (dermed ikke sagt at planlægning forud ikke kan være meget gavnligt)). For hvis der bliver nogle forhold i et fremtidigt (nært) system, hvor der vil være tydelige tendenser til at tilgodese de få til fordel for de mange i sådan en grad, at uligevægten ikke kan forsvares (i en grundig analyse) med, at den til gengæld tilvejebringer en hurtigere og bedre teknologisk udvikling, jamen så vil man netop i langt højere grad kunne analysere disse forhold i fællesskab via alle de udviklede prædiktive ontologier/modeller og med hele teknologien bag disse. Og ikke nok med dette, man vil også, mener jeg, med den samme udvikling opnå langt bedre muligheder for at indgå nye aftaler i de brede fællesskab. Jeg forestiller mig nemlig, at de nye decentraliserede digitale løsninger, der vil komme med det semantiske web osv., også vil bane vej for, at vi, især hvis nødvendigheden virkeligt opstår, får udviklet systemer til at forslå og stemme om nye aftaler om forskellige ting, hvor hver bruger anonymt kan tilkendegive (til den samlede statistik) til resten af den relevante gruppe, hvor de står henne, når det kommer til et vist spørgsmål, og altså særligt hvor villige de vil være til at deltage i nye aftaler, hvis stemningen generelt er for dem. Jeg forudsiger altså, at det vil blive ekstremt let, når først et issue (et\ldots\ problematisk forhold) er blevet identificeret af p-ontologi-fællesskabet, og diverse løsningsmuligheder er blevet kortlagt, så også at nå til enighed i de relevante grupper om, hvilken løsning man vil arbejde sammen om (om ikke andet så for alle på nær nogle enkelte, som slet ikke vil gå med til den givne indstemte løsning, men dem må man jo så bare klare sig uden, hvis løsningen kan bære dette afsavn) at gennemføre. Så lige for at præcisere så forstiller jeg mig altså anonyme (helt eller delvist) digitale afstemningsformularer, hvor folk i første omgang kan afgive stemmepoint til de forskellige løsninger, samt tilkendegive, hvor villige de vil være til at acceptere løsninger, de ikke stemte for, hvis alle andre (generelt) gør det. Og sådanne afstemninger vil således bare kunne fremkomme når som helst efter behov. Så snart der findes frem til et problematisk forhold, kan individer og/eller grupper gå sammen om at kortlægge løsningsstrategier og så videreformidle disse til det resten af det relevante fællesskab, hvorefter det selvfølgelig kan vendes en gang mere eller to, og så kan man ellers bare initiere afstemningen med det samme. Det vil således blive det letteste i verden (det er jo så meget sagt, men nu siger jeg det bare sådan alligevel), at forenes i menneskegrupper (befolkningsgrupper, faggrupper\ldots\ you name it) med fællesinteresser om at tage aktion for at løse et forhold. Jeg mener, at der er et utal af forhold i vores nuværende verden, som ville være på en hel anderledes måde (og altså i sagens natur for langt det meste på en bedre måde), hvis folk udnyttede deres forhandlingsposition i diverse mulige grupperinger mere optimalt (som udgangspunkt i forhold til, men selvfølgelig ikke nødvendigvis begrænset til, deres egeninteresse). I vores nuværende samfund er dette nemlig ikke rigtigt muligt i samme grad. Det er nemlig for det første sjældent i nærheden af at være muligt for den enkelte, at analysere situationen grundigt nok til at finde de mulige løsningsstrategier, ift. et spørgsmål/forhold, hvor vedkommende føler sig uretfærdigt stillet i samfundet (eller er det, men bare ikke er klar over denne uretfærdighed; det kan også ske). Selv når folk går sammen i grupper for at analysere problemer, er den nuværende teknologi, hvor folk enten mødes og tager en verbal diskussion (hvilket er en \emph{ekstremt} lavteknologisk analyseform efter min mening) eller tager en diskussion over en mail- eller forumtråd (hvilket også bare er horribelt). Og fra den enkeltes synspunkt er der tit bare ikke tid og overskud til at igangsætte og/eller deltage i sådanne analyser, så det bliver ikke gjort. Så folk accepterer som regel bare nogenlunde situationen som den er, og tilslutter allerhøjest en eller anden forening med nogle vage, slogan-agtige mål, som kan tolkes på alle mulige måder, og som regel uden nogen foranstaltninger til at sikre sig imod korruption, og generelt uden nogen særligt veludviklet strategi for, hvordan man bør arbejde sammen om at opnå de mål. Og ja, selv hvis man som enkeltperson skulle snuble over en god løsning, eller bare en god analyse af situationen, så er man bare én stemme iblandt alt for mange i vores nutidige verden, fordi vi jo ikke har nogen gode teknologier til at analysere problemer, hvor de gode løsninger og indsigter kan flyde tilvejrs og blive centrale i videre diskussioner/analyser. Selv diskussionsfora, hvor folk kan give point til gode kommentarer, fungerer ikke som analyse værktøj; det eneste der sker i disse er, at de allerede populære meninger flyder tilvejrs, og derfra kommer der ingen dybere analyse. Jeg har set, at der er en side, der hedder Kialo, som går ud på at skabe et kort over argumenter til diverse diskussioner. Dette er en rigtig god start; at udarbejde en argument-graf er en vigtig proces som led i en analyse. Men så mangler vi altså bare den efterfølgende analyse, hvor man for hvert argument-grafblad (grafen behøver ikke være et `træ,' men mon ikke man stadig kan kalde ikke-forbindende knuder for blade?) prøver at finde frem til, hvilken form for empiri, der kan besvare sådanne grundlæggende spørgsmål (og hvor man selvfølgelig, jævnfør p-ontologier, også bør gøre så hver brugere selv kan justere de grundlæggende prior-antagelser, der skaber grund for analysen (som jo så vil bruge Bayes' sætning, når vi når ned til dette punkt i analysen)), og ud fra dette så i sidste ende regner sig frem til (hele eller delvise) svar (et ``svar'' vil jo sjældent bare være ja/nej eller lignende absolutter, men vil være et slags sandsynlighedsrum, hvor man så bare kan håbe på at opnå, at sandsynlighederne kommer til at fordele sig med stor overvægt på én specifik type svar). Men ja, jeg tror altså at alt dette vil ændre sig, når vi først får udviklet teknologien omkring p-ontologier tilstrækkeligt. Herefter vil folk nok nærmest altid kunne finde frem til løsninger, der kan sikre dem retfærdig løn og retfærdig behandling osv. Og hvis vi så går tilbage til spørgsmålet om, hvorvidt bidragsydere til det semantiske web kan få en god og retfærdig løn for deres bidrag, jamen så er svaret også det samme; det vil man sagtens kunne finde måder at sikre sig til den tid.

Det næste man så kunne spørge om, er, hvad så med før den tid? Man må jo forudsige, at bidragene til det semantiske web og til teknologierne omkring det, navnlig teknologier omkring selve p-ontologierne, det er klart, vil være essentielle til overhovedet at nå til omtalte punkt, så hvordan kan man eventuelt sikre sig, at de tidlige bidragsydere/skabere til det semantiske web også bliver retfærdigt belønnet ift.\ hvor meget gavn deres bidrag får for andre? Jo, det tror jeg, jeg har et muligt svar på. Ja, eller dvs., der er selvfølgelig flere svar; ét svar kunne jo bare være, at visse iværksættere danner en eller flere (tech-)virksomheder, der bare kan fungere lidt ligesom YouTube eller Twitch, eller hvad har vi? hvor virksomheden ikke har sine skabere ansat direkte, men i stedet giver dem løn som en slags belønning bagud for deres bidrag. Dette kunne man også sagtens forestille sig for det semantiske web. Og så kan virksomheden nemlig sætte betalingsmure op og/eller tilføje reklamer til platformen, hvor indtægten så altså bagefter kan fordeles, så skaberne også for en vis del af de indtjente penge. Og fordi sådanne virksomheder godt kan fungere uden at de kommer til rent faktisk at eje indholdet på noget tidspunkt (men alligevel kan tjene penge på det), så vil skaberne altså også sagtnes kunne tilslutte sig sådan en virksomhed uden stor frygt for at virksomheden bliver korrupt eller går i sænk, for så kan man jo bare gå sammen om at danne en ny og overføre indholdet til denne. Og lige inden jeg begyndte på at skrive ``Ja, eller dvs., \ldots'' her ovenfor, kom jeg endda i tanke om, at fordi en sådan (normal, kapitalistisk) virksomhed i den tidlige fase vil have sine motivationer liggende meget på linje med målsætninger om at skabe glæde og gavn for brugerfællesskabet, vil denne simple, meget konventionelle mulighed faktisk være rigtig god. Jeg har altså også nogle idéer til noget mere ukonventionelt, men lad os lige følge tanken om denne mulighed til ende. For ja, kapitalisme er nemlig super godt og effektivt, når det kommer til at opstarte og udvikle nye teknologier, så længe der altså bare er en vej så at tjene profit på disse idéer. Hvis der er det, så vil virksomhedsledelsen have klar interesse i at skabe et produkt, der er så brugbart og gavnligt for brugerne som muligt for at cementere sig selv som virksomhed. Dette er dog på nær, at sådanne virksomheder fra start måske bliver nødt til at hemmeligholde teknologier og begrænse adgangen til produktet, men det kan jo også være svært at komme udenom i visse tilfælde. Men ja, om så sådanne virksomhedstyper typisk vil blive ved med at være gode i det lange løb, når de først har cementeret sig grundigt, det er så diskutabelt. Det gode er så her, når det kommer til det semantiske web og særligt til p-ontologi-teknologien, at folk så til den tid vil få mulighed for at udpege eventuel stagnation og/eller korruption (og bemærk i øvrigt, at jeg har det med at bruge `korruption' lidt løst i mine noter, så det bare betyder, at man kommer til at vige fra de oprindeligt opstillede mål (og med den oprindelige hjemmel/hensigt bag de mål) pga.\ diverse ydre eller indre kræfter, der får en virksomhed/organisation/bevægelse/et individ osv.\ til at drive (drift'e) væk fra disse mål mere og mere), når den opstår. Så selv hvis en virksomhed til at iværksætte det semantiske web m.m.\ ikke fra start af har nogen rigtige foranstaltninger mod korruption, eller sågar hvis de slet ikke har som mål andet en at skabe profit fra starten af, så vil den p-ontologi-teknologi, som virksomheden er med til at bane vej for, faktisk, efter min mening, gøre at dette heller ikke rigtigt bliver så nødvendigt. For når først denne teknologi er udviklet, mener jeg som nævnt, at folk meget hurtigt og effektivt vil kunne gå sammen om at tilvejebringe et skift i virksomheden (eller skabe en ny virksomhed/organisation), så den igen kommer til at arbejde for brugerfællesskabets bedste først og fremmest (plus eventuelle andre velgørende formål, hvis man vil dette). Og når man så får mulighed for at gøre dette i tide, inden det går helt galt for den oprindelige virksomhed (hvad man ligeså godt bare kan antage, vil ske), så vil det jo være helt naturligt for brugerfællesskabet at sige: ``denne virksomhed har virkeligt givet os meget og har bragt meget godt med sig, så derfor bør vi helt klart ikke vende den ryggen helt, bare fordi den jo er nået et punkt, hvor dens virke ikke ligger helt på linje længere med vores egne interesser, hvad der jo måtte ske før eller siden. Så nu forcerer vi, at der sker et retningsskift i virksomheden, men sørger alligevel for at ejerne af virksomheden samt de oprindelige iværksættere bliver gavmildt belønnet for deres indsats og for alt hvad de har tilvejebragt.'' Da dette vil være den naturlige tilgang; selvfølgelig vil man være tilbøjelig som brugerfællesskab til at se positivt på iværksætterne og synes at det er rimeligt, at deres personlige udbytte/belønning for foretaget sikres, især hvis vi kun er nået til en overgangsfase, hvor virksomheden kun så småt begynder at afvige fra brugerfællesskabets interesser. Så investorene behøver ingen gang at frygte, at brugerne tager over, selv ikke, mener jeg, hvis nu de vælger ikke at gøre noget for at forhindre dette (hvilket vil sige forhindre deres egen ``korruption''). Og selv hvis brugerne nu skulle smide deres gamle ledende virksomhed på porten helt, nå ja, så vil man jo alligevel have tjent en vis profit op til det punkt. Ja, og i vores økonomiske system, hvor investorer tjener deres penge på opsvinget og ikke så meget på, hvorvidt en virksomhed er langtidsholdbar eller ej (hvilket er endnu en mulig kritik af kapitalismen, som den er nu (som godt nok hører ind under den tidligere om ``at det er diskutabelt hvor effektivt det er, når virksomheder når forbi den tidlige fase og først får cementeret sig selv)) --- og man kan jo endda tjene penge på nedsving i dette system --- så kan det jo bare regnes for en del af spillet, nemlig at investorer enten må sørge for at tilvejebringe foranstaltninger, så der ikke er nogen risiko for at brugerskaren bliver vrede på virksomheden (og det burde slet ikke være svært), eller også må man bare som investor have ansvar for at trække sig ud i tide, hvis dette ikke lykkedes. Så ja, pointen er bare, at jeg ikke tror, at en sådan virksomhed behøver frygte overhovedet, at de jo selv vil tilvejebringe gode muligheder for, at brugerne tager over. Det virker for mig som om, at der er mange gode måder at eliminere risikoen for, at dette på en eller anden måde giver økonomisk bagslag, enten ved at bevare kursen ift. at blive ved med virke efter at skabe gavn for brugerne (og altså ikke begynde at lade dem i stikken i første omgang), eller ved bare at gardere sig på andre måder økonomisk mod bagslaget, som dette ville kunne give. Og alt dette kommer bare lige fra toppen af mit hoved, så virksomheder må jo selv lave vurderingen, hvad risikoen er, og om den overhovedet er der, som jeg her forudser (selvom jeg dog altså forudser at denne risiko er meget lille), når de når til det punkt. Så ja, jeg kan altså faktisk rigtig godt lide denne løsning, også selvom jeg jo som nævnt har udviklet et mindre konventionelt alternativ. 

%Det gode er så også, at jeg netop har formet idéen om dette alternativ på en måde %(lyder mindre prætentiøst)
%(det føler jeg i hvert fald, jeg er lykkes med), så løsningen spiller rigtigt godt sammen med konventionelle virksomheder. Og dermed mener jeg, at konventionelle, kapitalistiske virksomheder vil kunne adoptere min alternative idé i mere eller mindre grad, hvis man tror på at dette kan hjælpe én med at bevare det gode samarbejde med sine brugere. Man kunne således fint starte som en helt konventionel virksomhed og så bare lige tilføje nogle målsætninger om, at man på sigt vil prøve at adoptere flere og flere retningslinjer fra min alternative idé, når det bliver muligt og relevant. Og set fra den anden side kan en organisation, der har som et klart mål at implementere min alternative løsning, også sagtens arbejde sammen med og side om side med konventionelle, kapitalistiske (tech-)virksomheder. Den ene type virksomhed behøver altså ikke at bekæmpe den anden, især ikke i den tidlige fase af det semantiske webs udvikling. Når vi så når til at få udviklet teknologien omkring p-ontologier, mener jeg dog, at organisationer, der følger retningslinjerne fra min ``alternative løsning'' (som er en generel bevægelse om at tilvejebringe visse forhold, men det kommer jeg til), vil begynde at tage over, men dette skyldes de samme argumenter, som gør at jeg som sagt mener, at brugerskaren også vil tage over for den oprindelige virksomhed alligevel. Så på den måde skaber eksistensen af en bevægelse, som følger omtalte ``alternative idé,'' altså ikke nogen ekstra fare for de mere konventionelle tech-virksomheder, der prøver at iværksætte udvikling af det semantiske web m.m.; hvis der kommer stor efterspørgsel efter en organisation, der følger retningslinjerne fra min idé om en økonomisk bevægelse, jamen så kan disse virksomheder bare selv begynde at adoptere disse retningslinjer i god tid, og således vil de kunne bevare deres kunder/brugeres loyalitet. 

Da jeg begyndte at skrive denne sektion, havde jeg en lidt anden form for ``alternativ løsning'' på sinde, end de løsninger, jeg primært vil foreslå i stedet. Den tidligere ``alternative løsning'' var nemlig et forslag til en bevægelse, der skulle fungere lidt ligesom et KV-system, hvor visse værdier altså bunder i selve opbakningen til bevægelsen. Dette er jo på papiret egentligt fornuftigt nok, for alle værdier, som ikke er direkte naturalier, i vores nuværende samfund bygger jo også i sidste ende på en fælles tilslutning til en konsensus og en tro på et system. Men selv hvis min analyse omkring, hvorfor at denne bevægelse muligvis kunne holde, så er idéen stadig kompliceret nok til, at der ville være lang vej så at godkende den analyse i fællesskab, så man rent faktisk kan opnå den konsensus. Så i stedet vil jeg bare inkludere denne (eller rettere disse, for der var flere alt i alt) idé(er) om en KV-agtig økonomisk bevægelse i form af nogle overvejelser om, hvad et godt, velfungerende og stabilt fremtidigt økonomisk system kunne indebære. Nogle af de følgende paragrafer har jeg så skrevet, imens jeg stadig en anden kurs med denne sektion i tankerne, men det går nok. Jeg prøver bare lige (relativt i tid til skrivende stund) at væve dem lidt sammen, så det ikke er helt galt. Uh, men inden vi går videre til disse paragrafer, bør jeg forresten lige nævne, at de ``alternative løsninger,'' jeg i stedet primært vil slå et slag for her, bare er nogle idéer til nogle lidt alternative virksomheder -- én som er en slags fagforeningsvirksomhed *[Nej, den idé streger jeg faktisk lidt og nøjes med:], og én (som jeg er mest opsat på) hvor kundeskaren tilsammen (via stemmeretter) får magt over lønningerne til arbejderne i virksomheden --- men det vender jeg altså tilbage til senere.  

%%Nå nu bør jeg jo så forklare om, hvad min alternative idé går ud på ...eller vil jeg ikke stadig gerne lige snakke om nogle andre ting først (så jeg prøver at opstille argumentationen, så jeg nærmer mig den udefra..)? ... Nej, jeg kan sagtens bare forklare det lige på nu. For bevægelsen er bare ``at tage lidt forskud på, hvad der alligevel kommet til at ske.'' ...
%Nå nu bør jeg jo så forklare om, hvad min alternative idé går ud på. Jeg havde egentligt tænkt mig, at jeg ville prøve at skrive motivation for idéen ved at starte lidt udefra og analysere økonomiske systemer generelt. 
%%Hm, kommer jeg ikke også til at gøre dette alligevel? 
%Men nu kan jeg se en mere simpel motivation: Idéen går i store træk bare ud på at prøve tage lidt forskud på det retfærdige og gavnlige system, som man alligevel forventer, hvis man er enig i min argumentation, vil 
Lad os nu se på hvilket system (forhåbentligt gavnligt og retfærdigt som nævnt), der mon vil fremkomme, når først folk begynder at benytte de avancerede analyseværktøjer, og ``aftaleværktøjer,'' som vil komme med udviklingen af det semantiske web og p-ontologi-teknologien. %For at nå til hvordan dette muligvis kan lade sig gøre, lad os så først starte med at se på, hvad dette ``retfærdige system'' kommer til at indebære. 
Lad os holde fast i en antagelse om, at det økonomiske system vi går i møde fortsat vil være liberalistisk, således at den mest vigtige drivende kraft for systemet er folk egeninteresse. Dermed er det ikke sagt, at folk kun tænker på sig selv, men bare at systemet stadig kan antages primært at være drevet af at folk vil forsøge at varetage deres egeninteresse. Når vi så er færdige med denne analyse, kan vi så gå tilbage og se på, hvor sandsynligt og hvor gavnligt dette vil være. Men lad mig allerede pointere nu, at hvis et system i sidste ende alligevel medfører gavn og retfærdighed til alle dets deltagere, jamen så er det jo kun godt, hvis systemet så ovenikøbet baserer sig på nogle fundamentale drivkræfter i mennesker, således at systemet derved bliver meget robust.\footnote{
	Der er forresten lidt en tendens til, når folk udtænker og/eller er tilhængere af radikale ændringer i samfundet, liberitarisme, socialisme osv., at de så kommer til at gå ud fra, når man prøver så at beskrive, hvad det nye system vil føre med sig, at man kommer til at antage, at de ledende mennesker i bevægelsen, og tit også mennesker generelt, pludselig vil blive grebet af en træng til bakke op om bevægelsen og om at være generelt agere gode, solide, integritetshavende brikker i bevægelsen. Jeg mener altså herved, at folk har det med at antage, at der med bevægelsen vil følge en helt ny ånd (som i `spirit') iblandt folk, så disse vil begynde at agere mere uselvisk. Og dermed får man så fejet det faktum ind under gulvtæppet (sikkert ofte underbevidst), at mennesker altid vil være mennesker; de vil være de samme. Og alle tendenser til at lade sig korrumpere på alle mulige leder og kanter vil fortsat være til stede, og vil altså altid være noget man skal håndtere i systemet. Så ja, den eneste seriøse teoretiseren omkring alternative systemer/økonomier/bevægelser/foreninger/organisationer, og hvad har vi, er analyser, der meget aktivt prøver at overveje mulighederne for korruption, og som søger at finde en løsning, hvor korrumperende kræfter kan holdes mest muligt i skak. Teorier, som for at fungere helt bliver nødt til at antage, at folk vil agere moralsk og/eller uselvisk (uden at sørge for at implementere et ydre pres for at sikre dette), hvilket bl.a. socialisme/kommunisme helt klart gør, og jeg har også bestemt hørt liberitanere argumentere ud fra sådanne antagelser *(og for anarkister er en sådan naiv antagelse endda central for hele idéen, smh (\ldots Hvis jeg har forstået det rigtigt, det er ikke sikkert)), holder bare ikke, for hvis folk var så gode og moralske, så ville det nuværende system også allerede fungere fint ifølge samme teori, og der ville ikke være nogen grund til at prøve at finde alternativer. Og det duer altså ikke at antage helt grundløst, at en vis bevægelses fremkomst så pludseligt vil ændre menneskers tendenser --- især ikke hvis man så bare tænker én generation videre i det nye system (men i praksis vil der ingen gang gå så lang tid; de fleste mennesker vil nok begynde at give efter for fristelser inden for en lille årrække, og så vil dominobrikkerne allerede begynde at falde). *(Dermed ikke sagt, at der ikke godt kan være mange mennesker, der \emph{vil} blive ved med at undgå fristelser og handle moralsk, men man bør altså alligevel bare for alt i verden prøve at undgå, at fristelserne er til stede i første omgang bag de lukkede døre.)
} 

Okay, hvordan ville et godt liberalistisk system se ud, når først udviklingen af det semantiske web og af p-ontologi-teknologien har fundet sted og folk har haft mulighed for at bruge de nye teknologier til at finde frem til at godt, hensigtsmæssigt system? Tjo, man kunne jo meget vel forestille sig, at man kunne nå til et punkt, hvor hele det økonomiske system med alle dets arbejdsstillinger og alle dets investeringsmuligheder osv., samt alle de forskellige former for handlinger folk kan udføre i disse stillinger, blev kortlagt i en ontologi, og hvor man så gik sammen og fik indført, enten via en demokratisk proces eller via andre former for forhandlinger, et system, hvorved lønningerne for diverse handlinger jævnfør førstnævnte model bliver fastsat til en vis grad ved denne forhandlingsproces, og at man så også implementere et system til at håndhæve disse lønninger. Hvis man kan opnå dette, kunne det jo nemlig muligvis (og det mener jeg helt klart, det kunne) skabe et meget mere åbent arbejdsmarked, hvor folk med flere færdigheder (hvad folk altid i reglen har) kan overskue systemet og bidrage der, hvor der er behov, uden først at skulle igennem en masse ansættelseskontrakter osv. For hvis lønnen allerede er sat lidt mere centralt (i det pågældende lokale økonomiske system; disse lønningssystemer kan nemlig sagtens, og bør, være lokale for et specifikt område (geografisk og muligvis også fagligt), så man ikke skal til at nå til enighed på et globalt plan; det ville være alt for omfattende i udgangspunktet), så skal der ikke så mange forhandlinger imellem arbejder og arbejdsgiver til, før arbejdet kan påtages. Og hvis arbejdere har en digital profil, hvor alle tidligere arbejdsopgaver er dokumenteret til en passende grad (det behøver ikke være sådan, at alle mennesker kan se alt hvad alle andre har lavet; man kan helt sikkert godt lave en mere anonym, men stadig effektiv måde at at gøre det på), jamen så bliver det en let og hurtig proces at udbyde en arbejdsopgave. Og hvis lønningssatserne endda kan sættes så de tager højde for parametre omkring, hvor godt en arbejdsopgave er udført, hvad de sagtens kan i teorien, når først teknologien er der, så gør dette endda ansættelsesprocessen endnu nemmere. Og eftersom at de fleste arbejdsopgaver til den tid vil kunne løses foran en computer eller et andet digitalt interface (eksempelvis hvor man kan styre robotter, når vi når så langt), så vil et sådant system i særdeleshed blive gavnligt for den fremtidige økonomi. Vil vi så kunne opnå sådan et system i en nært forestående fremtid? Ja, sagtens. Især hvis altså hvis vores analyse-og-beslutningsfærdigheder kommer til at stige så meget som civilisation med p-ontologiernes fremkomst, som jeg forudsiger, de vil, så må dette blive nemt nok at opnå, mener jeg. Og hvis så folk altså også bare når frem til, at dette vil være en god måde at gøre tingene på, nemlig at det ikke kommer til f.eks.\ at skabe høj ulighed eller andet dårligt, jamen så vil det også blive sådan forholdene kommer til at blive i fremtiden. Vil et sådant system, hvor arbejdsgivere bare kan vælge og vrage mellem alle mulige udbud, så være godt? Der er jo allerede tale om ``the gig economy'' og dens negative sider, så vil sådan et system ikke føre en masse dårligt med sig? Nej, det mener jeg ikke, for til den tid vil det, som jeg har antydet, blive pærelet at gå sammen som arbejdere for at forhindre sådanne situationer. Ja, og hvad det vil sige at være ``arbejdsgiver'' vil også være noget helt andet til den tid. Det vil blive det letteste i verden for folk at gå sammen om at danne demokratiske virksomheder til at overtage markedet for eventuelle grådige virksomheder, hvis behovet bliver tilstrækkeligt stort, så i denne fremtid vil arbejdsgivere i langt højere grad være ansat på brugerne/kundernes, og arbejdernes, vegne frem for på enkeltindividers kapitalinteressers vegne. Så man vil dermed sagtens kunne opnå et økonomisk system herved, hvor folk for løn efter\ldots\ ja, man kan sige det helt kort: hvor arbejdere for løn efter lønsatser bestemt af kunderne/brugerne. Og så er det klart, at brugerne/kunderne jo vil være motiveret for at give arbejdere gavmild løn, hvis der er behov for at fremme lysten til at gøre den pågældende type bidrag. Og samtidigt vil kunderne/brugerne jo også udgøre hele befolkning, som også generelt set vil have lyst til at arbejde og at få en ordentlig løn for dette, så dette vil derfor også blive sikret ved sådan et system. Pointen er altså i virkeligheden lidt, at et liberalistisk system jo i teorien vil være styret af forbrugerskaren, hvilket det dog ikke bliver så meget i praksis i det nuværende system, men ved så at give folket bedre analyse- og forhandlingsmuligheder (og altså analyse- og aftale-teknologier), så mener jeg altså, at man så også vil kunne opnå denne realitet i praksis. Så ja, lønsatser vil altså med al sandsynlighed i sådan en fremtid (hvis denne altså forbliver liberalistisk (hvad den med al sandsynlighed vil, men det vil jeg som sagt vende tilbage til)) blive sat ud fra en lidt mere central (men dog ikke helt central; kun for det pågældende lokalområde, hvor stort dette så end måtte være) forhandlingsproces. Man kunne så her kritisere mine argumenter og sige, at sådanne vidtomfattende lønsatsmodeller vil være alt for komplicerede at udvikle og implementere i praksis. Og med den måde jeg skrev om det ovenfor, kunne det også godt lyde som om, dette ville indebære, at man udformede en ufatteligt detaljerede model til at tage højde for alle arbejds- og investerings- (og også forresten donations-)mulighederne. Dette mener jeg, slet ikke er en nødvendighed. Jeg mener at man sagtens i stedet ville kunne udvikle en lønsatsmodel baseret på mere abstrakte parametre, så man allerede fra start kan dække hele landskabet af bidragsmuligheder (altså arbejds- og investeringsbidrag m.m.), og hvor man derfra bare kan gøre dette landskab mere og mere detaljeret ved at føje flere og flere abstrakte parametre til modellen, således at arbejds- og investeringsbidrag m.m.\ kan belønnes på mere og mere detaljeret og sofistikeret vis. % (dropper måske eksempler..)
%Husk at forklar, at lønningsmodellen ikke behøver at være vildt detaljeret, fordi det godt kan lade sig gøre i stedet, at bruge nogle mere abstrakte parametre. (næsten-tjek; skal jeg lige give nogle eksempler? Jo...)

Nå ja, og man kunne også komme med den kritik, at der muligvis kan være en risiko for at et sådant system vil ende med kun at gavne en vis klasse af kunder og arbejdere (og investorer m.m.) men lade en vis underklasse i stikken, der ikke kan bevæge sig op og opnå midler, der skal til for at få forhandlingsmagt, og som muligvis også holdes nede af, at de eventuelt har svært ved at finde arbejde --- måske kunne man endda forestille sig en virkeligt grel situation, hvor en underklasse nærmest (altså mere eller mindre) bevidst holdes ude af arbejdsmarkedet. Men jeg tror dog, at det korte af det lange dog er, at der altid vil være mere at vinde ved at integrere en befolkningsgruppe i middelklassen frem for at holde dem uden for. Selv i et højteknologisk og ret digitaliseret samfund vil man stadig altid kunne udtænke masser af arbejdsfunktioner, som vil kunne varetages af næsten alle, og selv til funktioner, der kræver flere færdigheder, vil det også altid betale sig bedst at skabe en så stor mængde faglige dygtige individer som muligt, så man derved øger færdighederne generelt på arbejdsmarkedet. Ja, endnu kortere kan man sige: Det vil aldrig kunne betale sig at have en underklasse på slæb, der ikke er i stand til at varetage arbejdsfunktioner, hvad end de er kunstigt holdt ude, eller om det skyldes manglende uddannelse, så det vil altid kunne betale sig i et samfund, at hive en eventuel underklasse med op. Og i øvrigt giver det også bare generelt mere lykke i et samfund, når der er en sådan god lighed mellem folk. Dette er også bl.a.\ fordi, eksistensen af en underklasse jo også altid vil lægge et pres på ikke-underklasse-folk for at undgå, at de selv falder igennem og ned til underklassen, hvor der jo i stedet kunne have været et sikkerhedsnet til at gribe dem.

Så fremtiden ser altså ret lys ud i mine øjne, især fordi jeg også mener at udviklingen af det semantiske web og af p-ontologi-analyse-/diskussions- og beslutnings-værktøjerne er rimelig uundgåelig inden for en overskuelig fremtid, medmindre en singularitet (så som f.eks.\ en atomkrig eller lignende (så et stort 7, 9, 13 på den der ``rimelig uundgåelig''-udtalelse)%(\ldots 7, 9, 13, selvom det godt nok kun er et omvendt jinx\ldots)
) skulle forekomme i mellemtiden. Vi har dog stadig til gode at se på, om vi overhovedet kan gøre vores antagelse om, at vores system vil forblive liberalistisk i mellemtiden, og hvad det ville betyde ellers. Men der er faktisk ikke så meget at sige til dette punkt, nu hvor vi har set på, hvad der kan lade sig gøre med et liberalistisk system (læseren er måske ikke helt overbevist bare ud fra denne korte lille tekst, men jeg er sikker på, at hvis man dykker ned i sagerne og virkeligt tænker over diverse muligheder, så vil man i sidste ende komme frem til det samme med god sandsynlighed). For der er ikke noget som helst, der tyder på, at vores nuværende generelle samfund ikke vil fortsætte i liberalismens spor. Så en afvigelse fra dette spor må så enten skyldes en fejlagtig (altså hvor vi antager at den \emph{er} fejlagtig) analyse der konkluderer (fejlagtigt pr.\ vores antagelse), at det ville medføre noget bedre, men da vi kun vil blive bedre og bedre til at analysere alle forholdene og mulighederne \emph{korrekt}, vil dette jo være ret usandsynligt; det er allerede et usandsynligt sporskifte nu, og så vil det kun blive mindre og mindre sandsynligt som tiden går (givet antagelsen om at analysen var fejlagtig). Eller også vil analysen være korrekt, i hvilket tilfælde: Hvad er der så at bekymre sig over? Hvis vi sammen finder et godt alternativ til det nuværende spor, der faktisk med al sandsynlighed vil føre til mere gavn og glæde, og vi så tager dette sporskifte, jamen så er det jo bare godt. Jeg tror stadig ikke, at det på nogen måde er sandsynligt, at dette vil ske, før vi allerede er nået til de fremtidige forhold, jeg har beskrevet for det liberalistiske spor (og jeg tror egentligt også, det er mest sandsynligt, at vi så også vil forsætte i det spor i meget, meget lang tid efterfølgende), men selv hvis jeg tager fejl, så er der altså heller så meget at frygte ved dette, så længe beslutningen så bare gjordes på baggrund af en virkeligt, virkeligt omfattende og grundig fælles analyse. Så alt i alt er jeg altså kommet frem til --- og mon ikke læseren kan gøre det samme? --- at en sådan fremtid som beskrevet, altså liberalistisk og med en ret demokratisk (muligvis helt demokratisk (med én stemme pr.\ hoved), hvis man altså først får forhandlet sig frem til dette) forhandlet lønsatsmodel (der så også håndhæves), vil være meget sandsynlig.\footnote{7, 9, 13.}
%Kom tilbage til "hvor sandsynligt og hvor gavnligt dette [system] vil være." (tjek)

%Brain: Man kunne lige forklare lidt om, at systemer trods alt \emph{kan} være stabile, trods den selviskhed og alt det, jeg lige påpegede, om ikke andet så simpelthen fordi folk ikke vil være interesserede i at røste et velfungerende system bare for en udsigt til en lille smule ekstra penge/besiddelser. Dette gør, at folk heller ikke vil svigte den tidligere generation påny, påny, påny, og her kan jeg så også lige komme ind på det med at håndhæve et system ved straffe, som jeg jo mangler.. Og så er det jo nemlig, at min bevægelsesidé kommer til at bygge på, at den/de tidlige organisation(er), der tilvejebringer, hvad end lønsatsmodel-system, man vil starte med at have i den nære fremtid, så derfor kan være ret afhængige af denne/disse organisation(er). ..Kan de ikke?.. Det kan de jo, hvis organisationen altså danner en hel bølge, og opsamler en hel del kapital og indflydelse på vejen.. Hm... Tja, ville et alternativ ikke lidt være bare at fokusere på sem-web-virksomheder, og så give nogle hints til, hvad der kunne være smart at gøre for virkeligt at få folk med..?.. Tja, eller det er jo så netop det, der udgår den ene løsning, og det ville jo være dejligt med en lidt mere.. decentral.. løsning, hvor man også kan opnå lidt investerings- (og banking- sådan set) hype.. Ja, for tanken er jo lidt, at hvis idéen så ikke slår igennem via en normal virksomhed.. eller der ligesom bare ikke er den samme opbakning.. pointen er at der måske kunne være langt højere opbakning til en virksomhed/organisation, der ligesom kan love mere åbenhed og lighed/retfærdighed.. ..Tjo, men er det så ikke bare et spektrum af.. jo, det er jo netop det, det er, men er det virkeligt så også det man skal se det som: ikke som to alternativer til hinanden, men som den samme idé, der bare har visse paramtre at stille på så at sige? ... ..For når det kommer til stykket, kan man vel ikke være sikker på, at tingene vil forme sig på en bestemt måde, og er det ikke lidt det, jeg lidt har antaget, når jeg har overvejet idéen; at man kan regne med, at folk lidt vil være nødsaget til at respektere de her (tidlige) bidrags-værdipapirer..?... Tjo, men hvis man jo netop kan få nye medlemmer til ligesom at give løfter om, at de vil respektere de værdi-satser inden for de og de rammer ift. den fremtidige udvikling... Ja, og det smarte er jo netop så, at folk kan gå sammen om at "investere" i bevægelsen uden rent faktisk at lægge nogen penge på bordet; kun løfter..(!).. ..Ja..!! Og når bevægelsen så bygger på løfter, så fungerer det jo stadig også ret fint på omtalte spektrum, for en virksomhed/organisation kan så med fordel igangsætte et løfte-register så at sige. .. (03.05.21) .. Ja, det er jo en vigtig pointe ved det hele, som jeg lige har haft glemt lidt igen nu her, nemlig at "løfte-grafen" er vigtig for at oprette tillid og sikkerhed til, at "værdipapirerne"/"tokens'ne" bliver respekteret og håndhævet i sidste ende. ..Ja, så er det egentligt ikke bare én idé på et spektrum. Det vil sige, "idéen" er så egentligt bare flere små idéer til nogle ting man kunne gøre som sem-web-virksomhed for at tiltrække flere skabere/bidragsydere (og vendt om nogle idéer til, hvilke nogle målsætninger man bør søge efter hos en virksomhed/organisation, når man melder sig til en sådan (i flok eller som enkeltindivid))... ..Og ja, pointen er jo så, at folk vel nok får brug for deres etos og derfor ikke kan tilldade sig selv at bryde sine løfter.. ..Hm... Nå ja, jeg har jo netop formet (hvorfor er jeg så sløv?) idéen, så der er en stor frihed til de første organisationer, og så de også kan fungere ret konventionelt i princippet, så nej, der \emph{er} ikke to adskilte idéer; idéen er at have virksomheder/organisationer, som kan benytte sig meget af det omkringliggende samfund og også selv bruge konventionelle principper fra dette, og så er det netop op til hver enkelte virksomhed/org at beslutte, hvor mange af mine idéer, man vil følge, og i hvor høj grad. Så idéen (ud over at der også bare generelt er nogle overvejelser om, hvordan det nære fremtidige samfund kan være) er bare nogle mulige måder, hvor man måske kan booste en sådan sem-web-bølge-virksomhed. 
%Men ja, så kan jeg altså lige stadig starte med det nævnte om, at man sagtens (self.) kan have velfungerende systemer, og folk vil desuden også have en vis tendens til at støtte gode og retfærdige principper (for selv vis de ikke selv følger dem --- og jeg kunne i øvrigt også lige præcisere, at folk jo godt kan være ikke-så-korrumpérbare, men det skal man bare helst holde sig fra at regne med --- skulle de selv befinde sig i en situation, hvor de bliver fristet over længere tid, jamen så er det jo stadig nemt at bakke op om gode og moralske ting, når det handler om politik osv., hvor det ikke handler om én selv og i øvrigt også er mere offentligt ligesom (korruption (altså falden for fristelser) sker nemlig kun, når andre alligevel kigger væk eller ikke kan se (så bag "lukkede døre" typisk))). Og så kan jeg som nævnt også passende nævne om løfter (løfte-grafen) osv., og om håndhævelsen af dem. Men ja, ellers er det så bare at komme ind på, hvad en virksomhed så kan gøre for at være mere som min "bevægelses-org," og dermed altså muligvis så kan tilstrække meget større opbakning og funding.

%Inden vi når til min idé om en økonomisk bevægelse (\ldots eller hvad vi lige helt skal kalde den\ldots), så skal vi dog også lige se på, 
Og inden vi går videre, bør vi også lige se på, 
hvordan det omtalte fremtidige (altså sandsynligvis fremtidige efter min mening) system vil sikre, at investorer og/eller pensionerede arbejdere osv., der stadig har løn, besiddelser og/eller andre rettigheder til gode i det nuværende system, ikke bliver ladt i stikken af fællesskabet. For man kan jo ikke rigtigt have et system, hvor folk bare får deres løn på én gang i form af de faktiske goder vedkommende kan nyde. For hvis man betaler i penge (inkl. besiddelsesrettigheder m.m.) kan disse jo i princippet altid bare tages tilbage igen af fællesskabet/samfundet. Og desuden kan en løn, som jeg vist kun var en smule inde på ovenfor men nu kan understrege her også, jo sagtens afhænge af en fremtidig udvikling efter at bidraget var gjort. Det er jo umiddelbart ret essentielt for et liberalistisk system, at man kan belønne investeringer, både penge- og arbejdsinvesteringer (\ldots og reklamering/opmærksomhedsskabelse) osv., ud fra, om de blev gjort på frugtbare steder eller ej. (Og dette bør altså være en del af det omtalte parameterlandskab for lønsatsmodellen.) Men ja, hvordan sikrer man sig så, at sådanne løfter om fremtidig betaling/belønning kan overholdes i et sådant liberalistisk system. Jo, vi ved jo fra vores nuværende generelle system, at man sagtens kan opnå systemer, hvor det kan sikres, at løfter (i hvert fald visse typer: de ``juridisk bindende'' løfter i vores nuværende samfund) generelt overholdes. Og når det kommer til stykket, så er det bare en del af vores natur at være i stand til at indgå i selv komplekse sociale systemer, hvor man har mulighed for at indgå diverse aftaler, som man så også kan have tillid til bliver overholdt. Jeg nævnte i en fodnote ovenfor, at det dog også er i vores natur at lade os friste med en vis sandsynlighed, især når dette ligesom sker bag lukkede døre og med usandsynlige udsigter til, at dette vil forfølge en (enten juridisk eller hvad ens ry og omdømme angår). Men hvad angår, hvordan vi agerer som individer og grupper i fuld offentlighed, så er vi altså i den grad i stand til som art virkeligt at skabe stærke sociale rammer, som sikre at folk generelt følger de samme lovmæssige, og endda til en vis grad også de samme moralske%\footnote{
	%Og lad mig lige nævne, at der altså for mig er en klar forskel på begrebet om, hvad der er `etisk,' og hvad der er `moralsk.' En handling er for mig etisk, hvis... %etik.. men kan så ikke lade sig gøre i praksis for individet, men det er også derfor, at man må se det mere fra smafundsperspektiv, og så er det bare klart, at et sådant samfund så ikke kan forvente, at folk handler "etisk" (men kun moralsk). Moral er så en betegnelse for noget, som vi er i stand til at opretholde i et socialt system; definitionen er nemlig, hvad der er rigtigt i samfundets øjne. Bemærk derfor at moralske handlinger ikke nødvendigvis er etiske (og omvendt). Og angående denne forskel kan jeg så lige nævne et eksempel om, at det kan være etisk at give sit tøj til en, der har mere brug for det, men hvor det dog ikke er etisk for samfundet generelt at ophøje dette til en moralsk regel (og ligesom håndhæve dette), da det i praksis så vil have alle mulige andre konsekvenser (i og med at man jo så altid i en eller anden grad tvinger denne "gode" handling ud af folk).
%} %Nej, jeg gider ikke skrive denne fodnote færdig; det er meget simpelt, det er allerede sådan de fleste nok skelner mellem de to begreber, og jeg har et etik-afsnit nedenfor, der er fint nok. Så denne fodnote kan bare blive her i kommentarerne, selv i dens nuværende overvejende stikordsform.
, retningslinjer. 
Og det vil også være sådan for dette (forhåbentligt) fremtidige system, nemlig at det ikke vil kunne betale sig for folk at tillade sig selv og hinanden at begå løftebrud, der underminerer de lovmæssige principper (og moralske principper, hvilke jo i reglen vil være til stede som det grundlag, det lovmæssige system så kan bygges ovenpå). For hvis man først begynder på dette, efter at man egentligt har fået opsat (eller eventuelt bare har bygget videre på det tidligere, i.e.\ vores nuværende system, hvad der jo nok er klart mest sandsynligt (i hvert fald efter min umiddelbare mening)) et godt socialt, lovmæssigt system, så vil man jo ødelægge al den tillid man har opbygget til hinanden og til systemet, hvilket man så bliver nødt til at prøve at opbygge igen. Og hvem siger, at dette vil være særligt nemt? så der vil altså altid være god grund til at prøve at opretholde lovmæssige systemer og sociale aftaler generelt i et samfund. Og samtidigt vil et ``samfund'' i den nære fremtid også kun være et lokalsamfund i en eller anden forstand, da vi jo lever på en ret stor planet. Man kunne forestille sig et globalt samfund, men dette vil ikke være en særlig god ting at satse på, da det vil være for uholdbart ift.\ den langt bedre løsning at bevare en vis adskillelse (modularitet, om man vil) i verdenssamfundet. Så hvis man derfor som individ eller lokal gruppe af mennesker pludselig viser for omverden, at man er klar til at bryde aftaler og løfter bare for at pleje sin egeninteresse en smule mere, jamen så risikerer man jo også bare, at folk fra de omkringliggende samfund ser dette, og fremover så vil tøve meget med at indgå nye aftaler med pågældende gruppe/individ som følge af dette. Så ja, jeg ved godt, at dette er ret trivielt, men jeg føler alligevel, det er værd at udpensle analysen så meget her --- %vi skal nemlig bruge det her i det følgende. 
det bliver nemlig rart at have fået det på plads i det følgende.



Nu er vi så noget forbi de paragrafer, jeg skrev inden at gik lidt væk igen fra mine idéer om nærmest lidt KV-agtige økonomiske bevægelser og over til bare at ville foreslå to mere simple alternativer til den helt konventionelle løsning på at fremme det semantiske web, som er bare at starte en normal tech-virksomhed til formålet. Nu vil jeg så beskrive det grundlæggende i mine oprindelige idéer, men ikke hvor jeg gør rede for, hvordan de eventuelt kan fremkomme / udvikle sig. Jeg vil i stedet som nævnt bare beskrive dem i form af, hvad et fremtidigt system, som håndhæver en lønsatsmodel, kunne indebære. For det første vil jeg nævne, at det muligvis faktisk kunne være en holdbar idé at bruge lykke/velfærd som en grundlæggende parameter, når man skal vurdere, hvor meget en handling skal belønnes med. Jeg kom ellers lidt frem til for noget tid siden, at det ikke rigtigt ville holde, for hvad hvis man får en god idé, som bliver ved og ved med at bringe folk lykke? Det ville potentielt set føre til en rigtig høj belønning/løn, også selvom man måske bare lige var lidt tidligere ude end andre. Men pointen er så dog, at man jo bare kan tage højde for dette i beregningerne\ldots\ Tja, måske holder det ikke alligevel (for hvad med kunstneriske værker m.m.)\ldots\ Nå, men tanken var altså, at man måske faktisk nok godt i sådan et system kunne søge efter tilstand, hvor lønningsberegningerne i høj grad (og på lødig vis, det er selvfølgelig også målet) kommer til at se på den lykke/velfærd, som handlingen tilvejebringer statistisk set (ift. til den gennemsnitlige verden, hvor handlingen ikke blev gjort). Men ja, ellers har jeg tænkt meget i et system, hvor man i stedet bare søger imod en lønsatsmodel, der ikke nødvendigvis benytter et bestemt begreb så som lykke/velfærd, som man så har aftalt fra starten, men stadig søger imod en model, som prøver at bruge nogle ret abstrakte, generaliserbare koncepter til at bestemme lønningerne. Tanken er, at hvis man kan sikre dette, så undgår man at folk bare lige impulsivt kan stemme sammen om at justere nogle specifikke parametre i en kompliceret model, og dermed så kan opnå, at en vis arbejder eller arbejder gruppe pludselig ikke har udsigt til nær samme løn som før. Så med andre ord bør man søge mod nogle ret stabile koncepter, så lønsatsmodellen derved ikke bliver særligt flygtig. Og når man så har fundet nogle gode parametre, eller for den sags skyld inden man har dette, så bør man også sørge for, at der er en vis inerti i parametrene; at folk kun har mulighed for at ændre dem lidt ad gangen, igen for at skabe større ro og sikkerhed omkring modellen. Det skal så gerne blive sådan, at hvis en arbejder eller arbejdergruppe opfinder og/eller skaber nogle vildt gode nye forhold, hvorved de så får udsigt til en stor løn via modellen, så vil dette ikke bare medføre, at folk skynder sig at ændre modellen for ikke tilsammen at skulle betale lige så meget løn til pågældende arbejder/opfindere (selv bare antaget at folk vil benytte sådanne muligheder, så længe systemet tillader det), fordi inertien i modellen gør, at man kun kan ændre dette langsomt, og så risikere man bare at skræmme fremtidige opfindere mere væk fra deres eventuelle foretagender, især netop fordi der så også vil være en tilsvarende inerti i at stemme og justere lønsatsmodellen tilbage igen til dens første stadie. Ja, så det var vist egentligt bare det der skulle nævnes, nu hvor jeg ikke længere føler, jeg behøver at prøve at redegøre for en muliged, hvormed man kunne prøve at opstarte en bevægelse, der søger at få fremført sådanne lønsatsmodeller. Nu nøjes jeg nemlig bare nogle mere simple bud på virksomheder/organisationer, hvor kunderne i sidste ende også må komme til ultimativt at bestemme, om ikke andet så bare fordi, de nye teknologier omkring det semantiske web altså (som jeg forudsiger det) vil medføre bedre muligheder for at forenes og forhandle med deres købsmagt. Og hvad end system så opstår da, kan jo så passende benytte de forslåede principper fra denne paragraf, hvis dette anses for klogt. 

\subsubsection{*Idé til en mere kunde-/brugerdrevet virksomhedstype\label{kundedrevet}}\footnote{Stjernen i overskriften her betegner bare at overskriften er indsat, ikke at hele sektionen er.}
Okay, nu vil jeg så forklare om %en fagforeningsorganisation som et muligt alternativ til en konventionel tech-virksomhed til at udvikle og fremme det semantiske web. ...
min idé omkring en ny type virksomhed, hvor kunderne får magt til at bestemme arbejdernes lønninger. Det vil også generelt høre til idéen, at kunderne sandsynligvis vil tage mere og mere over, også når det kommer til ejerskabet af virksomheden, men det kommer vi til. Som jeg ser det, vil der, særligt med hjælp fra de nye digitale teknologier, på et tidspunkt i en rimelig nær fremtid. Tanken med denne idé er så, om man ikke kunne tage lidt forskud på denne fremtid ved at starte en virksomhed eller organisation, der allerede er ret kundestyret fra starten, og som så sigter imod at blive det mere og mere. For set fra en investors synspunkt må der jo være rigtigt gode penge at tjene på at fremme en virksomhedstype, der (forhåbentligt) kommer til at overtage markedet i fremtiden. Det nye ved denne virksomhedstype er, at investorerne/ejerne fra start af opsætter en politisk proces/struktur, hvor kunderne får magt over lønninger til skaberne/arbejderne alt efter, hvor meget de har bidraget til virksomhedens omsætning, nemlig ved at købe dens produkter og/eller services (og ikke ved f.eks.\ at investere i den). I starten bør kunderne dog ikke have så meget magt over de ansattes løn i virksomheden, men bare de uafhængige skaberes (da idéen antager, at virksomheden starter primært som en tech-virksomhed), der uploader bidrag til virksomheden. For disse typer bidrag skal der gælde, at jo mere en kunde altså bidrager til omsætningen, desto flere stemmer får brugeren altså til at stemme om, hvilken model der skal gives belønninger til bidragsydere ud fra. Iværksætterne af virksomheden sætter så fra start en forskrift for, hvor mange stemmer kunderne opnår pr. de beløb, de løbende betaler som kunder, samt selvfølgelig hvor mange stemmer iværksætterne, eller rettere aktiehaverne, har til at starte med. Iværksætterne sætter i øvrigt også en forskrift for, hvilket afkast aktierne giver til, som altså, ligesom alt andet, kan være afhængigt både af tid og af den samlede omsætning. I øvrigt sættes der også en forskrift for, hvor mange aktier der løbende deles ud til kunderne, for da det jo er tanken, at det faktisk bliver kunderne, der mere og mere kommer til at styre foretagendet demokratisk, jamen så vil det heller ikke være fair andet end at aktierne også med tiden uddeles kunderne --- gerne alt efter hvor meget stemmeret de også har, hvilket altså så i sidste ende kommer til at afhænge af den omsætning, de har bragt til virksomheden (via diverse køb og abonnementbetalinger som kunder). Man kan jo ikke retfærdiggøre, at investorende skal blive ved med at have ret til overskuddet fra virksomheden, når de egentligt ret hurtigt bare bør kunne læne sig tilbage som aktieindehavere og bare lade afkastene (den resterende mængde pr.\ omtalte forskrift), fordi magten og dermed ansvaret for virksomhedens virke alligevel kommer over på kundernes hænder. Hvordan fordelingen så skal være, når aktierne uddeles, det må jo så også være op til en forskrift sat af iværksætterne. Det er nok klogt at uddele disse ret retfærdigt ud fra, hvor meget kunden har spyttet i kassen gennem tiden, men man kunne muligvis også overveje, at udlove lidt ekstra bonus til de første kunder, eller noget i den stil, pga.\ den reklame og hype dette potentielt kan medføre på et tidligt (og potentielt lidt mere kritisk for virksomhedens succes, det må man jo bedømme som iværksætter) tidspunkt. Så iværksætterne skal altså bare (``bare'') udforme alle de her forskellige forskrifter --- samt selvfølgelig tegne de kontrakter, der kan binde dem til at overholde disse forskrifter, så kunderne kan stole på dem --- og de skal også lige strukturere den politiske proces for, hvordan man vedtager ændringer til lønsatsmodellen (og hvornår osv.) --- og så er tanken, at de derefter forhåbentligt ret hurtigt vil kunne lade videre arbejdsbyrder over på kundeskaren og skaberne/arbejderne. Angående at ansætte faste arbejdere i virksomheden (nok primært i starten), så må iværksætterne bare tage højde for dette, og sørge for at overskuddet, der tilgår aktieindehaverne vil være stort nok til at kunne betale potentielle ekstraudgifter, hvis der er behov for mere arbejdskraft her, end hvad var gennemsnitligt estimeret. Det er i øvrigt klart, at det nok ikke vil være klogt at være alt for grådig som iværksætter, for så lokker man nok derved ikke så mange kunder med på idéen, men hvis denne virksomhedstype har så stort et potentiale, som jeg håber på, så vil man sagtens kunne tage sig \emph{rigeligt} betalt som iværksætter/investor.\ \ldots\ Jeg er forresten lige kommet i tanke om ((14.05.21) i skrivende stund), at det at udlove (aktie-)bonusser til de tidlige kunder (med vigtig stemmeretsmagt) baseret på, hvor meget den pågældende delvirksomhed (og jeg kommer til om lidt, hvordan virksomheden kan opsplittes i flere dele), hvor kunderne har deres magt i, vokser og tiltrækker flere kunder og mere omsætning (efter at kunderne har været aktive i at medbestemme lønsatserne), det løser også et problem jeg ellers lidt havde, om at kunderne basalt set bare kunne give lønnen til sig selv (evt.\ bare indirekte via en eller anden lille konspiration imellem en kundegruppe og arbejdergruppe --- eller hvis kunderne selv bidrager med et arbejde i virksomheden). Men ved altså at sikre sig, at de tidlige kunder ikke bare har interesse i at virksomheden leverer dem et godt produkt for fremtiden, men også selv får en interesse direkte i virksomhedens vækst, så kan man jo dermed ikke bare opnå ekstra hype omkring virksomheden ved dette, men man kan endda også bruge det som et middel mod, at der sker stagnation som følge af, at kunderne bare giver lønnen mere til sig selv, end hvad der er gavnligt ift.\ virksomheden og dets produkt. Og da kunderne allerede har en interesse i virksomhedens produkter og services (ellers ville de jo ikke være kunder), så behøver man dog ikke at give de tidlige kunder samme bonus-udsigter som f.eks.\ investorene; jeg tror at en udsigt til moderate bonusser bør være nok til formålet (også lidt alt efter, hvor tidligt vi snakker om, for iværksætterne kunne jo godt beslutte sig for at satse meget på den omtalte hype, dette potentielt set kan generere i starten).

Udover stemmeret omkring lønsatserne og aktier uddelt til brugerne/kunderne og udover afkastet til dem selv og aktiehaverne generelt, så skal iværksætterne jo selvfølgelig også beslutte forskrifter for, hvor meget løn, der gives til arbejderne. Men i modsætning til de andre forskrifter kunne man godt her vedtage, at denne forskrift kan justeres løbende af aktieindehaverne via en politisk proces, hvor aktierne altså giver stemmer til ejeren. Så kan aktiehaverne (hvilket på længere sigt jo kommer til at blive kunderne/brugerne mere og mere, i takt med at aktierne bliver uddelt imod afbetaling af det lovede afkast) altså løbende justere, hvor meget løn arbejderne skal have i alt (og hvor kunderne så dog er med (mere og mere med tiden) til at bestemme fordelingen af denne samlede løn). %Skal jeg bare lige nævne eventuelle omforhandlinger af de andre forskrifter her, og så bare videre til nedenstående punkter (altså dem jeg mangler)? Er der ellers andet..? ..Der er vel det med at give kunderne mulighed for også at shorte aktierne --- dog forsikret imod at kunderne så vinder penge på dette --- som en slags løn.. Ja, fair nok.. (tjek)
Der er også endnu en forskrift-kontrakt man kunne inkludere i denne type virksomhed, som jeg har tænkt på, og det er at inkludere en forskrift for, hvor meget tidsafhængig løn, kunderne må stemme om at uddele til arbejderne. Nærmere bestemt tænker jeg, at man muligvis kunne gøre plads til (begrænset med en forskrift fastsat fra start af), at kunderne i bund og grund kan shorte virksomhedens aktier som en løn til arbejderne, hvorved den sande løn så kommer til at afhænge af, hvor meget aktierne stiger værdi under perioden inden at de pågældende arbejdere skal sælge aktien tilbage. Aktien skal altså så sælges tilbage ved en fast løn, på nær dog hvis aktien rent faktisk er faldet i værdi, for da hele dette er beregnet som en \emph{løn}, er det jo nok mest fornuftigt og rimeligt, hvis man forsikrer lønmodtageren, så denne ikke kan ende med at tabe penge på processen. 

Hvis man som iværksætter er bange for at komme til at sætte forskrifterne uhensigtsmæssigt fra start, så kan man sikkert godt finde nogle måder, hvorpå man får mulighed for at genforhandle forskrifterne imellem de forskellige parter. En idé jeg kunne forestille mig til at opnå denne mulighed, kunne måske være, hvis man indlagde en procedure, hvormed kunderne kan aktivere en genforhandlingsproces, og hvor denne procedure så skal være kostelig for kunderne (når det kommer til besvær og/eller penge). Et eksempel kunne være at kunderne fik mulighed for at igangsætte et planlagt boykot, hvor genforhandlingsprocessen så først bliver udløst efter at boykot-processen officielt er igangsat, hvis det lykkes kunderne at reducere omsætningen med en vis anseelig mængde. Genforhandlingsprocessen kunne så indebære en på forhånd fastlagt politisk proces, hvor parterne stemmer om nogle nye forskrifter. Man kunne så eventuelt lade kundernes samlede stemmemagt afhænge af, hvor dybt det lykkedes den at reducere omsætningen, for dette vil ligesom gøre, at kunderne måske bare kan nøjes med at markere deres magt over for de andre parter, således at omsætningen dermed ikke nødvendigvis behøves faktisk at reduceres helt vildt i processen, før at de andre parter kan give sig i de efterfølgende forhandlinger. (Dette er i modsætning til, hvis kundernes stemmemagt var fastlagt, og at reduktionsprocentdelen også var fastlagt, hvormed kunderne jo død og pine bare skulle ramme denne reduktionsprocentdel. Men hvis man giver stemmeretsmagt alt efter procentdelen, så kan man nok opnå en mere naturlig og dynamisk forhandlingsmagt til kunderne.) Og hvorfor i alverden skulle man på forhånd give magt til kunderne om at kunne bevirke, at forskrifterne gøres mere fordelagtige for dem selv? Tja, det er der måske heller ikke nogen mening i, men tanken er altså, at man muligvis herved vil forsikre sig imod, at brugerne/kunderne bare trækker stikket på virksomheden, hvis nu nogle uheldigt satte forskrifter bliver en torn i virksomhedens side (og dermed i kundernes tilslutning). Så ved at åbne op for, at kunderne bare kan besvære sig selv nok i stedet for at få forskrifterne genforhandlet, så kan man muligvis forhindre at kunderne i stedet simpelthen besværer sig helt og begiver sig for at opbygge en helt ny virksomhed fra grunden. Så tanken er altså at sætte en stopklods for en sådan proces ved i stedet at have en indbygget proces, hvor kunderne kan opnå deres vilje, hvis denne bare er stor nok, men til mindre besvær og omkostninger for begge parter (end hvis de altså boykotter og sænker virksomheden helt for at starte end ny). \ldots Hm, eller ved nærmere eftertanke er det jo nok bedre, hvis man bare giver investorene magt til at ændre forskrifterne, men \emph{kun} i kundernes favør. For så kan alt dette med at boykotte osv.\ bare ske på kundernes helt eget initiativ for at lægge pres på aktieindehaverne om at stemme for at ændre forskrifterne således. Ja, det giver klart mere mening sådan.

Jeg mangler nu særligt at beskrive mere præcist, hvad den politiske proces om at sætte lønsatserne kunne indebære, og jeg mangler også, som nævnt, at forklare om, hvordan jeg mener, at virksomheden løbende bør kunne opsplittes i forskellige afdelinger. Lad mig starte med den politiske proces. Jeg er kommet lidt frem til, at den overordnet set bør være ret ``normal,'' nemlig hvor man jævnligt har nye valg og hvor man så stemmer en model ind ved i sidste ende at vælge den, der får flest stemmepoint iblandt finalisterne. Jeg mener, at man fordel (da hele processen selvfølgelig bør være digital, og fordi kunderne således også bør have mulighed for at automatisere deres stemmeafgivninger (også så de kan afhænge af afstemningsforløbet)) kan have mange runder i afstemningen, hvor man i hver runde kun udelukker en vis nedre procentil (f.eks. det nedre kvartil eller en mindre procentil (er egentligt ikke helt sikker på at `procentil' er det korrekte term, men nu bør der ikke være forvirring; det der svarer til et kvartil med måske med en anden procentdel i stedet for 25 \%)), hvorefter folk får mulighed om at stemme på ny om de tilbageværende. Det folk stemmer på er jo så som sagt en lønsatsmodel. %(16.05.21) Jeg fik denne idé nu her, imens jeg skrev teksten (og nu kan jeg med det samme se, at den faktisk er ret vigtig):
Her kunne man så med fordel måske lade det være helt frit, hvordan en sådan model kan konstrueres, og hvor mange parametre der kan bruges i den, og så kan man i stedet bare sige, at parterne selv kan bestemme deres stemme i form af en fordelingsfunktion. På denne måde bliver metrikken i parameterrummet underordnet, selv hvis der bruges reelle tal (hvad der naturligvis vil gøres med al sandsynlighed). For hvis parterne stemmer via en fordelingsfunktion, så kan folk jo bare selv tilpasse fordelingen, så den passer med, hvad end metrik den givne variable model bruger. Det samlede model-rum bliver altså så bare et rum over alle strukturelle forskellige modeller (der dermed sagtens kan have forskelligt antal variable), så geometrisk bliver der derfor tale om en forening mellem mange parameterrum med forskellige dimensioner. Hvis to modeller er struktureret forskelligt, men er ækvivalente, når det kommer til lønsatserne, så kunne man jo måske enten bare sige, at det så er op til de interesserede parter at vælge en af dem, så stemmefordelingen nemmere kan nå op på en pointværdi, der overlever runden\ldots\ Tja, men her har vi så et problem, for ``pointværdien'' kommer så til at afhænge af metrikken\ldots\ 
%Uh, lige en indskudt bemærkning (imens jeg venter på at finde et svar): Jeg kom også lige til at tænke på, hvor langt man egentligt kan komme med en så åben organisationsstruktur som denne, også når det kommer til partier osv. For alt sådan som som gennemsigtighed osv. kan jo også bare komme ved, at folk begynder at inkludere det i de modeller, de stemmer på. (16.05.21)
\ldots\ Hm, kunne man gøre noget med, at folk efter hver runde hver især alle skal søge at forøge indre produkt (altså det typiske indre produkt, der bruges for funktioner som vektorer, eksempelvis i kvantemekanik osv.) *(og hvor fordelingerne så skal normeres ift.\ vektornormen) med de andre fordelinger summet sammen ved at justere sin fordeling så det kommer over en vis værdi, hvis det altså som udgangspunkt ikke gjorde dette?\ldots\ %Hm, eller kunne man ikke gøre noget med, at alle folk kan vælge en farve..? Hm... 
\ldots\ Nej, en bedre løsning er måske i stedet, at gøre hele stemmeprocessen dynamisk sådan at den hele tiden er i gang, og hvor hver deltagende part så bare kan sætte og løbende justere et kraftfelt i model-parameterrummet, hvor der så bør være en maksimal kraftstørrelse, som deltageren kan sætte i hvert punkt (hvad man jo så typisk altid vil vælge (med mindre måske punket er tæt på ens mål, og man bare har lyst til at stabilisere modellen omkring det punkt)). *(Nå ja, men hvis deltagere kan hæve ``flere stemmer,'' så kan dette jo med fordel implementeres med, at de så får en større maksimalkraft, de kan sætte i deres kraftfelt.) For med disse regler bør metrikken heller ikke betyde noget, andet end altså at den kan betyde noget for forandringsfarten, men det går jo nok. Der er jo så det ved det, at der er flere modeller i spil, hver med deres eget parameterrum, men så tænker bare, at deltagerne i princippet definerer deres kraftfelter for alle de nogenlunde populære modeller (og hvor ikke-definerede kraftfelter, eksempelvis for upopulære modeller, man ikke gider at bruge tid på, bare sættes til nul pr.\ default). Og så opstiller man bare alle modeller i deres eget overordnede parameterrum (med $N$ antal dimensioner, hvor en er antallet af mulige modeller), hvor parametrene går fra 0 til 1, og hvor man så definerer en eller anden lovmæssighed om, hvornår en model træder i kraft, medmindre man i stedet finder en eller anden måde at tage et gennemsnit af et antal vilkårlige modeller. \ldots Tja, man kunne jo bare sige, at den mest populære model gælder, og hvis man så vil undgå, at man kommer ud i, at modellen så kommer til at veksle frem og tilbage mellem to tilstande, som har markante (diskrete) forskellige i sig, så kan man jo bare stemme en tredje model på banen, som er en blanding af de to (ved så netop at definere en måde at blande dem), hvor blandingsforholdet så afhænger af en ekstra parameter (og når vi taler lønsatser, kunne dette jo f.eks.\ bare defineres ved at tage et gennemsnit af de to modellers lønfordelinger vægtet med denne ekstraparameter). 

Hvem skal så kunne tilføje nye modeller til den samlede mængde? Det skal alle i princippet kunne. Man kan dog lige sætte en begrænsning på, hvor mange modeller én person kan foreslå i et vist tidsrum, så man undgår spam. Hvordan sørger man får, at modellerne bliver fulgt? Dette skal faktisk indgå i selve modellen, mener jeg. Man kan måske også bare se det (og/eller implementere det) som en separat stemmeproces, men pointen er i hvert fald bare, at det også skal bestemmes demokratisk i dette system, hvordan systemet skal holde kontrol med ledelsen, der skal implementere modellen i virkeligheden. Lad os kalde denne del af det for kontrolmodellen. Så skal man bare sørge for som organisation, at der er en kontrolenhed med magt til at fyre ledelsen (og evt.\ sanktionere den på anden vis), og som så er juridisk bundet til at følge organisationens indstemte kontrolmodel (i rette sig efter ændringer i den). Man kunne også have sprunget kontrol-model og -enhed over, man jeg tror, det vil være mere stabilt at faktorere det ud sådan her, for hvis man har en kontrolenhed, hvis medlemmers løn kun afhænger af (og hvor potentielle sanktioner afhænger af), om man følger modellen eller ej, så tror jeg. dette vil være mindre korrumperbart. For jeg forestiller mig nemlig, at en organisationsledelse, der har fingrene direkte i alle mulige sager, lettere vil kunne komme til at befinde sig i situationer, hvor der er interesser og kræfter, der frister dem og trækker dem i en anden retning. Og når vi snakker en organisation som denne her, hvor ledelsen der sætter lønnen, vil skulle være i et tæt samarbejde med aktiehaverne, som skal overføre den løn, så\ldots\ Hm, whatever\ldots\ Det kan være, jeg har ret i denne indskydelse om, at det altid nok vil være godt at bruge en separat kontrolenhed til at styre ledelsen, men det må man jo bare lige overveje, om jeg har. %\ldots Hm, men jeg tror nu, den er god nok: Pointen må nemlig være, at man herved så bare kan nøjes med trusler om fyrringer, når det kommer til den samlede ledelse (inklusiv kontrolenheden). Men hvis ikke man havde en kontrolenhed, så skulle man alligevel blive enige i det brede fælleskab, hvornår man kan og skal fyrre ledelsesmedlemmer, og så er man jo i princippet ikke videre i den demokratiske proces, end hvor man startede. Men hvis man i stedet bruger en kontrolenhed, der følger en model for, hvornår der skal fyres eller ej, så skal man kun blive eninge om\ldots\ Hm... %Nej, lad mig lige prøve at tænke dette her om.. ...Eller måske bare lad det stå åbent, som jeg også lagde op til.. (ja)
\ldots\ Ja, der kan være mange måder at prøve at sikre sig, at der altid vil være en ledelse, der forsøger at følge den indstemte model så godt de kan, så den del af det må man bare prøve at løse så godt som muligt som iværksætter. Om dette så vil indebære flere moduler i den samlede ledelse, så forskellige moduler har magt og interesse i at holde hinanden i ørerne, det må man jo så se på.

*[Jeg bør faktisk også lige nævne, at det generelt vil være fordelagtigt, at have så mange parametre, som deltagerne overhovedet kunne ønske for modellerne, således at folk kan trække modellen i helt den retning, de gerne vil. Modeller, som unødvendigt blander to dimensioner sammen ved at have for få parametre, kan derfor med fordel nedstemmes af fællesskabet (til fordel for mere brugbare modeller med flere nyttige parametre). Jeg bør forresten også lige uddybe, at det selvfølgelig ikke er sikkert, at ledelsen kan lade sig diktere fuldstændigt af modellen (og det er nok meget godt\ldots). For eksempel når det kommer til denne idé om en kundedrevet virksomhed/organisation, så er det klart, at der i første omgang er nogle fastlagte rammer for, hvad der kan bestemmes af modellen (i dette tilfælde en vis andel af den samlede lønfordeling), og sådan vil det nok også være i alle mulige andre organisationer, man kunne forestille sig benytte denne form for afstemningsproces. Bare så det lige er helt på det rene.]

Efter at have skrevet dette kan jeg jo godt se, at en sådan høj-dynamisk og alsidig demokratisk struktur kan være brugbart inden for andre områder en lige denne idé. Et indlysende eksempel ville jo være et politisk parti, og jeg er sikker på, at man ville kunne finde på mange flere gode eksempler. Men for at vende tilbage til min idé om en mere kunde-drevet virksomhedsorganisation, så skal jeg altså også have forklaret, hvordan jeg mener, at organisationen bør kunne opsplittes løbende til forskellige fagområder. Jeg har tænkt lidt over det, og jeg tror bare, man bør vedtage, at enhver fornuftig opdeling, hvor produkter og services kan splittes i to grupper, og hvor et flertal, eventuelt med en vis procentmargin, af kunderne/forbrugerne af én gruppe (hvor forbrugernes stemme altså her også er vægtet ud fra, hvor stor en del af omsætningen de bidrager med) gerne vil have denne opsplitning (efter at der er gået en vis tid siden at produkterne servicesne kom på markedet), jamen så må man udføre en. Under en opsplitning splittes aktierne for de to områder også, men hverken kunderne eller investorerne tvinges dog til at bytte nogen af de resulterende aktier med hinanden (så alle vil derfor eje lige meget før og efter opsplitningen). Derudover vil forskrifterne under sådan en opsplitning så deles i to, sådan at kunder, der køber fra et af områderne men ikke rigtigt fra det andet, nu altså kun vinder stemmer indenfor dette område. Så iværksætterne skal altså fastsætte en protokol for, hvordan forskrifterne opsplittes herved --- selvfølgelig på en måde, så kunder der handler lige meget inden for begge områder ikke ville mærke personlig forskel, hvis de opsplittede virksomheder ellers fortsatte ret meget ligesom før. 


Jeg vil også lige nævne, at selvom arbejderne/skaberne bare gives løn og ikke umiddelbart gives anden magt i denne struktur, så kan de jo stadig altid forene sig i fagforeninger osv., hvis de vil have deres stemmer mere hørt. I den forbindelse kan jeg også nævne, at selvom den umiddelbare tanke er, at arbejderne videregiver rettighederne for det produkt, de skaber, til organisationen, så ville det heller ikke være helt dumt, hvis arbejderne/skaberne på visse punkter kunne holde fast i rettighederne til dele af deres produkter, især når produkterne er ret selvstændige (såsom alle kreative produkter (billeder, videoer osv.) som et godt eksempel), sådan at en eventuel fjernelse af indholdet fra en platform ikke vil ødelægge andre produkter. For hvis arbejderne/skaberne kan dette, så kan de bedre beholde en vis forhandlingsmagt i det store hele, og kan bedre sikre at deres interesser bliver retfærdigt tilgodeset, på trods af at de altså ikke ellers har nogen magt pr.\ organisationens indre politiske system (i hvert fald i denne version, som jeg nu foreslår; man kunne jo godt forestille sig andre versioner, hvor arbejderne også får stemmer tildelt i de politiske forhandlinger (men jeg ville nu nok satse på bare at undlade dette (men bare i stedet lægge op til, at de selv for dannet fagforeninger))).  

Og hvad er de markedsmæssige fordele ved at danne sådan en organisation som denne? Det er at kunderne/brugerne, mener jeg, hellere vil bakke op om en virksomhed, hvor de selv får medbestemmelse over produkterne, og som er socialt bæredygtig, fordi ejerskabet med tiden vil fordeles ud på kundernes hænder, end mere ugennemsigtige, topstyrede og ikke-socialt-langtidsbæredygtige virksomheder. Og dette, tror jeg, gælder især for virksomheder, hvor kunderne i forvejen i høj grad er med til at bidrage til og med til at forme produktet. Og desuden, hvis kunderne får mere magt over lønningerne, kan de også bedre få indflydelse til at guide virksomheden til at lave det produkt / de produkter, som passer bedst med kundernes interesser, hvilket altså også er en grund til, at brugerne nok vil bakke meget op om sådan en virksomhedstype.


*(12.06.21) Man kunne i øvrigt muligvis forstærke denne idé endnu mere ved at bruge mine lønsatsmodel-principper, som jeg fandt på sidst i \textbf{*Yderligere tanker omkring kryptovaluta}-sektionen nedenfor (i paragraferne fra d. 10-11/06, og særligt 11/06-paragrafen, selvom den godt nok bare står i kommentar-brainstorm-form pt.). Med disse idéer føler jeg nemlig endelig, at jeg har et muligt fornuftigt svar på, hvordan man kan sikre sig en fair lønsatsmodel, hvor folk kan forvente ikke at blive diskrimineret (f.eks.\ af den næste generation af medlemmer/deltagere), men hvor systemet stadig er til at implementere (uden alt for meget besvær). Overordnet set går idéen ud på, at binde sig til at give lønninger ud fra en model, der kun bruger atomiske semantiske parametre, udvalgt abstrakte nok til, at man simpelthen ikke kan diskriminere på baggrund af specifikke omstændigheder omkring bidragene, men kun kan se på mere overordnede forhold, omkring hvor meget gavn/lykke de medførte, hvor meget besværlighed og misbehag de medførte, til hvor mange mennesker og i hvor lang tid. (Og man kan så i øvrigt eventuelt have en konstant parameter, der giver en belønningsfaktor, hvis bidraget var med til at fremføre selve systemet, men det er en anden snak.) Pointen, der gerne skulle gøre systemet lettere at implementere, er så også, at man så løbende stemmer omkring lønsatsmodellen, 
men dog er bundet til at indføre en justering med tilbagevirkende kraft, hvis man justerer en parameter opad ift.\ den løn parameteren medbringer. Man kan altså godt ændre lønsatsmodellen også med tilbagevirkende kraft, men kun så den bliver mindst ligeså gavmild (eller mere) end den nuværende. Og tanken er så videre, at man som fællesskab så får mulighed for at udbetale lønninger til det relevante tidspunkt for at tiltrække bidragsydere, hvorved man så bare (til en vis grad) bliver tvunget til efterfølgende at sørge for, at modellen til det pågældende tidspunkt altid vil være gavmild nok til at kunne retfærdiggøre alle de her tidlige lønbetalinger. Man må dog godt give plads til, at der kan være en vis mængde outliers i denne forbindelse, således at en enkelt forkert udbetalt lønning ikke kan komme til at blive skæbnesvanger for hele resten af de tidlige lønninger i systemet. Man kan altså godt tillade sig selv en vis margin for afvigelser samlet set, men gennemsnitligt set skal lønsatsmodellen altså justeres (eksempelvis med tilbagevirkende kraft), så diverse tidlige lønbetalinger generelt passer med modellen. Og hvis folk i øvrigt så ender med at have fået for lidt udbetalt i sin tid, så skal det resterende beløn erstattes, så alle altså som minimum får, hvad modellen siger. Da modellen dog som antydet kan ændres løbende, så vil det altså være smart i denne forbindelse, hvis man kan udstede restbeløbet i form af en aktie, og så love som fællesskab, at man løbende vil købe sådanne aktier ind igen, eventuelt efter en vis fast tid (og hvis der også er økonomi til det), til hvad kursen nu står (ifølge lønsatsmodellen) til på pågældende tidspunkt for tilbagekøbet. Da jeg fandt på denne idé nedenfor, var det i forbindelse med et kryptovalutasystem (og særligt et med ``NL-fortolkninger'' i, som jeg kalder det her i disse noter), men principperne kunne også overføres på andre systemer, hvor man så f.eks.\ bare binder sig til alle disse forskellige ting som systemadministrator-enhed på gængs juridisk vis. Og fordelen ved at implementere disse principper i sit system, f.eks.\ i sin kundedrevne virksomhed, er altså, at man så binder sig til at belønne bidrag på en fair måde, hvilket så også gør (som jeg lige var lidt inde på), at folk kan bidrage mere åbent til fællesskabet, og uden nødvendigvis først at registrere bidragene på en bestemt måde eller at skulle indgå visse aftaler først, der kan sikre dem, at de får løn som fortjent. Dette kan altså muligvis lette en hel masse arbejde både for bidragsyderne og for administrationsenheden i det tidlige (og kritiske) stadie af systemet (fordi man så nok bare kan udskude lønningsvurderingerne af de fleste tidlige bidrag til det fremtidige fællesskab). Nå ja, og det hører så forresten med, at brugerne dog skal registrere deres bidrag i et aller andet omfang, så de kan udstede værdipapirer på baggrund af dem og derved få mulighed for at købe sig af med deres risiko på disse værdipapirer.

*(13.06.21) En tillægsidé omkring den dynamiske politiske model kunne i øvrigt være, at man inkluderede muligheden for at, der løbende kan erklæres fakta omkring den virkelige verden inde i modellerne, sådan at stemmererne kan få mulighed for (hvis de har tillid til korrektheden er, hvordan disse fakta erklæres) at ikke bare at sætte et kontant kraftfelt, men også et variabelt kraftfelt, der kan afhænge af, hvad der bliver erklæret. Stemmerer ville så hermed få mulighed for at indstille automatiske justeringer af deres stemmekraftfelt, som bliver udløst af visse begivenheder. Det gode ved denne idé er, at den, selvom den måske virker lidt kompliceret, faktisk ikke koster noget, for den giver jo bare muligheden for at brugere i endnu højere grad kan opnå, at deres stemmekraftfelt passer sig selv. Og da det nok kan være smart, at fastsætte metamodellen (i.e.\ den ydre model for alle de mulige politiske modeller) ret tidligt, så man ikke skal omformulere en masse kontrakter osv., så ville det nok være en meget god idé at gøre metamodellen åben for denne funktionalitet. Det eneste man nok lige skal tænke over i denne forbindelse, er nok, hvorvidt systemet kan være garanteret at have tilstrækkelige resurser til at regne disse ting ud, men ellers kan jeg ikke se, at der kunne være andre ting galt med at tilføje funktionaliteten (for det åbner som sagt kun flere muligheder for deltagerne, og de kan altid bare ignorere funktionaliteten, hvis de ikke finder den nødvendig). Nå jo, og så skal man i øvrigt lige overveje, hvordan man vil indrette systemet, der fører til opslåede erklæringer i systemet, og om man eventuelt fra start vil binde sig til at noget fast i denne forbindelse, så deltagerne får krav på, at man med jævne mellemrum opslår visse typer erklæringer --- hvilket eksempelvis kunne være ud fra en proces, hvor brugerne også får lov at stemme om, hvilke nogle typer erklæringer skal opdateres i metamodellen. Så ja, det kan blive en smule kompliceret, men man kan jo om ikke andet bare sørge for, at holde alle disse muligheder åbne i systemet, så man kan træffe alle beslutningerne, når de er relevante. 

%
		%Opsplitning. (Hvis der er nok kunder, der er for en splitning, hvor de selv tilsammen udgør en stor del af den ene gruppe, så må det bare blive sådan. Forskifterne bør så ikke afhænge af opsplitninger; den bør bare se på de individuelle omsætninger.) (tjek)
		%Politisk proces. (tjek)
		%Eventuel grunde til succes.
		%Fagforeninger (og at disse eventuelt prøver at holde lidt på IP-rettigheder). (tjek)
		%Men jeg skal vel lige tænke lidt mere over, hvordan man sikrer sig imod grådige kunder, der finder en måde at donere til sig selv effektivt set... (For man kan da ikke sige, at skabere ikke må være kunder eller omvendt?..) ... Uh, jamen dette problem løses da bare ved, at man netop giver bonusser til tidlige kunder, hvis de får del-virksomheden til at udvikle sig (og får tiltrukket flere kunder). På denne måde har kunderne selv interesse i delvirksomhedens vækst, hvis/så længe der er mulighed for dette. Og når der ikke vurderes at være det, jamen så vil denne "grådige" adfærd jo bare lige pludselig være en måde at nedjustere priserne kunstigt på fra kundernes side (ift. hvad den pågældende forskrift ellers har sat af til det). (14.05.21) (tjek)
		%..Og man bør næsten også lige overveje nogle måder, hvor investorer og kunder kan gå sammen om at ændre politikker lidt, og/eller måder, hvor forskrifterne kan have lidt rum til ændring.. måske via fælles afstemninger.. ..Tja nej, man må bare sætte noget fornuftigt fra starten af.. eller vent.. Hm, man kunne måske lave noget med at sige, at hvis kunderne kan "strække" (eller boycotte) og sænke omsætningen med en vis procent, så kan det udløse muligheden for, at forskifterne kan omforhandles ud fra en eller anden fastlagt forhandlingsproces.. ..Tja, jeg kan jo nævne den tanke, men så ellers bare lade spørgsmålet stå lidt åbent ellers.. (tjek)







%(12.05.21) Ah, der er faktisk et uafsluttet spørgsmål omkring fagforeningsidéen, så vidt jeg kan se.. For én ting er, at man kan underskrive fortrolighedsaftaler inden man køber bidrag (hvilket man kan gøre i alle tre løsninger/muligheder), men for fagforeningsløsningen er det jo vigtigt, at arbejderne beholder magten over deres bidrag. Og problemet er jo så, at man dog heller ikke kan have det sådan, at arbejdere bare kan banlyse brugen af vilkårlige bidrag, for der kan jo være andre bidrag, der bygger oven på disse, og det ville så skabe et meget usikkert og kaotisk forhandlingsgrundlag.. Jeg har så tænkt på, om man kunne gøre noget med, at foreningen er juridsik og/eller foreningspolitisk bundet til at erstatte udtrukne bidrag på en måde, hvor folk, der ikke kender noget til løsningen, skal prøve at opfinde den på ny, men dette er bare også svært at få til at du. For det vil så både blive meget svært i visse tilfælde, at finde uvidende folk (og at verificere denne uvidenhed), og derfra bliver det også svært formelt set at definere, hvad deres promt til at genopfinde løsningen må indeholde (for nogen idéer giver jo nærmest sig selv, når man først for defineret målsætnigen, og så er det det sidstnævnte, der egentligt er det vigtige ved idéen). ...Hm, kunne man gøre noget med, at forening og skaber/arbejder forhandler, hvad denne promt skal være, når løsningen registreres og optages første gang i foreningen (så måske netop forhandlet bag (komplet) fortrolighedserklæringer)..? Tjo måske, men dette løser dog ikke det førstnævnte problem.. ..Hm, men kunne ikke bare forhandle en pris for bidraget til at starte med, som.. Hm... ... Kunne man eventuelt modificere den kunde-orienterede løsning, så man sikrer arbejderne en anelse bedre, eller så de bare får mere at skulle have sagt? (Ikke at jeg er vildt opsat på at ændre den idé umiddelbart, men..) ...Hm, eller kunne man gå tibage til tanken om at man måler opfindelsestiden ved at vædde om den?.. ..Hvor køberne estimere et bud på opfindelsestiden, og hvor at skaberen så, hvis denne er utilfreds så har ret til at udfordre denne.. Hm, men problemet er dog, at.. nå nej, for tiden måles jo bare til, hvornår næste skaber tager kontakt, og ikke til hvornår den bliver offentliggjort. Hm, men hvad så med to idéer, der er ret ens men ikke 100 \%? Tjo, men dette kunne man vel godt gøre til en del af forhandlingen, kunne man ikke? Hvis man først forhandler om løsningsdefinitionen, sådan at køberen i første omgang går med til en eller anden (ikke alt for præcis, hvis køberen kan sørge for dette) definition, inden at køberen så går med til at sætte et tidspunkt, som denne så hermed tvinger sig selv til at vædde om, skulle skaberen afslå den tilbudte løn. ... Tjo, det må da kunne lade sig gøre..!..? I så fald vil det være en ret god idé, som man kan krydre (nok) enhver af mine løsninger med.. (12.05.21)
%(13.05.21) Jeg skal lige forklare, inden jeg går videre, at tanken med at kunne sætte denne opfindelsestid så er... Hm, nå ja, der var jo også et problem med denne idé i form af, at hvad så når idéen bare ikke er interessant nok for andre til, at de vil arbejde på den af sig selv, og at opfindelsestiden bare dermed bliver længere... Hm, men er det ikke netop noget af det, man så også gerne vil belønne?.. Skal lige samle tankerne lidt.. ..Hele idéen med denne løsning --- og jeg tror btw muligvis, den kommer til at smelte sammen med kunde-løsningen, men det må jeg lige se --- er bare, at arbejderne får mulighed for at beholde en magt over deres bidrag.. Men ja, det holder bare ikke helt, for det bliver meget svært at finde frem til nogle gode rettigheder, der er til at forhandle med.. medmindre altså jeg finder en eller anden god idé til det... (..og hvis man så finder en god idé, så kan den så bruges i alle løsninger --- og så kan man nok lidt bare se det som kryderi til resten og ikke som sin egen løsning nødvendigvis...) Hm.. ... Tja, og hele grundlaget for idéen var så nemlig, at man så kunne bruge eksisterende IP-lovgivning.. Tjo, men man kan jo stadig godt have interne bestemmelser for at fordele magt imellem bidragerne.. Men det duer så bare ikke rigtigt (vistnok), hvis de individuelle bidrageres forhandlingsmagt ligger i, at de kan forlade foreningen ("med deres bidrag" :s).. ...Hm, jeg skal i øvrigt også lige genoverveje, om der ikke kunne være en mulighed i en KV-system-bevægelse, hvor arbejderne får udstedt deres bidrag som tokens... Tja, nej.. Hvis man gerne vil ære tidlige arbejdere, så vil kunde-løsningen jo også ende med, at gøre dette.. Så, men derfor ville det stadig være dejligt nok, hvis man kunne sikre individuelle arbejderes forhandlingsmagt lidt bedre.. 
%Jeg tror nok, det bedste bare er at sige, at det er kunderne, der bør samle IP-rettigheder til sig (og investorerne i første omgang, det er fint). Og så må man vel bare bruge fortrolighedskontrakter til at handle med skabere/opfindere, der er bange for ikke at få nok ud af deres bidrag, hvis de bruger de normale kanaler, hvilket er bare at videregive sine IP-rettigheder til den kundedrevne organisation imod at blive belønnet pr. denne organisations (lønsatsmodel-)løfter.. Og jeg fik vist ikke nævnt dette, men arbejdere kan jo altid skabe fagforeninger til at opnå et forhandlingspres på kunderne, når disse skal sætte deres model, og bør i øvrigt også være frie til at gå sammen og forsikre hinandens arbejde, så man kan sikre sig yderligere imod, at kunderne (eventuelt i et ret grønt lønsatsvurderingssystem) kommer til at fordele lønnen skævt og unfair. Er det så bare det?.. Ja, det er det vel egentligt (13.05.21).. (Og hvis skabere \emph{kan} beholde IP-rettigheder til deres bidrag på en måde, så de kan forhandle med dem, jamen så må de jo endeligt bare benytte dette, og gerne også bl.a. til at kunne medbringe mere forhandlingsmagt til en fagforening (større eller mindre) eller anden interessegruppe.) 






%Nu kan vi så vende tilbage til min ``alternative løsning'' (til bare at bruge konventionelle virksomheder til at varetage og katalysere udviklingen af det semantiske web m.m.), som altså omhandlede en vis økonomisk bevægelse, og se på hvad denne går ud på mere præcist. Som sagt handler den i bund og grund om at prøve at tage lidt forskud på denne fremtid med dens retfærdige, centralt forhandlede lønsatsmodel. Men ja, som sagt er det altså ikke sikkert, at idéen er bedre end bare at have en mere konventionel (tech-)virksomhed omkring udviklingen af det semantiske web. Tanken er, at man har en organisation, der allerede fra start sætter sig for at danne en lønsatsmodel til at signalere, hvilken løn diverse skabere/bidragsydere(/programmører/sprogudviklere/\ldots) skal have, hvad man jo egentligt også ville skulle gøre i en normal tech-virksomhed, men hvor man dog her også åbner op for en smule demokrati (og mere og mere som tiden går) for, hvordan lønsatserne sættes. Samtidigt skal en sådan organisation så også sørge for at danne en juridisk ramme, der sikrer at bidragsydere kan begynde at stole på, at hvis de melder sig selv til ordningen, kan deres bidrag faktisk blive belønnet ud fra, hvordan denne lønsatsmodel udvikler sig i fremtiden --- samt også hvordan verden generelt udvikler sig og altså særligt så hvad det specifikke bidrag så kommer til at se ud til at have betydet for denne udvikling. Organisationen behøver ikke at gøre lønsatsmodellen helt demokratisk fra starten af, men kan sikkert med fordel faktisk i stedet følge nogle (officielle, og gerne også juridisk bindende) principper/målsætninger, så man uddeler stemmeret ud fra, hvad man ser som de mest vigtige bidrag til organisationen. Disse vurderinger skal så også ske ud fra selv samme lønsatsmodel, hvor ``vigtigheden'' altså bliver sat lig med den løn, som bidragsyderen (og dette er i øvrigt inklusiv investorer (så ``løn'' kan altså også betagne, hvad man normalt måske ville kalde afkast/fortjenester)) har i vente ifølge modellen. 
%Det hører så med til idéen, at man dog fra start indgår en klar fællesaftale (som også nye medlemmer skal skrive under på), om at parametrene bag lønsatsmodellen aldrig må være tidsafhængig, så de diskriminere imod en genration af bidragsydere, eller på anden vis være diskriminerende. ...
%Når de tidlige investorer i en sådan organisation så ...








%Brain: *(fortsat fra "... har i vente ifølge modellen.")
%Og idéen er jo så bare at lave en organisation, hvor man søger at udbygge en sådan lønsatsmodel på demokratisk vis allerede fra starten, og hvor man så allerede begynder at udvikle og udstede værdipapirer, som kan tage højde for, hvordan denne model udvikler sig, samt selvfølgelig hvordan forholdene omkring hele bevægelsen og omkring, hvad end specifikke bidrag der er tale om, udvikler sig. Og hvor man altså derved allerede fra start begynder at uddele værdier, og særligt også altid hvis org'en skulle gå nedenom og hjem, så kapitalen uddeles med en vis fordeling i så fald. Og så gør man det sandsynligvis altså bare ikke helt demoratisk fra starten, men sørger for at stemmeretter omkring modellen også bliver fordelt, så det passer med interessen i den succes. Og man vil altså således også bevare en rimelig mængde stemmeretsmagt hos investorene i org'en også.. Ja, og fordi man jo antager, at man også vil ende ud i et liberalistisk system, som altså vil være motiveret fra start til slut i at værdsætte løfter og (særligt) kontrakter med tilhørende værdier på spil, så kan man altså også sagtens sælge sin risiko på de "bidragstoken"-værdipapirer, man får via sine bidrag til org'en (ikke at der kun nødvendigvis er én org; der kan sagtens være flere, som så bare i sidste ende, pr. deres retningslinjer, bør komme til at arbejde sammen, ja og endda smelte sammen på en måde). (Og ja, som antydet kan disse retningslinjer så følges i høj eller mindre grad; det kan ses som et spektrum, hvor en mere konventionel virksomhed så er i den anden ende (men så sandsynligvis alligevel nok kommer til på et tidspunkt at blive presset til at gå mere og mere med på samme retningslinjer).)
%(05.05.21) Jeg bør i øvrigt også nævne, hvordan man nok godt kan regne med at "hvor meget lykke/velfærd handlingen i gennemsnit medfører" bliver en vigtig parameter for den fremtidige model. Og nævn i øvrigt, at man nærmset kan deducere bevægelsens succes (teoretisk set), hvis den først kommer i gang..
%Har lige skrevet "... har i vente ifølge modellen" her ovenfor, men nu har jeg det så lidt alligevel (igen, igen..) som om, der er noget jeg har overset.. Hm.. hvis der alligevel er en risiko for at de første investorer, nu bare hvor man tæller de første arbejdsinvestorer med også, bliver grådige.. hm, hvad er så pointen til forskel fra en mere konventionel virksomhed..?... ..Tja, den enseste forskel er vel bare, at investorene i bevægelsesorganisationen i modsætning til ellers ligesom lover nye potentielle bidragsydere at de vil få en større del af kapital-(og stemmerets)kagen, hvis modellen skulle udvikle sig i den retning.. Hm, men selv samme kapitalhavere, der så ikke har interesse i, at modellen skal udvikle sig i den retning.. (er dem, der styrer den mest,) men der er dog lige den forskel, at modellen jo så allerede kan indeholde parametre for, hvordan den udvikler sig.. Der skal jo, som jeg har fastlået før, netop være noget inerti i den.. ..Ja, det skal der jo være, men bliver det så ikke bare det samme som at benytte nogle fremtidsafhængige aktier, som bidragsyderne kan opnå?.. Ja, så jeg skal altså lige slå ned på, hvad der overhovedet er smart (potentielt) ved at bruge en mere central model i stedet for bare at udstede individuelle aktier til individuelle bidrag.. ..Tja, i princippet er det vel så bare, ud over at det måske er nemmere og mere gennemskueligt for bidragsyderne, hvis man benytter en mere central model, at man jo så også gerne ligesom skal kunne committe sig til som org, at modellen skal bevæge sig imod at rette sig efter et generelt lykke-bidrag frem for bare at forsøge i sidste ende at forøge kapitalejernes profit.. Hm, og handler det så ikke bare om, at man netop sørger for at.. dele stemmeretten mere ud end ellers, men måske er det i stedet mere bare, at man sørger for at modellen i sidste ende skal prøve at vurdere lønninger ud fra lykke-bidrag.. Tja, tjo, jeg ved det ikke helt...  
%(06.05.21) Tænkte et par tanker i går aftes, men det er nu lidt svært at finde ud af, hvad der præcis skaber min følelse af, at det ikke holder helt alligevel. Jeg tænkte bl.a., om man kunne overveje at sætte et princip om at kapital altid skal difundere ud til kunderne.. (Og hvad i den forbindelse så i øvrigt med opslittelser af orgs?..) Og inden jeg gik i seng tænkte jeg også på, at man måske nærmest kan vende "frygten" for at brugerskaren tager over rundt, så man søger at tilvejebringe denne magt hos brugerne og kan fungere som en slags fagforening.. eller noget i den stil; det var bare nogle løse tanker, som jeg lige skal tænke mere over, ligesom med de andre nævnte tanker.. Jeg tænker så nu lidt, at noget af problemet vel også er, at jeg er blevet i tvivl om, hvorvidt man faktisk kan sikre sig, at org'en har gode investeringsmuligheder.. For hvis man bare er en tech-virksomhed, der lover visse belønningssatser, men som ikke ejer indholdet, så ved man jo ikke, om brugerne ikke bare pludseligt kan finde på at gå over til en anden platform.. ...Hm, hvad med at man i stedet for at være en blanding mellem en virksomhed og en bruger-org, bare er en fagforening i bund og grund? Og så kunne det bare følge som en naturlig del, at brugere der ejer mange bidrag og/eller har et stor potentiale for at bidrage mere i fremtiden ligesom får mere stemmeret/forhandlingsmagt?..!.. Hm.. Tja, jo, og så kan man jo stadig med fordel søge at udvikle og vedligeholde en god stemmerets-/forhandlingsmagt-model, for bedre at kunne tiltrække mange arbejdere nok til foreningen til at man.. ja, og hvordan skal man egentligt så sælge.. nå ja, man sælger så rettighederne til at benytte og evt. tjene penge på indholdet, som man så vil råde over som forening. Hm.. ..Ah, og fagforeningen må så alligevel godt være lidt mere end dette, for de må vel så også gerne kunne købe bidrag fra arbejderne, så at selve foreningen, som jeg så derfor alligevel vil foretrække at kalde en organisation (for det bliver jo så lidt en blanding mellem en fagforening og en virsksomhed alligevel, og passer 'organisation' så ikke bedre?..), kommer til at eje bidraget. ..Hm, og vil det så være arbejderne, der så kommer til at eje de købte bidrag, eller vil det være.. nå ja, det må rigtigt nok være investorerne. Så man tillader altså, at folk kan sælge deres bidrag, og derved kan folk deltage som investorer i specifikke bidrag, ved at de altså køber disse, og herfra er det så kun et lille spring så også at danne en investeringsfond, hvor en bestyrelse kan varetage investornes penge ved at købe diverse arbejdsbidrag løbende, og hvor overskudet fra denne handel så kan blive omsat til afkast til investorerne, alt efter hvor meget de har investeret (hvilket man jo så kan regne i, hvor mange aktier de har). Ah, dette er virkelig en god idé/indsigt/måde at anskue tingene på, jeg har fået her, for dette bliver jo netop så en org a la den, jeg har forestillet mig, hvor værdien primært ligger i ejerskabet af bidragene, på nær eventuelle formuer, der ligger på diverse investeringsfonders konti, og hvor arbejderne så ikke behøver at sælge disse bidrag til nogen, men kan også beholde dem selv som en slags aktier i stedet (dog ikke aktier i virkeligheden (men altså IP-rettigheder), for de er nemlig så ikke knyttet til nogen specifik virksomhed og følger altså bare med vedkommende, hvis denne skifter fagforeningsorg). Og samtidigt vil der så alt andet end lige (det vil jo kun være gavnligt) også blive en slags lønsatsmodel for deltagerne, så de kan holde styr på, hvor meget deres bidrag anses for at være værd. Men i dette billede er det så bare en ren forhandling, der sætter denne model, fordi medlemmerne jo altid bare kan skifte org, hvis ikke de er tilfredse med modellen. ..Ja, vigtig tanke at se det som en fagforening i det grundlæggende lag, som så bare er afhængig IP-lovgivninger (men ikke af, hvordan det fremtidige samfund udvikler sig (og alt muligt langhåret analyse omkring dette)!), og hvor investeringsdelen af det så bare er en seperat ting, hvor investorer i bund og grund køber og handler med (og forhandler med) IP-rettighederne til diverse arbejdsbidrag. Der vil selvfølgelig også ligge en værdi i at have en cementeret fagforening, hvad angår dens struktur og dens brand (og særligt hvor god den er til at varetage lønsatsmodellen), men... Hm, gad vide om der så er sandsynlighed for, at denne org så bare ender med at købe alle bidrag og ikke tage imod bidragslicenser så at sige fra arbejderne..? Det vil man jo måske gerne undgå..? Tja, eller hvad? Det bliver vel bare en del af det, at arbejderne skal sørge for at forhandle sig til, at muligheden holdes åben for deltagende arbejdere, der udliciterer deres bidrag, i stedet for at det skal sælges (for herved kan man som arbejdere bedre sikre sig, at man bevarer magten i denne org..)?.. Ja, det bliver sådan. For investorerne har alligevel også brug for arbejdernes opbakning for at kunne tiltrække mange bidrag til, og arbejderne får i øvrigt bedre forhandlingsteknologier, så der bliver ikke fare for, at investorerne kommer til at tage over fuldstændigt; så vil organisationen bare falde sammen i sidste ende (og det man jo gerne undgå). Okay, det virker altså som om, jeg virkelig er kommet over på et godt spor nu; mindre (fælles) fremtidsanalyse og mere lige til. Og dette er selvom, det er et en virkelig, virkelig simpel idé (altså en super simpel struktur bag den foreslåede org), men sådan er det jo tit: Man får kun de idéer, man får, så man kan sagtens gå og overse en simpel løsning i lang, lang tid. Tja, eller jeg har jo faktisk nok været over nogenlunde denne version af idéen før, men så har jeg.. Hm, nu bliver jeg i tvivl, men uanset hvad, så har jeg nok bare ikke ledt så grundigt efter en idé, der.. ja, som fungerer fuldt ud liberalistisk og uden nogen kompliceret bevægelse om at følge visse principper, som så kan sikre, at der kan ske belønning bagud.. Ja, jeg har vist været ret opsat på tanken om en organisation, der kan belønne arbejdere bagud, hvor tanken så har været, at organisationen skulle følge nogle officielle principper, hvorved den så ville miste anseelse, hvis man pludselig lod arbejderne i stikken, der ellers fortjente løn. Ja, og så har jeg nemlig mest bare set muligheden for, at bidragsydere får lov at holde på deres bidragsrettigheder, som en måde at skabe en mere solid bund for org-strukturen frem for at se det som central del af løsningen.. tja, eller jeg har jo faktisk set det som en ret central del af det, efter jeg arbejde på idéen sidste sommer (2020)... Nå, det er også lige meget. Nu virker det til, at jeg muligvis har en lidt mere simpel løsning, som stadig er lidt anderledes end bare at have sem-web-tech-virksomheder, særligt fordi der i sidstnævnte løsning godt kan være risiko for, at en situation indfinder sig, hvor virksomhedsejerne ikke har interesse i, at arbejderne for mere forhandlingsmagt, som vil komme med sem-web-teknologierne (ikke at dette vil blive et stort problem, for tech-virksomhederne skal jo bare lige tage højde for dette så). Men pointen bliver nemlig, at en mere fagforeningsorienteret organisation (hvor investorer og arbejdere på en måde er i samme gruppe, og hvor disse så handler med tech-platforme, der så selv bare er uafhængige af organisationen (alt andet end lige)) så i modsætning faktisk kan benytte denne mulige bedre forhandlingsmagt i stor stil potentielt set. For de vil jo ikke have nogen modsatrettede interesser, så alt hvad arbejdere kan opnå i fælleskab ved de forbedrede forhandlingsmuligheder, jamen det kan man jo så selv sagt også opnå i, hvad der jo basalt set bare er en fagforening. Og hvis fagforeningen endda også samlet set ejer rettighederne til at bruge en god platform, der er helt fremme i skoene, hvad angår disse teknologier, så.. (ja, hvad så?..) ..ja, så tror jeg altså, at en sådan fagforeningsorg ville få en kæmpe opblomstring på meget kort tid, fordi folk så vil strømme til (digitalt) org'en (fordi jeg altså også mener, at der sandsynligvis vil være meget at opnå ved disse teknologier kombineret med en fagforening), og de tidlige investorer og arbejdere, der så ejer rettighederne omkring denne platform, (eller dem de eventuelt har solgt de rettigheder videre til (i.e. andre investorer)) vil jo så sikkert med lethed tage en rimelig god løn for så at lade alle de nye brugere strømme til.. Ja, det vil man selvsagt kunne, for det vil en normal virksomhed endda også; man kan selv sagt tage løn for, at folk for lov til at bruge ens teknologier, så at man kommer til at eje rettigheder til at bruge en meget eftersøgt teknologi, og at man samtidig også allerede har en god forening, hvor man har mulighed for at bruge denne teknologi, jamen det gør jo kun situationen bedre for de tidlige investorer. Der er altså, mener jeg, en ret god sandsynlighed for, at der kan tjenes en fornuftig mængde penge ved at være tidligt på banen og således være med til at tilvejebringe de nye, brugbare teknologier for folk (ja, det giver lidt sig selv, endnu en gang). Tjo, okay, men det giver ikke nødvendigvis sig selv, om der så virkeligt skulle være en \emph{større} fordel i denne forbindelse, når det kommer til fagforeningsorg'en frem for en sem-web-virksomhed.. ..Tja, men er dette også så vigtigt at finde frem til? Nu er alternativet jo selv en simpel og tæt-på-konventionel løsning, så er det ikke bare at nævne den, og... Hm, men det ville nu være rart, hvis man kunne sikre sig, at man får noget ud af... ..Jo. Det vil kunne betale sig for nye arbejdere, der gerne vil bruge de nye teknologier at have en allerede velstruktureret fagforening, med lønsatsmodel og det hele, som man kan joine, og hvis denne fagforening har opretholdt en god integritet og medmennskelighed, jamen så kan man jo sagtens forvente, at det nye medlemmer generelt er villige til at donere en god løn til de bidragsydere, der har gjort det hele muligt. Nå ja, og i øvrigt kan en fagforeningsfond også lave reklamearbejde for de investerede penge (udover bare at købe arbejdsbidrag), så hvis man bare er tidligt ude med reklamer, når teknologien er ved at modnes, så kan man helt klart (..eller "helt klart" og "helt klart;" det skal jeg lige uddybe..) tjene penge på at samle folk op og få den med på bevægelsen. Hm, men hvordan tjener man pengene her?... Nå ja, det gør man jo bl.a. bare på en betalingmur på teknologien, men det kan man jo også med de gænse virksomheder. Ja, faktisk er der ikke rigtigt forskel vel egentligt, for begge tilfælde har jo bare nogle investorer, der råder over visse IP-rettigheder. Forskellen er bare, at en normal virksomhed ikke nødvendigvis tillader at arbejdere kan udlicitere deres bidrag og uanset hvad er lidt på modsatte side af arbejderne og nok ikke har interesse i, at arbejderne skal blive bedre til at forhandle med deres IP-rettigheder, og hvor en normal tech-virksomhed nemlig også er bundet af et specifikt brand, hvorimod en fagforeningsorg virker på niveauet inden, at man udliciterer samlede IP-rettigheder til en tech-platform, og derved indeholder org'en altså kun "arbejdernes side." Hm, på nær at denne org jo så godt selv kan gå hen og blive arbejdsgiver (man kan jo købe arbejderes bidrag). Men ja, her er pointen vel bare, at det bør være en målsætning for foreningen, at arbejdere altid kan vælge selv at investere deres arbejdsbidrag i foreningen frem for at sælge dem. ..Dette opnår så (muligvis), at man kan bevare magten mere jævnt fordelt i foreningen, og dette vil jo så være en fordel, når man når til, at forhandlingsteknologierne modnes.. Hm... ... Uh, og med min idé beskrevet i mine 26.-28.-noter nedenfor tænkte jeg jo, at man kunne sætte sig for i fælleskab at have en demokratisk (delvist) model med meget inerti i, som et vigtigt princip, og også med et vigtigt princip / en vigtig målsætning om ikke at vedtage diskriminerende ting for lønssatsmodellen. Der er jo stadig muligvis en vis idé i dette, men så kunne man muligvis bare implementere idéen som en strategi for foreningsorg og/eller muligvis tech-virksomhed i stedet for at antage det som et bærende element, at denne målsætning også kan holdes. Så man kan altså bare som fagforeningsorg eller tech-virksomhed bare forsøge at danne en tillidsvækkende ramme for at implementere disse målsætninger, og hvis man gør dette troværdigt *(.."tillidsværdigt" ville være mindre tvetydigt..) nok, så kan det muligvis være, at man dermed kan tiltrække flere brugere/medlemmer/arbejdere, og særligt også måske herved kan overbevise samme brugere/arbejdere/medlemmer mere om, at det kan betale sig for dem at være gavmilde over for de tidlige bidragsydere (for ellers kan de jo selv risikere at lades i stikken, når den tid kommer, og næste generation ligesom kommer på banen og overtager..).. Hm, ja og som virksomhed/org har man jo mulighed for at tilstrække brugere/medlemmer også selvom disse så ligesom skal skrive under på som en del af det, at de gerne vil hædre og (rettere) belønne de tidlige bidragsydere ved generelt at holde sig til en inertifyldt, ikke-diskriminerende demokratisk lønsatsmodel. Og det man så lokker med er 1: at optage vedkommende i den udviklende org med tilhørende platform og dets muligheder, og 2: at man så ligesom sikrer sig mod selv at blive ladt i stikken af næste generation af brugere, for når først man når den stabilitet, hvor en inertifyldt, ikke-diskriminerende lønsatsmodel bliver fulgt, så må man vel forudsige, at det ikke vil være gavnligt for folk at bryde denne stabilitet.. ..Virker umiddelbart ret fornuftigt.. (06.05.21)
%(07.05.21) Lige lidt noter, som jeg tænkte på i går aftes: Det skal også nævnes, at det gavner begge parter, hvis man indgår i sådan et system, hvor der er stor inerti i lønsatsmodellen, og hvor alle parter har givet løfter om derved ikke at handle grådigt, for så kan de nemlig straffes for dette. For arbejdere sikrer sig sig så mere, at de tidlige investorer ikke begynder at blive grådige på baggrund af at de kan kontrollere brandet og hele foretagendet, og investorer sikrer sig imod at medlemmerne bliver grådige og lader investorerne i stikken for deres værdifulde bidrag til smafundet. Så på en måde gavner aftalerne først arbejderne (og kan muligvis lokke flere arbejdere til), og i sidste ende gavner det så de tidlige investorer, for arbejderne vil miste meget etos, hvis de bryder deres løfter, især hvis de tidlige investorer har holdt deres ende af aftalen. En anden lille note er, at jeg skal sikre mig, at jeg har nævnt det om løfter og etos (det har jeg vist nok.. ja. Det kunne jeg bare ikke lige huske i går aftes.). Nå ja, og det gode ved at indgå disse aftaler er så også, skal det nævnes, at man så måske derved kan slappe mere af i forhandlingerne, både investor kontra arbejdere, men også særligt indbyrdes mellem arbejdere, fordi man i stedet bare kan lade lønudsigten være knyttet til et værdipapir, som indehavere bare selv må vurdere, hvornår skal sælges/indløses. Og den sidste note fra i går aftes er så, at jeg faktisk godt kan gå tilbage til at fremhæve lykke/velfærd noget mere, som en vigtig parameter, man vil forsøge at sigte efter. For ja, som jeg sikkert har haft indset før (og derfor er det så vigtigt for mig at lave den opsamling af det hele (næsten, for jeg udskyder nok fysikken lidt, men de tanker er allerede i en dyb skuffe i mit hoved), så jeg også får samling på det i hovedet (og kan lægge det fra mig, uden at skulle gennemgå en masse halvkryptiske papir-noter som den eneste hjælp udover min hukommelse (som egentligt er udmærket, men der er bare mange idéer på én gang (men sådan går det vel, når man har tendens til at udskyde skrivearbejdet, hvis der stadig er ting, man føler, man bør overveje..)))), så kan risikotagning osv. stadig belønnes som normalt (..måske har jeg tænkt det før, måske ikke alligevel..), for det handler bare om at man forsikrer sine værdipapirer og/eller sælger dem videre eller slår dem sammen med andre. Og dette går stadig fint i spænd med, at man som lokalsamfund stadig sagtens kan sætte dæmpere på lønnen, hvis én persons handling tilføjer en masse gavn og glæde (men hvor det så ikke hjælper motivationen særligt, om man har udsigt til at blive absurd svinerig eller bare rigtig rig); man bevarer jo bare i så fald de samme afdæmpelser uanset hvem den/de endelige ejer(e) er blevet. Cool. Nå jo, og jeg havde også lige et lille punkt, der sagde, at man nok bør prøve at sørge for at der er aktiv handel på tværs af generationerne med værdipapirene, så man på den måde opnår en mere stabil situation, hvor de ikke bliver særlig stor forskel på generationernes interesser (fordi de så på formel vis får blandet deres interesser (deres stakes)). ..Tjo, men det kan man ikke rigtigt sørge for, for man kan jo ikke tvinge folk til at bytte deres interesser, for de skal jo netop have lov til at være ueninge omkring, hvor meget bidragene er værd (det hviler hele pointen jo på). Okay, jeg overvjer lidt, om det ikke kommer i vejen for idéen om at ophøje en ff-org/sem-web-virksomhed ved at indgå aftaler om at følge en inertifyldt model, at det måske er svært at forhindre korruption af modellen i den forbindelse, men jeg tænker så nu: er svaret ikke bare, at man jo regner med... Tja, eller \emph{hvis} man regner med, at org'en skal bestå, så vil diskriminitionerne jo alligevel jævnes ud med tiden..? ... Hm, en alternativ idé ville være at lave nogle interne regler, der forhindrer en i at reverse-engineer'e idéer... Hm, måske er der noget at komme efter her.. (07.05.21) Hvis man havde en god måde at sikre sig på, at folk ikke kendte løsningerne, men kun selve problemerne som forsøges at løses... (Jeg har jo kastet denne idé fra mig igen før, men måske man alligevel kunne finde på noget smart, der kunne gøre den mulig...) ..Uh, eller hvad med at spørge kunderne?..!.. Eller med at gøre kunderne bestemmende fra starten af..? Hmm... ..Okay, det var klart også en værdifuld idé... ..Og det gode er, at man nok godt kan finde en måde, at sikre sig, at det kun er de rigtige kunder (som altså indgår i en omsætningsproces), der bliver talt med, hvad man muligvis gerne vil.. Ja, og så kan man også lave processer, hvor eksempelvis parter med modstridende interesser, men også bare parter generelt, kan opslitte virksomheden.. ..Hm, og kan det ikke også godt lade sig gøre så, at sætte belønningskurver, så kunder også kan committe sig til at betale bagud?.. ..For hvis kunderne ligesom bliver medejere, eller i hvert fald bare medbestemmere, af foretagendet, så kan de jo forhandle med IP-rettigheder og sætte betalingsmure op, så de dermed kan sikre sig, at de kan holde deres løfter. Og kunderne vil så have en vis interesse i ikke at bryde løfter, men hvad endnu vigtigere er, så kan man også bare sørge for, at pågældende løfter bliver juridisk bindende, enten inden for foreningens egne bestemmelser (hvilke kunderne btw bliver \emph{nødt} til at følge, da det er disse, der giver dem magt i første omgang), eller ud fra det omkringliggende retssystem. Og jeg tænker i øvrigt så også, at løn til arbejdere også bare tæller med som et tilsvarende (eller måske lidt mindre) grad, når det kommer til stemme-/magt-fordelingen, som de intægter kunderne betaler. Med andre ord skal en arbejder, der har fået 100 kr. i løn få ligeså meget magt (eller lidt mindre (..eller lidt mere..)) som(/end) en kunde, der har betalt 100 kr. for et produkt eller en service. Man kunne også gøre noget tilsvarende med investerede beløber, hvis man tør, stadig potentielt med en justeret stemmeret-pr.-penge-sats ift. kundernes. Hm, man kunne eventuelt lade kunderne bestemme disse justeringer (dog ikke med tilbagevirkende kraft, kun fremadrettet!), så at det på den måde er kundernes job at justere dem, så de tiltrækker et passende antal arbejdere og/eller investorer til foretagendet.. ..Hm, ja og det vil være en god idé, at have investorers beløb tællende med i høj grad i starten. Ja, man kunne jo sige det sådan, at investorene jo bør være dem, der sætter satserne i starten for så at lokke arbejdere og kunder til, men hvor det så derefter er hele pointen, at kunderne og arbejderne skal tage over. Og hvad får man så ud af at sætte disse regler som investor: jo, at hvis man ikke selv griber muligheden, så gør andre det bare, eller også går "kunderne" (de kommende kunder) bare sammen fra starten, og gør arbejdet for én, hvorved man så vinder 0 gevinst. Så ja, pointen er, at man som investor ikke ville kunne kæmpe imod denne bølge (det vil man godt i realiteten, men lad os håbe, at tingene går mere roligt for sig), så man kan ligeså godt joine bevægelsen, og så \emph{kan} man faktisk gå hen og tjene en formidabel sum for denne indsats (da man jo, når alt kommer til alt, vil have investeret penge i et foretagende, der forhåbenligt sandsynligvis vil få en kæmpe fremgang, hvilket man derfor sagtens kan tage sig rigeligt betalt for (og det er nemlig ingen sag at sørge for, at kunderne ikke kan stemme ens afkastrettigheder ud, også selvom de får mere og mere magt, for denne magt vil de nemlig stadig trods alt kun have i kraft af virksomhedens bestemmelser (plus selvfølgelig diverse juridiske kontrakter oveni til at bakke op om disse bestemmelser))). ..Uh ja, og ift. "bagudbetaling," altså de belønningsløfter, som kunder kan give tidligt til arbejdere, inden at de faktsik har midlerne til at betale dem, jamen de kan jo bare være nogle juridisk bindende nogen. Bum. Okay, virkeligt nogle gode idéer, både den lidt nye vinkel, jeg fik idéen til i går, og så denne ret nye vinkel, jeg lige har fået idéen til (selvom den jo minder meget om en gammel idé). (Jeg skrev forresten også en note (bare på et tilfældigt stykke papir; ikke på min notesblok) i forgårs aften om, om jeg ikke måske skulle prøve at tænke på, at gå lidt tilbage min tidligere idé... Hm, jo men denne idé handlede godt nok mere om at tvinge en virksomhed til at dens kapital langsomt vil uddeles til kunderne; min nye idé handler mere om bare at lade kunderne få magten fra start af uden at tænke så meget på, hvor kapitalen ligger (kapitalhaverne kan så få ret til et afkast stadigvæk i denne idé, men de fragiver så altså deres stemmerettigheder fra start til dette alternative system med stemmeret i særdeles høj grad til kunderne og altså baseret på deres samlede "købsinvestering" i virksomheden).) 
%Og når det så kommer til kapitalen og IP-rettighederne, så kan man godt sige, at investorene beholder dem lidt i starten, men stille og roligt bør det så fordeles ud på kundernes hænder (i tilfælde af konkurs bare), da de jo også vil overtage administrationsarbejdet, og da det altså derfor ikke vil være fair, hvis ikke de så også ender med at eje virksomedens, i hvert fald IP-rettigheder, men også kapital, mener jeg. Når det kommer til opdeling af virksomheden, så handler det bare om at beslutte sig for nogle grænser, både hvad angår produkterne og servicesne, og hvad der så også angår de relaterede bidrag. Og det burde så ikke være noget særligt problem at sikre sig, at alle kunder så kommer til at eje lige meget lige efter opslittelsen som før. ...Hm, eller det kan man jo ikke helt sikre, hvis nu kunderne ikke har købt de "gode produkter".. Hm... Tjo, men kan man så ikke bare sige glem det, og så sørge for, at alle kunder kommer til at eje lige meget i... Nå nej, kunderne "ejer" jo ikke andet end det potentielle konkursudbytte, og dette kun efter noget tid.. men så kan man jo stadig bare sørge for så ikke at ændre på værdien af dette ejerskab. Og når det komme til stemmeret?.. Tja, hvad med at kunderne bare selv bestemmer, hvordan de så vil have deres stemmeret fordelt ved en opsplittelse? Jo, det må være en god måde at gøre det på. ..Og for investorene gælder så lidt det samme, eller rettere, de må bare se på, om de vil splitte deres aktier ad i samme omgang, og så eventuelt handle med dem på tværs af opslitningen efterfølgende. (Angående "tidlige idéer" forresten så er idéen mere lig det, jeg har kladt "demokratiske firmaer".. Ja, og idéen med disse var i bund og grund det samme som her.. Tjo, men jeg har nok alligevel tænkt det mere førhen, som at folk alligevel så stemte for at forøge firmaets overskud, hvorved denne idé så ligesom krævede en måde stadigvæk at få aktier mere og mere fordelt ud på kundernes hænder.. Men her undgår man dette problem, fordi folk i stedet bare stemmer efter, hvad de tror vil gavne dem bedst som kunder, ift. de produkter og services, de kan opnå..) Hm, men fortsat fra denne parentes, så skal man vel alligevel så overveje, hvem der sætter betalingsmurene og administrere den del af virksomheden --- bliver det kunderne? ..Eller bliver det investorene, eller bliver det først dem, men så senere hen kunderne?.. Ja, mon ikke bare det bliver kapitalejerne. Men som nævnt bør kapitalen jo så også uddeles mere og mere til kunderne (hvorved jeg kommer lidt tilbage til mit tidligere koncept om "demokratiske firmear" btw).. Er dette ok? ..Ja, det er det nok, for måden at dette ligesom også er "fair," selv hvis man sammenligner det med nutidige virksomheder, er at det jo også er kunderne, der laver en stor del af det administrative arbejde ved at bestemme arbejdernes løn (som btw kun skal kunne gives fra den omsatte mængde penge minus det som investorerne har krav på). Og man kan så bare se det som, hvilket er rimeligt nok, at kunderne tager sig betalt for dette arbejde ved så også at få del i mere og mere af aktierne, og særligt at stemmeretsmagten omkring, hvordan produkterne sælges (hvilken så måske i starten overvejene vil lægge hos investorerne). Og i takt med at kunderne også administrere mere og mere af denne side af forretningen, vil det så altså ende ud med, på fair vis, at kunderne kommer til at eje hele foretagendet til sammen (men selvfølgelig efter at investorene er "købt ud" (på en altså ret automatisk vis) for en god mængde penge). Så investorene får altså et potentielt set rigtigt stort afkast (for dette kan sagtens være sat ret stort uden at det skræmmer kunderne (og arbejderne m.m.) væk), og de slipper for en del af det administrative arbejde, imod at de så tilgengæld på et tidspunkt vil have alle deres aktier vekslet til afkast (pr. virksomhedens egne bestemmelser). ...Hm, et kritisk punkt er dog, om man kan forhindre kunder i bare effektivt set i at give sig selv løn som arbejder for derved at opnår mere stemmerets- og købsmagt, og at dette så kan løbe løbsk på en eller anden måde.. Hm..? Tjo, men der vil jo være mange flere kunder end arbejdere.. Hm.. Nej, dette vil man allerhøjest kun kunne regne med i starten, hvis man da overhovedet vil kunne det.. ..Hm, det kunne være, at man kunne gøre det sådan, at alle kunder efter en vis tid som kunder får én stemme, men hvor kunder også kan få stemmer oveni dette alt efter, hvor meget de har "investeret" i forretningen med deres køb, og hvor at antallet af ektrastemmer, som kunder kan opnå, vil gå imod et lavt antal med tiden, sådan at det sikres at stemmerne med tiden kommer til at fordele sig således, at baggrundsfordelingen bestående af kunders basisstemme (altså den éne) kommer til at blive den mest betydende i den samlede fordeling (således at man nærmer sig et demokrati ret meget). Og det disse stemmer så bruges til er så, i første omgang at indstemme (efter en eller anden fair valgproces), hvilken model man vil benytte i fællesskabet for så derfra at kunne stemme efterfølgende og løbende om, hvordan lønfordelingen skal være i virksomheden. Så man afholder altså en slags grundlæggende valg med jævne mellemrum, som så bestemmer, hvordan folk kan stemme derfra, når det kommer til individuelle lønspørgsmål. Og tanken var så her, at de stemmer så bare er sådan, at hver kunde stadig kan fordele arbejdsløn ud fra, hvor stor en investering denne kunde har gjort, men at det mere demokratiske grundvalg så bare afgør nogle begrænsninger på, hvordan man så kan fordele denne løn (så at man eventuelt f.eks. ikke kan give løn til sit eget fagområde, eller noget i den stil..).. Tja, det var bare lige en hurtig idé fra toppen af hovedet.. Hm... ..Tja, eller hvorfor siger man ikke bare, at det er en fuld lønsats model, man stemmer ind i første omgang via en valgproces (som svarer lidt til politiske valgprocesser, som vi kender dem), hvor man så bare glemmer det med den éne stemme og alt det (eller også gør man ikke, det lyder jo egentligt ret dejligt, hvis man fra start kan gå med at at lave sådan en ordning, som vil føre mere fremtidig demokratisk lighed med sig i virksomheden), men hvor man så bare har stemmeret ud fra ens "købsinvestering?" Det lyder da næsten bedre (mere simpelt, og man har jo stadig potentialet til at indføre det med den éne stemme og det der, hvis man vil).. He, jeg kan huske, at jeg har tænkt de tanker før, og ikke helt har kunnet komme udenom netop denne mulighed for korruption, nemlig at vis lønnen sættes ud fra folks donationer, kan de gå sammen i smug om at favorisere hinanden hver især i en intern gruppe, så de effektivt set bare donere til dem selv. Men nu har jeg altså den mulige løsning, at man faktisk bare vælger en mere konventionel demokratisk proces til at sætte lønningerne, hvilket jo er lidt sjovt, hvis det bare er den vej, svaret ligger.. Hm, men man kunne faktisk måske med fordel også lige fastsætte nogle forhold, der gør at stemmefordelingen på kunstig vis jævnes mere og mere ud, eller rettere at der i det mindste er nogle kræfter, der trækker i denne retning, både pga. at det ligesom er god stil, men om ikke andet så også bare lige have en ekstra foranstaltning imod, at velhavende kunder/medlemmer kan opnå et positivt feedback på deres velhavenhed og magt i systemet (ved altså fra start bare lige fastsætte et lille negativt feedback, der bør forhindre dette i det store hele (og måske især på længere sigt, fordi det jo så gør, at velhavende medlemmer lidt skal løbe for at stå stille, hvad der vel på et tidspunkt må briste)).. Ja, det ville sgu nok være en god idé.. både en god og en \emph{god} idé.. Ja, og jeg tror også bare det vil lette folks sind, hvis de kan se, at kursen ikke går mod ulighed, men går mod en fremtid med stor lighed. (Og som sagt behøves der altså nok bare et lille negativt feedback, i.e. en (kunstigt fastsat) jævnende kraft på magtfordelingen.)  
%(08.05.21) Jeg har tænkt lidt videre. Jeg kom på at arbejdere måske faktisk hellere skulle have mindre stemmeret pga. deres løn frem for mere. Tja, men nogen andet er også bare, at det generelt vil være let nok at sørge for, at brugerne ikke kan lave skjulte aftaler, hvor de effektivt set bare sender pengene til demselv, uden at man ligesom kan opdage dette. Ja, især hvis man bruger en domkratisk lønsatsmodel med en vis inerti i, og hvor man så også bør have et princip om ikke at sætte lønninger alt for specifikt, så der er mulighed for unfair fordelagtiggørelse af nogen. Og så modellen jo er offentligt bliver det ikke svært at opdage forhold, der lugter af unfair fordelinger og/eller korruption, og så kan resten af kundeskaren bare bekæmpe dette... Hm.. Tja, whatever. Hvis modellen er sat ved en demokratisk proces, så må det være fint nok; så bør korruptionsbekæmpelsen komme af sig selv derfra --- og hvis ikke den gør, jamen så er det fordi, det er en demokratisk beslutning ikke at gøre noget ved den. (Og uanset hvad vil det i det hele taget også bare være ret meget bagateller, hvis nogle få brugere ender med at kunne cheese sig til nogle lidt lavere betalingsgebyrer alt i alt (fordi de finder en måde at donere lidt til sig selv effektivt set).) Jeg er også kommet på tidligere i dag, at investorer bør kunne vælge at låne en vis mængde aktier ud til kundeskaren, som så kan stemmeom, hvordan disse så skal lånes videre til arbejderne, sådan at kundeskaren dermed får en simpel måde at kunne give løn tilarbejdere, hvis værdi bliver bestemt i fremtiden alt efter, hvor godt det går for organisationen. I øvrigt vil jeg lige nævne, at der jo er en væsentlig forskel på den demokratiske proces her og så i nuværende politiske systemer: Man stemmer ikke her bare på pæne og rare ansigter med tilhørnde slogans og "løfter;" man stemmer på modeller, som ledelsen \emph{skal} følge, og denne stemme proces har kan så have mange omgange, inden man når til en endelig udvælgelse. Jeg tænker så for det første, at man både vælger en model som fællesskab(/kundeskare) og en model for, hvordan førstnævnte model skal håndhæves ved denne proces. Og angående hvad denne proces bør indebære mere præcist, så tænker jeg en slags ranked choice voting, hvor man stemmer en hel masse gange, hvor man så hver gang skiller sig af med en vis (ret lille) procentdel af de mulige modeller, der fik lavest stemmevægtning.. eller rettere man beholder en vis (ret stor (det kunne være 90 \% sågar)) procentdel af de modeller med størst stemmevægtning (så man kan altså godt skille sig af med en stor procentdel af de samlede modeller, hvis de f.eks. ikke fik nogen stemmer..) ..Ja, så når jeg siger procentdel, så mener jeg af volumenet ander stemmevægtningsfordelingen, hvor man så altså beholder 70-90 \% (eller noget i den stil) af den volumen (ved at lave et passende cut-off på antal effektive stemmer for at komme videre). Og så skal ledelsen altså overvåges nøje af vagthund-modulet, som man altså også vælger i disse valg, og hvor denne overvågning meget gerne må raporteres til offentligheden (høj gennemsigtighed er ultra vigtigt for et holdbart system), og hvor at brud på den indstemte model (i dette tilfælde lønsatsmodellen) så med det samme udløser et nyt valg, så snart det opsnappes. Og oven på et sådant system kan deltagerne så indgå i alle mulige forskellige partier/interesse grupper på kryds og tværs og gerne med arbitrære overlab, hvormed deltagerne bedst muligt kan sikre sig, at deres forskellige interesser bliver kæmpet for (ved altså at bidrage med forhandlingsmagt til disse (mikro-)partier/-interessegrupper, som så kan benyttes når en ny model forhandles). (Der skal i øvrigt selvfølgelig også som nævnt være modelvalg med jævne mellemrum, uanset om der sker brud på en tidligere strategimodel eller ej.) Og så tror jeg nemlig, at jeg bare vil nævne mine idéer til bevægelser sådan lidt btw-agtigt, fordi jeg altså ikke føler, at jeg kan bakke dem op ordentligt (ligesom med mine KV-idéer; holdbarheden af dem kommer i sidste ende bare an på folk tilslutning og gejst for idéerne (og det kan jeg bestemt ikke forudsige, at der vil komme nogen stor del af)). Men nu har jeg nemlig også tre fine idéer.. eller rettere to, for den første er bare helt konventionelle tech-virksomheder.. som hver især nok kan bringe en lys fremtid med sig (7, 9, 13), og som jeg mener har lidt større potentiale for, at få en stor tilslutning/gejst omkring sig. Især min 3. løsning her har jo i bund og grund alle mulighederne, som jeg har drømt om --- for selvom bagudbelønning ikke sker af sig selv i denne idé (hvilket vist også er en lidt flyvsk drøm, når det kommer til vores nutid), så kan man både stadig implementere muligheder for at give løn, hvis værdi kan afhænge af udviklingen, man kan også sagtens implementere måder, hvor idé-scouts kan underskrive fortrolighedserklæringer, hvorved de så kan tage rundt på fælleskabets vegne og lave handler med nye potentielle opfindere, og så har man altså i denne idé, imodsætning til de to andre løsninger, allerede fra start et system, hvor kunderne bestemmer lønningerne. (De to førstnævnte ting man man altså dog også i de to første løsninger, og kundemagten skal nok også indfinde sig med tiden i disse løsninger, men ret fedt med en idé, hvor man bare starter med magten hos kunderne). 
%(09.05.21) Kapital skal nok uddeles til arbejdere i høj grad udover investorer, sådan at især kunderne, men også lidt investorerne taber på en konkurs. 










%Investorer tjener alligel penge på opstartsfasen, hvorfor denne mulighed kunne være mere tilstrækkende en mine bevægelsesorgs.. 
%Men ja, godt at min bevægelsesidé, som den er nu, jo alligevel er forsøgt designet, så at den netop går godt i spænd med normal kapitalisme, især i opstartsfasen, så at de to muligheder derved faktisk kan ses som et spektrum i stedet for to helt adskilte løsninger. (tjek)
%Overvej også lige, om man bør nævne, at man jo f.eks. i det danske (og sikkert også de andre skandinaviske) samfund allerede har et system, hvor folk går sammen i visse fællesinteresse-grupper og sørger for at der er nogle rammer for lønsatserne..









%Okay, men hvilken fremtid tegner dette så, hvis vi samlet set bliver meget bedre til at analysere økonomiske forhold (og andre forhold for den sags skyld (her tænker jeg også især på de mere humanistiske))? Og hvilket økonomisk system vil dette så ende med at medføre; et liberalistisk, et kommunistisk eller et helt tredje (/fjerde/femte\ldots) system?  Tja, tiden vil jo vise, præcis hvordan det økonomiske system kommer til at udvikle sig, men jeg tror alligevel godt vi kan komme med nogle bud på det. Om ikke andet kan vi gøre et forsøg. Jeg tror, det er fornuftigt så først at bryde tingene mere op og se på, hvad et økonomisk system grundlæggende er. ...
%Et velfungerende økonomisk system må så være et, hvor ...
%
%[Tja, eller måske skal dette bare lige være en indskudt bemærkning..:] Så nu har vi snakket om, hvad et ``velfungerende'' økonomisk system er, men lad os så lige se lidt på, hvordan man overhovedet definerer hvad `velfungerende' indebærer. Tydeligvis ligger der i begrebet, at det er noget der skal kunne bestå, og noget som er godt, men hvad definerer så `godt?' Det vil jeg prøve at nærme mig mere her. ... %Jeg tænker altså her lige at binde de gode principper fra forrige paragraf op på nogle grundlæggende etiske tanker (så det lige står lidt mere solidt, og så er det jo også en fin mulighed til at nævne disse ting).
%
%
%%Men selvom det nok (efter min mening) er der vi ender, så kan vi altså ikke bare sige, at nu prøver vi alle sammen i fællesskab at handle efter disse etiske principper. Selv hvis man kunne få folk til at erklære dette i fællesskab ville det ikke holde en meter alligevel. Vi skal altså først finde en vej til et sådant lykkeligt system. Jeg har et bud på, hvordan en sådan vej kunne tegne sig. Idéen falder lidt i to dele... Hm, nu er vi så her igen...
%[Passende overgang fra det overordnede mål] Et bud på et lidt mere konkret mål at sigte efter kunne så være...
%
%Men heller ikke dette system er vildt konkret, og vigtigere, siger ikke noget om, hvordan vi når der til.. Så hvordan gør vi det? Jo, jeg tror...








%Ny opsamlings-brainstorm (26.04.21):
%I forgårs faldt der lige nogle ting på plads, og i går fik jeg så bare lige tænkt lidt efter over det, og er stadig ret sikker på, at det holder. Det er næsten lidt svært at finde ind til, hvad jeg tænkte før, kontra hvad jeg tænker nu, men jeg kan i hvert fald se nu, at det at bruge en handling-nytte-graf som grundlag for at prøve at bestemme, hvad folk bør skylde hinanden, simpelthen bare gør det meget meget nemmere at skalere, og gør det nemmere i det hele taget at få folk med på det.. og ja, gør det bare i det hele taget meget mere foreneligt med liberalismen (med dette kan bevægelsen nærmest foregå fuldstændigt liberalistisk..!!), frem for hvis man har et system, hvor man skal prøve centralt at udregne hele lykke landskabet og kompensere folk, og.. ja, det ville blive et meget ustabilt system.. Men i denne nye version af bevægelsen handler det bare om, at folk får mulighed for at lade spørgsmålet om, hvor meget de skylder hinanden baseret på handlingsudvekslinger og den tilhørende nytte draget af disse, stå åbent, så det kan udregnes senere i henhold til den dynamiske, demokratisk styrede model. Hm, og angående hvilken model man skal bruge lokalt set, hvis der er meget variation i folks meninger, jamen så handler det vel bare lidt løst sagt om at finde et gennemsnit over de involverede parter, som så kan bruges til en given handling. Og fordi man så altid kan udskyde udbetalinger på tokens, så kan man ligesom være rimeligt sikker på, at man altid i sidste ende vil nå til den samme model.. Ja, for hvis man som en lokal gruppe i fremtiden vil have sine egne lidt aparte regler, så er det jo for egen regning; hvis der tydeligvis er en model, der fungerer bedre, så vil det koste alt for meget for en lokalgruppe at holde fast i denne. Men hvis der ender med at være sådanne forskelle i befolkningsgrupper i fremtiden, så er det også bare fair nok. Det må man bare tage med. Hvis man så har lavet innovative bidrag for en lokalgruppe, der ender ud med at blive til den mere dovne side, og man måske endda kommer til at udskyde token-vekslingen til problemet bliver større, jamen så er det bare ærgerligt. Og hvordan tegner man så skelende mellem befolkningsgrupper? Det gør man vel bare helt frit..? Ja, der er ingen grund til at se på nogen ydre parametre så som lokation. En stor del af pointen bag bevægelsen bliver nemlig også, at der altid skal være en ret stor inerti i folks afstemningsbeslutninger, så alle mennesker kan regne med, at lønningsmodellen for diverse grupper ikke bare ændrer sig helt vildt lige pludseligt. Det bliver altså en af grundprincipperne i bevægelsen --- og så tror jeg ikke det gør så meget, at denne grundsætning (ligesom de andre faktisk også) er ret vag, når det kommer til at implementere specifikke versioner af den, men jeg føler nærmest at dette er en styrke snarre end en svaghed ved bevægelsen. For når man først følger principperne overordnet set, så skal man nok i fælleskab kunne finde ind til de mest fornuftige og solide implementationsløsninger med tiden (og den lille usikkerhed, der er forbundet med spørgsmålet om, hvad den præcise implementering bliver, vil alligevel være lille i forhold til alle mulige andre usikkerheder). Og en af de ting, jeg har gået og overvejet i de seneste dage, er, at der jo gerne også skal være en løsning, som ikke afhænger af alle mulige fremtidsforudsigelser gjort i fælleskab, men hvor man bare kan have en mindre bevægelse, som fungerer inden i et system som det nuværende (et liberalistisk system, men det bliver bevægelsen også.. et præ-bevægelse liberalistisk system). Men det kan det også nu, for det handler så bare om, at firmaerne/organisationerne, der vil udgøre en igangsættelse af bevægelsen, selv holder en lønningsmodel, som aktive medlemmer, og her er firmaet/org'en fri til selv at vælge, hvordan dette defineres (for det bliver en del af de ret vage grundprincipper, nemlig at der sagtens kan være et begrænset optag i det stemmeberettigede fælleskab, så længe optaget dog altid sigter imod, at få flere og flere lødige medlemmer med, og når man samtidigt også følger et princip om, at man vil sørge for at "lødig"-prædikatet ikke i sidste ende kommer til at dele befolkningen i to, men at man vil optage alle i fællesskabet så snart man føler at det kan bære det (hvilket det også ret hurtigt kommer til, for det handler jo bare om at få gode nok midler til at identificere lokale grupper, så grupper med en anden holdning til verden ligesom bare får sin egen lønningsmodel at stemme om (og som de jo så bare altid kan ignorere, hvis de ikke tror på bevægelsens bliver fremtiden for dem selv))). Og det store er så lidt, at en sådan org faktisk ikke rigtigt binder sig på så mange punkter, når de erklærer, at de er en bevægelses-org. Det betyder bare, at man sætter et mål for, hvordan ens medlemmer bør belønnes givet deres handlinger, der påvirker folk internt i org'en eller folk omkring den. Og det en org-ledelse (og det er vel mere en org end et firma..) så skal sørge for, er så bare, at lægge handlinger bag belønningsmodellen, som viser, at man er seriøse omkring den, bl.a. ved at forsøge at give folk den løn, de fortjener (så at tokens bliver betalt af og ikke bare udskydes alle sammen), men endnu vigtigere sørger for faktisk også at uddele magt (i form af stemmerettigheder såvel som faktisk kapital, dog gerne med betingelser om, at denne kapital bare er til låns og bare er beregnet til at udgøre en forsikringsforanstaltning til de tidlige bidragsydere, snarre end den er beregnet som en løn) til bidragsyderne. Pointen er, at ledelsens fortjente løn jo kun er sat ud fra deres handlinger, hvilket skal regnes ud fra den dynamiske, demokratiske model på lige fod med alle andre (dermed ikke sagt, at der ikke kan være kæmpe store gevinster forbundet med dette tidlige, vigtige arbejde; det kan der helt sikkert godt), og det vil således ikke være ligeså holdbart, hvis ledelsen i det omkringliggende samfunds øjne ejer al kapitalen, for så vil de jo potentielt set kunne drage fordel af pludselig at trække stikken på denne org og løbe med byttet. Og dette er en fare, man skal gardere sig imod fra alle vinkler for de partier, der holder kapitalen (i det omkringliggende samfunds øjne). Så det eneste rigtige at gøre, hvis man tror på bevægelsen og vil vise dette overfor sine med-medlemmer, er at uddele af sin kapital, så man opnår at kapital fordelingen internt nogenlunde svarer til, hvordan den løn, som folk har til gode ifølge modellen, fordeler sig. Herved sikrer man sig nemlig også, at kapitalen (og dermed forhadlingsmagten, hvis modellen ikke kan følges, og man ender med at skulle gå tilbage til det omkringliggende samfunds love og regler (kun; dem skal man jo også følge alligevel), i det mindste for en tid) fordeler sig efter, hvem der mest interesse i bevægelsens overlevelse og positive udvikling. Og så kan det altså understreges, at kapitalen kun vil være til låns ifølge de interne bestemmelser i bevægelsen, og skal leveres videre på ordre fra fællesskabet (selvfølgelig altid kun lidt ad gangen (og hvis nu kapital-varetager laver en stor handel, hvorved retten til at sidde på en stor mængde kapital i mistes pludseligt mistes, så må man jo bare sørge for, at kapital-overførslen også er en del af handling (også med det omkringliggende samfunds øjne), så man sikrer sig imod, at kapital-varetageren ikke bare flygter fra org'en sammen med den pågældende kapital)). Det er altså bare en forsikringsanstaltning (der også generelt har til hensigt at skabe en mere solid magt-/forhandlings-balance i fællesskabet) for den enkelte, der har meget investeret i fællesskabet, så fællesskabet ikke pludeligt kan lade den pågældende i stikken. Og hvis kapital-varetagere pludseligt får lyst til bare at rende med kapitalen (og det gøres på en eller anden måde, hvor denne aktion ikke kan retsforfølges i det omkringliggende samfund, hvilket i øvrigt også ville være dumt ikke at prøve at sørge for ikke kan lade sig gøre; man bør selvfølgelig sørger for at opbygge så mange stabiliserende kontrakter omkring bevægelsen som muligt, gerne i sådan en grad at bevægelsens lovmæssigheder også kommer til at gøre sig gældende ret eksakt i det omkringliggende samfund, hvis og når dette kan lade sig gøre), så er det jo bare sådan, at bevægelsen selvfølgelig vil notere denne handling og sørge for at aktøren bliver negativt kompenseret (hvad vi så end lige kan kalde dette ('afstraffes,' men det virker en anelse hårdt)), skulle bevægelsen bestå (hvad jeg jo bestemt tror, den vil gøre uanset hvad (om så den bare skal genstartes et andet sted først)), og hvis denne handling ikke kan retfærdiggøres og samtidigt får negative konsekvenser for andre menneskers velfærd/lykke. Og når det så kommer til at opsætte betalingsmure, så man kan tjene penge på bidrag til det semantiske web, hvad er en ret vigtig ting at kunne (både pga. indtægtskilden i sig selv, men også fordi det repræsenterer en ret vigtig evne, man skal kunne have som bevægelse), så kan man jo bare sørge for, at medlemmer, hvad det omkringliggende samfund angår (dvs. at man gør det juridisk gældende), giver og/eller udliciterer deres bidrag til org'en, så denne får rettighederne til at tjene penge på dem, i alle tilfælde hvor dette kan lade sig gøre. Og når vi så snakke at udlicitere frem for at give rettighederne, så skal dette jo så kun være for bidrag, hvor det kan give mening at trække dette bidrag væk fra samlingen, uden at resten kollapser. Som en del af hele bevægelsens natur, vil der så meget muligt være en forskel i, hvor værdifulde de rettigheder er, hvis man ser dem i relation til det omkringliggende liberalistiske samfund, fordi en del af bevægelsen jo netop handler om at give belønninger til bidrag, som er svære at omsætte til penge ellers. Men så er det jo her, at man bare bør (som en velfungerende bevægelsesorg) give mere kapital-varetagnings-rettighed for sådanne bidrag, end de selv bringer af kapital direkte. Men dette er også okay, for idet at org'en så kommer til at råde over en stor mængde rettigheder, så kan org'en alligevel altid sørge for at sætte betalingsmure op på diverse systemer, således at der stadig kan tjenes penge på samme bidrag. For selvom andre partier så måske i princippet ville kunne sjæle pågældende bidrag, fordi de måske ikke er omfattet ordenligt af det omkringliggende samfunds IP-regler, så vil dette alligevel ikke kunne betale sig: Der vil nemlig ikke være nok at vinde på det, og samtidigt vil man så også påkalde sig bevægelsens harme, og vil således alt andet end lige bare blive straffet igen for et sådant forsøg på et foretagende. Og generelt tror jeg i øvrigt også, det ville være fint, hvis org'en faktisk ikke gør alt i verden for at bekæmpe pirateri og sørge for, at betalingsmurene bliver overholdt, for hvis man i stedet bare får en måde at holde styr på piraterne på (evt. ved en slags honeypot, hvor der så følger en fremtidig straf med, skulle bevægelsen bestå (hvad den jo sikkert vil efter min mening; så kan pirater... Hm, nej, på den anden side, ville dette så stadig være nemt nok at komme uden om.. Ja, never mind denne tankegang, men derfor behøver bevægelsen stadig ikke bekymre sig sindsyngt meget om pirateri alligevel --- ikke mere end hvad der let kan implementeres via IP-bestemmelserne fra det omkringliggende samfund)).. Og noget andet er, at man også bare kan gå sammen i bidragsgrupper/-foreninger, således at man kan gøre individuelle bidrag, som måske kan være svære at forsvare IP-rettigheder for, til et samlet bidrag ejet af hele den forening, som så samlet set kan forsvare IP-rettighederne, og så internt kan have aftaler for, hvordan rettighedskapitalen fordeles, evt. også bare via en demokratisk belønningsmodel, enten i form af bevægelsens model eller en egen model, eller bare via indbyrdes (liberalistiske) aftaler. Og dette leder mig så også videre (eller måske tilbage, om man vil) til at understrege, at belønningsmodellen kun skal være beregnet til at belønne bidrag, som helt eller delvist kan ses (efter en lødig og grundig (muligvis gjort i fremtiden) analyse og vurdering) som en form for donation til fællesskabet, og hvor bidragsyderen altså ikke har forventet fuld kompensation for handlingen, andet potentielt set end den kompensation, der i opnås via en belønning fra selve bevægelsen (altså beregnet ud fra modellen (som altså tager dermed tager højde for alle andre tidligere belønninger inden belønningen udregnes)). Og derudover skal det særligt også gælde, at der altid skal være en måde for medlemmerne i bevægelsen at sælge (og købe) og forhandle med retten til belønningen fra et givent bidrag. Herved kan man altså sagtens have helt normalt fungerende (liberalistiske --- og sagtens også kapitalistiske (det hører jo med, hvis de kan være helt liberalistiske)) firmaer, hvor arbejdere i firmaet har en kontrakt, hvor de med det samme afskriver sig alle belønningsrettigheder imod at få lønbetaling fra selve firmaet i stedet. Og det er derfor, der så egentligt vil være mulighed for at implementere selv det samme system vi har nu, inde i bevægelsen, hvis man vil det; bevægelsen bygger jo som sagt på liberalisme som ét af dens grundsten, så alt kan sådan set lade sig gøre. Forskellen er så bare, at der i praksis, medmindre bevægelsen altså korruperes, så den fuldstændig rammer ved siden af det originale mål, ikke vil være noget til hinder for, at disse arbejdere bare opsiger deres kontrakt, hvorefter de så... Tja, alt andet vil så nemlig være imod bevægelsens ånd, hvis ikke pågældende firma tillader, at det arbejder kan arbejde "frivilligt," således at de tjener deres egne penge på de bidragstokens, de generere herved. Ja, det er netop hele målet med denne forening: Hvis bidragsydere føler, at de ikke helt får som fortjent, eller får hvad der for den sags skyld er optimalt ud fra et velfærd/lykke-princip i pågældende lokalsamfund (inklusiv det globale samfund), jamen så skal de altid bare kunne tage teten og undlade at veksle deres bidragstokens (hvilket så eventuelt altså gøres ved først at opsige sine kontrakter om det modsatte), hvorefter fremtiden (den nært forestående, men det kommer jeg til om lidt) så i sidste ende bliver dommerne for, hvor meget bidragstokens'ne skal veksles for, når de endeligt bliver dette. Og ja, angående hvornår tokens skal betales, så er det jo bare lidt ligesom med kapital-varetægt-uddelingen, at man uddeler forsøger som bevægelses-org at betale belønningerne af, når de nogenlunde kan udregnes, og så kan der bare også eventuelt ligge en forhandling i det, hvor bidragsyderen vælger at sige, det var det, og dermed giver risikoen fra sig igen (ved altså at sælge sin token til lokalfælleskabet, og altså særligt til de specifikke skyldnere, men man kan jo godt simplificere det lidt i fælleskabet, så bidragsyderne ikke hele tiden skal rende og lave aftaler med deres specifikke nyttedragere fra yderens handling), således at pågældende ikke senere vil få krævet noget belønning tilbage, men så til gengæld heller ikke har chance for at få en større belønning. Der er mange måder, hvor man specifikt kan implementere systemer omkring dette. Det vigtige er bare, at bidragsyderne altid har mulighed (medmindre de har nogle særlige kontrakter, der forhindre dette, hvad man som bevægelse helst bør prøve at modarbejde; det strider nemlig imod idéen) for at sidde på deres bidragstokens og således tage chancen for så måske at blive belønnet mere i fremtiden. Og jeg har ikke nævnt det eksplicit her, andet end at jeg har nævnt, at der skal være en inerti i belønningsmodellen, men en af grundprincipperne i bevægelsen skal så være, at belønninger gives uden diskrimination af folk overhovedet, hverken baseret på lokation eller på tid, og at dette altså opnås ved kun at sigte efter at bruge belønningsmodeller, der fungerer ud fra abstrakte parametre, der beskriver hvordan de nuværende lønninger sættes, og hvor disse parametre så er valgt på helt lødig vis, så man ikke bevidst prøver at franare tidligere bidragsydere for penge. Og her bør man altså så bruge et skaleringsprincip, hvor at man hele tiden prøver at finde nogle tilsvarende, eventuelt tænkte, situationer i nutiden, hvor parametrene tilsvarer det daværende, og så giver lønning ud fra disse. Hvis en bidragsyder f.eks. så har været en af de tidlige bidragsydere, der har hjulpet med at sætte skub i hele bevægelsen, så bør dette bidrag (i form af den reklame, som bidraget også medførte til hele bevægelsen (og dette \emph{vil} jo blive en rigtig vigtig side ved de tidlige bidrag, også selvom selve bidraget i sin kerne egentligt er noget helt andet, f.eks. et konkret tekst-bidrag eller en anden form for investering)) også belønnes, og således bør man altså bare tænke sig til en lignende situation i den pågældende nutid, der svarer til det, bare eventuelt med en skaleringsfaktor på, og så lønne ud fra dette. Og hvad skal straffen så være, hvis man ikke lever op til disse principper, og hvem skal udføre straffen. Jo, jeg tror faktisk på, at man sagtens bare kan bruge (endnu) et grundprincip i bevægelsen, om at hver eneste gang at folk prøver at lave sådan fordækt handling, hvor de, enkeltvis eller samlet, prøver at løbe fra tidligere løfter, skal --- uagtet hvilken lykke/velfærd afstraffelsen skaber; det må man bare se ret meget bort fra (for så er følgende formulering af, hvor hård straffen skal være heller ikke værre) --- pågældende individer alle sammen straffes på en måde, så de som minimum (og man må så også meget gerne holde sig nogenlunde tæt på dette minimum til gengæld) kommer til at fortryde (med mindre de allerede er afdøde eller på anden vis er uden for denne lovs arm (hvad man jo så selvfølgelig skal prøve at forhindre; man skal forsøge at sørge for at folk så får deres straf rettidigt)) kommer til at fortryde, at de lavede den pågældende fordækte handling. Bemærk at dette princip jo ikke nødvendigvis altid vil følge, hvad der er etisk, eller det kan man i hvert fald ikke afgøre på stående fod, men denne bevægelse forsøger heller ikke at være det endelige system for mennesker. Det forsøger bare at være en bevægelse, der fører mod en lysere fremtid, hvor der så skal være plads til videre analyse og tilhørende vurderinger og afstemninger om, hvorvidt man skal forstætte med helt de samme grundprincipper, eller hvad man skal (men det bliver nok først en del tid efter at bevægelsen har udbredt sig og er blevet et stabilt system --- og hvis system har først positive ting med sig vil de fremtidige generationer, der har draget meget nytte af det, også formentligt være taknemmelige og vil respektere tidligere indbyrdes aftaler og forventninger fra de tidligere generationer (og nok allerhøjest udfase dem, snarre end at vende tingene på hovedet)). Eftersom systemet jo bør være liberalistisk (for det er jo bl.a. vigtigt, at man kan sælge sine risici, og også at man indgå indgå alle mulige aftaler og kontrakter med hinanden stadigvæk), så kan dette princip også føres videre og lade gøre sig gældende for folk, der bryder løfter i systemet generelt, således at man internt i systemet/bevægelsen opnår et retssystem, hvor man kan indgå intra-bevægelses-kontrakter, som man kan forvente at folk vil følge (for ellers bliver de jo bare straffet nok til at de vil fortryde det), i tillæg til det retssytem, der jo også vil være i form af det omkringliggende samfunds retssystem (inden man som bevægelse potentielt set \emph{bliver} det omkringliggende samfund på et tidspunkt), hvilket man jo også med fordel kan bruge flittigt undervejs. Jeg mener nemlig ikke, der vil være et fornuftigt alternativ til sådan et princip, og jeg mener ikke at bevægelsen kan være holdbar uden det. Og da jeg så mener, at bevægelsen bestemt vil føre meget godt med sig og dermed rigtigt vil være til at undvære i den nære fremtid, så vil der altid være rigeligt med incitament for de kommende generationer til at opretholde dette retsprincip --- udover at folk helt sikkert også vil synes at det er det moralsk rigtige at gøre; at afstraffe fordækt handling i moderat mængde, så aktørerne gerne lige akkurat (eller hvor godt man nu kan ramme det) kommer til at fortryde deres handlinger (og dette mener jeg bestemt er inklusiv de fordækte handlinger, der forsøger at lade tidligere bidragere i stikken på unfair vis). Så vi har altså alt i alt et princip om ikke-diskriminerende belønningsmodeller, et princip om at folk kan handle med tokens på liberalistisk vis og generelt bare har frihed til lave diverse liberalistiske handler og underskrive kontrakter osv., et princip om at ændringer i modellen skal ske langsomt, hvilket altså også skal være med til (som en ekstra foranstaltning), at bidragsydere ikke pludselig bliver snydt, fordi den kommende generation ændrer belønningsmodellen hurtigt og markant (hvilket er et ret vagt princip, men det er også fint), og så har vi altså også et næsten-grundprincip, altså et princip som jeg mener vil være godt at følge, om at en bevægelses org bør forsøge at uddele kapital løbende, så at man derved danner et værn mod, at folk bliver fristede til at stikke af fra bevægelsen (fordi kapitialen så alligevel er fordelt svarende til, hvem der har interesse i bevægelsens beståen). Nå ja, og så er der selvfølgelig også et princip om, at man vil forsøge at sørge for at bidragsydere kan have mulighed for at få deres belønningtokens vekslet, når de har lyst. Og jeg har så i øvrigt heller ikke talt så meget om selve handling-nytte-grafen, og om hvilke principper belønningmodellen bør følge nogenlunde, samt hvordan ydelser og modtagning af ydelser kommer til at udgøre "tokens." Men det er også rimeligt enkelt. Det handler således bare bare om at have et system til at holde øje med, hvad folk skylder hinanden basalt set. Og så er det altså ikke længere, hvor man sigter efter et mål om at maksimere lykke med systemet, ikke direkte, men i stedet handler det så bare om at sørge for at folk bliver belønnet fair ift. nutidige lønninger (når man ser på eller tænker sig til lignende situationer, muligvis op- eller ned-skaleret, i den pågældende nutid, hvor belønningen betales (eller hvor man eventuelt kræver penge tilbage)).
%(27.04.21) Okay, jeg mangler lige at nævne et par småting, og så kan jeg gå videre til at skrive noterne over det her til denne sektion. Lad mig starte med at understrege, at bevægelsens udbredelse nu er meget mere sikker i denne version, for den bygger jo kun principper, som anses for fair i vores kultur alligevel, og som vi allerede stræber efter i vores liberalistiske systemer. Det eneste nye er bare, at man nu kan forvente, hvis man først har erklæret sig som en del af bevægelsen, at blive straffet hvis man bevist prøver at underminere folk, man ellers, hvis ret skulle være ret (og hvis altså alt var fair), ville skylde løn/tilbagebetaling til. Dette bringer mig i øvrigt også hen til: Hvad med tidligere handlinger og handlinger generelt af partier, som ikke har erklæret sig som del af bevægelsen? Svaret ligger faktisk lidt i, hvordan jeg allerede har formuleret det: Kun bidrag, hvor der ligger en oprigtig velgørenhedstrang bag, altså handlinger, hvor yderen i princippet burde fortjene en større løn end den, yder har til udsigt, hvis man ser bort fra eventuelt forventede belønninger fra selve bevægelsen (og man bortregner i øvrigt også alle bløde belønninger så som at man eksempelvis vandt respekt blandt sine medmennesker ved handlingen osv.), skal belønnes af bevægelsen i sådan en grad, at den samlede løn for handlingen nu bliver passende. Og "passende" vil jo så altså sige udregnet ift., hvad man på demokratisk vis i den relevante lokaløkonomi ville give til nye bidrag med lignende (lødige, ikke-diskriminerende) parametre. Så det vil altså sige, at man kun bør belønne bagud til open source bidragsydere (hvilket i øvrigt kun er smart, for man vil alligevel for alt i verden gerne appelere til denne type af mennesker som bevægelse) og velgørenhedsbidragere m.m. Nå, men hvad så med den negative side af dette; hvad med personer eller instanser ('partier' under ét), der bevidst har gjort handlinger, der negativt har præget folk liv og velfærd, men som ikke har kunne retsforfølges med det oprindelige (og i starten omkringliggende) system, og som måske særligt så alligevel faktisk har draget løn af dette? Nå ja, og nu hvor jeg skriver dette, kommer jeg også i tanke om, at jeg også bør se på negative handlinger generelt; de kan jo ikke altid straffes økonomisk bare, og tit vil der også være en straf i det omkringliggende system, så hvordan regner man dette fra og til, og hvad gør man lige helt præcist? Tja, til det sidste spørgsmål, så handler det jo egentligt bare om at sige, at ingen straf som udgangspunkt er ekskluderet fra bevægelsen (selvom man i praksis derfra dog stadig bør ekskludere en masse strafformer i praksis! (Jeg taler altså her kun om, hvad man skal vedtage i det grundlæggende lag for bevægelsen)), og så skal straffe jo bare følge det samme princip som belønninger; de skal sættes så de er fair ift. lignende situationer i samtiden. Nå ja, og jeg bør i øvrigt også nævne, at når man har en situation, hvor det er svært at trække belønning/straf tilbage, så er det smartere hele tiden at sørge for at lade den margin, man forventer kan være på straffen/belønningen i henhold til modellens drift, lades stå tilbage ubelønnet/ustraffet, indtil den ligger mere fast, eller indtil nok tid er gået (for man vil jo også gerne give begge rettidigt, så folk ikke først får deres straf og/eller belønninger som gamle), eller for den sags skyld indtil vedkommende godkender belønningen/straffen (ved i bund og grund at "sælge" sin risiko). Men ja, når det så kommer til handlinger gjort før eller uden for bevægelsen.. Tja, det simple er jo bare at sige, at det lades være et åbent spørgsmål og således op til den enkelte bevægelsesorg, hvad man vil prøve at indføre, også selv for de positive handlinger. Men kan dog sige, at der sandsynligvis vil være langt større gevinst ved kun at kompensere de positive præ-bevægelse-handlinger, fordi man så får disse typer mennesker nemmere med på bølgen. Modsat er der nemlig ikke rigtigt nogen gevinst ved at straffe præ-bevægelses-handlinger, for pointen med straffe (i et etisk samfund) er jo kun, at man så forhindre efterfølgende handlinger, men jeg tror altså ikke nødvendigvis på, at det her så vil hjælpe at straffe præ-bevægelse-handlinger (udover den straf der tilkommer vedkommende i det omkringliggende samfunds strafferamme). Og når det kommer til økonomiske straffe, hvor man kunne tænke det som en mulig indtægtskilde, så kræver det jo, at vedkommende først ligesom bliver en del af bevægelsen, og hvorfor skulle vedkommende det? Ja, hvis det stod til mig, så ville jeg nok allerhøjest lige lege med tanken om at kræve særlige bidrag fra velstillede "syndere," så deres handlinger derved ligesom kan tilgives af folket, men dette er kun en tanke. Umiddelbart ville jeg helt klart bare sige, at negative præ-bevægelse-handlinger må straffes efter præ-bevægelsens strafferamme, færdig. Jeg mener nemlig, at det sidste man har lyst til som bevægelse er at skabe fjender unødvendigt. Hvorfor skulle man det, når nu man i stedet bare kan fremføre bevægelsen udelukkende med en kæmpe positiv energi og med principper, som nærmest alle et eller andet sted må bakke op om ret fuldt ud? Jeg tror helt klart på, at en sådan positiv energi er vejen frem. Ja. Jeg gentager: Det sidste man har lyst til som bevægelse er at skabe fjender og negativ omtale (for man ved aldrig hvad det kan føre til; det kan nemt så komme til at føre mere skidt end godt med sig, og så er det langt bedre at spille sine kort sikkert i stedet), især når man ellers snildt havde muligheden for at undgå dette. Oaky dokes. Jeg vil lige nævne og understrege nogle flere ting til slut. Noget vigtigt ved denne version af min idé er også, at det nu bliver ret nemt at investere penge i bevægelsen. Dette kommer nemlig bare til at handle om at give kapital til en org. Uh, og her er det værd at nævne, at indtil denne org så investere denne kapital videre og rent faktisk køber relateret arbejdskraft og/eller ejendomme/rettigheder for pengene, jamen så får investoren jo bare selvsamme kapital tilbage at råde over skulle org'en gå falit... Tja, eller det går de jo faktisk netop ikke, for noget af de kan jo gå til at give andre bidragsydere kapital til varetægt også. Men ja, uanset hvad bliver det i hvert nemt nu for investorer at investere i bevægelsen gennem diverse opstartsorganisationer. Jeg skal så også lige præcisere, hvad det vil sige at "varetage kapital," og bagefter kan jeg lige snakke om, hvad man gør i tilfælde af org-kollaps generelt, og om hvordan der godt kan være flere bevægelses-orginasitioner i gang på samme tid, endda hvor de kommer til at overlappe hinanden (det kan de nemlig i denne version af idéen). "Varetage kapital" vil sige, at man laver nogle kontrakter på tværs af bevægelsesorganisationen, som kan træde i kraft, hvis org'en på en eller anden måde går uopretteligt i opløsning. Det kunne f.eks. være pga. uenigheder og/eller en korrupt ledelse (eller andre vigtige parters korruption). Og når disse omstændigheder gør sig gældende, udløses kontrakterne så at omtalte kapital fordeles ud på den aftalte måde. Og pointen er så særligt med alt dette (udover at få en retfærdig opløsning, skulle dette ske), at dette så kan virke som en stabiliserende kraft for org'en, fordi folk så som sagt ikke er fristede til at løbe med byttet, og fordi kapitalen så kan fordele sig ret meget i henhold til interessen omkring bevægelsens overlevelse. *(Når det kommer til varetægt af kapital og de kontrakter, man indgår i organisationen for at sikre sig imod, at den går i opløsning, så skal jeg jo også nævne, at man selvfølgelig også bør forsøge at gøre, så at kontrakterne også kan straffe de medlemmer, der var grund til kollapset, hvis det kan lade sig gøre.) Nå, og så kommer vi til: Hvad når/hvis der så er flere org'er i gang på samme tid. En god måde at sikre sig mod korruption er jo ved at have en masse små organisationer, der hver især kun har ansvar for lokale interesser/befolkningsgrupper. Det gode ved denne version af idéen er så, at man ikke er afhængig som sådan af, at organisationerne består. En organisation starter jo først med at vise tegn på korruption eller på andre grundlæggende fejl, når denne begynder ikke at sørge for at midlerne bliver fordelt retfærdigt mellem medlemmerne, og eftersom organisationen så også skal sørge for at fordele kapitalen tilsvarende, så kan medlemmerne egentligt bare trække stikket i så fald for så efterfølgende at starte en ny organisation op med samme mål. For de ledere, der har gjort et ærligt stykke arbejde i den afdøde organisation, vil deres bidrag selvfølgelig bare stadig blive husket og værdsat, og self. belønnet retfærdigt på sigt alt andet end lige, og får de korrupte ledere vil deres gerninger også bare følge med dem og alt andet end lige ende med passende straffe (formentligt økonomisk, hvilket de fleste "straffe" jo helst skal være, og helst i form af en slags boykot fra den fremtidige bevægelse (eller økonomsike sanktioner, om man vil..)). Og fordi ledelserne i diverse orginasationer jo i sidste ende bør have samme mål, og hvor der så allerhøjest bør være forskel på deres nuværende modeller, hvilket med tiden bør jævnes ud, da modellerne jo er demokratiske (og da organisationerne også alle bør bevæge sig mod at inkludere så mange som muligt (de bør være så inkluderende som muligt, for nyoptagne medlemmer kan jo som sagt altid bare knyttes til en midlertidig medlemsgruppe med lavere rettigheder osv. (men hvor det så altså altid skal være meningen, at dette netop kun er midlertidigt, og at org'en skal søge efter at få alle medlemmer optaget på retsmæssig lige fod med (/ med samme rettigheder som) alle andre))), bør disse (modellerne) også bevæge sig imod hinanden alligevel (ellers er der jo noget galt). Så dermed kan bevægelses-organisationer nu altså fint overlappe hinanden og kan derfor også nemt finde frem til måder at smelte sammen på på en god måde, så snart deres interesseområder begynder at overlappe og så snart procuderen for sammensmeltningen er gennemtænkt. Hvis to org'er har fundamentale forskelle, som gør at de ikke har lyst til at smelte sammen, og vi er nået forbi det punkt, hvor man kan skyde skylden på, at man stadig er i en opstartsfase, jamen så må det simpelthen skyldes, at én eller begge af organisationerne ikke følger bevægelsens grundprincipper ordenligt; jeg kan ikke umiddelbert se, hvad der ellers kunne gøre sig gældende. For hvis grundprincipperne følges, så bør det altså ikke være noget problem at smelte sammen inden for en rimelig tidsgrænse, som jeg forudser det. Okay. Jeg føler så, at jeg også lige mangler at snakke lidt mere om diverse betalingsmure. Man kunne jo spørge, hvad er pointen med betalingsmure, hvis målet er at give så mange bidrag til folket som muligt? Jamen det er målet nemlig ikke med denne version af bevægelsen; målet er i stedet bare at sørge for, at de bidrag, der gives bliver belønnet fair, og så er det tanken, at dette så kan lægge grund til et system, hvor folk har interesse i ligesom at give så mange gode bidrag til samfundet som de kan (dog uden at stresse sig selv; det bør selvfølgelig kun være efter behag og efter behov for / higen efter reprocitive (vi ved, hvad jeg mener) gerninger). Og betalingsmure er så noget man passende kan gå sammen om at opsætte omkring sine intelektuelle bidrag m.m. Og måden man så kan gøre dette på, er så bl.a., at man underskriver kontrkter, der forener flere bidrag under én paraply (og hvor man så indbyrdes ingår nye kontrakter --- gerne helt adskilt fra belønningsmodellen, men selvfølgelig også gerne ud fra nogenlunde de samme principper alligevel --- om lønfordelingen for den indtægt, man opnår på baggrund af det samlede bidrag), så man sikre sig, at der kan opstilles IP-rettigheder omkring dette samlede bidrag, og at man så dermed kan forsvare disse rettigheder sammen. Ja, så det var egentligt bare det, jeg lige ville understrege om dette (mest det første med, hvad hele pointen med dem var; det sidste har jeg vist nævnt). Jeg har også noteret, jeg jeg nok bør sige noget mere om at sælge og handle med risici omkring model-belønningerne (og alt muligt andet for den sags skyld), men jeg tror egentligt, det bør være nogenlunde klart allerede. Jo, så vil jeg lige give et understregende eksempel hurtigt: Hvis man som arbejdsgiver ansætter mennesker til at arbejde på at bidrage til bevægelsen, hvor man altså så selv regner med at blive belønnet af selve bevægelsen, jaman så bør hele det reklame-bidrag, der også ligger i denne aktivitet, såvel som aktiviteten i sig selv, jo så stadig ultimativt regnes som arbejdsgiverens bidrag. For arbejderne arbejder jo bare for deres løn alt andet end lige. (Og det duer altså slet ikke, hvis man skulle prøve at definere bevægelsen anderledes, end hvor dette gælder..) Så det er altså kun i tilfælde af, at man også på lødig vis kan betragte en arbejders individuelle valg om at arbejde for denne virksomhed som en generøs handling imod bevægelsen; hvis det f.eks. kan påvises, at arbejderen kunne have sagt ja til et andet bedre betalende job eller lignende, men valgte at deltage i bevægelsesfremmende virksomhed med henblik på at styrke bevægelsen (og dermed også som regel med henblik (for dette tager man jo ikke med i udregningen) på at få en yderligere løn fra selve bevægelsen). I disse tilfælde vil en sådan handling dog heller ikke trække fra fra selve arbejdsgiverens handling, medmindre det kan påvises efterfølgende, at der var mangel på arbejdskraft og at arbejdsgiveren burde have sat en højere løn, hvis ikke det havde været fordi at man som arbejder og arbejdsgiver regnede med den efterfølgende bevægelses-belønning. Tja, det er måske faktsik en lidt mere kompliceret diskussion, om hvordan og hvorvidt man eventuelt kan fraregne arbejderes generøsitet fra arbejdsgivers, men det er heller ikke sindsygt vigtigt; det skal man nok hurtigt finde ud af. Så ja, det er ligesom pointen. Og selvom det måske for nogen kan virke lidt uretfærdigt, at arbejdsgiveren i dette tilfælde for ret til hele belønningen, så er dette altså bare en ting, man ikke kan komme uden om på nogen god måde alligevel, selv hvis man ville. Én ting er, at det ville ødelægge hele simpliciteten i bevægelsens grundprincipper, og en anden, meget vigtig, ting er også, at man simpelthen bare virkeligt har brug for at tiltrække investeringer som bevægelse, og det kan man nelig lige præcis rigtigt nemt, når tingene er på denne måde. Hm, nu får jeg det lidt til at lyde som om, det er en dårlig ting; nej, det mener jeg ikke. Jeg argumenter bare ud fra et tilfælde, hvis man skulle prøve at overbevise nogen, der synes det virker uretfærdigt. Det skal derfor heller ikke ses som en bivirkning ved at bevægelsen skal have mulighed for at komme godt fra start; nej, jeg tror altså på, at disse grundprincipper kan holde i lang tid og skabe grund for et rigtigt godt etisk/lykkeligt samfund i fremtiden. For lad mig nemlig gentage: Hvis arbejdere så kommer til at føle sig uretfærdigt behandlet, kan de jo under bevægelsen netop bare sige, okay nu arbejder jeg ikke længere for at få den pågældende løn fra arbejdsgiver, men i stedet arbejder jeg alene med henblik på at blive kompenseret for mine handlinger af bevægelsens belønningsmodel i sidste ende. Og så er det så her, at man også, nærmest som et grundlæggende princip, i bevægelsen så skal sørge for, at dette altid kan lade sig gøre for arbejdere, således at man bekæmper enhver arbejdsgiver, der ikke tillader sådant "frivilligt arbejde," når først bevægelsen er kommet godt i gang. Nå, det blev alligevel en lang tekst til at beskrive det eksempel, men det gør jo ikke noget. 
%Okay, jeg har lige nogle sidste noter tilbage til denne brain storm: "Lige lidt om temp. CS generelt og om demokratiske firmaer generelt (altså nok bare lige nævn det)..", "Og om kurver!", "Og nævn også, at det kan være nyttigt aktivt at prøve at måle stemningen særligt ift. risici hvad angår bevægelsen (så man bedre kan udregne belønningen til startbidragsyderne)." og "Uh og "kurver" ift. belønnings- og straf-caps også..!". Jeg vil bare lige opsummere dem hurtigt her til sidst. "Temp.(orary) CS (closed source)" handler bare om, at jeg generelt synes det er en god idé med virksomheder, der kun holder på kapital og IP-rettigheder midlertidigt, og hvor dette så langsomt bredes ud på brugere/kunders og arbejderes hænder. Det er nemlig den eneste etisk forsvarlige måde at gøre tingene på, med mindre at man direkte kan påvise, at der alligevel vil være denne effekt i samfundet/systemet i forvejen, således at kapital og rettigheder ikke bare ophober sig på de fås hænder. Og i forbindelse med en bevægelses-org synes jeg klart også, at enhver org skal have dette for øje: Al kapital og alle rettigheder skal kun ses som midlertidige rettigheder, som har til formål at sørge for, at org'en kan virke ordentligt, men man skal alligevel altid sørge for, at denne kapital og disse rettigheder langsomt vil blive mere og mere allemandseje med tiden (kun medmindre der bliver et klart og bestandigt skel mellem to økonomiske systemer uden noget overlap (dette kunne eksempelvis være i tilfælde af at man kollonisere en anden planet, hvor der ikke er mulighed for handel, eller hvis man opdager en rumvæsen civilisation på lige fod med en selv; her må man også godt holde på midlerne indtil at økonomierne bliver mere integreret i hinanden og kan smelte sammen..)). Jeg tror heldigvis dette kommer til at ske lidt af sig selv, men hvis ikke, må man altså meget gerne faktisk tage de som et princip i org'en. Og hvad får man så ud af at gøre dette som org, jo, man sørger for at signalere til alle medlemmer, at de kan være sikre på, at fremtiden, som org'en tilvejebringer, vil være lys, og at ingen vil blive ladt i stikken i sidste ende (hvilket sikkert vil tiltrække flere medlemmer, for når det kommer til sådan nogle overordnede samfund-i-fremtiden-spørgsmål har mennesker det faktsik, efter min mening, med at være rigtigt godsindede generelt (og det er jo nemlig også ret let at være det på den front, så det er jo bare skønt)). Nå, det næste er så "demokratiske firmaer." Dette er bare idéen om, at demokratisk styrede virksomheder (så som en bevægelses-org), hvor man giver folket mere magt på bekostning af investorernes (m.m.) faktisk vil kunne betale sig generelt, fordi investorerne så til gengæld køber brugernes fulde tillid. Dette vil jeg komme mere ind på senere i denne "øvrige"-sektion, men det er bare værd at bemærke, at vores bevægelses-org i denne idé faktisk bare hører ind under konceptet om sådanne "demokratiske firmaer" (som så er et mere løst begreb, der også kan passe på andre typer virksomheder/organisationer). Og "kurver," ja, det er lidt blevet mit nøgleord, når jeg skriver noter, for at man jo med fordel kan lade diverse kontrakter og værdipapirer afhænge af, hvordan ting udvikler sig i fremtiden. Dette gælder jo både de kontrakter osv., man opsætter omkring en org for at få den til at fungere som beregnet, men særligt gælder det selvfølgelig også for belønningsmodellen; her kan belønninger jo godt afhænge, ikke bare af selve arbejdet, men også af efterfølgende udfald i den fremtidige udvikling. Og angående det med at prøve at måle risici, så giver det lidt sig selv, men lad mig lige forklare: Da belønningsmodellen sikkert kommer til at tage højde for risici forbundne med et foretagende ved foretagenets start, og fordi modellen jo som sagt fungerer ud fra et skaleringsprincip til nutiden, så vil det være smart at prøve at måle allerede tidligt i bevægelsens opstart, hvad folk generelt mener om dens chancer. For generelt kan det jo være fordelagtigt at belønne succesforetagende ekstra, hvis der egentligt ikke generelt var tiltro idéen til at starte med, men hvor innovatørerne så alligevel fik overkommet den mistro og fik ført idéerne ud i livet til success. Med andre ord mener jeg, at folk vil lade det kunne betale sig at tage risiko med investeringer (ligesom det jo er hele fundamentet bag tanken om investering her i nutiden). Og hvis man så vil have bedst mulig mulighed for at belønne start-bidragere retfærdigt (og det kan i øvrigt ses fra begge vinkler (så man heller ikke opfinder en langt større fortidig usikkerhed, end den man rent faktisk anså dengang)), vil det altså være smart at tage stikprøver (hvor man prøver at sørge for at svarende heller ikke er økonomisk biased) på folks generelle forestillinger om bevægelsens sandsynligheder. Og så sidste af de punkter, nemlig omkring "caps:" Dette punkt handler om, hvordan man skal "skalere" belønninger. Man kunne jo godt prøve at holde et simplistisk princip om, at belønningen f.eks. skal være "proportionelt med den lykke, man skaber" eller noget i den stil, men dette mener jeg ikke vil være holdbart; ikke indtil at man så finder et eller andet godt, "magisk" princip, som man kan følge. Men indtil i første omgang er det altså langt bedre at lade det spørgsmål stå åbent og gøre det til et demokratisk valg, hvilke principper og parametre, der skal belønnes ud fra. Og dermed kan der altså også sagtens sættes nogle bremse-effekter på belønningskurven, ift. hvis man ser den langs en akse af, hvor meget gavn handlingen skabte, især hvis der f.eks. er tale om meget få mennesker. Ikke for at sige at det sikkert bliver sådan, men man kunne altså godt forestille sig nogle bremsende parametre for, hvor meget løn skal gives til enkeltpersoner, hvis sådanne kurver findes fair af folket, og hvis man så heller ikke tror at sådanne forhold vil sænke innovation- og bidragslysten i for høj grad, og, sidst men ikke mindst, hvis man kan forsvare at forholdene ikke er bevægelses-integritetsbrydende i form af at de kommer til i praksis at blive diskriminerende mod fortidige bidrag til fordel for nye bidrag. Og det giver i øvrigt sig selv, at man kan gøre det samme for straf også.
%Det er gået hen og blevet den (28.04.21), inden jeg fik forklaret ovenstående punkter, og nu er jeg så også kommet i tanke om nogle sidste ting, som jeg mangler at forklare i denne brainstorm. Jeg mangler at forklare om, hvordan man i en bevægelsesorg (og altså også i bevægelsen generelt, på tværs af orgs) sørger for at håndhæve belønningsmodellen, og også hvordan brugere registrerer bidrag og handlinger generelt, og hvordan de kan skabe værdipapirer (det jeg går og kalder "tokens" her) ud fra deres handlinger. Angående registrering, så er svaret bare: det er lidt lige meget. Man skal bare finde måder, hvorpå folk kan registrere deres indbyrdes bidrag/handlinger til/for/mod hinanden, så alle får rig mulighed for at gøre dette. *(Og jeg skal lige huske at nævne også, at man så selvfølgelig også skal sørge for, at det kan verificeres, hvem der gjorde handlingerne (..og at de blev gjort!), men så er man ellers også good to go.) Og det gør så ikke noget, at registrene ikke nødvendigvis er globalt tilgængelige, når bare de kan verificeres og bringes videre til mere globale retssystemer, skulle der blive nødvendighed for dette. Så hver lokalgruppe i det globale samfund har bare ligesom ansvar for at holde et fungerende registreringssystem og et fungerende lokalt retssystem, som kan vartage den lokale bidrags-handlings-graf og håndhæve den lokale belønningsmodel samt den lokale løfte-samtykke-graf. Og inden jeg kommer til, hvordan disse håndhæves, så lad mig lige understrege, at en stor fidus ved bevægelsen jo netop er, at uenigheder kan løses ved simpelthen at udsætte bedømmelser, hvorved man altså tager sin skyldte løn/straf for en vis samling personlige handlinger og udsteder et værdipapir på bagrund af det, hvor folk så formelt kan handle med sådanne værdipapirer, og hvor belønningen/straffen (hvis man skal kunne handle med straf; i praksis kommer man nok til at vedtage nogle begrænsninger, hvad angår handel med straf (så det ikke kan lade sig gøre i samme grad at handle med negative værdipapirer.. især ikke hvis straffen kan være andet end økonomisk, men det bør man i det hele taget også bare holde sig fra om ikke andet så i starten af bevægelsen)) og den tilhørende risiko(/chance) så altså kan handles med. Sådanne udstedelser og handler skal selvfølgelig så også varetages af det lokale bevægelses-rets- og registrerings-system. Okay, og bemærk, at alle sådanne handlinger så, hvad end de bare er rigistreret eller også efterfølgende er udstedt som værdipapir, så de kan handles med, medfører at de personer, der har haft gavn af handlingen, så skylder bidragsyderen, eller den person der er kommet i bisiddelse af det pågældende værdipapir, en løn givet ud fra belønningsmodellen. I øvrigt kan "handlinger" også indebære, at man har draget nytte af andres bidrag, så kan også gå den anden vej ift., hvem der skylder hvem penge. Så dermed er der altså også i ligeså høj grad mulighed for negative værdipapirer (u-udstedte eller ej); ja alle positive værdipapirer har jo faktisk negative værdier fordelt ud på bevægelsen (mere eller mindre lokalt alt efter typen af handling), og alle disse handlings-registreringer er altså bare noget man gør for at holde øje med, hvem skylder hvad. Dette er i modsætning til visse tidligere (..muligvis alle..) versioner, hvor jeg tænkte mig, at de ville være skyldsspørgsmål imellem individet og bevægelsen generelt. Men nu tænker jeg det altså helt klart som skyldsspørgsmål imellem individer kun; det mener jeg holder meget, meget bedre end det andet, især hvis vi tænker på en bevægelse, der skal opstartes inden i et allerede liberalistisk system, og også gerne skal kunne trække på de muligheder forbundet med denne situation. Og hvordan håndhæver man så alt dette. Jo, simpelt: En bærende ting, vi bliver nødt til at antage, hvis vi tror på bevægelsen, er, at den rent faktisk kan føre et bedre økonomisk system med mere samarbejde og innovation på et højere niveau med sig. Dette tror jeg jo, helt bestemt er tilfældet. Og når vi så har denne antagelse i lommen, jamen så vil det aldrig betale sig i længden for individer... Hov, lad mig lige indskyde følgende, som jeg også vil indskyde i teksten ovenfor: Når det kommer til varetægt af kapital og de kontrakter, man indgår i organisationen for at sikre sig imod, at den går i opløsning, så skal jeg jo også nævne, at man selvfølgelig også bør forsøge at gøre, så at kontrakterne også kan straffe de medlemmer, der var grund til kollapset, hvis det kan lade sig gøre. Okay og tilbage til: Og når vi så har denne antagelse i lommen, jamen så vil det aldrig betale sig i længden for individer at prøve at komme i fuldstændigt unåde i bevægelsen, for det kommer, da bevægelsen så netop givet denne antagelse vil overtage mere og mere, så til at svare til, at man prøvede at klare sig uden det nuværende økonomiske system her i nutiden. I takt med at bevægelsen udbredes, vil folk altså simpelthen blive afhængige (som i 'dependent') af den. Så det eneste man bør behøve at gøre i bund og grund for at håndhæve belønnings(/skylds)spørgsmålet (givet ud fra den pågældende belønningsmodel og den pågældende handlingsgraf og løftegraf (i.e. de indbyrdes kontrakter der findes mellem folk, der jo har mulighed for at ændre på skyldsspørgsmålet)), er at sørge for, at medlemmer, der forsinker de betalinger, de skylder, for meget ift., hvad de er i stand til at give, og ift. hvor sikker man er på udfaldet i store træk (for hvis der er tilstrækkelig stor sikkerhed omkring et udfald, så bør medlemmer også begynde at betale denne forventede skyld af stille og roligt), kan komme i fare for at blive ekskommunikeret af bevægelsen og således blive boykottet fra at kunne lave intra-bevægelseshandler og også for at det bliver svært for dem at drage nytte af bevægelsesrelaterede bidrag i det hele taget. Hvis bare man sikre sig dette som bevægelsesorg, forudser jeg altså, at dette vil blive rigeligt, når først bevægelsen får rigtigt fat. Og indtil da kan man jo som sagt hjælpe stabiliteten på vej med kontrakter (i det omkringliggende samfund), der sørger for at det heller ikke kan betale sig, hvis medlemmer ikke lever op til bevægelsens krav (fordi de så kan komme til at miste deres investeringer). Okay, jeg tror faktisk, det var det, jeg skulle forklare i denne brainstorm. :) Så skal jeg så til nu at omstrukturere det, så det bliver nemmere at følge, og skrive det i selve teksten til denne sektion.









%Brain:
%Okay, så idéen til et fremtidigt system, som kan være det umiddelbare mål for en forenings/forretningsbevægelse, er altså, hvor man sørger for, at når enhver person skal betale for en ydelse eller ejendom m.m., som vedkommende har gavn af, og man så ser på, hvilke ydelser, der har tilvejebragt muligheden for, at vedkomne kan få den gavn, så bør betalingen gå til de mennesker, der har givet disse ydelser på retfærdig vis. Og hvad er så retfærdig vis, jo, det bestemmes så ud fra en ikke-diskriminerende (er det er så en del af princippet, at den ikke må være dette (ellers er det et integritetsbrud)) model, som folk i lokalsamfundet, hvor stort det så end lige defineres (eller et over-samfund til dette, når ydelserne sker på tværs af lokalgruppe-grænser), stemmer omkring og justerer løbende, men hvor belønningen så altid gives ud fra modellens bestemmelser på det tidspunkt, hvor ydelsen blev gjort (tja, og hvis man så har behov for at bryde sidstnævnte regel, så skal det kun være i et ekstremt nødstilfælde, og dette vil så betegne, at målet (om samfundssystemet) ikke var nået alligevel). 
%..Okay, men så langt her (og der kommer mere) er vi stort set ved en generel beskrivelse af et velfungerende system, så det kunne jeg jo nævne der.. Ja, det bør jeg. Og så bliver de etiske mål også nævnt i den fobindelse i øvrigt.. [...] Okay, tænk på tastaturet: ..Jeg skal lige finde ud af [...] om der er en stor idé ved at tokenize'e bidrag, for... Hm, ja det vil jo altid være smart, så bidragsyderen selv kommer til at eje risikoen/chancen.. i hvert fald hvis der så bare altid er gode muligheder for at forsikre tokens på fair vis, hvilket der jo så klart bør være.. ... Men jeg går vel så helt væk fra at man tokeniserer start-bidrag. Eller er der stadig en eller anden lille mulighed for dette? Ah, her er der faktisk et hul!.. For... Nej vent... Når del-bevægelsesforeningerne starter, skal de så udforme en kurve hjemmel for de første bidragere, som har ydet bidrag inden at modellen er sat ordentligt op, og/eller hvor der er tale om centrale bidrag, der meget muligt bliver svære at veje i nogen overskuelig fremtid.. Ja, og del-bevægelsen vil så stadig have ret til disse bidrag, pr. den gennemgående aftale mellem delbevægelserne, der er med til at definere dem som en del af bevægelsen.. Men ja, og lad mig så lige repetere eller finde ud af, hvordan delbevægelserne/underforeningerne skal forhandle så at kurverne bliver overholdt.. Ja, det handler jo om, at folk, når de melder sig som medlem af underforeningen, samtykker til, at de skal hjælpe med at belønne startbidragerne... Og er det bare det så; fungerer dette i sig selv?... Tja, det gør det vel, især idet man så også bør sørge for at fordele ejerskabsrettighederne og lignende magt i foreningen, så startbidragerne også bare har magt til at sikre sig disse ting (hvor der så selvfølgelig vil være naturlige konsekvenser ved et magtmisbrug herved til gengæld). Men så kommer spørgsmålet jo dog så til mergers igen... Hm, for man kan jo ikke bare sådan balancere magten hensigtsmæssigt på samme måde på tværs af foreninger.. Hm... Hm.. Tja, måske er det vigtigt at gå sammen fra starten om at lave en overforening, der ikke gør andet end at bestyre IP-rettigheder osv...? Hm.. Jo, det er vel nok nødvendigt.. Og al fordelings... Tjo, men hvad så med fordelingen af midler imellem underforeninger (altså overordnet set), der er jo stadig et svært problem så... Hm... Ja, der er faktisk et problem her..!.. Det ville være fedt, hvis bare der var en eller anden smart formel, som beskrev en retfærdig måde at fordele midler imellem underforeninger baseret på deres bidrag (og muligvis på deres medlemstal også)... .. Uh, fik lige en idé..! Kunne man mon lave noget med at gøre det til en del af bevægelsen, at alle idéer til nye tiltag først skitseres, og at folk så løbende kan indsende bud på løsninger, og hvor vinderen i sidste ende får en pris, som bl.a. så afhænger af tiden det tog (men hvor man så altså heller ikke vil bruge for lang tid, for så kan andre jo bare snuppe prisen)? Jeg ved godt, at jeg skal arbejde mere med denne idé, men dette er altså den indledende idé. Man kunne jo så fra start af udlove den tidsafhængige præmiekurve... Uh, msåke kunne man så have, at folk stemmer på forslag, som de gerne vil se, og at prisen så forhøjes på baggrund af dette..!.. Ja, så at folk ligesom kan stemme om at forhøje visse dusører så at sige.. ..Okay, jeg er ikke 100 på, at dette løser det oprindelige problem, men det kan jeg lige vende tilbage til, for: Og så kunne man måske også gøre så at folk, der allerede har idéerne kunne skitsere betydningen af dem løst, og så indgå i en slags auktion, hvor folk køber idéen ud af idéhaveren, hvor at de så, så længe idéhaveren endnu ikke har fået det ønskede beløb, prøver at gætte på, hvad idéen består i, og jo længere dette trækker ud før de gætter det, desto mere får idéhaveren alligevel --- altså således at det kun er de tøvende købere, der bliver straffet ved ikke bare at købe idéen med det samme, hvis prisen var fair, men hvor at hvis prisen er sat for højt, så har de mulighed for at afsløre, at dette er tilfældet ved selv at gøre den efter (dog fraregnet de hints de vil have fået fra skitsen, hvilket dog vil være svært, fordi ikke alle er klar over, hvor meget der ligger i denne del, når man får en idé: noget af det største er at fornemme, at der er en mulighed, således at man kan få sig selv til at bruge tid på det...).. Hm.. Men ja, og så er tanken så også, at købere og idéhavere så også kan prøve at finde et kompromis for at spare tid, hvor idéen bare bliver belønnet ud fra visse forhold om, hvad den kommer til at medføre (hvor man altså så venter og måler på dette, inden man betaler bidragstoken'en af).. (Det er d. (23.04.21) i dag btw. (..ikke at jeg ikke førhen har haft lignende idéer, det har jeg helt sikkert, men det må alligevel være lang tid siden; har ikke tænkt rigtigt i de baner i nyere tid.)) 




%Nu hvor jeg har fået forklaret ovenstående tanker og forudsigelser, kan jeg så endeligt gå videre til at give et bud på, hvordan en mulig udvikling kunne være, der baner vej for en sådan lysere fremtid. Der er faktisk to lag i det bud, jeg vil komme med her. Det ene er sæt retningslinjer omkring et overordnet system, som det er meningen, at man skal sigte efter --- og hvis man ikke formår at bevæge sig i retning af de mål i sidste ende, vil systemet så, efter min mening, med al sandsynlighed blive vippet af pinden igen til fordel for nye økonomiske aftaler ... (Hm, er dette virkeligt den bedste vej ind til emnerne? Man kunne da også bare starte fra bunden af, og så putte de fremtidsstabiliserende mål på efterfølgende... Ja, det er nok en bedre vej til at starte fra bunden.. )





%Så lad mig...
%Hm, beskrive nu, hvad man så kan bruge skabelonsforeninger til? Hvor pointen så er den balance-gang, fordi brugere så hurtigt kan skifte bane, hvis de løfte-hjemler, de har indgået, giver hjemmel til dette (nemlig vis ledelsen altså opfører sig grådigt).. Tja, det er jo lidt bare konklusionen (selvom de så også lige kræver lidt løfte-graf-m.m.-teori), og er dette ikke en fin vinkel på det?.. Åh, jeg føler, at jeg har været doven her i går og i dag (også lidt i weekenden, men det var ok; det var der sådan set gode nok grunde til). Vejret gør det selvfølgelig heller ikke nemmere at koncentrere sig, men der er også ligesom det ved det, at det næsten føles som om, at det er blevet for simpelt nu her.. Jeg ved ikke helt, men jeg kan ikke helt lade være med at have det som om, der er noget jeg misser her.. Og måske er det altså med til at gøre, at jeg har lidt svært ved at få hul på det nu her; medmindre det vitterligt bare er en dovenskab, der er kommet lidt over mig i de her dage... Jeg må i hvert fald lige prøve at blive lidt mere ihærdig.. 
%(22.04.21) Nå okay, der var faktisk også lige noget jeg missede, som jeg har fundet ud af nu.




\subsubsection{Noter omkring muligheder for det fremtidige marked generelt}
I denne sektion vil jeg bare lige lave en hurtig oversigt over nogle af de idéer og tanker, jeg har, omkring hvilke andre forhold, der kunne gøre sig gældende i et fremtidigt samfund, end dem vi er vant til. Jeg tror for det første, at der ligger et stort potentiale i, at folk generelt forener sig som forbrugere og forhandler forskellige forhold, både hvad angår priserne for de produkter og services de forbruger, og også hvad angår de pågældende virksomheders aftryk på verden generelt, globalt såvel som lokalt. Det kræver selvfølgelig, at man bliver overbevist først om, at det nuværende samfund ikke nødvendigvis er optimalt, hvad angår forbrugeres (og menneskers generelt) glæde og gavn, men måske er optimeret lidt skævt, så det særligt er et vist fåtal af mennesker, hvis gavn og glæde bliver tilgodeset. Dette er jo en diskutabel sag, så på den led giver det god mening, hvis folk ikke har været så opsatte på at komme en sådan skævhed til livs, men jeg tror altså på, at når først man kommer i gang, skal man nok finde tusind ting, man kan ændre til det bedre. For eksempel tror jeg, at de samfund vi har nu, har en vis tilbøjelighed til stagnation, fordi korrelationen mellem at være velhaven og at være magthaven er for stor, og fordi der også i øvrigt er andre forhold med positive feedback, der gør at store formuer har nemmere ved at yngle sig selv. Og når først folk er velhavende og samtidigt også magthavende, så vil de påvirke samfundet, så konserveringen af deres rigdom bliver prioriteret i højere grad end f.eks.\ på at holde hjulene i samfundet godt i gang (og end mindre rige folks glæde og gavn generelt). Så jeg mener altså klart, der vil være meget at opnå, hvis man forenede sig som forbrugere. Vi kan også bare se på, hvor meget industri, der er bygget op om reklame. Det viser jo soleklart, hvor meget magt man egentligt potentielt har som forbruger, når man basalt set kan betale for visse services (tv og internetindhold f.eks.) ved bare at lade sig (potentielt) påvirkes en lille smule til at ændre sin forbrugeradfærd bare en anelse. Hvad kunne man så ikke opnå, hvis man rent faktisk tog en stor del af sit forbrug i sine hænder og brugte det (sammen med andre), til at forhandle diverse tilbud på produkter/services (altså en slags grupperabat-forening) og/eller til at forhandle bedre forhold for sig selv (og andre) på andre områder? Hvis man bare forestiller sig en forening, der sørger for at brugere på internettet selv ejer deres data og kan tjene penge på at sælge det videre, så ville selv denne idé kunne gøre en betydelig forskel for brugerne. Men der er altså ingen grund til at stoppe der, for grunden til at denne data er så meget værd, er at den kan bruges til at forudsige dine købsvaner en anelse, og der må altså være langt mere at opnå en selv dette, hvis man i stedet faktisk finder en måde at handle med sine købsvaner mere direkte. 

Når brugere/forbrugere går sammen, kan de også generelt opnå stor indflydelse på, hvilke virksomheder der klarer sig godt, og hvilke firmaer der ikke klarer sig. På denne vis kan forbrugergrupperne faktisk aktivt vælge at ``investere'' i visse firmaer frem for andre. Og når man når dette punkt, så er der ingen grund til, at forbrugerne ikke også kan opnå visse af de samme fordele, som pengeinvestorerne får; man bør altså simpelthen kunne handle sig til at få del i overskuddet fra virksomheden, som løn for at man vælger at bakke op om foretagendet. Her skal man dog passe meget på ikke at blive for grådig med denne proces, men umiddelbart tror jeg, det kunne være meget fordelagtigt i vores samfund, hvis forbrugerne samlet set også kunne gøre sig gældende som ``investorer,'' og hvis de dermed også kan komme til at sidde på nogen af værdierne i samfundet samt rettighederne til at få del i diverse overskud. Så der er altså også nogle muligheder her, men jeg vil lige understrege igen, at man dog bør træde varsomt med dette. Særligt skal man f.eks.\ sørge for aldrig at pisse den bredere befolkning af (og det gælder jo også helt generelt), fordi andre folk måske så kommer til at se sig snydt, hvis de ikke får del i disse gevinster (hvilket man jo i øvrigt også kan undgå, hvis bare foreningen er bred nok).

Med min nyfundne idé til at have en modelstyret organisation (jeg forklarede om den (og fandt på den) i forbindelse med min idé til en mere forbrugerdrevet organisation/virksomhed, men den kan bruges andre organisationer også), hvor ``afstemningen'' er en løbende proces, hvor alle deltagere bruger deres stemmekraft til at trække model i en vis retning, tror jeg også virkeligt man kan nå langt med at danne alle mulige forskellige interessegrupper. Det kan man også helt generelt, men jeg tror altså, at en så alsidig demokratisk proces også vil være nyttigt for sådanne foreninger. Men ja, det er ikke kun som forbrugere, at folk kan forene sig for at få indflydelse over forhold. Det kan f.eks.\ også være politiske partier (eller endda bare mindre grupperinger inden for et parti (eller uafhængige politiske grupperinger)(!)), fagforeninger, humanitære organisationer osv., hvor der altså kan være stor interesse i at få ændret visse relevante forhold i verden, men hvor dette vil kræve en fælles analyse og indsats. Jeg tror så, at teknologien omkring prædiktive ontologier kan skabe det nødvendige analyse værktøj til dette, og som værktøj til at indgå de nødvendige aftaler i fælleskab for at opnå målene, man har analyseret sig frem til er mulige, tror jeg altså, at min nyfundne idé om en kontinuert justeret, demokratisk ledelsesmodel måske kan være en nyttig kandidat. (Det bliver dog ikke det eneste nyttige aftaleværktøj: Det vil også være nyttigt med gode værktøjer til at tage (anonyme) stemningsstikprøver, samt værktøjer til at indgå i kontrakter, hvis binding afhænger af, om nok andre brugere vil have skrevet under på det samme i sidste ende.)


Nu nævnte jeg, at man også eventuelt kunne forenes i mindre politiske grupperinger. Helt generelt synes jeg det er værd at nævne, at folk med fremtidens teknologier nok får rig mulighed for at indgå i alle de forskellige foreninger/grupperinger, de har interesser til fælles med. Og jeg tror så også, at det vil være sundt for samfunde, hvis folk hermed også kommer til at danne så mange interesseforeninger, store og små, som man kan tænke sig, således at der for hver interesse hele tiden er en løbende analytisk og politisk proces i gang, der afsøger, om der er forhold, der kan være værd at tage aktion for at opnå i gruppen, og samtidigt ser, om der så vil være stemning for disse aktioner, når de opfindes, og som så i sidste ende kan igangsætte disse aktioner, hvis der er det.



Og hvad hvis de her mulige forandringer ikke vil føre noget bedre med sig i det store hele? Jamen så finder man ud af dette og kan derefter bare gå sammen om at finde tilbage til de tidligere forhold. Selvfølgelig bør man altid tænke grundigt over tingene i fælleskab, inden man begiver sig ud i at lave store omvæltninger, så man bedst muligt kan forhindre skader i at ske ved dette, og også så man bedst muligt kan planlægge tingene, så de kan reverseres på fornuftig vis, hvis man skulle opdage store fejl ved dem (hvis de altså fører dårlige ting med sig). Man bør altså altid sørge for at holde rimeligt koldt vand i blodet med alt, man gør, og sørge for at analysere situationen grundigt for inden i fællesskab, før man går i gang med at prøve at ændre tingene til det bedre. Jeg kan dog lige tilføje, at jeg heldigvis tror, vi også vil få bedre analyseværktøjer med den kommende teknologi, så vi vil formentligt også blive bedre til at analysere situationen i takt med at magten kommer til at kunne ændre på tingene. 



Jeg tror i øvrigt også, at fremtiden kan byde på gode økonomiske systemer / virksomhedstyper, der formår at gøre, så folk (altså alle mennesker) frit kan indsende nyskabende idéer til fællesskabet uden at tænke på, om det ikke var smartere at holde lidt på kortene, så man selv kan opnå flere IP-rettigheder eller noget i den retning, men hvor belønningen for nyskabende idéer for det meste nok skal være mindst lige så stor, hvis man udbreder sine idéer med det samme. Dette er et emne, jeg har brugt meget tid på for at prøve at finde en idé til en virksomhedstype eller et økonomisk system, der opfylder dette allerede tidligt i dens/dets dannelse. Men nu hvor jeg har mine seneste idéer omkring en mere kundedrevet virksomhedstype (grundidéen er egentligt gammel --- nok næsten lige så gammel som idéen omkring et system med stor idédelings-frihed) og har genovervejet mulighederne inde for denne idé, så tror jeg altså nu, at den sagtens kan står alene, og at opfinderfriheden bare vil følge med med tiden som en konsekvens. For når lønninger sættes på demokratisk vis, og når de dermed bliver bestemt mere i henhold til de globale interesser i samfundet og ikke bare i henhold til lokale interesser (f.eks.\ et fåtal af investorer), så vil der ikke være en fordel i at være nærig omkring belønningerne til opfindere (og innovatører og nyskabere generelt), og særligt ikke på en måde, der får disse til at holde på deres idéer i stedet for at dele dem til diskussion i fællesskabet med det samme.



Bankvirksomhed er også et område, hvor fremtidige teknologier kan gøre en forskel og være med til at gøre systemet mere lige. Teknologier som blockchain prøver allerede at opnå\ldots\ ja, noget i denne retning, men det er diskutabelt, hvad det præcist er, de nuværende blockchain-teknologier prøver at opnå. Men de påpeger i hvert fald, at der er et vist behov hos adskillige folk for alternative bankvirksomheder. Vores nuværende banksystemer bygger på en fælles konsensus og tro på opretholdelsen af visse lovmæssige anstalter. Men for de fleste er det lidt svært at gennemskue, hvordan disse nuværende systemer fungerer, og om de eventuelt fører en vis ulighed med sig, f.eks.\ hvis de nu gør, at der bliver printet penge til nogen instanser men ikke til alle i samfundet. Men når vi fremtidens diskussions- og analyseværktøjer bliver udviklet sammen med teknologier til nemmere at indgå store aftaler i store grupper, så bliver det det nemmeste i verden (selvfølgelig hyperbolsk sagt) eventuelt at udtænke nye banksystemer, hvis man mener, at dette kan gavne samfundet --- måske ved at gøre det lidt mere lige og retfærdigt.


Man kan i øvrigt sige det samme omkring alle mulige andre virksomhedstyper og instanser: Folk vil være i stand til at analysere hele billedet og træffe beslutninger, som ikke kun er optimeret til at fremme økonomien, men som også fremmer social bæredygtighed (og fremmer glæde generelt og bæredygtighed generelt). Jeg ser `social bæredygtighed' som betegnende det, at et system ikke fører til større og større ulighed og mindre retfærdighedsfølelse, men at de sociale forskelle konvergerer mod en opdeling med stor lighed i (og sandsynligvis mod større og større lighed, når vi når langt nok ud i fremtiden, men man kan også sagtens have plateauer/mellemstadier, hvor der er en vis (sund) ulighed, hvis man mener, dette er bedst for fællesskabet (hvad man jo nok vil)).   

Boligmarkedet (inkl.\ udlejning) er nok også et område, hvor de fleste ville have glæde af et mere åbent og demokratisk system ved f.eks.\ at skære mellemmænd fra. I det hele taget er der nok mange forretninger, hvor helt eller delvist monopol (eller andre årsager) gør, at der er et urimeligt markup på prisen, som kunne gøres meget mere til at betale, hvis man havde en demokratisk virksomhed på markedet. 

Reklamebranchen er også et område, hvor det ville kunne betale sig, at få virksomhedsledelsens interesser mere rettet på linje med befolkningens interesser generelt. Set udefra kan samfundet nemlig spare en masse krudt på ikke at skulle gå og prøve at forføre hinanden til at købe visse produkter frem for konkurrenten. Med p-ontologi-teknologien vil vi også blive i stand til som forbrugere bedre at kunne træffe forbrugsrelaterede beslutninger på et oplyst grundlag, så der vil i det hele taget heller ikke blive så meget behov for sofistikeret reklame i samme grad. Dermed er det ikke sagt, at al reklame er dårligt, for der kan jo også være noget nyttigt i, at folk kan inspirere hinanden til diverse trends og fascinationsområder, hvilket jo i reglen gøres bedst ved at spille på de mere underbevidste strenge. Så vi bør nok ikke gå helt væk fra al ikke-rationel reklame og anden influering, men der er dog stadig meget af det, som vi godt som samfund kunne være foruden. 


Hvis man læser mine tidligere noter omkring min ITP-idé (sektion \ref{ITP_feb-maj}), så har jeg jo også forklaret om, at jeg tror det vil blive nemmere at oprette nye virksomheder generelt, fordi man i højere grad kommer til at kunne følge open source, gennemprøvede skabeloner, som man kan læne sig op ad. I den forbindelse vil jeg så også nævne her, at der også i det hele taget er, mener jeg, et stort potentiale i mere open source virksomheder, bl.a.\ fabrik-virksomheder. For jeg tror at fremtidens marked vil og bør geares mere imod, at kunder kan customize (customize'e) deres produkter i høj grad, også de fysiske. Og virksomheder der har deres processer mere åbnet op udadtil kan også nyde godt af, at de kan modtage forslag til forbedringer og hurtigt kan få del i nye teknologier, som udvikles i open source fællesskabet. Fysiske maskiner kan jo også gøres (og er det sikkert også i høj grad allerede, men det har jeg jo ikke så meget forstand på) meget modulære, således at forskellige open source virksomheder sagtens kan dele nye maskinløsninger, også selvom deres maskinsystemer gør noget forskelligt, så på den måde tror jeg, at der kan være stor gavn i at indgå i et stort open source-fællesskab, også for produktionsvirksomheder. En fordel ved dette er så også, at man hurtigere kan lægge sine fabrikker om, hvis der sker ændringer på efterspørgslen, hvis dette bare handler om at gå over til en anden fabrik-skabelon, og om lige at udskifte diverse maskinmoduler med nogle andre, man har brug for i stedet. En sådan open source-bølge inden for produktionsvirksomheder kunne også i øvrigt være med til at gøre, at det bliver nemmere for u-lande at få gang i industrialiseringen (privatejet af lokale, så der altså ikke bare er tale om udenlandske og/eller internationale forretninger, der så tager al profitten med sig ud af landet), så at små virksomheder kan poppe op hurtigt og sætte gang i lokaløkonomierne. 



Jeg håber også, at `selv-overvågning' bliver langt mere normalt i fremtiden, hvor man altså i diverse ledelser, politiske partier, fabrikker, forskningscentre, nyhedsbureauer osv.\ sørger for at overvåge sig selv for at kunne vinde kunders/medlemmers/lægmænds/medforskeres/seeres/osv.\ tillid. Der bør altså være en høj gennemsigtighed i sådanne foretagender, og ikke bare ved at man kan se, hvem der varetager visse funktioner, men hvor man også kan overvåge, om de gør det korrekt eller ej. Det gælder også forskere/rapportere/undersøgere i felten, som så må bringe selv-overvågningsudstyret med sig rundt. Man kan så sagtens finde på systemer, der ikke bryder pågældendes privatlivsfred, og gør at diverse bemærkninger og samtaler omkring sociale aspekter og vedkommendes dagligdag, samt alle andre ting, man kan tænke sig, som de selv-overvågede ikke bør være tvunget til at broadcaste til omverden i deres dagligdag, kan sorteres fra, inden det bliver offentliggjort. Dette gælder både hvad foregår på en arbejdsplads eller på arbejdsmissioner. Man kunne f.eks.\ bare have nogen ansat til at skære al unødvendigt selv-overvågningsmateriale fra, og sørge for at disse har motivation både til ikke at skære for meget fra eller for lidt. Og hvordan kan man sikre sig, at de ikke skærer for meget fra, når nu offentligheden ikke kan se med så? Det kan man gøre på mange måder, bl.a.\ ved at have flere uafhængige filtreringsenheder i gang på samme tid, og gøre så at hver filtreringsenhed ikke ved, hvem ellers er i gang på et givent tidspunkt. Hvorvidt mere `selv-overvågning' bliver udbredt i fremtiden kommer så bare an på, om man kan gå sammen i diverse (nye) demokratiske ledelsesfællesskaber og nå til enighed om, at dette bør være en mere udbredt ting end pt. 


Og hvad så hvis selv-overvågne ikke lever op til forventningerne? Og hvad generelt hvis en ledelse ikke lever op til forventningerne. Jamen så må man jo give dem reprimander og/eller fyre dem og indsætte nye ledere. Men hvad hvis man ikke helt har magt til dette, f.eks.\ hvis der er tale om en ledelse af et politisk parti (og muligvis et lands regering også), hvor disse er valgt for en periode og ikke bare lige kan skiftes ud igen, og hvor man i det hele taget ikke har opsat måder, hvorpå medlemmerne kan holde deres ledere i ørerne løbende? Jamen så er man næsten også for dum, er svaret. Vi skal væk fra de nuværende konventionelle systemer for politiske partier, hvor politikere bare skal vinde folks stemmer én gang ved et valg, hvorefter de får tildelt magt i en fastlagt (lang) periode. Vi skal have mere dynamisk stemmemagt også i politiske partier. Og dette bliver heldigvis ret let at indføre, for det kræver bare et parti, der ikke markedsfører sig på specifikke holdninger eller idéer, men på en ledelsesstruktur, hvor medlemmerne løbende kan omfordele mandater for visse ledelsesretninger, som enten kan være repræsenteret af mennesker, men alternativt også bare kan være repræsenteret af (dynamiske) modeller, som den samme ledelsesgruppe så er kontraktbunden til at følge (med). På en måde har jeg således allerede nævnt denne idé, for idéen er jo således bare at bruge min idé om en dynamisk, demokratisk ledelsesstruktur til politiske partier. Men jeg synes alligevel idéen om at bruge dette på politiske partier er så vigtig, at den fortjener sin egen paragraf. Især fordi det vil være ret let hurtigt at danne sådanne partier; det vil ikke kræve den vilde teknologi, især fordi man ikke skal vægte stemmerne, og i modsætning til virksomheder med investorer osv.\ kan man for partier sagtens starte med nogle lidt blødere løfter om at implementere systemet, for hvis man ikke holder de løfter, kan en ny forening bare tage teten op og fortsætte, hvor den anden tabte tråden (også uden nogen IP-rettigheds-komplikationer). Og disse partier så også binder sig til en høj grad af selv-overvågning, så tror jeg virkeligt man kan vinde folks interesse hurtigt. Især fordi denne parti-type ikke udelukker de mere konventionelle partier; folk kan sagtens være medlemmer af begge. Folk skal således stadig have frihed til at give deres stemmer til repræsentanter, de har tillid til, eksempelvis ledelsen fra gamle, kendte partier. Forskellen er så bare, at hvis man stemmer det åbne, dynamiske parti ind først, så får man lige pludselig mulighed som vælger for løbende at rette sin stemme, hvis man bliver skuffet over sine gamle helte, og får i det hele taget mulighed for faktisk at indgå i forhandlinger efter valget også, og får altså mulighed for fortsat at bruge sine magt løbende.



Apropos selv-overvågning så kan dette også blive gavnligt for arbejdere, der gerne vil dokumentere, at de har gjort sig fortjent til deres timeløn, og i det hele taget gerne vil dokumentere, hvor meget energi de har lagt i et projekt. Dette kan også gøres på mange måder, og når VR-systemer i øvrigt bliver mere og mere almindelige arbejdsstationer (også for folk, der så kommer til at arbejde med at styre robotter m.m.), så bliver det jo ret nemt også at danne (nonintrusive) algoritmer til at (selv-)overvåge aktiviteten. 


Jeg har også en lille idé, når det kommer til selv-overvågning af de nederste lag af en organisation, dvs.\ arbejdere, (for)brugere, medlemmer, kunder osv. Her kunne man gøre det så at folk skal finde sammen i grupper, gerne med så ligesindede og ligestillede parter som muligt. Tanken er så, at folk i disse grupper giver hinanden mulighed for at overvåge hinanden internt, og at tilfældigt udvalgte stikprøver fra det samlede overvågningsmateriale så indimellem videregives til en enhed i et højere lag. Hvis denne enhed så opdager brud på protokoller og/eller på aftaler/løfter, som gruppen har givet samtykke til, så er tanken, at hele gruppen får en reprimande. Herved kan det så også undgås at følsomt data bliver lækket i denne proces; man kan så sagtens holde hele denne proces ret anonym. Gruppemedlemmer til en snyder/løftebryder har så bare ansvar for at handle korrekt, når integritetsbrudet opdages, og hvis disse ellers selv skjuler over det, jamen så får de selv reprimanderne. Det bliver så gruppens eget ansvar at skille sig af med medlemmer, der bryder deres løfter (så grupperne skal altså være dynamiske). Men hvordan skal man lige administrere at overvåge så mange arbejdsgrupper (for grupperne må jo nok helst ikke blive for store; så kan den interne overvågning ikke så godt lade sig gøre), og er det særligt smart hvis et fåtal af enheder får stor indsigt i alle mulige følsomme forhold. Nej, det er det selvfølgelig ikke, så et svar på begge ting kunne altså være også at opdele overvågningsenhederne i flere lag, således at overvågningsenhederne også overvåges med tilfældige stikprøver på samme måde. På den måde sikre man sig, at alle har motivation for at overholde protokollerne hele vejen op, men at en vilkårlige overvågningsenhed aldrig har kendskab til mere end et vist begrænset mængde `sladder' om de underliggende lag. (Og det bør så selvfølgelig være en del af protokollen, at folk ikke må sladre videre uhensigtsmæssigt om følsomt `sladder,' men må kun videregive oplysningerne, hvis protokollen kræver det i en stikprøve, og må ellers bare foretage de fornødne handlinger uden at lække oplysningerne til offentligheden.) I systemer hvor arbejdere kunne have forskellige grader af tillid, eksempelvis ved at visse arbejdsgrupper har mere stake i foretagendet/organisationen (og således større motivation for ikke at blive taget i fersk gerning), så kunne man endda lade dem varetage funktionen som overvågningsenhed for visse underliggende grupper i hierarkiet. Og da arbejdsgrupper på en måde ansætter sine gruppemedlemmer, ville det endda blive naturligt at have et system, hvor arbejdsgrupper prøver at stige i graderne i et hierarki, men hvor mere afslappede arbejdere også bare kan være tilfredse med en mere underliggende position. Så der er altså nogle forskellige muligheder her ved at danne sådanne arbejdsgruppe systemer, men pointen er altså her, at det kan bruges til selv-overvågning af medlemmerne er foreninger m.m. 

Der er et helt særligt punkt, hvor et sådant system kan blive vigtig for den fremtidige teknologi, som jeg kan forestille mig, og det er, når det kommer til at autentificere brugere som enkeltindivider. Dette bliver i første omgang vigtigt, hvis man har et system, hvor folk ikke bare har stemmeret efter hvor meget de betaler, men også har stemmeret (og/eller andre rettigheder) pr.\ næse bare. (Det løser nemlig end masse besvær i min idé omkring kundedrevne virksomheder, at stemmeretten bare kan være vægtet efter noget så måleligt som betalinger. Ja, og for politiske sammenhænge har man allerede gode metoder --- i hvert fald i mange lande --- til at ID'e folk i forvejen. Men for andre organisationer, især meget digitale organisationer, er det altså ikke nær så nemt at ID'e folk.) Og en anden rigtig vigtig sammenhæng, hvor det kan være vigtigt at ID'e folk og sikre sig, at én person ikke udgiver sig for flere, er, når det kommer til p-ontologier --- og til videnskabelige sammenhænge generelt, hvor man udspørger folk online. Det bliver således vigtigt, når man vil måle stemningen for diverse ting, og måle folks meninger generelt, at sikre sig, at folk ikke kan give sig ud for flere, end de er. Bemærk dog, at i alle disse sammenhænge, kan det også være smart, hvis dette kan sikres på en ret anonym måde, så man kan sikre sig, at folk kan give deres stemmer og meninger netop én gang, man samtidigt kan sørge for, at folks meninger kan holdes anonyme i statikkerne og ikke lægges til offentligheden. Så her kunne der altså blive rigtig stor behov for systemer a la det, jeg lige har foreslået, hvor integriteten kan sikres i interne processer, med uden at følsomme informationer bliver lægget til offentligheden. Så er der altså bare ikke nødvendigvis tale om ``arbejdsgrupper,'' men bare medlemsgrupper generelt.

Et trick man så eventuelt også kunne benytte til at vedligeholde integriteten i et sådant system med medlemsgrupper, der overvåger sig selv og hinanden på en ret sikker måde ift.\ ikke at lægge følsomme oplysninger udadtil eller til resten af organisationen, kunne så også være fra en højereliggende enhed at give midlertidige privilegier til underliggende folk om at bryde protokollen, og simpelthen hyre dem til at gøre dette. På den måde kan man så tjekke, om de mellemliggende overvågningslag fungerer (hvilket eventuelt kan afsløres i statistikken, hvis man ikke ligefrem vil ind og rode i, hvordan stikprøverne bliver udvalgt). Jeg har i øvrigt kaldt tricket at fordele ``påskeæg'' i systemet i mine noter, men man kunne også finde et andet udtryk for det. Altså et lille trick, der muligvis (jeg kan jo ikke vide det på stående fod) kan hjælpe til at tjekke løbende, at systemet fungerer korrekt.


En generel lille ting man i øvrigt kan sige om det at udvikle ledelses- og organisationsstrukturer, er, at man i princippet altid bør forsøge at identificere alle potentielle muligheder for integritetsbrud, og særligt sørge for at tage alle (eller i hvert fald så mange som man kan finde på) potentielle motivationskilder til at bryde systemets integritet hos diverse parter med i analysen og overveje disse grundigt.

Jeg nævnte at folk stadig kan bruge repræsentanter til at varetage deres interesser, også selvom de så dog beholder retten til at tage denne repræsentationsret tilbage til hver en tid. Og begrebet om at repræsentanter er jo heller ikke helt dumt (selvom det på mange punkter i praksis nærmest er `helt dumt' i de nuværende systemer\ldots), for så kan man derved spare en masse energi og arbejdsbyrder med at holde øje med, hvad der kan betale sig politisk set med ens stemme, når man vil bruge den til bedst muligt at opnå sine interesser. Princippet om at flere personer med lignende interesser kan gå sammen og give deres stemmer til en repræsentant, de har tillid til, er bestemt gavnligt, hvis man så også kan finde sådan en tillidsværdig repræsentant. Men problemet er så, at hvad hvis selve det at blive kvalificeret nok til at kunne varetage sådan en position, og bl.a.\ til at få indsigt nok i det pågældende system til at være i stand til at regne ud, hvordan man skal arbejde bedst for at kæmpe for sine repræsenteredes interesser, (hvad hvis det) kommer til at bringe repræsentanten ind i en anden klasse i samfundet, hvorved denne altså får nogle andre motivationer end dem, han/hun repræsenterer? Hvem siger så, at repræsentantens eget, således ændrede, motivationslandskab ikke i så fald kommer til at få indflydelse på dennes gerninger som repræsentant? Godt spørgsmål (tak, tak), men ikke lige et der er så nemt at finde en god løsning på. Ikke andet end at man så må sørge for virkeligt at sikre sig, at repræsentanten udfører god selv-overvågning (og at der er høj gennemsigtighed generelt omkring dennes arbejde), og at denne har høj motivation for ikke at lade yderlige personlige motivationer have indflydelse på dennes arbejde og beslutninger, eksempelvis ved at der er en reel risiko for at blive fyret, hvis man lader de personlige motivationer styre i visse sammenhænge. \ldots Hm, dette var egentligt ikke helt det, jeg ville komme ind på med denne paragraf\ldots\ Nå, jeg lader den stå. Men jeg havde egentligt mere i sinde bare at nævne, at når der så kan være sådan nogle skel i befolkningen korreleret med, hvor meget indsigt folk har i visse områder, hvor det ellers er vigtige nok for de befolkningsgrupper, der ikke har så meget indsigt i disse områder, stadigt at kunne kontrollere, at deres repræsentanter inden for disse områder ikke snyder dem, så kunne det muligvis være en god idé med en god tradition om, at pågældende befolkningsgrupper med jævne mellemrum bør udnævne folk til at bruge tid på at sætte sig ind i tingene (imod et stipendium), således at denne i sidste ende kan vende tilbage til sin tidligere tilværelse, men kan så forhåbentligt inden da bekræfte, at tingene foregår lødigt nok. \ldots Tja, bare en lille tanke, jeg følte var værd at nævne.


En lille idé, jeg også lige vil nævne hurtigt, er, at det muligvis vil være en god idé med en politisk samtale-platform, hvor folk anonymt kan erklære deres interesser, og muligvis deres meninger, og hvor platformen så kan finde interessante diskussionspartnere til en. Tanken er så at diskussionspartnerne ikke skal vælges så de i høj grad er folk med samme meninger og baggrunde, men snarere tvært imod. Hele tanken er, at jeg tror, det vil være sundt, hvis folk fik større mulighed for at tale på tværs af menings- og befolkningsgrupper, så man bedre kan prøve at forstå hinanden, og at der så på en eller anden måde også blev en god tradition blandt folk for så at benytte sig af dette (hvis man altså ligesom kan få folk godt med på idéen; hvis de også synes, det er en god idé). Internettet har pt.\ en ret splittende og konfliktskabende indflydelse på vores (gennemsnitlige) interaktioner. Og som enkeltperson kan det være meget svært at forstå, hvorfor i alverden folk i god hensigt kan mene visse andre ting end en selv; ``de må da være hjernevaskede eller noget, de mennesker??'' (kunne man spørge sig selv). Men hvis man så får mulighed for at snakke med disse mennesker, og især hvis man ligefrem prøver at fokusere på de ting, man har til fælles, først, så vil ofte dermed kunne udvide sin egen horisont (også selvom man ikke nødvendigvis skifter mening!) og altså ende med at forstå andre menneskers bevæggrunde bedre. Igen bare en lille tanke. %En mulighed der måske også kunne gøre platformen mere kommerciel kunne så være også at opnå op for muligheden for at broadcaste sine diskussioner, men pointen er dog stadigvæk mere, at samtalerne skal foregå iblandt almindelige mennesker, der ikke har ydre motivationer for at "vinde" en diskussion, og det er også bedst, hvis man selv indgår i samtalerne i stedet for bare at se på andres samtaler. 
*Forresten kunne denne idé også være brugbar, når det kommer til folk der gerne vil lære andre sprog og/eller om andre kulturer. Som det er nu kan det (så vidt jeg ved) nemlig være svært, når man gerne vil lære et andet sprog (og om et andet land generelt), at man mangler nogle lokale at snakke med. Og det er på trods af, at der nok altid vil være mange i et land, der vil være lykkelige for at snakke om sit eget land og sit eget sprog til fremmede (især hvis personen også er interesseret i det modsatte sprog, når det kommer til sproglæring). Så denne mulighed gør nok faktisk i sig selv, at sådan en platform potentielt vil kunne have store kommercielle muligheder (som jeg umiddelbart ser det).


Jeg har også lige nogle eksempler på nogle områder, hvor p-ontologier kan blive rigtigt anvendeligt. For det først bliver IT-sikkerhed et område, hvor p-ontologier-teknologien kan hjælpe meget, og også bare sikkerhed generelt. Ontologier over alverdens produkter samt deres priser, produktoplysninger og brugervurderinger bliver helt sikkert også en rigtig nyttig ting. Jeg tror også at en social-netværksagtig ontologi over politiske profilers samt almindelige folks meninger og holdninger til forskellige ting, og ikke mindst deres argumenter for og imod. I det hele taget tror jeg denne idé kunne være gavnlig, uanset den underlæggende teknologi; et politisk orienteret socialt netværk, hvor folk kan stille sig frem med argumenter --- evt.\ også bare hvor man leger djævlens advokat --- og hvor folk så får mulighed til at poste spørgsmål og modargumenter (og med-argumenter) til ens profil. Pointen er så, at disse meningsprofiler skal rates af andre brugere, bl.a.\ ud fra, i hvor høj grad meningsprofilen formår at opretholde en god aktivitet, hvormed relevante modargumenter og spørgsmål ikke bare bliver ignoreret af profilen over længere sigt. 
*Hm, det kan være at en lægevidenskabsontologi også kunne være gavnlig for folk, så man nemmere kan undersøge og få svar på sygdomssymptomer selv. Og ja, man kan sikkert finde på mange, mange flere eksempler.


Her til næsten sidst i denne sektion vil jeg også lige for det første understrege, det jeg nævnte før, om at man muligvis som nystartet virksomhed, der eksempelvis prøver at markedsføre sig på, at kunderne med tiden skal tage mere og mere over, kunne overveje at udlove belønninger (f.eks.\ i form af aktier og/eller muligvis ved basalt set at shorte sin egen aktie til dem, som jeg beskrev før) til de tidlige kunder og/eller arbejdere (eksempelvis programmører) og/eller andre mennesker, der er med til at skabe hype og interesse omkring virksomheden og/eller er med til at sætte skub i den generelt.


%Hm, hvad er egentligt værd at sige om temporary closed source, som ikke allerede er trivielt; selvfølgelig skal der helst være en udløbsdato på IP-rettigheder osv..? Jeg kunne lige understrege, at der bør gælde for enhver virksomhed, hvor kunderne/brugerne bidrager med meget indhold, at kunderne så med tiden vil overtage en passende mængde af aktierne og/eller medbestemmelsesretten, og at hvis ikke dette er tilfældet umiddelbart, må kunderne altså gå sammen om at sørge for, at det bliver sådan, men jeg tror allerede det fremgår fint af teksten hidtil..


Til sidst i denne sektion vil jeg så lige komme ind på spørgsmålet om: hvad gør man, hvis man har en eksisterende (mere normal, om vi vil) virksomhed og indser at mere kundedrevne virksomheder nok er vejen frem, også for ens egen branche, og gerne vil deltage i bevægelsen, men også gerne alligevel vil bevare sin virksomhed i store træk, som den er nu --- i hvert fald på kort sigt? Er der en måde, hvor man selv kan tilslutte sig lidt blødt til bevægelsen, og begynde at give kunderne større bestemmelse på sigt? Ja, det kan man jo faktisk sagtens, for det handler bare på samme måde om lige at fastsætte et system, hvor særligt de mest betalende kunder gives mere og mere medbestemmelses magt, og så på et tidspunkt (ikke alt for sent) binde sig juridisk til dette system. Og hvis man gerne vil bevare tingene, som de er i rimeligt lang tid, så kan man bare lade kurven over den samlede kundemedbestemmelse stige tilsvarende roligt i dette system. (27.05.21)


*(26.07.21) Jeg kan ikke huske, om jeg har fået understreget det, men hvis ikke, vil jeg bare lige nævne, at selvom jeg nok har tænkt meget i baner af økonomisk ``retfærdighed'' i det her, så er det altså slet ikke det eneste mål. Det er i stedet bare en del af vejen til, hvad jeg tror også kan blive en langt mere teknologisk effektiv og produktiv økonomi i fremtiden. Det handler jo basalt set om, for det første at give værktøjerne til, at folk i fællesskab kan analysere situationer og nå frem til løsninger i fællesskab, og for det andet handler det om, at folk kan finde frem til en fælles retning ved at man får løst og fundet kompromiser for alle modstridende interesser i befolkningen på en effektiv politisk vis, så man derfra kan fokusere om at arbejde sammen om de samme mål, hvor kompromiserne altså allerede er faktoreret ind, så man ikke længere skal bekymre sig om denne politiske del. Jeg tror nemlig på, at med fremtidens analyseværktøjer (som kommer med det semantiske web (løbende), som jeg tænker det) samt værktøjer til dynamisk at gøre sin stemme gældende politisk (og hvor man nok ofte bare kan afgive sin stemme én gang for alle, medmindre man altså skifter mening), gør at man relativt hurtigt kan finde frem til, hvad de modstridende interesser er helt præcist, og få indgået de fornødne kompromiser som samlet samfund. Og når først man har sig sine fælles mål (hvor alle har haft deres stemme) og har analyseværktøjerne til virkeligt at arbejde effektivt sammen på tværs af samfundet om at finde løsninger, så tror jeg altså på, at dette virkeligt vil sætte skub i udviklingen.


%*(18.08.21) Jeg skriver bare lige denne paragraf herude i kommentarerne, men jeg har også før haft tænkt på, hvis nu vi (forhåbentligt) går en fremtid i møde, hvor teknologier er meget mere open source, om der så ikke kunne være en praktisk idé i at bygge moduler til diverse konstruktioner endnu mere alsidigt ved indføre et fast interface-skema for, hvordan ting kan sættes sammen. I bund og grund tænker jeg altså, om ikke det i såfald kunne være en idé at gå over til en slags real-world lego, i.e.\ hvor vi designer en masse alsidige moduler, der kan sættes sammen til at danne, hvad end vi har brug får (næsten). Så spørgsmålet er altså, om det (givet at alt er mere open source) vil være værd sikkert at ofre lidt ift.\ hvor kompakt og effektivt, vi bygger vores maskiner og andre konstruktioner, imod at modulerne så sikkert bliver billigere at producere og samle. Bare en tanke. 



\subsubsection{*Yderligere tanker omkring kryptovaluta}
*(Det blev til ret meget ekstra tekst omkring kryptovaluta, så nu gør jeg det lige til sin egen sektion.) *((23.06.21) Jeg har besluttet, at jeg ikke længere tror så meget på vigtigheden af mine KV-idéer, men da denne undersektion første til nogle gode idéer, nemlig dem i 11/06-paragrafen (og læs 10/06 med), så udkommenterer jeg den altså ikke.)\\
\\
(03.06.21) Lige en ekstra ting (som jeg slet ikke er særlig sikker på, men som måske er værd at nævne): Jeg har overvejet tanken om et kryptovaluta-system, hvor folk kan lave udstede mønter/tokens til sig selv baseret på handlinger, de selv har lavet, og hvor værdien så kommer til at afhænge af, hvor meget lykke/gavn denne handling bringer til verden. Dette system, hvis det kan lade sig gøre, kunne muligvis gavne verden ved så i højere grad at gøre folk villige til at lave spontane gode handlinger. Jeg har dog ikke rigtigt kunne få idéen til at holde indtil videre (i den forrige sektion om \textbf{Økonomien omkring det semantiske web og i det hele taget} prøvede jeg at lægge op til et forsøg på at retfærdiggøre holdbarheden af et sådant system, men droppede det igen, da jeg skiftede mening). Nu vil jeg så nævne, at jeg faktisk tror, at der måske godt kan være en mulighed for sådant et system --- ikke bare i en fjern fremtid, men muligvis også i en nær. For man kunne jo bare skabe et simpelt system, hvor folk kan udstede mønter/tokens, hvor de hævder at de har gjort visse handlinger, og så kan det jo i og for sig godt være, at folk vil være villige til at bakke op om dette system (hvis mange nok mener, at det rigtignok kan føre meget godt med sig), og at der derfor vil være en stor vilje til at anerkende pågældende mønter/tokens for at have den pågældende værdi, som systemet foreslår, hvilket jo så dermed kan medføre, at det effektivt kommer til at få denne værdi. Da det må formodes at meget lykke/gavn i fremtiden kommer fra mere eller mindre automatiske kilder, og ikke direkte fra medmenneskers arbejde for at gavne en selv, så må man også formode, at et sådant system i fremtiden vil medføre at gavnlige handlinger for andre mennesker som regel vil fortjene (i pågældende fremtid) mere gavn til en selv. Samtidigt er der dog en risiko i, om dette system overhovedet kommer op at køre set med nutidsøjne, hvilket altså vil presse værdien nedad i starten. Men dette gør så også tilgengæld, at det kan være interessant at investere i, hvis man som investor tror på, at folk vil bakke mere og mere op om systemet, for så vil værdien af mønterne/tokens'ne jo stige sammen med denne opbakning. Igen: bare en lille idé, jeg synes var værd at nævne. Om ikke andet er det rart at tænke på, at vi som samfund nok på et tidspunkt skal udregne et økonomisk system, som effektivt fungerer på denne måde (så reel gavn-/lykke-skabelse bliver fundamentet for det økonomiske system, og ikke alle mulige andre arbitrære ting). *((10.06.21) Uh, måske har jeg en udvidelse til denne idé. Se længere nede (hvor datoen når d.\ 10/06).)

Ah, man kan da også bruge mit dynamisk-model-demokrati-system (med stemme kraftfelter) til et KV-system. Jeg tror jeg har undgået tanken lidt, fordi jeg tænkte, at det ville være for nemt at få skabt positive feedback til de rige brugere, men det kan man jo godt undgå ved bare at antage nogle grundlæggende ting om modellen, så den dynamiske afstemning sker omkring lidt mere sikre ting. Hm\ldots\ Og KV-systemer er jo også lidt interessante for tiden, hvis bare man kan finde en løsning, der ikke er så energikrævende, men stadig er lige så decentral som PoW-kæder (og måske er det også godt at sørge for, at den stadig har miners ret meget som i PoW-kæder, hvis man vil ramme den samme åre af interesse). Hm\ldots\ Tja, det det kan jeg jo eventuelt tænke lidt over i min ``fritid,'' hvis jeg har lyst. Nå, men angående en dynamisk model for et KV-system, så kunne dette jo muligvis også være vejen til, hvad der svarer til min kundedrevne virksomhed, bare hvor folks ``bidrag til omsætningen'' nu udskiftes med mønt-stake i KV-systemet\ldots\ Og man kunne vel så sørge for som fællesskab at finansiere virksomheder, der lover at sælge produkter for KV'en\ldots\ He, her er endnu en gang noget, som ligger meget, meget tæt op af en tidligere idé, men hvor jeg så nu måske kan se det i et lidt andet lys\ldots\ \ldots\ Hm, ah ja, så KV-systemet lidt egentligt bare kommer til at fungere som en investeringsfond, bare hvor investeringsbeslutningerne tages i en online, demokratisk (dog ikke med stemme pr.\ næse, men pr.\ mønt), og hvor man dog ikke kan få pengene tilbage, hvis systemt crasher, men hvor det tilgengæld så er nemt at købe sig af og på. Det var da egentligt en meget god idé: at danne en demokratisk (igen dog med stemmer vægtet efter stake) og bureaukrati-mæssigt effektiv investeringsfond i form af et KV-system. Det ville nok ikke være helt dumt så også at bruge min idé om, at tillade at ikke-formaliserede NL-udsagn kan blive afgørende for kontrakters udfald (og andre forhold, der kan bestemme pengefordelingen), så man nemt kan udstede selv komplicerede smart-kontrakter i fællesskabet. For hermed behøver man ikke at gøre sig afhængig af tredjeparter til at bedømme komplicerede kontrakter, og hvis det alligevel bruges som en grundlæggende del af systemet, så vil ingen af dets medlemmer være interesserede i på nogen måde at nå til et punkt, hvor den tiltænkte semantik bag bare én velformuleret kontrakt eller tilsvarende bliver ignoreret og fejlfortolket med vilje, for så vil hele konceptet bag systemet kollapse. Så selvom folk midlertidigt godt kan agere, som om en kontrakt eller vedtagelse mente noget andet, end hvad den tydeligvis gjorde, så vil det samlede fælleskab alligevel i sidste ende vurdere det korrekte om kontrakt-/vedtagelses-semantikken (og derfor vil det ikke give mening for lokale grupper at lade som noget andet). Hvis man derimod ikke tør at bruge sådan en type KV-system, så må man så bare benytte tredjeparter i sine smart-kontrakter og vedtagelser, og derved altså bare bringe dem til konklusion via diverse certifikater (hvorimod konklusioner i det mere bløde system aldrig 100 \% bringes til konklusion, men hvor konklusionen af dem altid i bund og grund er et fortolkningsspørgsmål). Selvom man i så fald ikke behøver at afhænge af tredjepartsdommere til sine kontrakter, så kræver det dog lidt mere arbejde, hvis man gerne vil gøre, så at fortolkningen af NL-udsagn kan være betydende for pengefordelingen til et givent tidspunkt, fordi man så også lige skal sørge for at sætte en klar lovmæssighed op for, hvad man gør, hvis udsagn nu ikke er velformuleret, eller hvis de ser ud til ikke at kunne besvares inden for en overskuelig fremtid. Jeg gav dog vist nogle bud på, hvad man kunne gøre her, i KV-sektionen ovenfor i dette notesæt. Cool nok. (03.06.21)

*(15.06.21) Jeg har lige indset at min idé om at oprette en investeringsfond på baggrund af en kryptovaluta ikke afviger fra at oprette et firma, bare hvor kryptomønterne erstattes med aktier. x) 

(08.06.21) Nu kom jeg faktisk lige på nogle flere ting omkring blockchain i går aftes. For det første \emph{bør} en god kæde jo faktisk indeholde NL-fortolknings-semantik i pengefordelingsfunktionen, for ellers kan man jo ikke undgå i fremtiden (givet at kæden skal kunne fungere som en fremtidsvaluta), folk ikke bare kan udstede lejemordskontrakter i øst og vest. Det vil altså så blive et kæmpe problem for kæden, hvis den ender med at være kilde til grov kriminalitet, og hvis altså hele fundamentet omkring kædens værdi bygger på, at sådanne smart-kontrakter\ldots\ Nå nej, man kunne jo argumentere for, at man bare som samfund kan retsforfølge diverse tredjeparter, der underskriver\ldots\ information omkring verden --- nej, den går heller ikke, for det kommer alle informationshjemmesider og nyhedshjemmesider jo til at gøre, så nej, det går slet ikke. Så derfor er det nok en god idé faktisk at sørge for, at friheden på kæden bliver begrænset ift.\ smart-kontrakterne. Det gode er så, hvis min idé om at bruge NL-fortolkninger for blockchain- (og særligt pengefordelings-) semantikken altså holder, at dette stadig så kan gøres på en ikke-central måde. For man kan så bare (``bare;'' det kommer sikkert til at kræve en del omtanke) erklære skarpe sætninger omkring pengefordelingsfunktionen, der siger, at smart-kontrakter ugyldiggøres, hvis det opdages, at de har indeholdt kriminelle væddemål og/eller betalinger (hvor man så definerer fra start, hvad der er kriminelt i kædens øjne), og hvis det erklæres på kæden (af vilkårlige personer) inden for et vist tidsrum efter at smart-kontrakten udløstes. Hm, men ville dette være besværet værd, for kriminelle kan jo bare bruge deres egne blockchains osv.?\ldots\ Ja, men så har man så som samfund mulighed for at prøve at bekæmpe de kommunikations- og betalings-netværker, uden at det går ud over det pengesystem, som almindelige mennesker bruger. Så er NL-fortolkninger ikke herved potentielt ret vigtige, hvis man vil have en fremtidsholdbar kæde? \ldots\ Tjo, det kan man i hvert fald lave et ok argument for.

Og det andet er så, at jo, man kan godt lave en energiforbrugs-holdbar PoW-kæde! Man skal bare gøre så at møntejere kan melde deres møntbeholdning til forskellige grupper med forskellige mining-belønninger (og måske også med forskellige blokskabelseshastigheder osv.), hvor mønterne så bare tages fra pågældende brugere i den gruppe. Og hvis man gerne fortsat vil have en voksende møntmængde på kæden, så kan man bare justere fratrækningen relativt til, hvad mining-belønningen bør være, hvis alle nye mønter skulle skabes hos minerne alene. Men mon ikke det er bedre for en kæde bare at have konstant møntmængde? For man får jo ikke mindre handel på kæden af denne grund, for minerne vil stadig typisk skulle sælge deres mønter, efter de har fået dem, uanset om de opstod i mining-processen eller blev overført fra andre konti i processen i stedet. Man bør i øvrigt også gerne kunne danne grupper, som kan definere deres egne protokoller for, hvornår belønningen stiger og falder, enten automatisk (måske periodisk) eller på bagrund af forespørgsler fra brugerne (som så måske skal godkendes af den samlede gruppe), hvis nu brugerne f.eks.\ pludselig skal udføre mange betalinger på én gang, eller hvis det af andre grund skal gå hurtigt. Nogle grunde til at det måske godt kunne være smart at sætte en periodisk voksende og aftagende belønningskurve, kunne være, at man herved kan aktivere flere minere, fordi det så ikke bliver et klart ja/nej spørgsmål for den individuelle miner, om det kan betale sig, men mere et spørgsmål om hvornår. En anden ret vigtig grund (men som også kunne bevirke det samme i øvrigt) kunne være, at man gerne vil fremstå miljøvenlige, og at man derfor sætter en højere belønningsrate om dagen, hvor strømforbruget ikke koster strøm netværket ligeså meget, fordi lagringsteknologien jo ikke vil være fuldt udviklet før om lang tid, hvilket gør at de vedvarende energikilder bedst kan bruges om dagen. *(Hm, men mineren kan jo ikke bevise, hvor denne er henne, og kan jo derfor snyde sig til en lokation i en dags-tidszone\ldots\ Hm, kan man lave et system, så noderne skal vælge en lokation?\ldots\ Hm, ja det kunne man faktisk. Hvis man eksempelvis laver et system, hvor noderne skal arbejde sammen med nabo-noder for at mine mønter\ldots tjo, men så vil noderne jo bare placere sig i klynger\ldots\ Nå, men ellers kunne man måske også bare sige, at hver node skal vælge en aktivitetscyklus med et voksende og aftagende aktivitetsniveau, som noden så skal overholde i noget tid, før den kan begynde at få dagtime-rabatter. Hvis denne aktivitet så justeres sådan, at det så kun kan betale sig at have en node, med aktivitetscyklus afstemt med den faktiske lokation\ldots\ nej vent, for det kræver at elselskaberne i forvejen har kurvecyklusser på elpriserne, og så løses problemet jo bare her. Tja okay, men måske skal man så bare sige det som kæde-fællesskab; at hvis samfundet vil have mere miljø-venlige løsninger, så må de jo bare overtale alle el-selskaber til at sætte varierende kurver på deres el-priser. \ldots Ah, men man kunne dog godt lave et system, hvor man kan kortlægge noderne, ved at løbende at kræve, at noderne selv sætter en afstandsbedømmelse til forskellige tilfældigt udvalgte andre noter, hvor de så aldrig kan betale sig at vurdere en for stor afstand eller for kort afstand (hvilket bl.a. må kunne lade sig gøre med et slags prisoners' dilemma-agtigt spil), og hvor det så \emph{særligt} ikke kan betale sig at være en node, som andre noder ofte gætter en forkert afstand til. Ja, så man vil kunne lave et sådant system, men ellers kunne man jo også bare bruge argumentet om, at det mere må være elselskabers ansvar i stedet.) Bemærk som læser, at strømforbruget omkring blockchains er et emne, der er meget aktuelt lige for tiden, så en løsning, der kan gøre blockchain-teknologien mindre energiforbrugende, er nok (vil jeg tro) også i særlig høj interesse for tiden. 

Så nu kan jeg altså både fremføre en idé om at bruge en blockchain som en slags investeringsfond, hvor det så i øvrigt også kunne være særligt smart med mulighed for at bruge NL-fortolkninger lovmæssighederne. Generelt har jeg så også et argument for, hvorfor NL-fortolkninger ikke bare kan være smarte, men måske endda ret vigtige for en god kæde. Dertil har jeg så også en mulig løsning på energiforbruget til PoW-kæder, som så faktisk er helt uafhængig af min NL-kædesemantik-idé. Og sidst (og måske også mindst) har jeg så også lige en idé omkring at lave en kæde, hvor folk kan tokenize ``gode handlinger,'' hvor disse handlinger enten kan indebære generelle gode handlinger ud fra nogle omfattende etiske (medmenneskelige) grundsætninger, eller mere en særlig mængde gode handlinger, der relaterer sig mere til kædens vækst og dens medlemmers glæde og gavn --- eller en blanding af disse to, så man både kan tokenize generelle gode handlinger og mere kæde-relaterede handlinger (i forskellige kategorier). (08.06.21)

Uh, og kunne man ikke også lave noget tilsvarende i forhold til graden af PoS?! Hm, tjo, men på den anden side, så kan dette jo allerede sikkert lade sig gøre med state chennels, ikke?\ldots\ Jo, men derfor vil dette nok slet ikke være en dum idé at inkorporerer det som en grundlæggende ting i kæden, så brugerne nemt kan gøre dette dette, uden at skulle udforme alle mulige kontrakter hver især, og blive enige om, hvordan det skal gøres i pågældende gruppe. I stedet kan man så bare have nogle standard måder, hvorpå folk kan melde sig til forskellige grupper (altså med forskellige PoS-protokoller), uden først at skulle kontakte andre brugere off chain og danne en masse kontrakter sammen (i stedet skal de bare lige vælge nogle parametre for den gruppe, de vil tilslutte sig). Ja, jeg tror sgu altså næsten, at man kan komme rigtigt langt bare med denne lille idé om en kæde, hvor brugerne frit kan slutte sig til (og skifte imellem) alle mulige grupper med forskellige mining-protokoller (og særligt altså med forskellige protokoller for, hvor store belønninger skal gives til minere, men også hvordan hele mining-processen foregår). (08.06.21)

Hm, jeg mangler så faktisk lige at finde ud af, hvordan gør, at minernes arbejde for en gruppe mest kommer denne gruppe til gavn for det første, men også bare i det hele taget hvordan den samlede kæde så hænger sammen. For grupperne ville jo så komme til at svare ret meget til state channels, men så skal man jo netop finde ud af, hvem der betaler for main chain-minearbejdet og hvordan\ldots\ Hvis bare man kunne koble disse ``state channels'' helt af main kæden (midlertidigt)\ldots\ Dette har så lidt fået mig til at tænke på også: gad vide om man i øvrigt med NL-fortolkninger helt kan undgå 51 \%-angreb?\ldots\ Hm, jo men måske kan man endda også forhindre dette uden NL-fortolkning\ldots(!) Man kunne jo måske have et system, hvor noderne løbende godkender hinanden i fællesskabet, når de aktiveres og re-aktiveres, så man i det samlede fællesskab monitorere nodeaktiviteten og i samme omgang også autoriserer pågældende node-aktivitet. Så skal man altså bare sørge for, at noder generelt har gavn af, at registrere andre, nyaktiverede noder, samt monitorere gamle noders aktivitet, således at fællesskabet således kommer til ad naturlig (egoistisk) kraft kommer til at være inklusiv. Og så er pointen så ellers bare, at man gør det til en del af kæde-lovmæssighederne, at aktiviteten skal have været registreret\ldots\ Hm, hvad hjælper dette?\ldots\ Tja, det kan jeg lige tænke lidt mere over, men for at vende tilbage til NL-fortolkningerne, så er det jo faktisk super nemt at undgå 51 \%-angreb her, for så kræver det bare, at man erklærer, at kæden skal have været offentlig (og så må man bare lige definere, hvad dette betyder ret præcist), og fordi sådanne spørgsmål netop ikke skal bedømmes løbende (ifølge min idé), men hele tiden bare vil afhænge af, hvad man mener, at de i fremtiden (ærligt) vil mene om spørgsmålet, så kræver kæden altså stadig ikke nogen central enhed til at bedømme sandheden omkring sådanne fortolkningsspørgsmål. Og dette, mine damer og herrer, vil jo så være en super nem måde, at nedbringe energiforbruget på, fordi man så kan være helt ligeglad med skumle 51 \%-angreb (fordi man jo pr.\ kædens grundlæggende fortolkning vil skulle se bort fra alle delkæder gjort i smug (når altså det kommer til main chain'en)). *(Uh, og noget andet er: hvis man kan være ligeglad med 51 \%-angreb, så kan man jo også frit lave state channels, som kan være helt ligeglade med main chain, inklusiv pengefordelingen og møntdannelsen på den ellers, så herved opnår man altså med ét slag en universel kæde, ud fra hvilken folk frit kan danne uafhængige state channels med andre protokoller! Så alene ved denne NL-fortolkning kan man altså opnå, hvad der svarer til en ``selv-opdaterbar'' kæde. (Så skal man i øvrigt bare lige sørge for, at definitionen på ``offentlig'' er så skarp og velovervejet, at folk heller ikke kan angribe denne (ved f.eks.\ at forsøge at lave grænsetilfælde --- så man skal altså bl.a.\ sikre sig, at forekomsten af eventuelle grænsetilfælde ikke bliver skadelige for den samlede kæde og de almindelige brugere)). *(Uh, angående dette, så kunne man måske benytte, at lovmæssighederne for at kunne tale om, hvad den offentlige kæde var på et givent tidspunkt, ikke behøver at gøre dette muligt til hver en tid. Hvis man benytter dette, gør det således ikke noget, hvis det til nogen tidspunkter vil kunne være tvivl om, hvilken en kæde er den rigtige offentlige kæde, hvis der er flere konkurrenter; så må brugerne bare vente på betalingerne i denne tid, indtil der pludselig igen vil være en klar vinder pr.\ de pågældende (NL-)lovmæssigheder.)) Okay, men lad mig så alligevel gå tilbage til at se på mulighederne for kæder uden NL-fortolkning. Min origanle tanke omkring grupperne var jo bare, at mønt-ejere så selv må betstemme, hvor meget de betaler til minerne, og da der så vil være nogen, der vil være interesserede i mining-arbejdet (jo nemlig når en bruger venter på at en betaling cementeres i kæden), jamen så vil der være brugere, der er klar til at betale? Er dette så ikke en fin idé sådan, eller skal man overveje, hvordan man kan motivere gruppen samlet set om at betale miners. Ville dette system forringes ved, at der så kommer ``freeloader,'' eller vil ``freeloading'' blive en helt uskyldig del af kæde-fællesskabet (for hvis man bare har sine penge stående stille, så har man jo heller ikke brug for mining-arbejdet)? Hm, uanset hvad, kunne man måske også alternativt bare have et system, hvor brugere betaler alt efter, hvor aktive handlere, de er\ldots? \ldots Uh, og kunne man så ikke bare i samme omgang sige, at noder kun sjældent må skifte gruppe?\ldots\ Eller kun gøre det sidstnævnte\ldots? \ldots Tja, tjo, men hvad hvis der oven i købet også bliver en gruppeskifte-afgift?\ldots\ Og hvis man så måske kun kan handle primært med brugere fra den samme gruppe\ldots? 

(09.06.21) Måske bliver freeloaders faktisk et problem, hvis man lader folk selv helt frit bestemme, hvor meget de betaler til miners. Jeg kom i tanke om i går aftes, at man dog måske kunne have en demokratisk (PoS-vægtet, selvfølgelig, og gerne med ``stemmekraftfelter'') protokol til at bestemme mining-belønningskurven. Og ja, her er da en mulighed, men der er jo også bare generelt et problem med PoW-kæder: Man kan ikke få energiforbruget under en vis størrelse, for det må aldrig kunne betale sig at lave angreb, hvor man bygger en længere kæde end den officielle i smug. Derfor skal energiforbruget gerne være måleligt i omkostninger med, hvor mange penge bliver handlet på kæden til det givne tidsinterval. Og dette, nu hvor jeg skriver det (og tænker ordentligt på det), er jo virkeligt et problem, for så kunne man ligeså godt bare købe strømmen eller regnekraften fra minerne; det ville give en et beløb måleligt med, hvad der bliver overført til en, hvilket minerne alligevel bruger strømmen og regnekraften på at få til at ske. Jeg skal lige gennemgå argumentet ordentligt, men håber faktisk virkeligt, at jeg har ret, for et klart argument imod PoW-kæder er faktisk noget af det bedste, jeg har kunne håbe på i denne forbindelse --- jeg har tænkt på, at det ville være vildt dejligt at have, men har ikke turde håbe på at finde det --- for så kan jeg jo bare for alvor fokusere på mine NL-fortolknings-idéer, og ville så kunne lave et ekstra stærkt argument for dem. \ldots Hm, tja, måske går den heller ikke; man kunne sige det, hvis alle transaktioner involverede den samme instans i et tidsinterval, men ellers vil det ikke kunne betale sig for den enkelte handler at bruge computerkræfter på et angreb, for så skal vedkommende også betale, hvad der svarer til, hvad alle andre handler kræver i form af mining-arbejde. \ldots Medmindre de handlende parter arbejder sammen om angrebet\ldots Og det ville vel ikke være så svært at bygge en form for ``kriminel'' nodetype, så alle opportunister kan danne et skummelt delnetværk, der giver sig ud for normale noder, men som indimellem snakker sammen om, hvorvidt et angreb kan betale sig --- og måske også arbejder sammen om at frembringe gunstige forhold for et angreb --- for så, hvis forholdene ser gunstige ud, at gå sammen om at lave et angreb. Hm, har jeg lige fundet en speciel ny angrebsmetode imod PoW-kæder (i form af et slags 51 \%-angreb, men ikke med én instans; med mange instanser, der arbejder sammen, i stedet)? I så fald vil min lille drøm i denne sammenhæng (som jeg lige nævnte) være gået i opfyldelse\ldots\ Ja, det ser altså sådan ud\ldots\  (09.06.21) Jo, men 51 \%-angreb bliver også bare i det hele taget nemmere, hvis man vil prøve at indføre strømbesparelse i kæden, for så skal man jo kun kæmpe med de noder, der er aktive som minere på pågældende tidsinterval. Der må i øvrigt også være andre opportunist-angreb, man kan lave med et ($>$)51 \%-netværk af skumle noder (eller `knuder' skulle jeg måske sige). Minere må således kunne forøge deres indtægt ved sådan skummel aktivitet (hvor man altså f.eks.\ sørger for løbende at prøve ignorere ny-minede blokke udefra til fordel for blokke inde fra delnetværket selv). 

Nå ja, og fik ikke nævnt, at PoW-kæder også må skulle slås med angreb, hvor man får blokke til at indeholde ulovligt digitalt indhold (f.eks.\ copyrighted). Dette kan i øvrigt bruges som led i et 51 \%-angreb på en strømbesparende PoW-kæde. Men for NL-kæder bliver det ingen sag at sørge for, at man løbende kan implementere protokoller til at erstatte gamle delkæder ved at rotere nogle ækvivalente blokke ind i stedet for. Og desuden kommer NL-kæder også bare til at fungere langt bedst til investeringsfond-kæder, som jeg ser det. Ja, \emph{langt} bedst\ldots\ For med en kæde, hvor integriteten af fortolkningen af NL-lovmæssighederne er vigtige, og altså særligt også når det kommer til de demokratiske beslutninger på kæden, så kan man herved jo som kæde ret let lægge vægt bag diverse løfter, også endda uden at bakke dem op juridisk (eksternt) først. Herved kommer man altså til at kunne fungere ret effektivt, når det kommet til fra beslutning til handling, og man vil let kunne udstede vægtige smart-kontrakter i fælleskab, som automatisk kommer til at medføre en betaling til firmaer, der f.eks.\ sælger deres aktier til kæden --- og ikke mindst i øvrigt til iværksættere, der indledningsvist udlover aktier til kæden imod startkapital. Investeringsfond-kæde-idéen kan endda blandes med kunde-medbestemmelses-idéen også, fordi man jo kan vedtage i den demokratiske kæde, at kunder får tokens med i købet, der også giver stemmeret (ligesom at de normale kæde-mønter gør det). (09.06.21)

(09.06.21) Men det er vel dog stadig en god idé at have en opdaterbar/justerbar PoS-kæde, hvis man sammenlinger med PoS-kæder ellers. Så idéen om kunne have en (PoS-vægtet) demokratisk kæde, hvor protokollen kan ændres løbende inden for et vist parameterrum --- som gerne må være så åben som muligt, på nær at man dog skal sørge for, at de velhavende møntejere ikke skal kunne bruge deres magt til at skrabe flere mønter til sig --- er faktisk en idé, der måske ville kunne vække interesse hos folk.

(10.06.21) Inden jeg begynder at overveje lykke-/gavn-kryptovalutaen igen (har nemlig lige fået nogle tanker, som nævnt overfor), vil jeg også lige pointere, at man måske skal slå lidt koldt vand i blodet i starten omkring NL-smart-kontrakter, indtil man kan danne et godt lovmæssigt system omkring dem på kæden. Om ikke andet må man i hvert fald bare lige sørge for, at man godt kan begrænse friheden af smart-kontrakter, sådan at man kan slå til så snart, man opdager kriminel aktivitet, man endnu ikke har fundet ud af, hvordan man skal håndtere. Når man så finder ud af dette, kan man så igen åbne mere og mere op for NL-smart-kontrakter, og samtidigt også, ikke mindst, gøre den frihed mere stabil (når først man har forstand på det), så brugerne kan være sikre på, at reglerne ikke lige pludseligt ændrer sig igen.

Nå, jeg havde så lige den tanke, om ikke bare fra start kunne have en demokratisk (dynamisk) model for, hvor meget gavn/lykke diverse handlinger vurderes (af deltagerne) at bringe. Så kæden nærmest fra starten er en slags kundedrevet virksomhed (ligesom den, jeg har i tankerne), bare hvor der faktisk er én stemme pr.\ næse (så et faktisk, ikke-vægtet demokrati). Folk kan i princippet så bare jævne belønningsfordelingen helt ud, men så får kædemønterne dog heller aldrig nogen værdi. Værdien af kæden ligger i den positive indflydelse, den kan få på verden og på fællesskabet. Hvis kæden kan formå at lokke folk til mere frit at gode gode og gavnlige handlinger for hinanden (uden så meget bureaukrati og dit og dat nødvendigvis er på plads først for at sikre aktørens løn), så vil folk også være interesserede i at håndhæve valutaens værdi, da dette så bliver måden at håndhæve de løfter, som kæden i brund og grund er et udtryk for; en samling løfter mellem folk indgået på demokratisk vis (hvorfor folk så ikke nødvendigvis vil føle sig så individuelt bundet til dem, men hvis de viser sig at bringe gavn til verden, så vil folk altså bakke op i høj grad, også selvom de måske stemte for noget andet). Og hvis folk altså kan se, at dette kommer til at skabe et holdbar grundlag for en fremtidig global økonomi, jamen hvorfor skulle den så ikke holde. Og hvis man er lidt i tvivl samlet set, jamen så giver dette jo bare anledning til spekulation, hvilket måske endda gør dette mere positivt end negativt (altså at der er usikkerhed), for det kan måske tilføje hype omkring systemet. For ellers er det nemlig ikke en så spekulativ kæde. Uh, noget andet er, at man kunne blive enige om en hvis startformue, som alle personer har, og man kunne også blive enige om en vis progressiv skat i form af en udjævnende kraft, der omfordeler pengene i kæden. Og hvis man så tror på, at dette ender med at blive tilfældet (pr.\ de fremtidige demokratiske beslutninger), så vil almindelige mennesker derved også kunne spekulere i kæden, fordi alle folk, der ikke har solgt rettighederne til denne start-formue væk, vil så, alt andet end lige, have stake i kæden, antaget at verdens nuværende pengefordeling er skæv imellem rig og fattig, ift.\ hvad den vil være i fremtiden. Og en vigtig tanke omkring alt dette er nemlig også, at man jo bare kan lade personidentifikationen blive udskudt, til når teknologien er til det. Så de første afstemninger vil altså mere blive en slags (muligvis biased) stikprøver af den generelle stemning, men hvor man så med tiden kan verificere folk identitet bedre og bedre, så man i sidste ende når frem til, hvad det egentlige resultat af afstemningen blev. Og alt dette kan netop lade sig gøre, hvis man lader pengefordelingen i kæden kunne afhænge af NL-fortolkninger. Og i modsætning til mine tidligere idéer/tanker, så behøver lykke/gavn dog ikke være defineret fra starten af, hvilket ville være meget svært, for det første, og ikke være særligt anvendeligt. Her er det en demokratisk proces, der løbende bestemmer det (og man bør i øvrigt gerne prøve at skabe noget inerti i modellen, men det kan så bare være op til demokratiet at implementere), og fordi denne proces nu faktisk godt, som jeg ser det, kan benytte, at folk har én stemme pr.\ næse, så må udfaldet jo automatisk svare til, hvad folk generelt har vurderet, vil bringe mest gavn og glæde. \ldots Hm, hvad skal mere tænke over? Jeg tænkte, at jeg skulle forklare, hvordan man skal kunne handle med fremtidigt bestemte start-formuer, men det er vel egentligt ret trivielt\ldots\ \ldots\ Uh, man bør dog nok vedtage noget fra start om modellens inerti (både ift.\ ``start-formuer'' og/eller UBIs/deltagelsesbelønninger og til handlingsbelønningerne)\ldots\ Uh, og forresten: Med NL-semantik kan man også understøtte valutaen nemmere, fordi man så kan love, at man som fællesskab løbende vil donere mindre (eksterne) værdier til medlemmer som betaling for at opkøbe deres mønter. Og ved at love dette, binder man sig altså så til en proces (for hvis man så ikke holder løftet, vil hele kæden og fællesskabet jo bare gå i opløsning), som så sørger for, at kursen bliver holdt stabil løbende (og altså kun daler, hvis det hele krakker). Kædefællesskabet skal så altså også varetage eksterne værdier --- og dette gerne ved at fungere lidt som en demokratisk investeringsfond, som jeg jo også bare generelt vil foreslå i forbindelse med NL-kæder --- og så bør de altså binde sig til, at ville sælge de eksterne værdier for mønter til en vis kurs (som godt kan være sat på en kompliceret måde), så længe værdibeholdningen bare er stor nok. God idé, synes jeg. Jeg skal i øvrigt også lige se på, hvordan kæden kan opsplittes i lokale demokratier, men jeg tror bare man må gøre noget tilsvarende det det, jeg foreslog for min ``kundedrevne virksomhed,'' nemlig at enhver gruppe bare skal have lov til at fraskille sig fra resten, når en stor nok del af den givne gruppe stemmer for forslaget. Angående inertien\ldots? Nå nej, inden da: Det skal nok ikke kunne være en omfordelingskraft; det er nok bedre at lade dette blive ved skatten i det omkringliggende samfund. Derimod bør fællesskabet committe sig til at betale handlings- og UBI-mønter af til deltagerne med eksterne værdier gerne i form af produkter, men måske også i form af lokale FIAT-valutaer (som så altså dog \emph{skal} kunne bruges med lethed som betalingsmiddel for de fleste produkter i pågældendes lokalmiljø (det skal med andre ord altså være i form af en stærkt accepteret valuta i pågældendes lokalmiljø)). Okay, og ift.\ inertien\ldots? Uh uh, kunne man ikke bare prøve at finde frem til at godt parametersæt til at karakterisere handlinger, dog udvalgt så abstrakte, at man ikke kan komme til at diskriminere mod bestemte brancher eller til bestemte tider?\ldots!! Kunne dette ikke være en løsning?! Det kunne så være abstrakte parametre, der kan beskrive noget om, hvilken overordnet type af gavn handlingen gør, samt hvilke typer negative konsekvenser, og som også kan beskrive basale ting så som antal mennesker (\ldots og dyr også, hvor man så skal finde en måde at adskille dem på via parametersættet) handlingen berører og på hvilke måder, samt hvordan hele risikobilledet var, da aktøren tog handlinger\ldots\ Nå ja, og måske også noget med nogle parametre til at beskrive, hvor privilegeret vedkommende var til at kunne udføre handlingen. Ja, og man kan sikkert finde mange flere ting, men som antydet skal parametersættet så stadig ikke kunne diskriminere ift.\ nogle uhensigtsmæssige ting såsom tid, befolkningsgruppe (hvad angår både aktøren og de gavnende parter) eller branche --- og her tænker jeg især på teknologistadiet; det går ikke at man kan diskriminere imod produktionen af forældede teknologier, og altså på den måde indirekte diskriminere ift.\ handlingstidspunktet. Det gør endda ikke noget, at kæden fastsætter en håndfuld afstemninger i fremtiden for at gøre ændringer til dette parametersæt, for alle generationer langt frem, vil selv have gavn af, at fremtiden heller ikke kan diskriminere imod dem selv. Umiddelbart: Nice! (10.06.21) 

En parameter kunne forresten også være, hvor meget handlingen relaterer sig til kædens udvikling. Lad mig lige præcisere i øvrigt, at i denne nye version af lykke-/gavn-KV-idéen er det vedtægterne, der var indstemte på det pågældende tidspunkt, hvor handlingen udførtes, der bestemmer hvordan den belønnes, dog hvor man justere disse vedtagelser med tilbagevirkende kraft, så snart man finder ud af, at en identitet ikke kan verificeres (og her må man så definere, hvad der skal til som minimum for at tjekke dette, så folk ikke bliver sorteret fra på unfair vis, ved at de dømmes som ikke-verificerbare). En anden ting, jeg vil nævne, er, at en grundindkomst eller en deltagelsesindkomst nok er at foretrække frem for en ``start-formue.'' Noget tredje er, at man jo også kan blande alle mine idéer, nemlig om en (demokratisk) investeringsfond-kæde, det at give kunder til virksomheden mere og mere bestemmelsesret, og så det her med en demokratisk valuta, hvor de stemmes om, hvordan man udsteder og donerer mønter til aktører på bagrund af deres handlinger, hvor man muligvis har en grundløn, en deltagelsesløn eller begge, og hvor den demokratiske model så altså skal begrænses, så pr.\ konstruktion ikke kan diskriminere lønningssatser på en unfair måde (ift.\ bl.a.\ tid eller diverse tilhørsforhold). Hvis man blander sidstnævnte som resten, så handler det altså om, at sørge for at lønningsmagten langsomt går fra investorernes og muligvis kundernes hænder, hvis man blander den idé med, og så over på almindelige folks hænder, så systemet sigter mod en jævnt fordelt lønbestemmelsesmagt. En mulig fordel ved at blande min nye idé her med, kunne så måske være, at man netop kan lokke med omtalte deltager-/grundindkomst --- ja, og her er det nok især deltagerindkomsten, der er gavnlig --- for dermed at vække interesse og opbakning hos en større befolkningsmængde og lokke flere deltagere til kæden. Og hvis man så (mens man som investorer stadig har størstedelen af bestemmelsesmagten) kunne indstemme nogle gavmilde lønsatser på pågældende tidspunkter for at handle med mønterne og herved presse prisen opad (og her kan de handlende brugere jo overvåge sig selv, så de senere kan bevise deres handlinger), så ville dette måske ikke være helt dumt, fordi man jo så kan tiltrække mange nye deltagere, som kan presse prisen på deres købte KV-mønter op, og så faktisk vinde dobbelt på det, fordi der så også er en tilsvarende belønning hvis de holder efterfølgende. Hm, dette bør så dog ikke være over en hvis grænse, for man vil heller ikke skabe en boble; deltagernes interesse skal helst bare matche ens egen\ldots\ Tja, om ikke andet, var denne mulighed da lige værd hurtigt at nævne. Nå ja, forresten, alt dette kommer jo så bare til at høre ind under gode/gavnlige handlinger, hvis man er med til at fremføre valutaen, og altså ikke bare ind under ``deltagelse.'' Ja, simpelt nok; deltagere, der er med til at fremføre den gavnlige valuta til verden, må jo også kunne forvente belønning for deres handlinger, jævnfør hele pointen med valutaen. Så min lykke-/gavn-valuta-idé ser altså nu igen ud til alligevel at kunne komme til at virke. (10.06.21)

(11.06.21) Jeg har haft nogle yderligere tanker, hvor jeg blandt andet har opdaget, at jeg manglede at overveje nogle ting ift.\ hvad jeg kom frem til i går, og altså hvad der står i de to ovenstående paragrafer. Men derudover er jeg også kommet på endnu en idé, som jeg føler virkelig er vigtig. Jeg bør egentligt skrive ovenstående paragraf(er) om og så tilføje de nye rettelser og idéer, men indtil videre kopierer jeg altså bare lige mine brainstorm noter ind her (indtil jeg får tid til at rette det):

{\slshape
\%Brain (11.06.21): Okay, nu er jeg lidt i tvivl igen, og jeg har nok blandet noget sammen ift. inerti: hvad skal lige afgøres ved den samtidige model for handlingen, og hvad skal afgøres efterfølgende?... Ah, men som jeg tænkte i går, er det ikke bare kursen? Og så gør den demokratiske model... Hm, eller bliver den så bare et signal i første omgang..? Kan man ikke finde en måde, hvor den skal respekteres lidt..? Jo, det kan man vel --- det var vel nærmest også lidt det jeg havde i tankerne i virkeligheden.. Men hvad skal man så sige?.. Er det bare, som jeg også lidt har tænkt, at der skal være en fast grundløn til deltagerne, men som kan opjusteres løbende, så længe man så også bare opjusterer med bagudvirkende kraft, således at kurven kun kan være (ikke-skarpt) aftagende i sidste ende?.. Og at den resterende løn, der gives for handlinger gives ud fra den samtidige demokratiske model for samme..? Hm, men så bliver det faktisk nærmest en hæmsko, ikke at kunne udtrykke sig præcist ift.\ lønningerne. Medmindre man måske bare sørger for, at efterfølgende modeller passer med grundprincipperne.. Ja, det var måske en idé!.. Så man må godt udlove beløber ret præcist til folk, men hvis man så efterfølgende finder ud af, at beløbene er i karambolage med grundprincipperne, så må man bare opjustere (og altså kun opad) de gamle lønninger, så modellen passer. Men så skal man bare kunne finde en god måde også at være retfærdig med denne opjustering... Hm, men kan man ikke bare sige, at dette bare skal ske ud fra, hvordan kædefællesskabet sparer mest muligt i første omgang, og hvis der stadig er frihedsgrader tilbage, så kunne man sige, at parametrene så må sættes, så de ekstra summer gives så bredt som muligt ift.\ de mennesker, der gøres rigere af justeringen (og her må man så antage en formel, som man herved prøver at optimere (eller flere formler med forskellige prioriteter, så man er sikker på at kunne eliminere alle frihedsgrader i sidste ende, så at omjusteringen altså kan følge en protokol eksakt)). Åk, nu begynder det da virkeligt at ligne noget..! (Og disse principper kan man så bruge for sig, men, måske endnu bedre, også i forbindelse med andre systemer, f.eks. investeringsfond-kæden (hvilket særligt ville være en god idé), og måske også blandet med kundemedbestemmelses-idéen, hvor principperne så eventuelt bare kunne tage mere og mere til (ved at en større og større del af økonomien følger dem), og altså ikke nødvendigvis gøre sig gældende særligt meget i starten.) (Umiddelbart:) Nice! (11.06.21) ..Uh, og man kunne så med fordel tillade nogle outliers lidt alligevel, så man ikke absolut er tvunget til at opjustere alle arbejdere/bidrageres lønne, hvis man kommer til at love nogle enkelte personer måske lidt for meget; man kan sagtens lave et system, hvor man tillader en vis afvigelse fra middelværdien, så længe det bare kommer til, når man justerer modellen bagudrettet, og ikke til... nutidige lønninger.. men man kunne måske også bare tillade afvigelser den ene vej (opad i løn), og aldrig den anden. Jo, men man skal stadigvæk ikke kunne favorisere folk i pågældende nutid ved at give dem forøget løn.. eller hvad? Lønnen sættes jo demokratisk i fællesskab, så hvorfor ikke? Ja, måske går det, hvis man bare tillader afvigelser den ene vej. Og her skal det så siges, at man så ikke regner med en margin for hvert enkelte lønning, men vi snakker altså om en samlet afvigelse, der ikke må overskride en vis grænse. Skal man så være omhyggelig med at fastsætte, hvad denne afvigelse må være?.. Tja, man bør nok overveje det grundigt alligevel, men jeg tror faktisk ikke, det kommer til at betyde så meget, for det kommer nok egentligt bare til at handle om, hvad start-bidragerne kan håbe på ift.\ opadgående lønningsjusteringer. Jo, man skal dog nok sørge for, faktisk at sikre sig, at den tilladte afvigelse ikke aftager for hurtigt, for vi må ikke forvente, at teknologien udvikler sig for hurtigt; det kan sagtens være at man fortsat vil have behov for, at kunne benytte muligheden for at have en vis afvigelse, så man fortsat har mulighed for at give mere konkrete løfter til folk (inden man får teknologi til at følge modellen eksakt), uden at det så kan komme til at koste kæden dyrt, hvis en lønning så bliver sat for højt. Man kunne sikkert faktisk med fordel undgå helt at lade den mulige afvigelse aftage, for som jeg ser det, vil den ikke blive nogen hæmsko at have i fremtiden (for man kan jo altid bare ignorere muligheden, hvilket man nok også generelt vil gøre). ... Det er egentligt lidt halv-genialt, det her med teknikken, hvor man kan love ret præcise ting, og så bare opjustere modellen, så det hele i sidste ende også bliver fair. Og med disse tanker, som altså særligt kan være smarte for NL-kæder (men man kunne nok også omtænke dem til andre systemer), så kommer NL-kæderne jo endnu mere til potentielt at være "fremtidens kæde." Og i sig selv er tanken om en sådan demokratisk valuta, hvor det i sidste ende bliver en demokratisk beslutning, hvordan lønsatsmodellen for samfundet skal justeres, og også bare en fed tanke i sig selv, så dermed fedt, at jeg nu kan se en mulig teknik til at nå dertil også..! (11.06.21)
}

Om ikke andet, så forklarer jeg det jo nok også igen lidt mere sammenhængende nedenfor i skitse-noterne (som også fungerer lidt som en opsummering).

(12.06.21) Lad mig lige nævne også, at hvis man skal inkludere parametre i sit model-sprog, som gør at man kan favorisere bidrag, der fremmer selve valutaen, så skal man også sørge for, at denne favorisering ikke kan tildeles et negativt fortagn, så man derved kan få en mulighed for at diskriminere mod generationen, der fremførte kæden. Hm, på den anden side er det nok i virkeligheden bedst bare at fjerne denne/disse (måske lidt for konkrete) parameter/re, eller, måske endnu bedre, sætte den til noget konstant som ikke kan ændres senere hen i fremtiden (hvor den nemlig alt andet end lige heller ikke vil være relevant længere). \ldots Hm, nå nej, på den anden side vil muligheden for at sætte sådan en parameter jo bare i sidste ende give en undskyldning for, at man så har kunne tillade sig at favorisere tidlige bidragsydere, uden at dette så binder en til\ldots\ nå nej, kurverne må jo gerne være aftagende i tid. Ah, men så er der heller ingen ko på isen; så kan det kun være gavnligt for fremtidsfællesskabet med den ekstra frihedsgrad, som en faktor for, hvor meget bidraget har fremmet selve valutaen, medbringer. Men man kunne så spørge sig selv, om man skulle kunne sætte visse parametre, så som denne selve-valutaen-fremmende-faktor, til en fast værdi i starten, og så kræve, at den maksimum kan aftage i sidste ende med en vis begrænsning på hastigheden? For så kan man nemlig selv sætte nogle begrænsninger på, hvordan ens belønninger som kædefællesskab i kædens begyndelse til diverse bidragsydere skal fortolkes, således at fremtidens kædefællesskab f.eks.\ ikke er frie til bare at omfortolke den lovede selve-valutaen-fremmende-faktor til noget andet i stedet. Og hermed kan man så sende nogle klarere signaler til potentielle bidragsydere, om hvad det er, man efterspørger, så man herved slipper for at vurdere hvert enkelte bidrag centralt i starten, men gør at bidragsydere netop kan bidrage mere frit og uden en masse lønaftaler først, og dog stadig være rimeligt sikre på, i hvilken størrelsesorden belønningen for deres bidrag vil være. Ja, så dette kan muligvis være en gavnlig tilføjelse til idéen.



%Brain (11.06.21): Okay, nu er jeg lidt i tvivl igen, og jeg har nok blandet noget sammen ift. inerti: hvad skal lige afgøres ved den samtidige model for handlingen, og hvad skal afgøres efterfølgende?... Ah, men som jeg tænkte i går, er det ikke bare kursen? Og så gør den demokratiske model... Hm, eller bliver den så bare et signal i første omgang..? Kan man ikke finde en måde, hvor den skal respekteres lidt..? Jo, det kan man vel --- det var vel nærmest også lidt det jeg havde i tankerne i virkeligheden.. Men hvad skal man så sige?.. Er det bare, som jeg også lidt har tænkt, at der skal være en fast grundløn til deltagerne, men som kan opjusteres løbende, så længe man så også bare opjusterer med bagudvirkende kraft, således at kurven kun kan være (ikke-skarpt) aftagende i sidste ende?.. Og at den resterende løn, der gives for handlinger gives ud fra den samtidige demokratiske model for samme..? Hm, men så bliver det faktisk nærmest en hæmsko, ikke at kunne udtrykke sig præcist ift.\ lønningerne. Medmindre at man måske bare sørger for, at efterfølgende modeller passer med grundprincipperne.. Ja, det var måske en idé!.. Så man må godt udlove beløber ret præcist til folk, men hvis man så efterfølgende finder, at beløberne er i kambolage med grundprincipperne, så må man bare opjustere (og altså kun opad) de gamle lønninger, så modellen passer. Men så skal man bare kunne finde en god måde også at være retfærdig med denne opjustering... Hm, men kan man ikke bare sige, at dette bare skal ske ud fra, hvordan kædefællesskabet sparer mest muligt i første omgang, og hvis der stadig er frihedsgrader tilbage, så kunne man sige, at parametrene så må sættes, så de ekstra summer gives så bredt som muligt ift.\ de mennesker, der gøres rigere af justeringen (og her må man så antage en formel, som man herved prøver at optimere (eller flere formler med forskellige prioriteter, så man er sikker på at kunne eleminere alle frihedsgrader i sidste ende, så at omjusteringen altså kan følge en protokol eksakt)). Åhk, nu begynder det da virkeligt at ligne noget..! (Og disse principper kan man så bruge for sig, men, måske endnu bedre, også i forbindelse med andre systemer, f.eks. investeringsfond-kæden (hvilket særligt ville være en god idé), og måske også blandet med kundemedbestemmelses-idéen, hvor principperne så eventuelt bare kunne tage mere og mere til (ved at en større og større del af økonomien følger dem), og altså ikke nødvendigvis gøre sig gældende særligt meget i starten.) (Umiddelbart:) Nice! (11.06.21) ..Uh, og man kunne så med fordel tillade nogle outliers lidt alligevel, så man ikke absolut er tvunget til at opjustere alle arbejdere/bidrageres lønne, hvis man kommer til at love nogle enkelte personer måske lidt for meget; man kan sagtens lave et system, hvor man tillader en vis afvigelse fra middelværdien, så længe det bare kommer til, når man justerer modellen bagudrettet, og ikke til... nutidige lønninger.. men man kunne måske også bare tillade afvigelser den ene vej (opad i løn), og aldrig den anden. Jo, men man skal stadigvæk ikke kunne favorisere folk i pågældende nutid ved at give dem forøget løn.. eller hvad? Lønnen sættes jo demokratisk i fællesskab, så hvorfor ikke? Ja, måske går det, hvis man bare tillader afvigelser den ene vej. Og her skal det så siges, at man så ikke regner med en margin for hvert enkelte lønning, men vi snakker altså om en samlet afvigelse, der ikke må overskride en vis grænse. Skal man så være omhyggelig med at fastsætte, hvad denne afvigelse må være?.. Tja, man bør nok overveje det grundigt alligevel, men jeg tror faktisk ikke, det kommer til at betyde så meget, for det kommer nok egentligt bare til at handle om, hvad start-bidragerne kan håbe på ift.\ opadgående lønningsjusteringer. Jo, man skal dog nok sørge for, faktisk at sikre sig, at den tilladte afvigelse ikke aftager for hurtigt, for vi må ikke forvente, at teknologien udvikler sig for hurtigt; det kan sagtens være at man fortsat vil have behov for, at kunne benytte muligheden for at have en vis afvigelse, så man fortsat har mulighed for at give mere konkrete løfter til folk (inden man får teknologi til at følge modellen eksakt), uden at det så kan komme til at koste kæden dyrt, hvis en lønning så bliver sat for højt. Man kunne sikkert faktisk med fordel undgå helt at lade den mulige afvigelse aftage, for som jeg ser det, vil den ikke blive nogen hæmsko at have i fremtiden (for man kan jo altid bare ignorere muligheden, hvilket man nok også generelt vil gøre). ... Det er egentligt lidt halv-genialt, det her med teknikken, hvor man kan love ret præcise ting, og så bare opjustere modellen, så det hele i sidste ende også bliver fair. Og med disse tanker, som altså særligt kan være smarte for NL-kæder (men man kunne nok også omtænke dem til andre systemer), så kommer NL-kæderne jo endnu mere til potentielt at være "fremtidens kæde." Og i sig selv er tanken om en sådan demokratisk valuta, hvor det i sidste ende bliver en demokratisk beslutning, hvordan lønsatsmodellen for smafundet skal justeres, og også bare en fed tanke i sig selv, så dermed fedt, at jeg nu kan se en mulig teknik til at nå dertil også..! (11.06.21)




%\subsubsection{Diverse øvrige idéer og tanker omkring organisationsstrukturer m.m.}
%I denne sektion vil jeg give en hurtig oversigt over diverse øvrige små idéer, der relaterer sig til økonomiske systemer, firma-/organisations-/parti-strukturer osv. ... 

% "Privat overvågningssystem" (F.eks. til at ID'e folk). Tillidsrepræsentanter. "Kontraktbaseret repræsentatskab," "donationer med kontrakt-forbehold" og "dynamiske mandater."  



%Emner (og huskenoter):
		% - At brugere anses og gør sig gældende (ved at de kræver dette) som investorer. (tjek)
		% - Forbrugsmagt --- og man kunne endda i teorien tage det ud på et ikke så fint punkt, hvor man prøver at overtage bare for magtens skyld og ikke for det gode samfunds skyld (hvad man self. ikke bør). (Btw: "Social bæredygtighed.") (tjek på nær "social bæredygtighed")
	% - Kontrol og ejerskab over data med åbne og samtidigt private systemer (og husk jo at forklare om "privat overvågning" og sådant). Nå ja, og dette handler så også særligt om fremtidige forbrugsoplysninger, hvilket så også får det til at hænge sammen med det følgende punkt (altså "Grupperabat-foreninger"). (Btw: "Mindre lyssky.") (tjek på nær "privat overvågning")
		% - Grupperabat-foreninger. (tjek; det er vist dækket i min første lille paragraf)
		% - ...Og arbejdere kan jo også prioritere de mere socialt bæredygtige foretagener.
		% - Og så er der jo bagud-belønnings-organisationer. (tjek) Og der er også lige min lille lykke-KV-idé. (tjek; har nævnt (dog ikke uddybet så meget, men det er vist også fint)) (Husk: reklame vigtig. Og det er også vigtigt at se på, hvad nye brugere vil tillade af kurver i et eksisterende system, som de skal til at joine..) (hænger sammen med at "starte en ny valutta generelt" (følgende punkt); det vil jeg muligvis ikke bruge tid på her)
	% - Det med hvad man skal tænke på i forbindelse med at starte en ny valuta generelt. Og husk her at mere brugerdrevet banking også generelt er et betydeligt emne. (gider jeg egentligt at skrive om det?)
		% - Jeg tror ikke jeg behøver noget med at handle med investeringsrettigheder, vel? Kom jeg ikke frem til, at det var rimeligt banalt (tror jeg, jeg gjorde)?.. (nej, det er ikke værd at bruge tid på her)
		% - Bør jeg mon også lige komme mere ind på selv-udviklende p-ontologier og det der, eller har jeg ordnet det?.. (det er ikke så vigtigt, nu hvor jeg ikke er så interesseret i KV-systemer længere --- så kan det nemlig bare siges som: hvis man gerne vil have at p-ontologierne udvikles og vedligeholdes, kan man gå sammen om at donere løn til dette arbejde)
		% P-ontologier, og her (ligesom med certifikater generelt) er det vigtigt at bruge intuition i stor stil. Således bør man også være ihærdig med at prøve at foreslå forskellige pointsystem til diverse ting, som så kan evalueres og omformes til andre systemer efterfølgende (eksempel: Denne tekst har en høj grad af at være forklarende (osv.)).. (tjek; jeg tror det kommer til at fremgå klart nok, når jeg forklarer om idéen som et "stackoverflow-alternativ")
		% - Eksempel: Menings-profiler til p-ontologier. (tjek)
		% - "Mikro-partier," dvs. partier (effektivt set) til enhver lille gruppe. (tjek) Kontraktbaseret repræsentatskab.. Ja, og overvågning og gennemsigtighed self. (Btw: "Ledelses-licitering" kunne man også kalde det.) (Btw: Jeg har også noteret: "Donationer med kontrakt-forbehold" i mine papirnoter, bare lige for at minde mig på den del af det.) Nå ja, og "dynamiske mandater." (er på vej, så på forhånd tjek)
		% - Reincarnationsprincip hvad angår lykke/velfærd. (Også: "Dynamisk demokrati, hvor folk stemmer ved at ændre deres præference-parametre.") (Ha, sjovt at jeg fik den idé her (hvis man specificerer 'parametre' til 'model-parameterrums-kraftfelt,' så har vi jo min nye idé)) (reincarnationsprincip kan jeg skrive om under 'etik')
		% - OS-fabrikker sågar. *(Med mere modul-kompatibilitet også.) (Btw: "Globalisering kan vendes til noget godt.") (Og: De kan også være mere fleksible og omstillingsdygtige.) (tjek (på nær at jeg ikke har nævnt globalisering.. Hm, hvad mente jeg egentligt helt præcist med det?.. Tjo, vel bare, at man alt i alt for neutraliseret globaliseringens negative sider ved at gøre produktionsmidler osv. mere styret og ejet af kunder (inkl. arbejdere), men det behøver jeg vist ikke at nævne, så ja: tjek))
		% - Selv-overvågning af fabrikker, partier/politikkere, ledelser/ledere, rapportere og videnskabsmænd m.m. (Også selv-overvågning for arbejder, men det er vel en lidt anden snak.) (tjek)
		% - Ontologi eksempel: Produktoplysninger. (tjek)
		% - Har jeg allerede understeget, at der skal være meget fokus på at klassificere brugere, eller skal jeg gøre det her? (har jeg gjort)
		% - Ontologi-eksempel: IT-sikkerhed og sikkerhed generelt. (tjek)
		% - "Nedefra-og-op tillidsstruktur," dvs. hvor man hele tiden sørger for, at hver undergruppe har en tillidsrepræsentant, som har samme interesser som gruppen, men som så har tid til at studere forholdene (uden at blive korrumperet undervejs..).. (tjek)
	% - Samtale/diskussion-platformsidé ("to-vejs-twitch-agtig idé"). (på vej)
	% - ~"Idé med ikke at lønne ledere ud fra overskudet, men måske i stedet ud fra den løn, den gives til arbejderne." (tror jeg ikke jeg vil nævne; det er bare en lille idé og kan nok bare blive her i kommentarerne..)
	% - Eventuel automatisk opsplitning af diverse åbne orgs.. (tja, at opslitte virksomheder, hvis de bliver for store, er ikke nogen ny tanke, så det må man bare se på, om giver mening at indføre regler om (så bliver foreløbigt her i kommentarerne))
		% - "Bekæmp korrelationer." (Og prøv at danne koncepter, når korrelationer findes (eller forudses)). (tjek)
	% - Eventuelt: "Automatisk generering af (grafiske) assets".. (gider jeg ikke nævne nu her; bliver foreløbigt her i kommentarerne)
	% - Vil nok ikke nævne men: Lyssky ting/hemmeligheder i foreninger/instanser er altid dårligt; om de så end dækker over individuel eller bare en samlet korruption (også endda selvom denne korruption gavner dem, det bliver holdt hemmeligt for (jo på beskostning af andre)). (bliver foreløbigt her i kommentarerne)
	% - Det med at gå sammen som fælleskab for at sætte betalingsmure på hidtil ellers åbne systemer.. *Hm, skulle man nævne noget om dette, og så komme ind på det med at belønne tidligere OS-bidrag?.. ..Og så kunne jeg måske også lige understrege muligheden i at belønne tidlige kunder m.m. i samme omgang?.. (tjek, men jeg nævner foreløbigt ikke det med at belønne tidligere OS-bidrag, hvor tanken i øvrigt altså var, at man muligvis kunne rekrutere folk, der allerede er aktive i open source-verden, ved at fremhæve dem blandt folk, der fortjener belønning, hvis virksomheden kommer på banen. Men jeg synes altså ikke at idéen giver ligeså meget mening, nu hvor jeg er gået væk fra min KV-agtige idé igen.)
		% - At prædiktive ontologier aldrig behøver at sortere data helt fra; der kan jo komme et paradigmeskift, når nok data lige pludselig peger i en anden retning. (tjek)
		% - Btw: "Udliciteret modding." (tja, det hænger vel sammen med, at det udgivne produkt bliver mere åbent for modifikationer --- det hænger også sammen med mine tanker om et alternativt Git, så det skal jeg nok huske)
		% - Burde jeg lige nævne noget mere om feedback til tekster (uploadet eller på anden måde delt til andre personer)?.. (nej, det bliver en ret central ting nu er mit "stackoverflow-alternativ")
	% - "Udløbsdatoer på brugerdata." (er der egentligt mere, jeg skal have med om brugerdata-foreninger?..) (nej, det gider jeg ikke at skrive om her (det handler bare om, at brugere gerne må eje deres indsendte data på den ene eller den anden måde, men det giver jo også lidt sig selv))
		% - ""Påskeæg" til at fremhæve selv-eftertjek." (tjek)
	% - Eventuelt: "Dark democracy" (uh-uuh).. (nvm)
% - "Alle firmaer kan deltage" (ved at åbne mere op osv.). (eller ved særligt altså at gøre det mere kundedrevet?..) (ja, helt klart værd at nævne)
		% - ~"Semi-private lokalgrupper til at ID'e brugere." (tjek)
		% - Smart at opkøbe boliger osv. til en fælles.. fond.. (Der kan også i det hele taget skæres mellemmænd fra..) (er ikke gået i detajler angående 'fond' eller noget, men tjek alligevel)
		% - Man kan spare meget arbejde på reklame og pr., og ende med meget bedre reklamering generelt for (gode) varer. (tjek)
		% - ..Alt det (lidt gentaget) med at det bliver nemmere at oprette virksomheder og finde kapital (fordi man bare kan følge nogle solide og gennemprøvede skabeloner). (tjek; nævnte det kort igen, og det er vist fint)
	% - ..Ikke skabe rivaler; nedton grådighed meget (jeg har nævnt dette i en kort parentetisk bemærkning et sted, og måske er dette nok..)
	% - Temp. C.S. samt gennemsigtighed (self.). (kan lige nævne temp. CS igen kort..) (nej, jeg tror allerede de vigtige ting omkring det fremgår)
		% - Sandhedsmining.. (gentaget) (det dropper jeg vist at skrive om, nu hvor jeg ikke tror på, at KV-systemer kommer med til at igangsætte udviklingen)
		% - (Husk kurver og prognoser..) (det har jeg vist været inde over)
%
		% - "Ledere = stuarts".. (tja, der er vist ikke så meget mere at nævne her; ja, det er en god idé, hvis ledere bl.a. kan fyres af kunder/medlemmer/arbejdere i det system, de leder)
		% - Håber mange vil deltage i og fokusere på arbejdet om at opbygge org-strukturer osv.. (ja, men føler ikke, jeg behøver at nævne dette i ovenstående tekst)
	% - Nu er jeg lidt kommet frem til (det er d. (19.04.21) i dag), at et godt KV-agtigt system / en god økonomisk bevægelse bare handler om at aktivere deltagernes fremtidige forbrug, arbejde og interesser og gøre dem til kapital på en måde. Og et sådant system kommer så til at bygge på bruger-etos. Hvis deltagere så vil tage lån i denne kapital, så må sådanne lån jo bare være begrænset i forhold til en beregning omkring, hvor sandsynligt det er, at brugeren ikke vil holde sine løfter (nemlig omkring dennes 'etos,' som jeg bruger udtrykket). (Dette bliver foreløbigt her i kommentarerne)
	% ..Man kan sagtens investere på mange måder, bl.a. måske ved brug af KV-agtige systemer, men også bare på mere normal vis. Og så vil der jo nok (forhåbentligvis) være en hvis motivation fra både ledere og medlemmers side om at gå en balancegang om ikke at være unfair over for den anden part, for det er jo hvad hele organisationen går ud på (så det duer ikke at organiasitionen går imod hele denne grundlæggende tanke; det vil folk jo gøre oprør imod), og det duer ikke at svigte sin ledelse som medlemmer på unfair vis, for så mister man jo etos (og alt andet end lige ved ingen jo noget om, hvor vigtigt etos bliver for fremtiden; det kan sagtens være, at der gør lang tid, før der indfinder sig systemer, som ikke behøver at gøre brug af etos (og vil sådanne systemer så egentligt overhovedet være at foretrække?)). (Og her tænker jeg så på en temp. CS-org, hvor investorene stille og roligt sælger ledelses-/stemmeretten ned til underlæggende medlemmer osv. ..Ja, og hvor man jo så fra start giver nogle prognoser (kurver) for, hvad man så sælger det for, altså hvad man forventer at tage for sin investering (alt efter hvor tidligt den gjordes) med andre ord, så folk ligesom skriver under på dette, når de melder sig til.) (Dette bliver foreløbigt her i kommentarerne)
	% - Tjek at det her med, at folk vil gøre oprør, når hjemlerne er mere fastsatte *(og hvis ledelsen så viger fra disse hjemler (hvilket jo er meget klart; hvis man formulerer målsætninger grunddigt fremfor vagt, bør dette være langt mere tillidsvækkende, for så lægger man derved også større bånd på sig selv)), er med (har jeg nævnt det i KV-sektionen?). (Dette er en fin nok pointe, men jeg tror ikke, jeg har vildt meget brug for den i teksten, har jeg vel?..) (Nej, det må lige blive foreløbigt her i kommentarerne)
		% - Har jeg nævnt her?: Husk demokratisk firma generelt. (he, tjek *('he' fordi jeg lidt genopfandt denne idé, da jeg kom til at tænke på kundedrevne virksomheder igen))
		% - Nævn åbenhed: Det må være rart, hvis man som opfinder bare kan dele rimeligt frit. (tjek)
		% - Hvis jeg ikke har nævnt det: Husk princippet om at finde en repræsentant til at undersøge mere magtfulde og/eller indsigtsfulde lag. (tjek)






%Lige en lille brainstorm over emnerne til denne sektion:
%Jeg vil jo nu også skrive her min "økonomiske bevægelse," som dog i sin kerne ikke længere helt er det; jeg er på en måde lidt tættere på et tidligere stadie, som jeg kaldte.. mere en "forretningsidé".. Ja, så jeg vil altså bare lige for det første redegøre for, hvordan sem-web m.m. kunne give anledning til nogle smarte og åbne organisationer, som generelt så nok kan give mere dynamik og positiv fremgang i samfundet (og især i forbindelse med selve udviklingen af sem-web m.m.). Dette kunne jo så følges op af mine andre økonomiske tanker, som jeg vil skrive om her. Jeg vil dog nok også gerne lige nævne lidt om mine tanker, jeg har arbejdet på omkring at skabe en.. nærmest en lykke-valuta.. Ja, værd at nævne de tanker.
%Ah ja, lykke-handlinggraf-valutta er faktisk en spændende nok (KV-agtig) idé, som dog ikke kan garranteres at virke ligesom med alle andre KV'er; det kommer jo helt an på opbakningen --- plus den efterfølgende generelle accept. Ok.
%Så jeg skal skrive om de økonomsike foreslag, hvor jeg så passende kan starte med den generelle hurtige (meget) analyse af et godt økonomisk system, samt også hvad et etisk samfund er. Så kunne jeg jo derfra lige nævne lykke-KV-tankerne/-ideérne. Og så kan jeg gå videre til at beskrive, hvilke nogle problemer, man gerne vil løse generelt i forbindelse med et opstartende, jo meget åbent, sem-web, og så videre til alle diverse idéer jeg har til, hvad man kunne gøre af forskellige ting; alle de forretningsfordele osv., jeg har (eller "alle" og "alle".. Jeg har da et par stykker..). 
%Når det så kommer til ting, der har lidt mindre med økonomi og virksomheder (og samfund) at gøre, og mere med de teknologiske fordele, som bl.a. sem-web åbner op for, så er det vel egentligt bare at uddybe ontologier lidt (altså komme med flere forslag og forklare, hvor vigtige jeg tror, p-ontologierne bliver), og så også beskrive nogle flere områder (YT, spotiy, steam, whitebox-søgealgoritmer..), hvor der kan være klare fordele ved at åbne teknologierne mere op.? Hvad er der mere? Nej, der er vist ikke lige andet (men der er jo som nævnt også lige et par ting, der skal drøftes omkring mine idéer til forretningsfordele og lignende..). 
%Åh, og angående lykke-KV, så er det også værd at tænke på, at det at melde sig til systemet og begynde at indgå løfter i forbindelse med det, kan ses som et bidrag, og derfor kunne det være i orden at sætte en kurve på, således at tidlige "investorer" (m.m.) får forøget deres pengebeholdning, når KV-systemet vokser. Og ja, generelt bør jeg jo sige noget mere generelt om KV-systemers holdbarhed osv., samt hvilke muligheder der kan være for faktisk at sørge for at tidlige investorer (m.m.) bliver belønnet.  






% Husk:
			% - At webbet kommer til i høj grad også at handle om, at klassificere brugere, så man kan bruge råd og anbefalinger fra grupper, der matcher én selv. (tjek)
	% - "Modelérvoks-metafor." (i gang)
% - "Og uanset bevægelse eller ej, så er det godt med et godt donations-/tipping-system."
	% - "NL i sem-web." (tjek, men kan godt nævne, at NL fra resten af webbet også kan processeres) (tja, det er vel egentligt trivielt..)
			% - Det med at ITP er vigtigt for servere i det mere åbne web. (tjek)
	% - "Bayes-ontologier er vigtige for brugerdrevent web."
	% - Princippet om en slags omni-side (skal jeg lige uddybe langt mere). (Tror egentligt ikke, der var mere om denne idé, end hvad jeg har skrevet nu om sem-webbet..)
% - "Og husk: Big data på en ny (anonym) måde."
			% - "Sem-dokumenter." (tjek)
% - Det med at få det sådan, at brugere ejer (..selvfølgelig) deres bidrag, så de også nemt kan trække det væk, og dermed har en handlekraft ift. donationer osv. (og "fagforeninger").
% - "Domsmænd og trust-grupper på sem-web." ..Giver næsten lidt sig selv..
% - I forlængelse af forrige punkt: "Det behøver ingen gang at være så avanceret, for når "kontrakterne" er for grupper, så kan grupperne bare samle på (og kæmpe om) de mest troværdige medlemmer."
			% - Fra mine papir-noter er også et punkt, der lidt kortere kan skrives: "Semantiske (og dynamisk opdatérbare) moduler og interfaces til konventionelle applikationer også." (tjek)
% - (Også lidt omskrevet:) "Big data i forbindelse med science."
% - Steam, spotify, youtube, søgemaskiner, pricerunner m.m. (tja måske ikke; måske bare tage Steam som eksempel..), spiludvikling (digitalt), ... (kan måske godt skrive mere)
% - Jeg bør nu nævne noget om firmaer (mere end foreninger, selvom foreninger i virkligheden ville være bedre), der dannes via (templates) formelle procedurer, hvor brugere er ret magtfulde og hvor man godt kan belønne efter arbejdet er udført, men stadig bare som en løn; jeg behøver ikke at komme ind på fremtidfortidsbelønninger.. Tja, eller "ret magtfulde," det er så lige spørgsmålet. For jeg kan også bare holde mig til at man gerne "vil aktivere så mange brugere som muligt." .. Ja, og at en ret åben struktur så tilgengæld vil kunne konkurrere med gængse platforme, netop fordi man så aktiverer brugere mere. 
			% - Man kan altid bidrage.. (tjek)
% - "Og overvej noget om offline-reklamer.. ..Tja, sådan noget kan man gøre, når vi når til, at brugere kan overvåge sig selv på en ikke-kompromiterende/-pinlig/-afslørende og så-godt-som-anonym måde. Og dette kunne bare kræve, at der er en overvågningsmekanisme, der slår til en gang imellem, hvor dem, der skal tjekke det, så kan bede om klip (video-, lyd-, bevægelse-, keyboardtryk- m.m.) fra tilfældigt udvalgte tidpunkter, og hvor brugeren så lige bør gennemse dem og sikre sig, at ingen af de adspugte klip er kompromiterende, og så får lov at veto'e et lille antal af dem, uden at de medfører en bøde. (Men dette kunne man jo betragte som "øvrige noter.")"
% - Forklar om at bruge template- (formel protokol-) foreninger til det mere åbne web i 'øvrige noter.' Forklar også her om whitebox-algoritmer (nemlig at de slår blackbox), og foklar om teknikker til brugernetvæk omkring at overvåge hinanden på en stort-set-anonym måde. 
% - "Skulle man nævne noget omkring, at almindelig tekst på nettet så også løbende bliver mere og mere semantisk?.. Og skulle man undertrege noget mere omkring at man bare kan benytte sig af URL-antagelser til sine programmer, så man også kan søge informantion på nettet, uden at lokationen er tiltænkt til at være semantiske..?.. (måske er det ikke så vigtigt...)"
% - "Firmaer, der lover kurver, på bekostning af løfter til aktionærene, men hvor der så herved muligvis kan skabes større efterspørgsel til gengæld."
% Husk at overvej, om "løfte-baseret (KV-)system" og/eller de tanker om, hvorfor det er mere robust, skal med... Nå ja, man kunne jo sørge for at noget tilsvarende er nævnt for de generelle tanker om økonomiske systemer (indeholdende en "løftegraf" osv.).
% - Husk at tjekke at hjemler bag foreninger er nævnt de steder, hvor det kan være relevant..









\subsubsection{Flere noter omkring fremtidens internet m.m.}
(18.05.21) Nu hvor jeg har fået idéen til en mere kundedrevet organisation med en kontinuer ledelsesmodel-afstemning og har (lige) tænkt lidt mere over, hvordan og hvornår vi mon når til denne udvikling, så kan jeg se, at udviklingen måske godt kan starte mere direkte med de kundedrevne organisationer og/eller det brugerdrevne web. Min ITP-idé er god og solid, men den vil ikke nødvendigvis bringe en særlig hurtig udvikling med sig i sig selv. Selvfølgelig er det smart nok, at alle ITP'er bare kan samles til én (fordi man i stedet bare bygger det hele over `meta-antagelser,' og så er det lidt lige gyldigt, hvad man starter med), og selvfølgelig bliver det vigtigt at få udbredt brugen af certifikater, men hele denne udvikling vil alligevel bygge på mange menneskers engagerede arbejde, og hvem siger at folk vil blive hypet til at engagere sig så meget? Nu har jeg jo heller ikke længere nogen stor blockchain-idé, jeg kan hvile på, som kan danne hype omkring brugen af matematik og p-ontologier, som jeg ellers tænkte, da jeg begyndte dette notesæt (for to måneder siden ret præcist). Til gengæld har jeg nu en ret simpel virksomhedsidé, men det vender jeg tilbage til. Hvis jeg tænker over det, så vil det vigtigste, som hele min ITP-relaterede idé muligvis kan tilvejebringe på relativt kort sigt, være mere åbne applikationer og hjemmesider, hvor brugerne har stor indflydelse på opbygning, og hvor der samtidigt også stor mulighed for customization, i hvert fald set fra brugernes synspunkt. Set fra programmørernes synspunkt kunne man også forestille sig, at mere semantisk, matematisk programmering kunne blive stort inden for en\ldots\ `overskuelig fremtid' var jeg lige ved at sige, men det er netop problemet: Den er slet ikke overskuelig. Det afhænger af vildt meget, inklusiv af folk engagement. Og som jeg har tænkt det indtil nu, havde jeg jo forestillet mig, at udviklingen af matematisk programmering skulle føre udviklingen af et mere åbent web med sig. Men nu tænker jeg altså, at det måske også kunne være omvendt; at udviklingen af mere åbne og applikationer og hjemmesider kunne blive det drivende element i sig selv. Ja. Brugbarheden af det brugerdrevne web afhænger jo ikke af, at programmørerne kan lave smarte transformationer og skrive teste kode på et højteknologisk plan, og der er derfor ingen grund til at vente på at hypen omkring matematisk programmering skal komme og drive resten med sig. For de basale elementer i det brugerdrevne web behøver slet ikke at være særligt sofistikerede. Det handler jo bare om, at brugere kan få indflydelse på, hvordan database-servere skal prioritere diverse brugeruploads, som så indgår i en samlet data-ontologi, hvorfra serverene skal servere datastrukturer tilbage til brugerne. Brugerne bør i øvrigt også have indflydelse på en model for, hvordan denne data serveres af serverne. Database-serverne behøver ikke at være styret kun af én central enhed, men fællesskabet kan også sagtens bruge P2P-netværk, hvor hver knude i netværket så kan have sine egne (offentlige) bestemmelser for, hvad der kan uploades, hvad der gemmes og hvad der serveres og hvordan. Da sådanne server-politikker jo vil være relativt konkret i forhold til f.eks. organisationsledelse, tror jeg så bestemt at man også her bare kan bruge demokratiske modeller a la dem, jeg foreslog for de kundedrevne virksomhedsorganisationer, hvor alle deltagere hele tiden kan ``trække i forskellige retninger'' med en fastlagt kraft. Ja, og man kan selvfølgelig sagtens blande de to idéer så; man kunne f.eks.\ sagtens forestille sig en server-virksomhed, hvor brugerne får stemmemagt(/``stemme\emph{kraft}'') alt efter, hvor meget de bidrager til omsætningen (via deres abonnement f.eks.). Og man kunne ligeledes også forestille sig, at en kundedrevet virksomhed med fordel kunne engagere sig i brugerdrevne applikationer og hjemmesider. 










(23.06.21) Jeg oprettede denne sektion, fordi jeg ikke lige vidste, hvor ovenstående paragraf ellers ville passe ind. Nu har jeg endnu noget igen, som jeg ikke helt ved, hvor jeg skal indsætte henne, så det indsætter jeg også bare her. Jeg har nemlig lige nogle noter fra i går og her lidt tidligere morges omkring dynamiske, demokratiske modeller og om kundedrevne og/eller folkedrevne virksomheder. Lad mig først lige understrege, at en vigtig kundedreven virksomhed er en til at udvikle og varetage en alt-muligt-hjemmeside/(desktop- og mobil-)applikation. Det har jeg sikkert nævnt før, men jeg følte altså lige træng til at understrege det igen. Internettet (og app-markedet) er nemlig det oplagte sted at give brugerne meget mere kontrol over indholdet og indstillingsmulighederne. Så min idé bare om at have en hjemmeside-virksomhed, hvor kunderne bestemmer (mere og mere) --- og måske også som så får mere og mere af ejerskabet --- er altså (som jeg kan se det) en virkeligt gangbar idé i sig selv. Og denne virksomhed kan jo så med fordel prøve at sigte imod at have refleksive servere, hvilket vil sige at det selv kan verificere og implementere bruger-indsendte og -indstemte opdateringer af sig selv, og tjekke at de overholder visse betingelser --- og her kan man også benytte menneske-vurderinger i form af certifikater, så længe at teknologien ikke er til, at serverne kan tjekke alt selv. Virksomheden bør selvfølgelig også benytte dynamiske model-afstemninger, det er klart.

Når den kommentar er ude af verden, så vil jeg altså også godt nævne noget lidt vigtigere (for det andet er jo bare en gentagelse sådan set, af ting jeg har skrevet tidligere), nemlig at man faktisk bør lave en samlet applikation (og oprette et samlet fælleskab) omkring dynamiske afstemninger-via-modelkraftfelter, hvor alle mulige afstemnings-varianter kan vises, hvor man kan inkludere vilkårlige bruger-undermængder og også give disse mængder vilkårlige vægtninger. Samtidigt bør brugere så selv kunne oprette vilkårlige grupper på kryds og på tværs, hvor de kan implementere deres egne vægtninger --- gerne i form af en funktion, der kan ændre sig med en udvikling (som hver bruger i princippet selv må sørge for bliver indputtet korrekt til applikationen, hvis de vil få vist den korrekte opdatering). Brugere skal så også kunne give diverse point (med brugerdefinerede pointsystemer) i grupperne, og sidst men ikke mindst skal de selvfølgelig også kunne opgive deres "kraftfelter." Tanken er så, at man i virkelige virksomheder, fagforeninger (eller andre foreninger) eller partier så også kan bruge denne app og gå ind og lave aftaler med diverse brugergrupper, man er interesserede i, eksempelvis som kunder.  Brugergrupperne kan altså gå sammen om at kontakte diverse foreninger/virksomheder, men det kan som sagt også være den anden vej. Så udover at fungere som en godt visningsværktøj til at se folk meninger og holdninger, og parathed til at gå med på diverse handlinger og aftaler ikke mindst, så kan denne app så altså også udgøre udgangspunket for at igangsætte sådanne aftaler og handlinger. En simpel men brugbar mulighed i denne forbindelse kunne være at opnå henholdsvis grupperabatter og sikre sig en mere pålidelig kundebase i den nære fremtid for (henholdsvis) en brugergruppe og så en virksomhed. Og hvis en brugergruppe sørger for at identificere medlemmerne før eller som led i at aftalerne indgås, så vil brugerne have incitament for at holde deres løfter, for ellers bliver både samme og andre grupper sandsynligvis nødt til at ekskludere pågældende i fremtiden for at kunne fremstå troværdige. Grunden til at det kan være smart med stemme(-kraft)-vægtninger i grupperne er så bl.a., at mere betalende kunder så kan få en større vagt, og hvis der er tale om, at gruppen skal bestemme over noget, kan man også starte med at give de mest kyndige brugere mest stemmemagt først, alt sammen som led i at indgå aftale med den virksomhed eller forening, der kommer til at lægge under for brugergruppens beslutninger efter aftalen. Jeg synes, dette er en rigtig vigtig pointe, for bemærk, at det herved bliver meget mere åbent og liberalt, hvordan vægtningerne skal være for diverse systemer, der bruger dynamiske afstemninger, og som inkluderer kunderne og/eller folket i sine beslutninger, end hvis jeg eller andre skulle sidde og finde på et godt universelt vægtningssystem til at begynde med, som alle så skulle følge i denne teknologiske og forretningsmæssige bevægelse. Og noget der næsten er endnu vigtigere: Hvis man så får valgt et ikke helt optimalt system, så kan man så med denne applikation stadig gå en og få vist, hvad alternative valg ville have medført. Og da modellerne kommer til at blive så alsidige (fordi de bør, mener jeg, baseres på naturlige sprog, således at afstemningerne i bund og grund svarer til løbende at stemme om ændringsforslag til manifester/ledelseshjemler og/eller kontrakter), at folk faktisk kan idé-udvikle via stemmeprocessen (ved at trække modellen over på til versioner, der påpeger disse idéer), så bliver det jo særligt vigtigt også at kunne iagttage mindre gruppers resultater (eller grupper med mere centreret stemmevægtning), fordi disse jo så potentielt kan indeholde inspirerende løsninger.

Så det var altså tanken. Når vi så f.eks.\ tænker på mulighederne for grupperabat-foreninger og for politiske foreninger i sig selv, så kan man nok se, at vis først én (stor nok) befolkningsgruppe begynder at bruge denne teknologi, så må andre også følge trop (for ikke at få en markedsmæssig eller politisk ulempe). På denne måde mener jeg altså, at appen hurtigt vil udbrede sig til hele befolkningen, når først konceptet vinder lidt frem. Og som jeg så kom i tanke om her i morges, bør jeg jo alt andet end lige faktisk begynde på en prototype nu her, for det ville for det første hjælpe meget på forklaringen, og vil også bare gøre lanceringstiden væsentligt kortere, fordi folk så bare lige skal udvikle applikationen færdig som en fungerende app. Så det tror jeg altså næsten, jeg vil gøre i den kommende tid. (23.06.21) Nu skal jeg i øvrigt også lige se på, om det ville være muligt for mig også at bygge en simpel prototype til min ``Stack Overflow-idé,'' som jeg har kaldt den på det sidste --- altså min idé til en mere semantisk vidensdelingshjemmeside (som så måske i starten kan fokusere meget på programmeringsløsninger). \ldots\ Det må man da faktisk næsten kunne ret nemt, altså bygge en prototype, for det er vel bare ligesom jeg skrev det sidst i Milawa-sektionen (``\textbf{Min idé er opfundet\ldots''}), og hvis man som sagt kan implementere det med relationelle databaser, hvad man sikkert sagtens kan, så ville det jo blive ret nemt at komme fra start\ldots\ Og så skal jeg bare lige selv indføre nogle gode argumentationsopbygnings-grafkanter til at starte med\ldots\ Jo, men lad mig altså lige tænke lidt over det.

Angående dynamiske modeller fik jeg lige en idé til, at man måske skulle have muligheden for at indstemme nye punkter andre steder i metamodellen, og så derefter få mulighed for at teleportere mellem to punkter, hvis begge er\ldots\ eller måske i hvert fald bare hvis destinationspunktet er stationært. Dette kunne nemlig så blive en nem og bekvemmelig måde at skifte landskabet omkring modelpunktet på, så man ikke er tæt på en uhensigtsmæssig ``kløft'' --- eller så en udmærket løsning ikke ligger bag uacceptable områder ift.\ punktet i rummet, så man ligesom skal trække punktet udenom disse områder. (23.06.21)
\\

(26.06.21) I forgårs, d.\ 24/6, fik jeg nogle virkeligt vigtige (for mig) tanker omkring de dynamiske model-afstemninger og også om ontologier og det semantiske web. Angående de dynamiske model-afstemninger kom jeg frem til, at man nok med stor fordel kunne bruge machine learning (ML) -teknikken, hvor man basalt set finder frem til korrelationsvektorerne (de mest betydende ført) og så parametriserer rummet ud fra disse. Herved kan man så bruge en række test spørgsmål (som kan varieres af brugerne og over tid), som brugerne dermed kan bruge til at klassificere sig selv og hinanden. Og dette vil så med det samme give en geometri, der vil være et godt udgangspunkt, når man så skal indlægge kraftfelter og begynde at trække modellen i diverse retninger. Dette ledte mig så videre til at tænke på pointsystemerne i ontologierne (i forbindelse med min ``Stack Overflow -idé,'' som jeg dog vil stoppe med at kalde det, for nu er idéen bare en generel idé til at få gang i det semantiske web, hvor man så i samme omgang også lige kan pointere mulighederne omkring at bruge teknologien til program-dokumentation og -verifikation også), og her må man jo også virkeligt kunne komme godt og hurtigt fra start, hvis man bruger denne ML-teknik. Nu forestiller jeg mig så, at det semantiske web faktisk kan startes meget mindre formelt og nede-fra-og-op, end hvad jeg ellers har lagt op til i dette sæt noter, og i stedet startes meget blødere (og båret delvist af brugerstatistiker), hvor brugerne så til gengæld bare løbende kan tilføje diverse mere præcise og formelle metoder til systemet, så disse også kan benyttes.

Da jeg nedskrev nogle tanker omkring, hvad database-systemet burde inkludere sidst i ``\textbf{[\ldots] Milawa \ldots}''-sektionen, kunne jeg se, at muligvis blev ret simpelt overordnet set. Nu har jeg så tænkt videre over, hvad en semantisk web (sem-web)-database skulle indeholde, som jeg tænker idéen nu, er jeg tror virkeligt det \emph{er} ret simpelt. Forskellen fra gængs ontologi-teknologi, der bruger tripletter, er nok mest bare, for det første at termerne skal kunne have en indre struktur og ikke bare være atomare. Systemet skal altså inkludere komposit-termer også. Det ville nok ikke være helt dumt at bruge et typesystem *[her mener jeg at bruge typer a la dem, som er konventionelle for funktionel programmering (til at konstruere termerne)] til at implementere dette, men det kan man jo overveje. Grunden til, at dette er så vigtigt, er, at det er helt essentielt, at tekster løbende kan opsplittes af brugerfællesskabet, hvorved delelementerne kan vurderes, bl.a.\ for korrekthed og forståelighed/entydighed, uddybes, refereres fra og til, erstattes og/eller grupperes med alternative versioner. Men, kunne man spørge, kunne man ikke bare bruge tripletter til at danne disse strukturer? Jo, og det ville give god mening, hvis teksterne var mere fastlagte, men da de netop gerne i høj grad skal være dynamiske og varierende, så er der ikke rigtigt andet at gøre, end at give mulighed for komposit-termer, således at man ikke skal gemme en ny kopi af hvert barn for hver version af forælderen. Samtidigt vil det også være fornuftigt at skelne mellem udsagn om opbygningen af tekster/datastrukturer og prædikater og relationer omkring dem, da den næste vigtige ting omkring dette system frem for gængse triplet-ontologi-systemer er, at alle prædikater og relationer kun skal betragtes som forslag, som så kan tildeles en visse point (bl.a.\ korrektheds-/sandsynlighedspoint) af brugerne, og aldrig som absolutter. Men da data-/tekststrukturer \emph{er} absolutte, så er det jo meget fornuftigt at skille de to ting ad således. *[Dette udsagn kommer jeg til at ændre igen; de to ting skal faktisk ses som to sider af den samme sag, hvor brugerens point-valg så i sidste ende kommer til at afgøre, hvordan termen skal vises og hvilke relationer skal inkluderes og på hvilken måde (skal det være eksterne referencer eller skal de inkluderes i selve det viste, sammensatte term?). Men det kommer jeg tilbage til --- og det betyder ikke så meget her.] Alle termer skal så kunne tildeles atomare point af brugerne, hvilket blot vil sige, at en tupel med bruger-id, point-id (svarende til et navn, men det er nok fornuftigt at bruge en reference i stedet, så man herved ikke får problemer med slå- og stavefejl) og point-værdi, samt også lige dato, for det bliver meget nyttigt at holde styr på for alle uploads til databasen. Relationer (og prædikater) varierer fra den mere gængse/konventionelle ontologi-teknologi primært fordi disse også skal kunne tilføjes point på samme måde som termerne. Bemærk at ``korrektheden'' af en tekst-term siger noget om selve teksten, hvor ``korrektheden'' er en relation (altså en variabel proposition --- ikke en database-relation eller sæt-teoretisk relation) selvfølgelig siger noget om, hvordan teksten relaterer sig til noget andet. Brugere skal så alle have muligheden for at instansiere nye point-id'er, relations-id'er (for her vil man på samme måde også gerne bruge id'er og ikke strenge) samt nye relationer, som vedkommende jo så kan være den første til at vurdere. (Og brugere skal selvsagt også kunne uploade nye atomare termer og nye kompositioner af eksisterende termer.) Så langt, så godt. Så kommer vi til de afhængige, ikke-atomare point og til pointaggregater. \ldots Nå nej, inden da skal jeg lige nævne, at det ikke behøver at være et bruger-id, der er tilknyttes diverse uploads, men også kan være certifikater, som ikke nødvendigvis behøver at være forbundne med brugerkontoer, men som også kan komme fra tredjepartsinstanser og fra ikke-brugere generelt. Et certifikat kan jo så bare signere den resterende del af den uploadede tupel *(inklusiv datoen \ldots\ Hm, og måske man også kunne kræve en løsning til et captcha-puslespil genereret som en del af signaturen, så det er bevist, at et menneske har lagt noget tid i at forme uploadet\ldots\ Hm, sikkert en meget god idé, men det bør dog nok ikke være påkrævet, men i stedet være valgfrit i det grundlæggende lag). \ldots Hm, nu hvor jeg tænker over det, er det nok bedre, hvis man udelukkende bruger certifikater frem for bruger-id'er, således at der ikke er nogen brugerkontoer i det underliggende lag af systemet. *(Hov, jeg glemte også at nævne: Brugere i sig selv skal også kunne have uploadet vurderinger omkring sig fra andre brugere. Hvis brugere vælger at uploade ufine atomare point og bruge disse til at vurdere brugere, gør det ikke så meget, for andre brugere får ikke disse ting at se, medmindre de selv inkluderer disse point i deres queries. Og da der vist ikke som sådan bliver nogen begrænsninger i dette grundlæggende lag på serverne, så kan server-instanserne jo også markedsføre sig på ikke at inkludere sådanne pointtyper.) Nå, tilbage til pointene: Hvis vi tænker tilbage på, hvad man kan med nævnte ML-teknik, så har vi jo mulighed for at selektere en mængde af termer og relationer/prædikater og se på, hvordan brugere (som kan identificeres via deres certifikater/signaturer) vurderer denne del-ontologi samlet set, og finde frem til korrelationsvektorerne. Det er så ikke ufornuftigt at gå ud fra, at disse korrelationsvektorer kommer til at udgøre relevante akser, hvorpå brugerne kan variere i deres interesser og holdninger. Ved dermed at finde frem til nogen sigende del-ontologier, har man så her en ret nem måde, at finde frem til forskellige tendenser, som man kan bruge til at kvalificere sig selv og sammenligne sig med andre brugere. Endnu et springende punkt ved idéen er så, at brugere, ligesom at de selv kommer til at tilføje de atomare point, også kommer til selv at kunne uploade afhængige point. Dette skal så gerne kunne ske i lag, hvor hvert lag af point kan have dets point defineret på baggrund af pointdefinitionerne fra de underliggende lag. Disse definitioner må gerne være selv ret komplekse, for det kan altid bare være op til serverne, om de vil understøtte en vis pointdefinition ved så at se på, hvor populær pointtypen er versus, hvor tung den er at opdatere og lave udregninger på baggrund af. Så hvis vi går tilbage til nævnte ML-teknik, så kunne et afhængigt point eksempelvis defineres ved at tage en vis brugermængde, konstant eller helt eller delvist defineret ud fra visse brugerpoint, og eventuelt med en vis vægtning af brugerne i mængden tilføjet, og så se på, hvad disse brugere vurderer er en repræsentativ del-ontologi (da udregningen måske bliver for tung, hvis man bare skulle se på al data på én gang), og fra denne del-ontologi kan man se på, hvad brugerne har givet af vurderinger, når det kommer til en vis mængde af pointtyper, og finde frem til korrelationsvektorerne. Man kunne i øvrigt også inkludere en vægtning af brugerne i sidstnævnte udregning, hvis dette findes nyttigt (enten pga.\ omfanget eller kvaliteten af udregningen eller begge). Det korte af det lange er her, at man kan udpege visse brugermængder og se på, hvordan de giver visse typer vurderinger og udregne en parametrisering, hvor akserne (sandsynligvis) kommer til at repræsentere holdnings- og/eller interessegrupperinger for brugermængden, hvilket så forhåbentligt videre kan være relevant for brugerskaren generelt. De herved opnåede parametre kan så plukkes ud (og med nævnte ML-teknik vil formlen så bare blive en linearkombination af atomare point) og foreslås som afhængige point til systemet (eller måske bare de mest betydende af parametrene). Hvis disse parametre så rigtigt nok peger på relevante tendenser for brugerne, så kan sådanne point jo blive nyttige for videre udregninger. Dette kan så eksempelvis være udregninger, hvor en server skal servicere en bruger med noget, der forhåbentligt er relevant for brugeren, eller det kunne være udregninger som indgår i definitionerne af endnu mere komplicerede point. Bemærk, at hvis man definerer pointet som nævnt, så kan det afhænge af udviklingen af systemet. For hvis nogen af de involverede brugere kommer med nye vurderinger i vores eksempel, vil dette jo ændre udregningen, og man vil opnå en lidt anderledes linearkombination til sidst som resultat af dette. Man kan undgå dette ved at begrænse data efter en vis dato, men man kan måske også med fordel i visse tilfælde tillade, at pointdefinitionerne er flydende. Det er nemlig tanken at pointdefinitionerne skal uploades til serverne, så serverne selv kan opdatere pointværdierne af disse afhængige point løbende. Så det er altså tanken med de afhængige point. Serverne skal så i princippet kun gemme én værdi for de afhængige point (som serveren nemlig selv bare regner ud), men ellers kommer de til at fungere ligesom de atomare point i bund og grund. Serverne kan dog vælge også at beholde gamle værdier, så brugerne også kan benytte nogle mere faste parametre, hvor dette er gavnligt, og/eller så de kan ``gå lidt tilbage i tiden'' på deres søgninger. Og afrundingsvist kan det så slås fast, at måden hvorpå brugere kan finde frem til gode afhængige point --- og atomare point for den sags skyld --- fra andre brugere vil så selvfølgelig være, hvis at selv point kan vurderes med point af brugerne i dette system (hvilket så selvfølgelig vil være i form af mere simple pointtyper, så som (men ikke begrænset til) den kendte op-eller-ned pointtype, som man ser de fleste steder til at give et mål for populariteten af et objekt). Så pointtyperne bliver faktisk også en slags termer i systemet. Og når vi behandler point lidt ligesom termer, hvorfor så ikke også åbne op for, at de kan få uploadet relationer/prædikater omkring sig? For så kan deres betydning nemlig forklares og sidenhen vurderes af brugerfællesskabet også.

Da der ikke som sådan skal være nogen begrænsninger i det grundlæggende lag for, hvor komplicerede servere kan vælge at tilbyde deres afhængige point at være, så kan vi hurtigt se, at alle mulige formelle metoder til at verificere relationer, og/eller point med en bestemt, præcis semantisk fortolkning, kan tilføjes løbende, også selvom systemet starter helt blødt. Man kunne eksempelvis tilføje afhængige point, der måler den gennemsnitlige ordlængde i en tekst, ser på om to tekster dækker samme område, tæller antal sætninger i teksten, verificerer grammatikken og/eller andre konventioner, som teksten kan overholde eller alle mulige andre ting. Bemærk, at nogen af disse eksempler ikke behøver at afhænge af andre point, og at `afhængige point' altså kan afhænge af alt muligt; både af termernes og relationernes opbygning samt deres samlede point (med pointtyper fra et lavere-liggende lag end pågældende afhængige point selv). Man kunne eksempelvis også forestille sig en afhængig pointtype, som benyttes til automatisk at vurdere korrektheden af en term eller en relation ved at lave udregninger på dem, a la hvad en refleksiv theorem prover ville kunne lave. Og på denne måde kan systemet altså gå hen og inkludere alle de samme muligheder, som hvis vi byggede systemet på baggrund af matematisk logik (som jeg tidligere har foreslået det i dette notesæt). 

%Træ-/farve-point, skal jeg nævne dette? Ah, jamen det har jeg jo lige gjort. x) (så tjek)

Hvilke nogle ting skal man så dele i dette netværk? Jo, svaret er jo selv sagt alt muligt i bund og grund. Men det vil nok ikke være nogen dum idé, hvis termerne alligevel begrænses til en undermængde af (sanitized) HTML. Ved at bruge HTML bliver det så nemt at implementere et interface til at se de del-ontologier, som man får serveret. \ldots\ \ldots



(02.06.21) Okay, jeg var ved at skrive videre her, men de næste paragrafer, jeg var i gang med blev aldrig gjort færdige, fordi jeg så fandt nogle ting, jeg skulle tænke over. Det har jeg gjort nu, og nu er jeg klar til at skrive videre, men jeg kan så se, at det nok bør blive i en sektion for sig. Så nu laver jeg endnu en sektion her nedenfor, hvor jeg forklarer om mine seneste idéer. Da disse idéer også, som min plan er nu, vil være de første, jeg kommer til at fokusere på, når dette notesæt er afsluttet, så kan det godt være, at jeg udformer sektionen, så den også kommer til at redegøre for denne samlede plan, jeg har nu. Og så kan jeg bare referere tilbage til sektionen i min afrunding af notesættet, hvorefter jeg så med det samme kan gå videre til renskrivning og prototype-design. Inden jeg afslutter denne sektion bør jeg dog lige forklare noget omkring min idé til dynamiske afstemninger om politik-/ledelses-modeller. Jeg bør nemlig lige understrege, at idéen altså har ændret sig nu, så mine tidligere paragrafer og sektioner (ovenfor og nedenfor, selvom jeg måske udkommenterer det, der står nedenfor) omring emnet er altså ikke tilstrækkelige. Jeg vil dog ikke forklare mine ændringer til idéen her, men vil gøre det i den nævnte (under)sektion, som så altså kommer til at efterfølge denne. 
*(Nej, det passer ikke. Jo, jeg vil skrive om mine seneste tanker omkring de dynamiske afstemninger i den følgende sektion, for det passer nemlig også godt til det overordnede emne. Men jeg vil dog gøre sektionen omkring min nye tanker omkring en mulig begyndelse til det semantiske web til en selvstændig sektion efter denne.)













\subsubsection{Mere om, og rettelser til, min idé om dynamiske (model-)afstemninger}
I en sektion nedenfor, som nu vil blive udkommenteret, skrev jeg følgende:\\
\\
{\slshape
\textbf{Idé til en form for online afstemning (skitse-noter)}\\
Denne idé (som jeg også har beskrevet under sektion \ref{kundedrevet} ovenfor) handler om en demokratisk afstemningsproces, hvor folk løbende kan ændre deres stemmer når som helst online, hvorved de hermed så kan begynde trække de politiske retningslinjer i en ny retning. Disse politiske retningslinjer behøver så ikke nødvendigvis at handle om regeringers politik, men kan også sagtens handle om den politiske retning for en virksomhed eller forening, og måske endda kun for et helt bestemt område for denne virksomhed/forening. Et eksempel kunne være, hvad jeg foreslår omkring lønningsmodellen for en `kundedrevet virksomhed,' hvilket jeg bl.a.\ skriver om her i \textbf{Mine idé til en mere kundedrevet virksomhed}-undersektionen. 

%Da det hele faktisk bliver noget teknisk og detaljeret er det ret vigtigt, at forklare målene og fordelene først. Så det må jeg lige huske på til når jeg skal skrive en bedre og mere ren version.

Idéen er så i første gang at formulere en metamodel --- altså en model over, hvilke nogle modeller man kan danne i systemet, hvor en sådan metamodel altså eksempelvis kunne være i form af en definition af nogle formelle byggesten til disse, lad os kalde dem `indre modeller' her --- over hvilke nogle (indre) politiske modeller, som stemmedeltagerne samlet set kan indstemme i processen. Her er det så meningen, at virksomheden eller det politiske parti, eller hvad det kan være, skal binde sig juridisk (eller via andre former for løfter, man er interesseret i at holde) til at følge denne model. Hvis der i denne forbindelse kan være undtagelser til, hvor præcist man kan følge en indstemt model, så kan man så vurdere, om man vil prøve at inkludere disse undtagelser som en del af modellernes byggesten (i.e.\ af metamodellen), eller om man vil formulere undtagelserne for sig, samt en protokol for, hvad man gør i stedet, hvis modellen stemmes ind på et område, hvor man ikke kan følge den præcist. Det kan nemlig være, som jeg ser det, at det sidstnævnte kunne blive lettere i praksis.

Man kan så se metamodellen som en definition af et samlet \emph{parameterrum} (og hvis læseren ikke kender dette til begreb, menes der altså et slags koordinatsystem af parametre --- man kan således forestille sig et to- eller tredimensionalt koordinatsystem i hovedet, men der kan dog godt være tale om flere end tre parametre), for det hører sig nemlig med, at metamodellen gerne skal indeholde muligheden for variable (reelle) parametre i dens indre modeller. Metamodellen må dog også gerne kunne indeholde diskrete (i.e.\ ikke-kontinuerte) overgange imellem forskellige indre modeller. Samlet set kan vi altså tænke på det samlede parameterrum, som metamodellen definerer, som en mængde af kontinuerte parameterrum, hver især med et vist helt antal dimensioner og med en euklidisk geometri (hvilket bare betyder, at de kan ses som helt normale koordinatsystemer med vinkelrette akser).

Mit forslag går så på, at folk stemmer ved at påføre den nuværende politiske model, som kan ses som et punkt i et af disse parameterrum, en hvis \emph{kraft} (som begrebet kendes fra fysikken (søg på `klassisk mekanik' eller på `Newtons love')), hvorved de altså så kan trække ``punktet'' i en vis retning (og jeg burde næsten også sætte anførselstegn omkring `retning' her, hvis ikke metaforen omkring at ``trække noget i en politisk retning'' var alment kendt og forståelig). Og for så at folk ikke nødvendigvis skal holde øje hele tiden, og se om nu de skal ændre kraftretningen, hvis punktet ændrer placering, så foreslår jeg, at folk derfor bare får lov at stemme med et helt \emph{kraftfelt} (som også kendes fra fysikken, se f.eks.\ `elektromagnetisme' og særligt måske `det elektriske felt'). Herved kan stemmererne således i princippet bare stemme én gang i hele sit liv --- eller i metamodellens liv, hvis nu denne kommer til at blive opdateret undervejs --- med et kraftfelt, der fylder hele metamodellen, og så kan ens stemmekraft derfra bare være givet ud fra denne. Men hvis en stemmer ændre holdning og/eller får ny information, skal denne altså til hver en tid, som nævnt, have mulighed for at logge på og ændre sit `stemmekraftfelt,' som vi kan kalde det. 

Nu mangler jeg dog, som man måske vil have bemærket, at svare på, hvad man gør, når det kommer til diskrete overgange mellem de indre modeller. Med andre ord hvad gør vi for at sikre, at brugere også kan bevæge punktet, der angiver den nuværende politiske model, imellem to adskilte koordinatsystemer? 
%Mit forslag er her, at man sørger for, at alle særskilte, %variable\footnote{Og med `variable' modeller mener jeg altså her nærmere bestemt modeller, som er parametriseret udelukkende med reelle parametre.}
%kontinuerte indre modeller, eller alle koordinatsystemerne, om man vil, selv bliver klistret op i et  %(euklidisk\footnote{Medmindre jeg tager fejl, og at euklidiske rum ikke kan betegne rum med et uendeligt antal dimensioner, hvilket jeg nemlig vil foreslå, hvis nu metamodellen udgør en uendelig mængde af kontinuerte indre modeller.}) %Nå nej, jeg vil jo egentligt foreslå en hypersfære, så never mind..
%geometrisk rum, således at hver stemmers stemmekraft også kan gives kraftretninger, der forsøger at trække modelpunktet på tværs af de særskilte indre modeller. Dette kan implementeres ... Hov, der er noget jeg har overset her. Enten skal metamodellen sørge for at de forskellige modeller er arrangeret på en måde, så de faktisk selv kan beskrives via et begrænset antal parametre, og således at "koordinatsystemerne" selv kan sættes op i et overordnet koordinatsystem med et konstant, endeligt antal dimensioner, eller også skal folk jo bare stemme om de særskilte, kontinuerte indre modeller, hvor den mest populære så vinder; for en hyperfsære med antal dimensioner lig med antal særskilte modeller vil jo bare give det samme i praksis. Hm... Oh well, metamodellen vil jo alt andet end lige kunne ordnes i et endeligt antal dimensioner, for den vil jo i sidste ende være defineret ud fra en endelig mængde information. Og det vil nok ikke være vildt svært at beslutte sig for en fornuftig ordning. Så ja, jeg vil faktisk forslå at metamodellen parametriseres fra start, så det hele bliver ét rum effektivt set, og hvor alle diskrete overgange så modelleres som kontinuerte, men bare hvor der så sker en overgang, når den kontinuerte parameter overskrider en tærskel (halvvejs i rummet) imellem to modeller med diskrete forskellige. Så ja, lidt kort sagt skal metamodellen defineres så alle diskrete forskellige gøre kontinuerte, men hvor der så bare "rundes op og ned" for alle indre modeller med en ikke-heltallig værdi, for noget der kun giver mening som en diskret, heltallig størrelse. Hvis vi ser metamodellen som en samling model-byggeklodser, hvor nogle klodser så selv kan indeholde parametre, så skal både alle disse klods-parametre altså transformeres om til reelle parametre, men også alle spørgsmål om, hvorvidt man placerer en given klods eller ej, skal altså også modelleres via reelle parametre (således at de samlede "figurer," man kan bygge, altså også opstilles i et kontinuert koordinatsystem).
Mit forslag er her, at man sørger for at transformere hele den samlede metamodel om til en kontinuert model ved for det første at omdanne alle diskrete parametre, der kan indgå i de variable indre modeller, til reelle parametre, hvor man så bare runder værdien af til nærmeste heltal, når det kommer til semantikken. Dette er dog ikke helt nok, for man må sikkert forestille sig, at den oprindelige metamodel (i dens mere naturlige form) nok også bestod af diskrete valg ift., hvilke nogle ``klodser'' man vælger til at bygge de indre modeller op. Et realistisk billede vil således nok være, at metamodellen udgør en samling ``byggeklodser,'' som hver især kan indeholde diskrete og kontinuerte parametre i sig. Hvis vi ser bort fra disse ``byggeklods-parametre,'' er der altså stadig et helt rum af muligheder, ift.\ hvordan man sætter disse klodser. Det bør dog også her være muligt, at parametrisere dette mulighedsrum, så det bliver et faktisk parameterrum --- og endda med endeligt mange parametre realistisk set. Disse parametre skal selvfølgelig så selv vælges som reelle, således at valget mellem at sætte en klods eller ej modelleres som noget kontinuert, men hvor man igen så bare runder udfaldet af til den nærmeste meningsfulde diskrete værdi. Herved kan metamodellen gøres fuldt kontinuert. %Bemærk, at hvis vi antager, at ``byggeklodsernes'' antal af parametre er begrænset (altså aldrig overstiger et vist antal), så vil metamodellen således kunne transformeres om til en kontinuert rum med et endeligt antal dimensioner, nemlig af $n+m$ dimensioner, hvor $n$ er den øvre grænse ... Nå nej, for antallet af klodser er jo alt andet end lige ikke begrænset..
Bemærk dog, at hvis antallet af ``byggeklodser'' ikke er begrænset, og/eller hvis antallet af klods-parametre ikke er det, så vil rummet have uendeligt mange dimensioner. Dette gør dog ikke umiddelbart noget. For om man spiller et spil, hvor hver deltager kan hive et punktobjekt i 10.000 forskellige retninger, inklusiv blandede retninger, eller i et uendeligt antal retninger, betyder ikke noget i praksis; efter et vist punkt vil det ikke ændre billedet længere, hvis man tilføjer flere nuancer til de, i dette tilfælde politiske, retninger, man kan trække i. Lad mig dog lige for illustrationens skyld antage, at antallet både af klodser og af klodsparametre er begrænsede, og lad os kalde disse grænser for $m$ og $k$. Hvis antallet af parametre, man skal bruge for at definere det samlede rum af mulige ``byggekonstruktioner,'' så er lig $n$, så vil samlede antal dimensioner af det rum, som stemmererne skal definere deres kraftfelt i, være lig $n+m k$. Bemærk så i denne forbindelse, at der vil være områder af det samlede rum, hvor en parameters værdi ikke vil have betydning for semantikken bag modellen. I dette tilfælde vil den eneste grund til at lægge sin kraft delvist på denne parameter så bare være, hvis man er i grænselaget tæt på en anden model (hvor der er en diskret overgang i modelsemantikken), hvori denne parameter igen bliver betydende, og man derfor på forhånd derfor vil sørger for at justere denne, så den har en fordelagtig værdi, hvis denne grænse skulle overskrides. Og bemærk videre, at selv hvis vi ikke begrænser $m$ og $k$, så vil der i dette billede (altså det med byggeklodser og klodsparametre, hvilket jeg mener, sandsynligvis bliver et godt billede på forholdene) til hvert et tidspunkt kun være et endeligt antal klodser i spil, og dermed vil der altid kun være et endeligt antal parametre, der er betydende for modelsemantikken til et givent tidspunkt. Og hermed kommer folk altså ikke til i praksis at skulle overveje uendeligt mange parametre, selv på trods af at rummet altså i teorien er af et uendeligt antal dimensioner. 

Nu kunne man så spørge sig selv, hvorfor bruge så meget energi på at forklare, hvordan metamodellen kan opbygges? Kan man ikke bare sørge for at finde frem til en eller anden model, der af semantisk dækkende for området, og som kan parametriseres med reelle parametre, og så må den være god nok? Tjo, det er ikke sikkert, efter min mening. For én ting er, om den samlede model i sidste ende bliver semantisk dækkende for området, men en anden ting er, hvordan metrikken (altså den geometriske udformning, der definerer afstandene imellem punkter) af modellen kommer til at blive, for dette bliver jo betydende, når man skal til at hive og trække %... Hov.. Hm, kom jeg ikke netop frem til, at metrikken ikke vil betyde noget, når man gør tingene på denne måde??... ..Hm, det går vel i virkeligheden ud på, at majoriteten, der nogenlunde er tilfredse med, cirka hvor modelpunktet havner, hvis man ser bort fra interne uenigheder, ikke bare er interesseret i selve området, men også i, hvordan landskabet ser ud omkring punktet, for hvis farlige konsekvenser truer lige rundt om hjørnet, så kan det blive farligt at fokusere for meget på de interne stridigheder. Og det er nok derfor, at jeg i bund og grund vil/ville have mange semantiske gentagelser i metamodellen, så man også i sidste ende får frihed til at vælge landskabet omkring sig som overordnet majoritet. Men hertil kan man så på den anden side måske sige, at hvis bare metamodellen designes på en måde, så der altid vil være... Hm, så der altid vil være semantiske gengangere, men med vilkårligt transformerede lokale metrikker, eller så landskabet bare generelt designes glat nok?.. Det sidstnævnte kan man måske ikke forvente...(?) Hm, man kunne måske også bare indføre en variabel metrik, som folk kan bruge noget af deres stemmekraft på at ændre lokalt.. Og hvis dette potentielt giver for meget informations-garbage, så kan man bare løbende approksimere metrikken, så den bliver næsten-identisk, men får cuttet en masse information fra sig. (Eller også kan de lokale ændringer bare propagere ud i resten af parameterrummet på en måde som bevarer/begrænser informationsmængden/entropien..) Hm.. Tjo, tja, der er nok flere muligheder her, men på den anden side er det jo stadig vigtigt nok at nævne de her ting, for jeg kan sagtens tænke mig situationer, hvor man gerne vil have en metamodel i form af et eller andet specielt, muligvis konventionelt, sprog, såsom et konventionelt logisk sprog eller et naturligt sprog ligefrem (f.eks. i forbindelse med mine NL-kæder). Så kan jeg ikke bare nævne dette faktum (nemlig omkring hvorfor man måske kunne have lyst til at vælge konventionelle sprog), og så i den forbindelse nævne, at man så måske lige skal sikre sig, at model er dækkende, ikke bare ift. semantikken, men også ift. metrikken omkring ethvert givet semantisk punkt, som folk har tendens til at kredse over? Og i denne forbindelse kan jeg jo så sige, at, jamen, hvis sproget er stærkt nok, så vil man også kunne definere både metrik og semantik, som man vil have dem i sine indre modeller, og dermed kan man få det som man ønsker sig det (samlet set) i sidste ende. Ja, det må være sådan..

%Lad mig nu lige nævne et par småting angående den samlede metamodel. For det første kunne man måske tænke, om ikke det kunne blive et problem, hvis koord
%Lad mig i øvrigt lige nævne, at der alt andet end lige godt kan være


Nu har jeg så skrevet `kraftfelt' i det ovenstående, og det er også en meget god måde at se det som, fordi man ligesom kan forestille sig, at man ``trækker'' i en hvis retning (med en hvis kraft). I virkeligheden bør der dog i stedet hellere være tale om et hastighedsfelt, således at summen af alle alle folks ``kræfter'' et det givne punkt for modellen, ikke kommer til at resulterer i en acceleration, men en hastighed. Jeg tror dog, jeg vil blive ved med at kalde de et `stemmekraftfelt,' for det føles mere intuitivt (fordi tanken om at bidrage med en hastighed ikke, efter min mening, er særlig intuitiv ift.\ at bidrage med en kraft). Dette forstærkes også af, at vi er så vant til friktion i vores dagligdag, og er dermed vant til, at ting, der trækkes/skubbes, for det første går i stå af sig selv, hvis man stopper med at trække i en vis retning, og for det andet kun opnår en begrænset maksimal hastighed, hvis man vedvarende trækker i samme retning --- og billede er altså næsten ækvivalent med, hvad man opnår med et hastighedsfelt. 
Det kan faktisk også være, at det ville give mening, at folk påførte en kraft (og dermed altså en acceleration) på modelpunktet i stedet for et hastighedsboost, for det kunne gøre visse ting mere belejlige så som, hvis man genre skal tilbagelægge, hvad der viser sig at være store afstande i parameterrummet ift.\ den generelle hastighed. Man kunne så også se på, om stemmererne skal have lov til at omdanne deres kraft til en friktionskraft, der modvirker bevægelse. Men her vil jeg dog i stedet foreslå, at man bare har en ekstern afstemning (som i øvrigt også kan fungere på samme måde, hvor folk så trækker med kræfter hver især, men så bare hvor dette parameterrum altså er endimensionelt), hvor folk så kan stemme om og derved justere en skaleringsfaktor for hastighedsstørrelserne relativt til parameterrummets afstandsmetrik. På denne måde opnår man nemlig også herved, at folk ret let kommer til at kunne stabilisere modellen, hvis der er stemning for dette. %Ja, det var en bedre løsning. (14.06.21) 

%[...] Så jeg vil altså også fortsætte med at kalde det for `kraftfelter' i det følgende. %Hm, men ville det egentligt være så dumt rent faktisk at bruge kraftfleter, for så kunne man måske også implementere det, jeg lige har tænkt på angående, at deltagerne nok helst gerne også effektivt set skal kunne sætte bremsende effekter op for modellen, ved simpelthen at inkludere sådanne bremsende effekter i kraftfeltet, således at brugere kan vælge at konvertere deres kraft til en friktionskraft, som så kommer til at fungere som en statisk friktionskraft, hvis ikke den samlede trækkende/skubbende kraft overstiger den samlede friktionelle (og hvis punktobjektet også allerede står stille). Og ydermere ville det så heller ikke være så farligt, at der kan blive store afstande imellem brugbare modeller, for hvis folk styrer en acceleration og ikke en kraft, så bliver alle sådanne afstande, som man vil overkomme i flok, jo i praksis meget mindre. Så ja, er det ikke faktisk kræfter, man skal bruge, frem for hastighedsboosts (..ah 'boosts' var et godt udtryk lige at huske i øvrigt)? Det virker da næsten sådan.. (14.06.21) ..Hm, man hvis man så kan opnå en høj hastighed, som tager lang tid at bremse, så skulle man måske sætte en automatisk opskaleringsfaktor på friktionelle kræfter, hvis de over stiger 50 \%, så folk i flok endnu hurtigere kan bremse modellen igen, hvis de fleste vil dette.. Men ja, det korte af det lange er lidt, at der er mange ting, man kunne gøre her, så måske jeg bare lige kan nævne de overordnede muligheder.. Uh, en mulighed kunne jo også være i øvrigt, at folk også løbende stemte om hastighedsstørrelsen ift.\ parameterummets metrik som en ekstern afstemning.. Hm..

%Det bør være klart for enhver, at stemmerernes mulige kraftfelter skal være begrænset hver især ift., størrelsen på kræfterne. Kraftstørrelserne behøver dog ikke at være begrænset nedadtil, for der er jo intet i vejen for, at brugerne kan melde sig ubeslutsomme og/eller ligeglade i visse områder. Dette kunne f.eks.\ være i området omkring deres politiske mål, hvor de så kan have mulighed for at lave en lille kraftfri lomme. %... Hov, det skal jo ikke være en kraft som sådan; det skal jo nærmere være et hastighedsfelt...
Dette var altså de primære hovedtræk i idéen. Til disse kan jeg så lige præcisere, at der selvfølgelig skal være en øvre grænse på kraftstørrelsen i folk stemmekraftfelt. Denne grænse vil jo så i et typisk demokratisk system være det samme for enhver deltager, men i visse andre sammenhænge så som virksomhedspolitikker eller andre mere lokale og/eller private sammenhænge kkan det godt give mening at denne øvre grænse kan variere fra person til person, og også for den sags skyld variere over tid afhængigt af en eller anden udvikling. Jeg bør også lige nævne, at folk så selvfølgelig skal have mulighed for at definere deres stemmekraftfelt ud fra en matematisk funktion (som dog selvfølgelig gerne må være stykvist defineret), så at de ikke er tvunget til at justere kraftvektorerne i hvert punkt af parameterrummet på egen hånd. Bemærk i denne forbindelse, at der heller ingenting er til hinder for, at folk bare vælger deres stemmekraftfelt ud fra visse populære muligheder. Dermed behøver folk altså ikke selv lægge en masse arbejde i det, hvis de ikke har lyst; de får muligheden for lægge meget arbejde og tanke i deres stemmekraftfelt, men kan altså også bare vælge én, de har fået anbefalet. Derfor har de så også mulighed for at bruge meget mindre energi på afstemningssystemet, end hvad et konventionelt afstemningssystem kræver, for her skal man jo jævnligt møde op og afgive sin stemme, også selvom ens stemme måske forbliver det samme det meste af ens liv. 

\ldots%...
}
\\
Men nu har idéen altså udviklet sig lidt siden da. 

For det første\ldots \\\\
%Det bliver egentligt ret simpelt. Basalt set bliver det bare et stemme propositioner ind og ud af mængden, der kan stemmes om (så folk er forberedt inden at afstemningen om en ny proposition sættes i gang), og derefter er der bare at have en afstemning her. Så kan der så være begrænsninger ovenpå, der så at sige retter den endelige beslutning, og som måske også bringer en inerti på banen, så der går noget tid før, at afstemningsresultatet ændrer sig, og til at man vedtager ændringerne (og hvor man så muligvis også kan have det sådan, at det heller ikke må bevæge sig for meget frem og tilbage før vedtagelsen og/eller at man i en eller anden grad kigger på et moving average frem for de egentlige udsving..). Og så skal de bare lige nævnes, at propositioner godt kan snakke om repræsentanters (fremtidige) bestemmelser osv., i stedet for at lave beslutningerne direkte, og i øvrigt kan begrænsningerne ovenpå også sættes til at indebære, at der inden for visse emner skal vælges repræsentanter frem for, at afstemningen afgør beslutningerne direkte. En pointe er dog i øvrigt, at begrænsningerne kun skal lægges ind over de egentlige stemmeresultater, og at disse alligevel skal være helt synlige, så man altid også har mulighed for at se, hvad den egentlige folkstemning ser ud til at være. Eller bare 'stemning;' der behøves ikke være tale om et folk. Det kan være alle mulige grupper og foreninger, der kan bruge et sådant system. Og det koster aldrig særligt meget, og det kan virkeligt blive et gavnligt redskab, tror jeg, for grupper til at se, hvad vej vinden blæser i gruppen og hvilke muligheder man har, når det kommer til at handle og/eller kæmpe sig til diverse tiltag --- og også bare til at finde frem til mulighederne i første omgang. Så derfor tror jeg, at denne simple, simple teknologi kan blive rigtig vigtig. Og man kunne derfor også med fordel lave en app, der implementere systemet. På denne app skal folk altså have frihed til at oprette diverse grupper og begynde denne afstemning. Og når der så ser ud til, at kunne komme noget fornuftigt ud af resultaterne, kan man så handle med diverse instanser om at oprette en aftale, hvor der (muligvis) sættes en begrænsning ovenpå afstemningen, og hvor resultatet så efter denne begrænsning kan bruges til at føre politik hos instansen. Hvad instansen kan opnå ved dette, vil så være, at gruppen kan give løfter om visse ting til gengæld, f.eks. at forøge den indbyrdes efterspørgsel af instansens produktion. Og hvis folk bare er grundigt nok identificerede før aftalen, vil det være dumt for folk ikke at holde disse løfter, for så vil andre grupper på appen jo ikke bryde sig om at optage en i andre nutidige eller fremtidige grupper (for det giver ikke rigtigt mening at optage løftebrydere i ens gruppe, for hvad kan man så bruge gruppen til?). Og herved mister folk så faktisk effektivt set kapital på at bryde sine løfter, fordi de så vil gå glip af fremtidige tilbud, som de ellers kunne have været en del af.



Lige indtil jeg får det skrevet bedre: *(Eller måske nøjes jeg bare med det i den følgende form (hvad jeg også overvejede, kunne blive tilfældet, da jeg indsatte det)\ldots)\\\\
{\slshape
\%Det bliver egentligt ret simpelt. Basalt set bliver det bare et stemme propositioner ind og ud af mængden, der kan stemmes om (så folk er forberedt inden at afstemningen om en ny proposition sættes i gang), og derefter er der bare at have en afstemning her. Så kan der så være begrænsninger ovenpå, der så at sige retter den endelige beslutning, og som måske også bringer en inerti på banen, så der går noget tid før, at afstemningsresultatet ændrer sig, og til at man vedtager ændringerne (og hvor man så muligvis også kan have det sådan, at det heller ikke må bevæge sig for meget frem og tilbage før vedtagelsen og/eller at man i en eller anden grad kigger på et moving average frem for de egentlige udsving..). Og så skal de bare lige nævnes, at propositioner godt kan snakke om repræsentanters (fremtidige) bestemmelser osv., i stedet for at lave beslutningerne direkte, og i øvrigt kan begrænsningerne ovenpå også sættes til at indebære, at der inden for visse emner skal vælges repræsentanter frem for, at afstemningen afgør beslutningerne direkte. En pointe er dog i øvrigt, at begrænsningerne kun skal lægges ind over de egentlige stemmeresultater, og at disse alligevel skal være helt synlige, så man altid også har mulighed for at se, hvad den egentlige folkestemning ser ud til at være. Eller bare 'stemning;' der behøves ikke være tale om et folk. Det kan være alle mulige grupper og foreninger, der kan bruge et sådant system. Og det koster aldrig særligt meget, og det kan virkeligt blive et gavnligt redskab, tror jeg, for grupper til at se, hvad vej vinden blæser i gruppen og hvilke muligheder man har, når det kommer til at handle og/eller kæmpe sig til diverse tiltag --- og også bare til at finde frem til mulighederne i første omgang. Så derfor tror jeg, at denne simple, simple teknologi kan blive rigtig vigtig. Og man kunne derfor også med fordel lave en app, der implementere systemet. På denne app skal folk altså have frihed til at oprette diverse grupper og begynde denne afstemning. Og når der så ser ud til, at kunne komme noget fornuftigt ud af resultaterne, kan man så handle med diverse instanser om at oprette en aftale, hvor der (muligvis) sættes en begrænsning ovenpå afstemningen, og hvor resultatet så efter denne begrænsning kan bruges til at føre politik hos instansen. Hvad instansen kan opnå ved dette, vil så være, at gruppen kan give løfter om visse ting til gengæld, f.eks. at forøge den indbyrdes efterspørgsel af instansens produktion. Og hvis folk bare er grundigt nok identificerede før aftalen, vil det være dumt for folk ikke at holde disse løfter, for så vil andre grupper på appen jo ikke bryde sig om at optage en i andre nutidige eller fremtidige grupper (for det giver ikke rigtigt mening at optage løftebrydere i ens gruppe, for hvad kan man så bruge gruppen til?). Og herved mister folk så faktisk effektivt set kapital på at bryde sine løfter, fordi de så vil gå glip af fremtidige tilbud, som de ellers kunne have været en del af.
}


%DM-afstemning:
%repræsentanter..
%Det der med: ikke boost-størrelse, men metrik.

\ldots

(18.10.21) Nej, jeg gider egentligt ikke sige så meget mere til denne idé. Jo, der kan godt en vis idé i at danne en app / en platform, hvor folk kan stemme udsagn ind i modeller, som så f.eks.\ kan repræsentere handlings-/løsningsforslag, men sammenlignet med mine nuværende idéer til diskussionsplatforme og til ``civilforeninger,'' så synes jeg ikke, at det giver så meget mening, at tænke mere i detaljer med denne idé. Jeg har vist skrevet det, der er værd at sige for nu. :)







\subsection{Ny tilgang til det semantiske web %m.m.\ 
%	--- en samling af nogle af mine tidligere idéer men på opdateret form
	(02.06.21)}
%\ldots Og disse idéer (hvis vi skal fortsætte på overskriften) bliver så også, som planen er nu, hvad jeg vil gå videre med først, så denne sektion kommer muligvis til i det hele taget også at inkludere en opsummering af alle de idéer, jeg vil arbejde videre med efter denne (selv-rettede (e.i.\ af mig selv og til mig selv)) noteskrivning. 

%Det første jeg vil fokusere på at gå videre med, er min
I denne sektion vil jeg fortsætte med at skrive om den
idé, som jeg begyndte at komme ind på i forrige sektion, om at starte det semantiske web ret blødt og altså ikke med et grundlag af formel logik, men i stedet bare et grundlag af brugerdefinerede prædikater (/relationer, men jeg vil bare kalde det hele `prædikater' i det følgende) og point (som jeg nævnte i forrige sektion), hvor brugerne altså også bare selv tillægger disse en fortolkningsdefinition, og hvor denne så ikke behøver at være så formelt beskrevet. På en måde kommer min nye version af idéen altså til at minde meget om gængs sem-web-teknologi (`sem-web' = `semantisk web'), hvor man jo heller ikke tænker på, at prædikater i sidste ende bør bindes op på nogle semantiske atomer (som jeg jo ellers lidt har tænkt selv), men hvor man også bare instantierer, hvad end prædikater man lige har brug for, og ikke nødvendigvis tænker på at prøve at bygge dem ud fra eksisterende byggesten. Forskellen er dog, at 
%mit fokus ikke er på, at tilføje mark-up til det eksisterende internet, så man hurtigere kan finde frem til data, man på at have en database af semantiske ontologier (eller én samlet ontologi, om man vil), hvor folk kan analysere ...
%Ikke bare (wiki)data men hele wikiapedia skal trækkes ind som ontologi, svarer denne idé til..
jeg vil have brugerdefinerede vurderingspoint og brugerdefinerede algoritmer som noget centralt, så brugerne selv kan bestemme, hvilke algoritmer og filtre der bliver på de endelige ontologier, de får serveret af servernetværket.

Jeg kan starte med at præsentere idéen om et nyt vidensdelingsdelings-paradigme (if I may be so bold), som en vidensdelingsside altså eksempelvis kunne benytte, hvor brugere kan give point-vurderinger til vilkårlige udsnit i teksterne på siden, og hvor disse point så bl.a.\ skal kunne repræsentere `korrekthed,' `entydighed,' %hvor grundigt uddybet sætningen/paragrafen/teksten er (e.i.\ hvor lidt man selv skal tænke), 
`grad af uddybelse,' `vigtighed' (for konteksten), `let-læselighed' og `dækkende for emnet' (som teksten/tekstudsnittet var tiltænkt at skulle belyse). Man kan sikkert godt finde på flere nyttige prædikater om tekstudsnits funktion i konteksten, og der er heldigvis ingenting til hinder for, at man kan tilføje flere løbende *(hvilket man altså helt klart gerne skal kunne), men allerede med disse få tror jeg, man kan komme rigtigt langt i sig selv. Jeg tror nemlig, at hele denne tilgang, hvor man giver mulighed i brugerfællesskabet for at lave argumentationsanalyse af teksterne, om end bare med nogle vagt definerede point, hvor hver person kan have lidt sin egen fortolkning af pointene, kan bane vejen for meget større effektivitet, når det kommer til at udarbejde argumenterende og/eller forklarende tekster i et fællesskab. En del af idéen er så også at sørge for at teksterne er opbygget i en struktur, hvor man ret let kan skifte en sætning eller et tekstudsnit ud i en tekst, hvis man i fællesskabet udformer et alternativ, der opnår højere point. Da der for det første er flere point i spil, hvor forskellige brugere kan have forskellige præferencer ift.\ de mest vigtige point, og da pointene jo også kan variere, er der dog bestemt ingen grunde til at smide gamle tekstudsnit ud, bare fordi der opstår et lidt mere populært alternativ. Det er bedre at gemme alternativerne, så brugere både fortsat kan vurdere dem, men også så de simpelthen har mulighed for at se nogle alternative formuleringer, hvilket jo kun kan gavne den overordnede tekst, at der er denne mulighed. %Og med en sådan struktur i teksterne, kan man ydermere sørge for, at sætninger og paragrafer kan opdeles i en ydre argumentationsstruktur og så de indre (del-)argumenter. Og herved er det så, at `grad af uddybelse' (og også `dækkende for emnet') kan deles op i bidder. For argumentationer %... Hm, ja, hvad med at man i stedet for direkte point vurderer en formel for, hvordan børnenes point skal omregnes til at give forælderens point..? ..Lyder umiddelbart ikke helt dumt; det er jo netop lidt det, der er hele pointen med at splitte argumentationsanalysen op..
%Hm, og bliver der så en måde (og skal der blive det?), hvor man kan pege på, hvad man savner som bruger, via pointene..?.. ..Hm, for programmering er det ret simpelt: Paradigme, hvor man kan vurdere del-argumenter, samt hvordan de fungere ind i deres forælders struktur, hvorved man så bedre kan holde styr på, hvor grundigt alle del-argumenterne er blevet gennemgået i dokumentationen. Ja, så man kunne måske også bare starte der. ..For det er først, når vi når til forklarende tekster, at det først rigtigt bliver vigtigt nok at tage højde for forskellige brugeres præferencer; det behøver man næsten ikke nok, når det bare kommer til simple formelle argumentationer inden for programmering --- eller inden for matematik, når det kommer til formelle tekster og altså navnligt beviser. Ja, så egentligt ret vigtigt at splitte det op i formelle argumentationer og forklarende tekster.. Og når vi så når skridtet videre til en vidensdelingsside a la Stack Overflow eller Stak Exchange, så kommer der vel allerede lidt mere komplicerede ting ind i billedet, fordi man kan fortolke spørgsmål..? Eller nej, for program-dokumentation kunne jo i princippet også videreføres til at inkludere mere bløde paramtre om programmodulerne, f.eks. når det kommer til UX og alt det (hvilket jo også kan testes og dokumenteres formelt trods alt).. ..Ja, men derfor er der stadig ikke behov for den vilde teknologi; ikke før "den vilde teknologi" først alligevel er der, og er blevet velkendt at bruge, men der er nok ingen grund til at udvikle teknologien til dette formål.. ..Nej, det bliver mere først, når man begynder at ville have åbne applikationer, hvor brugerne kan indstille dem, som de vil have dem (og hvor ændringer af disse indstillinger så potentielt kan medføre ændringer langt nede/inde i kildekoden..).. ..Hm, og kunne man ikke også komme langt med forklarende tekster, hvis bare man løbende kan tilføje flere tekstudsnit-point, så man som læser kan grænske teksten på flere forskellige (kritik-)punkter..? For eksempel "skjuler teksten et åbent argument (som der ikke er enighed om) i forklædning som en velkendt del af hjemlen (for at aflede opmærksomheden fra, at her er et åbentstående spørgsmål)?" og alle mulige andre ting.. Hm..? Ja, og så kunne man indføre prædikat-skabeloner, der tager en pointtype som input, til at kunne tillægge forklaringer på, hvorfor at et point bør gives til et tekstudsnit (hvilke andre brugere så kan istemme sig i ved at give disse forklaringer korrekthedspoint), således at brugere, der ser, at et tekstudsnit har fået mange af en vis type point, så kan få mulighed til at navigere til at se, hvorfor at folk har givet disse point.. ..Ja, og dette er også vigtigt, så brugere direkte kan gå ind at se forklaringen (hvis en er givet), når der er et par enkelte vurderinger, der siger "ukorrekt" som eksempel, hvilket man jo må skulle kunne.. Så for de mere lukkede systemer, hvor man kan have højere grad til andre brugeres intentioner, så kan man altså nå rigtigt langt, hvis man bare giver mulighed for at tilføje flere atomare point. Og på vidensdelingssider kan man det samme i høj grad, hvis man bare også lige giver mulighed for point-aggregater, så folk bedre kan søge i tingene, men her bliver det så en god fordel, hvis man også går videre til at kunne klassificere brugerne og have dette med i pointene.. Hm.. ..Ja, og det er jo ikke vildt meget, man skal gøre for at give mulighed for disse ting.. så hvis jeg bare måske starter i det simple, så behøver man måske ikke de vildt store argumenter, for at man skal åbne det op på denne måde..? Hm.. ..Det vender jeg tilbage til. Hvad så med, om man kal kunne efterspørge tekster via pointene..? Tjo, men der kunne man jo netop bruge mere abstrakte point til at signalere ca., hvad man mener, og så tilføje forklaringer med sin bedømmelse, som specificere, hvad man mener.. Og hvad så, når det kommer til at udskifte teksudsnit --- og til at holde rede på alternativer i det hele taget..? Jo, men det er vel bare, at bruge det system jeg har tænkt mig med en database, der implementerer en semantisk struktur i teksterne og deres relationer..? ..Ja, så en stor del af hele idéen, selv når vi når til vidensdelingssiden, kan egnentligt holdes ret simpelt.. 
%Hm, kunne man ikke også med fordel bruge en konvention med at bruge emne-lister, så man kan vurdere, hvilke områder et tekstudsnit dækker, og så kunne man måske bruge dette videre til at sammenligne, om to teskter dækker det samme.. Jo, sikkert en rimeligt god idé, for dette er jo også bare i det hele taget vigtigt som læser, at kunne få en god oversigt over, også så man ret hurtigt kan se, hvad teksten mangler/ignorerer noget.. Hm, og hvad bliver der så af automatiske point i det hele taget, og hvordan kan man sikre, at ens vurderinger ikke går til spilde, fordi man brugte en senere udgået pointtype, eller hvis man bedømte en tekst, der nu er erstattet..? Hm... Tja, hvis man skal genbruge disse bidrag, så skal man jo netop kunne relatere de gamle tekster (og point) med erstatningerne.. Og brugerne kan oversætte pointene, hvis de gider.. Hm, og hvad så med tekster, kan man gøre noget smart her --- noget med formular-forælder-point og/eller point for ækvivalens mellem to tekstudsnit..? Ja, og også især ift., hvad teksterne dækker af emner (fra en liste). Ja, og det smarte man kan gøre af lige netop at bruge formular-point i stedet for bare simple point, for så giver man en ramme for, hvad teksten skal leve op til, før den kan beholde det omtalte point. ..Hm, jeg kom lige til at tænke på noget helt andet, nemlig at det må blive ret nemt at lave en bot-netværk med alle mulige meningstyper, og så bruge disse til at fremhæve særlige ting... Hm, men hvor stort et problem vil dette være, og kan man gøre noget smart imod det?.. Ja, man kan bruge person-identifikationer.. Og hertil kan man jo bruge delvist anonyme netværk, hvor man bare godkendes som en faktisk person af sine venner, men uden at det afsløres.. Hm, afsløres, hvilken konto man rent faktisk har?.. Hm.. Det kan jeg lige tænke over, men i starten må man folk i netværket jo bare identificere sig selv til en vis grad.. Ah ja, og ellers kan man bare bruge en trejdepart til at skjule sin identitet, så denne trejdepart f.eks. kan sige, "jeg har en (muligvis betalende) bruger, som jeg kender identiteten på, men er kontraktbundet til ikke at afsløre, og som siger det og det". Og tanken er så, at hvis man bruger tilstrækkeligt store tredjepartsforeninger (eller hvis de anonyme brugere skal betale, og man ikke er bange for, at instanser med mange penge laver angreb.. hm..), så behøver man ikke at bekymre sig om gengangere.. Ah, eller man kan også bare gøre, så at anynyme brugere jo kan smide ting ud i æteren, og så kan netværket af identificerede brugere jo stå for at fremhæve og vurdere disse ting.. ..Og når der kommer donationer på, så kan det betale sig, at være en bruger, der udgiver ting på andres vegne.. Ok, tilbage til formular-point.. Hm, men det ikke nærmest være nemmere bare at omvurdere de nye versioner..? Hm, dette spørgsmål hænger vel sammen med spørgsmålet om, hvor brugbart det ville være, hvis man kan efterspørge tekster, der opfylder visse parametre, via pointene; så man f.eks. kunne give formular-point, der slår til, når visse betingelser er opfyldt.. Tja, jeg kan vel om ikke andet foreslå det, for det kunne f.eks. måske især være brugbart, når det kommer til, hvor dækkende tekster er for et område, hvor man som bruger via formular-pointene kan efterspørge emner og samtidigt sørge for, at pointene ikke skal justeres for hver alternativ version, når bare emnerne stadig er der. Hm, ja, og angående korrekthed og ækvivalens mellem tekster, så kan man sikkert også ret let finde på måder at bjærge point fra tidligere versioner på.. Så ja, det kan jeg nævne.. ..Ja, det er vist meget godt, alt sammen..


Jeg tror som sagt, at denne tilgang både kan bruges, når det kommer til forklarende tekster om generel viden, der prøver at lære læseren om et specifikt emne, og også til argumenterende tekster. Sidstnævnte kunne eksempelvis være matematiske beviser eller argumentation for programmodulers (funktioner og klasser osv.) korrekthed. Og lige netop når det kommer til programmering, så tror jeg særligt min idé kunne være brugbar her. \ldots Og jeg burde måske endda starte her (når jeg skal omskrive disse noter), for det er samtidigt også lidt nemmere at beskrive. Men ja, jeg tror nemlig, at en ny konvention om at tillægge, ikke bare forklarende, men også argumenterende dokumentation til sine programmoduler, og hvor man altså bruger så min tilgang med at kunne give forskellige point (såsom dem jeg nævnte ovenfor) til tekstudsnit i et fælleskab, kunne blive rigtigt stort. Lad mig dog først lige sige, at denne ekstra dokumentation dog også helst kræver, at man benytter en IDE, hvor den uddybende dokumentation kan skrives i filer for sig, og hvor IDE'en så er i stand til automatisk at lægge et link fra kildekoden til denne ekstra dokumentation. \ldots %Hm.. Man burde lige overveje, hvordan det nemt kan implementeres.. Nu tænker jeg lidt på, om man ikke bare kunne udnytte GoTODefinition eller lignende og så have nogle filer.. vent, kunne man ikke bare bruge rent kommentarer, og så have et eksternt.. hm, på nær IDE-navigationen.. ..Men hvis man selv kan definere parsingen for at finde dokumentationsreferencerne.. Ja, det må kunne lade sig gøre!.. ..Hm, kan hele dokumentationen så ligge i en database (som netop så kan være delt), hvor man så har et program til at skrive passende dokumentationsfiler til den enkelte brugers program-filsystem, og hvor formatet så kan varieres, så IDE'en kan udnyttes til at give de passende links i program-kildekoden, men uden at man låser sig fast på en bestemt IDE eller noget? ..He, det var hurtigt, jeg fandt på dette (tog det hvad, 10 minutter?.. 15?) Nå, men det lyder altså umiddelbart som en ret god idé. ..Hm, så min idé kan faktisk nu betragtes også som en konkurrent til Git-metoden (men ikke den samme, som jeg før har tænkt; det var mere noget med at have mapper med begrænsede muligheder for aktioner..).. ..Hm, og så kan man måske forene de to teknologier ved.. tja, men hvorfor skulle man det..?.. ...Hm, kunne man sigte mod en semantisk Git..? ..Og måske et semantisk programmeringsparadigme..? Lad mig lige tænke over det første først.. Ja, et semantisk Git vil være alt for oplagt på sigt; det kommer til at ske. Men hvad inden da..?.. ...Jo, men inden da må man jo netop dele det lidt op, så.. Ja, så man bare har et sammenspil med en database, og hvor man så på GitHub kan vedlægge et link til den database.. tjo, men jeg vil også godt konkurrere med Git, så hvornår bliver skellet?.. Hm, det handler vel egentligt bare om, kommer kildekoden med ind i databasen, eller er det bare dokumentationen.. Og hvorfor ikke kommunikere rettelser i kildekoden via databasen, hvis man alligevel deler dokumentationen her?.. Ja, simpelthen. Så projektet udformes ved at man uploader kode og dukumentation til en delt database.. En database, som så selv kan have internetadgang, kan hive open source-bibliteker ind fra nettet, og hvor man i øvrigt så også på automatisere det at få den rigtige licens-tekst på sit program, hvor det også er noteret korrekt, alle de copyleft-ting, man har brugt.. Men hvor man altså ikke nødvendigvis skal dele sine egne "uploadede" løsninger med andre end sig selv eller med resten i et lokalt netværk.. Hm, og dette bliver måske så "semantisk programmering;" at man uploader ting til en database, som så gør at programmet bliver bygget op kontruktivt efterhånden, og hvor man så ikke behøver at overskrive og/eller kassere udgåede løsninger, medmindre altså man vil skabe mere plads i databasen, hvis der skulle blive brug for dette. ..Bum. Og dette kan så ske lidt ligesom at arbejde med en lokal GitHub, hvor man bare committer løsninger på lidt mere semantisk vis, og hvor man altså informerer databasen om, hvad formålet med hver ændring er. .. Ja, og hvis semantisk-database-programmet så bare kan parse ens program-filsystem, så kan en standard ændring, man ikke lige gider at tilknytte kommentarer til (hvad man dog bør, men hvis nu det bare er noget, man selv lige arbejder på hurtigt...), bare kan noteres som "en ændring til dette kode- eller dokumantations-udsnit," og så kan man jo altid tilføje kommentaren bagefter.. Ja.. Et semantisk Git.. Hm.. Og database-programmet skal jo bare opdatere hver enkelte brugeres fil-system med fodnoter omkring ændringer i starten, som brugeren så kan adoptere on-the-run, hvis det ikke overlapper, hvad man selv arbejder på, og særligt, hvis det bare drejer sig om point.. Ja, og hvis det overlapper, så kan brugeren jo bare få det at se, i form af en kommentar, der også undslipper database-programmets parsing.. Eller man kan få det i xml, som viser til semantisk, at der er tale om alternativer, hvor brugeren selv kan toggle imellem det ene og det andet, men hvor databasen kan være ligeglad med, hvad der for IDE'en er ud- og indkommenteret.. Ja, det virker alt sammen meget lovende, og (forhåbentligt) ikke vildt svært at komme i gang med. Vil lige tænke lidt mere over, hvordan man kunne implementere de første versioner af det... Ja, det bliver nok (7, 9, 13) ikke så svært at få gang i. Det er jo bare en parser, der kan oversætte brugerens ændringer til semantiske uploads, og som kan konvertere en ontologi (med point) til et filsystem.. Og der behøver ikke være nogen specielle begrænsninger på dette parser-database-program; det handler bare om at kunne oversætte fra ontologi til programmeringssprog og tilbage igen, og her kan man bare udvikle konventionerne løbende.. Umiddelbart ret nice med de her tanker; det virker som et rigtigt godt sted at starte (ift. den samlede idé)..! (06.07.21)

%(09.07.21) Jeg har fået tænkt nogle tanker i forgårs og i går --- nogle ret gode tanker faktisk. Nu har jeg muligvis et ret godt billede af, hvordan man kunne komme godt fra start med den her idé --- og i det hele taget hvordan idéen kan fungere godt i praksis. Jeg kom på, at man måske bare skulle have forklarende (NL-)titler i stedet for nøgler, og videre at folk så måske bare kunne bruge simpelt mark-up til at give en "semantisk," men stadig bare skrevet i NL, definition af et tekstudsnit, og at dette så kan ske både i en dokument-editor, men også, og dette bliver nok ret vigtigt, i alverdens uploads til internettet. Pointen bliver så, både når vi snakker editors, hvor folk udformer dokumenter i et lokalt netværk, eller når vi snakker diverse web-2.0-uploads på internettet, at folk skal kunne bruge diverse ("semantisk") mark-up'ede tekstudsnit ved at query'e et program i form af en editor- eller browser-udvidelse (som muligvis snakker sammen med et særligt program over localhost, som har adgang til databasen) til at se, hvilke nogle erstatninger der kan være til det mark-up'ede udsnit. Blandt andet skal brugere altså kunne indsætte et tekstudsnit som er et ArgumentHvorfor:LasereErFarligeForØjnene eller ArgumentHvorfor(eller 'argw' eller 'aw'; tekstudsnit-kategori-tagget kan nok med fordel forkortes, for det bliver ikke vildt mange forskellige, og det vil blive brugt ofte):Funktionen,myFun,ErKorrektIParameterintervallet,x osv. Folk der læser tekstudsnittet kan således få vist, hvis fælleskabet har udarbejdet en bedre version af udsnittet (med samme semantiske funktion), der måske har fået flere point end den oprindelige, og skal så få mulighed for at toggle mellem teksterne. En yderligere idé er så også, at folk også skal kunne definere tekstprædikater, som man så kan klistre på sit mark-up. For eksempel kunne man så sige: "LasereErFarligeForØjnene.uddybende" (selvom man også her nok gerne vil ende med at forkorte dem lidt) eller "LasereErFarligeForØjnene.ultraKort" osv. Så kan systemet både se, hvad den oprindelige tekst var ment til at være, og så kan læserne i første omgang prøve at søge på erstatninger med tilsvarende kvaliteter, hvis de gerne vil bevare formen af den originale tekst nogenlunde, men skal selvfølgelig også være frie til at toggle til eller følge en sti til og få vist eksempelvis en mere uddybende tekst i stedet for. Hvilket sprog teksten er på kunne også være en type af anden oplagt tekst-kvaliteter/-prædikater, som kunne være brugbare, og selvfølgelig også kvaliteter af, hvad teksten forventer læseren ved og kan forstå i forvejen. Min tanke er så faktisk, når det kommer til online tekster især, at brugere bare tilføjer mark-up'et, også selvom de ikke nødvendigvis er med i systemet, eller måske bare ikke er logget ind / har tilkendegivet sin brugeridentitet, men at alle folk, der læser teksten, og herved kan se, at tekstudsnittet endnu ikke er en del af databasen, simpelthen så bare selv får mulighed for at uploade pågældende tekst til databasen (hvor de så selvfølgelig kan tilkendegive, hvis de ikke selv var forfatteren til teksten, men bare uploader det på en andens vegne). Og bum, så kan systemet fungere som et overlay på alle gængse web 2.0-sider. Så mangler jeg at forklare om pointsystemet, om serverne og service providerne og om hvordan de samentiske ontologier kommer ind i billedet. Angående det sidstnævnte, så skal brugere bare simpelthen have mulighed for at oplaode forslag til alternative titler til tekstudsnittende. De skal i øvrigt samtidigt også have mulighed for at oploade forslag til, hvilke (NL-)titler er dublianter af hinanden, hvilket dog også kan gøres ved at "foreslå en alternativ titel" som så bare er en eksisterende titel. Og hvad gør man nu, hvis der er to forskellige tekst-funktioner, der er beskrevet med samme (og dermed tvetydige titel)? Ah, men alternativerne skal forslås for hver enkle tekst og ikke for titlerne i sig selv. Hermed kan folk så nemlig præcisere titlerne efterfølgende, også uden at ændre på de originale uploads. Og disse titler behøver så ikke at være i NL, men kan også være i et formelt sprog. En formel semantisk titel vil jo så være en træstruktur, hvor hver knude referere til et ord i en sprog-ontologi. Og her kan man så bare bruge et tag eller et NL-ord til disse, hvor brugeren så selv kan søge på ordet/tagget; så bemærk at jeg i det hele taget går væk fra formelle referencer i form af hrefs eller nøgler, og så simpelthen bare til ord/tags/titler, som folk selv kan slå op i databasen. Så for altså at besvare spørgsmålet om, hvordan ontologierne kommer ind i billedet, så kan dette jo simpelthen bare ske ved, netop at folk får uploadet alternative titler med formelle ord/tags i, som folk så kan navigere til. Og her mangler jeg jo bare lige at nævne, at 'OrdDefinition'/'wdef'/'wd' også bør være en essentiel tekst-kategori, og så er man jo allerede i mål ift. ontologier. Nå ja, og så skal editor-/browser-udvidelserne bare inkludere muligheden for at klikke på ord i titler og slå dem op i databasen, og konventionen for titlerne bør så lægge op til, at folk bruger mellemrum og/eller parenteser i titlerne, selvom jeg her selv skrev titeleksemplerne uden mellemrum (for det var lige nemmere i denne sammenhæng). Så "lasere er farlige for øjnene" er en mere oplagt standard titel og "er<farlige>[for(øjne)][generelt](lasere)" kunne være et eksempel på et formelt alternativ til samme titel (hvor jeg så her har brugt [] til at påhægte den adverbielle sætning til verbet (og <> til at specificere selve verbet), og hvor jeg lige også har indsat "[generelt]"-adverbiet for sætningen for at specificere dette, men man kunne jo også finde på andre konventioner). Med denne konvention skulle man så inkludere alle bøjningerne (flertalsbøjningerne i denne sammenhæng), så det er nok egentligt badre, for det første at prøve at formulere det i ental, men også ellers indføre muligheden for at give morfemer endelser i sætningerne på en formel måde i stedet, hvis man endelig vil bruge denne kasus. Bemærk så også, at omtalte "wdefs" også kan påhægtes alternative titler, så herved kan man også gøre sprog-ontologien sprog-uafhængigt (så det ikke behøver at være engelsk det hele, når man vil bruge sprog-ontologien --- og ontologien generelt). Angående pointsystemet (og serversystemet) så skal folk bare være meldt ind til en "lokal" (som dog godt kan blive vilkårligt global) service provider-instans, gerne i form af et distribueret netværk, men det kan også være private virksomheder. Det er denne instans, der har ansvar for at gemme et upload, hvis en klients underskrift er på det. Og uploads skal så selvfølgelig også kunne være point-tilkendegivelser. Her skal point-tags bare være helt frie for brugeren. Point er nærmest en slags tekst-prædikater/-kvaliteter med værdier tilknyttet, men i modsætning til de teskt-prædikater/-kvaliteter, jeg har nævnt allerede, er pointene ikke en del af titlerne, men er seperate uploads. De skal så bare være på uploads a la "bruger med brugerid x (og gerne med tilknyttet underskrift) synes at denne database-entity (som nok stort set skal være alt muligt på nær andre point-uploads), referet til via titel og nummer (for der kan jo godt være flere uploads med samme titler) skal have y point med pointnavnet z." Grunden til, at der bl.a. gerne må være underskrifter på, også selvom man stoler på sin service provider, er, at at så kan andre service providers også bedre stole på uploads'ne og så bliver det også meget lettere at rede dataen (og lave recovery), hvis en service provider skulle gå nedenom og hjem og/eller blive kompromiteret. Jeg ved ikke, om jeg lige har glemt noget, men det var nok stort set det. Nå jo, jeg har i øvrigt en idé til, hvordan man... Nå nej, inden jeg kommer ind på det: En stor idé, som jeg fik her (i går, var det), er at de afhængige point faktisk ikke behøver at blive behandlet af serverne, men faktisk kan udregnes mere på brugernes ende.. Ikke at databasen bare skal være fri for dem, og at browseren skal lave udregningerne hver gang, nej, de "afhængige point" skal gemmes i database systemet. Man kan bare gøre så at det er brugernes egen opgave at gemme pointene korrekt, og altså gøre det, så udregningerne slet ikke foretages af serverne, og dermed slet ikke er en del af det grundlæggende lag egentligt, på nær at man dog lige skal inkludere en anden lille ting, for at gøre det grundlæggende lag åbent for denne mulighed. Det eneste, man hermed skal gøre, er, at give brugere mulighed for at erklære en liste af moderatorer (i form af andre bruger-id'er, og hvor listen selvfølgelig skal kunne forkortes med en nøgle, hvor listen så i første omgang skal uploades selvstændigt til databasen). Pointen er så simpelthen bare, at disse moderatorer skal have mulighed for at slette og overskrive pågældende point-uploads i database systemet. ..Hm, måske kunne man også faktisk bruge det her, hvis brugere så faktisk \emph{kan} give point til denne type point-uploads.. Ja, det ville ikke være en dum idé, for så kan moderatorerne let finde frem til de modererede point-uploads, der så er blevet "flagged" af andre brugere (og så skal modereringen ikke afhænge af ekstern kommunikation for at flagge disse uploads til moderatorerne *(selvom dette dog vil være godt at have uanset hvad)). Ja, sådan set kunne man sikkert også åbne op for denne mulighed helt generelt for uploads i systemet, da det måske også kunne tjene andre formål. Og "afhængige point," eller rettere de "automatiske point," kan så implementeres simpelhen ved, at man bare (foruden moderatorlisten) opgiver en formel for en type point, som denne type point så bør udregnes ud fra, og så kan fællesskabet bare arbejde sammen om, med moderatorerne som overdommere, at fjerne alle point-uploads, der har udregnet forkert i en væsentlig grad ift. datoen for uploadet (hvilket skal på alle uploads, så man derved altid kan "gå tilbage i tiden"). Bum! Jeg kan så lige nævne hurtigt, som jeg skulle til at nævne før, at man også måske kan vælge at finde løsninger, så fællesskabet kan gruppere point-uploads, hvor brugerne så bare signere den ækvivalente erstatning i stedet, så alle de tidligere point-uploads så kan erstattes med en mindre pladskrævende erstatning. Men det gode er nemlig, at dette kan man gøre som man vil; der behøver nemlig ikke --- og bør faktsik ikke --- være noget krav om, at lokale fællesskaber bevarer tidligere uploads --- andet selvfølgelig end, hvad de har lovet resten af fællesskabet. Så det er altså helt frit for, hvis et lokalt fællesskab gerne vil give sig selv lov til at smide gamle tekst-uploads ud, der ikke længere bliver brugt, måske fordi de for længst er erstattet med noget bedre, og til at smide irrelavante point-uploads ud. Der er faktisk i og for sig slet ikke nogen faktiske begrænsninger i det grundlæggende lag, men bare en samling konventioner for det overordnede system, samt nogle retningslinjer for, hvad det skal kunne og hvordan det nogenlunde bør fungere.. Selvfølgelig er det dog også klart bedst med en klar konvention for databasernes udformning fra start af, så man let kan kommunikere data på tværs af alle lokale fælleskaber, så på den måde bliver der et "grundlæggende lag af begrænsninger," men her skal "begrænsninger" så bare tages med et gran salt. ..Fuck, hvor er det altså bare fedt det her..! Det her lille nye spin på idéen gør bare så meget..! Dette \emph{bliver} altså bare, hvad vi kommer til at forstå ved web 3.0, og det bliver endda ret hurtigt at fremføre! Og jeg har \emph{virkeligt} gået og savnet sådan en idé med stor og hurtig hurtig gennemslagskraft, og det føler jeg altså, at dette har..! Og så leder det direkte videre til mine p-ontologier, hvilket jeg jo ser som noget af det allervigtigste af alle mine idéer, hvilket bare gør det så meget desto federe --- for så bliver der så meget desto "færre skridt" i hele processen med at få mine idéer ud. ..!! Virkeligt. Fedt.!! (Jeg troede næsten også, jeg havde et spin på idéen, der opnåede dette, forleden, da jeg fandt på at sælge idéen med fokus på et socialt netværk, men jeg mistede lidt troen igen på, at det helt var nok. Og nu kan jeg nærmest ikke forestille mig, hvordan man kan få godt gang i sem-webbet uden disse nye idéer, ha! Hm, det er måske lidt meget sagt, men jeg føler altså virkeligt, de er vigtige.. :) Nå, men jeg kan jo selvfølgelig nå også at skifte mening igen med disse idéer, så koldt vand i blodet og 7, 9, 13 på at de også holder i længden. :)) Okay, og det sidste hængeparti, som jeg faktisk kom på her tidligere, imens jeg skrev dette (nemlig da jeg skulle begynde at skrive om point og server-fælleskaberne), er lige, at jeg måske lige bør overveje, om der kan være en måde, hvorpå server-fællesskabet kan uddele nøgler imellem sig, hvorved signaturer kan vise, at de kommer fra en af de private nøgler i fællesskabet, men ikke fra hvilken?.. Hm, det kan jeg lige tænke over, samt også tænke mere over, hvad man kan opnå med dette, men det behøver jo ikke at være noget, man behøver at overveje i det grundlæggende lag. Nå jo, jeg skal forresten også lige nævne, at der ikke bare skal være mark-up til at give en "semantisk" titel på diverse tekstudsnit, men at der også skal være mark-up for at give fod-note-referencer og/eller hyperlink-agtige-referencer (hvor man altså trykker på et ord eller en sætning (m.m.)), som altså ikke er beregnet til, at man skal kunne toggle imellem tekster, men kun er beregnet til, at man skal kunne følge linket via sin editor- eller browser-udvidelse. Det skal også nævnes at "tekst-udsnit" ikke kun skal være begrænset til tekst-tekst, men skal kunne være alt muligt hypertext-indhold, så man også bl.a. kan inkludere markup for billeder, gifs og videoer (og alt muligt andet), eksempelvis på web 2.0-sider, der tillader sådant indhold fra brugerne. (09.07.21)
%(10.07.21) Ah, og så skal det også lige nævnes, at uploads også gerne må kunne tage udgangspunkt i tidligere uploads, så man uploader en liste af rettelser, f.eks. i form af "fra punkt x, overskriv (eller indsæt) med strengen a," hvor rettelsen altså så referere til et tidligere upload, og evt. et upload, som selv er en rettelse. Herved kan serverne jo nemlig spare lidt plads. Hvis der så kommer mange rettelser nok, og/eller hvis uploadet bliver populært nok, så kan serveren jo altid transformere rettelses-uploads'ne til selvstændige tekster (så hele teksten kan hentes ét sted fra). Man kunne i dette tilfælde endda vælge at give mulighed til, at tidligere uploads, hvis disse ikke længere er så efterspurgte som de rettede versioner, transformeres omvendt, så disse bliver gemt som "rettelser" (i praksis; selvfølgelig skal brugeren stadig se det hele på den samme måde (og altid kunne se, hvad der kom først)) til de nyere versioner. Nå ja, og tekster skal i øvrigt også kunne tilføjes nyt mark-up omkring indre sektioner, hvorved serverne så bør tilføje en seperat tupel i databasen også, som svarer til en almindelig tekst-upload-tupel, men som evt. bare kan gemmes ved at referere til den tidligere tupel (hvis man vil spare lidt plads). (Igen behøves alt dette ikke at kunne ses af brugerne af systemet i deres almindelige brug af det, da det bare handler om pladsforbruget (selvfølgelig kan brugere have begrænsninger på, hvor meget de må uploade til den lokale database, hvorved serverne jo godt kunne forestilles at give mulighed for brugerne til at se, hvor meget plads deres uploads optager)). Dette leder mig så videre til at snakke om point-givning til indre sektioner. Jeg kom nemlig i tanke om i går nat, at det jo bliver meget vigtigt faktisk, at brugerne også skal kunne give point til indre sektioner, uden at disse behøves at trækkes ud som selvstændige tekster først. Dette skyldes især, at sætninger jo sagtens kan være udformet, så det kun skal forstås i den givne kontekst; ikke alle tekstudsnit er kontekst-løse (og her snakker vi ikke bare om, at ting allerede kan være introduceres og dermed have stedord refererende til sig allerede fra starten; dette kan man sagtens håndtere... Hm, bør jeg egentligt ikke lige nævne noget om dette? ..Jo, det gør bagefter dette). Sandheden af teksters udsagn kan nemlig godt afhænge af konteksten... Hm, og dette tænkte jeg jo faktisk også på i går, hvor jeg kom frem til, at man nok bør kunne give hypoteser til tekster. ..Ja, det bør man jo.. Okay, det har jeg ikke nævnt noget om, men det gør jeg bare nu: Teksters samlede titel skal foruden den normale titel (samt tekst-prædikater) også kunne indeholde hypoteser. ..Hm, og hvordan implementerer man dette, skal man bare udvide syntaksen, så der kan klistres flere parametre bag på?.. Hm, hvis jeg tænker på min '.'-notation, som jeg har brugt her for tekst-prædikaterne, kunne man så ikke bare bruge denne hele vejen, så man nærmest betragter listen af antagede hypoteser som et prædikat om teksten?.. Ja, hvorfor ikke? Så "tekst-prædikaterne" bliver bare betragtet som en del af titlen, og kan altså snakke både om indholdet såvel som formen på teksten. Og punktummerne (eller hvad man nu bruger, men punktummer er et godt forslag) bruges så bare til, at man kan klistre flere sætninger sammen i titlen, hver med forskellige kategori-tags foran. Ja, og hypoteserne til et argument kan så bare få et specielt tag for sig, f.eks. 'assumedHypothesis,' 'ahyp,' 'ah,' eller bare 'h,' og så tænker jeg bare, at serverne og brugerne tilsammen bare kan bestemme sig for, hvilket alternativt tag man søger på, hvis man doppeltklikket (eller ctrl-klikker) på sætningen/strengen efter tagget i en titel.. Hm, eller skal man bare opgive dette a la 'h(argw)'/'h(arg)'/'h(a)' eller 'h(c)' (med 'c' stående for conclusion)?.. Ja, måske er det bedre. For det er få ekstra karakterer, og så skal man ikke tænke så meget mere over dette. Og så er systemet også mere åbent for, at man kan finde på nye tags til specifikke områder, som måske bør linke til andre kendte (eller ukendte) tekstkategorier, når brugeren følger titel-strengen til den pågældende del-titel (bestående af kategori-tag plus titel-streng (hvorved vi kan sige at "samlede titler" består af en række "del-titler" sepereret med punktummer)). ... Alternativt kunne man tillade, at flere tags bliver opgivet i rækkefølge, så man i stedet for f.eks. 'h(a)' kan skrive 'h:a:' (og så er der tale om en hypotese, men den næstede titel, 'a:<str>' kan stadig følges som normalt). Så ja, det kunne være nogle muligheder. Nå, tilbage til pointgivning til indre tekstudsnit.. Nå nej, jeg bør også lige kommentere på introducerede begreber.. Ah, jamen dette kan man jo også ordne med et "tekst-prædikat," som siger f.eks. siger hvad 'it' og/eller 'he' osv. referere til. Man kunne evt. bare bruge stedordene som tags'ne her, så at "she:Tina Dickow" kunne være et del-titel. Ikke at man skal bruge stedordsdeklærationerne til dette, men del-titlerne kan så også bare blive måden, hvorpå folk kan tilføje søgnings-tags til teksten (for her er der jo netop også tale om prædikater til teksten). Og det bør i øvrigt give sig selv, at folk også skal kunne foreslå rettelser til titlerne, hvorved den oprindelige også bør få en reference til den nye titel i databasen, således at folk kan navigere hen til de rettede titler fra den originale. Især fordi ordnen som regel ikke vil betyde noget bør del-titlerne gerne splttes op i databasen, så brugerne f.eks. også kan søge på andet end bare hovedtitlen, eksempelvis på nogen af hypoteserne også eller på nogen af søgnings-tags'ne. Og angående indre point... Skal man bare hive tekst-sektionerne ud og give dem titler.. Det synes jeg nu, nok bliver alt for meget besvær.. ..Ah, men man kunne måske bare undlade at give den en titel, så den bare får en placeholder titel, og så kommer database-entiteten bare til at være defineret ud fra referencen til forælderteksten samt intervallet, hvor udsnittet står i teksten.. ..Eller skulle man bare lade point kunne refere til et interval i stedet? Eller bagge dele?.. Hm, jeg tror faktisk, det er meget fornuftigt at sige, at hvis et udsnit er vigtigt nok til at gives en vurdering, så bør man også lige hive det ud (dog uden at kopiere dataen), så det også bliver sin egen database-indgang, og så skal der altså bare være mulighed for at lade titlen på indgangen stå åben. Jep.
%Angående det, jeg skrev om i går omkring, om man kunne finde en måde at dele anonyme nøgler ud, hvor det kan ses, at de kommer fra en vis gruppe men ikke fra hvilken person i gruppen, så er der en ret nem måde at gøre det på, der ikke involvere nogen speciel krypteringsteknologi. Hvis man bare antager, at folk har VPNs tilgængelige til dem, hvilket de bør have alligevel, hvis de skal kunne bruge resultatet til noget, så kan man bare lave en protokol, hvor fællesskabet (muligvis i form af en central enhed) bliver eninge om en række offentlige nøgler til en central enhed (som enheden så kommunikere til gruppen i hemmelighed), som kun bliver delt med den relevante gruppe for protokolen, hvorved folk så kan vælge tilfældigt og kryptere en ny-lavet offentlig nøgle til dem selv med en af de nævnte offentlige nøgler. Den centrale enhed godkender så hver herved tilsendte (som så kan tilsendes via en VPN) én ad gang efter et simpelt først-til-mølle-princip og informere om, hver gang en nøgle bliver taget af en anden offentlig nøgle. Hvis ingen i gruppen hapser mere end én nøgle, så vil alle herved få en offentlig nøgle leveret til den centrale enhed i hemmelighed. Nå ja, og der skal altså være én offentlig nøgle i udbud pr. gruppemedlem i protokollen (foruden den centrale enhed). Hvis brugerne så ikke kan afslutte protokollen med at sige at de hver især er tilfredse og har fået én nøgle, så man man be alle involverede om at opgive deres private nøgler lavet til sammenhængen (som jo så bare \emph{skal} laves fra ny som en del af protokollen, så de dermed sagtens kan afsløres, hvis protokollen fejler). Herved kan man så se, hvem der har fået to nøgler, og hvem der.. nå nej, dette går jo ikke helt, for herved kan det jo både ske at ærlige folk ikke får sig en nøgle, men en trold kan også bare sige, at han ikke fik en nøgle, og herved er min idé endnu ikke skalerbar, for hvis det skal fungere for tusinde mennesker på en gang, så går det ikke, hvis trolls ikke kan luges effektivt fra (og muligvis gives karantæne af fællesskabet / af den centrale enhed). Hm.. Hm, men behøver der at være én nøgle pr. person? Kan man ikke bare lægge op til, at man godt må hapse flere nøgler, hvis bare alle, eller bedre, de fleste, kan nå at få? Og den centrale enhed har jo ingen grund til at prøve at give nogen brugere flere nøgler end én, og i øvrigt har brugere generelt ikke vildt behov for flere nøgler i processen (og hvis de har, kan den centrale enhed jo bare servicere gruppen med et godt antal nøgler), så kommer vel til at gå meget godt.. Og man kunne jo så spørge, kunne centralen ikke bare løbende udgive en masse nøgler, og så kan fællesskabet bare løbende haspe nøgler, også selvom de ikke skal bruge det, men bare for at... Tjo, men så må det bare ikke være den samme, der hapser alle nøglerne, for så kan denne jo afsløre nytilkomne brugeres offentlige nøgler (eller kan i hvert fald afsløre hvilke nøgler, der blev taget hurtigt..). Hm, men hvis folk bare indimellem tager nøglerne hurtigt..? Ah, men centralen er jo interesseret i.. at brugerne kun får én nøgle hver, eller i hvert fald får et ligeligt antal, for så kan serveren nemlig begrænse servicerne pr. person.. Hm... Hvad med at man bare i gruppen på anonym vis opbygger en pool af offentlige nøgler.. Hm, og så kunne man spørge i en runde af gangen, hvem der har nølger i poolen.. (hvor at det så kan godkendes, hvis mange nok er med, og hvor man så kan be folk om at trække overskudende nøgler ud).. (Kom også i tanke om, at en bruger jo så skal stole på, at alle de andre brugere i gruppen ikke er én person eller arbejder sammen..)... Hm, men hvad er der i vejen med, at man bare signere uploads'ne med en nøgle, som centralen kender, og som er uddelt hemmeligt fra den centrale enhed til en gruppe, og så en ny offentlig nøgle, og så.. nå nej, for så kan én bruger stadig sætte sig på mange nøgler.. Hm, oh well.. Om ikke andet kan man jo bare benytte servere, som kan stoles på til at køre programmer, hvor den følsomme data bliver slettet med det samme, hvad man jo alligevel hele tiden gør med gængse servere.. ..Ja, det må også være fint. (Men jeg kan jo altid lige tænke videre over dette, hvis jeg får lyst.)
%(10.07.21) (stadig) Okay, hvad vil jeg så gøre nu? Vil jeg bare skrive disse noter færdig, og inkludere det af denne idé, som jeg har nu, eller vil jeg gå videre til at designe nogle detaljer til en prototype af systemet og arbejde på alt det? Jeg mangler jo stadig at skrive om mine nye tanker omkring dynamiske model-afstemninger (inklusiv nogle idéer til, hvad man kunne bruge det til, og særligt hvordan man eventuelt kunne komme i gang med det..).. ...Jo, lad mig bare lige skrive om de idéer som det første. Og så kan jeg bare lave en ny sektion, så denne sektion kommer til kun at handle om denne sem-web-idé, sådan at jeg bare kan fylde på her, når jeg kommer frem til flere ting, og så kan den nye, følgende sektion bare handle specifikt om de dynamiske afstemninger. Ok. (10.07.21)
%(11.07.21) Lad mig bare skrive alle mine seneste tanker ind nu, og så kan jeg altid finde et sted at putte de noter hen, som jeg kommer til at generere under en mere detaljeret udvikling af idéen. I går aftes i sengen kom jeg i øvrigt frem til, at man måske skulle gemme hver deltitel med en nøgle, som nok skal dannes med et hash (sikkert SHA-2), så databaserne kan være mere uafhængige, og hver hele titel kan så gemmes med en nøgle fra hashet mellem den sidste del-titels nøgle lagt sammen med nøglen fra den samlede titel minus pågældende del-titel. Dette vil så sige, at enhver titel med n deltitler i sig skal gemmes i form af både de n seperate deltitler plus n samlede titler, hvor man lægger én deltitel til ad gangen. Selvom dette umiddelbart kan virke som en masse plads at bruge, så kommer nok nok til at give rigtig god mening i en relationel database. For deltitler vil jo typisk blive genbrugt alligevel og tekster vil nok gives mange kombination af titler, så i sidste ende kommer den ekstra plads, vi skal bruge på dette nok mest bare være i form af de få titler, vi skal gemme, som ikke har nogen tekst linket til sig, hvilket jo også er ok, for det letter bare søgningen (fordi de så til gengæld vil have links til relaterede titler i databasen). Og man opnår så altså ved alt dette, at brugere kan søge på alle deltitler for sig selv, og finde relaterede samlede titler (med tekster tilknyttet), og man kan også hurtigt søge på, hvilke andre titler relaterer sig til en samlet titel --- både i form af længere titler og kortere titler. ..Tags kan vel bare gemmes som en del af deltitlerne..? Nå nej, for man vil jo gerne kunne søge på tværs af tags, men med samme tekster.. ..Hm, hvad med at tagget, eller en byte, half-int eller int, der repræsenterer tagget, klistres på hashet..? For så kan man alligevel hurtigt søge på tværs af tags, ja, det var da en meget god idé.. Ja, sgu. ..Jo. Og så skal man altså bare vælge hvilket tag, man vil prøve at søge på, hvis man vil finde frem til andre deltitler, der bruger samme sætning, men har et andet tag. ..Hm, og hvad med det med 'h(a)'/'h:a:' osv.? Kan man ikke bare fortælle sit program, hvilket tags hænger sammen, så databasen kan give alle relevante forslag, hvis man klikker på en deltitel? Jo, det er nok meget godt på den måde. ..Hm, men det går da forresten ikke, hvis tags gives tal-repræsentanter, hvis systemet skal være åbent for alle mulige brugerdefinerede tags?.. Hm, skulle man så bare sige, fint, men så får I bare fire chars at bruge til alverdens tags..? Og så må brugerne bare "slås" om deres semantik?.. Hm, måske faktisk ikke helt dumt at gøre det sådan.. ..Ja, det tror jeg faktisk bliver planen umiddelbart.. Hm, alternativt kunne man have to tags pr. deltitel; én der beskriver den overordnede type, hvor man så typisk vil søge på denne, og én der beskriver den specifikke semantiske funktion i pågældende (samlede) titel, hvilken man så ofte med fordel kan vælge se bort fra, når man søger.. Hm, hvad er bedst?.. Det sidstnævnte her er nok nemmere alt i alt, men koster jo lige lidt mere plads.. Tja, men det må næsten være det værd. Okay, så planen er to tags pr. deltitel. ..Vent, men kan de type-specificerende tags så ikke bare gemmes som en del af selve titelsætningerne? Hm, jo, medmindre man også gerne alligevel vil kunne søge på tværs af disse tags.. Vil man gerne det..? ..Og i så fald, kan man så ikke bare implementere dette i et lag over selve databasen, hvor dette overliggende lag så bare kender de type-tags, som er populære hos brugerne, og dermed kan foreslå forskellige muligheder i søgningen? Jo, det tror jeg, bliver svaret.. Ja, for man vil alligevel skulle implementere et tilsvarende lag, efterhånden som titlerne bliver mere og mere formelle. For her skal brugerne jo alligevel netop kunne trække termer ud af titlerne og søge på dem i andre sammenhænge osv. Så jo, det bliver i et overliggende lag, og databasen i det grundlæggende lag bør derfor nok bare have enkle (int-)tags, som specificere den semantiske funktion af deltitlen i den pågældende samlede titel. 
%Okay, så der skal altså være en relation over deltitler og tilhørende samlede titler, som de afslutter (også hvis de kun står som indre deltitel i praksis, og at der dermed ikke er knyttet nogen tekst til pågældende samlede titel). I denne relation kan man så både finde frem til alle samlede titler, der bruger en vis deltitel, og man kan finde frem til alle udvidede titler til en vis samlet titel. Og hvis man fra en samlet titel vil navigere til kortere versioner af samme titel (altså med deltitler taget fra i enden), så kan den relevante nøgle (til tilen med én deltitel trukket fra) findes i en kolonne i de samledes titlers primære relationen. Denne relation skal altså både have en (nullable) nøgle til en kortere version af titlen, en (ikke-nullable) nøgle til titlen selv, og selvfølgelig også en (nullable) nøgle til en tekst, som er uploadet med pågældende titel. Og hvad mangler jeg så mere?.. Nå nej, forresten, den sidstnævnte relation skal selvfølgelig splittes op, så en seperat relation linker titler til deres (nullable; muligvis ikke-eksisterende) kortere titler, og en seperat relation linker titler med deres (nullable; muligvis ikke-eksisterende) tekster. ..Og hvor skal tags'ne være? ..Vel egentligt ikke i deltitel-samlet-titel-relationen?.. Indskudt: ..Hm skulle man foresten.. Hov, indskudt indskudt: for den første deltitel må der jo gerne være et tag, der angiver tekst-typen..! Hm, men så kunne man jo bare bruge tagget til dette her, og så bare bruge en konvention om, at første deltitels semantiske funktion altid bare er at give den første sætning til at beskrive, hvad tekten handler om, og hvor efterfølgende delttiler så kan præsicere dette (og kan bruge deres tags til at vise semantisk funktion af deltitel-teskten, og altså ikke kærre sig om nogen teksttype her). Og det første, jeg var ved at indskyde, var så bare, at man måske skulle bruge en anden værdi end null, så man bedre kan søge specifikt på start-titlerne (jeg husker det nemlig som om, man ikke kan sige noget a la 'T = null' som en condition i (en query til) relationelle databaser (men jeg kan nu tage fejl..)).. Okay, og tags'ne kan bare få en kolonne i samlet-titel-kortere-titel-relationen, og skal så også indgå, når hash-nøglen til den pågældende samlede titel skal dannes. ..Hov, jeg mangler forresten at behandle lange titler. Her tænker jeg faktisk (hvilket jeg også kom frem til i går aftes i sengen), at hver deltitel bare gemmes som den første del af titel-teksten og så en reference til den afsluttende titeltekst samt en nøgle, som dannes ud fra et hash af de to førstnævnte attributter. Så jeg bør altså udvide deltitel-samlet-titel-relationen med dette, hvilket altså vil sige, at den kommer til at indeholde disse tre attributter samt referencen til den samlede.. ah, nej, så skal der jo også være tale om to seperate relationen; den jeg lige har beskrevet med de tre attributter, der definerer deltitel-nøglen, og så den tidligere nævnte relation mellem deltitel-nøgler og fuld-titel-nøgler. Og så skal der så bare også være en relation over titel-afslutninger, som eventuelt kan implementeres ligesom fuld-titel-kortere-titel-relationen, bare uden tags og uden tekster, men man kunne også implementere den på en anden måde. ..Tjo, men jeg ville nok bare implementere den på den nævnte måde.. ..Tjo, på den anden side kan det godt være, at man hellere vil prøve at reducere alle de "hop," som databasen så skal tage mellem vidt forskellige tags.. Ja, så det kan man jo lige tænke over. Okay, men ellers: så langt så godt..*(!) Så skal teksterne måske gemmes, så de muligvis kan være "rettelser," og dermed altså referere til en tidligere (eller senere, hvis man transformerer om som nævnt) tekst.. ...Uh, men hvis man er bekymret over store hop på harddiskene, så kan man jo altid bare lige tilføje en ekstra relation med alle nøglerne (..eller eventuelt også bare en relevant undermængde af alle nøglerne), og så med lokale nøgler, hvorved man så kan udskifte alle (eller de relevante) nøgler med lokale nøgler. Når data så kommer fra internettet, skal man så bare lige igennem denne relation først, men ellers kan man bare bruge lokale nøgler. Og disse lokale nøgler kan så bare ordnes på en måde, så beslægtene entities typisk vil lægge tæt på hiannden i relationerne, og dermed (så vidt jeg jeg, og i henhold til hypotesen, der skaber hele denne problematik) typisk komme til at ligge tættere på hianden i databasens (fysiske) lager. Cool. Og angående teksterne, så kan disse også gemmes lidt på samme måde, hvor en tekst-entity har en tekst-begyndelse, en reference til den afsluttende tekst(-entity) samt en nøgle, som enten vil være et hash af begge lagt sammen eller muligvis en lokal stedfortræder-nøgle (til førstnævnte, globale nøgle). Og når det så kommer til rettelser, så kan vi også nok nøjes med en enkelt "rette-tekst" med en fast maks-størrelse, og så en reference til tidligere tekst-entitet, for hvis man vil lave længere rettelser, så kan man jo altid bare gemme dem som to rettelser oven i hinanden (så nummer to rettelse altså så bare referere tilbage til nummer ét rettelse i stedet) --- ja, eller endnu flere rettelser kædet sammen for den sags skyld. Cool-cool.. 
%Jeg kan også lige nævne, at jeg i morges tidligt kom på en mulig syntaks for mark-up'en: Jeg har tænkt lidt på "sem{}," men nu hælder jeg mere til "#{}," eller rettere "#{[_]_}," og endnu mere specifikt: "#{[<title>]<inner text>}." Så tanken er altså at man f.eks. på twitter kunne skrive "#{[argw:why lasers are dangerous for the eye. txtq:short] There is a high risk of frying the retina.}," og så kan brugeren altså erstatte dette med andre forklaringer og se på, hvilke point hver enkelt forklaring har fået, særligt inklusiv den pågældende forklaring, men måske endnu vigtigere også særligt for den mest populære forklaring. Her har jeg forresten altså ladet 'txtq' stå for 'text quality.'
%Hm, og tags'ne er ikke lige lovligt begrænsede med denne tilgang?.. ..Hm, man må faktisk næsten kunne gøre noget bedre, hvor deltitel-teksterne selv beskriver deres funktion via tags i et overliggende lag, men hvor man så bare nemt kan søge på indre dele af deltitlen, f.eks. på nestede deltitler, sådan som jeg også tænkte det, da jeg skrev om 'ahyp:argw:...'/'h:a:...'.. ..Og pointen var her, at jeg mente, at man burde kunne søge på titel-teksterne direkte i databasen, og altså ikke bare på deres nøgler.. Hm... Det vil jeg lige tænke lidt over, for det skal heller ikke eksplodere helt med pladsen.. Og man kan ikke bare lave.. "aggregat"-queries, hvor man trækker nogle chars fra titel-strengene.. Virker lidt uldent, men jeg tænker lige over tingene...  
%(12.07.21) Okay, man kan (vist) gøre det noget smartere, end jeg tænkte her i går. Det må man kunne. Jeg tror bare, man skal gemme alle titler tilhørende en tekst (eller flere) som én titel, og altså ikke gemme alle kortere versioner samtidigt med. Man kunne muligvis bare bruge det, jeg tænkte for deltitlerne, hvor man giver en reference til titel-afslutningen i relationen. Og så tænker jeg nemlig i stedet, at deltitler bare kan tildeles links efterfølgende (manuelt) (hvis man ikke har gjort det til at starte med). Hm, og dette kan man vel egentligt bare gøre ved at have en relation beregnet til at linke fulde titler med deltitler. ..Og hvor man altså så bare skal sørge for, at man altid kan navigere direkte til titler med deltitlen som start-titel.. Ja, hvilket vel kan gøres, hvis hver fulde titel også bare har en relation mellem dem og deres start-titler. I øvrigt bør brugere også kunne uploade links mellem tekster, som ikke behøver hverken at have med links i teksten at gøre eller med deltitler, og som så skal kunne gives point på samme måde som tekst-uploads'ne (altså tekst-entiteterne bestående både af en tekst og en tilhørende titel --- og hvis man vil give gode point til teksten men ikke til titlen, eller omvendt, så må man nok bare designe pointene, så man kan se hvad de handler om, hvad man jo alligevel skal gøre).. Angående markup-syntaksen så hælder jeg nu til, at man ikke behøver "{}," men at man bare opsætter regler for (som kan tilpasses og varieres løbende i starten), hvordan tekst parses efter et startpunktet (hvor man så meget vel kunne vælge at benytte "{}," hvis tekstudsnittet f.eks. er en del af en paragraf og ikke bare hele (eller resten af) paragrafen), som altså kunne starte med et hashtag ('#'). Syntaksen kunne så måske være "#[<text>]," hvis... Tja, eller måske bare udelukkende (hvad jeg også ville foreslå uanset hvad) "#<nøgle>," hvor jeg så dropper, det med at titler og tekster uploades af læserne, men hvor de skal uploades direkte til databasen altid. Og det gode ved at bruge meget deterministiske nøgler er så, at man som bruger ikke behøver at vente på at databasen godkender uploadet og giver en nøgle tilbage, som man kan bruge; man kan bare udregne nøglen selv og bruge den med det samme. Ja, det lyder altså ret fornuftigt..(!).. ..Og der gør i øvrigt ikke noget, at syntaksen kollidere med twitter og facebooks syntaks for hashtags, for konventionen har er aldrig at slette/erstatte hashtagget, men bare at give det et links, så hvis browser-udvidelsen heller ikke sletter det, men bare tilføjer yderligere links til det.. Hm, hvis teksten alligevel skal indsættes, så kan man jo tilføje linket der, men man bør dog også bare kunne give en reference (hvis nu mijøet på siden ikke rigtigt tillader at man indsætter en hel masse; hvis det ødelægger (css-)strukturen).. Men ja, det må også kunne lade sig gøre at give flere links til hashtagget uanset hvad; man må f.eks. kunne gøre det, så en drop-ud-menu kommer ud, når man trykker på det, hvor man får de forskellige links, bl.a. det, som selve hjemmesiden havde oprindeligt, hvis der er sådan et. ..Og lad mig lige nævne, angående links mellem fulde titler og deltitler, at jeg her tænker, at databasen bare kun skal acceptere gyldige forslag, hvor forslaget matcher en deltitel helt. Ja, og her kan man så bare definere parsingen, så man kan se bort fra eventuelle tags foren deltitel-teksten.
%Okay, så en relation over titel-begyndelse, titel-afslutning, og et hash af den fulde titel (i form af ren tekst, så altså ingen nestede hashes her), som så bliver titel-entitetens nøgle. En relation med titel-afslutninger, som evt. kan bestå af en tekst-begyndelse, en nøgle til den videre afslutning (nullable, for vi må kunne regne med at ingen tekst-afslutningers hash kolliderer med null), samt et hash af de to ting sat sammen, som så bliver denne entitets nøgle. En relation over titler og tilhørende tekster (hvor en titel altså godt have flere tekster og omvendt) (og "nøglen" bliver her bare de to kolonner tilsammen, så altså to nøgler). Lad mig foresten kalde disse entiteter, altså titel plus tekst for 'artikler,' hvilket så altså ligesom bliver den mest centrale entitet i databasen. Så skal vi altså også have en fuld-titel-del-titel-relation og en "eksterne links"-relation over to artikler, den artikel linket skal være fra, og den linket skal være til. I øvrigt skal der også dato-kolonner, hvorend det kan være relevant, og også kolonner til brugersignaturer.. Nej, det kan forresten bare komme i en seperat relation, hvor der så skal være dato her, så man kan se, hvem der kom først.. Hm, måske skal denne signatur-dato-artikel-relation --- og en tilsvarende signatur-dato-ekstern-link-relation --- simpelthen bare være der alle datoer (og signaturer) hører hjemme.. Lad mig sige det for nu. Uh, og bemærk at "deltitel" nu bare gemmes som en almindelig titel (som jo findes i sin egen relation, så der behøver ikke at være nogen tekst tilknyttet den), og så er det bare kolonnerne i deltitel-fuld-titel-relationen, der afgør, hvad er hvad. Tekst-relationen kan bare være, som jeg tænkte den i går (altså som en mulighed). Så mangler jeg vel nærmest bare pointene? Nå jo, der kan muligvis også være en relation over globale nøgler (hashes) og lokale. Point skal også både kunne gives til artikler og til eksterne links. Og hvis man f.eks. så vil give point til en pointtype, så må man bare give det til en artikel, der beskriver pointet i stedet. Herved kan man altså bare danne et point, der beskriver "hvor nyttigt er pointet, som denne artikel beskriver" og så kan man bare danne et afhængigt point, der faktorisere "hvor korrekt er denne artikel," "hvor populær er denne artikel" og så altså "hvor nyttigt er pointet, som denne artikel beskriver" sammen til ét point, som så kan bruges til at give overblik over nyttige og populære point. (Man kan i øvrigt sikkert også finde på bedre point-sammenblandinger til at implementere dette, men det var bare for at give et forslag.) Hm, mon ikke altid point-entiteter dog skal have en signatur?.. For man regner vel ikke med at skulle genbruge point-værdier..? Tjo, medmindre der gives diskrete point-værdier i et lille interval.. Hm.. Nå, men dette er vel uanset hvad at betragte som nede i implementationslaget, og skal være frit til at kunne gøres på forskellige måder. Så lad mig bare gøre det så simpelt som muligt her.. Hm, men først og fremmest skal der vel være en relation med pointtype-entiteter.. Pointtyperne skal så have et navn og.. og hvad mere? Skal de have en forklarende tekst tilknyttet sig (nok ikke, for denne må gerne kunne skiftes ud og/eller rettes, men hvad så?)? ..Ah, kunne man ikke bare give pointene en titel? Denne titel kan jo så i sig selv være ret sigende for pointen, men så vil man i øvrigt også kunne tilknytte tekster til samme titel, hvor det så bør være en underforstået konvention at hvis pointtitler er meningsfulde, skal tekster der tilknyttes til dem forventes at forklare pointets betydning (hvilket også vil være det naturlige alligevel, så længe man bare sørger for at titelteksten selv signalerer, at der er tale om en pointtitel, og ikke noget andet). Men sikkert stadig en ok idé med et pointnavn også, som så tilgengæld ikke behøver at udgøre nøglen til pointen, og så forskellige points navne altså godt i princippet må kollidere, selvom man dog (måske) bør bruge at holde navnene adskilte for populære point.. Okay, så en relation over pointtitel, pointnavn, og..? Interval? Nej, det er bedre hvis intervallet altid er det samme (signed int sandsynligvis), men at fortolkningen bare kan variere. Nå ja, vi skal også have mulighed for moderator-liste, medmindre dette bare skal ske i en separat relation, eller at man adskiller de to point-kategorier (i modererede og ikke-modererede point).. Hm, jeg kan dog godt lide tanken om, at modererede point bygges af et (ikke-modereret) standard-point og så en moderatorliste, så man let kan udskifte moderatorlisten og/eller have mange forskellige, som forskellige brugere kan bruge.. Ja, det lyder meget fornuftigt. ..Og jeg mener ikke, databasen behøver at indeholde en liste over kendte brugere; en reference til en "bruger" kan altid bare referere til en offentlig (krypterings)nøgle og ikke andet.. ..Okay, så giver point-relationerne så nu lidt sig selv, eller mangler jeg at tænke over mere?... ..Jo, man skal dor det første stadig kunne give point til brugere, så der skal også være en relation til dette. I øvrigt føler jeg lige, jeg bør nævne, at de foreslåede links som browser-udvidelsen i sidste ende giver, ikke kun behøver at være deltitel-links og eksterne links. Man er også fri til at gøre det, så at programmet selv kan foreslå links baseret på titlen og/eller på baggrund af de eksterne links. Eksempelvis kunne man give et enkelt eksternt link til et emne, hvor dette så kan medføre, at man også for vist andre populære (eller hvilken point-parameter, man nu går efter) links inden for dette emne, men altså uden at disse links skal uploades manuelt. Og så tænkte jeg i øvrigt også på, at jeg også skal huske, at overveje de indre sektioner i artikler også skal kunne indsættes i form af artikel-referencer. Dette gøres så bare også med samme "#{<nøgle}"-syntaks.. nå ja, men så skal man jo lige sørge for, at denne syntaks udvides, så den kan vise... (om den mark-up'er den efterfølgende tekst eller er en "indsæt her"-reference), men skal man så ikke kun bare have "indsæt tekst her"- og "indsæt link her"-referencer, og ikke nogen.. "følgende text har denne titel"-referencer.. nej, sidstnævnte skal netop ikke være med. Så jo, kun de to første ting, medmindre altså man også skal dele "indsæt tekst her" op i andre kategorier så som "indsæt billede her osv..? ..På den anden side må tekster jo også gerne indeholde billeder osv., så i princippet er al "tekst" bare hypertekst, og så er det underforstået, at dette også kan inkludere billeder osv. ..Og selvom de i brugeren ende i sidste ende bliver til html, så behøver syntaksen og implementationen dog slet ikke have noget med hverken html eller http at gøre. Databasen kan bare have sine egne relationer med medie-entities, hvor man også kunne lave en standard protokol for at finde nøglen, hvilket nok bare bør være: tag et hash af hele det binære objekt. Ja, så vi behøver kun "indsæt (hyper)tekst" og "indsæt link." Syntaksen kunne være noget så simpelt som at sætte et udråbstegn eller et '@' foran eller bagved (hælder selv til foran, tror jeg), hvis man gerne vil præcisere at det henholdsvis er meningen, at teksten skal foldes ud i artiklen, eller hvis det ikke er meningen. Man kunne også bruge '&' i stedet for '@,' for den er lidt nemmere på tastaturet, og den passer også med C-syntaksen, så det er ikke bare taget ud af det blå. Hm, vis vi skal følge C, så kunne man dog også bruge '*' i stedet for '!'.. Ja, det giver egentligt god mening. Fint. ..Hm, på nær at mange forbinder '*' med "pointer" (og ikke programmører forbinder det måske også mest med en reference i form af en fodnote-reference)... Tja, på den anden side, hvis man skal være lidt utilitaristisk, så ville dette jo bare så komme til at udgøre en god huske regel, når folk skal lære C, efter de har lært syntaksen til dette system.. Hm.. Ja, jeg vil nok forslå henholdvis '*' og '&' som valgfrie tegn foran '#<nøgle'-syntaksen til at præcisere, om teksten helst skal indsættes eller helst ikke.
%Angående moderatorlister så kom jeg til at tænke på, at moderatorerne kunne få lov i databasen til at tilføje undermoderatorer samt et start- og et slut-timestamp pågældende pointtype, hvor tolkning så er, at moderatoren har set at pågældende bruger ("undermoderatoren" --- eller vi kunne også bare kalde det en whitelist over godkendte brugere) har givet gyldige pointangivelser for pointtypen i pågældende tidsinteval, og hvor godkendelsen til whitelisten derudover betyder to ting, nemlig at moderatoren nu forventer, at folk indimellem selv hjælper med at eftertjekke pågældende whitelistede brugers pointbidrag, og altså rapportere til moderatoren, hvis pågældende uploader ugyldige point (altså ud fra en formel fortolkning af pointtypen, som er beskrevet i pointtitlen samt tilhørende tekst --- hvilken i øvrigt selv bør være godkendt af moderatorerne), og, som nummer to ting, at moderatoren så vil sørge for at reagere på sådanne rapporter, og fjerne pågældende fra whitelist'en igen, hvis rapporterne holder. På den måde kommer moderatorerne til at kunne trække på brugerfælleskabet, når det kommer til at tjekke gyldigheden af pointene. Hermed skal man så kun mest tænke på moderatorernes oprigtighed, når man vælger dem, og ikke nødvendigvis så meget på, om de har ressourcerne til at gennemtjekke og/eller give alle de relevante point selv. Hermed behøver man heller ikke at skifte pointtypen ud med en ny, hvis en tredjepartsinstans, der udbyder den service at udregne pointene, falder fra, for så kan moderatorerne bare ændre whitelisten en anelse. ..Hm, men hvad så når moderatorer falder fra, kunne man så spørge, og det var jo så her, jeg tænkte mig, at moderatorer-listen også skal kunne udskiftes for en pointtype, men hvordan?.. Hm, ved at gøre, så man kan sætte start- og slutdatoer på moderatorlisten, og så bare lave en ny pointtype baseret på den gamle, men hvor den tidligere moderatorliste så kun indgår med en allerede-udløbet slutdato, så man herved stadig beholder alle de gamle bidrag, men nu bare kan fortsætte med en ny moderatorliste? ..Hm, og hvordan.. Ah, og så skal man bare sørge for, at man indimellem (hvilket gerne må ske automatisk) får tjekket pointtypens titel-entitet for at se, om der skulle være kommet en mere populær version af pointen, hvorved brugeren så bør få besked, så denne kan vælge, om vedkomne vil gå over til den nye version. ..Ja, det lyder umiddelbart nice nok. 
%Hm, men burde man ikke kunne give tekst-prædikat-efternavne til sine referencer, uden at de behøves at gemmes som en seperat titel..? ..Hm, eller kunne man måske i stedet give point-efternavne (.."point" bliver jo en slags prædikater).. Nå ja, og jeg kom forresten også til at tænke på tidligere, om point mon skal kunne have input-parametre og alt sådan noget, så her har vi lige endnu et åbent spørgsmål.. Men ja, skulle man mon bruge point som tekst-prædikater..? ..Og hvis point så kan have input parametre, så kan man f.eks. opnå at sige sådan noget som: "teksten (eller billedet) må ikke være større end x gange y målt i tekststørrelse-enheder.".. Ja, det virker da umiddelbart ret fornuftigt, men lad mig lige tænke lidt over det.. ..Jo! Point kan jo mere, end hvad mine tekstprædikat-deltitler fra før kan, så hvorfor ikke droppe alle deltitler, der ikke har med semantikken af indholdet at gøre, og så lade point klare alt dette i stedet? Og min tanke er nemlig så, at folk skal kunne tilføje point-præferencer bag på '#<hashnøgle>'-syntaksen, så man kan signalere til læseren, hvis der er nogle ekstra hensigter med tekstens form osv., udover selve sementikken, som nemlig bare bør beskrives af titlen (i hvert fald hovedsageligt --- men typisk også udelukkende, så at pointene altså kun beskriver formen og populariteten og ting i den retning). ..Nicehed. 
%Og ja, så må man jo nok hellere kunne have point med input-parametre og/eller med flere output-værdier.. Hm...
%Ah, ja, man skal da helt klart kunne have input-parametre til pointene, og det gode er så, at hvis man bare har mindst én point-kategori, hvor værdi intervallet er stort nok til at kunne rumme databsens hash-nøgler, jamen så vil man i praksis kunne give pointene hvad som helst; alt hvad databasen kan rumme, i hvert fald hvis man så også lige har lov og mulighed for at uploade det. Nice. Og ja, man bør så også helt klart have pointkategorier med output-værdiintervaller store nok til at rumme hashnøgler, for så kan man tilsvarende også give hvad som helst som point-værdi i praksis. Og dette gør så bl.a., at man også (som del-brugerfællesskab) kan implementere sit eget system til f.eks. at foreslå relevante eksterne referencer til artikler osv. Fedt. Cool. (12.07.21)
%(13.07.21) Jeg bør også lige skrive om, hvordan jeg forestiller mig, at sprog-ontologien, og andre ontologier for den sags skyld, kan komme i spil, og så skal jeg også lige genoverveje det med whitelist'ene og alt det, nu hvor man bare kan opdatere pointtyperne.. ..Hm, nu har jeg også lige tænkt over, at systemet jo også ret hurtigt vil kunne bruges til at finde frem til, hvor på nettet medieindhold så som billeder og videoer stammer fra, fordi man bare kan sørge for at tage et hash af filens indhold, og så har du dit link til databasesystemet der. Og så kan folk også lynhurtigt flagge indhold for "potentielt misvisende" med en tilhørende forklarende tekst, der påpeger, hvis billedet eller filmen har været fremstillet i en forkert kontekt, hvor på nettet den har det, samt hvad den rigtige kontekst menes at være. Og alt dette kræver ikke vildt meget arbejde og kan gøres af ret få personer, hvis man tænker over det.. Det kræver bare, når man ser et tvivlsomt opslag: find hashet (mon ikke man i øvrigt kan få JavaScript til at gøre dette..?), lav et upload til databasen under hash-nøglen, hvor du uploader URL'en (og/eller anden identificerende information; bare så andre brugere kan finde frem til samme og verificere det) og muligvis også den kontekst-tekt, der var givet, hvis den ikke er så stor, og derudover er det bare lige, hvis man kender den egentlige kontekt, at forklare denne med nogle få ord (men hvis ikke man selv kender denne, er det også fint nok). Herved bliver det nemlig nemt at se, hvilket opslag kom først.. nå ja, og man bør derfor helst også uploade timestampet for opslaget, hvis man kan finde dette, og så kan folk ret hurtigt gennemgå de forskellige opslag, flagge hvis der er modstride i konteksterne, og stemme på hvilken én, de tror er mest rigtig. Og idet det ikke er vildt svært i brugerfælleskabet at holde øje med, hvilke kontoer, der har det med at stemme lødigt på sådanne ting (og fordi brugerne hver især selv bestemmer over, hvilke pointtyper med hvilke moderatorlister, de vil bruge), så vil man ret let kunne ignorere spam i disse afstemninger (ved at bruge pointyper, der lytter mest til de brugerkontoer (dvs. de offentlige kryperingsnøgler), der ser ud til at have en god og lødig historik). ..Og dette tror jeg altså, kunne være et ret godt og simpelt salgspunkt omkring systemet. Og det ville så heller ikke være et vildt skridt at gå over til udsagn, der circulere.. Her ville det jo så være bedst, hvis man enten får en sprog-ontologi op at køre, eller hvis man bare på forhånd udvikler et formelt sprog (at udvikle en sprog-ontologi svarer nemlig til at udvikle et formelt sprog i/via selve databasen)... Hm, men ellers er der jo også de eksterne links, hvilket i øvrigt også bliver brugbart for medieindholdet, fordi folk jo altid kan ændre lidt i videoer (e.g. beskæring, spillelængde eller opløsning), og gør det også tit af flere grunde, så derfor kræver førstnævnte system også en god, hvis ikke ligefrem sprog-ontologi, så i det mindte emne-ontologi, så man kan tildele videoer alle mulige relevante tags, hvorved folk så nemmere kan finde frem til, hvis der er gengangere i databasen. Og bemærk, at hvis browserudvidelsen bare selv kan finde hashet (og måske ved bare først at poste indholdet til localhost-programmet, hvis dette bliver hurtigere), så behøver selve mediefilerne altså ikke at uploades til databasen (hvilket jo er ret betydeligt). ..Hm, forresten, nu hvor links kan gives som ouput til pointtyper, vil det så ikke være redundant med de eksterne links i databasen..? ..Eller vil det stadig spare lidt plads..?.. ..Tja, man kan jo altid bare erklære eksterne links forældede senere, hvis der ikke bliver behov for dem længere.. ..Hm, dette får mig til at tænke på, at det måske er en ret god idé, hvis visse pointtype-kategorier også kan have to (og måske tre; hvorfor ikke?) outputværdier. For det koster bare databasen nogle ekstra relationer, som brugerne altid kan se bort fra, hvis de ikke bliver nødvendige, og ellers koster det bare lige, at brugerne må udforme deres "point scripts languages" en anelse mere omhyggeligt, når de skal konstruere de formelle sprog til at kunne bygge scripts til de automatiske point (som brugerne jo så selv står for at håndtere ved brug af moderatorlisterne). ...Hm, og angående spog- og emne-ontologi, hvor 'emne-ontologi' altså bare ligesom er en simplere version af en sprog-onologi (forskellen på at kunne opbygge emne-tag-lister eller (komplicerede) sætninger), så kan man jo netop komme rigeligt langt med bare en emne-ontologi, når man skal klassificere medieindhold eller udsagn.. ..Hm, og hvordan skal nu bruge emne-ontologien helt (eller mere) præcist..? ..Det ville jo være rart med en træ-agtig graf, men der bliver jo umiddelbart ret mange fobindelser på kryds og tværs, medmindre man på en eller anden måde får ordnet tags'ne.. hvad med bare alfabetisk (eller noget simpelt i den stil)?.. ..Hm, eller måske ordnet i popularitet... ..Ja.. Det må faktisk allerede blive ret godt sådan.. ..Hm, og angående medieindhold må man vel kunne komme ret langt ved bare at beskrive, hvad der bliver sagt, og hvad der optræder i billedet/videoen (m.m.).. ..Og det er ikke alle brugere, der hver især skal finde frem til duplikanter; så snart to er fundet, så kan man tilføje "eksterne links" til dem.. Uh, og timestampet kan også blive ret vigtigt, i hvert fald når der er tale om noget nyt indhold, som deles på internettet, hvilket jo ikke altid er tilfældet (nogen gange har folk det jo også med at lave opslag med gamle videoer i en forkert kontekst (som om de er aktuelle)), men alligevel ret vigtigt redskab for nye ting.. Hm.. Men ja, bare én bruger kan finde frem til duplikanterne (via emne-ontologien (..Og indre-(evt. talt, når det kommer til videoer)-tekst-i-indholdet-ontologi.. Og andre ontologier; man kan lade flere ontologier arbejde sammen om at klassificere indholdet, og altså ikke bare bruge emne-tags..)), så vil det efterfølgende blive nemt for resten af fællesskabet at fremstemme forslaget om det eksterne link imellem duplikanterne.. ..Hm, skulle man så indføre "tags" til databasen i form af tekster (a la titelteksterne som eksempel)..? Hm, ah, men det kan man vel netop allerede ved at uploade artikler uden tekster tilknyttet.. Ja, og da titel-tekst-relationen er en seperat relation, så vil det ikke fylde mere.. Ah, det kræver dog lige, at eksterne links så kan gives til titler og ikke kun til artikler, for ellers så vil det lige kræve et null hver gang på størrelse med en hash-nøgle for alt det tekst, der ikke er beregnet som titler, men f.eks. bare kunne beskrive, hvad der bliver sagt i en video osv.. Okay, men er det så ikke bare at sidestille tekst-entiteterne med artikel-entiteter... Hm, eller vent, bør tags forresten ikke bare indføres som "brødtekst" i tag-ontologierne..? Hm.. Indskudt: Og et solidt salgspunkt fra starten er altså også det med, at man så, når man har browset frem til et medieobjekt eller et populært tekstopslag, så vil man kunne navigere hen til databasen og se, hvad hele ens (eget!) samlede fællesskab (af de internetbrugere, man tror, deler interesser og værdier med en selv --- som man enten har bemærket selv, eller som man indirekte er kommet ind i gruppen, fordi andre brugere har lukket dem ind (hvilket aldrig behøver at være permantent, og i øvrigt kan ske med en tillidsvægtning på brugeren)) siger om pågældende indhold/emne..! Tilbage: Det virker umiddelbart fornuftigt nok at bruge brødtekst til tags, for så bliver ontologi-kanter jo bare ekterne links mellem.. Hm, men hvordan indfører man så en klasse i ontologien.. ah, for ontologi-klasser kan brødteksten jo bare være forklarende for klassen, i stedet for at være instanser af klassen. Hm, men hvordan kender man så forskel.. på point muligvis.. ..Så om en artikel fungere som en klasse-definition i en ontologi eller en instantiering af en klasse, det kommer så bare an på de point, den er blevet givet.. Tja, og man vil jo alligevel være afhængig af pointene, hvis man gjorde det på en anden måde, så hvorfor ikke bare sådan her? Hm, selvfølgelig kunne linket mellem klassedefinition og instantiering jo også bare være et eksternt link og ikke selve titlen, hvilket måske nok ville være mere oplagt.. ..Ja, det ville være bedre. Hm, eller det kan jo netop faktisk med fordel være et titel-deltitel-link, som jeg jo har haft tænkt på. Ja, ikke dårligt..
%Hm, jeg tænkte på, om det ikke ville være rart, hvis brugerne lige kunne få lidt mere frihed til at vælge deres egne point, men så kom jeg lige på, at med ML-teknik-point oveni, så må man da egentligt kunne komme rimeligt langt.. Så skal der så bare lægges op til, at brugerfællesskabet sørge for at opdele deres point i mange værdier, så brugerne selv i sidste ende kan bestemme den vægt, de selv synes om, når browser-udvidelsen har fået tilstrækkeligt data og altså lige skal sætte det sidste filter på.. Hm, og folk vil jo for det meste stadig også være interesseret i.. i hvert fald i socialt-medie-sammenhænge.. at få vist nogenlunde det samme som mange andre brugere (så ens filterbobel heller ikke bliver alt for niche).. Hm.. .. Hm, ja, man må jo i sidste ende kunne komme vildt langt med ML-point, men kunne man eventuelt også lige gøre lidt mere end dette, så man f.eks. i starten kan fortælle browseren.. Hm, men kan man ikke bare bruge ML-point over en bruger-ontologi også, hvor brugere så kan uploade (og underskrive) bruger-bruger-interesse-kanter (lidt ligesom at følge personer på gængse platforme)..? Hm, jo også kan man jo søge på point fra specifikke brugere, så.. Hm, jeg overvejede lige hurtigt, om man så måske skulle tilføje nogle aggregat-kommandoer til serverne, men.. tja, hvorfor egentligt ikke?.. Men ellers kunne man i øvrigt også bare downloade alle pointgivninger fra brugere, man følger, og så kan man selv bruge dette i det sidste filter i sin egen ende.. Ja, så det korte af det lange er vel bare, at man sagtens kan komme til at få det, så man kan bruge specifikke brugere man følger som en del af sit samlede filter --- enten ved at få sit lokale program til at spørge til disse brugeres (og måske også disse brugeres fulgte brugere.. men så er vi nok ude i, at man i stedet så hellere skulle udregne parametre, som så nogenlunde fanger denne brugermængde..) pointgivninger, eller ved simpelthen at få det til bare at downloade selvsamme pointgivninger, og så bruge dem i et lokalt slutfilter. ..Ja, der er rigeligt med muligheder --- også især fordi man godt en hel del vektorer i vektorrummet, og fordi man også sagtens bare kan variere disse vektorrum (til f.eks. at klassificere brugere) alt efter emne.. (13.07.21)
%Whitelists'ne er forresten gode nok; det bliver nok gode at have. En god ting i øvrigt, når jeg skal sælge det med at kunne se kommentarer af indhold på tværs af sider og at kunne sætte sine egne filtre på kommentarerne, er også, at man kan argumentere for, at man ret hurtigt vil kunne drage fordel af systemet allerede med helt fundamentale inddelinger af systemets brugerfællesskab. Dermed vil visse brugere altså hurtigt begynde at få gavn af systemet, og så kan det jo bare udvikle sig derfra. 
%(14.07.21) Til databasen skal der også være seperate fil-entiteter, hvis nøgle så skal være hashet af selve fil-objektet. Jeg har i øvrigt overvejet, om man ikke bare kunne tage et hash af f.eks. starten af en video, men hvis man gerne vil sikre imod f.eks. nsfw eller lignende, så kræver det hashes af hele videoen. Men derfor kunne man jo godt have begge ting. Jeg tænkte også, at man måske kunne gå efter højeste lyd-spike eller noget i den stil, hvis man vil prøve at gøre start-tidspunkts-beskæringen mindre vigtig (når vi taler om hashes kun af dele af en video). Nå, men tilbage til fil-entiteterne i databasen: Disse skal så være på ret meget lige fod med artikler (og eksterne links), idet de også skal kunne gives point. Fil-entiteters "brødtekst" bliver så bare selve indholdsobjektet, som dog sagtens kan være null --- og som kan være null i én lokal database, men ikke-null i en anden; tilgængeligheden af fil-objekter må godt kunne variere (hvor man jo så altid kan requeste til sin lokale database, at den henter filen til en).. I øvrigt tænkte jeg også i går aftes, at man måske også kunne bruge mere syntaks end bare '*' og '&,' for eksempel kunne man måske bruge en syntaks til at give et tekstudsnit linket, så man kan trykke på denne tekst for at følge linket. Jeg tror dog, at jeg bare bør starte i det helt simple med prototypen (hvis jeg selv laver den), og måske bare nøjes med '&' som standard og eneste mulighed (hvorved man så ikke behøver at tilføje symbolet nogen steder). Og noget andet er, at jeg jo forestiller mig, at artikel-("brød")teksterne bør være hypertekst og dermed html, men hvor man måske så bare skal begrænse html'en lidt, og så var det, at jeg tænkte, at hrefs så kunne begrænses, så de kun kan følges, hvis de kan slås op i databasen og have et tilstrækkeligt antal point af diverse pointyper, som hver enkelt bruger selv bestemmer (de bestemmer altså både pointtyperne samt hvor høj værdigrænsen skal være, før at href-linket kan følges). Jeg overvejer så lidt, hvis jeg skal lave en simpel prototype, så bare at åbne en ny tab, når et databaselink følges, og så skal man måske bare få vist artiklen (som så kan have indre links i sig), de eksterne links og deltitel-linksne (måske ude i siden) og så muligvis en søge-formular, så man selv kan søge efter (alverdens) yderligere links til databasen. Og så kom jeg nemlig også lige i tanke om, at da disse tabs skal køre (har jeg i sinde) over/via localhost, så kan man også komme til at få mulighed for at åbne programmer fra tabben, og man kan også køre diverse scripts, der måtte være, så man f.eks. kunne bruge små applikationer (måske i vinduer) på en artikel-side, som så kan køres i mere effektive sprog end JavaScript. Så sørger man bare for at localhost-programmet sender grafisk info (f.eks. hvor i billedet befinder sprites'ne sig) til browseren, som så bare kan bruge JavaScript til at opdatere grafikken i applikationsviduet, men som altså ikke bruger JS nogen af de underliggende udregninger. Herved kan man altså ret let få det, så brugere kan bygge applikationer, der kan køre i selve browsertappen, men hvor den eneste begrænsning på effektiviteten rigtigt, er at det grafiske interface så skal implementeres med JS/CSS i browseren. Og hvis dette alligevel bliver begrænsende, så kan man altid bare lave programmer, som bare åbner i separate vinduer i brugerens dekstop environment, når brugeren klikker på programmet. Og grunden til, at man pludselig kan alt dette, er jo simpelthen pointene; fordi alt internetindhold i dette system nu har point tilknyttet sig, hvor man altså uden de vilde udfordringer kan bygge forskellige pointtyper med meget høj troværdighed, så brugere med det samme kan se, om en fil, et program eller et link er sikkert. 
%Hm, jeg mangler måske at tænke lidt over betalingsmure osv. (det har jeg lige overvejet lidt, om jeg gør), men nu kom jeg også lige i tanke om, at man da næsten bør kunne have variabelt indhold i brødteksterne også.. Nå ja, det var jo lidt en del af min idé om, at man skal kunne udskifte tekstudsnit invendigt.. Hm.. Ja, så '*'-udsnittene er ret vigtige på længere sigt, og vigtigt det med, at man kan give intended pointtyper til dem som forfatter. Hm, og så skal brugerne vel bare kunne indstille sine programmer i egen ende til at parse disse pointforslag (også hvis ingen er opgivet), og så justere dem til egne præferencer. De ville så være smart, hvis forfattere så også kan kommunikere typen/klassen/kategorien af det indsatte objekt, så læserne bedre kan erstatte det med et, der passer bedre til dem.. Og kunne man så bare gøre dette som en del af "pointforslagene?".. Hm, sker dette bare via titlen, det gør det vel nok..? ..Hm, måske skulle man så fremhæve ontologi-kant-links som noget fundamentalt for sig selv, ligesom med deltitellinks'ene..? ..Tja, det kunne dog også bare være i form af specielle deltitler.. ..Og deltitellinks kunne jo implementeres, så de kan få tags tilknyttet sig, og som kan parses direkte fra titlen.. Og, så tænker jeg altså, at man måske kunne indføre en konvention om, at browser-udvidelse/localhost-program lige sørger for altid at navigere til klasse-artiklerne, som altså refereres til i klasse(-tag)-deltitler, ... eller bare kigger på referencen direkte (måske behøver man ikke at query'e databasen), for så at se om pågældende klasser har nogen specielle point-præferencer hos brugeren. ..Hm, okay, lyder vist fornuftigt nok, men så skal jeg også lige tænke på.. lister: Brugere skal også kunne specificere lister over artikel-objekter via pointene. Dette er faktisk rigtigt vigtigt, for ellers skal hver enkelte (html-)struktur, som brugerene kan få vist gemmes som en indgang i databasen, men hvis pointene til gengæld bare kan afgøre indholdet, rækkefølgen og størrelsen af lister, så er man good to go, og så kan hver brugere få vist artikel-objekter, der er skræddersyet til deres præferencer. Ja, også sådan noget som CSS-stilarter kan man også styre i så fald. Vigtigt. Så der skal altså være visse entiteter i databasen --- muligvis bare artikler, men det vil jeg lige tænke over nu --- som kan udfoldes (måske med '**'-syntaks..?) som lister, og hvor listernes form altså kan afhænge af pointtyper. ..Hm, en mulighed kunne måske være at indføre (html-)struktur-entiteter, som så kan bruges i artiklerne, og hvor man så kan bygge mere komplicerede strukturer af allerede definerede strukturer.. Hm, lyder da meget interessant.. Og så skal det, jeg lige sagde om klasse-links og personlige præferencer så stadig gælde.. Okay: Ja! Ja, man skal absolut have struktur-entiteter i systemet (så folk bl.a. kan linke til en struktur og/eller (CSS-)stil, som så kan opdateres eksternt, og som brugeren også kan præcisere ud fra egne præferencer som en ting for sig; uden at det blandes sammen med præciseringen af indholdet)..!! Fedt. Okay, så hvordan skal disse struktur/stil-entities implementeres?.. ...Hm, man kunne måske for det første have en måde at lave et liste-link inden i en tekst, der foldes ud til en liste af artikler, med eller uden overskrifterne, og hvor man så.. Hm, ja, én ting kunne være at vælge en titel og en række point-forskalg, hvor en liste af objekter med samme titel foldes ud i et vist antal, sorteret efter pointene (med mindre pointfoslagene så altså ændres i brugerens ende).. Hm, og kunne man så ikke gøre noget med klasser også, så man ikke får artikler med titler af samme navn, men i stedet får artikler hørende til samme klasse..? ... Okay, to tanker: Ét er, at man måske skulle kunne kalde strukturer via links, hvor man starter med den grundlæggende liste og så kan tilføje ting og tilføje stil til listen derfra ved at sætte flere stil-/struktur-referencer bag på reference-/link-strengen.. Og noget andet er, om man ikke bare skulle bruge sepererere titler og deltitler, så deltitlerne bare optræder som normale titler i princippet, måske bare med tags foran..?.. ..Det giver på en måde mere mening (altså fra nogen synspunkter).. ..Og hvis man så kunne have point, der refererer til en tidligere titel, så kunne man jo sige, "denne deltitel er god til teksten, hvis denne (anden) titel kommer først"..!.. Hm.. ..Tja, eller en mulighed kunne måske også være, hvis titler bare kan referere andre titler, hvorved det så forstås, at det skal ses som en deltitel/undertitel, hvis ikke denne reference er null.. Hm, måske ikke helt dumt.. Hm, eller disse undertitler kunne så.. hm, de skal vel refere til artikler, så den specifikke tekst hører med.. eller nej..? ..Jo, de må skulle referere til artikler.. Hm, lad mig lige tænke over, hvad man kan bruge disse ting til, og om det er det hele værd.. ..Hm, én ting er jo, at man altid kan blive ved med at hælde hypotese-deltitler på i princippet, så her er det nok meget godt, hvis deltitler kan foreslås til en eksisterende titel men samtidigt holdes seperat fra hovedtitlen.. ..Hm, kunne man evt. implementere deltitlerne i form af artikler, hvis text bare referere til artikler og ikke "brødtekser"..? Tja, hvorfor ikke?.. Ja, det lyder altså ok fornuftigt. Og så kan klasser altså f.eks. indføres ved at folk tilføjer klasse-undertitler til artikler, og dette kan så bare gøres frit og til hver en tid. ..Hm, men ville det mon give mening at gøre begge ting (altså de med formel parsing af deltitlerne og så det med de frie deltitler)?.. (Indskudt: I relationen over lokale og globale nøgler, må der gerne kunne være flere lokale nøgler til én global nøgle.) ..Nej, det giver ikke så meget mening i så fald, men måske skulle jeg gå tilbage til tags foran titler og deltitler så. For nu kom jeg nemlig lige i tanke om, at hvis man giver plads til tags store nok til, at de kan være ontologi-links, så kan man jo tilføje et vilkårligt semantik-præfiks til hver titel og deltitel..! Uh, og så får vi også hermed direkte givet bolden op til at bruge en "sprog-ontologi" mere og mere i brugernes titel-konventioner..!! NI-ICE!.. (14.07.21) ..Åh, fuck hvor fedt, altså..(!) Det tror jeg, kommer til at lette det hele en hel del.. ..Ja, det bliver sgu sådan.. 
%Og når man skal indsætte/udfolde lister, som brugeren gerne selv må få indflydelse på at præcisere, så skal man så bare basalt set linke til under-/deltitlerne (og så typisk til deltitler med tags, der anfører, at der er tale om en slags klasse-definition eller en tekstprædikat-specifikation) samt indsætte nogle pointforslag, samlet set eller til individuelle undertitler, og bum, så kan programmerne i brugerens ende (selvfølgelig ved også at parse hele den omkringliggende syntaks omkring linket --- hvor man måske kunne bruge '**,' eller hvad ved jeg? (skal dog nok finde på et rigtigt forslag)) justere pointforslagene efter brugerens præferencer (bl.a. ved også at læse hvilke klasser, der er tale om), og så udføre de queries, der finder de bedste forslag og ordner dem i rækkefølge ud fra brugerens endelige pointsammensætning.
%Angående strukturerne så skal disse nok stadig være med, som jeg foreslog, men er sådan set uafhængige af klasserne; hvis brugerfællesskabet har en klasse, der sætter begrænsninger på strukturen, så er det brugerfællesskabets eget ansvar, at sørge for at disse er overholdte (ligesom at alt muligt andet er deres eget ansvar). Men ja, brugerne bør kunne uploade og udbygge struktur/stil-entities på konstruktiv vis, så at komplicerede strukturer kan bygges som kompositter af andre strukturer (og hvor man også har operationer til at gå ind og tilføje ting og ændre ting i struktur-listerne).. Og hvis man så som forfatter gerne bare vil pege på en vis overordnet stil, men gerne vil tillade, at denne stil kan opdateres løbende og i øvrigt justeres og præciseres af brugeren (og hvorfor skulle man ikke det?), så kan man altså bare linke til en klasse i stedet (og evt. give nogle point-forslag med), og så kan bruger-ende-programmerne bare hente den mest passende stil inden for denne klasse, og indsætte denne. ..Hm, men det bringer mig så til: Skal der være syntaks for at folde forskellige ting ud fra klassen så som ét objekt, flere objekter eller stil/struktur-objekter (i stedet for tekst/indholds-objekter), eller skal disse valg ske via klasserne?.. ..Hm, jeg tænkte at foreslå, om man kunne gøre begge ting, men nu kom jeg i øvrigt også lige på, om klasser på en eller anden måde kunne tilføjes strukturer til titlen, som så automatisk sættes først inden al videre konstruktion..? ..Man kan sige, at strukturer/stile jo er så fundamentalt forskellige fra tekster/indhold, og også børe implementeres som to forskellige ting i databasen, så det ville nok ikke gøre noget også at udskille syntaksen på dette punkt. Men hvad så med syntaks for at indsætte/udfolde lister eller enkelte objekter, og hvad med muligheden for klasser med en inbygget, underliggende struktur..? Hm, nej man må vil bare sørge for, at strukturerne passer til deres klasser (så at klasse-undertitler kun bliver stemt rigtigt frem af brugerfællesskabet (styret af visse brugervalgte moderatorer for pointene, selvfølgelig), hvis pågældende artikels stil passer med klassen).. Hm.. Og hvad så med stil-klasserne..? ..Hm, men kunne man egentligt ikke blande indholds-artikler og ren-stil/struktur-artikler lidt sammen..? ..Ah, jo for man kunne jo bare implementere de variable stil-referencer med en syntaks, der finder en artikel på efter samme protokol som altid, men i stedet for at indsætte indholdet (med eller uden dennes stil(/struktur)..) indsætter den bare stilen fra atiklen og ikke indholdet.. Og så tænker jeg altså bare, at artikler også kan have faste stile, som så bare kan refere dirkete til stil-entities.. Hm, hvordan virker dette --- og giver det i øvrigt mening, hvis man kun kan vægle at ignorere stilen fra en artikel eller tage det med, eller skal man også have flere valgmuligheder imellem disse to..? (Sidste spørgsmål handler altså om at indsætte indhold..) ..Hm, jeg tror faktisk bare, at man bare skal kunne indsætte med eller uden header-stilen, og så er det det.. For folk vil så bare typisk sørge for først at lave den grundlæggende struktur til en artikel, og så bagefter lave en artiklen bestående af en stil-header og så med underlæggende artikel sat ind, for så får man det bedste af begge verdener; man får en artikel med læsbar stil, og man får en artikel, hvor stilen lettere kan justeres, fordi den bare sættes på efterfølgende.. ... Ja, jeg tror altså umiddelbart, at alt dette kommer til at fungere meget godt. Og man kan jo passende bare tage udgangspunkt i HTML, CSS og JavaScript, hvor man så bare lige begrænser det, så links kun kan følges med godkendelse fra pointene fra databasen.. (14.07.21) 
%(15.07.21) Uh, og lokal-global-nøgle-relationen behøver ikke dække alle nøgler, men kan godt bare dække selv en lille mængde af nøgler, så man kan sagtens sørge for, at f.eks. alle populære tags (som altså bruges ofte) får en mindre datatype. ..Hm, og som det er nu, så er det altså meningen at brugere selv skal opdele databasen i klasse-artikler og indholds-artikler, hvor sidstnævnte handler om tekster beskrevet af deres titler, og hvor førstnævnte handler om titler beskrevet af deres tekster, altså titler som så kan bruges til at kategorisere indholds-artikler.. Eksterne links mellem klasser/tags vil så beskrive forholdet imellem to mængder i stedet for individuelle artikler.. ..Hm, men man kan vel ligeså godt adskille de to entitetstyper så, for dette skal man jo alligevel gøre, når man skal implementere brugerende-programmerne (for hvad et program skal gøre med et link vil altid afhænge af klasse-eller-indhold-typen alligevel, vil det ikke?).. ...Hm, men er "titler" og "klasser" så synonymer..?.. ..Kunne man også i stedet se det som, at tilknyttede tekster kan regnes for at have en type alt efter, om de beskriver, hvad titlen repræsenterer, eller om de repræsenteres af titlen..? ..Ville måske ikke være helt dumt, for så kan vi netop lade 'titler' og 'klasser' være lidt den samme ting.. Hm.. Ja, det virker helt fornuftigt. ..Og jeg kan ikke rigtigt se, at man mister noget på ikke at have "samlede titler" på objekter; det må egentligt være helt fint bare at kunne føje deltitler til uafhængigt af hinanden. ..Og så har jeg jo konceptet om "tags," som jo egentligt var (og måske fortsat bliver) beregnet til at ændre den semantiske funktion af en titel/klasse, men hvor man stadig beholder linket til titel/klassen uden den semantiske præcisering (så man f.eks. kan klassificere en tekst med dens hypoteser, og altså understrege, at de er hypoteser/antagelser, men stadig få det link, der kan føre til deres argument).. ..Hm, "klasser" kommer vel så til i praksis at kunne ses mere som "termer" (hvor man så konstruere en klasse-titel ved et "tilhører klassen"-tag efterfulgt af termet).. ..Hm, så titler kan måske ses som en "relation" (nemlig hvad jeg har kaldt "tags") og så et "term," der godt også kan være en sætning, således at man kan danne titler/prædikater som f.eks. "opfylder" + "<sætning>" eller "antager" + "<sætning>".. ..Hm, man kunne i øvrigt måske kalde dem "prædikat funktioner" i stedet; det lyder måske en anelse kluntet, men så er der i hvert fald ingen forvirring.. Så de tiltænkte typer (uafhængigt af, om de bliver implementeret som seperate typer (men muligvis)) er altså 'prædikat funktioner,' 'termer,' 'prædikater,' "term-definerende/forklarende tekster" og "indholds-tekster".. (Og artikel-entiteterne og link-entiteterne osv. bygges så oven på disse i næste lag..) ..Nå ja, og så er der også "stil-definitioner".. ..som altså måske bare kan være CSS og JavaScript (hvor hyperlinks så bare begrænses).. ..(og Posts og Gets osv.).. *(Og prædikat funktioner kan så særligt beregnes til at gives kortere nøgler, og sikkert i form lokale nøgler..) ..Okay, men det bringer mig vel tilbage til, hvad man lige gør, når strukturer skal udfoldes i en hypertekst.. ..Hm, men det er vel bare at kunne folde lister ud, og så have concat-operatorer og andre liste operationer, så de eftersøgte ting kan foldes ud hver især og sammensættes til en hvis struktur, der så passer til det stil-link, man så kan sætte som header til fold-ud-strukturen.. ..Og point-forslag sættes så bare bagpå hver enkelt fold-ud-liste i den samlede (fold-ud-)struktur, og så kan man evt. også give stil-headere til hver enkelt liste, hvis dette bliver nemmere end kun at lave en overordnet stil-header til den samlede struktur.. Gode tanker.. ..Ah, og forfatteren kan så selv vælge, hvor faste antallene af liste-elementer skal være, men så snart der kan være præferenceforskelle, er det så bare en god idé at sætte grænserne via passende point i stedet for de faste operatorer, så brugerne selv kan justere det.. Ja, eller man kunne dog også netop indføre "løse" operatorer, som altså forslår et antal, men hvor brugerne netop kan justere selv. ..Okay.. er jeg ved at være der nu så..? (altså for denne omgang..) Hvad mangler jeg mon at tænke over..? ..Eksterne links vs. point med link-outpus..? ..Nej, jeg er ret overbevist om, at eksterne link-entiteter skal med.. 
%Hm, jo, jeg kunne måske tænke lidt over, hvor godt termer kan inddeles flere dele... og også om prædikat-funktioner må kunne tage flere inputs.. ...Point skal forresten være seperat fra resten, for det er ikke meningen, at point selv skal kunne gives pointvurderinger, men et spørgsmål er til gengæld, om eksterne links ikke bare kan blandes ind under prædikater?.. Hvordan bliver term-ontologien og prædikat-funktion-ontologien så; kan man opbygge disse uden eksterne links..? ...Uh, i øvrigt kunne hrefs/hyperlinks/URL'er (alle refererer til det samme) også være deres egne typer af entiteter.. Ah ja, og så får man også ret automatisk, at der bliver forskel på indholdspoint, og href-sikkerheds-point, hvilket altså er ret dejligt.. ..Ah, og så kan man også vedligeholde hrefs i databasen!.. (Altså modulært.) Nice nok. ..Nå, jeg tror på, at "ekterne links" skal erstattes med prædikaterne for artikler, men hvad med term- og p-fun-ontologierne?... Hm, men det handler jo bare om, at alle skal kunne have prædikater tilknyttet sig, så.. Hm, kunne termer og p-funktioner ikke bare være tekster..? ..Det ville da umiddelbart give rigtig god mening..(!).. ...Hm, og det gør ikke noget, hvis term- og p-fun-teksterne også kan være xml, for så kunne man jo herved lave konventioner for, hvad linksene skal erstattes med af tekst i titlerne, når brugerne læser dem, og hvad der f.eks. også kunne være af mouse-over-tekst osv.. ..Ni'ce.. ..Og hvis man stadig vil dele det op i typer, så kan man jo bare lige tilføje den byte mere til linkene.. Men ellers bør bruger-ende-programmerne jo også bare selv holde øje med typerne (for det kan de, og det skal de nok alligevel).. (15.07.21)
%Jeg synes ikke, der skal være typer implementeret i databasen. Der er bare tekster og prædikat-tekst-relations-entiteter, hvor det så er konventionen, at første del af prædikatet er en tekst-reference, og hvor anden del af prædikatet nok kan være enten eller.. Ja.. For så kan man netop relaterer termer til termer i ontologierne. Jeg tænkte så at foreslå, at der kunne være liste-typer i databasen, men hvad med, at man bare implementerer alle lister og sekvenser via teksterne. Hvis man således vil brugere højere-ordens- og/eller fler-variabel-p-funktioner, så må man bare uploade en liste/sekvens i form af en tekst, og så referere til denne. Så skal man nok bare lige indføre nogle konventioner (som kan justeres og ændres af brugerfællesskabet) for, hvordan bruger-endeprogrammerne skal parse disse sekvenser, og hvad de skal gøre, når der opgives sådanne reference lister i stedet for bare én reference.. (15.07.21)
%..Og det vil ikke være helt dumt at have enkeltords-tekster, som man så kan bruge til at søge på, om så i første omgang kan lede hen til flere betydninger af ordet og så derfra lede videre synonymer, underkategorier eller overkategorier, og herfra kan man så få sig sine søge forslag, hvis man har indtastet flere søgeord.. Fil-objekterne skal jo have sin egen entity-type, men her kan man også bare bruge enkeltordstekster til at søge på dem, nemlig i form af fil-hashet (som vist gerne må indeholde præfikser, så de adskiller sig fra tekst-nøglerne). Hm, skal prædikat-tekst-entiteterne egentligt så have nøgler, eller skal man altid bare navigere til tekster med prædikater som kanter så at sige..? I øvrigt kan jeg lige sige, at selvom point måske godt skal kunne gives til tekster og filer, så er deres primære formål altså at blive givet til prædikat-tekst-entiteter. Point skal ikke kunne gives til andre pointgivninger.. Så hvis man alligevel vil gøre noget a la dette, må man bare implementere en konvention for at referere til specifikke pointgivninger i teksterne. ..Nå ja, angående mit spørgsmål så både ja og nej: Der skal nemlig være nøgler, men nok kun for at pointene kan referere til entiteterne. Ja. ..Hm, værd at bemærke forresten, at systemet nu grundlæggende set (foruden fil-, stil- og url-objekterne) er et triplet-baseret system, hvor tripletterne kan gives point; så altså på en måde ret meget ligesom, hvad jeg ellers har tænkt (men jeg er uden tvivl nået længere tankemæssigt siden da :)). ..Tja, og så er der jo det, at det hele er en (delt) (relationel) database, og at URL-links'ene kun indgår som en ekstra-ting (og altså på en ikke-central måde), og så er der også det med at benytte hashnøgler som globale nøgler. ..Og det med fold-ud-links... tja, men nu sagde jeg også i det "grundlæggende"... ..Og lige for at runde en lille ting af, så synes jeg bestemt at point skal kunne gives til alle de grundlæggende objekter (tekster, filer, url-objekter.. stil-objekter, selvom jeg egentligt nu synes, at stil-objkterne bare skal implementeres som tekster... Hm...), også selvom det kan virke lidt redundant, for så kan man f.eks. flagge indhold direkte, så behandlingen af faresignalerne kan ske i et seperat lag... Ja, umiddelbart virker det ret fornuftigt sådan.. Ja. Så kan prædikat-tekst-entitets-vurderinger komme til primært at handle om ontologi-kanterne, eller databse-linkene, om man vil, og objekt-point kan så være alle vurderede prædikater.. Hm, eller..? Prædikaterne kan i øvrigt også handle om semantik omkring selve objektet og ikke bare links.. Hm..? ..Tja, eller hvorfor ikke bare kræve, at pointgivninger altid hæfter på et prædikat? Man kan jo bare lave standard prædikater til point, så det sker automatisk.. Ja.. Lyder umiddelbart forbuftigt.. ..Ja. (15.07.21)
%(16.07.21) Angående fil-hashes og det med at søge på filer i databasen, så kan man også bare uploade hashet som en tekst og så foreslå prædikater til det, der forbinder det med kilder, og også evt. med den tekst, man også har uploadet til at beskrive, hvad man gerne vil sige om filen. Angående "tripletter" så er det vigtigt at pointere, at selvom det godt kan udvikle sig til, at man uploader udsagn i form af tripletter, hvor p-fun-delen af tripletten bliver verbet i den uploadede sætning, så er dette altså slet ikke meningen i starten. Mening er i meget højere grad, at tripletterne skal tolkes som bestående af henholdsvis tekst, semantisk tag og titel... Nå nej, på nær når der er tale om links / ontologi-kanter; så kan det i stedet tolkes mere som en relation imellem tekster/termer. ..Hm.. Måske skal man gå væk fra, at indføre url- og... tja, nej, der skal jo være specielle fil-entities, men skal der være specielle url-entities? Man kan jo stadig implementere, at de bliver vedligeholdt modulært, bare hvis man bruger teksterne.. Hm, men hvordan får man ellers linket point til, om man kan følge url'erne eller ej..? ..Tja, det ville vel være ved at indføre en konvention i brugerende-programmerne.. ..Hm, men i praksis vil man jo aldrig tilføje url'erne direkte i teksterne.. Hm, men jeg har jo også det med at tilføje et link til et efterfølgende tekstudsnit.. ... Hm, skal lige tænke over nogle ting.. 
%(21.07.21) Okay, det blev alligevel til en del tænkedage, men nu tror jeg også, jeg har et rigtigt godt billede af, hvordan systemet skal være --- og har i øvrigt også et ok billede af, hvordan jeg nok bør gå til det, nemlig at jeg faktisk bør fokusere på programmeringsdelen først, som jeg også har tænkt på før. I går kom jeg også på i den forbindelse, at man nok bør sigte efter at få det godt integreret med Git som noget af det første. Men nu tror jeg bare, jeg vil stoppe denne brainstorm her, og så faktisk bare skrive mine tanker direkte ned som ikke-kommentar-noter (og egentligt så bare fortsætte, hvor jeg slap). 



(21.07.21) Da jeg afbrød ovenstående paragraf her gik jeg i gang med en af mine længere brainstorme gjort i kommentarene (altså udkommenteret) af dette dokumentet (altså kildekoden), hvor jeg har tænkt over design-detaljer omkring, hvordan man kunne implementere systemet. De seneste dage har jeg så tænkt nogle flere tanker (men uden at skrive brainstorm-kommentar-noter her), og nu har jeg vist et ret godt billede af, hvordan systemet skal være. Og i stedet for at fortsætte min kommentar-brainstorm, vil jeg så bare skrive de noter ind her med det samme:

I det helt grundlæggende lag af systemet har vi en delt database, hvor hver node i databasen godt kan have sit eget data med begrænsninger på, hvem der kan få lov at se / få serveret denne data (så man kan godt have private netværk, hvilket bl.a.\ er vigtigt, fordi systemet skal kunne fungere lidt ligesom Git, hvor brugere altså gerne skal kunne have, hvad der svarer til private mapper), men hvor der selvfølgelig også er en offentlig mængde data i databasen, som alle skal kunne se. I denne database skal brugere kunne uploade, hvad der svarer til termer og tripletter (i.e.\ ontologi-opbyggende data), men det er kun det overordnede billede; der ligger mere i det end dette. For det første er tripletter ikke bare tripletter; propositions-entiteter i databasen kan godt komme til at bestå af længere sammensætninger. Hvis brugerfællesskabet således godt vil oprette et prædikat med $n$ input, og der ikke i forvejen er en propositions-relation i databasen med $n+2$ attributter, så anmoder man bare databasen om at oprette sådan en relation (og evt. bare lokalt hos et delnetværk af systemet i første omgang). I praksis vil vi dog ikke få brug for vildt mange af sådanne relationer, og da det ikke koster databasen noget rigtigt, kan man bare starte med alle propositions-(database-)relationer med $n\in\{0, 1, 2, \ldots\}$ og op til $n=20$ eller $n=40$ eller noget i den stil. (Jeg forestiller mig dog, at man nok ikke får brug for meget mere end en 10 stykker.) *[Nej, jeg kommer faktisk til at lave dette om, så databasen kommer til at understøtte propositionsentiteter af vilkårlige længder, og hvor der så bare bruges linked lists basalt set, når længden overstiger en vis grænse. Ydermere mener jeg så, at dette bør være skjult i databasens interface, %så brugerne altså bar... Hm, skal det nu også være skjult, eller skal man bare give et inputtal med?.. ..Måske det sidste.. ..Hm, en fordel ved at have det skjult, og at databasen kan have det i samme relation, kunne måske være, at man herved så kan opnå, at sammenhængende data (i form af sammenhængende strukturer) så kan lægge tættere på hinanden (med lokale nøgler)... Ah, men man kan gøre begge ting: Man kan gøre det skjult samtidigt med at man dog lige giver inputlængden med i nøglen (for hvad sker der ved det).. Ja, det må da næsten blive svaret; brugeren skal jo alligevel bruge længden, så det er slet ikke dumt, at denne bare kan aflæses med det samme --- ja, især når der jo er tale om linked-lister; så giver andet jo nærmest ikke mening alligevel. Ok. ..Hm, men hvordan bliver query-funktionerne så?.. ..Hm, skulle man mon gøre noget i retning af, at man bare ikke kan lave relationel algebra-operationer efter en vis længde..? ..Hm, eller ville databasen egentligt ikke godt kunne håndtere sådanne rimeligt effektivt..? ..Hm, man kunne også bare.. ja, man kunne flere ting.. ..Ah, nu ved jeg det.. Og man kan i øvrigt også ret nemt sammenblande flere reletioner af forskellige størrelser til én ved at forkorte de længere, så man kunne enten have et interface, hvor alle længere closures også ses at optræde i alle de mindre closure-relationer, eller man kunne bede brugerne om.. at lave sammenblandingerne selv.. men nej, jeg tror nu, at det bliver smartest med omtalte interface. For så kan databasen bare selv implementere en vis makslængde på closure-del-listerne, og hver gang en mindre relation query'es, så sørger programmet under dette interface bare lige for at inkludere alle de længere relationer skåret ned også. Og så skal brugerne nemlig ikke tænke på den del, og så kan det også samtidigt variere frit med, hvor lange dellisterne kan være i databasens implementation. Med dette er det så bare vigtigt, at closure-længden ikke bliver en del af nøglen, men bliver en attribut, som sættes automatisk via de underliggende queries, som bruges bag interfacet, for så må compileren (tror jeg, håber jeg) kunne skære de andre relationer fra igen, hvis brugeren kun er interesseret i closures af lige netop én længde. Ja, det må kunne lade sig gøre, at sørge for at compileren kan dette. Cool.:) ..Ah ja, og selv hvis man ikke kan stole på, at de konventionelle compilere vil gøre det, så kan man bare give programmet under interfacet ansvaret for selv at kompilere det nok til, at de kan! Nice.
så brugerne ikke kan se, hvordan listerne er gemt, men bare ser en masse relationer af forskellige længder. Og nu tænker jeg endda også, at længere relationer bare automatisk kommer til at skæres over og ``kopieres'' ind i de mindre relationer, dog med den oprindelige længde tilknyttet som en selvstændig attribut, så disse relationer let kan adskilles igen. Pointen er så, at dette bare kan implementeres i de resulterende queries, som brugernes queries oversættes til inde bag omtalte interface, så databasen dermed stadig kan holde entiteterne adskilt i hver deres relation (af forskellige længder) inde bag dette interface (og dermed behøver databasen altså ikke fysisk at kopiere entiteterne ind i de mindre relationer). I øvrigt skal relationerne ikke længere bare indeholde ``propositioner,'' som jeg ellers har lagt op til, men skal bare kunne indeholde generelle (HOL-)typer, hvorfor jeg også er begyndt at kalde den `closure'-entiteter i stedet (taget fra Lisp-terminologi).]
%)%*(Ja, man vil nok kun primært få brug for 0-2 for propositioner og så lidt flere for point, men det er nu lige meget; friheden må rigtigt gerne være der.) 
Grunden til, at der kan være denne frihed i databasen, er, at systemet faktisk bliver en smule ``refleksivt,'' kunne man næsten sige, fordi brugerne faktisk selv kommer til at kunne programmere algoritmerne og meget mere, ved at uploade dem som termer i databasen, hvorved de så kan stige i omtale og i anerkendelse hos brugerfællesskabet og til sidst blive taget i brug som algoritmer. Tanken er nemlig nu, at selv de første algoritmer alligevel implementeres ved at de uploades til databasen, og at brugere så vælger at bruge disse algoritme-termer i deres browserinterface, så på den måde bliver det altså ret oplagt, at brugere selv kan lave deres egne algoritmer, og det er derfor, jeg mener, at man nærmest kan kalde det ``refleksivt,'' selvom det dog er en meget blød form for refleksivitet (og også lidt søgt ift.\ den semantiske betydning af begrebet\ldots) --- og som selvfølgelig ikke adskiller sig synderligt fra ethvert andet open source-projekt, men stadigvæk\ldots\ Men ja, så derfor er det kun en fordel, hvis databasens funktionalitet kan justeres efter brugernes behov. Men derfor skal det grundlæggende lag dog heller ikke være helt uden begrænsninger og konventioner. Før jeg når til de specifikke detaljer, vil jeg dog også lige pointere, at der udover termer og propositioner jo også skal være `point,' som noget helt centralt. Som attributter skal disse tage et bruger-id/pointgiver-term, et en point-definition (som også er en speciel term, der hører til en speciel relation i databasen) og et grundled-term samt $n\in \{0, 1, 2, \ldots\}$ ekstra attributter, som bl.a.\ kan udgøre selve ``pointværdien'' af point-entiteten og også en timestamp-attribut for, hvornår pointet blev uploadet. Så der skal altså også være den samme åbenhed for point-relationer af vilkårlig længde som for propositions-relationerne. Er point-entiteter og relations-entiteter så selv termer? Ja, alle database-entiteterne er ``termer'' (også inklusiv prædikaterne til propositionerne i øvrigt) og kan altså refereres til helt frit af både (andre) point og af (andre) propositioner. Så alle forbehold over for uendelig rekursion, enten semantisk eller i algoritmer, må brugerne altså selv tage højde for i overliggende lag. 

Der kommer dog nogle restriktioner i forbindelse med pointene. Som nævnt skal point-definitions-termerne i første omgang være entiteter af en speciel relation. Attributterne skal så være en/et forklarende proposition/prædikat, der beskriver semantikken bag pointtypen, samt en reference til en `moderator-liste,' som er en liste over offentlige krypteringsnøgler tilhørende brugere/brugergrupper/instanser, som så før magt til at moderere pointtypen, hvilket bl.a.\ vil sige at godkende point og fjerne point af den pågældende type.  Moderator-listen skal bare, tænker jeg, være et atomart tekst-term med et bestemt format, som bare kan være de offentlige nøgler adskilt med kommaseparatorer, hvilket database-server-programmet så kan parse og bruge til at tilføje moderatorerne til en (database-)relation mellem pointdefinitioner og moderatorer, som serverne selv har eksklusiv kontrol over (så denne relation er altså ikke åben over for bruger-uploads). Hvordan moderatorer så kommunikere til database-serverne og får dem til at slette eller godkende point, der kan man så lige overveje; det kan gøres på flere måder (evt.\ via specielle pointuploads fra moderatorerne, men det kunne også ske bare via en ekstern kommunikationsprotokol). Så langt så godt ift.\ moderator-listerne, men der hører også mere til historien. Moderatorer skal også kunne tilføje whitelists til samme pointtypen. Disse whitelists er så i sig selv en moderatorliste, men disse whitelistede moderatorer bliver dog ikke tilføjet direkte til den grundlæggende moderatorliste. I stedet bliver hver ny whitelist tilføjet som børn til et ``moderator-træ'' (sådan kan man forestille sig det), hvor altså også hver whitelistede moderator selv kan tilføje børn under sig. Pointen er dog, at det rent er op til de algoritmer, der bruger moderatorernes svar, i hvor høj grad de vil medregne hvert led af børn i dette træ. Ah ja, så dette betyder egentligt også, at moderatorernes bidrag faktisk skal kommunikeres via specielle point-uploads, ligesom jeg foreslog det. Hm, eller skal disse være specielle; kan de ikke bare implementeres som normale point-uploads\ldots? \ldots\ %Hm, nej det må faktisk gerne være specielle point...
Ah jo, pointene skal være specielle, idet de %uploades %... Eller skal de bare have en bestemt type "bruger-id"-underskrift..? ..Uh, eller skal alle point fra modererede pointyper ikke bare have en ekstra attribut, der fortæller, hvor bruger-id'et var i træet til den pågældende dato?..! ..Jo..! 
skal have en speciel attribut, som database serverne har ansvar for at sætte korrekt, og som fortæller, hvor i moderator-træet, pointgiveren var på tidspunktet for uploadet. Herved kan brugere så nemt finde frem til denne information, i stedet for selv at skulle udregne dette --- hvorved de jo så ville skulle efterspørge og downloade alle point af den pågældende type, hver gang de skal\ldots Hm, eller de kunne jo i princippet også bare bede databasen om aggregater, hvor moderator-træet så på en eller anden måde bliver omsat til den pågældende, ønskede attribut-værdi, men det er dog stadig nok langt nemmere, hvis database-serverne bare har ansvar for at sætte disse attributter én gang for alle. Ja. Så point fra modererede pointtyper har altså altid en ekstra attribut i relationen, som sættes med det samme af database-serverne. Hvis en whitelistet bruger så bedømmer et modereret point forkert er det så bare den pågældende overmoderators ansvar for det første at sørge for altid at få nys om dette i tide, og derefter er det dennes ansvar selv at give et korrekt point (ifølge pointets semantiske definition). Fordi brugere så kan søge fra toppen af grafen og ned, når de søger på et modereret point, behøver man ikke også at slette det ukorrekte point, for brugere vil dermed aldrig nå til dette point, og selv hvis de gør, så ville det jo være dumt ikke lige at gå et trin op for at se, om moderatoren ét trin over ikke har reageret på det. Angående modererede point skal det så også nævnes, at moderator-listerne også gerne må være delt op i flere, som hver især kan have forskellige dato-intervaller på, hvor de er gældende. På denne måde kan man så effektivt opdatere modererede point og deres moderator-lister ved simpelthen at uploade en ny pointtype, som dog har den samme point-definition og som deler alle moderator-listerne fra den tidligere pointtype, men som måske bare har ændret udløbsdatoen på nogle af de ikke-udløbede moderator-lister, samt eventuelt tilføjet flere med startdato fra og med oprettelsen af den nye version af pointtypen (eller evt.\ fra en startdato lidt ude i fremtiden for den sags skyld). 

En moderators whitelist kan dog varieres i løbet af hele tidsintervallet, så der skal man lige sørge for, at databasen også har en relation over moderator, id på dennes whitelist og så lige pointtypen som en tredje attribut, for whitelists'ene kan jo godt ændre sig fra pointtype til pointtype, hvis pågældende er moderator på flere. Database-serverne får så ansvar om at kunne snakke sammen med en moderator om at justere dennes whitelist, men behøver ikke at have ansvar for at gemme historikken. Bare serverne sætter de korrekte moderator-niveau-attributter med det samme, når et modereret point bliver uploadet, så bør det hele gå, for moderatoren kan jo altid rette alle fejl fra whitelisten ved bare selv at udregne og sætte sit eget point, og herefter betyder alle de tidlige fejl ikke længere noget. \ldots Hm, nå nej, det passer faktisk ikke helt: For at man kan gå moderatorer efter i sømmene, så er det godt at bevare alle whitelist-konfigurationer i noget tid, så folk kan finde frem til årsagen (til kilden så at sige), hvis nu et ukorrekt sat point skulle propagere og blive til flere ``ukorrekte point'' --- hvilket det netop ville kunne se ud som, hvis man bare slettede instanser fra whitelistene og glemte dem, når de satte et point forkert. 
I øvrigt må moderatorer gerne føje folk til blacklists også, således at moderator-niveauet, medmindre det bliver overskrevet af en whitelist højere oppe i moderator-træet (som jo også generelt bør overskrive mindre værdier, så den maksimale mulige moderator-værdi altid sættes), sættes som negativt af, hvad den ville have været, hvis moderatoren, hvis blacklist vi taler om, selv havde sat pointet. %Med denne fra-høj-til-lav-konvention for moderator-niveauerne (hvis vi skal bruge denne (men hvorfor ikke?)), så vil almindelige brugere altså få værdien 0 for deres moderator-niveau. Resten af niveauerne kan i øvrigt så være kommatal, så vi starter ved 1.0 for selve overmoderatorerne --- men det er nu ikke sikkert, at man rent faktisk vil bruge denne numeriske værdi til noget\ldots Hm...
Værdien for Moderator-niveauet kan bare eksempel vis gemmes som 1 for overmoderatorerne, 2 for moderatorerne under disse (på deres whitelists),\ldots\ Hm, det er umiddelbart en fin idé, det her, men det kan godt være, at jeg finder på noget bedre, så lad mig bare lige vende tilbage til det.


Bemærk at database-server-instanserne selv med fordel kan fungere som moderatorer, så deres ansvar behøver ikke at være begrænset til kun at sætte moderator-niveaue-attributterne på pointene fra de modererede pointtyper. Samtidigt skal det også nævnes, at der i det grundlæggende lag her heller ikke er nogen egentlige restriktioner på, hvad databasen skal modtage og gemme fra diverse brugere, eller hvad de må og ikke må smide ud, men sådanne løfter bør selvfølgelig gives af database-knuderne (for ellers er de jo ikke meget værd). Og her kan en databaseknude så sagtens f.eks. binde sine løfter ift.\, hvornår databasen må smide visse uploads ud, op på visse modererede point (som de evt.\ selv er med til at moderere), således at termer f.eks.\ siges at må smides ud, hvis de får så og så mange af disse point --- eller måske med negative fortegn på pointværdierne, så vi kunne sige ``så og så lidt'' i stedet. 


Okay, så det var nogenlunde et billede over den grundlæggende opbygning af databasen (minus nogle ting og nogle detaljer). De første, jeg så bør forklare allerede, er, hvad jeg forestiller mig, at de modererede point kan bruges til. Jo, de kommer faktisk til at åbne op for vilkårlige algoritmer i database-systemet. For ligesom at database-serverne alle giver et løfte om at sætte moderator-niveau-attributten korrekt efter en formel forskrift, så kan instanser, der moderere en `modereret pointtype' jo gøre det samme. Og her kan forskriften jo bare være opskrevet formelt i pointdefinitionen. Og disse instanser kan som nævnt være database-instanserne selv, men det kan altså også være vilkårlige andre instanser, inkl.\ brugere eller brugergrupper. Hvis en pointtype er defineret, så pointene kan udregnes automatisk, jamen så vil det jo så ikke være en dum idé, at definere den via det script, som en computer ville skulle køre for at udregne og sætte det korrekt. Og hermed ville de jo heller ikke være en dum service, hvis en vis moderator-instans udbød et point, som gives til andre point, og hvis korrekthed afhænger af, om et gyldigt script kan parses direkte fra input-pointtypens definition (selvfølgelig efter en formel konvention, så man ikke f.eks.\ bare parser $p$, ``dette point gives ikke ved følgende script: $p$''), og som instansen så simpelthen bare tjekker automatisk ved at køre det pågældende script og se, om de satte point var korrekt sat. Så altså med andre ord en pointtype til at verificere andre point, hvis pointdefinitioner følger formelle konventioner, hvor et script for at sætte pointet korrekt for hvert enkle point kan parses direkte fra dennes definition. En sådan service kunne jo f.eks.\ udbydes af database-instanserne (i hvert fald i den tidlige opstart), og så er man herved allerede godt i gang med et system, hvor brugere effektivt set kan uploade og foreslå brugen af nye algoritmer til at sætte pointværdier i databasen. Især hvis man også oveni introducerer en modereret pointtype til at måle på populariteten/efterspørgslen af gamle og nye point-algoritmer og dermed danner end protokol, hvor brugerne selv løbende kan indstemme og udskifte point-algoritmer. Dette system kan så med tiden udskiftes med noget mere skalerbart, men det kunne altså meget vel være en god måde at starte på. 


Så det er altså de primære tanker bag det grundlæggende lag. De atomare termer i databasen skal også bare være frie på samme måde som propositions- og point-relationerne, nemlig ved at brugerne bare skal kunne be database-serverne om at oprette nye relationer efter behov, her kun med én attribut udover nøglen, hvilken jeg i øvrigt heller ikke har nævnt noget om endnu, men det kommer til i denne paragraf. *(Nej, det er forkert; her behøver brugerne faktisk ikke at kunne se, hvordan de atomare termer er gemt, og faktisk kan delen af databasen, der gemmer de atomare termer, fra brugernes side lidt bare ses som en key-value store. Så nej, brugerne skal ikke bede databasen om at oprette relationer til at kunne holde atomare termer effektivt; det finder databasen selv ud af.) De atomare termer skal også bare kunne være af vilkårlig type, i.e.\ hvad end den relationelle database understøtter. Det bliver dog UTF-8-tekster (af forskellige længder i databasen), der bliver det primære omdrejningspunkt, men diverse BLOBs til f.eks.\ mediefiler osv.\ bliver selvfølgelig også relevante. Det lidt mere interessante bliver så, hvordan konventionen skal være for nøglerne, for her bliver det med godt med en fornuftig konvention som udgangspunkt --- selvom den dog kan udskiftes; det er jo altid bare at transformere nøglerne i databasen. 
%Godt nok skal man så også transformere nøgle-referencerne inden i teksterne i databasen, men konventionen for at give referencer inden i tekster bør også være en rigtig yderlig konvention %(i.e.\ hvis man nu har en HTML-tekst, der er encoded med UTF-8, så bør nøgle-reference-encoding'en ligge imellem de to, så hvis man vil f.eks.\ vil beskrive, hvordan man formaterer nøgle-referencerne i html-teksten, så skal man først undslippe/escape nøgle-reference-encoding'en). Herved bliver det så altid nemt at parse tekster for nøgle-referencer og ændre dem. Hm, kunne man så egentligt ikke bruge html-kommentar-indkodningen, så der aldrig kommer konflikter imellem de to (så man ikke absolut skal parse og transformere hver uploadede HTML-tekst fra en ekstern kilde)? (Og HTML kommer nemlig til at blive rigtig central for systemet, i hvert fald i starten.) Hm, nej det går ikke helt, for man må ikke have nestede kommentarer i XML\ldots\ Ah, men det gør jo ikke noget!\ldots Hm... Nå nej, hvad snakker jeg om, jeg vil jo gerne have, at referencerne kan behandles af JS..
%--- dog bør nøglerne stadig kunne behandles af JavaScript. Hm\ldots\ Tja, kunne man mon ikke bare gøre det til en del af syntaks-definitionen, som jeg jo alligevel har tænkt mig som noget ret grundlæggende/yderligt (jeg ved ikke helt, hvorfor jeg sætter `yderligt' = `grundlæggende' i hovedet, men det gør jeg altså)\ldots\ Jo, så konventionen kan blive at nøgle-referencerne kan implementeres i brugernes HTML/JavaScript efter konventioner, som de selv i princippet sætter, men...%Hm, jeg afbryder, ikke fordi det er en dårlig idé, men fordi jeg lige skal overveje en anden mulighed også. Jeg tænkte her at have en konvention, hvor brugerne har firhed til at sætte deres egne nøgle-reference-konventioner, hvis bare de lige erklære disse via en formel syntaks først. Og jeg forestiller mig nemlig, at syntaks-definitioner bliver et tæt-på-grundlæggende lag af systemet. Så var det så, jeg tænkte, om ikke det faktisk skulle gøres til en del af det helt grundlæggende lag, så database-systemet allerede i det grundlæggende lag har en protokol for at transformere nøgle-referencer.. Og nu kom jeg så på, at man måske i stedet også bare kunne starte med en omfattende konvention, og at brugerne så bare kan definere nye konventioner som en del af syntaksen, hvis de vil skifte... Hm, og så kunne databaserne altså selv lave deres egne maps til potentielt mere effektive konventioner?... Hm, tja i praksis vil en konvention om, at bruge et (utf-8-)tekst-eller-blob-flag, en længde-værdi og så et SHA-2-hash jo sikkert være ret holdbart, mon ikke?... Hm, og længde-værdien kunne bare være log_2 af længden (rundet op).. Og da flaget bare skal være en byte, så kan dette præfiks evt. også bruges til, hvis man vil have værdier, der selv signalerer, om de er en reference eller en numerisk værdi/karakterstreng.. Hm, og så er det bare lige, om man skal kunne udvide længde-værdien, med præciserende suffix-værdier, og hvordan man ændrer længden på hash-nøglen... ..Hm, eller hvis dette bare skal være en helt omfattende konvention først og fremmest, så skulle man måske bare give længde-værdien som en værdi, der repræsenterer den nummeriske længde direkte.. Hm, og samtidigt kunne man så have en præfiks på hashet, der fortæller algoritmen og længden.. Hm... Hm, og så kan alle reference-attributters størrelse så bare skjules for brugerne sammen med de faktiske nøgler, og databasen kan i princippet bare vægle at transformere alle nøgler inden de serveres til brugere (ikke nødvendigvis af sikkerhedsårsager eller noget, men bare for ikke at give nogen krav til databasen om, hvordan den skal gemme nøglerne).. Ja, og den kan jo nemlig også passende transformere til lokale nøgler, som jeg jo egentligt også i forvejen har tænkt, nok vil være en god idé.. ...Hm, i øvrigt er det da ikke en super dum idé, at scriptet, der håndtere database-queries'ne med det samme kan se, hvor længe teksterne er, inden de overhovedet query'er efter det.. ..Hm, og længden kan så også fortælle, hvor stor hash-længden skal være i den efterfølgende del af nøglen.. Hm, og da hash-kollisioner sker så sjældent, som de gør, så kunne man bare tilføje en konvention i det grundlæggende, om at alle database-noder med det samme kommunikere med hinanden, hvis en kollision findes, og så beslutter sig for en rækkefølge af de to kolliderende tekster/blobs, så man bare lige kan sætte det sidste suffiks i nøglen, som siger, hvilken én af de kolliderende tekster det er (og hvor der så i princippet godt kan være flere end to; så finder man bare en rækkefølge på, hvor mange det nu end er (hvilket gøres uden at ændre allerede vedtagne suffikser)).. Og til hver en tid vil de maskimale nøgle-længder så være en byte plus den maskimale længde for database-objekterne i log_256 plus den maksimale hash-længde i bytes plus det maksimale antal kollisioner fundet for samme hash-værdi log_256 (hvilket typisk vil være nul).. Hm, og så kan man endda bare droppe algoritme-flaget, for hvis SHA-2 bliver brudt i stor stil, så vil hele internettet og webbet alligevel skulle omstruktureres en hel del; meget mere end bare lige at transformere nogle nøgler i nogle tekster i en delt database. ..Og den skal virkeligt brydes fuldstændigt fra hinanden, før det kan skabe sikkerhedsproblemer for systemet, for faren bliver bare i praksis ellers, at man bare kan være uheldig i en kort periode, at få serveret en i praksis fuldstændig tilfældig tekst i stedet for den ønskede... Hm, tja, men der kunne måske være steder, hvor dette kunne give et sikkerhedshul, men faren er stadig virkelig, virkelig lille. Ja, så alt i alt: det behøver jeg helt klart ikke at tænke på; jeg kan bare sige at SHA-2 bruges i de globale nøgler. ... Jo, det er altsammen meget fint, men man vil så også egentligt gerne have lokale nøgler, som også kan bruges selv i kommunikationen med en lokal database-knude/-instans og så dens brugere.. ..Hm, skulle man så have offentlige global-lokal-nøgle-relationer, som dog kun database-instansen kontrollerer?.. ..Og så altså også gøre det til noget ret grundlæggende, det med at definere nøglerne i syntaksen.. Hm.. ...Eller kan man ikke egentligt bare bruge globale nøgler og så bare JS-biblioteker? ..Jo, det er da bare det man gør, er det ikke? (..Min hjerne har virkelig været langsom i dag..) Jo, så al det med "syntaks for at parse nøgler" osv., det ligger bare i (java-)script-laget, og kan f.eks. bare implementeres ved, at man definerer konstanter og funktioner, man kan bruge til at get'te de globale nøgler, uden at brugerne/programmørerne behøver at arbejde med nøgle-værdierne direkte. Og hvis en database-knude så vil konvertere om til mere effektive nøgler, jamen så kan denne jo bare oprette oversættelse til lokale nøgler. Cool. 
%(23.07.21) Jeg overvejer lidt omkring rettelser og datoer.. Det ville nemlig måske være meget smart, hvis man kan referere til en tekst med den rettelser til et givent tidspunkt (med et givent pointsæt).. På den anden side kunne man jo også bruge point-forskrifter til at bedømme den samlede tekst.. Hm, men så skal man jo følge de samme konventioner alle sammen.. hvilket dog bare kan være så simpelt som at propagere minimalværdien.. ..Ja, det er nok mere fornuftigt netop at bedømme ydre struktur og det individuelle indhold af tekser for sig (og efter nogle konventioner).. ..Hm, kunne man ikke finde en måde at gøre det usynligt, hvad der er seperate (atomare) tekster, og hvad der bare er parsede udsnit..?(!..) ..Hm, i JS bliver det måske lidt usynligt, hvis man (netop) benytter biblioteks-konstanter og -funktioner til at navigere med (som så kan implementeres som klasser med fordel, så man kan navigere ved at smække en række funktionskald bag på et iterator-objekt).. ..Jo, så man kunne altså overveje helt at usynliggøre, at der findes bare tekster uafhængigt af pointvalg og af datoen (eller rettere timestamp'et, men jeg har det med at skrive dato i stedet).. Hm.. ..Det lugter lidt af en god idé.. ..Ja, og så kan datoer for point jo muligvis bare samles via et nyt point, der holder øje med, om tekster har bevaret deres semantik og klarhed, så man kan aggregere pointgivelserne lidt på tværs af versioner... Hm, men hvad så med, når man gerne vil tilføje struktur til eksisterende tekster (og uddybe udsnits semantiske funktion i den overordnede tekst)..? ..Ah, men skal man ikke bare altid dele en tekst op i to; dens struktur og dens indhold?.. ..Ah, og "ydre struktur" kan så måske være helt flettet sammen med point-propagerings-forskriften..!.. Ja, det må da næsten være sådan..? ..Hm, og hvordan laver man så ændringer i den ydre struktur? ..Nå jo, denne vil jo indeholde prædikaterne og point-forslagene, så det bliver jo bare at ændre struktur-teksten (og så må folk altså bare bedømme (point-propagerings-)forskriften på ny).. Okay, men så tilbage til, hvordan man tilføjer struktur og rettelser til oprindeligt atomare tekster.. ..Tja, så skal man jo kunne give prædikater og point til udsnit, hvilket man kan, men skulle man måske gøre det mere.. usynligt ligesom, om man referere til en atomar tekst eller et udsnit, måske ved at indføre reference-entiteter.. Kunne også godt lugte hen af en god idé, men skal dette virkeligt være i det grundlæggende (eller i JS-laget)..? Hm, det kunne egentligt nok være ret nice, hvis man bare kunne gøre det i det grundlæggende lag.. ..Ah ja, det ville faktisk blive rigtigt nice, for så kan man endda gøre det helt usynligt, således at database-instanserne har frihed til at omstrukturere teksterne og smide tekster ud (inklusiv dele af tekster), uden at det gør noget.. Yes..! For ja, hvis man gemmer propositionerne og/eller pointene med reference til den originale tekst, og så bare selvfølgelig start- og slut-punkterne for det relevante udsnit, så bliver det jo noget rod, når man så begynder at slette det omkringliggende i den fulde tekst. ..Ah, så i databasens interface kan man udtrække udsnit fra tekster, som så bliver "kopieret" og får sin selvstændige reference (med den deterministiske globale nøgle, hvor hashet altså bare tages på udsnittet, og det hele), men i virkeligheden kan den egentlige database bare gemme den tilhørende lokale nøgle i form af en (lokal) reference (/nøgle) til den originale tekst samt start- og slut-punkt for udsnittet. Og så skal brugerne af dette interface altså aldrig tænke over lokale nøgler eller bekymre sig over den tilsyneladende ekstra plads, der skal til, når man "kopiere" udsnit ud som selvstændige tekster. Fedt. Nu har jeg så dog fjernet hovedgrunden til, hvorfor jeg rigtigt gerne ville have ubegrænset input i pointtyperne og til propositionerne, så måske jeg lige skal genoverveje dette?... 
%Okay, der skal helt klart være det her interface, som jeg snakkede om, og så skal der bare lige være en relation i den underlæggende database, der linker tekster med underafsnittende, hvilken gerne bare må være offentlig, for mening er så, at man nemt kan finde frem til underteksterne og så, om de er givet nogle relevante point for en. Denne funktionalitet kommer dog så nok primært til bare at blive brugt til at alarmere resten af fællesskabet om en fejl eller et vist andet forhold, men bliver ikke en del af.. nå jo, det bliver en del point-propageringen, men kun når tekst-udsnittet mathcer en indsat tekst i en struktur. Og ellers kan det altså også bruges til at signalere generelt og uden om strukturene og deres propageringsforskrifter. Jeg er så kommet frem til, at struktur-tekster bare skal implementeres som tekster, og altså ovenpå det grundlæggende lag, og nu er jeg ved at overveje, hvordan point-forskrifterne kan implementeres.. ..De bliver nok et eksempel på, at det er godt at kunne have vilkårlige point-(database-)relationer, men lad mig nu lige se.. 
%Okay, nu har jeg så fået den idé, om ikke man måske på en eller anden måde kunne gøre det til noget ret grundlæggende, at udsnit af struktur-tekster altid bedømmes i forhold til konteksten... nå jo, men det giver vel egentligt lidt sig selv..? Eller gør det, det gør det da ikke..? Nej, så måske kunne det altså være en idé.. Hm.. Tjo, men kunne dette så ikke implementeres ved faktisk at identificere start- og slut-punkterne i point-parametrene i stedet for at give selve udsnittet (og "kopiere" det i databasen så at sige) point (hvor der jo netop ikke vil være en kontekst, når teksten ses som en separat ting)..? ..Og så er det nemlig ikke meningen, at man skal rette ydre strukturer --- kun hvis strukturen selv er delt op i flere ("ydre") strukturer. Man skal i stedet bare udskifte dem, simpelthen (og de vil også være ret korte tekster). Ja, det må da bare være det..? Og så er tanken nemlig nu, at point-propagerings-halløjet kan implementeres ved at folk først stemmer om kategorier af undersektioner af en struktur --- og muligvis også om tilhørende parameterværdier til disse kategorier, hvis de bruger sådan nogen. Og så lægger man ellers simpelthen bare en forskrift ovenpå, som brugerne så kan vælge og kan justere selv for hver tekstklasse (og også bare efter behov for en pågældende søgning/browsing), som så tager visse point fra udsnittende alt efter, hvilken kategori de har fået, og så også udregner et nyt (eller flere nye) point til den overordnede tekst ud fra disse værdier alt efter, hvilke kategorier de kom fra. Nice, fed løsning! (Og nemlig stadig også rigtigt fedt med, at man bare kan udplukke tekster frit, og hvor de så behandles helt på lige fod med selvstændigt uploadede tekster (i hvert fald udadtil (efter det første interface))!)
%(26.07.21) Okay, jeg har tænkt lidt siden sidst (afsluttede sidste paragraf d. 23.), og to ting jeg lige hurtigt kan nævne er, at serverne også gerne må kunne oprette eller sammenkæde nye forældertekster til tekster, der oprindeligt kom fra en anden forældertekst, og så at der måske skal være specielle pointtyper, som serverene ved kan bestemmes automatisk, så de dermed ved, at de ikke nødvendigvis behøver at gemme resultatet. Og derudover har jeg tænkt en del i baner af, hvad man skal gøre ift. databasen, som jeg jo gerne har ville have, skulle være delt. Det vil jeg i øvrigt stadigvæk. Men spørgsmålet er altså ikke helt så trivielt, som jeg ellers lidt har tænkt; og nu har jeg en idé om at bruge mine point til at opdele databasen, som måske ikke er helt dum i sig selv. En tilhørende idé til denne er så, at brugerne faktisk selv holder styr på, hos hvilke servere er hvilke del-ontologier, ved at de forventes at foreslå et godt server-til-server-link, når de oploader et relationship mellem to termer, der ikke holdes af samme server. Så brugerne uploader altså (gerne mere eller mindre automatisk) selv et link til den server, hvor man bør kunne finde det eksterne term. Og når ansvaret således (i teorien) kommer til at ligge hos brugerne, jamen så er det enste serveren selv skal gøre pludeselig bare at holde styr på sin egen del-ontologi, samt lige at kommunikere med andre servere om at back'e dataen op, selvfølgelig imod at serveren selv back'er deres data op til gengæld. Så hvis bare brugere kan komme til at følge URL'er for at forbinde til andre servere, så har vi jo allerede et rigtigt godt system!.. Og her søgte jeg så lidt i går aftes på nogle muligheder, og det virker som om Tor/onion-netværket og Tor-browseren kunne blive en rigtig nem måde at implementere det på. Og så får man endda et meget tillidsværdigt krypteringssystem med gratis, så brugere og servere hurtigt kan ende med at kunne blive ligeså anonyme som de ønsker det (som man kan slippe for censur og sikre privathed). Og man behøver ingengang at bruge Tor-browseren som bruger, for man kan helt sikekrt ret nemt lave et localhost-program, der bare formidler indholdet videre til brugeren yndlingsbrowser. Og man kan endda lave en slags firewall (nærmest) i dette program, som kun tillader godkendte .onion-addresser at blive formidlet videre til og besøgt af Tor-browser-programmet, så man herved kun holder sin aktivitet til godkente servere, som er en del af (sem-web-)fællesskabet. Og det er nemlig virkeligt fantastisk, at man bare kan oprette servere kvit-og-frit, og som browsere så bare kan navigere til via URL-er.(!) Idet brugerens browser så er forbundet til localhost, vil URL'erne som brugeren ser bare have .onion-adresserne sat på som postet data (efter `\#'-tegnet), men det er jo helt fint. Og det er så lidt interessant, at min løsning måske kommer til at blive en løsning på noget, andre har søgt at gøre, for folk har nemlig tilsyneladende (hvilket er klart nok) tænkt i at lave semantiske og delte databaser, hvor man kan benytte de naturlige opdelinger i den samlede ontologi til at lade serverne i et P2P-netværk (nemlig et "SP2P"-netværk, hvor 'S' står for 'semantic') bestemme selv over, hvilke del-ontologier de vil varetage. Og her kan min idé altså muligvis være lidt interessant generelt, men det behøver jeg til gengæld ikke at fokusere særligt meget på; jeg kan bare præsentere idéen og nævne, at det er en form for SP2P-netværk.  
\ldots\ Okay, jeg har tænkt lidt yderligere yderligere over nogle ting. %Nøglerne skal bare være omfattende, og så kan den enkelte databaseinstans altid bare konvertere til lokale nøgler, som så kan blive mere effektive. Jeg tænker, at de globale nøgler (som er dem brugerne ser) indeholder et byte-flag for, om der et tale om en proposition, en ...
%Hm, hvad med mine contructors, som jeg tænkte på for noget tid siden..? ..Ah, men der er jo faktisk ingen grund til at begrænse "propositioner;" jeg kan bare udvide det til vilkårlige HOL-typer..!.. Hm så propositioner bliver bare en slag "closure" i stedet.. ..Og ja, man kan også lave sine closures i form af tekster, men det kan netop være smart, hvis man også kan lave dem som database-relations-entiteter, så man herved har mulighed for at bruge de computationelle kræfter, der følger med dette. Ja, og så kan jeg jo bare foreslå en syntaks for at danne tilsvarende closures i teksterne, som altså har den samme semantiske betydning, men som passende altså bare kan laves som tekster, hvis man endnu ikke har en algoritme i tankerne, hvor det er smart at inkludere dem som db-entiteter. Uh ja, og når man så pludselig finder på en algoritme, så skulle man jo så gerne bare kunne be databasen om at løfte alle closures'ne af den pågældende type op, så de bliver entiteter.
Der bliver også nogle rettelser til det ovenstående ting, så lad mig lige lave en ny paragraf og få styr på de ting, og så kan jeg vende tilbage til atomare termer og nøgler efterfølgende.

For det første skal propositions-relationerne ikke kun kunne konstruere propositioner, som jeg ellers har lagt op til, men de skal også bare kunne konstruere ting generelt af vilkårlige semantiske typer. Så vi kan se entiteterne i relationen som generelle `closures' (hvor dette term er taget fra Lisp).  %Dog forestiller jeg mig nu også, at der også bør være en grundlæggende syntaks for at opbygge sådanne closures i UTF-8-teksterne. Men ved at brugerne også kan bygge dem som entiteter i databasen, kan de få mulighed for at udnytte databasens effektivitet, når det kommer til relationerne, hvis man altså gerne vil have sine closures til at indgå i algoritmer. Og grunden til at have en grundlæggende syntaks til at opbygge tekst-closures er så, at brugere så bare kan starte med at implementere deres closures i form af tekster, og så hvis de finder ud af, at det kunne være smart at have dem som database-entiteter i en relation i stedet, så kan de så bare be databasen om at løfte tekst-closures'ne op, så... %Hm, men vil det ikke bare være nemmere altid bare at gemme closures som db-entiteter..?  ...Hm, det skulle da aldrig blive sådan, at man så også implementerer strukturene..??... ..Hm, og hvis man nu gjorde det, kunne man jo altid bare have et flag, som viser om sidste reference er slut-termet, eller om det er en reference videre til en fortsættelse på parameterlisten... ..Det virker da egentligt totalt oplagt, nu hvor jeg tænker over det.. (Jeg troede ellers ikke helt på det, da jeg skrev indskydelsen, men jo, det giver da super god mening..) He. ..Hm, og struturer, som jeg tænker dem, vil så altså blive en liste af prædikater samt en struktur-constructor i starten.. ..Ja; totalt oplagt.. 
Ikke-propositionelle closures kommer så faktisk til at blive ret centrale, som jeg ser det, for jeg forestiller mig nu, at de kommer til at implementere, hvad jeg har kaldt `tekst-struktur' i mine (udkommenterede) brainstorm-noter, hvilket simpelthen vil sige, at de kommer til at implementere tekster, eller rettere en mængde af tekster, hvor hvert afsnit så refereres til i closure-entiteten, men ikke direkte. I stedet optræder en reference til en prædikat-typet closure (altså med den implicitte type af `prædikat,' for der er nemlig ikke faktiske typer i systemet, når det kommer til closures'ne), som fortæller, hvad tekstafsnittet gerne skal indeholde, og hvad det også i øvrigt skal opfylde generelt --- og hvor hver enkle bruger så kan få indflydelse over valget af den resulterende tekst til afsnittet via point-valgene. Dette er i øvrigt absolut et af nøglepunkterne ved hele idéen --- lige nu er jeg jo bare ved at bygge fundamentet op, og jo, min idé til pointene er nok ret vigtig, men det rigtigt vigtige kommer først, når jeg når til nogle af mine forestillinger om, hvad man skal bruge fundamentet til. Så hæng i! For ja, der går lige lidt, før vi når dertil.

Det skal ikke længere være sådan, at brugerne skal anmode om, at databasen skaber nye relationer med plads nok til visse closures, hvilket jeg nu også har nævnt i en indskudt bemærkning ovenfor. I stedet skal det for brugernes synspunkt bare se ud som, at databasen er klar til at oprette vilkårlige relationer. Og ydermere skal det endda se ud som, at databasen simpelthen bare kopiere hver enkle closure-entitet ind i alle de mindre closure-relationer, bare hvor halen af closure-entiteten skæres over (så bemærk i øvrigt også at inputtet til closures'ne altid er ordnet). Så skal der bare være en fast attribut, som beskriver closure-længden, og som så altså \emph{ikke} ændres, når en closure skæres op for at kopieres ind i de mindre relationer. Alt dette er dog kun, hvordan tingene ser ud fra databasens interface af. Inde bag dette interface kan databasen så implementere de lange closures via linked lists, hvis den vil, og den kan også sørge for at længde-attributterne sættes automatisk i selve queries'ne og altså ikke skal gemmes fysisk nødvendigvis. Og selvfølgelig kan databasen så hermed også undlade rent faktisk at kopiere alle closure-entiterne ind i de mindre relationer, men kan bare implementere dette som en automatisk del af de underliggende queries også. Så brugerne har altså et interface, hvor databasen ser ud på én måde, men i virkeligheden kan et program til oversætte alle queries, så det passer til den underliggende database under interfacet. 

Noget andet databasen kan gøre under sit interface, er at benytte lokale nøglereferencer i stedet for de ``globale,'' som brugerne ser og benytter til interfacet. Når jeg kommer til nøglerne, vil jeg nemlig foreslå, at disse konstrueres via (SHA-2)-hashes, hvilket bliver smart, især når databasen skal deles ud på mange knuder/shards, men det gør dog, at beslægtet data kommer til at lægge arbitrært og tilfældigt langt væk fra hinanden. Dette kan dog gøres bod på, hvis den underliggende database (under interfacet) oversætter de globale nøgler til lokale via en look-up-tabel. Og her kommer også en mulig pointe i for databasen at implementere de underlæggende closure-relationer via linked lists, for så kan beslægtede closures (hvilket f.eks.\ en closure har andre closures som børn) endda også komme til at lægge tæt sammen på harddiskene, hvis man ønsker dette, også selvom de har forskellige længder. 


De globale nøgler skal så bare være omfattende --- man behøver ikke at spare på længden, for den underliggende database kan altid selv bare oversætte dem til lokale nøgler, hvis der skal spares (eller af andre grunde). %, og så kan den enkelte databaseinstans altid bare konvertere til lokale nøgler, som så kan blive mere effektive. 
Jeg forestiller mig, at de for det første kan indeholde et byte-flag for, om der et tale om en closure, en UTF-8-tekst, en BLOB, en pointtype, en point-definition, et point (i.e.\ en specifik pointgivning) eller et certifikat (som vi ikke har snakket om endnu) --- eller hvad end, jeg enten ikke lige kan huske nu, eller som jeg bare ikke har fundet på endnu. *(Ah jo, der er også lige moderator-listerne\ldots) I øvrigt kunne samme flag endda også repræsentere for en vilkårlig entitet-attribut, om der er tale reference i første omgang, eller om der i stedet er tale om en direkte værdi, så som et tal, hvilket jo blandt andet bliver brugbart for pointene. Ja, så dette byte-flag kan altså høre med til alle term-attributter, og for nøgle-referencerne tages flaget så bare med som en del af nøglen. Desuden skal nøglen, som jeg forestiller mig det, indeholde en længde, hvilket enten kan referere til en closure-/point-entitets-længde eller til tekst-/blob-længden. Og sidst men ikke mindst skal der altså også være et hash (SHA-2 tænker jeg bare) af det pågældende objekt. 

Angående det, jeg kaldte ``bruger-id/pointgiver''-attributten, så forestiller jeg mig nu, at dette faktisk skal være et hash af, hvad bliver endnu en term-type, nemlig `certifikater.' Da det er meningen, at disse certifikater bl.a.\ skal kunne indeholde et hash af det pågældende objekt (f.eks.\ i form af dets nøgle), så skal certifikat-hashet altså dermed undtages fra at indgå i point-entiteternes nøgler. Hermed kommer samme point-entitet altså til at kunne underskrives af flere forskellige brugere. Certifikat-hashet skal også være nøglen til certifikat-entiteten, så det både kommer til at fungere som en underskrift og en nøgle til, hvad der underskives på samtidigt. \ldots Hm, nu bør jeg så faktisk også lige genoverveje, om man overhovedet skal adskille point og closures, og/eller om man skal adskille underskrivningen i sin egen relation, som så muligvis bare bliver mellem termer generelt og så certifikat-hashes(/-nøgler)\ldots? \ldots Ja, de to ting skal stadig være adskilte, og pointene skal være ret meget ligesom, jeg har lagt op til. Og ja, det giver også bedst mening, hvis certifikaterne kun linkes til pointene. Men der er nu ingenting i vejen med at adskille det, så der bliver en certifikatnøgle-pointnøgle-relation, nu hvor man også alligevel kommer til at kunne have det sådan, at den samme pointnøgle, fordi denne ikke må være afhængig af certifikat-hashet, at samme pointnøgle kan underskrives med flere forskellige certifikater. Så ja, lad mig endelig adskille dette. \ldots Hm, men så bliver pointene jo dog i bund og grund bare prædikater, da de jo nu ikke længere har et bruger-id tilknyttet\ldots\ Hm\ldots\ Hey, og kræver min idé omkring systemet med moderator-listerne, som jeg beskrev ovenfor, ikke også for meget tillid til databaseknuderne?! Det kan jo sagtens være, at man gerne vil have en delt database, hvor man helst vil stole mest på certifikaterne frem for selve serveren (hvad jeg faktisk selv tænker meget på at lægge op til), og så duer det da ikke, med det beskrevne system?\ldots\ Hm, jeg tænkte jo, at det skulle munde ud i en moderator-niveau-attribut, men ja, så skal man jo netop have god tillid til den pågældende server\ldots\ %Hm, så kan man finde en god måde, hvor brugerne nemt selv kan udregne, hvad der svarer til disse moderator-(godkendelses-)niveauer..? ..Ah jo, det kunne man vel godt. Eller rettere serveren kan bare selv opgive moderator-niveauet osv., men samtidigt bare sørge for, at der også medgives det pågældende certifikat, der medfører det pågældende nivuea. Ja, det må kunne du.. Ja, simpelthen. Nice.
Ah, nu har jeg det vist. Man kan sørge for at serveren også altid lige giver det pågældende certifikat med, der medførste det pågældende moderator-niveau. Så kan klienten bare tjekke dette selv. På en måde kan alt dette så også oversættes til, at brugerne får databasen til at gruppere visse certifikat-kilder, så de kan lave søgninger efter certifikaterne ud fra et gruppe-tag, som så i vores tilfælde, hvis man beholder systemet sådan, altså vil være en numerisk værdi, der repræsenterer et sikkerheds-niveau basalt set (i.e.\ et ``moderator-niveau,'' som vi har kaldt det her). Så et naturligt spørgsmål vil jo så lige være, om man ikke kan implementere denne generaliserede funktionalitet på en mere simpel måde?\ldots\ (Indskudt: Hov, og måske bør jeg i øvrigt stoppe med at kalde det closures, for Lisp-closures har jo (så vidt jeg husker) altid kun tre elementer. Det kan jeg altså lige tænke over, og også tænke over, hvad jeg ellers kunne kalde det. Men indtil jeg finder på noget andet, fortsætter jeg nok bare med at kalde det `closures.') %Hm, det ville jo nok kræve noget lidt tilsvarende til det, jeg har tænkt, hvor pointtype-uploaderen altså er den, der sætter moderator-hierarkiet i gang.. ..Hm, kunne man mon så ikke på en måde bare gøre dette til forskellen mellem point og closure-prædikater, nemlig om de er modererede eller ej..? ... Hm, for især hvis man ikke skal stole på serverene, så bliver der jo ikke rigtigt nogen ikke-modererede point i praksis?.. ..Og så kunne jeg måske fint sige, at alle closures kan underskrives med certifikater, og så kan certifikaterne måske bare indeholde selve point-informationen, og hvor forskellen så netop ikke bliver mellem... hm.. vent så certifikaterne kommer til at kunne erstatte point til en vis grad.. Hm, kan det komme til at give mening på en god måde..? ..Man skal dog gerne kunne søge på pointtypen.. Hm, men det kan så bare være en del af "certifikatet" (så altså at point-entiteter nu kommer til at indeholde sine egne certifikater, og at referencerne så fungerer som jeg har beskrevet det for "certifikaterne" (før jeg blandede dem sammen med point)).. ..Hm, i virkeligheden kunne man også implementere point i applikationslaget, så at sige, ved at lave closure-verber, der siger "bruger x siger, at der gælder p om a1, a2, a3, ..., og giver certifikat c med for at andre kan tjekke dette udsagn.".. Hm, ja så udsagn i databasen kunne altså få en sandhedsværdi ved også at snakke om et selv-refererende certifikat i princippet (hvor sandhedsværdien så kan føres videre til børnene) som en mulighed.. Hm, det virker dog lidt spild, hvis alle point entiteter skal gemmes med en ekstra closure-funktion-reference for at signalere, at de er point.. Hm.. Tja, men der kunne man nok godt bare sige, at det er databasens ansvar (enten sofwarens eller ejerne, hvis det simpelthen kræver specialiseret programmering at sørge for), hvis der kan spares plads ved at oprette særlige relationer, hvor en ganganger-attribut bliver faktoriseret ud. Så det behøver jeg nok ikke at bekymre mig så meget om.. For ja, det kan jo altid bare indføres i laget lige under database-interfacet ligesom for de lokale nøgler osv., hvis det endelig skal være det. ..Og ellers kan man jo også altid opdatere interfacet.. hvilket sikkert ikke bliver et vildt stort problem, for der kommer sikkert alligevel til at blive en API lige over dette interface, før vi når til de faktiske applikationer.. (Ja, helt klart, så brugerne ikke skal kode i SQL for de basale ting --- inklusiv tjek af point/certifikater osv.) Hm, ja så det er ikke alt eller intet, men det er jo også klart nok.. ... Og så kunne man måske oprette "certifikat-grupper" som særlige entiteter, hvor databasen så kan vælge at gå med til dem, hvorved databasens så får til ansvar at sætte og justere moderator-niveauerne pr. request.. Og dette gøres så bare i en seperat relation, hvor de pågældende termer relateres til moderator-niveauer samt certifikater, der beviser niveauet.. Hm, der kunne endda også føjes et prædikat til, så at serveren dermed kan (eller rettere dette er så mere oplagt) vælge at have ansvaret for moderator-gruppens bidrag kun særligt når det kommer til nogen prædikater. Dette prædikat kunne jo også bare være en del af termet, men det gør det måske lidt mere besværligt for serveren at se, om moderatorerne har rettigheder til at bedømme termerne.. Tja, eller på den anden side går det vil ca. ud på ét... Hm, jeg vil lige tænke lidt mere over det her i aften og så i morgen. Det kan i princippet være, at det bare er det med denne her moderator-niveau-relation, som databasen kan tage ansvar for, og at resten bare skal implementeres i laget lige over databaseinterfacet, men lad mig nu se (for der er sikkert flere muligheder)..
\ldots%...

%(30.07.21) Okay, jeg tror jeg har rimeligt godt styr på det nu. Databaser kan bruge dictionary compression og, endnu mere vigtigt, indexes til at comprimere og effektivisere ting, så jeg kan godt sagtens bygge det op på et grundlæggende lag af HOL (med mulighed for overlodede funktioner --- hvilket f.eks. bruges til tekststrukturerne, som jo kan have et variabelt antal sektioner i sig). Og de vigtigste typer, som skal bruges til mit system er så tekst-prædikater, komposit-tekster, atomare tekster og point, som indeholder certifikat-id-reference (som også gerne skal holde et ('underforstået') prædikat i sig (når de de-adresseres)), certifikat-underskrift og pointværdi(er). Jeg tror ikke, det bliver relavant at kopiere længere "closures" ind i mindre closure-relationer, for det er meningen at man søger via propositions-entiteterne, hvor closure-længden altid vil fremgå i inputtes nøgle.. Hm, men propositioner.. kan de være af variabel længde?.. eller skal prædikater altid være unære..? Ja, tekst-prædikater skal altid være unære, så ja, det går. Prædikater kan dog godt bygges som komposite termer, så man f.eks. kan lave prædikater med f.eks. niveuaer af requirements, og som i sidste ende kan indeholde noget så blødt som mine 'point-forslag.' Databasen skal i øvrigt have en relation mellem nøgler, som ikke findes i selve database knuden/shard'en, og så 'emner'/tags skrevet i plaintekst, og som serveren så faktsik bruger til at finde frem til, hvilken server klienten skal sendes videre til, hvis denne vil følge de pågældende link. Det er brugeres eget ansvar i princippet at foreslå gode 'emner,' når referencen ikke matcher nogen intern reference, men serveren kan selvfølgelig assistere, da den jo også selv bør holde en privat relation over, hvilke 'emner' andre servere i netværket slår sig op på. Og servere slår sig nemlig så bare op på nogle emner samt en mængde af point (måske med en vis vægtning), som kommer til at afgøre, om serveren vil holde på et term eller ej, og hvor længe for den sags skyld. ..Og det er faktisk mere eller mindre det.. Og man kan så, som jeg også har nævnt med fordel gøre brug af Tor-netværket eller noget tilsvarende, så enhver nemt kan oprette servere, som specialiserer sig i et emne, som vedkomne er interesseret i (f.eks. en af deres GitHub-mapper). Og hvis jeg ikke har nævnt det, så kunne det også i øvrigt være en god ting som noget af det første, at arbejde på at lave et program, så folk kan parse deres Git-mapper og konvertere det direkte til en database graf (og hvor man så efterfølgende kan godkende eller ændre forslagene til, hvor kommentarerne hører til osv., men hvor meget af arbejdet altså kan ske automatisk). Angående syntakser for at indsætte tekster fra database-netværket til diverse web 2.0-sider, så kommer dette bare til at være adskilt fra hvordan strukturene fungere. Så det bliver altså bare noget, man bygger ovenpå, og altså hvor man beslutter en syntaks samt laver en hmtl-tekst, der kan fungere som en browser udvidelse til opgaven. Så tænker jeg, at brugere kan navigere til denne html-side, åbne deres web 2.0-side i en iframe, og så kan et script i førstnævnte, ydre html-side stå for, at indsætte, hvor der er referencer (og når det kan verificeres, at referencerne er sikre --- og overholder brugerens pointrestriktioner generelt). De atomare tekster i databasen skal som sagt ind bag et interface, så man altid kan udtrække udsnit fra gamle tekster, så de (tilsyneladende) oprettes som selvstændige tekster i databasen. Databasen holder så en offentlig relation over forælder- og børne-tekster, hvor den i øvrigt også (pr. request) kan matche børne-tekster med forælder-tekster, også selv hvis barne-teksten ikke oprindeligt kom fra forælderteksten. Og hvordan databasen gemmer teksterne under interfacet, samt hvilke transformationer den laver, det styrer den altså selv.. Ah, og folk bør jo også kunne sammensætte tekster til nye, længere tekster, hvor databasen altså heller ikke i udgangspunket behøver at gemme de nye tekster (men kan oprette nogle virtuelle tekster, der bare referere til de gamle tekster), men kan altså også på sigt, hvis disse sammensatte tekster bliver meget brugt, transformere over til, at disse bliver de materialiserede tekster i databasenm og måske samtidigt gøre, at de gamle tekster bliver de virtuelle. Hvis folk så gerne hurtigt vil fremhæve et tekstudsnit i en tekst, der endnu ikke har fået tilføjet struktur men bare er atomar, så kan de så gøre brug af forælder-barne-tekst-relationen, fordi de så kan trække udsnittet ud som en selvstændig barnetekst og give den et point direkte, og når browseren viser atomare tekster, kan den så bare lige se, om der skulle være nogle barne-tekster til den, og om disse så har fået nogen "alarm-point," som man måske kunne kalde dem. Jeg tror, at dette bliver smart især for nye tekster, som måske gerne skal have lidt tid at blive rettet i, før man begynder at bruge energi på at tilføje struktur til dem --- så man altså lige kan få taget de værste fejl på en hurtigere og nemmere måde. ..Nå jo, og alle de nævnte typer så som (tekst-)prædikater og komposit-tekster (hvilket altså vil sige mine tekst-struktur-"closures") må vist gerne fremgå af terments nøgle. Det tror jeg på, vil være en god idé. Og det vil så være underforstået, hvad der menes, når komposit-tekster har prædikater eller ikke-prædikater (så som andre komposit-tekster eller atomare tekster) som input. I førstnævnte tilfælde er meningen, at man skal søge på et term, der opfylder prædikatet, og i de to sidste tilfælde (som ikke er ligeså recommented, men som stadigvæk vil være ret praktisk at have mulighed for), vil det bare være underforstået, at prædikatet er: 'Er' (som i: "Er atomare teskt x" eller "Er kompositte teskt y"). ..Og det bliver altså i bund og grund det.. Og nøglen i hele idéen er så især det her med "variable tekster, hvor sektioner kan udskiftes løbende og indsættes så de passer til brugerens specifikke behov," hvilket så også hænger sammen med "fri og brugerdrevne pointsystemer," hvilket videre også hænger sammen med (eller 'resulterer i,' kan man måske også sige) "frie og brugerdrevne algoritmer til at søge på og filtere indhold, så det passer bedst muligt til ens præferencer." Så her har vi altså nogle nøglepunkter, og "brugerklassifikation og ML-teknikker (begge brugerdrevet) som led i at danne et alsidige og langt mere effektive pointsystemer, end hvad man har set før" bliver så også nogle vigtige elementer af den smalede idé. Mit eksempel på et simpelt point-system til at vurdere korrekthed og uddybelsesgrad i argumenterende tekster m.m. er også i øvrigt ret vigtigt, for jeg tror virkeligt at man kan komme langt, selv med dette simple pointsystem (og det vil helt sikkert blive en central del af alle fremtidige vidensdelings-sider/-applikationer/-netværk). Og mine versioner for, hvordan systemet/netværket kan forbedre Git (og Stack Overflow), og hvordan det også kan forbedre web 2.0-sider så som bl.a. sociale medier, tror jeg også begge er ret vigtige at forklare. I øvrigt kan folk måske opslå deres interesse-'emner' i deres biografi-tekst, hvilket bl.a. så måske kan gøre, at man kan forkorte hashes'ne i sine links (i sine opslag), især hvis en dato altid også kan parses fra opslaget. For hvis interesseemne-nøgleordene og datoen for et oplag altid kan parses, så vil det nok sjældent kræve særligt meget af hashet, før database-termet kan identificeres. Nå ja, og ellers kunne man også bruge en konvention, hvor man bare svarer sig selv med link-endelserne, så hele links'ne kommer med, men samtidigt uden at gøre den egentlige tekst ulæselig (eller svært læselig, rettere) for mennesker. ..Ah, eller vedkommene kunne også bare oploade point til pågældende termer, som er beregnet til at gøre, så der effektivt set bliver en relation (som altså kan udledes) i databasen mellem et bruger-id (i.e. en offentlig krypteringsnøgle) og så de termer, denne bruger har linket til. Ja, det var egentligt meget smart; så kan man altså herved ret nemt få det, så man kan indsætte korte links i sine web 2.0-opslag. 
%Men ja, "et semantisk netværk, hvor tekster er dynamiske på den måde, at de i stedet for f.eks. atomare tekster bare har prædikater, hvor det så er meningen, at den bedst mulige tekst, der opfylder prædikatet på det pågældende tidspunkt (og også ud fra brugerens eget valg af point-parametre), kan indsættes" er et meget centralt punkt ved idéen.  



\ \\\\
Lige indtil jeg får det skrevet bedre, så er her mine udkommenterede brainstorm-noter over, hvordan det bliver, som jeg tænker det nu: *(Jeg tror ikke, jeg vil skrive det om.)\\\\
{\slshape
[...] Så et naturligt spørgsmål vil jo så lige være, om man ikke kan implementere denne generaliserede funktionalitet på en mere simpel måde?\ldots\ (Indskudt: Hov, og måske bør jeg i øvrigt stoppe med at kalde det closures, for Lisp-closures har jo (så vidt jeg husker) altid kun tre elementer. Det kan jeg altså lige tænke over, og også tænke over, hvad jeg ellers kunne kalde det. Men indtil jeg finder på noget andet, fortsætter jeg nok bare med at kalde det `closures.') \%Hm, det ville jo nok kræve noget lidt tilsvarende til det, jeg har tænkt, hvor pointtype-uploaderen altså er den, der sætter moderator-hierarkiet i gang.. ..Hm, kunne man mon så ikke på en måde bare gøre dette til forskellen mellem point og closure-prædikater, nemlig om de er modererede eller ej..? ... Hm, for især hvis man ikke skal stole på serverene, så bliver der jo ikke rigtigt nogen ikke-modererede point i praksis?.. ..Og så kunne jeg måske fint sige, at alle closures kan underskrives med certifikater, og så kan certifikaterne måske bare indeholde selve point-informationen, og hvor forskellen så netop ikke bliver mellem... hm.. vent så certifikaterne kommer til at kunne erstatte point til en vis grad.. Hm, kan det komme til at give mening på en god måde..? ..Man skal dog gerne kunne søge på pointtypen.. Hm, men det kan så bare være en del af "certifikatet" (så altså at point-entiteter nu kommer til at indeholde sine egne certifikater, og at referencerne så fungerer som jeg har beskrevet det for "certifikaterne" (før jeg blandede dem sammen med point)).. ..Hm, i virkeligheden kunne man også implementere point i applikationslaget, så at sige, ved at lave closure-verber, der siger "bruger x siger, at der gælder p om a1, a2, a3, ..., og giver certifikat c med for at andre kan tjekke dette udsagn.".. Hm, ja så udsagn i databasen kunne altså få en sandhedsværdi ved også at snakke om et selv-refererende certifikat i princippet (hvor sandhedsværdien så kan føres videre til børnene) som en mulighed.. Hm, det virker dog lidt spild, hvis alle point entiteter skal gemmes med en ekstra closure-funktion-reference for at signalere, at de er point.. Hm.. Tja, men der kunne man nok godt bare sige, at det er databasens ansvar (enten sofwarens eller ejerne, hvis det simpelthen kræver specialiseret programmering at sørge for), hvis der kan spares plads ved at oprette særlige relationer, hvor en ganganger-attribut bliver faktoriseret ud. Så det behøver jeg nok ikke at bekymre mig så meget om.. For ja, det kan jo altid bare indføres i laget lige under database-interfacet ligesom for de lokale nøgler osv., hvis det endelig skal være det. ..Og ellers kan man jo også altid opdatere interfacet.. hvilket sikkert ikke bliver et vildt stort problem, for der kommer sikkert alligevel til at blive en API lige over dette interface, før vi når til de faktiske applikationer.. (Ja, helt klart, så brugerne ikke skal kode i SQL for de basale ting --- inklusiv tjek af point/certifikater osv.) Hm, ja så det er ikke alt eller intet, men det er jo også klart nok.. ... Og så kunne man måske oprette "certifikat-grupper" som særlige entiteter, hvor databasen så kan vælge at gå med til dem, hvorved databasens så får til ansvar at sætte og justere moderator-niveauerne pr. request.. Og dette gøres så bare i en seperat relation, hvor de pågældende termer relateres til moderator-niveauer samt certifikater, der beviser niveauet.. Hm, der kunne endda også føjes et prædikat til, så at serveren dermed kan (eller rettere dette er så mere oplagt) vælge at have ansvaret for moderator-gruppens bidrag kun særligt når det kommer til nogen prædikater. Dette prædikat kunne jo også bare være en del af termet, men det gør det måske lidt mere besværligt for serveren at se, om moderatorerne har rettigheder til at bedømme termerne.. Tja, eller på den anden side går det vil ca. ud på ét... Hm, jeg vil lige tænke lidt mere over det her i aften og så i morgen. Det kan i princippet være, at det bare er det med denne her moderator-niveau-relation, som databasen kan tage ansvar for, og at resten bare skal implementeres i laget lige over databaseinterfacet, men lad mig nu se (for der er sikkert flere muligheder)..\\
...
\\\\
\%(30.07.21) Okay, jeg tror jeg har rimeligt godt styr på det nu. Databaser kan bruge dictionary compression og, endnu mere vigtigt, indexes til at comprimere og effektivisere ting, så jeg kan godt sagtens bygge det op på et grundlæggende lag af HOL (med mulighed for overlodede funktioner --- hvilket f.eks. bruges til tekststrukturerne, som jo kan have et variabelt antal sektioner i sig). Og de vigtigste typer, som skal bruges til mit system er så tekst-prædikater, komposit-tekster, atomare tekster og point, som indeholder certifikat-id-reference (som også gerne skal holde et ('underforstået') prædikat i sig (når de de-adresseres)), certifikat-underskrift og pointværdi(er). Jeg tror ikke, det bliver relavant at kopiere længere "closures" ind i mindre closure-relationer, for det er meningen at man søger via propositions-entiteterne, hvor closure-længden altid vil fremgå i inputtes nøgle.. Hm, men propositioner.. kan de være af variabel længde?.. eller skal prædikater altid være unære..? Ja, tekst-prædikater skal altid være unære, så ja, det går. Prædikater kan dog godt bygges som komposite termer, så man f.eks. kan lave prædikater med f.eks. niveuaer af requirements, og som i sidste ende kan indeholde noget så blødt som mine 'point-forslag.' Databasen skal i øvrigt have en relation mellem nøgler, som ikke findes i selve database knuden/shard'en, og så 'emner'/tags skrevet i plaintekst, og som serveren så faktsik bruger til at finde frem til, hvilken server klienten skal sendes videre til, hvis denne vil følge de pågældende link. Det er brugeres eget ansvar i princippet at foreslå gode 'emner,' når referencen ikke matcher nogen intern reference, men serveren kan selvfølgelig assistere, da den jo også selv bør holde en privat relation over, hvilke 'emner' andre servere i netværket slår sig op på. Og servere slår sig nemlig så bare op på nogle emner samt en mængde af point (måske med en vis vægtning), som kommer til at afgøre, om serveren vil holde på et term eller ej, og hvor længe for den sags skyld. ..Og det er faktisk mere eller mindre det.. Og man kan så, som jeg også har nævnt med fordel gøre brug af Tor-netværket eller noget tilsvarende, så enhver nemt kan oprette servere, som specialiserer sig i et emne, som vedkomne er interesseret i (f.eks. en af deres GitHub-mapper). Og hvis jeg ikke har nævnt det, så kunne det også i øvrigt være en god ting som noget af det første, at arbejde på at lave et program, så folk kan parse deres Git-mapper og konvertere det direkte til en database graf (og hvor man så efterfølgende kan godkende eller ændre forslagene til, hvor kommentarerne hører til osv., men hvor meget af arbejdet altså kan ske automatisk). Angående syntakser for at indsætte tekster fra database-netværket til diverse web 2.0-sider, så kommer dette bare til at være adskilt fra hvordan strukturene fungere. Så det bliver altså bare noget, man bygger ovenpå, og altså hvor man beslutter en syntaks samt laver en hmtl-tekst, der kan fungere som en browser udvidelse til opgaven. Så tænker jeg, at brugere kan navigere til denne html-side, åbne deres web 2.0-side i en iframe, og så kan et script i førstnævnte, ydre html-side stå for, at indsætte, hvor der er referencer (og når det kan verificeres, at referencerne er sikre --- og overholder brugerens pointrestriktioner generelt). De atomare tekster i databasen skal som sagt ind bag et interface, så man altid kan udtrække udsnit fra gamle tekster, så de (tilsyneladende) oprettes som selvstændige tekster i databasen. Databasen holder så en offentlig relation over forælder- og børne-tekster, hvor den i øvrigt også (pr. request) kan matche børne-tekster med forælder-tekster, også selv hvis barne-teksten ikke oprindeligt kom fra forælderteksten. Og hvordan databasen gemmer teksterne under interfacet, samt hvilke transformationer den laver, det styrer den altså selv.. Ah, og folk bør jo også kunne sammensætte tekster til nye, længere tekster, hvor databasen altså heller ikke i udgangspunket behøver at gemme de nye tekster (men kan oprette nogle virtuelle tekster, der bare referere til de gamle tekster), men kan altså også på sigt, hvis disse sammensatte tekster bliver meget brugt, transformere over til, at disse bliver de materialiserede tekster i databasenm og måske samtidigt gøre, at de gamle tekster bliver de virtuelle. Hvis folk så gerne hurtigt vil fremhæve et tekstudsnit i en tekst, der endnu ikke har fået tilføjet struktur men bare er atomar, så kan de så gøre brug af forælder-barne-tekst-relationen, fordi de så kan trække udsnittet ud som en selvstændig barnetekst og give den et point direkte, og når browseren viser atomare tekster, kan den så bare lige se, om der skulle være nogle barne-tekster til den, og om disse så har fået nogen "alarm-point," som man måske kunne kalde dem. Jeg tror, at dette bliver smart især for nye tekster, som måske gerne skal have lidt tid at blive rettet i, før man begynder at bruge energi på at tilføje struktur til dem --- så man altså lige kan få taget de værste fejl på en hurtigere og nemmere måde. ..Nå jo, og alle de nævnte typer så som (tekst-)prædikater og komposit-tekster (hvilket altså vil sige mine tekst-struktur-"closures") må vist gerne fremgå af terments nøgle. Det tror jeg på, vil være en god idé. Og det vil så være underforstået, hvad der menes, når komposit-tekster har prædikater eller ikke-prædikater (så som andre komposit-tekster eller atomare tekster) som input. I førstnævnte tilfælde er meningen, at man skal søge på et term, der opfylder prædikatet, og i de to sidste tilfælde (som ikke er ligeså recommented, men som stadigvæk vil være ret praktisk at have mulighed for), vil det bare være underforstået, at prædikatet er: 'Er' (som i: "Er atomare teskt x" eller "Er kompositte teskt y"). ..Og det bliver altså i bund og grund det.. Og nøglen i hele idéen er så især det her med "variable tekster, hvor sektioner kan udskiftes løbende og indsættes så de passer til brugerens specifikke behov," hvilket så også hænger sammen med "fri og brugerdrevne pointsystemer," hvilket videre også hænger sammen med (eller 'resulterer i,' kan man måske også sige) "frie og brugerdrevne algoritmer til at søge på og filtere indhold, så det passer bedst muligt til ens præferencer." Så her har vi altså nogle nøglepunkter, og "brugerklassifikation og ML-teknikker (begge brugerdrevet) som led i at danne et alsidige og langt mere effektive pointsystemer, end hvad man har set før" bliver så også nogle vigtige elementer af den smalede idé. Mit eksempel på et simpelt point-system til at vurdere korrekthed og uddybelsesgrad i argumenterende tekster m.m. er også i øvrigt ret vigtigt, for jeg tror virkeligt at man kan komme langt, selv med dette simple pointsystem (og det vil helt sikkert blive en central del af alle fremtidige vidensdelings-sider/-applikationer/-netværk). Og mine versioner for, hvordan systemet/netværket kan forbedre Git (og Stack Overflow), og hvordan det også kan forbedre web 2.0-sider så som bl.a. sociale medier, tror jeg også begge er ret vigtige at forklare. I øvrigt kan folk måske opslå deres interesse-'emner' i deres biografi-tekst, hvilket bl.a. så måske kan gøre, at man kan forkorte hashes'ne i sine links (i sine opslag), især hvis en dato altid også kan parses fra opslaget. For hvis interesseemne-nøgleordene og datoen for et oplag altid kan parses, så vil det nok sjældent kræve særligt meget af hashet, før database-termet kan identificeres. Nå ja, og ellers kunne man også bruge en konvention, hvor man bare svarer sig selv med link-endelserne, så hele links'ne kommer med, men samtidigt uden at gøre den egentlige tekst ulæselig (eller svært læselig, rettere) for mennesker. ..Ah, eller vedkommene kunne også bare oploade point til pågældende termer, som er beregnet til at gøre, så der effektivt set bliver en relation (som altså kan udledes) i databasen mellem et bruger-id (i.e. en offentlig krypteringsnøgle) og så de termer, denne bruger har linket til. Ja, det var egentligt meget smart; så kan man altså herved ret nemt få det, så man kan indsætte korte links i sine web 2.0-opslag.\\
\%Men ja, "et semantisk netværk, hvor tekster er dynamiske på den måde, at de i stedet for f.eks. atomare tekster bare har prædikater, hvor det så er meningen, at den bedst mulige tekst, der opfylder prædikatet på det pågældende tidspunkt (og også ud fra brugerens eget valg af point-parametre), kan indsættes" er et meget centralt punkt ved idéen.  
}

I øvrigt har jeg vist ikke forklaret så meget om mine tanker, når det kommer til SoMe-platforme (andet end i de udkommenterede noter), men tanken er bare, at man skal kunne linke til argumenter og forklaringer for/på ting, og at andre læsere af det givne web 2-opslag så ved at bruge en browserudvidelse (muligvis implementeret som en html-side, hvor man bare åbner web 2.0-siderne i en iframe) til at parse linket og enten erstatte det med at hyperlink til en tekst, der opfylder prædikatet (bedst muligt ud fra brugeren valg af point), eller simpelthen indsætter det direkte i kommentaren, hvis det ikke er alt for stort. Og ellers er tanken sådan set bare, at folk ikke kun har værdifulde og konstruktive diskussioner, når det kommer til tekniske områder, men der er også et kæmpe stort potentiale i, hvis man for indført mere strukturerede diskussioner i web 2.0-siderne, så folk kan få deres opslåede indsigter mødt med ordenlig analyse, hvor de mest konstruktive input flyder til vejrs i pointsystemet(/erne), i stedet at folk (som det desværre ofte forholder sig nu) bare plaprer forbi hinanden. Så brugerdrevne pointssystemer til at fremhæve de konstruktive argumenter/modargumenter vil altså blive virkeligt stort, og det vil det også, det at folk hurtigt kan linke til andre folk forklaringer og diskussioner af det samme emne, så man ikke skal huske og gentage andre folk tidligere argumenter i hver ny diskussion, men altså altid bare kan bygge oven på andre menneskers tidligere arbejde, npr det kommer til diskussioner. Og hvad jeg mener med potentialet i ``ML-teknikker'' kan også læses i mine udkommenterede noter. Men kort sagt er det bare, at jeg tror man lynhurtigt kan komme i gang med en effektiv `brugerklassifikation,' som jeg jo også har snakket om i tidligere sektioner, ved at bruge Machine Learning-teknikken, hvor man udleder korrelationsvektorer. Og dette bliver også kæmpe sort for webbet, for nuværende klassifikationsalgoritmer rundt omkring på konventionelle sider er bare alt alt for dumme og for simple/generaliserede ift., hvad jeg mener, at man ret let kan opnå, hvis man arbejder sammen i brugerfællesskabet om finde frem til mere værdifulde og nuancerede opdelinger.  




%(16.08.21) Nu har jeg fundet ud af, at min idé her altså også sagtens kan implementeres via den konventionelle sem-web-teknologi. Det handler bare om at starte en konvention om, at uploade prædikater osv.\ med fokus på tekststrukturer i stedet for bare de ret atomare semantiske genstande, som de nuværende(hidtilige...) forsøg på det semantiske web... %..mit bidrag kommer nok i bund og grund til primært at blive, at man skal gå mere meta....

(17.08.21) Okay, jeg kom hjem fra ferie for en uges tid siden, hvor jeg nåede at tænke lidt videre over idéen her, og hvor egentligt kommet til et punkt, hvor jeg virkeligt troede, der kunne være et stort potentiale i at starte idéen som et slags programmeringsparadigme om at strukturere koden top-down ved brug af et system a la det, jeg har tænkt, med tekst(/kode-modul-)-prædikater som skelettet for kodebasen og med brugerdefinerede pointsystemer til at analysere og verificere kode-modulerne/-udsnittene i et samarbejde. Denne idé tror jeg stadig på, kan have rigtigt stort potentiale, men grunden til, at jeg har haft en uges skrivepause nu, er at jeg for det første, da jeg kom hjem, fandt ud af, hvor tæt min idé lå på det konventionelle sem-web, nu hvor jeg er gået lidt tilbage til at bruge URL'er. Og dette fik mig til at søge på nogle ting, og nu har jeg fundet ud af, at rigtig mange af mine idéer, som jeg hidtil bare har overvejet og udviklet på selv, faktisk er tænkt, og at der faktisk er en del, der har mange af de samme visioner, når det kommer til de idéer. Særligt fandt jeg frem til (i tirsdags for en uge siden, mener jeg), at mine visioner om at implementere et debat-web via det semantiske web faktisk i høj grad er delt af andre --- og ikke nok med det, der er også aktive grupper pt., der arbejder på de visioner! (Aktiviteten er vist dog ikke vildt stor, fordi der ikke er funding, men den er der\ldots!) Noget af det første, jeg fandt frem til, var, at der er en w3c-gruppe i gang omkring `credibility assessment,' og jeg er vild med deres manifest (eller hvad man skal kalde det)! Jeg er særligt vild med, at tanken omkring at have en bruger-netværks-graf (FOAF), hvorfra brugerne er frie til at lægge deres egne tillidsfordelings-algoritmer over, og at brugerne således selv er i fuld kontrol, også er et grundlæggende standpunkt for gruppen; jeg havde slet ikke forestillet mig, at denne tanke/idé var så main-stream! Jeg troede, det var en af mine helt store idéer/indsigter, men åbenbart ikke! Dette ledte så videre til, at jeg fandt frem til `the Argument Web' og også en `web annotation'(eller noget i den stil)-gruppe, og nu kan jeg altså se, at der i den grad allerede er tænkt mange af de samme tanker og visioner, som jeg har haft omkring videns- og debat-ontologier. *(Og i forbindelse med disse ting, fandt jeg altså også ud af, at min tanke om et semantisk og socialt-netværk-agtigt lag over det gængse web (så altså et overlay med annotationer osv.) også er tænkt af andre.) Og i bund og grund undrer det mig faktisk slet ikke. Det er nogle ret simple idéer, og det er også klart nok, at andre har tænkt i, om ikke man kan designe det på en decentral måde (og at der også er en vilje til dette). Og i modsætning til andre tilfælde, hvor der har været lidt blandende følelser med at opdage, at visse af mine idéer allerede er tænkt (hvor jeg dog sjældent har været mere ked end glad for at opdage det, men det er en anden historie (eller andre histori\emph{er} rettere)), så er dette faktisk ret lykkeligt. For alt dette handler nemlig om min kerne-vision, kan man sige: Det er den vision, jeg virkeligt tror, kan ændre verden, og som derfor på en måde lidt har været målet med alle mine andre idéer. Jeg vil nemlig gerne ændre verden til det bedre, og hvor at jeg er f.eks.\ er utroligt glad og stolt over min QED-teori, å kommer den nok ikke ligesom til at ændre så meget. Men, har min tanke været, hvis jeg har mine mere godgørende idéer på plads først, og jeg så opnår berømmelse via f.eks.\ min QED-teori, jamen så må jeg kunne bruge den interesse til at få nogen med på mine andre idéer.\footnote{Og dette er i øvrigt grunden til, at jeg endnu ikke har fået udgivet min teori, for det har været så vigtigt for mig, at få mine andre idéer nogenlunde på plads inden. Jeg vil nemlig gerne i så fald kunne ride på den interesse, der vil komme, og ikke lade den dø ud, inden jeg kan få udarbejdet nogle sammenhængende idéer for, hvad jeg vil efterfølgende.} Så nu er det altså slet ikke ærgerligt at andre (selvfølgelig nok) har tænkt de samme tanker --- især fordi der endda stadig er gejst omkring de idéer, selvom der jo har været meget dødvande i det. Det undrer mig godt nok lidt, at jeg ikke er stødt ind i de ting noget før, når jeg har søgt på det semantiske web (efter at jeg opdagede dette (story of my life; jeg har slet ikke tal på, hvor mange gange tilsvarende er sket\ldots)). Men jeg har i hvert fald fået det indtryk, at det hele bare handlede om atomart og skalart (hvem-hvad-hvor-og-hvor-meget-)data. Det er øvrigt stadig mit indtryk, at fokusset i alt for høj grad er på dette (ift.\ hvor mine interesser og visioner ligger), men det kan jeg komme til. Samtidigt var jeg på det tidspunkt også meget opsat og positiv omkring tanken om at (gen)starte interessen for det semantiske web og udbrede teknologien med udgangspunkt i mine ITP-tanker, hvor jeg altså tænkte (hvad jeg ikke helt gør mere) at disse nye metoder til programverifikation kunne være med til at sætte skub i et ``mere åbent og brugerdrevet web,'' hvilket så kunne føre det semantiske web med sig derfra. Så dette må altså være grunden til, at jeg ikke fik set mig mere om dengang (og det kunne i øvrigt lige ligne mig, at gøre noget tilsvarende nu, og altså gå for hurtigt videre, inden jeg får læst stoffet grundigt nok). *(Ah, og jeg kan også i øvrigt huske, at jeg så `trust'-delen i den visuelle model over sem-web-teknologien som noget centralt, og jeg har altså aldrig fattet, at decentrale trust-/credibility-algoritmer var en gængs tanke.) Nå ja, i øvrigt er min `dynamisk demokrati'-idé også vist nok ret meget tænkt. Jeg faldt i hvert fald også over emnet `e-demokrati,' og det lyder rigtigt meget i retning af mine tanker. 

Jeg har nu tænkt lidt over tingene, og er lykkeligvis kommet frem til, at mine idéer måske egentligt kan koges ned til nogle ret simple punkter, givet det nuværende stadie af teknologien/videnskaben, og at der dermed måske virkeligt faktisk kan være kort vej, til at få disse idéer ud i verden. Efter at jeg lige samler trådende i denne sektion, vil jeg skrive en afsluttende tekst til dette note-sæt, som opsummerer, hvor mine idéer er nået til. Derefter vil jeg lige skrive en renere version (og gerne på engelsk), og så vil jeg ellers nok gå i gang med så småt at finde frem til netværk, hvor idéerne kunne være interessante, samtidigt med at jeg så bruger den resterende tid på (endelig!) at skrive min fysikartikel.  

Nå, men lad mig så prøve at samle trådende for emnet i denne sektion. Som sagt ligger mit system, som jeg har beskrevet her i denne sektion, ret tæt på den konventionelle sem-web-teknologi, og ikke nok med det, jeg tror nu på, at mit system sagtens bare kan implementeres via tripletter også (og dermed via den konventionelle teknologi, pretty much). 
For selv om mit system vil have funktionelle afhængigheder i visse tripletter, og hvor eksempelvis det sidste term vil være afhængigt af de to første (i.e.\ der vil være visse én-til-én-relationer i systemet), så kan man altid bare identificere disse afhængigheder og bruge det til at optimere databasen i laget under et interface, som bare udelukkende kan bruge tripletter. \ldots Hm, jeg troede, jeg kunne give et godt eksempel på dette, men nu er det faktisk ikke sikkert. Så det er egentligt heller ikke sikkert, at jeg overhovedet vender tilbage til dette emne, men nu må vi se. 
Til gengæld skal jeg huske også at nævne, at mine idé om at have en relation med lokale og globale nøgler i et lag under database-interfacet jo egentligt er omsonst. %Jo, man kan godt have skjulte lokale nøgler i et underliggende lag for måske at hjælpe databasen med at gemme relaterede ting tæt på hinanden... %hov nej, det er da ikke omsonst, for man kan jo muligvis spare plads ved ikke at gemme de globale nøgler alle vegne.. Hm.. ..Ah, men man kunne da netop bare gemme den hos den entitet (ved at tilføje en ekstra attribut), som nøglen refererer til. Ja.. ...Jo tak, men så skal man jo stadigvæk transformere alle nøglereferencerne, så jo, man kan med fordel hive global-lokal-nøgle-relationen ind i de grundlæggende relationer, men det kræver så også en hel transformation af alle entiteter.. Hm..  
\ldots Ja, for man kan i stedet bare oprette en ekstra attribut med den lokale nøgle for hver relevant entitet, og så kan man altså herefter (eventuelt) transformere de globale nøglereferencer i entiteterne %og skifte dem ud med lokale. 
om til lokale. Og hvis man så skulle få en global nøgle i hånden og ikke en lokal, kan man jo så bare søge på den globale nøgle i stedet. Så hvis nøglerne bare selv viser, hvad de er (lokale eller globale) behøver man altså at ændre i databaseopbygningen ellers.

Og mine ``komposit-tekster'' behøver heller ikke at implementeres som relationsentiteter, som jeg ellers var begyndt at tænkte, men kan også bare implementeres som tekster i en speciel syntaks (muligvis en undermængde af JSON, eller noget i den stil). Og så bør man altså bare, som jeg også tænkte originalt, have browser-applikationer til at parse komposit-teksterne og enten udskrive den (eller dele af dem), som de er med deres tekst-prædikater, eller hvor passende tekster indsættes på tekstprædikaternes pladser. 


%Inden jeg giver et godt eksempel på dette, kan jeg lige forklare følgende, der kommer til at vise, hvor der bl.a.\ vil blive sådanne funktionelle afhængigheder i systemet. (Det var knudret sagt, men det kommer til at give mening). 
%Mine ``komposit-tekster'' kan nemlig implementeres ved... %Hm, de kan vel implementeres på flere måder; man kan både bruger hash-nøgler, og man kan lade være... ..Hm, de kan jo implementeres som html-dokumenter med tomme sektioner, hvor pradikaterne er tilføjet som html-attributter, for det første.. ..Og ja, man kunne jo også alternativt lave en mere simpel syntaks.. Hm, jeg hælder til en mere simpel syntaks, sådan at det bliver komprimeret, og så at dataen bare ekstraheres og hældes ind i et html-dokument, når brugeren skal have vist strukturen eller det resulterende (deadresserede) dokument.. Hm, så hvad er spørgsmålet egentligt..? ..Jeg har jo tænkt mig, at globale nøgler skal gives med URL'erne, så de bl.a. lettere kan transformeres, og jeg har også tænkt på at bruge hashes direkte i stedet for URL'er.. Hm, men får man brug for at gøre det sidste? (Der har vi spørgsmålet.) ..For man kunne vel også bare bruge URL'en, muligvis med en global nøgle post'et med, og så kan vi jo have hele strukturen gemt på lokationen.. Ja.. Strukturerne (og closures'ne i det hele taget) kan vel bare gemmes som tekst alt sammen.. ..Hm, og så kunne man dog måske lave en fast syntaks, så det også kan gemmes som relations-entiteter... (i laget under interfacet..) ..Ja ja, det er fint. Så ja, komposit-tekster kan bare implementeres via syntaks, og er det så det?.. Og angående point, og det faktisk ikke sikkert, at man kan gemme det mere effektivt end tripletter, nu hvor jeg lige tænker lidt over det (men har ikke tænkt så meget endnu).. ..For den funktionelle afhængighed, der er i fobindelse med pointene, den kan man vel alligevel ikke slippe udenom..? ..Og nu behøver jeg måske ikke liste-operator-agtige relationer (hvis jeg ikke bruger hashes, men bare holder mig til det konventionelle med URL'er)..? ..Nej, det gør jeg ikke. Okay, så det eneste jeg tilføjer, bliver måske nærmest bare: "Prøv at gå lidt meta, og fokuser på udsagn om komposit-teskter, i.e.\ hvor der altså er tale om en struktur med tekst-prædikater i stedet for faktsike tekster i sektionerne.".. (Og jeg har også et par flere pointer, bl.a. om at gøre stilen og applikations-scriptene (til at browse det semantiske web) til en ontologi i sig selv, men den nævnte pointe er måske den mest vigtige.. (..og den hører så sammen med bare: "Prøv at fokuser på forklarende tekster og sammenhængende udredninger --- også når det kommer til Argument Web --- i stedet for primært bare den skalare (inkl. hvem-hvad-hvor) data..)) ..Ja.. Ah, der er godt nok lige en pointe også om, også at implementere et hurtigt netværk, hvor man bare skal vise lidt PoW for at joine, hvilket kan skabe en hurtig og god base for mere sofistikerede (og mere ligesom FOAF, men også mere sofistikeret endnu) netværk. (Og her har jeg så også nogle få hurtige idéer omkring captcha..) Ja, så det hele er altså pludseligt vældigt simpelt.. (..umiddelbart!..)


%Okay, men inden jeg begynder at forklare ... %Nej, lad mig bare starte med at forklare... hm, eller skulle jeg skifte sektion til den afsluttende tekst..? 

Pointene kan også bare implementeres som tripletter bestående af point-id, den omhandlende entitet og så en reference til pointgivningens data, hvilken så kan inkludere bruger-id og point-værdi (og husk, at det hele er i form af referencer) samt en tilhørende underskrift, der så underskriver hele point-entiteten ved både at underskrive point-type, grundled og point-værdierne med det pågældende bruger-id. Underskriften bør også indeholde datoen (eller timestamp'et rettere). Og så kan man altså stadig bare bruge point fra tillidsværdige parter (som man selv vælger som bruger) til at udregne diverse aggregater af pointene\ldots\ Hm, men hvordan sørger man så for, at disse aggregat-point kan findes direkte, hvis bruger-id'et ikke indgår i forælder-tripletten? Hm, så må man jo næsten sørge for, at pointtyper kan indeholde whitelists over folk, der må uploade point af denne type, hvor databasen så skal afvise alle uploads, hvor bruger-id'et ikke passer med whitelist'en (og i øvrigt også alle uploads, hvor signaturen ikke er fyldestgørende)\ldots\ Ja, men det kan man også godt forvente af en konventionel sem-web server, nemlig at den kan have et filter, så den ikke bare accepterer alle uploads. %Hm, og hvordan opdaterer man så denne whitelist..? ..Tja, man giver vel bare lov til at bruge aggregaterne fra de gamle point, når point fra de opdaterede pointtyper skal udregnes..? Ja.. 

%(18.06.21) 
Når det kommer til at opdatere pointtyper, eksempelvis ved at opdatere deres whitelist, så handler det så bare om at oprette en ny pointtype, som har lov at bruge aggregater fra den tidligere pointtype, og så ellers bare sørge for at tilføje det til en pointtype-ontologi, så folk kan få overblik over udviklingen og forgreningerne.

Ja, så alt i alt er det nemt nok, at implementere alt det, jeg havde i sinde med tripletter udelukkende, og det gør nemlig heller ikke noget, at komposit-teksterne ikke er relations-entiteter men bare er tekster med en fast syntaks. 

Jeg vil dog komme med et forslag til, hvordan det måske ville være en god idé at beslutte sig for et skema til at danne globale nøgler i systemet til teksterne og (måske også) tripletterne, måske a la den måde jeg har tænkt det, hvor man altså bruger et hash som en del af nøglen, og at man så altid giver denne nøgle med i triplet-URL'en efter `\#'-symbolet (tror jeg nok er bedre en efter `?'). For så vil det nok blive væsentligt nemmere at transformere gamle tripletter om, så deres URL-referencer bliver up-to-date, kunne jeg forestille mig. Måske ville det endda også lette semantikken en del, hvis disse nøgler bliver mere menneske-læselige\ldots\ Hm, kan man egentligt følge en URL kun med en `\#'-streng (hvad det nu end lige hedder)\ldots? Ja, man kan have relative URL-stier. Fedt nok, for så kunne man sådan set bare bruge triplet-URL'er på formen `\texttt{/\#<nøgle>}' eller `\texttt{/\#<emne-nøgleord/-sti><tag><nøgle>}' eller varianter af disse, hvor man altså muligvis giver den absolutte nøgle og/eller giver hjælpende emnenøgleord, eller en emne-sti ligefrem, og giver et menneskelæseligt tag for, hvilket objektet der refereres til (som så helsts skal kunne forstås entydigt givet emne-konteksten). Og så skal brugeren bare være på en side, hvor et javascript-script kan læse disse strenge og sende brugeren hen det rigtige sted. Man kunne så endda også bare bruge `\texttt{/\#[emne-nøgleord/-sti]<tag>}' og så inkludere den absolutte nøgle (og muligvis emne-stien, hvis man ikke inkludere den direkte, og hvis dette overhovedet bliver en hjælp (for det pågældende javascript-script), når den globale nøgle jo allerede er givet) i en header-sektion over tripletten, for med en relativ sti, hvor det kun er en `\#'-streng, bliver man jo bare på den samme side, og scriptet kan derfor bare læse disse headers direkte. Og for at det ikke skal være løgn, kan man endda så også bare inkludere biblioteker af headers, så man uden videre bare kan bruge alle de tags, som biblioteket definerer. Så tripletter kan altså også herved komme til at blive rigtigt overskuelige at skrive og at læse. Jeg ved dog ikke, om det i første omgang er et problem, at konventionelle tripletter er ret svære at læse, for man må jo alligevel forestille sig, at mange bruger specielle interfaces til at konstruere og uploade tripletter, men det må folk, der har mere viden om emnet end mig, jo bare finde ud af. Men uanset hvad, er det sikkert ikke en dum idé at ``poste'' (bare altså til et script i sin egen ende og ikke til serveren, hvis man altså bruger `\#' og ikke `?') de globale nøgler i triplet-URL'erne, så man let kan omtransformere dem, hvis f.eks.\ domænerne går ned (medmindre der allerede findes en eller anden simpel teknik til dette, som godt kan reparere URL'erne som de er, hvilket jeg dog har lidt svært ved at forestille mig). \ldots Hm, og alt dette passer egentligt også godt med, at `\texttt{\#<nøgle>}' (eller `\texttt{\#[emne-nøgleord/-sti][tag]<nøgle>}') ikke er noget dårligt valg af syntaks, når man skal angive et objekt i teksterne, f.eks.\ når man skal referere til prædikater i en komposit-tekster. Ja, så det vil jeg lige foreslå, men ellers kan mit system altså bygges ret nemt oven på den eksisterende, konventionelle teknologi bare (og jeg vil nu stoppe med at sige ``mit system,'' for det er altså heller ikke banebrydende rent teknisk set; idéen er stort set bare, at lave en ontologi med sammenhængende forklarende tekster ved at bruge komposit-tekster, hvor sektionerne har prædikater i stedet for de faktiske tekster, og så at tekstopbygningerne og selve sektionsteksterne altså dermed er adskilt flere (mindst to) lag). (Hm, nå ja, og så har jeg også den idé, at have en ontologi over selve de applikationer, man kan browse ontologierne\footnote{Jeg er forresten blevet i tvivl, om jeg bruger det ord rigtigt. Jeg er usikker på, om en `ontologi' inkluderer selve dataet i grafen, eller om det bare er en begrebs-graf til at organisere data\ldots} med, men også her kan dette bare implementeres ret nemt via det konventionelle system. Og dette er jo rigtigt fedt i øvrigt: Fedt, at min idé bare indebærer at bygge nogle lidt nye former for applikationer i en eksisterende teknologi. Hermed kan det nemlig bl.a.\ også forklares ret kort, hvilket jo er skønt --- og det kan jo sikkert også implementeres så meget desto hurtigere.)



%Husk g-nøgle-posts..  (tjek)

%Okay, så hvad mangler jeg at skrive om? Jeg skal vel forklare, hvordan min idé kan tage ret meget udgangspunkt i "en wikipedia, hvor hver artikel ikke bare har en overskrift men også har prædikater, der bl.a. beskriver, hvor dybdegående og/eller hvor introducerende teksten er, og måske videre også kan begynde på et tidspunkt at inkludere ting som biases og lignende.. Og så skal jeg jo netop også føre det videre til, at hvordan det kunne bruges som et argument web, der bare er fokuseret mere på sammenhængende udredninger i stedet for (atomare) påstande mod påstande. Og herefrer skal jeg også nævne det med, at have selve browsing-applikationen som en del af en ontologi, hvor brugerne altså kan komme med alle mulige forslag (som kan verificeres for sikkerhed af andre brugere) til, hvordan hele applikationen kan virke, og dette kan så faktsik føre til en slags omni-side, som både kan fungere som en twitter, reddit, youtube, you-name-it. Så altså en helt ny form for web, hvor brugerne er i fuld kontrol. Og så har jeg nogle pointer omkring dette, som gør, at jeg mener, det kan komme til at virke. *(Her tænkte jeg på "tiltag-kanaler.") Så er det ikke ca. det?.. ..Og så har jeg lige et par småting til andre sektioner, plus jeg gerne vil lave en afrundende tekst (der måske også gentager idéerne fra denne sektion lidt). Jo, så skal jeg også tale om "blade med forslag til ny empiri".. "psykologi-ontologi".. Nå ja, og mine tanker omkring at bruge det i programmering, selvfølgelig.. Husk "fokus på kilder" også.. ..Snak lidt om, at der er mange ting man kan gøre, hvis bare interessen var der (..og måske man vil kunne vække en ny debat-kotume-bølge frem ved at påpege, hvor vigtigt det er..).. "Model-orienteret argumentation" og "street-credits" (og "psykologi-grupper").. "Kanaler" med gode udredninger, algoritmer, stile, og generelle (UX-/usability-)tiltag.. Lidt om captcha-spil osv.. Forklar om, at der må være penge at tjene (for det må der jo).. Og hvis vi zoomer ud, så nævn at fokuset (med sem-web) også kan være mere på forklarende tekster og generelt indhold, samt endda også UX/usabilty og applikationsmuligheder, i stedet for bare at handle om skalart data (b-b-b-boring!). ...Hm, man kunne måske endda begynde at snakke om "semantic web of data," "semantic web of content" og "semantic web of applications" som tre separate ting..! *Og holdnings-ontologi og mine simple point..


\subsubsection{Ny ny tilgang}
(18.08.21) Okay, nu vil jeg så prøve at opsummere, hvad mine idéer går ud på i overordnede træk, og hvor jeg så ikke går så meget ind i det tekniske, for det synes jeg, jeg har redegjort for i det ovenstående, og nu bygger det jo også stort set bare på konventionelle systemer. 

%Jeg har overvejet, om ikke man kunne introducere idéen på denne mpde, og sige at det semantiske web af data bliver en vigtig teknologi, især når vi også får bedre ML/AI, men det semantiske web kommer også til at bruges til mere end bare at kunne søge effektivt på data. Jeg tror således for det første, at et `semantic web of content' bliver mindst ligeså vigtigt, og desuden tror jeg også et `semantic web of applications' vil blive vigtigt. Man kan argumentere for at `content' kan høre ind under `data,' men det ændrer ikke på, at fokusset pt., af hvad jeg kan se, er meget på atomart (hvem-hvad-hvor-hvor-meget-)data, og ikke så meget på længere tekster eller... % Hm, men en del af den gængse tanke er jo også at tilføje semantisk metadata til videnskabelige artikler og blog-artikler, så den holder da ikke alligevel.. ..Nej, jeg kan ikke begynde at hævde, at det er en ny tankegang (i hvert fald ikke når det kommer til "SW of content").. 


%(19.) Hm, følte, at jeg kom med et ret godt argument for, hvorfor man skal gå over til SCID (software code in database), hvis man ville gå over til mit kommentar-og-point-system, som jeg lagde lidt op til i starten af denne sektion, og som jeg har tænkt videre over. Jeg fandt nemlig frem til, at jeg nok skulle starte idéen som en idé til et not programmeringsparadigme (ikke et stort paradigme så som funktionel, objectorienteret og dem; bare et mindre et), og den plan syntes jeg altså lidt, jeg fik bekræftet på Bornholm, fordi jeg altså syntes jeg kom frem, at et sådant system ville kræve semantisk SICD, og dermed kunne lede videre til resten (ikke helt af sig selv, men alligevel). Men nu fik jeg lige nogle tanker i går aftes, der muligvis kommer til at ændre det syn.. Mit argument var nemlig, at det ville kræve alt for mange kommentarlinjer, hvis man ikke bare for det første skal forklare kodeudsnit (som man allerede skal efter de normale konventioner) og videre skal skrive bevisene kommentarer til kodeudsnittene, og ikke nok med dette oven i købet også skal skrive bevisende kommentarer til, hvorfor kodeudsnittene tilsammen, hvis de virker korrekt, også giver en korrekt overordnet funktion eller klasse. Men jeg tror nok, da jeg havde disse tanker, at jeg tænkte, at sidstnævnte kommentarer ville skulle være inde imellem kodeudsnittene.. Hm, altså det ville jo være godt med en top-down opdeling, hvor man til hver funktion/klasse først bygger et skellet (med tekst-prædikater i stedet for faktiske tekster), for disse vil også være nemmere at kommentere og vise korrektheden omkring, end hvis man referere til den fulde, verbose funktions-/klasse-kode. Så ja, min idé til også at opdele kode på en mere struktureret måde ved brug af, hvad der svarer til mine "komposit-tekster" med "tekst-prædikater" i, er stadig sikkert en rigtig god idé, men nu er det bare lige før, jeg tror, at man også kan komme langt med alternativer. For, lige for at færdiggøre forrige tanke, man kun jo bare indsætte kommentarerne, som forklarer den 'afhængige korrekthed' af funktionen/klassen, oven over denne, eller endda bare indsætte en reference til dem endnu bedre. Hm.. (I øvrigt kan denne her omstrukturering sikkert også forbedre version control..) Hm.. ..Kom lige til at tænke på også, at mit paradigme måske godt kan lade bugs gå igennem tjekkene, hvis man har en ikke-præcis beskrivelse en et kodeudsnits opførsel, og så misfortolker det i en anden sammenhæng. Og hvis man så skal undgå dette, skal man altså kræve præcise definitioner på kodeudsnittenes forventede opførsel, hvilket jo nok måske gør paradigmet for regidt og restriktivt igen.. (ligesom mit 'matematiske paradigme' jo nok lidt var (med vores nuværende teknologi)).. (eftersom paradigmet så går hen og faktisk bliver til det 'matematiske paradigme,' jeg før har tænkt (og som andre i øvrigt helt sikkert også har tænkt; det er ikke fordi, jeg skal tage credit, btw..)).. Hm, eller også kan man komme rigtigt langt bare med pointsystemer, der f.eks. kan sige: "ingen bivirkninger".. ..Jo, så det handler jo bare om at indføre pointsystemer til kodekollaborationer, sammen med bevisende kommentarer, som så selv kan bedømmes med point også.. (og det skal forklarende kommentarer, i.e. dokumentation, selvfølgelig også).. ..Og fordi der skal fyldes bevisende tekster ind i kodebasen, så vil det sikkert ikke være dumt at opdele det ved brug af "komposit-tekster," hvilket så måske endda også kan hjælpe med version control --- og også med at visualisere koden grafisk, hvis man vil dette (hvad man jo vil på et tidspunkt i fremtiden om ikke andet). Ja, fint. ..Og kan komposit-teksterne stadig også bare gemmes via syntaks her?.. Ja, det er faktisk det nemmeste endda. ..Uh, og forresten: Dette paradigme åbner også op for mere decentraliserede collabs..! Det er da et ret fedt salgspunkt i sig selv..! ..Ah ja, og det hører jo sammen med 'semantic web of applications.' (/'software.')

%(19.) Okay, nu overvejer lige lige lidt omkring, hvordan man kunne lave et PoW-FOAF-netværk, hvor der er noget krypto, der begrænser, hvor mange nye et enkelt individ kan optage i netværket... ..Eller også kunne man nemlig bruge et PoMennesketimer-netværk.. ..Ah: Proof-of-Kontruktive-Mennesketimer.. Hm, vent kunne dette egentligt ikke også være en basis for et valutasystem..??.. Hm, jeg tænker altså et decentraliseret netværk (som dermed også kan have arbitrært mange forks, men det gør ikke noget her (imodsætning til KV'er)), hvor folk bedømmer hinandens bidrag og i samme omgang giver vedkommende mere stemmeret i de forks, hvor parterne optræder i.. Hm, og hvis vi glemmer forks lidt, så at man bedømmer folks bidrag og giver dem mere stemmeret dermed også.. Hm, og man behøver måske så ikke at bekymre sig så meget om det politiske i det, for hvis man bare bedømmer bidragene efter, hvad man selv har behov for, så vil man jo med al sandsynlighed give stemmeret ud på en måde, så selvsamme behov vil være repræsenteret.. Hm.. Jo, og skal de forskellige forks så kunne stille lidt på parametre, så man kan sørge for, at pågældende fork holder sig inden for nogle interesserammer..? ..Ja.. Så det jeg tænker her er altså for det første en måde at tænke pointsystemer på, men spørgsmålet er så, om man også kan få lidt økonomisk værdi ind i billedet..? ...Tja, hvis der skal det, så handler det vil bare om at skabe en forening, som man selv i gruppen og/eller andre folk kan donere til, og så fordele pengene efter tokenfordelingen --- måske imod at fordelingen så også har en udfladende kraft i form af en skat, der jævner fordelingen løbende (efter hver donationslønrunde). Ja.   

%Og lige en hurtig note om lykke-valuta: Jeg har jo tænkt en del på, hvordan man dog kunne starte en/et lykke-/gavn-orienteret valuta/økonomisk system, og jeg husker det som om, jeg kom frem til noget ret fornuftigt (bl.a. med mine 11.-juni-idéer), men hvis først man når til et punkt, hvor interessen er udbredt nok til, at man kan starte det ad politisk vej (som en stor fællesbeslutning), så kan det vil egentligt bare implementeres som et demokratisk lønsystem, hvor lønmodellerne stemmes om demokratisk (med e-demokrati, og altså dermed et mere direkte demokrati). Disse lønmodeller kan jo stadig med fordel følge generelle principper, så lønnen fordeles retfærdigt. Og i forhold til interesseforskelle genrationer imellem, så tror jeg ikke dette bliver noget problem, men hvis der bliver en strid her, så kan man bare indføre en konvention i samfundet, der kræver at man straffer den forrige generation i passende grad (imens de stadig er i live), sådan at man afskrækker sin egen og efterfølgende generationer fra at gøre det samme (og belønningen til selve generationen, der afstraffer den tidligere, er så bare, af straffen kan være at den tidligere generation skal arbejde ekstra for den efterfølgende). Men ja, jeg tror nu ikke, at dette bliver et problem. Man ved dog ikke; det er ikke helt utænkeligt, at en generation kan fristes til at være for hård ved mijøet, som et godt eksempel --- det ville jo nok ikke være første gang.. 

Jeg tror faktisk, at jeg simpelthen kan koge meget af det ned til følgende. Lad os prøve at strukturere både tekster (artikler og svar m.m.) på vidensdelingssider og også software kode mere på en måde, hvor hver artikel eller funktion, eller hvad det nu kunne være, har en indre struktur delt i flere lag, hvor man i første omgang bare har selve dispositionen % (det jeg har haft kaldt ``komposit-tekster'' i det ovenstående) %Hm, eller "komposit-tekster" refererer måske også mere til hele teksten, om så afsnittene er indsat eller ej..?..
af teksten, hvor selve tekstafsnittene altså mangler, men hvor der i stedet står et samlet prædikat, der beskriver, hvad tekstafsnittet skal opfylde. Tanken er så, at når brugere browser en tekst, så skal browseren finde de mest populære (eller hvilken parameter, man nu præcist vil gå efter) tekstafsnit, som opfylder pågældende prædikater, og indsætte dem, så brugeren kan læse den resulterende sammenhængende tekst. Selve artiklen eller funktionen m.m.\ skal så også selv have prædikater tilknyttet sig og kan dermed selv indsættes som element i en anden disposition. Og her vil ret vigtige ``dispositioner'' så faktisk bare være monadiske (eller `unære,' kan man også sige) dispositioner, der sådan set bare siger: ``Indsæt tekst der beskriver emnet, som denne titel, $x$, passer til.'' Så hvis vi nu betragter en en slags Wikipedia, der opfylder dette princip, og lad os sige at du lige har søgt efter et emne og har valgt en titel, der virker interessant i forbindelse med din søgning. Nu står du så på den yderste (monadiske) disposition og venter på, at serveren og/eller din browser udfylder dispositionen. Det næste, der skal ske, er så, at din browser finder frem til en ny (ikke-monadisk) disposition, som opfylder det første prædikat, som altså bare er på ovenstående form. Denne nye disposition kan så have andre prædikater tilknyttet sig (og husk, vi snakker et system a la det semantisk web, så det er brugerne selv, der uploader prædikaterne). Og nu kan din browser måske se dine præferencer og finde den disposition med det samme (i samarbejde med serveren i et eller andet omfang), men det kan også være, at den prompter dig om, hvilken disposition, du vil foretrække. Et rigtigt vigtigt eksempel, som jeg tror bliver ret central for idéen, kunne være, om du vil have en kort, letlæselig tekst, eller om du vil have en mere uddybende tekst. Graden af hvor introducerende kontra avanceret teksten er for emnet er også et andet vigtigt eksempel. Så med denne form for Wikipedia er der altså plads til alle typer af samme artikel (i.e.\ med samme titel) og brugerne kan selv vælge, hvilken type de vil læse. Og hvad end valget i første omgang blev gjort automatisk eller af brugeren selv, skal brugeren selvfølgelig nemt kunne skifte til de andre versioner af artiklen. Hvis den valgte (ikke-(nødvendigvis-)monadiske, ikke-trivielle) disposition f.eks.\ er valgt til at være `kort' eller `letlæselig,' så vil disse prædikater selvfølgelig typisk gå igen, når det kommer til sektions-prædikaterne. Og angående disse, så vil browseren også med det samme fetch'e passende tekstafsnit ud fra brugerens præfererede pointsystem (som f.eks.\ kunne prioritere `popularitet,' som nævnt) og indsætte dem i dispositionen. Og så kan brugeren straks læse den sammenhængende tekst, der passer med dennes valgte parametre. 

Så angående alternative vidensdelingssider a la Wikipedia giver denne idé altså en form for Wikipedia, hvor artikler ikke bare har titler men også har prædikater tilknyttet sig, så men dermed kan vælge imellem flere typer af samme artikel, hvilket jo i sig selv er en meget god idé; det synes jeg i hvert fald selv. Idéen med at dele tekster op i flere lag på denne måde er slet ikke nogen vild tanke. Jeg tror endda muligvis allerede Wikipedia har sine sektioner på en modulær måde, hvor artikler kan genbruge sektioner (og så står der oven over sektionen: ``Gå til den originale artikel''). Og det hele læner sig rigtigt meget op ad gængse tanker inden for `det semantiske web,' så det undrer mig egentligt noget, at jeg ikke kan finde et wiki-leksikon, der bruger prædikater til artiklerne, som jeg tænker det her --- ingen gang bare når det kommer til titlerne, hvilket i sig selv ville være en god idé, mener jeg. Men jeg har dog også nogle flere idéer, der gør platformen/systemet endnu bedre, så selv hvis der er en god grund til, at ingen har indført tekstprædikater til wiki-leksika, så kan det sagtens være, at idéen kombineret med mine andre idéer holder.

Den næste del af idéen handler så om de nævnte point, som browser/server skal bruge til at vælge, hvilken disposition eller tekst bedst passer til de givne prædikater. Et vigtigt punkt her er, hvordan man laver aggregater af point i sådan et decentralt system (som det semantiske web jo er). Min løsning her er ret simpel, og jeg er næsten sikker på, at andre også her har tænkt den samme løsning før mig, men efter en del søgning i går (og det er forresten blevet d.\ (21.08.21) i dag) fandt jeg dog kun frem til tekster omkring algoritmer til at fordele `trust.'\footnote{Hvilket forresten dog er super fedt, at folk har tænkt meget i disse baner. For det er jo super vigtigt for det semantiske web, og det siger også rigtigt meget om fællesskabet omkring det, synes jeg.} Jeg har hverken (endnu i hvert fald) kunne finde noget rigtigt om mere komplicerede point%...%Hm, jeg overvejede lige at skrive noget om, at jeg kun finder (tommelfinger-)op-ned-ratings og stjerne-ratings for, hvor godt man kan lide noget, men det må da give lidt sig selv, at disse kan tilføjes arbitrære prædikater på det semantiske web, og så har vi jo allerede "mine mere nyancerede (men stadig ret simple) point"... Ja.. Ja, så never mind, at jeg ikke følte, at jeg kunne finde noget, der lignede mit pointsystem; det gjorde jeg faktisk. (Jeg fandt nemlig denne artikel, som er et eksempel: Ziegler, emph{Semantic Web Recommender Systems}.) Hm, men min ting omkring at bruge trust til visse brugere/instanser om at uploade aggregater, det er jeg da ikke stødt ind i..? Hm, forresten skal jeg lige passe på, for jeg har arbejdet så længe med, at alle uploads er faste og har en fast dato tilknyttet sig, men det kan man jo ikke nødvendigvis regne med, hvis man bygger systemet over de konventionelle SW-teknologier.. Hm.. ..Hvilket netop kan blive problematisk, når man vil benytte tidligere aggregater.. Hm, men der kunne jo bare være en time-to-live tilknyttet aggregaterne.. ..Ja, kan man ikke.. Nå ja, vent, jeg tænkte jo endda timestamps faktsik burde tilknyttes seperat, så never mind. Man kan opnå timestamps (/"datoer") som er arbitrært tillidsværdige, og ellers kan man indtil da også bare bruge TTL på aggregaterne. Nå tilbage til, om ikke jeg bare må have overset, at man kan bruge tillid til at få brugere/instanser til at uploade aggregater, som man kan benytte i (browserapplikations-)algoritmer... ..Tja, ikke nødvendigvis. Det kan sagtens være, at det er en lidt særlig tankegang, også selvom der selvfølgelig helt sikkert er nogen, der har tænkt det samme før (det må der jo næsten være).. 
%
\ldots 

\ldots

(24.08.21) Nu har jeg tænkt mere over det, og jo, mine idéer omkring mere nuancerede rating-systemer kan ikke siges at være nogen stor ny idé heller. Jeg skal i det hele taget ikke vinkle mine sem-web-idéer som noget stort og nyt, som jeg har fundet ud af, men mere bare som nogle overordnede tanker og visioner --- hvoraf der så måske kan være nogle, der potentielt kan bruges til at iværksætte visse løsninger, der så kan være med til at sætte skub i hele udviklingen, men det kan jeg så nævne mere som en slags ``efter-tanke'' til mine overordnede tanker/visioner. 
Det gode er så, at jeg i mellemtiden er kommet frem til (læs eventuelt mine brainstorm-noter i kildekoden til dette dokument) at min lykke-valuta-idé faktisk måske har meget mere gennemslagspotentiale, end hvad jeg ellers var kommet frem til forinden, fordi jeg altså nu mener, at teorien omkring den kan gøres meget mere simpel! Så nu tror jeg faktisk lige pludselig rigtigt meget på tanken om at starte en ny kryptovaluta til at bære mit lykke-valuta-system! Og noget andet er så, at jeg nu mener, at alle mine andre idéer i dette notesæt, som er værd at fremføre, faktisk kan koges rigtigt meget ned! Og specifikt når det kommer til mine sem-web-tanker, så tror jeg faktisk det ville være en hel fin idé bare i første omgang at introducere dem i forlængelse af min lykke-(krypto-)valuta-idé (som jeg måske nogen gange vil forkort LKV eller LV). For jeg kan bare tilføje kommentarer bagefter, omkring hvordan nogle af disse idéer også måske har potentiale til at starte noget (iværksætteri), også selv hvis min LV-idé ikke tager særligt hårdt fra land til at starte med. 

Så jeg har af den grund nu planlagt, at jeg vil lave en tekst, først som den afrundende tekst af dette notesæt og senere som en mere renskreven tekst, hvor jeg starter med at introducere min LV-idé, og så efterfølgende nævner mine sem-web-tanker, og muligvis også nogle af mine andre vigtige tanker såsom mine kundedreven-virksomheds-tanker, min QED-teori og mine eksistenstanker. Selvom mine sem-web-tanker så introduceres som en del af mine visioner omkring LV'en, så vil jeg som sagt altså efterfølgende også sørge for at skrive om, hvad mulighederne efter min mening kunne være, selv hvis LV-idéen ikke får den start, jeg håber på. 

Jeg bør derfor næsten også lige som det første afslutte denne sektion ved at skrive om, hvad mine sem-web-visioner nu indebærer, hvorved jeg så bare kan sørge for at veksle min (overordnede) analyse af idéerne imellem to udgangspunkter, alt efter om man kan forvente, om bidragsydere til systemet vil blive belønnet med penge (i form af LV eller i form af almindelige donationer bare) for deres bidrag, eller om de kun vil blive belønnet i form af web 2.0-agtig opmærksomhed for bidragene (hvilket jo som bekendt dog også kan føre til pengebelønning). 


\subsubsection{Ny ny tilgang --- nu med en ny vinkel}
%"[...] ikke derfor behøver at binde idéerne helt op på LKV-idéen, fordi jeg bare kan nævne, at "jeg tror i øvrigt også, at disse idéer er gangbare i sig selv, også selvom bidragsyderne bare lokkes af de sædvanlige web 2.0-grunde, der er for folk til at uploade ting til et web 2.0-netværk." Her er en af idéerne nemlig så, at prøve at gøre det til noget socialt, når man bidrager med tekniske løsninger (f.eks. UI-indstillinger) til platformen, og at folk kan følge og like'e [red.] sådanne open source-bidragsydere. [\n] Ja, og dette er vel en stor del af idéen sådan set, og så er der lige alt det med sammenhængende tekster *(især når det kommer til argumenter), simple point (folksonomy-ratings), at bruge dette til programmering og version control også, og... Og hvad mere (jeg har jo nævnt det med WoApps og at gøere det til en social ting --- hvis ikke en LKV-indbringende ting --- når man bidrager med programmerings- og/eller designløsninger til platformen..)..?  Jo, og at det kan startes ligesom en gængs web 2.0-platform.. Og det med at lave psykologi- og holdnings- ontologier (hvor jeg altså bruger begrebet, som om ontologier også indeholder selve deres data, hvilket jeg ikke er helt sikker på, er den gængse definition af ordet..).. [...] ..Og er det virkelig bare det..? ..Nå jo og også: "blade med forslag til ny empiri," "fokus på kilder," "Model-orienteret argumentation," "street-credits" og "captcha-spil." ..Men er det så virkelig bare det..? ... Tjo, og så bør jeg også lige understrege 'semantisk version control med prædikater i grafknuderne' og så skal jeg også komme ind på 'ML-teknik og mange forskellige trust-/point-fordelinger.' Desuden skal jeg lige overveje, om ikke min "kraftfelt-idé" kan noget ift. et e-demokrati. I øvrigt skal også nævne, at man bør huske at fokusere på, at 'credibility er en ting, man løbende skal justere (på Bayesk vis).' Og det var vist det primære. [...]" 
%*(Mine tanker omkring simplere FOAF i det hele taget er også vigtig --- og det her med, at man kan måske starte web 3.0 gennem en (open source --- i hvert fald på sigt (så muligvis "temporarily closed source"..)) web 2.0-platform.)

%Lad mig lige brainstorme kort over, hvordan opdelingen måske kunne være for dette: Et stort punkt kunne vel være semantisk version control, hvor tektsmoduler har prædikater omkring sig, således at en sammensat (og "komposit" er forrsten måske ikke den bedste oversættelse af compound (sammensat..)..) tekst kan ses som en graf, hvor der er prædikater i hver knude (og gerne med et rod-prædikat, der bare angiver titlen/emnet). Hm, dette punkt lægger så i første omgang op til at tale om min wiki-vision. Skulle man så komme direkte ind på debat-delen af det..? Ja, hvorfor ikke. Og så videre på mine ratings, som altså gerne skal implementeres, så læsere kan få en oversigt, hvor tekst bl.a. kan være highlightet ift.\ hvor sandsynlig (og også hvor entydig, hvis man skifter highlight-briller) teksten er (for at være korrekt/sand). Og så kan jeg passende nævne direkte efter, at dette jo også kan bruges til programmering. ..Så det kan altså være den hurtige rundgang-gennemgang af idéen, og hvad har vi så mere af punkter?.. Skulle jeg mon ikke prøve på at gøre det mere kortfattet, når det kommer til diskussionsdelen i denne første rundgang?.. Nå ja, og et andet stort punkt er det med, at gøre applikationsdesign bidrag som en del af "det sociale" omkring en (open source) web 2.0-agtig platform (muligvis på vej mod web 3.0).. Og så er der ellers lige nogle idéer, der primært relaterer sig til debat/diskussionsdelen af det.. ..Og så er der jo lige kraftfelt-idéen.. Tja, ja, det er vel sikkert værd at nævne som en idé til at implementere et e-demokrati.. ..Og hvordan var idéen nu, som den var nået til at være..? (Hvad kom jeg frem til?) Hm, kan ikke huske præcis, hvor meget jeg tænkte "kraftfelt" ind i mine sidste tanker om det, men en løsning kunne vel være, at man løbende stemmer byggeklodser ind i form af prædikater og relationer, og at disse byggeklodser bruges til at danne ramme for en kraftfelt-afstemning, hvor man kan "bevæge sig rundt" i et konfigurationsrum af mulige propositioner bygget af de givne prædikater/relationer. Og grænsen for, hvornår et/en prædikat/relation(-sfunktion) kan optages bør så bare være ret lille, så enhver faktisk minoritet kan tilføje prædikater/relationer til afstemningen (men dog med en begrænsning på, der forhindre spam og/eller i det hele taget, at afstemningen ikke bliver unødvendigt cluttered).. Hm, og \emph{er} denne idé så virkeligt værd at nævne?.. (I øvrigt hører det så også med til idéen, at prædikaterne/relationerne gerne må vælges, så de implementerer repræsentanter, der overtager en arbejdsbyrde fra de stemmende, og det skal selvfølgelig også nævnes, at der kan lægges vedtægter ind over afstemningen, før at partier går med til at rette sig efter den --- i.e. instanser behøver altså ikke nødvendigvis gå med til at rette sig helt og ubetinget efter de indstemte propositioner, men kan rette sig efter en udgave af modellen, efter den har været "igennem et filter" så at sige (som eventuelt også kan indebære et salgs moving average, så man heller ikke bliver så "sårbar" over for tilfældige fluktuationer)). Ja, jeg kan godt nævne denne idé også, og så kan jeg jo altid bare disclaime den ved at sige, "jeg har ikke tænkt så meget over, hvor triviel denne idé er ift. gængse e-demokrati-idéer, men nu nævner jeg den lige." Så kan jeg i øvrigt også lige nævne, at man måske kunne udvikle en app, hvor folk frit kan deltage i sådanne (legetøjs)-afstemninger, som så måske kan gribes af instanser, der har lyst til at rette sig efter dem, selvfølgelig imod en vis lovet kundeopbakning fra selvsamme fællesskab, som er medlemmer af omtalte afstemning (for afstemninger kan så også sagtens være for helt eller delvist lukkede grupper). Cool-cool.. Hvis jeg så lige går tilbage, og tænker lidt over, hvad mine "at gøre programmerings-/design-tiltag til noget socialt"-tanker indebærer... Jo, men er det ikke bare, at man folk kan have "kanaler" med applikationsbidrag i, hvor folk så kan følge kanalerne og/eller personen bag (og like'e osv.), og derudover er der så bare lige punktet omkring at arrangere alle sådanne bidrag via semantisk version control samt at oprette donationsforeninger til at donere til disse personer (og/eller gå med til at se reklamer, hvor pengene så kan gå til personerne igennem en forening/organsation).. Ja, det er vel bare det, der er pointerne.. ..Hm, okay, så lad mig egentligt præsentere omtalte første rundgang uden at komme særligt ind på debatter og grupperinger, andet end lige overfladiske kommentarer omkring det, og så lade det vente til trejde punkt. Så kan andet punkt være om at have en.. "bootstrapped web 2.0-platform".. Ja, det lyder meget godt.. Og så kan sidste punkt omhandle debat samt menings-/præference-grupperinger (om ML-teknikker). Yes..? Ja, og "street-credits"-tanken (altså at folk kan bringe et socialt spil ind i diskussionerne ved at man bl.a. vædder (værdiløse) tokens på, om man får ret i en vis forudsigelse eller ej) kan så følge efter de to nævnte hoved-underemner.. Nå jo, og så var der også lige tankerne om captcha-spil, som den sidste lille ting (ser det ud til).. ..Hm, som jo handlede om, at man kan starte med et mere simpelt system, ja, end FOAF-systemet --- hvilket faktsik er en vigtig ting i sig selv: Jeg skal også nævne det her med (som ikke lige kom med i den kopierede tekst ovenfor, men som jeg lige har tilføjet en note omkring efter denne tekst nu), at man nok bør starte med en open source web 2.0 platform, som altså har et specifikt design, men hvor man så bare sørge for at holde systemet åben for arbitrært videre udvikling, og desuden gør det nemt at konvertere dataen til SW-tripletter (for man behøver ingen gang at starte med SW-tripletter i så fald, hvilket er meget nice). Ja, så det kan jeg alt sammen nævne (også mine captcha-tanker) i et slags punkt 2½ (som så måske kan blive det trejde punkt, hvorved debat-meningsgruppe-punktet bliver det fjerde).. Yes? Yes! 

%(25.08.21) Jeg kan også lige nævne, at folk i fremtiden vil kunne finde på schemes til at snige sig uden om regninger, men dette kan man komme efter i fremtiden, når først det bliver muligt at track'e handlinger generelt, for så kan man begynde at påklistre regninger til eksterne værdier (altså andre ejendele). Men dette bliver altsammen kun i fremtiden, hvilket i øvrigt ikke gør noget, for jeg tror ikke at 'regninger' bliver vigtige for systemet heller, før at systemet bliver udbredt nok til at konkurrere med andre valutatyper (om at være meget brugt). *(Jeg burde næsten flytte dette ned, hvor det hører hjemme..)


Okay, så mange af mine idéer omkring det semantiske web er altså mere mainstream, end jeg havde gået og forestillet mig. Jeg har dog stadig nogle overordnede idéer til nogle mulige %applikatio...%Hov, jeg kom lige på en ny idé. Burde man ikke lave et system, hvor folk kan lave HOL-agtige funktioner, når det kommer til relationer og prædikater osv., i stedet for bare at have tripletter, og at man så bare altid definerer dem ud fra tripletter, så der automatisk kan oversættes frem og tilbage imellem de to formater..? Hm jo, virker som en fin idé umiddelbart.. Jeg kan måske lige nævne det under punkt 2½..
applikationer, som benytter principper fra `det semantiske web' i en eller anden grad, og som hver især forhåbentligt kan være med til at fremme det semantiske web. 


%(01.09.21) Dispositionen kunne også være: Overordnet om SW-muligheder, ny wiki, debat (måske her *(ja, men måske i kort version)), programmering, bootstrapped web 2.0, meningsgrupper og ML, debat (måske her *(..og ja, men så måske som en opfølger på det, der så allerede er nævnt)) og så slutte af med LKV-idéen (hvor jeg her bare kan referere til næste sektion).
%(02.09.21) Hm, det kan faktisk godt være, at det med at bruge det til programmering bare skal nævnes mere som en sidenote til WoApps.. For måske er det nemlig først her, at paradigmet rigtigt kommer til at blive værdifuldt. ..Og det vil jo blive værdifuldt at tænke meget i "beviser," når projekterne først bliver decentrale nok, så ja.. 


Lad mig starte med at nævne, %at jeg mener, at man må kunne dele mulighederne omkring det semantiske web op i to dele. Den første del handler så om mulighederne for effektivt at søge på data, man har brug for i en situation, og desuden også at finde svar på faktuelle spørgsmål ved at få effektive algoritmer til at analysere en relevant mængde data og finde frem til svaret helt automatisk. Dette må siges at været kernen af de tidlige visioner omkring det semantiske web. Men der er også en del af mulighederne ved at strukturere indhold på nettet semantisk, og den handler om at søge på den rigtige \emph{udgave} af det indhold, man ønsker. Her snakker vi så heller ikke bare om skalart/atomart data, men også om indhold på nettet generelt såsom tekster og videoer osv. Denne del af drømmen handler altså om at ...%Hm, tja.. På en måde er der jo et mellemtrin, som handler om at give semantiske tags til ting, men hvor det stadig bare fokusere på, \emph{hvad} indholdet handler om, og ikke på yderligere prædikater om indholdet... (Så måske jeg ikke kan dele det op sådan alligevel..) 
hvordan jeg overordnet set vil dele mulighederne for det semantiske web op, så vi lige kan få opsummeret, hvad disse muligheder er. 
For det første er en grundlæggende del af visionen omkring det semantiske web jo, at brugere kan komme til at søge effektivt på hvad end data, de har brug for i en situation, og at de desuden kan finde svar på faktuelle spørgsmål ved at få effektive algoritmer til at analysere en relevant mængde data og finde frem til svaret helt automatisk. Og når det kommer til tekster, videoer og andet indhold på internettet, som altså ikke bare handler om faktuelt data og faktuelle svar, så er der også muligheder med det semantiske web, fordi man bl.a.\ kan annotere indholdet med semantiske tags, som kan gøre det lettere at finde rundt mellem forskelligt indhold og finde frem til indhold, man søger. Sådanne annotationer kan selvfølgelig for det første handle om, hvilke emner indholdet relaterer sig til, hvilket jo svarer lidt til at give det folksonomy-tags. Men der kan også gives andre typer prædikater til indholdet, som f.eks.\ kan beskrive ting såsom, hvor godt er teksten opbygget, hvor letlæselig er den, hvor korrekte er diverse udsagn i den, og hvis det er en video, hvor ``safe for work'' er den, hvor humoristisk er den, hvor lærerig er den, samt hvor godt er videoen opbygget, og hvor korrekte er diverse udsagn formuleret i den. Og man kunne finde mange flere eksempler end dette. Lad mig i øvrigt lige nævne, at når det kommer til bl.a.\ vurderingerne af ``korrekthed,'' og i det hele taget når det kommer til at vurdere, om et prædikat eller en relation passer til en tekst eller andet indhold, så er tanken omkring det semantiske web altså, at dette alt sammen skal foregå på en grundlæggende decentral måde, hvilket vil sige, at det altid i sidste ende er et åbent valg, hvilke grundantagelser og hvilke procedurer den enkelte bruger vil benytte sig af til formålet. Man kan således se alle uploadede prædikater og relationer, ikke som sigende ``det og det er sandt,'' men i stedet som i sigende ``undertegnede bruger/instans mener, at det og det er sandt,'' og herfra kan brugere så frit lave deres egne algoritmer til at aggregere denne data og vælge, hvilke prædikater/relationer skal medtages dennes browserbrugerflade, samt hvor høj sandheds-/troværdighedsværdi skal annoteres for de medtagne prædikater/relationer.

Så indtil videre har jeg altså nævnt, at det semantiske web kan bruges til mere effektivt at forespørge data og faktuelle svar, at forespørge tekster og andet indhold, der relaterer sig til et specifikt emne, samt at finde frem til de udgaver af indhold inden for et vist emne, der passer bedst til brugerens behov. Den fjerde mulighed, jeg kan komme i tanke om, som det semantiske web vil åbne op for, er så, at selve det at browse webbet kan blive en bedre og mere effektivt oplevelse for brugeren, fordi der i højere grad vil kunne findes alverdens relevante links til det pågældende indhold, som brugeren ser på i sin browser. I henhold til denne del af visionen kan vi altså forestille os, at brugeren altid har adgang til et overlay og/eller til sidepaneler med relavante links til det viste indhold og med eventuelle annotationer omkring indholdet, så man f.eks.\ kan se, hvad andre folk fra ens (sociale) netværk mener om indholdet og/eller se aggregerede vurderinger/meninger fra mere omfattende udsnit af det samlede brugernetværk. Denne del af visionen handler altså om i større grad altid at have relevante links og informationer lige ved hånden ift.\ det indhold, man betragter på webbet. 

Angående den tredje nævnte mulighed (hvis vi skal dele det sådan op) omkring at bruge det semantiske web til at finde frem til den passende \emph{udgave} af den type indhold, man søger, så er jeg vist egentligt ikke stødt på særligt meget litteratur, der har handlet specifikt om denne side af det --- i hvert fald ikke af, hvad jeg kan huske. Da det er en meget naturlig tanke, når man jo allerede tænker på annotationer til indhold i form af prædikater/relationer, så må der dog være mange andre, der har tænkt på denne del. Men måske er det generelt en lidt overset mulighed, når vi ser på, hvordan vi generelt har det med at ordne ting på internettet eller andre steder; fokusset er som regel bare er på emner, når vi skal kategorisere ting. Hm\ldots\ %...%Hm..
Om ikke andet har jeg en idé til, hvordan en mere nuanceret brug af prædikater omkring indhold kan bruges bl.a.\ til en ny wiki-side, hvor jeg faktisk tror at denne idé har potentiale til at slå igennem og bl.a.\ være med til at udbrede det semantiske web videre derfra. Når først denne side kommer op at køre, vil den nemlig udgøre et godt trin op til de andre teknologiske visioner. Det er denne idé samt andre idéer, der også efter min mening kan komme til at udgøre de første trin på vej mod det semantiske web, som jeg vil præsentere i det nedenstående.

Jeg vil således starte med at præsentere min idé om en ny form for wiki-side, hvor tanken er at hver en artikel og hvert et afsnit ikke bare har en overskrift, men også yderligere prædikater om teksten, hvorved redigeringen således ikke bare skal bestå i at finde den bedste tekst til at forklare om emnet bag titlen, men skal også skal redigeres så den bedst muligt passer de pågældende prædikater.\footnote{Og det kan så lige noteres, at Wikipedias artikler og afsnit jo også på en måde har implicitte prædikater omkring sig, nemlig at de skal følge Wikipedias generelle konventioner for, hvad der forventes af de specifikke tekster og afsnit. På denne wiki-side kan man også med fordel benytte tilsvarende konventioner, men her skal konventionerne bare noteres med prædikater, og der skal altså også være plads til andre konventioner --- alle dem som brugerne efterspørger.} 
%Derefter vil jeg præsentere en idé til en måde at strukturere og annotere programmoduler, når man programmerer i et fællesskab, som nemlig bruger meget de samme principper som fra min wiki-idé. 
Det umiddelbare område for denne side vil så være vidensdeling af faktuelle og forklarende tekster samt diskussion, hvorved brugere i fællesskab kan uploade og redigere tekster, der forklarer forskellige synspunkter på en sag, samt tekster, der sammenfatter og reviewer diskussioner. %Og når det kommer til at sammenfatte meninger, så skal det også høre sig til siden, at brugere frit kan danne deres egne (decentrale) pointsystemer til at give forskellige point til tekster...

Det næste, jeg så vil komme ind på, er dog, at man også kan bruge den samme tilgang, ikke bare til at bygge forklarende og/eller argumenterende tekster, men også til at bygge softwarebiblioteker. For det eneste man så skal indføre her, for at giver mening, er jo bare prædikater, der siger noget om programsemantikken af teksten (og man kan sagtens beholde nogle af de samme typer prædikater såsom `letlæselig' og `velopbygget' (ifølge konvention det og det) også). %Jeg vil så hermed foreslå en slags (decentraliseret) ``bootstrapped web 2.0-platform,'' kan vi kalde det. %...
Dette leder så videre til min idé, som man kunne kalde en ``bootstrapped web 2.0-platform.''
%
%...Jeg vil derefter foreslå en idé til en ``bootstrapped web 2.0-platform,'' kan vi kalde det. 
Idéen handler om et open source web 2.0-netværk, hvor der bl.a.\ er et særligt fokus på, at folk kan opslå designløsninger til forskellige ting og især til selve platformen. Tanken er altså at gøre tiltag til web 2.0-platformen som en del af det, folk uploader, like'er og subscribe'er til på platformen, så skaberne herved for de sædvanlige web 2.0-incitamenter til at være aktiv på platformen, men hvor kreationerne så altså ikke bare er i form af indhold, men også tiltag til det ydre design. 
%Jeg vil så også komme ind på en donations-blockchain, som jeg bl.a.\ tror kan sætte gang i denne form for web 2.0-platform ...%Hm, er det nu også den rigtige vinkel at tage, at præsentere det som "idéer, der kan fremme det semantiske web?".. Hm.. Tja, jeg må jo bare sige, at de jo også er deres egne idéer..
%Hm, hvordan var det nu, havde jeg ikke en generel pointe med at starte web 3.0 via web 2.0-sider..? ..Hm, men sammenhængen er vel, at man jo netop bør bruge SW-struktur til et sådant decentraliseret Web of Apps, så det er vel ligesom bare pointen..? ..Ja.. Det er vel bare det..
Da wiki-siden også gerne skal være decentraliseret kan man så passende slå de to idéer sammen. For hvis der rigtigt nok er en fordel i som netværk at fremhæve folk, der bidrager med nye designtiltag, jamen så kan enhver decentraliseret platform jo benytte sig af dette. Og hvis vi kigger den anden vej, så vil jeg også argumentere for, at der er klare fordele ved at bruge den samme semantiske top-down-struktur som for wiki-side-idéen, hvis man skal arbejde sammen om softwareprojekter i et decentraliseret fællesskab. 

Min tredje idé, jeg vil introducere, er en idé til et donations-netværk implementeret via en blockchain, hvor folk kan donere til gode formål og særligt også til personer, der har gjort et arbejdsbidrag, til samfundet generelt eller til et specifikt netværk, hvor donoren selv har fået gavn af dette. Jeg tror nemlig på, at denne idé kan åbne op for flere donationer til open source bidrag, såsom dem de to første idéer skal bygges på, og dermed tror jeg altså, at den kan fremme disse to idéer, og dermed indirekte også fremme det semantiske web. 

Og den forbindelse vil jeg så også lige komme ind på en fjerde idé om en forbruger-web 2.0-platform, som jeg mener potentielt kan sætte yderligere fut i donations-netværk-idéen. Så jeg vil altså præsentere fire hovedidéer, og så ellers plus det løse af små tanker og idéer, som jeg lige kan komme ind på, når det passer. 



Så lad mig redegøre for ``wiki-idéen'' lidt mere grundigt. \ldots
%Okay, det her \emph{er} måske faktisk en lidt mærkelig vinkel på det. Måske skulle jeg hellere bare introducere indéerne hver for sig, og så bare lige nævne, hvordan de (måske) kan føre resten af det semantiske web med sig.

(03.09.21) Okay, det giver egentligt ikke vildt meget mening, at forklare om alle disse idéer med en motivation om at fremme det semantiske web, for den del af det semantiske web, jeg er interesseret i at fremme, er overordnet set bare det, der svarer til min ``wiki-side,'' og så også bare lige sat sammen med min ``bootstrapped platform.'' Og i det ovenstående har jeg forklaret førstnævnte idé grundigt nok, og samtidigt er sidstnævnte idé også helt straightforward derfra, så jeg føler egentligt ikke, at jeg har et hængeparti længere ift.\ denne sektion. Jeg har lige et hængeparti ift.\ næste sektion, hvor jeg lige bør opdatere den med min nye overordnede idé omkring min donations-kæde. Og så vil jeg i stedet oprette en sektion efter den, hvor jeg opsummerer disse fire idéer, plus det løse. Jeg tror muligvis stadig, at jeg vil komme ind på, at første og anden idé kan fremme det semantiske web, men nu ser vi. Teksten er forresten ment til (ligesom denne undersektion var det) at fungere som en salgs første brainstorm-udkast, som jeg så vil skrive rent efterfølgende. Jeg tror dog forresten at jeg vil gøre selv den renskrevne tekst meget kortfattet, og måske endda lidt løst skrevet, for jeg skal jo bare dele det ud til interesserede, og jeg håber desuden, som planen er nu (krydser finger for, at det holder), at min donations-kæde-idé kan slå igennem for mig. Men ja, denne sektion fortsættes, eller rettere omskrives, i en ny sektion efter den følgende omkring ``lykke-valuta.''







%Denne note hører måske ikke til her *(eller måske gør den...), men jeg skal lige overveje, om jeg måske skulle inkludere nogle noter omkring at være vakse overfor medier (især ift.\ statistiske faldgruber, når man ser et bevis passe med den ene udlægning, men glemmer at se på, hvor godt beviset passer med den modsatte udlægning), og altså hvad man måske kunne gøre for at være dette..










%Brain (21.08.21): Okay, lad mig lige lave en barinstorm over, hvad jeg så skal skrive om. ..Så jeg tror jo, at det kan blive stort med de her.. vi kan vel nærmest kalde dem annotations-point.. folksonomy-point..? Hm, altså point, hvor brugeren.. hm, basalt set kan give rating-point til arbitrære prædikater om et objekt (teskt, tesktudsnit osv.).. Hvilket jo netop er en gængs del af 'det semantiske web,' når det kommer til området omkring indhold og indholdsbedømmelser.. Hm, og hvor man så (..naturligvis) med fordel kan benytte tillid til tredjeparter til at udregne og signere aggregater af disse ratings.. Ja.. Ja, så jeg kan faktisk gå helt væk fra.. at præsentere noget af det som "en særlig (ny-esque) idé" og i stedet bare præsentere det som nogle tanker over, hvad disse (mere eller mindre gængse) teknologier kan føre med sig inden for en kort tidsramme. ..Så jeg kunne altså vinkle mine idéer mere som: "Hey, lad os fokusere mere på det semantiske web igen, for der må da være de her muligheder, som kan implementeres på ret kort sigt.".. Og så skal jeg nok ikke præsentere det som 'idéer' men mere som 'tanker' bare.. Og selve "Idéen" kommer så bare, når jeg har redegjort for de muligheder, jeg ser, i form af at jeg kan sige, og hvorfor ikke starte en virksomhed eller et projekt, som fokusere mere direkte på denne vision, især fordi der, efter min analyse, jo faktisk \emph{er} (store) penge at tjene på det.? :) .. Hm, det lyder da meget godt.. Det skal i denne forbindelse forresten nævnes, at jeg i går aftes kom frem til, at mine forestillinger omkring en open source-forening, der belønner bidragsydere bagud, faktisk sikkert ret nemt kan blive en realitet i forbindelse med fremkomsten af det semantiske web!.. Selvom jeg (forståeligt nok) har brugt en del tid på at overveje, hvordan man kunne lave mere solide (og korruptionsresistente) systemer, så er det jo vigtigt at huske, at en meget simpel udgave af idéen, jo sikkert også har god sandsynlighed for at nå rigtigt langt! I min forrige brainstorm ovenfor her (fra d. 19.) kom jeg jo ind på, at det kunne være værd for brugerne at benytte et system, hvor de vurdere værdien af hinandens bidrag, og måske lader dette være en del af en tillidsfordelingsalgoritme, og så er der jo som nævnt ikke langt til også at lave en forening, som tager udgangspunkt i en vis tillids-/bidragsværdi-fordelingsprocedure til at uddele donationer (til foreningen) ud til bidragsyderne. Så med andre ord er dette bare idéen om et helt frit system, hvor folk bare donere efter egen lyst (i bund og grund), og hvor de så bare selv vælger, hvilken forrening med tilhørende fordelingsprocedure de vælger at donere til --- uden nogen umiddelbare bindinger. Dog kan fællesskabet så løbende indføre små goder til folk der har doneret til fællesskabet, og kan også med tiden sanktionere folk lidt, som ikke har doneret, men som pludselig selv vil tjene som bidragsyder i fællesskabet, men dette kan nok, mener jeg, være en rigtig blød udvikling, hvor man altså i lang tid bare takker for donationer samt giver et par goder til gode medlemmer af fællesskabet (og ikke prøver at forhindre og "piske" freeloaders rigtigt, men bare lader sem være).. Hm.. Hm.. ...Jeg overvejede lige lidt, om mon ikke man kunne lave et semi-lukket system, men nu kom jeg i tanke om, at handler det ikke bare om, at have P2P-server-netværk, hvor servere kan blive enige om algoritmer for, hvilke brugere, de vil servicere, hvor meget og med hvad..? ..Hm, så et semantisk web, hvor ikke alt indhold nødvendigvis kan tilgås af alle..? ..Hm nej, ikke rigtigt, for det er for nemt at kopiere indhold løbende... Tja, men det kan man jo også fra f.eks. youtube, kan man ikke..? Hm, og man kunne vel ikke have et sem-web, hvor brugerne ejer deres bidrag; bliver det ikke også lidt underligt..? (Bemærk, lige nu genbesøger jeg bare gamle spørgsmål og idéer, men det er jo også sundt nok at gøre indimellem.) Hm, og det hjælper ikke rigtigt på problemet, for spørgsmålet handler ikke om refærdig løuddeling (det tror jeg nemlig ikke længere bliver et særligt stort problem i praksis, for folk er genrelt glade for retfærdighed --- hvilket var, hvad jeg lidt indså i går aftes, lige for at gøre den tråd færdig), men om at få folk til at donere.. ..Hm, nu kom jeg lige på en lidt vild (måske far out) idé omkring noget med arvtagere, men lad mig lige prøve at tænke over mulighederne.. ...Uh, er der egentligt ikke en god mulighed i bare at benytte visse særlige bidrag i foreningen (som kan opnås ditributionsrettigheder til) til at lokke abonnenter til..? ..Ah, og ift. software, så kan man vel godt have copyright-licenser selv på åbne gitmapper, kan man ikke?? Og så kan man jo i foreningen bare eje rettighederne til software i github-mapper med licenser tilknyttet, hvorved arbejdet på softwaren kan ske i et åbent fællesskab, men hvor resultatet kan være ejet af foreningen.. Og på den måde kan man så som forening løbende samle noget indhold, som man har eksklusive rettigheder til, og som man så kan bruge til at lokke abonnenter til med..(!) Hm.. Og så er hele tanken nemlig bare, at foreningen dog ikke internt skal værdsætte de bidrag, der har rettigheder med sig, i særlig høj grad over andre bidrag, men at kundernes vurderinger af værdien af diverse indhold bliver mere bestemmende for lønfordelingen.. Hm.. :) ... Nå ja, og så giver det egentligt også sig selv, at donorerne bare kan bestemme i hvor høj grad man skal prioritere at belønne bidrag, som foreningen kan få eksklusive rettigheder til, hvis donorerne altså ønsker at gøre, at flere ikke-donorer kan tvinges lidt til også at give lidt, hvis de vil have adgang til alt indhold. :) ..Og så giver de der "11. juni-idéer" (som jeg endnu ikke har fundet et navn til) også lidt sig selv, for selvfølgelig skal man ikke spænde ben for sig selv og/eller prøve at tilbagekræve penge, hvis man lige uheldigvis kommer til at fordele lidt for meget til en gruppe bidragsydere til fordel for andre.! For nu er hele systemet jo bare frit og demokratisk (uden nogen bindinger!), så det giver alt sammen sig selv.! ..Ja, nice!.. ..Ja, nu tror jeg faktisk ret meget, at min 'kundedreven virksomhed'-tanke bare kan startes som en donationsforening, hvor donorerne for (semi-)demokratisk bestemmelse over den løbende lønfordeling alt efter størrelsen på deres bidrag. Og finten ved at donere igennem foreningen i stedet for at donere direkte, er så bare, at bidragsydere så dermed bedre kan stole på lønsatserne, hvilket netop er vildt vigtigt, når vi taler "bagud-belønning." Og hvad gør man så med ejerskabet over rettighederne (med andre ord de "aktier", man opnår i foreningen)?.. Kunne man ikke bare låse dem til foreningen, hvorved tanken så er, at denne beslutning kan lokke flere bidragsydere til, fordi "aktierne" så bliver brugt på.. andre bidragsydere.. medmindre donorerne bare beslutter, at de selv er vigtige "bidragsydere"... Hm.. ..Hm, tja, nu kan jeg mærke, at jeg tænker i de samme baner og træder de samme spor, som jeg også har gjort flere gange før angående dette emne.. ..Hm, nu fik jeg lige en idé.. Hvad med, at donorer, eller rettere investorer, med deres bidrag hver især skal købe halvdelen af rettighederne til et stykke arbejde... Hm, jeg kunne uddybe mere, men lad mig lige gå lidt videre.. Hvad med at jeg bare gik tilbage og sagde "ingen betalingsmure, ingen aktier?".. ..Og måske at, når det kommer til ting, der kan fås rettigheder til, at ophavmændene bare bevarer rettighederne, men gør det tilgængeligt for donorerne til foreningen..? ...Hm, kunne man gøre noget med, at bidragsydere skal gå med til et beløb, som foreningen kan købe bidraget for, hvorefter de så udliciterer bidraget til foreningen imod løbende donotionsudbetaling (pr. den donor-bestemte fordeling), og hvis de så ikke synes, de får nok, så kan de så sige, at de vil aflyse licensen, hvorefter foreningen har mulighed for at købe bidraget, fjerne det pågældende indhold eller prøve at forøge donations(fordelings)lønnen til de pågældende bidragsydere (selvom dette dog kræver, at donorskaren går med til dette (hvilket bestemmes på (vægtet) demokratisk vis)).. Ville dette ikke være meget godt..?.. ..Hm, men hvad er meningen med dette, hvis ingen skal eje noget?.. Nej, er idéen, altså denne version af den, ikke bare at droppe al ejerskab og så bare have en donationsforening..? Hm ja, for "problemet" ligger i, at grunden til at opnå rettigheder egentligt bare handler om interessemodsætningsforholdet mellem gamle donorer og nye brugere.. ..Hm, tænkte lige noget med, om man kunne lave cut-offs, hvor alle donationer før et nyt cut-off, kan forventes at tilbagebetales med en ny faktor til forskel for dem efter... ..Hm, men i forhold til dette giver intet af det rigtigt mening, hvis man bygger det fra, at bidragene bare er åbne. ..Ah..! Nej, skal idéen ikke bare være donationsforeninger (hvorved man kan opnå stabiliserede lønsatsmodeller i modsætning til, hvis folk skal donere direkte), og så kan det bare lige krydres med, at donorer kan matche mindre donorers beløber i større eller mindre grad for på den måde at give incitament til mindre donorer om at donere mere. Ja.. Bum..

%(22.08.21) Jeg søgte lige lidt på igen, om ikke det allerede er en gængs tanke, at man bruger trust'ede kilder/instanser til at udregne f.eks.\ gennemsnit eller andre aggregater over brugeres ratings, og jo, jeg fandt da en sætning i en artikel der nævnte gennemsnit af "andre bruger-ratings" også, nemlig andre end bare 'trust' og 'credibility,' i forbindelse med emnet 'web of trust.' Så ja, dette er altså heller ikke noget, jeg skal præsentere som "mine idé" og/eller som en "ny idé." Jeg har alt i alt derfor bare nogle "tanker" om, hvad man kan bruge SW-teknologierne til. 

%Hm, det var da ikke en helt dum idé, den i går med cut-offs.. Hm, som jo dog er en idé jeg har tænkt meget over før.. (Og tegnet som en aftagende kurve med vertikale streger igennem i mine papirnoter..) ..Men kan man regne med, at folk vil bedømme arbejde retfærdigt i dette system..? (Har jeg mon før undervurderet muligheden for, at folk bare kan motiveres af deres lyst til retfærdighed, nok til at systemet kan fungere..?) ..Hm, nå nej, vent, min nuværende idé omkring en sådan valuta er jo nok endnu bedre, hvor man bare medregner risiko, når lønnen skal bestemmes. Og så kan værdi-papirerne bare handles og så indløses til hver en tid (så man kan også holde på dem og satse på, at de stiger i værdi), så snart at virksomheden/foreningen har kapital nok til det (og bare med et først-til-mølle-princip). ..Ah, og man kunne endda lige tilføje en... tilfældig udtrækning.. men i virkeligheden ville en bedre idé nok bare være, at sætte en maks grænse på, hvor længe man må holde på værdipapirerne, så man stadig godt kan holde på papirerne og satse på, at de stiger i værdi, men kun indenfor en tidsramme, hvor samfundet ikke nødvendigvis har udviklet sig fuldstændigt, så måske denne tidgrænse kunne være et århundrede (så vi altså kommer til at snakke om nedarvede værdipapirer primært) eller et halvt.. eller halvandet.. Ja, og er der ikke bare det, min lykke-valuta-idé er? Og så never mind krypto og/eller det med at bruge en masse krudt på at definere systemet juridisk; det kan sagtens i stedet bare tage lidt løbende --- for idéen baserer sig jo på en rimelig simpel tanke, og det er heller ikke svært at opdage, hvis bevægelsen pludselig begynder at handle ulødigt, og så vil den jo bare miste opbakningen. 
%Ja, så min lykke-/gavn-valuta-idé er faktisk virkeligt simpel. Selvfølgelig skal man også lige nævne ting angående, hvordan man håndtere variation i, hvordan folk har lyst til, at lønsatsmodellen skal være i samfundet, og i øvrigt kan man også give plads til, at folk får lov at belønne ting, der kommer dem selv til gode, mere end ting der bare kommer folk til gode generelt (hvilket faktisk er meget vigtigt). I øvrigt bør man også måske kunne trække lønninger ned, hvis der hersker mistanke om, at vedkomne kan have været med til at starte det problem, der løstes (hvis man f.eks. ligefrem starter en katastrofe af en art, som et ekstremt tilfælde). Men det gode er, som jeg tænker idéen nu, at alt dette bare kan findes ud af løbende. For hele valutaen er nu helt fri og er bare en overordnet og muligvis rimelig implicit aftale imellem folk, der deltager i en bevægelse. Så alle sådan nogle forhold kan man altid bare implementere løbende; så længe man bare ikke bevidst prøver at implementere forhold, der giver de første bidragsydere til bevægelsen mindre end, hvad de på en måde var blevet lovet ved bevægelsens begyndelse. Så ja, rimeligt simpelt..! Og min kundedrevne-virksomheds-idé er i øvrigt også super simpel: Fra iværksætters synspunkt er den jo sådan set bare, at lokke kunder til et firma ved at begrænse sine egne aktier og sin egen magt på længere sigt, og altså udlove denne fremtidige magt og disse fremtidige værdier til kunderne. Det er i øvrigt en ret interessant ting, at denne idé faktisk kan blive brugt som et argument emph{for} kapitalismen og det frie marked, for "hvis der virkeligt var et problem med det kapitalistiske system, jamen så ville der jo bare være nogen, der fremførte sådanne virksomheder/foreninger, men hvis ikke der er nogen, der gør det, jamen så må problemet jo være særligt signifikant alligevel." Og hvis folk fremfører denne vikrsomheds-/forretnings-type, jamen så kan man jo sagtens argumentere ret godt for, at det er i henhold til kapitalismen (hvad jeg i øvrigt nok også egentligt ville sige). 
%For at gå lidt mere tilbage til emnet omkring det semantiske web, som denne sektion jo bør handle om, så skal jeg lige brainstorme over en tanke, jeg fik i går aftes, nemlig at serverne i sem-web-systemet jo også kan ses som bidragsydere. Hm, og hvordan kan jeg lige bruge denne tanke (hvis jeg kan)?.. Tjo, men hvis jeg begynder at tage min lykke-valuta mere seriøst, hvad jeg vel bør gøre, og altså fremfører idéen som en central løsning omkring donationssystemet, jamen så giver alt sådan noget jo sig selv.. Tja, og det gør det egentligt også alligevel; selvfølgelig er det smart at tænke server-services som 'bidrag' til systemet, det er klart..
%Okay, jeg bør lige lave en opsummering over mine økonomi-relaterede idéer, og så opsummere i samme omgang (lige efter), hvad jeg nu tænker omkring mine sem-web-idéer (så altså det, jeg var ved at beskrive her i denne sektion).. :
%Hm, nu har jeg lige skimmet mine sektioner under "Øvrige tanker [...] økonomiske [...] emner"-oversektionen, og det er altså virkeligt vildt, hvordan det hele nærmest bare er kollapset... Og mærkeligt nok kan jeg sige dette nærmest helt uden negative følelser, hvilket i sig selv også er ret vildt. Men disse noter har jo også bare været til for at jeg skulle få styr på en hel, hel masser tanker og idéer, der fløj rundt i hovedet på mig fra i sidste sommers af (og mange af tingene også inden da), og som krævede, at jeg arbejdede videre på dem. Og nu står jeg, foruden selvfølgelig min eksistensteori, QED-teori og plus det løse i de sidste sektioner her, med en lykke-valuta idé, som er ret simpel og nice, en (kundedreven) forretningsidé, som sikkert også er rimelig nice, og eller har jeg bare nogle overordnede idéer til, hvad folk kunne opnå økonomisk, hvis de samlede sig og var lidt mere aktive politisk i forhold til deres samlede økonomiske fællessituation og i forhold til deres forbrug (og også deres data, som en lidt mindre (og også mere mainstream) ting).. Og ja, så har jeg jo mine tanker omkring, hvad der viser sig, selv når det kommer til mine "p-ontologier," at lægge sig virkeligt tæt op ad de gængse tanker om det semantiske web. ..Og det er faktisk bare super nice..(!..) Det føles faktisk virkeligt godt.. En lettelse og en glæde.. Det hele føles bare meget, meget mere simpelt nu, og det er ikke fordi, jeg ikke stadig har nogle interessante idéer, som er værd at dele. Og ja, uanset hvad, så vil jeg altid have emnerne omkring det semantiske web og argumentations-/diskussions-principper og videnskabsteori generelt, som jeg, sammen med teorietiseren omkring, hvad skaber lykkelige liv og lykkelige samfund --- i vores nutid og i fremtiden, vil kunne finde nok kød på til resten af mit liv, hvis det skulle blive nødvendigt (i.e. hvis jeg ikke også finder andre interessante ting at beskæftige mig med, hvilket dog slet ikke kan forestille mig :)). Så det ser altså lyst ud, hvilket det i øvrigt hele tiden har gjort, men nu er det lige pludselig på en meget mere afslappende måde..! Jeg tror nemlig ikke nu, at jeg kommer til at skulle kæmpe særligt, for at få mine nuværende idéer ud, og det er jo bare \emph{skønt}..! Hm.. Nå, men jeg mangler altså stadigvæk at få styr på, hvad jeg stadig har af brugbare idéer omkring det semantiske web, så det vil jeg gå i gang med at tænke over nu.

%(23.08.21) Okay, i går aftes fandt jeg på en vinkel på mine sem-web-idéer, som jeg virkeligt blev glad for. Den er sådan set bare den, at det semantiske web kan startes som en web 2.0-side med frie rating-systemer (og hvor ratings kan vises som annotationer på specifikke tekstudsnit) og også frie algoritmer, hvor man så altså (som jeg har tænkt på før (i.e. når jeg har nævnt "tiltag-kanaler")) kan følge en slags "kanaler" a la YT-kanaler, men hvor kanalen tilbyder indstillinger til (open source) platformen. ..Hm, og hvordan skal det nu fungere mere præcist..? ..Nå ja, og jeg kan også nævne, at der jo bør være mere fokus på sammenhængende tekster, og dette kunne man (kom jeg lige i tanke om nu) måske netop også føre videre til, når det handler om et feed... hm, tja, både og.. Et feed skal selvfølgelig være personligt, men man kunne måske tænke mere i kompilationer.. Tja, nå, men jeg må lige tænke lidt over den overordnede tanke (denne "vinkel") generelt... Hm, nu har jeg lidt følelsen, at der måske ikke er helt ligeså meget kød på idéen, som jeg tænkte i går..(?) Tja, eller på den anden side, så er det vel den mest oplagte vinkel..(?).. 
%Jeg fik lige \emph{Idéen!!} imens jeg var ude at gå en lille tur nu her..!!! Eller dvs., det er en gamel idé langt hen ad vejen, og det var også i løbet af de sidste dage, at jeg begyndte at tro på (en ret simpel udgave af) idéen igen. Vi snakker min lykke-valuta-idé, som jeg nu faktisk tror vildt meget på.. Det er sjovt, for det er virkeligt nogle få og små ting, der på en måde ændrer billede fuldstændigt.. For det første kom jeg jo i tanke om, at hvis dealinen for at fastsætte aktieværdien og udbetale den (hvis muligt) er på de omkring halvandenhundrede år, jamen så kan det ikke lade sig gøre rigtigt at diskriminere de tidligere generationer af bidragsydere, for de fremtidige mennesker, der skal bedømme kursen i sidste ende, vil ikke have nogen grund til at lave denne diskrimination. Bum! Og så kan det hele jo bare køre på gentleman-regler..! Og hvad var det mere, jeg tænkte?.. Nå jo, og valutaen skal heller ikke forsøge vildt på at erstatte normale valutaer..! Og dette fjerner en del af de potentielle cirkelslutninger... eller "cirkelmekanismer," om man vil.. Og man skal jo bare sørge for, at dommerne, der afgør lønnen i sidste ende (i fremtiden) er upartiske, hvilket kan gøres bedst, hvis det er en demokratisk beslutning. Og så må man bare sige: never mind alt pjat om at se bedømmelsen som et bidrag. Bedømmelsen er demokratisk og kommer bare til at være motiveret af, hvordan man gerne vil have de næste hundrede eller mere år til at forløbe (fordi man så sætter præcedens for værdibedømmelserne af de forskellige typer bidrag i et samfund). Og, også vigtigt nok, motivationen bliver også bare: retfærdighedsfølelse, hvilket man altså ikke (det gør jeg ikke længere) skal underkende hos mennesker. ...!!!!! (23.08.21)
%Og nu bliver dette nemlig nok min vinkel..!! Nå ja, det skal også nævnes, at.. nå ja, for det første, at jeg nu faktsik også tænker, at implementere valutaen som en krypto-kæde igen! :D Og på denne kæde skal faktisk være sem-web-tekster (altså tripletter (og/eller andre formelle strukturer; det er frit for), og sikker formuleret i xml til at starte med). Så kæden kan i starten også fungere som en lille sem-web-database. Sværhedsgraden af at mine'e blokke skal så bare være konstant! ..I hvert fald i starten, og så kan man altid ændre den senere (alle regler omkring denne kæde er nemlig lidt løse; det er gentleman-regler! :)). Og noget fedt er så, at der faktisk bliver en slags "mønter" i starten, fordi den hype minerne skaber jo faktsik er et vigtigt bidrag til bevægelsen. Så det er altså en slags selv-refererende værdi, men en værdi regardless. Så det er jo fedt! :) Og så tænker jeg nemlig næsten, at jeg bare kan nævne mine sem-web-idéer i forbindelse med denne idé..! Jeg skal måske lige tænke lidt mere over det, men pointen er lidt, at jeg jo ikke derfor behøver at binde idéerne helt op på LKV-idéen, fordi jeg bare kan nævne, at "jeg tror i øvrigt også, at disse idéer er gangbare i sig selv, også selvom bidragsyderne bare lokkes af de sædvanlige web 2.0-grunde, der er for folk til at uploade ting til et web 2.0-netværk." Her er en af idéerne nemlig så, at prøve at gøre det til noget socialt, når man bidrager med tekniske løsninger (f.eks. UI-indstillinger) til platformen, og at folk kan følge af like sådanne open source-bidragsydere. 
%Ja, og dette er vel en stor del af idéen sådan set, og så er der lige alt det med sammenhængende tekster *(især når det kommer til argumenter), simple point (folksonomy-ratings), at bruge dette til programmering og version control også, og... Og hvad mere (jeg har jo nævnt det med WoApps og at gøere det til en social ting --- hvis ikke en LKV-indbringende ting --- når man bidrager med programmerings- og/eller designløsninger til platformen..)..?  Jo, og at det kan startes ligesom en gængs web 2.0-platform.. Og det med at lave psykologi- og holdnings- ontologier (hvor jeg altså bruger begrebet, som om ontologier også indeholder selve deres data, hvilket jeg ikke er helt sikker på, er den gængse definition af ordet..).. I øvrigt skal jeg lige huske at nævne omkring LKV'en, at man også nemt skal kunne merge forks (hvor det så bare skal noteres eksplicit, hvis der er delte blokke i de to mergede forks, så man let kan skippe gengangere, når man traversere kæden baglæns).. ..Og er det virkelig bare det..? ..Nå jo og også: "blade med forslag til ny empiri," "fokus på kilder," "Model-orienteret argumentation," "street-credits" og "captcha-spil." ..Men er det så virkelig bare det..? ... Tjo, og så bør jeg også lige understrege 'semantisk version control med prædikater i grafknuderne' og så skal jeg også komme ind på 'ML-teknik og mange forskellige trust-/point-fordelinger.' Desuden skal jeg lige overveje, om ikke min "kraftfelt-idé" kan noget ift. et e-demokrati. I øvrigt skal også nævne, at man bør huske at fokusere på, at 'credibility er en ting, man løbende skal justere (på Bayesk vis).' Og det var vist det primære. Så kom jeg også lige på den idé, at hvis man skulle have web 2.0-agtige diskussioner, så kunne man måske i det mindste indføre et pointsystem til at rate folks diskussionsdyd i deres opslag hver især. Men jeg tror nu ikke, at sådan en frem-og-tilbage, replik-dreven diskussionsform bør fokuseres særligt meget på, så denne lille idé kan bare blive herude i kommentarerne. Og lige for at vende tilbage til LKV'en, så skal løndommerne altså aldrig, som en del af grundsætningerne, fokusere på, hvem der nu er kommet til at eje værdipapiret, men kun på, hvad værdipapiret dækker over. Dette princip vil nemlig altid være værd at overholde (medmindre man finder på et helt nyt system (hvorved man så dog stadig alt andet end lige vil være retfærdigt indstillet over for det gamle)), for ellers kollapser hele valuta-systemet bare (og med god grund). At dommerne heller ikke må diskriminere i tid er i øvrigt noget man kan tjekke (at de ikke gør rettere) ved at se på, om de daværende kurser passer med de på samtidige lønsatser, som deles ud; hvis ikke de gør det, så er der nok noget uldent på færre. (Lige for at præcisere: Hvis der ikke er noget uldent på færre, så skal kurserne gerne presses ned, hvis lønsatserne uddeles med en lavere kurs for tilsvarende bidragstyper.) Okay.. (!!! :D)   

%(24.08.21) Man bør også indføre (hvilket sagtens kan gøres efter kædens start) protokoller for at bevise, at man lagre visse dele af kæden, så man dermed også kan danne nogle PoStorage-"mønter" (dvs., det er ikke mønter, det er selvfølgelig noterede bidrag ligesom alt andet værdi på kæden). Jeg fik i øvrigt ikke nævnt, at det stadig giver super god mening, hvis man kan opnå IP-rettigheder til sine bidrag --- måske ved at arbejde sammen i delvist lukkede (hvis det da er nødvendigt) --- og så bare eventuelt udlicitere brugen af det rimeligt kvit og frit. Pointen er at selvom man gerne vil have sine bidrag noteret som gode bidrag, så kan det jo godt være, at man pludselig finder ud af, at man godt kunne bruge kapital til et nyt projekt (eller til at udvide det gamle self.), jamen så er det jo kun godt at holde muligheden åben for at kunne tjene kapital op på sit tidligere bidrag. Og hvis det nye projekt er velovervejet og lovende nok, jamen så må denne handling jo tælle som et godt bidrag. Jeg skal også lige ind på emnet omkring "negative bidrag" (selvom det bør hedde noget andet..), f.eks. når man modtager gavn af et bidrag. Dette skal helt klart være en del af valutaen, så man bør altså sigte efter allerede fra start af, at få det mere og mere som en del af bidragsregisteret, hvem der modtog bidraget. Disse individer kan så få en bidrags-aktie med negativt fortegn på værdien.. lad os bare kalde det en modtagnings-aktie.. eller -regning.. hm, eller bare kalde det en regning ligeud.. Her skal der så gælde, at hvis man på et tidspunkt... Hm.. Eller skulle man bare sige, at folk \emph{kan} slå deres regninger sammen med aktier.. Hm, udover at det rejser nogle spørgsmål, så er det også bare problematisk, hvis der ikke er noget indbygget system, der tvinger folk til at gøre dette, for så går der jo lang tid, før man kan få denne del til at virke.. ..Hm, og vi gider i øvrigt (som en side-note) ikke have et system med arvesynd.. ..Hm, men er det ikke bare at sørge for, at aktier altid kan ordnes indbyrdes (hvilket er trivielt at gøre) og så bare have en protokol, hvor regninger automatisk klistres på alle aktier, som vedkommende kommer til at eje.. nå nej, så skal det jo lige præcis være i rækkefølgen af, hvornår vedkommende kom i besiddelse af aktierne. Så hvis man har gæld, så vil alle ens aktier blive 0 værd (hvis alle ellers er enige om at gælden overstiger besiddelserne, hvad ikke behøver være tilfældet), og en ny-erhvervede aktier (købte eller i form af ny-noterede bidrag) blive gælden mindre værd --- eller blive 0 værd, hvis gælden stadig overstiger værdien, hvorved processen fortsætter. Og at der så kan være helt uenighed omkring, hvornår gælden er betalt, det gør ikke noget; det er bare en del af systemet, nemlig at der er risiko på værdierne i form af, at man prøver at forudsige, hvor meget bidrag er værd, samt hvor store regninger bør være, for at det er fair. Og lige en ting omkring regninger, så bør held og uheld figurere som en del af det bestemmende, når det kommer til det skyldte. Jeg mener således f.eks., at hvis nu en person bliver tilfældigt udtaget til at modtage en vis gavn fra et eller andet event og/eller en uddeling af goder, og hvis alle mennesker på hele jorden havde samme chance for at blive udtaget, jamen så bør gælden faktisk være nærmest 0 --- den bør i hvert fald ikke overstige, hvad selv et fattigt menneske sagtens kan betale. For hele pointen med regninger er bare at udjævne "handlingsgrafen"/"bidrags-grafen" (di-grafen), så at bidrag flyder mest muligt rundt i grafen uden at der er grupper, der udgør markante og længerevarende sources eller sinks i bidragsflowet, men at alle sources og sinks bliver bedst muligt modsvaret på et senere tidspunkt (så de selv kortvarigt bliver sinks eller sources henholdsvis (så altså byttet om)). Og herved er der jo ingen grund til at straffe held eller uheld i nogen særlig grad, især ikke hvis det i høj grad er noget der kan "ske for os alle." ..Bum. ..Jo, og lige angående hype, så kan det jo være smart at bede en upartisk gruppe om at vurdere løbende, om bevægelsen har brug for mere kryptovalutakurs-hype eller mindre, hvorved den efterfølgende kurs så kan stige eller falde (givet at folk stoler på gruppens vurdering). Dette kræver så en hel del forudsigningsevner, så det kan godt være, at der går en del tid før man kan tæmme hype-kurserne på denne måde, så i starten må de bare passe sig selv mere eller mindre, og så må man bare håbe, at den pågældende hype kommer bevægelsen mest muligt til gode (hvilket den jo vil, hvis folk samlet set har bare en lille smule fornemmelse af, om hypen er god i dens nuværende mængde, eller om den er for meget). Og så er der nogle, der vil vinde penge på denne opstart, og nogle, der vil tabe penge, men sådan er det jo bare. ..Ej, jeg kan næsten ikke tro, at denne idé igen ser så stærk ud.. Det er næsten for godt til at være sandt. ^^ He, det skal det sikkert nok være på en eller anden led, men forhåbentligt ikke i særlig høj grad; forhåbentligt er den god nok overordnet set. :) 
%..Ah, og angående hype, bør jeg faktisk også lige nævne, at hype-skabelse kan tage rigtigt mange former. Ja, al aktivitet omkring kæden kan nok med fordel noteres som et hype-skabende bidrag i starten. Og noget vigtigt at tænke på i denne sammenhæng er dog så, at hypen helst skal hænge sammen med en efterfølgende teknologiesk udvikling af bevægelsen. Denne udvikoing kan dog bare være, at man får flere mennesker med på idéen (eller holder ild i idéen), men derfor er det stadig meget godt at notere sig, at hvis hype-kursen stiger, men at der ikke sker nogen teknologisk (eller foreningsmæssig) udvikling af bevægelsen, så er der altså noget galt, og så bør folk altså helst sænke deres forventninger til den pågældende kurs. ..Åh, jeg håber btw, at det er klart, at alle kæde-relaterede bidrag er tidafhængige. Ellers må jeg gøre det nu: Der er ikke mønter som sådan på kæden ligesom hos BitCoin og hvad har vi, der alle har den samme værdi. I stedet udgør mining på kæden, at man kan notere sit bidrag på kæden, og værdien af dette bidrag kan så i sidste ende afhænge af en masse faktorer, bl.a. hvornår bidraget gjordes, ikke fordi man må diskriminere ift. tid på et abstrakt plan, men fordi der kan vurderes at have været mere eller mindre behov for pågældende bidrag på det pågældende tidspunkt (hvilket man jo selvfølgelig godt må vurdere et bidrag ud fra --- det man bare ikke må gøre, er at nedvurdere et bidrag frem for et andet, hvis behovet (og alle andre faktorer) var af samme størrelse til de to tider). Så det korte af det lange er, kursen af kæde-relaterede bidrag bør følge en vurdering af, hvor meget udvikling af bevægelsen, som teknologi og/eller som fællesskab, bidraget var med til at medføre.   



















%[slet ikke nogen vild tanke], og der skal nok være andre, der har leget med samme tanke. Men jeg har søgt på diverse vidensdelingssider/wiki-leksika, og jeg er ikke stødt på noget, der minder om denne tilgang. Så måske er idéen altså nytænkende... %Hm, måske skulle man så starte fra pointene..?.. ..Eller er det fint at starte med at beskrive mit wiki-alternativ..?.. ...Hm, lad mig egentligt lige søge noget mere på (aggregat-)algoritmer over sem-web-annotationer... 
%Hm, jeg kom frem til i går, at mine simple point, hvor man kan give udtryk for generaliserede holdninger til en tekst eller et tekstudsnit, og hvor pointsystemet altså er væsentligt mere kompliceret og rigt end gængse pointsystemer på nettet, men hvor man stadig nemt kan aggregere pointene til brug i algoritmer a la, hvad man ser på web 2.0-sider, men nu fik jeg lige den her tanke, at måske kan man ligefrem ophøje det til at være vejen til en glidende overgang fra web 2.0 til web 3.0... Hm.. ...Ja, det kan man vel. Og ja, det er værd at fokusere på den pointe.. 


%Den næste del af idéen, som måske også kan siges at være nytænkende\footnote{
%	Det er der nemlig meget af denne samlede idé, der \emph{ikke} kan. Meget af min idé her lægger sig nemlig meget op ad gængse tanker omkring det semantiske web (om end de ikke nødvendigvis er totalt centrale for den gængse vision, der ser ud til at være meget orienteret omkring data, i.e.\ skalart data, og ikke helt så meget omkring forklarende tekster og om internet-``indhold'' generelt, som jeg kunne ønske mig).
%}

%At folk så også kan give hver enkle disposition (inklusiv underdispositioner) eller atomare tekst point lidt ligesom på diverse kendte web 2.0-sider, men bare ud fra decentrale og brugerdrevne pointsystemer, gør også rigtigt meget for idéen, mener jeg. For decentralt system bliver vigtigt, når man blandt andet gerne vil begynde at indføre tekstprædikater, der siger noget om, ud fra hvilket synspunkt teksten er skrevet. Brugergrupper af forskellige interessegrupper, befolkningsgrupper eller politiske grupper generalt kan således lave deres egne pointsystemer, hvor de vægter hinandens%... Hm, jeg tror nu lige jeg hellere bør snakke om det basale ved pointene og "mine simple pointtyper" først. (Også fordi det er en mere novel idé, hvor det her er mere gængst.)




%\subsubsection{} %til hurtig navigation %"nivi" xD








%Hm, kan  man ikke også sælge det bare på, at industrien kan opnå mere forbrugerdata, hvis de fonder idéen?.. 


%Pointér det her med, at der skal være mulighed for at blande pointsammen, så man f.eks. kan vægte en rating med populariteten.

%Husk alligevel det med fortolkningerne af tekster versus RW-ting og det..

%De afhængige point skal også kunne tage termer som input ...



%Sammenspillet, hvor brugeren søger i og navigerer rundt i ontologien (ved at forstærke visse point bl.a.)..

%Hvorfor det faktisk skal være anonymt i det grundlæggende lag.

%At brugerfællesskabet kan designe point til at maskere faktiske interesser (og danne et kunstigt billede), således at de bedre selv kan sælge deres egen brugerdata. 

%Eksempel med video-/spil-point, hvor man kan komme længere, hvis man har mere end bare konventionelle ratings (og folksonomies).   

%Hvor vigtigt det bliver med det alsidige pointsystem (og at det så kan bruges som plug-in..)..
%Eksempel/ler med programmering..

%Eksempel med videnssamfund og sem-dokumenter!

%Find eksempel/ler på søgningsfordele...
%(Husk at understrege, hvor vigtigt det bliver med brugerkategoriseringer osv..! (også godt i sig selv med flere estimatorer end bare gennemsnittet hele tiden.. (e.g. kult-ting..)))

%HTML (markup) centralt..
%Konventioner..

%(Copyright/left og korrekt forfatterskab...)

%Husk at understrege, hvor langt (længere) man komme selv bare med rigtigt simple typer point (som bare lige fortæller lidt mere end nutidens konventionelle point, man støder på rundt omkring).

%Husk at nævne alt det med brugerspecificerede applikationer, for man behøver jo ikke matematisk dokumentation, før man kan komme i gang med åbne applikationer..!! ..Nej, det gør man jo nemlig ikke! :D ..Og så kan man hurtigt implementere point, som de ledende applikationsudviklere vil stole på (det kan de altså selv finde), så man herved så småt kan få automatiske opdateringer i spil.

%Nævn måske politisk ontologi-diskussion..

%DM-afstemning:
%repræsentanter..
%Det der med: ikke boost-størrelse, men metrik.



%Denne stikordsnote hører ikke til her, men burde jeg ikke nævne noget om mulighederne med "fremdtidsception" for et retssystem generelt, samt for regerende enheder generelt (og uden at det nødvendigvis har noget med (NL-)KV-semantik at gøre)..? Det kan jeg nemlig ikke huske, at jeg har været inde på, og det tror jeg da umiddelbart kunne være ret godt..

%Hører heller ikke til her, men jeg kom lige i tanke om, at tanker omkring frie valutasystemer jo hører ret godt sammen med folkelige løfte-systemer, så lad mig lige have dette i mente, og tjekke at jeg får fremhævet overvejelser om dette i notesættet, hvis det giver mening..

%Og endnu en overordnet note: Skal jeg prøve at skrive om min idé til et PoW-kæde-angreb? (hvor man sælger en masse valuta og så lokker minere til med penge, at mine på en alternativ fork, hvor man stadig har pengene, og hvor man så også som en del af idéen har en pyramidespilsagtigt lokkestrategi, hvor man belønner rouge miners, hvis de holder en fod inden for på den rouge fork og belønner dem mere og mere, jo mere energi de ligger over i den rouge fork frem for den oprigtige..) Hm, eller måske kan det bare stå her i kommentarerne, for jeg har heller ikke regnet super meget på, om den virker, og jeg gider heller ikke rigtigt at gøre det.. *(Og dette angreb vil især give mening, når først andre teknologier og/eller instanser er blevet lidt afhængige af at bruge kæden, for så er der ikke samme fare for, at hele kædens værdi bare styrtdykker --- den mulige udsigten til hvilket nemlig kan skræmme miners væk fra at gå "rouge.") ..Hm, hvad var det nu forresten, jeg tænkte på i går eller i forgårs?.. noget med en kæde, hvor værdien ligger i, at man har krav på, at andre holder live i kæderne for en...? Hm... ..Hm, tja måske, men det vil ikke blive nogen særligt skelsættende teknologi, så det gider jeg heller ikke gå videre med.





\begin{comment}
Nu er jeg så kommet frem til, at jeg nok vil sælge idéen som en idé til en ny type socialt netværk (hvilket altså også giver rigtig god mening). En af de helt store styrker ved dette netværk bliver så, at man kan bruge det til at se passande vurderinger af indhold på internettet samt at få gode anbefalinger til ting fra ens netværk. En anden styrke kommer i form af det diskussions-/tekst-analyse-paradigme, jeg også vil lægge op til, skal fremføres af netværket. Og her vil jeg så foreslå programmering som et særligt område, der kunne få gavn af dette paradigme --- fordi det altså er et område, som er meget afhængigt af, at en masse argumenter bliver gennemgået for korrekthed af et fælleskab, og fordi det samtidigt er et stort kommercielt område, så der er også meget at vinde ved at forbedre det. Broen mellem disse to, i princippet adskilte, idéer er så, for det første at de begge kan implementeres via en semantisk database (hvilken jeg dog har i sinde skal implementeres over en relationel database), hvor brugerne selv bidrager med pointsystemerne og tilhørende serveralgoritmer/-filtre, og hvor hver en sætning i hver en tekst kan gives point, og for det andet at brugbarheden af det sociale netværk også selv kommer til bl.a.\ at bygge på, at brugerne i netværket bliver i stand til at dissekere og analysere --- og rette og udbygge --- hinandens opslag. 

Så lad mig starte med at beskrive, hvad det sociale netværk samt teknologien under det skal kunne. Der er et velkendt problem med gængse sociale netværksplatforme, i form af at disse platforme for at fange brugernes interesser bedst muligt benytter sig af algoritmer til dette formål, der så dog kan have tendens til at fange brugerne i såkaldte `ekkokamre,' hvorved at brugerne i højere og højere grad for serveret det, der passer med dennes eksisterende bias, og som følge af dette bliver dette bias så bare mere og mere forstærket. Dette er en naturlig konsekvens af at bruge algoritmer, der servere brugerne indhold, der passer til deres eksisterende interesser, og kan nok aldrig helt undgås i praksis for et socialt, men det hjælper bestemt ikke på problemet, at disse algoritmer typisk er styrede helt fra servernes side, og hvor det overvejende er ugennemsigtigt, hvordan de virker. I min idé til et socialt netværk skal alle disse algoritmer være helt åbne for brugerne. Ydermere skal brugerne faktisk selv være primusmotor på at udvikle algoritmerne, så de blandt andet kan undgå at algoritmernes foretrukne indhold kommer til at være påvirket af, hvad tredjepartsinstanser kan have betalt platformen for at fremhæve, hvilket man også kan beskylde mange gængse platforme for. Man kan i øvrigt også beskylde platforme for, har jeg hørt, med vilje at prøve at indfange brugernes opmærksomhed i længere tid, end brugeren måske egentligt ville foretrække, hvis indholdet var ordnet på en måde, så brugeren bedre kunne tage ting lidt i bidder og uden frygt for at gå glip af noget. Det kan være, at folk, der har studeret sociale meddier i dybden, kan bidrage med nogle flere eksempler, hvor mangel på gennemsigtighed og brugerstyring på algoritmerne kan være en ulempe for brugerens gavn af netværksplatforme. Men nu vil jeg gå videre til at understrege, at brugerne i min idé selvfølgelig også skal have mulighed for at uploade og udvikle mange forskellige algoritmer på platformen, så alle brugere ikke bare skal blive enige om én god løsning, men så alle kan få lov at bruge den algoritme, der passer bedst til dem. (Og selvfølgelig vil der i teorien være en fysisk grænse for, hvor mange forskellige algoritmer brugernetværket kan uploade og benytte samtidigt, men i praksis tror jeg ikke, at denne grænse vil blive noget problem.) 

Når vi så snakker om algoritmer til at finde frem til, hvilket indhold er interessant for brugeren, så bruger gængse (web 2.0-)platforme jo som bekendt pointsystemer, typisk simple systemer, hvor enten alle eller en vis mængde brugere (måske dem som har oprettet og benytter en konto) kan give en tommelfinger op eller, i visse tilfælde, ned, eller kan give en rating mellem f.eks.\ nul og fem (stjerner e.g.). Udover dette kan der i nogen tilfælde også være mulighed for at tilknytte et tag til indholdet, som så kan accepteres, hvis nok brugere stemmer på det samme tag, men det er vist også noget af det mest komplicerede, jeg kender til, når det kommer til brugervurderinger på web 2.0-platforme. Derfor er det nok værd at fremhæve tidligt, ...

%Hm, ja, man kan vel nærmest adskille det som en grundlæggende idé for sig selv, bare det her med simpelthen at udvide vurderingsmulighederne og så altså samtidigt gøre det muligt at vurdere hvert lille udsnit af teksterne (eller medie-objekterne) i stedet for bare dem hele..? Ja, det er det jo, for dette kunne jo faktisk godt fungere i sig selv som en idé, også uden brugerdefinerede algoritmer, og selv også uden brugerklassifikationer..!.. Godt nok kommer det til at spille ret godt sammen med resten, men ja, det kunne jo være en idé for sig, og \emph{er} en rigtig vigtig del af min samlede idé. ... Ah, men så \emph{er} det jo også to idéer..! Yes, så bør jeg altså dele dem op, hvilket også er meget dejligt. Nice nok.
%((05.07.21); dagen efter jeg skrev ovenstående) Det er rigtigt nok, at argumentationsanalysedelen kan stå lidt på egne ben, men det giver nu alligevel god mening, at behandle det hele lidt som en idé med flere trin, hvor jeg så bare starter med den del og bygger videre for til sidst at nå til det sociale netværk og ML-teknik-point.

\end{comment}





%Jeg har nu --- og det er blevet d. (01.07.21) i mellemtiden --- fundet ud af, at opbygningen af typesystemet med atomare termer, relationer/prædikater og point osv.\ skal være HOL-agtigt, idet det skal være opbygget i flere lag, hvor vi i bunden har substantiv-morfemer samt atomare tal-værdier og strenge. Oven på disse kommer bl.a.\ verber og adjektiver i form af relationer/prædikater. Men disse relationer/prædikater kan jo selv indgå som en slags substantiver i andre sammenhænge, f.eks.\ i ``at rejse er at leve'' eller i ``udholdenhed kan være bedre end styrke i denne disciplin'' (hvor vi så bl.a.\ på dansk skal lave adjektivet om til et substantiv, før det virker, men semantisk set handler de om at sammenligne to prædikater, og i praksis vil man derfor også have `udholdenhed' og `udholdende' gemt tæt på hinanden i morfem-ontologien). Så allerede inden vi når ud af domænet af tekst-sætninger, så skal der altså allerede være flere lag (og adverbier kommer så f.eks. i et tredje lag). Systemet skal dog faktisk ikke eksplicit være typet, hvilket bl.a.\ kommer til at medføre, at man godt kan vælge at bruge f.eks.\ morfemet, `hurtig,' både som adverbium og tillægsord, om ikke andet så indtil at man for delt morfemet op i to i ontologien. ...
%Brain: Jamen, så er det jo i bud og grund ikke HOL-agtigt. Jeg er ret overbevist om, at der ikke skal være noget type-system, når vi kommer op til de første relationer (og altså ikke bare er helt nede ved tekster i form af symbolstrenge (inklusiv tekster i mark-up sprog)). Men så er der bare ingen grund til at bringe HOL på banen, er der? Det bliver jo bare ligesom tripletter, bare hvor der også kan være flere indgange, som bl.a. skal bruges til point, og hvor prædikat/relations-parameteren ikke behøver at være atomar men også kan være et vilkårligt term (og hvor hver tupel i øvrigt også har en indgang beregnet til en bruger-/konto-signatur --- og hvor brugere så i øvrigt ikke bare skal kunne uploade tupler, men også point-definitioner og større tekster til databasen, der implementere systemet). Og dermed følger det så også, at prædikat/relation-inputtene selv kan være prædikater osv., og der er dermed altså intet typesystem (hvis altså vi ser bort fra de "atomare" termers indre opbygning). Og spørgsmålet er vel så, kan dette ikke skabe problemer for point-algoritmerne. Hm, vel ikke, idet disse ikke må være rekursive (hvad de nemlig ikke må; det har jeg vist ikke nævnt endnu)..? Jo, men hvordan skal man lige se om en algoritme er rekursiv eller ej, hvis der ikke er nogen typer? Det kan man jo ikke.. ..Medmindre at lige netop de "afhængige point" har typer, hvad man vel sagtens kan sørge for..? Og hvad så med relationer omkring pointene --- det har jeg jo skrevet ovenfor, vil være en god idé?.. Men skal de så ikke bare gøres typede, og så er det fjong?.. ..For point og relationer/prædikater skulle vel.. Hm, skal de nu også blandes sammen, eller hvad?.. Og hvorfor egentligt hægte point på selve prædikaterne; det er jo egentligt bedre, at de bliver hægtet på bag efter, og kan man ikke så netop klare sig fint kun med tripletter --- dog plus signaturer? (Altså jeg ser ikke rigtigt nogen grund til at begrænse sig til tripletter, men derfor er spørgsmålet stadig relevant..) ..Nej, vi skal enten have mere en tripletter, eller også skal der bare være typer til at skabe kompositte termer selv for de ikke-"atomare," således at man kan danne en liste-type af input eller danne prædikat/relations-skabeloner, hvor et specifikt prædikat (eller relation, men jeg bør næsten gå over til bare at kalde det prædikater hele vejen (også i øvrigt så det ikke forvirres med sæt-relationer..)) så kan dannes som en sammensat term..  Hvad er mon bedst?.. I fohold til database-effektivitet er tripletter (foruden signaturerne) måske meget fordelagtige så... Hm.. Ja, og det er måske faktisk endda også nemmere at arbejde med, og måske endda oven i hatten også mere effektivt at søge i, så... Jo, klart nok.. Så selvom systemet skal være typeløst, så skal der stadig være grundlæggende *(nej; brugerdefinerede) term-kontruktorer, så man kan lave f.eks. liste-typer (som så dog bare ikke får anden type end bare 'term') og sammensatte prædikater. Hm, men så skal de jo ikke være grundlæggende, de kontruktorer, men brugere skal selv kunne tilføje hvad end kontruktorer, de har lyst til. Hm, og konstruktioner behøver vel i øvrigt ikke at signeres, så... Eller hvad..? Jo, det er meget godt, at folk kan se, hvilken konto, der uploadede termen, hvis konto-indehaveren gerne vil have dette (og altså ikke bare bruger en anonym signatur). *(Tjo, men hvis termer alligevel først bliver "udgivet" så andre brugere kan se dem i forbindelse med den første relation, de indgår i, så kunne man måske spare dette væk?.. Ja, værd at overveje..) Men ja, fint nok. Hm, skal konstruktorerne bruge typer, eller fuck det?.. Det tror jeg faktisk rigtigt nok, man skal undlade (for slet ikke alle er vant til typer, og entydigheden af termerne kan jo bare bedømmes med point..). Ja: ingen typer (på nær algoritme-pointene, der altså skal have ordner). Og selvom der så bliver term-kontruktorer ad libitum, så skal brugerne stadig være mere end velkommen til bare at uploade hele tekster i vilkårligt mark-up og gerne især jo i HTML. For så er tanken nemlig, at man altid bare i fælleskab kan splitte det op i bidder efterfølgende, og her i denne forbindelse er det så nu klart, at dette bør gøres via disse brugerdefinerede constructors. Så jeg er hermed virkeligt ikke langt fra tripletter, men som jeg også tænkte over her under min gåtur for lidt siden, så bliver mit fokus jo ikke så meget på søge-effektiviteten osv. ved denne teknologi. Det bliver i stedet mere på argumentationsparadigmet omkring det. Idéen her skal altså ikke sælges så meget på det tekniske, men skal i stedet sælges mere på mulighederne for sådan et fælleskab, der arbejder sammen og at opbygge ontologier med tekster, hvis argumentationer dissekeres og overvejes. Ja, og så er det altså særligt mine argumentationspoint som f.eks. 'korrekthed' osv., der bliver i højsædet.. og så også det her med, at brugere selv kan uploade deres algoritmer, og at dette særligt kan bruges til at klassificere andre brugere, så man kan få vist de termer, man tror mest på, er relevante for en selv.. Ja, så selvom at algoritmerne med tiden skal kunne gøre alt muligt fancy i henhold til idéen (den gængse) omkring et semantisk web, så vil jeg måske nok sælge dem mere på, at de i starten kan sørge for, at folk så kan få sorteret skidt fra kanel (både ift. indhold og til argumentationspoint).. Ja, for én ting er, alt hvad det semantiske web vil bringe med sig af lækre fordele, men en anden ting er, hvikle fordele de første brugere kan få gavn af, og her tror jeg altså nok primært, det bliver i form af redskaber til at dissekere tekster i deres argumetationsopbygning osv., så man kan diskutere og analysere ting, inklusiv bl.a. programmer, som så i sig selv kan blive til stor gavn. Og hvis der så skulle komme hurtigere gang i de andre fordele, såsom at det måske bliver nemmere at søge præcist efter ting, jamen så er dette jo kun godt. Ok. (01.07.21) ...Hm, men kunne tekster ikke også godt uploades i tupler? For så ville det jo også blive nemmere.. ja! Så vil det blive nemmere at indsætte og erstatte sætninger, uden at der skal gemmes nye tekster hver gang (hvor den samme tekst alt andet end lige ville skulle gemmes flere gange, også selvom denne tekst er lang). Jeps! Nice nok. (01.07.21) 
%(02.07.21) Selvom at jeg rigtigt nok skal starte med at fokusere på idéen omkring et nyt argumentations/dokumentations-analyse-paradigme --- og altså et semantisk web der udspringer mere af behovet for at kunne analysere tekster og diskutere på en god og effektiv måde frem for behovet for at strukturere data på internettet (hvilket jeg nemlig kan se (for jeg læste lige en lille smule op på 'semantic web' i går), nok meget har været fokuset hidtil) --- så bør jeg dog helt klart også ret hurtigt lufte tankerne omkring mulighederne for at bruge ML-teknik-point til at klassificere sig som forbruger, både af produkter men også af digitalt indhold. Det er jo helt vildt, hvad platforme såsom reddit eller steam (osv. osv.) indeholder af tekt-kommentarer. Og for steam er det jo alle brugere, der prøver at hjælpe andre brugere --- også selvom at flere tusinde har skrevet på samme tråd før en. (Jeg browsede nemlig lige Steam i går og så på, hvor mange reviews der er til spillene i almindelighed.) Så der er så meget energi at høste, hvis bare man kunne semantificere alt dette en smule. Og det gode, som jeg lige kom til at tænke på her til morgen, at de fleste folk jo egentligt ikke vil have noget imod at dele deres forbrugsoplysninger, i hvert fald når bare de selv er herre over, hvilken del af deres forbrugsinteresser bliver delt, og hvilken en der ikke gør. Og der har vi så enddu en stor kommerciel vinkel potentielt set..! Hvis man kan lave en virksomhed, der arbejder på at opbygge at system af gode ML-teknik-parametre over folks forbrugstendenser og hjælper folk til selv at bidrage til at finde disse parametre/point osv., og så i sidste ende også får mulighed for at købe (..eller bare eje som en del af aftalen, hvis man kan slippe af sted med dette (hvad man sagtens kan så længe en konkurrent bare ikke melder sig på banen, som så betaler for oplysningerne (hvilket så vil være mit skjulte håb))) rettighederne til at sælge disse forbrugsoplysninger videre, jamen så har man vel en forretningsstrategi, der kan slå alle de virksomheder, der samler data i det skjulte. For hvis folk selv bliver herre over, hvad de deler, vil de foretrække dette langt over at virksomhederne bare samler, hvad end data de kan få fingrene i, og ved at gøre brugerklassificeringsalgoritmerne helt åbne, så brugerne selv kan deltage i udviklingen af dem, så vil brugere lige pludselig selv have rigtig, rigtig stor gavn af at dele den data, ikke mindst også fordi man så kan klassificere sig på tværs af platforme og forbrugsområder --- men altså også fordi klassificeringerne (med tusinder og tusinder af kreative mennesker bag) hurtigt kan udvikles til at blive virkeligt sofistikeret; langt mere sofistikeret end hvad gængse platforme kan opnå --- langt! (Og det kan jeg nemlig sagtens argumentere for.) Og dertil kommer så også, at man i dette system kan sikre anynomitet af brugerne meget mere effektivt. For i modsætning til hvordan det i praksis fungerer på gængse platforme (hvor man bl.a. som regel skal have en mailadresse tilknyttet), så kan man sagtens lave systemet anonymt selv i de grundlæggende lag, så brugerne ingen gang skal stole på, at virksomheden ikke levere deres oplysninger videre (hvorved deres bidrag så kan linkes typisk til en aktiv konto og/eller en aktiv mail). Man vil sagtens kunne lave et system, hvor alle brugere er anonyme helt fra grunden. Brugeres incitament for at opbygge en (anonym) profil bliver så ikke, at denne profil så kan fungere som et interface til at få serveret ting, der passer til ens præferencer, men bliver i stedet bare, at brugeren gerne vil hjælpe andre brugere i fællesskabet ved at bidrage til pointsystemet, hvad folk jo lige netop gerne vil! Så vi kan altså adskille behovet for at få serveret passende data selv med lysten til at bidrage til, at andre brugere (og også brugeren selv for den sags skyld) kan få serveret passende indhold (og måske vil lysten også typisk delvist bunde i, at brugeren har lyst til at påvirke folk til at prøve de samme ting som brugeren selv). Ja, så sådan en anonyme-brugere-men-åbne-algoritmer(-og-åbent-men-anonymt-data)-idé vil også virkeligt have et stort potentiale i sig selv..! Og at man så kan kombinere det, hvis man virkeligt vil sælge idéen til potentielle brugere (måske på bekostning af, at man så ikke selv kan ende med at tjene ligeså meget som investor/iværksætter), med at man kan betale brugerne for deres bidrag, således at der nu for brugernes sidde bliver al god grund til at benytte dette system frem for gængse platformes ratingsystemer og kommentar/review-tråde, og hvad har vi, gør jo bare idéen endnu mere solid. Så ja, den side af idéen må jeg også gerne fokusere på tidligt. 
%...Hm, men hvordan kan der være den kommercielle vinkel, hvis også dataen er åben?.. ..Hm, godt spørgsmål; det er er der måske heller ikke så.. Hm, medmindre man på en eller anden måde kunne gøre systemet lukket udadtil, men hvor det stadig er brugbart.. Hm, nej, det går ikke rigtigt. Den kommercielle vinkel bliver således først, når man laver en (over)forening/virksomhed, hvor folk så kan samles ud fra deres (fundne) forbrugs(-fælles)interesser, og begynde at handle med denne interesse.. (Hvor man så altså kan tage sig betalt som iværksætter ved at administrere de relevante handler og donationer..)  ..Hm, men hvis der nu på en eller anden måde var en forretning, der særligt gavnede af den åbne information.. ..Tja, det kunne vel være alle forretninger som synes, de har et bedre produkt end konkurrenterne, trods størrelsen af sidstnævntes popularitet.. Ja..! Så man kunne søge fonding hos sådanne virksomheder?.. ..Og måske også alle virksomheder, der mener at de kan levere et mere bæredygtigt produkt end konkurrenterne; "hvis bare forbrugerne ville indse dette."... Jo men skønt, lad mig nævne dette; lad mig nævne at man kunne søge fonding hos sådanne virksomheder. Ja, det er faktisk virkeligt cool, for mange virksomheder \emph{vil} sikkert også gerne gå i bedre og mere bæredygtige retninger, hvis bare der var opbakning til det, og vil derfor altså meget muligt reagere ret positivt på udsigten til, at denne type opbakning forstærkes i samfundet. Nice nok. (02.06.21)

%(03.07.21) Fortsat fra mine papir-noter (på post-it-lapper) så overvejer jeg altså nu, hvad der helt skal være de røde tråd igennem det, når jeg skal sælge idéen (på skrift f.eks.). Jeg kom lige i tanke om nu her, at man måske kunne starte med et fokus på 'et socialt medie, hvor brugerne selv bestemmer (og altså uploader) algoritmerne til at opdale og klassificere brugerne i brugerskaren,' og så kunne man måske også hurtigt få brugt diskussioner og argumentationsanalyser på banen i denne forbindelse. Lyder umiddelbert ret godt som en mulighed.. (Men jeg er ikke færdig med overvejelserne endnu.) En anden mulig indgang til idéen (som så måske er værd at fremhæve i sig selv uanset hvad, hvilket også er grunden til, at jeg skal fortsætte med disse overvejelser) kunne være ift. det semantiske web, hvor man kunne påpege, at en bedre løsning må være en, hvor alle kan bidrage, og hvor man ikke behøver at stole på nogen central bruger-/bidrager-gruppe. Hm, men det semantiske web er jo så data-fokuseret, og når det kommer til sådanne ting, er det jo egentligt fornuftigt og gangbart nok bare at gøre sig afhæng af tilliden til en sådan gruppe. Ja, så idéen omkring at tage udgangspunkt i 'et socialt netværk' er måske faktisk rigtig god.. Hm, og hvad er der ellers af muligheder?.. Nå ja, og så kunne man også starte fra 'et programmerings paradigme'.. ..Og fra et 'mere diskussionsorienteret vidensdelingsparadigme' og fra et 'indhold-vurdering og -anbefalings-netværk på tværs af platforme,' men disse to ting kan jo lige netop føres sammen, hvis man tager udgangspunkt i 'et socialt netværk.' .. Hm, og man kan nok ikke rigtigt indføre det første af de to uden at komme godt ind på det andet.. ..Ja, sgu. Så det bliver vist 'socialt netværk' først, og når jeg så kommer ind på diskussionsparadigmet, så kan jeg føre det videre på 'programmering(sparadigme).' Cool. Nu tror jeg faktisk næsten, jeg bare vil ordne nogle ting for resten af i dag (og så ellers se fodboldkamp i aften). 







%Hvilke nogle ting skal man så dele i dette netværk? Jo, svaret er jo selv sagt alt muligt i bund og grund. Men det vil nok ikke være nogen dum idé, hvis termerne alligevel begrænses til en undermængde af (sanitized) HTML. Ved at bruge HTML bliver det så nemt at implementere et interface til at se de del-ontologier, som man får serveret. Og når det så kommer til fil- eller medie-indhold kan serverinstanserne så vælge, om de selv vil gemme kopier af disse, så de kan serveres af selve serveren, eller om man bare vil give et eksternt link til det refererede objekt. Der vil altså ikke være noget krav i det grundlæggende lag om, at serverne af systemet skal gemme og servicere, hvad end fil- og medie-indhold der er refereres til i de uploadede tekster. 

%HTML-termerne må endda gerne kunne indeholde udfyldningsformularer osv., hvorfra brugeren ...Hm...






%...
%
%Og så kommer også det vigtige spørgsmål: hvordan kommer brugerne til at kunne navigere rundt i ontologierne, og hvor starter man, når man vil finde frem til noget? Her forestiller jeg mig, at serverne i første omgang supplere en simpel oversigt over populære begyndelses-termer samt populære point til søgningerne og også giver en normal søgefunktion, så nye brugere kan søge på termer via normale søgeord, hvorved serveren kan give forslag ud fra en speciel sprog ontologi, som brugerfællesskabet naturligvis vil bygge op over de forskellige udtryk, der er relevante for den data, de ellers uploader. Ja, så denne sprog-ontologi, som vil være en del af den samlede ontologi, bliver altså selv sagt meget central. Herved kan nye brugere så hurtigt komme i gang. Bemærk, at da termerne (som altså her referere til tekst- og data-objekterne) jo som nævnt bør formateres som html, så kan termer jo være startsiden til en hjemmeside. Så `populære start-termer' kommer altså med stor sandsynlighed til at indebære populære ``hjemmesider'' i form af html-dokumenter. Så hermed for brugerne altså mulighed for at konstruere deres egne nyttige oversigter til nye brugere. Mere avancerede brugere kan benytte specialbyggede browser-programmer, som selv holder deres egen personlige oversigt over favorit-lokationer(/-termer) og favorit-point. hvis de vil dette. For da hele teknologien i dette lag skal være åben, er dette jo frit for. Alternativt kan man logge på specielle servere beregnet til at gemme brugeres data sikkert og fungere som interface for sem-web-netværket, hvor brugeren ikke selv behøver at downloade noget. Når brugeren så har sine start-punkter (inklusiv favorit-point) til ontologien, kan brugeren så bruge disse til at konstruere første forespørgsel (query) til serveren, alt efter hvad de lige er interesserede i. %Det kunne f.eks.\ være, at man siger ... %Hm, og hvad når man løbende vil søge på søge ord?.. Hm, man skulle ikke ligefrem indføre brugeralgoritmer i form af point-udregninger (helt a la de "farve-regler," jeg tidligere har tænkt for ITP)? Det ville da kun være oplagt, ville det ikke?..! ..Hm, men ift. søgning, så ville "navigation" i sprogontologien også gøres, så man i praksis søger på ord, hvilket man jo så bare kunne bruge.. Hm... Ja, og det er jo skønt, at man kan det, og så kan (og bør!) dog stadig foreslå det her med, at afhængige point også kunne bruges til at føje brugerdefinerede algoritmer til serveren. Nice nok. 
%til serveren, ``giv mig start... %..Hm, får man ikke brug for mere avancerede serveralgoritmer allerede her?...
%Uh, det er da egentligt totalt smart med mulige farve-regler til serverne i form af de 'afhængige (aggregat-)point' (for de kan konstrueres ret abstrakte, men stadig potentielt blive rigtigt effektive (fordi de så kan benytte naturlige database-optimeringer))! Nice-nice. ^^ (26.06.21)
%(27.06.21) Hm, men man kunne da også give mulighed for at html-dokumenterne skal kunne query'e serverens ontologi?.. hvilket man jo netop kan, fordi hele strukturen er offentlig.. Jo, god idé, og så kan det bare ske via "pointene," som så også skal kunne have karakter af strenge.. Ja, sgu. Ja, og "via pointene" er måske så meget sagt; brugerne bør som sagt få en oversigt over databasen (og de kan også selv opsætte servere; det skal være helt frit for), og herved kan de så kontruere sql-queries i... Hm, men var "pointene" ikke netop en god måde at sørge for, at serveren har et valg ift.\ hvilke queries, den vil understøtte? Plus at et ekstra lag imellem html-kommandoerne og sql'en er alligevel nødvendigt, så folk ikke kan sige 'drop table' eller andre forbudte ting.. Og med boolske point kan man få vilkårlige søgninger, så det burde ikke være noget problem.. (Og selvfølgelig kan man også bare bede om de første x antal tupler med en hvis sortering ud fra et tal-point.) Og servere bør jo bare sætte et filter, så de fleste nye pointtyper kan godkendes automatisk. 





%Brain (27.06.21): Man burde nok gøre det sådan, at et start term bare den "hjemmeside," man søger fra, og som også definerer, hvordan relationer og termer sættes ind, når man navigerer rundt i ontologien (fra start-punktet/erne af).. Og angående javascript og php/ajax, så må det næsten være noget med at bruge specielle sprog, som serveren så kan oversætte til pågældende (så altså ikke bare at bruge en undermængde, men bruge helt nye sprog).. Og det er nemlig også i øvrigt fint, at man kan begrænse den tilgængelige ontologi ved at vælge visse startpunkter, så man f.eks. kan gøre internet-sessionen børnevenlig osv. Lad mig lige summe lidt over de tanker, og bl.a. tænke lidt mere over, hvad disse specielle sprog så skal kunne. ... Never mind det med at begrænse ontologien via start-punkterne, hvor jeg altså tænkte, at man kun kunne navigere videre til fobundne punkter; det er meget bedre bare at lave alle begrænsninger via pointene i stedet. Det bør også være sådan, at browser-applikationen bare snakker med en lokal server (localhost) i første omgang, som så bare bør opføre sig som en abstrakt database (i interfacet til browser-applikationen), hvor der kun kan laves queries og postes uploads til. Og her kunne jeg så foreslå P2P-databaser som en mulighed, men den lokale server skal sådan set bare være fri til at query'e internettet helt frit ud fra, hvad den er indstillet til. Den lokale server kan så fungere som en P2P-node, hvilket også kan skabe anonymitet. Den skal så også (uanset hvad) selv implementere verifikation over alt tilsendt data, f.eks. ved at tjekke diverse certifikater/signaturer, og skal i øvrigt også tage stilling til, måske med brugeren input som hjælp, om data skal gemmes i den lokale server eller om cachet data skal slettes. Hm, jeg håber lidt, der er en god fri P2P-database API, man kan bruge, for så kunne man jo ret hurtigt komme ret langt med en prototype.. Jeg har også lige overvejet lidt ift., hvad man evt. kunne gøre med betalingstokens.. Ja, man kunne jo i det mindste foreslå det med, at man bare kunne have et tillidsbaseret system, hvor en bruger efter noget tid kan betale mere og mere med tok... Hm.. Tja, men uanset hvad, må det da næsten være i et lag for sig, ikke? Jo, det må blive i server-server-interaktionslaget, hvor ens lokale server altså skal være fri til at implementere forskellige strategier, men hvor man bare kan starte helt simpelt og gratis (i et netværk af folk interesseret i teknologien). 
%(28.06.21) Fik nogle tanker i går, som jeg ikke nåede at skrive ind her, og så vil jeg også bare fortsætte brainstormen her ellers. Angående en distribueret database, så bør man ikke bare bruge hash tables og alt det, for man bør hellere udnytte, at den samlede data-ontologi allerede er ordnet i en graf, og endda pr. konstruktion på en måde, hvor nabotermer ofte vil skulle serveres sammen med hinanden. Så servernetværket skal selvfølgelig hellere kunne melde sig på at varetage diverse ontologi-dele. Jeg tænker, at man kunne benytte timestampsne her, således at ontologien kan grupperes i faste enheder, der ikke ændrer sig efter en vis tid. Og når det kommer til aggregatpoint, som afhænger af data på tværs af servere, så er det smarte jo bare, at hver server... Hm, skal hver server nu også holde relationerne til hver andre server; det biver jo så dermed ikke vildt skalerbart.. Hm.. Og man kan ikke bare udregne pointene konstruktivt i tid, så man altså altid kan lave tidslige checkpoints..? Hm, jo, for kan man ikke godt nøjes med udregninger a la sum og gennemsnit osv., hvor man ret let kan ændre beregningen, hvis data bliver tilføjet? Det må man da kunne.. Hm, hvad hvis der er ikke-lineære og/eller diskrete vægte i en udregning?.. Ja, det må også fungere; når bare hvert term i en udregning kommer med sit eget uafhængige bidrag, på nær måske en overordnet faktor afhængigt af antallet (så som i en gennemsnitsudregning), som man nemlig godt også kan styre, hvis man bare gemmer tælletallet, hver gang der kan være behov for det. ..Tja, eller bare tæller om for den sags skyld.. Tjo, men det handler jo netop om at gøre ontologi-dele mere uafhængige af hinanden, så man ikke skal sende så meget data og lave omregninger på kryds og tværs (så ja, god idé at gemme tælletallene). Nå, men da jeg skrev "uafhængige bidrag" her, kom jeg jo til at tænke på, kunne der ikke godt være brug for point, der tager højde for korrelationer mellem term-bidrag? ..Det burde der jo faktisk blive mulighed for.. Hm, men kan dette ikke ligesom sepereres ud, hvis man deler det op og sørger for at lave en speciel vektor langs denne korrelation?..!..? Jo, det må man sgu kunne..! Ja, det må man helt sikkert (for sådanne regnestykker kan jo udformes med lineær algebra). Nice. Så når man vil udregne point kommende fra en relation på tværs af servere, så skal man kun tjekke hos de servere, der holder de seneste versioner af termet (og som i øvrigt så selv ikke behøver at query'e andet end dem andenseneste version), og i denne forbindelse kan man så bare sørge for, at serverne, hvis "levende" ontologi-dele kan afhænge af hinanden sørger for at sende signaler til hinanden, når der sker opdateringer (de behøver altså heller ikke at have en kopi hver af hver relation, som jeg ellers tænkte lige før). Cool cool. *[Nå ja, og en idé angående et distribueret database-system også er lige, at man kunne have en protokol, hvor servere af og til skal regne det næste hash ud i en kæde, hvor hver blok er en hash af en vis datamængde samt det tidligere hash.. Ah, og lige en ydre faktor også, som ikke er kendt på forhånd (enten bare ved brug at tredjepart eller, hvis decentralisering af en eller anden grund bliver vildt vigtig (f.eks. hvis man blander en KV ind i billedet på en eller anden måde), ved brug af en blokchchain), sådan at serverne ikke bare kan udregne dem på forhånd. Herved kan man så sikre sig, at det ikke kan betale sig for serverne ikke at holde fast på den data, de hævder at de har på lagret.] Angående opbygningen så kom jeg i øvrigt frem til, at der skal være prædikater og relationer, der er beregnet til at tale om et term, som det optræder i en kontekst; nemlig i et tekst-term. Måske ville det faktisk i øvrigt generelt være en god idé at opdele modellen i tekster og RW (real world) -termer.. Og hvis en tekst vil referere til en anden tekst i ontologien, eller til et knudepunkt, så kunne man jo enten bare indføre en passende reference-relation, eller måske kunne man ligefrem lave en fast type term.. Nej, en relation til at referere til den i teksten omtalte eksterne tekst må være passende. Og så er det nemlig, at jeg tænker, at tekst-relationerne så faktisk skal være HOL-agtige, så højere-ordens-relationer kan referere, ikke bare til tekster, men til ontologi-knuder, der også kan indeholde relationer (af lavere ordner) mellem tekster. På denne måde kommer man nemlig bl.a. til at kunne vurdere, hvor godt en tekst er blevet tilknyttet passende referencer, og man kan så også sige noget om bl.a. forklaringsgraden af teksten, når man inkluderer dens referencer. Kontekst-prædikater og relationer omkring tekst-termers forhold ift. at opbygge en argumentation/forklaring vil så selvfølgelig være højere-ordens (med orden alt efter, om de inkluderer reference-relationerne eller ej). En vigtig kontekst-relation bliver i øvrigt (indså jeg i går), om hvor godt et tekst-term kan erstatte et andet tekst-term i det overordnede tekst-term (..Nå ja, så termerne skal jo dermed selv have flere ordner.. Ah, ja, for det er nemlig ikke helt HOL-agtigt; det er mere... Hm, eller er det? Det er også lige meget, det handler i hvert fald bare mere om, at relationer kan blive behandlet som termer (altså med et udvidet domæne for, hvad et "term" kan være) af relationer af højere orden). For dette bliver en rigtig vigtig ting, når man skal udforme interaktive tekster, hvor læseren kan variere teksten (i stedet for at man bare har faste html-objekter, som man kan loade en ad gangen; i stedet kan man således få interaktive html-objekter). Relationer til at tilføje ting til en liste kunne nok også være brugbare, især når listen ikke behøver have en bestemt orden/struktur. ... Og point og relationer (og prædikater) kan nu også blandes sammen til én ting, bare med to muligvis forskellige fortolkninger (det er nemlig ikke altid, man vil fortolke et point som et prædikat som den foretrukne (fortolknings)mulighed). Point/relationer/prædikater skal sammenlagt så bare have mulighed for at have arbitrære domæner (særligt ift. antal input --- og selvfølgelig dog med overholdte ordensinddelinger), og man kan også så tænke på de automatiske (afhængige) point som automatiske relationer/prædikater. Jeg har nu også tænkt lidt i typer, og særligt hvis man skulle have nogle specielle relationstyper, så lokalservere ved, hvordan html-dokumentet skal struktureres, og om relationer skal tilføje alternativer med lavere prioritet eller med højere prioritet, og om referencer skal inkluderes eller ej, hvis de nu mere er relevante den anden vej, nemlig fra den eksterne tekst til den pågældende, men ikke er så relevant i den pågældende tekst.. Jeg har i øvrigt også tænkt, at det også ville være en god idé med særlige alternativer, som repræsenterer oversættelser, enten til et formelt sprog (så man også kan tilføje flere søgbare termer til en tekst) eller bare til et andet NL.. Hm, men mon ikke alt dette skal ske et lag over det grundlæggende, ved at man laver konventioner for lokalserveren, hvordan de skal give termerne til browser-applikationen?.. Skal dette så bare være ved at udpege nogle specielle relationer til disse formål fra starten? ..Ja, det er det jo simpelthen bare; ikke så meget egentligt at diskutere der.. ...Hm, skulle man så kunne referere til visningskonventioner i (høj-ordens)-relationer..? Tja, måske; jeg kan jo foreslå det.. Ja, lad mig det.
%..Ah, point skal i øvrigt også kunne bruge "relations"-id'er (såvel som point-id'er; de er jo blandet sammen), så samme point bl.a. på den måde kan "sættes på" flere forskellige relationer. Således kan man så komme til at give point, som f.eks. siger, "brugergruppen, x, vurderer at den populære relation, p, har en korrekthedsværdi på a."
%Jeg tror ikke, der skal føjes typer til, og man kan nok bare bruge en god (ret grundlæggende) pointtype som betegner entydighed/meningsfuldhed til at skille relationer fra, der ikke har de rigtige "typer" af indeks. Så selvom folk kan blive forvirret, hvis domænet ikke er korrekt, og give gode korrekthedspoint, fordi de misforstår relationen og/eller ikke opdager det ugyldige domæne, så gør det altså ikke noget, så længe man bare husker at tjekke meningsfuldheds/entydigheds-pointet. Og man kan bare faktorere dette point sammen med korrekthedspointet eller lignende, så man ikke får for mange ikke-så-relevante point at holde styr på. 
%Jeg vil også tilføje, at jeg lidt har glemt, hvordan browseren skal kunne variere termer, idet brugeren variere point-prioriteten. Nu er jeg så kommet frem til, at alle termer (af vilkårlig orden) skal kunne selekteres og have ændret point-prioriteten for sig, enten for sig selv eller rekursivt, så alle relaterede børn og ekterne referencer selv bliver selekteret ud fra den nye point-prioritering. Og hvis man således vælger det yderste term, man arbejder på, kommer point-præferencerne altså til at ændre sig for hele sessionen (indtil man skifter om igen) og for alle de termer man navigerer til fra det pågældende. Jeg forestiller mig så, at browseren skal give en pop-up-indstillingsbar, hvor brugeren kan bruge knapper, formularer og/eller sliders i en eller to dimensioner til at justere den endelige point-prioritet, som så bør, tænker jeg, være en linearkombination af alle brugerens foretrukne pointyper, som brugeren har valgt at have i spil. Brugeren bør i øvrigt også kunne indstille grænser for disse point forud, så man ikke f.eks. kommer til at slide over på nsfw/nsfl-indhold (eller hvad man nu ellers ikke er interesseret i at se i sin normale browsing), fordi man lige fik lavet en forkert handling. Og ja, jeg tror altså det er godt at have gang i flere point på en gang. Så kan man f.eks. også slide frem og tilbage mellem korte og længere forklaringer af en paragraf, eller slide mellem engelsk og dansk, osv. Det ville i øvrigt ikke gøre noget, hvis browseren også kan give en form for nogenlunde oversigt over, hvilke alternativer der er, så man ikke bare skal slide frem og tilbage i blinde; måske kunne browseren endda vise en liste af muligheder for det selekterede term, som kan nås inden for brugerens point-begrænsninger, og så bare vise point-værdierne ud for, at brugeren kan se hvad alternativerne opnår. (28.06.21)
%(01.07.21) Jeg blev i tvivl om, helt hvordan jeg skulle sælge mine idéer her, og om jeg virkeligt havde nok til et hurtigt og nemt gennembrud, eller om der skulle mere til. Nu har jeg så igen fundet frem til en vej videre. Jeg skal bare ikke fokusere ligeså meget på de potentielle fordele, når det kommer til at få indhold og andre ting anbefalet sig, og mere på vidensdelingsdelen og programmeringsdelen i første omgang. Og hertil har jeg så også indset at der jo er nogle gode kommercielle vinkler, som endda kan bygge oven på et helt frit grundlæggende lag, nemlig hvis man laver en donationsforening til at tilvejebringe, at folk kan give donationer, som kan fordeles ud fra point, når det kommer både i første omgang til vidensdeling generelt og også specielt til programmeringsvidensdeling, hvor der måske særligt kan være en del penge involveret, og hvor man endda også kan lave en forening, der begynder selv at sælge de programmeringsløsninger, som udvikles, hvilket så kan gå hen og udvikle sig til en hel omni-IT(-applikations)-virksomhed. I øvrigt vil jeg så også sælge idéen om et programmeringsparadigme med HOL-agtige point til programdokumentationerne (både til atomare enheder og til sammensatte helheder). Og dette paradigme har så særlige fordele online, men kan sikkert også med fordel indføres i større IT-virksomheder, så folk bedre kan gennemgå og tjekke hinandens kode (og på en formel måde, som kan propagere op og ende med at konkludere på korrektheden af hele systemet). Så det skal ligesom være mit udgangspunkt for resten --- altså det jeg vil skrive om og udbrede mig om først --- og så behøver jeg ikke tænke så meget i anonymitet osv.; det kan altsammen komme i øvre lag på det grundlæggende (selvom det grundlæggende lag på samme måde, som jeg har tænkt, skal lægge op til anonymitet ved kun at identificere brugere via deres underskrifter, men hvor man altså hvor brugerne ikke skal have konto og password og alt det jazz). Det næste emne vil så blive idéen om at bruge pointsystemet til at lave kraftfeltafstemninger, og så det her med at folk jo ikke behøver alt muligt papirarbejde osv. for at komme i gang med at handle som grupper, fordi hver (i bund og grund anonyme) bruger bare samler etos op undervejs, og grupper kan så bare skille sig af med medlemmer, der ikke holder deres løfter/trusler ("trusler" fordi en væsentlig del af mulighederne også handler om at lægge pres på forretninger til at give mere favorable handler til medlemmerne). Og til dette fik jeg så den tanke i går, at man også her kunne finde en kommerciel vinkel på at fremføre denne idé, nemlig hvis man laver en overforening, som diverse grupper/underforeninger (som ellers er rimeligt selvstyrende) så kan handle igennem. Så overforeningen bliver ligesom "intefacet" mellem grupperne og omverdnen, og kommer herved til at administrere penge- og varestrømmene (både fra grupperne til erhververne og omvendt (og når det kommer til varer, vil det jo i reglen være omvendt)). Og for dette administrative arbejde kan man jo så tage sig betalt. Igen har vi her en virksomhed, der bygger oven på en helt fri teknologi, men hvor der alligevel er en god potentiel indtjeningsmulighed (hvis bare man så er hurtigt ude af fjerdene). Jeg kan så også lige sammenholde denne idé med videndelingsidéen, for den kan også generelt blive en god måde at måle sine kunders præferencer og holdninger, og kan altså blive en måde for virksomheder at åbne op for, at kunderne for medbestemmelse (med en vis vægtning). Nå ja, jeg fik ikke nævnt, at omtalte programmeringsparadigme så også kan føre mere åbne applikationer med sig, hvor man kan stille på applikationen via point (ligesom at man kan indtille termerne i ontologien, som jeg beskrev her ovenfor). Og i de situationer, hvor der skal vælges én mulighed af den ene eller den anden grund, så kan man netop bruge sådanne afstemninger i stedet. Og når jeg først begynder at foreklare om mulighederne ved denne idé, så vil det være naturligt også at snakke om kundedrevne virksomheder osv. (hvortil jeg også kan nævne mine idéer til... ja, mine "11/6-idéer" omkring et åbent økonomisk system med "bagudbetaling," som muligvis kunne udformes som en NL-KV, så det kan jeg så også lige nævne kort (og i øvrigt fik jeg lige en blockchain-idé i de her (to) dage (fik den i forgårs) som jeg lige vil skrive ind under min senste blockchain-brainstorm *(som er nedenfor under "opsummering..."), men jeg er nu stadig ikke særligt opsat på blochchain længere *(nope, never mind: idéen holder heller ikke rigtigt alligevel))). Så det er ligesom det, der er min plan nu. 
%I øvrigt fik jeg også lige de tanker/idéer i går (d. 30/6), at der gerne (til at starte med) må være konventioner for lokal-serverne, hvor de har ansvaret for at udregne de endelige point, som brugeren ser, efter at denne har valgt en pointsammensætning, så som bl.a. 'korrekthed,' 'entydighed,' 'let at følge' (hvor en lav score altså signalerer, at teksten skal uddybes noget mere og sikkert derfor gerne opdeles yderligere i mindre argumenter), og også point som så før indflydelse på opbygningen af termet, og særligt omkring, hvilke termer skal vises ud af en liste af muligheder, samt også hvorvidt en relation skal inkluderes, og i så fald om den skal inkluderes som en ekstern reference, eller om referencen skal foldes ud som et barn af det pågældnede (sammensatte, højere-ordens) term. Og i forbindelse med dette var en af idéerne så også at have tekst-highlights *(eller det kunne også fremhæves på andre måder... tja, man hvis man bruger highlights kan man bare skifte imellem de forskellige slut-point (så som de nævnte)) til at vise de tre nævnte korrektheds-forståeligheds-point, som man så kan slå til og fra, så læseren hurtigt kan få et overblik over, hvad andre brugere mener om de individuelle sætninger i teksten, selvfølgelig vægtet ud fra ens valgte pointsystem. Hm, alternative slut-point, som vi kan kalde dem, kunne i øvrigt være omkring, hvor relevant en sætning er for teksten, og hvor godt en sætning er underbygget i selve teksten... Ja, man kan finde på mange, så hvis det skal være highlights, man bruger til at få overblikket, så må man jo, som jeg lige har nævnt i en indskudt parentes, være sådan, at man kan skifte imellem at se forskellige highlights (inklusiv ingen highlights selvfølgelig). 
%Videre på brainstorm (01.07.21): Jeg mangler lige at finde ud af, hvordan man søger på termer, men jeg er så lige kommet i tanke om, at man også bør have relavans-point, hvor man automatisk kan regne ud (eller det kan serverne og/eller lokal-serverne), hvor relevant ét term er til et andet. Dette skal så kunne gøres ved at vægte relationer mellem vilkårlige termer og den pågældende ift. relevanspoint. Og så bliver løsningen simpelthen bare, at en bruger også kan inkludere vilkårlige strenge i disse point-queries, således at strenge altså kan tilføjes på ligefod med andre termer, også selvom de ikke på forhånd af en del af databasen, som point-parametre, hvorefter point-udregningerne så kommer til at bruge strengen --- eller det kunne også være en tal-værdi for den sags skyld --- som input. Så brugerne kan herved definere en algoritme via en afhængig pointtype, som tager et streng-input og traversere en morfologi-ontologi for at finde de mest relavante termer til strengen (pr. algoritmens indstillinger). Og bum, så har vi både konventionel tekst-søgning og også semantisk søgning, for brugere kan så også definere mere komplicerede søgninger, hvor f.eks. specifikke, formelle relations-termer (som kan repræsentere speciffikke verber med præcist defineret semantik bag) kan gives med som input, og meget mere. Ja. Nice.
















\subsection{Lykke-valuta-idé revisited}
(25.08.21) Jeg har ovenfor skrevet om en lykke-(krypto-)valuta-idé, og efter at jeg færdiggjorde \textbf{Yderligere tanker omkring kryptovaluta}-afsnittet her tilbage i næsten midt juni havde jeg en ret god fornemmelse omkring idéen, men den var dog stadig så kompliceret, at jeg ikke havde planer om at prøve fremføre den som en selvstændig idé lige med det første (men i stedet bare prøve at fremføre nogle andre idéer først og så komme ind på den efterfølgende). Jeg vil selvfølgelig ikke love, at dette lige netop bliver den første idé, jeg satser på at fremføre, men pointen er, at nu er jeg kommet frem til nogle nye tanker omkring idéen, der gør den meget mere simpel (og sikkert meget lettere at sælge som idé)! \ldots Og lettere at implementere sikkert også, ikke mindst! Jeg har på en måde overtænkt idéen. Ikke fordi mine overtænknings-tanker ikke var sunde nok at gøre, det var de nu nok, men nu har jeg altså indset %... %Hm, jeg kan mærke, at jeg lige skal summe lidt mere over idéen nu her.. og specifikt omkring hvordan den relaterer sig til tankerne, jeg havde forinden, om en "blød" donationsforening.. ...Jamen det holder altså bare! Og ja, idéen passer stadig super godt med min idé forinden omkring donationsforeninger, for 'donationer' kan endda meget vel blive udgangspunktet for hele bevægelsen; nemlig at man har en kryptokæde, hvor man kan tage ejerskab over donationer og andre bidrag, og hvor folk så enten kan donere til tidligere bidragsydere (og selv notere denne donation som et bidrag) --- muligvis ud fra en fordelingsformel.. og muligvis en demokratisk en af slagsen, som jeg jo også tænkte for idéen forinden (kom jeg lige i tanke om nu) --- eller de kan købe dele af vedkommendes bidrags-aktie.. hvilket også sådan set kan noteres som et bidrag: al aktivitet kan i starten med fordel noteres som et, om ikke andet, hype-skabende bidrag. ..!! :D^^ 
at jeg måske faktisk på en måde har angrebet idéen fra et lidt for hårdt udgangspunkt, da jeg analyserede den. Jeg analyserede den nemlig ret meget ud fra et udgangspunkt af, at alle mennesker efter kort tid ville begynde at handle selvisk\ldots\ Hm, hvilket jo også var et rigtigt sundt udgangspunkt, især fordi at jeg dengang også tænkte meget i baner omkring, at valutaen ligefrem skulle erstatte gængs valuta (lidt ligesom målet på en måde er for andre kryptovalutaer). Og jo, det var sundt at angribe idéen på den måde, men nu har jeg altså indset, at jeg nok kan tage et helt andet udgangspunkt for idéen, hvor systemet bare kan være drevet af en undermængde af befolkningen, som tror på, at systemet, pga.\ den godhed det medfører sig, vil overleve til langt ind i fremtiden og vil respekteres af en mindst ligeså stor undermængde i den halvnære fremtid. Og herved behøver systemet slet ikke at have særligt mange formelle restriktioner på sig, men kan bare være styret primært af nogle overordnede (og endda ret løst formulerede!) principper. Og den omtalte undermængde vil så bare overholde disse principper pga.\ deres tro på det gode i hele idéen, og dermed at på at deres ligesindede vil blive ved med at holde den i hævd. 

Nu skal vi derfor bare forestille os en kryptokæde med en enorm stor frihed på --- og det kan jeg komme ind på om lidt, hvad dette indebærer mere præcist --- således at det ikke koster noget rigtigt at tilføje blokke til kæden (så sværhedsgraden kan sættes arbitrært lav), og hvor man har rig mulighed for at merge forks og gøre dem til en samlet kæde igen (som så bare har en forgrening i sig, som samler sig igen). På denne kæde skal der så ikke være nogen underlæggende mønt, som minerne belønnes med. I stedet vil al værdi på kæden komme sig i form af `noterede bidrag' i blokkende, hvilket altså er tekster, der referere til et `bidrag,' eller vi kunne også kalde det en (god) `handling,' og en person, og hævder, at denne person gjorde denne handling, og som så muligvis også indeholder vedlagte beviser på dette. Pågældende person kan så identificere sig selv på kæden, og hævde at en vis offentlig krypteringsnøgle tilhører vedkommende. I denne forbindelse oprettes så en token, der referere til de to foregående påstande, og som vil tillæges en værdi, kun hvis disse påstande kan påvises at være sande (og at beviserne kan sørges for at blive gemt for eftertiden (og altså særligt den omtalte ``halvnære'' eftertid)), og så ellers hvis folk altså vurderer handlingen som værende værd at belønne. Og størrelsen af værdien vil så altså afhænge af, hvor høj værdi folk vurderer handlingen til at have. Inden vi når til at diskutere, hvorfor at dette kan lade sig gøre, og at værdier på denne måde kan skabes på kæden, så skal det lige nævnes, at folk selvfølgelig skal kunne handle med disse tokens. Der skal altså være en protokol for at overføre hele eller dele (for tokens'ne skal nemlig kunne splittes op) af ejerskabet over et token via blokke på kæden, der indeholder sådanne transaktioner. Et særligt eksempel på et `bidrag' / en god `handling' vil så være i form af at mine blokke til kæden, og måske mindst lige så vigtigt at lagre gamle del-kæder. I forhold til sidstnævnte bør man derfor indføre en protokol, hvor instanser løbende kan bevise, at de rigtignok gemmer på pågældende delkæde (med andre instanser, som også gemmer delkæden, som meddommere) --- så altså en slags PoStorage-protokol. Alle sådanne protokoller behøver dog ikke være defineret fra start af fra kæden, da de jo bare i sidste ende vil indgå som en del af bevismaterialet for et `bidragstoken,' hvilket netop ikke har nogen særlige restriktioner på sig i den grundlæggende kæde, og altså kan have form som hvad som helst. 

Førhen har jeg jeg også måtte slås lidt med, hvornår bidragstokens'ne skulle veksles til anden valuta, men nu kan jeg også se en ret simpel og effektiv løsning på dette. Det skal bare høre med til kæden\ldots\ \ldots%... %Hm, skulle mon man, i stedet for at starte med et antal år, bare starte med et løst princip om, at... Hm.. ..Jeg tænkte på noget med, at man bare sagde, at man bare skulle være noget på afstand nok til, at kunne bedømme bidragene.. fair.. Hm, kunne man mon lade spørgsmålet om, hvor lang tid der skal gå være en del af de abstrakte bestemmelser, således at man ved at sætte en deadline og/eller en deadline-trigger sætter en præcedens for, ikke hvordan, men hvornår den samtidige aktivitet skal bedømmes..?.. Hm, problemet er bare, hvornår bliver man nogensinde klar over, hvornår et bidrag kan måles?.. ..Hm, man kunne man ikke bare have et princip om at godkende, som fællesskab, at et bidrag bliver bedømt, hvis man mener, at omfanget af bidraget i god nok grad kan siges at være kendt, og at folk med bidrag så selv må søge om, at få dem indløst/vekslet..? ..Hm, nu kom jeg også lige til at tænke på: vil det ikke være sådan, at foreninger til at varetage IP-rettigheder og til at veksle bidragstokens'ne vil melde sig på banen..?.. ..Eller hvordan skulle det ellers foregå(?).. Jo, foreninger til at varetage IP-rettigheder, vil helt klart melde sig på banen, men vil disse også tage ansvar for bedømmelser og token-vekslinger..? Og hvem vil ellers? Bare vilkårlige, måske separate foreninger..? ..Ja.. ..Ja: foreninger af mennesker, der gerne vil holde liv i systemet.. Og ja, disse foreninger kan så også godt varetage IP-rettigheder (og altså sælge produkter/services) samtidigt med.. ... Hov, har jeg mon undervurderet faren for alligevel, at folk bare udvikler et alternativt system i fremtiden, hvor folk belønnes bagud (med måske bundet på mere konventionel jura), og at der så ikke bliver brug for at betale de nuværende ejere af bidragstokens'ne..? Hm.. ...Ah! Nu kommer problemet!: Ja, folk kan udvikle andre systemer, der fungerer fint i stedet for dette, og selvom man måske så alligevel gerne vil respektere og værdsætte dette system, hvem skal så betale? Det bliver for nemt at undsige sig, og hvis betalingsbyrden så bare bliver på nogle få, så kan det også let bare ende med, at de heller ikke vil betale (hvis deres donationer skal betale det alene).. Og så vil der heller ikke være den nævnte motivation, hvor man betaler, fordi man gerne vil holde systemet kørende.. (!) Hm.. ..Hm, måske er der stadig en mulighed, men ja, dette bliver et klart kritikpunkt af idéen.. 

%(26.08.21) Åh! Jeg var virkeligt klar til at afskrive meget af idéen..!.. Jeg har i øvrigt også gjort nogle andre tanker omkring kundedrevet virksomheder og forbruger-foreninger (inkl. en forbruger-web 2.0+-platform..) og ting i den retning. Men de var nu overvejende ret pessimistiske, ift. hvor jeg var i går formiddags.. Men nu her til aften kom jeg lige i tanke om, at LV-idéen måske kan virke alligevel!.. For man kan måske bare opbygge det sådan, at folk starter med at melde sig på et system (ved brug af tillid ('trust') og certifikater og tilhørende algoritmer osv.), som er beregnet på at konstruere og vedligeholde et lødigt kort (eller graf) over, hvilke bidrag er blevet gjort i fællesskabet og hvem i fællesskabet har gavnet af dem. Og pointen skal så bare være, at man ved at melde sig ind i sådan et fællesskab --- og der kan altså være flere af sådanne fællesskaber logget på samme kæde og kørende uafhængigt af hinanden --- erklærer, at man vil respektere processen, medmindre at der forekommer evidens på korruption af processen (hvilket man så kan erklære og melde sig ud, men så kan alle bare se dette, og vil generelt holde sig fra at stole på en i fremtiden, hvis folk generelt vurdere, at udmeldingen skyldtes selviske grunde), og at man vil sørge for at donere penge til folk, hvis bidrag man ifølge processens resultater har draget gavn af i en eller anden passende grad, som aftales på forhånd (mere eller mindre formelt). Så med andre ord melder man sig ind og siger: "Jeg vil gerne donere penge til folk, der yder bidrag som gavner mig, og jeg stoler på at proces x vil være i stand til at bedømme dette i god nok grad (og ellers må jeg bare gøre indvendinger og melde mig ud og håbe på, at folk vil se tingene fra min side i så fald), og jeg tror også på, at samme proces vil finde frem til fair donationsbeløber, som jeg så altså vil rette mig efter, hvis jeg har midlerne på hånde (og ellers må jeg jo bare erklære, at jeg holder pause fra hele systemet (inkl. andre konkurrerende varianter), og at jeg så betaler beløbet, hvis jeg melder mig ind igen)." Og hvis vi skærer al usikkerheden fra og antager, at systemet holder, og at folk overholder deres løfter, så melder man sig altså ind og siger: "Jeg vil gerne betale for handlinger / donere penge til folk, der har gjort handlinger, der i sidste ende gavner mig, og jeg vil gerne betale/donere et fair beløb." Folk der så yder et bidrag til hele eller dele af fællesskabet tager så selvfølgelig lidt en chance, for det kan jo være, at folk vil undsige sig fra at holde deres løfter. Men fordi folk generelt gerne vil holde sådanne løfter (og generelt altså gerne vil give tilbage, hvis nogen yder for dem), og fordi folk også kan holde øje med, hvem der tilsyneladende er svindelere, og så lukke dem ude af fællesskabet, indtil de går med til at betale, hvad der er fair (i det gennemsnitlige medlems øjne), så tror jeg altså, at det kan fungere! Ja..!! Så kæden i idéen, som den er nu, kommer altså til at bestå af, at folk notere deres bidrag helt generelt ligesom før, men at folk så også hver især melder sig til et fællesskab (som også er erklæret ved at uploade definerende/erklærende tekster til kæden) med en tilhørende proces for at bedømme, hvad individet skylder eller skyldes. Folk kan så melde sig ud og ind af fællesskaber, og kan også endda melde sig til flere på én gang, hvilket dog fra individets synspunkt helst kræver, at fællesskaberne arbejder sammen, så man ikke bliver bonget dobbelt for noget. Og hvis man melder sig ud af et fællesskab kan det så altså komme til at koste for ens omdømme, hvis man ikke gør det på en god måde. Ni-iice! (26.08.21)
%..Og man kan sagtens bare starte med at melde sig til nogle ret vage fællesskabserklæringer, hvor man så i bund og grund bare erklære, hvilke nogen senere, mere avancerede fællesskabssytemer/processer, man vil melde sig til efterfølgende. På den måde kan man f.eks. melde sig til og love de første minere belønninger, men uden så at blive bundet til selve fællesskabet, så man måske får svært ved at melde sig til bedre udformede fællesskab-processer/-systemer senere. Man kan f.eks. bare melde sig til en fælleserklæring om at: "Jeg vil gerne give så og så meget til de første minere på kæden (der jo er med til at skubbe tingene i gang) alt efter, hvordan kæden udvikler sig, i henhold til sådan og sådan," og så kan man jo altid melde sig til andre erklæringer bagefter, sålænge man bare sørger for at lave nogle passende offentlige donationer til minerne. Så ja, man styrer helt selv, hvor formelle og/eller præcise erklæringerne, man melder sig til, skal være, og hvor bindende tilmeldingen skal være osv., og erklæringerne kan sagtens altså bare udformes selv som nogle lidt løse sætninger (med tilsvarende løse løfter i). Jep. :) 




%(31.08.21) Om LKV-idé: Jeg er faktisk kun blevet overbevist mere om brugbarheden i idéen om min donationsløfte-blockchain her i weekenden. Kæden kunne nemlig også selv sætte gang i arbejde på at udvikle web 2.0-agtige platforme, såsom især et forbruger-SoMe, som kunne spille rigtigt godt sammen med kæden. Og folk kan tilegne sig bedre handlesmæssige positioner i deres liv, hvis de beviser, at de er i stand til at holde løfter, så for den enkelte er der sikkert økonomisk og politisk magt at vinde ved at deltage i kæden og donere til ting. Værd at nævne også, at kæden jo vil kunne blive en god begrundelse til at sænke skatten for almindelige mennesker og lade tilsvarende, skatteagtige penge flyde mere direkte fra folk og til, hvor de synes. (Så vejen til at folk får mere direkte indflydelse over, hvor deres skattepenge går (hvilket mange sikkert vil se positivt på, og resten vil nok mest bare se mere neutralt på det frem for direkte negativt) --- og da det er en løbende proces, vil den jo også være rimeligt reversibel ift., hvis man bare gik over til mere direkte indflydelse fra den ene dag til den anden.) Jeg håber på, at folk vil vælge at se på samlede (og lødige) oversigter over, hvilke bidrag på kæden har bragt dem nytte, når de skal donere, og ikke på mere selviske varianter, hvor man kun ser på bidrag gjort efter, at man selv oprettede sig på kæden. Jeg tror dog, at folk faktisk gerne generelt vil vise denne godhed, og at de derfor gerne vil tilslutte sig (hvilket jo også gøre offentligt, så alle kan jo se, hvad andre tilslutter sig) de mere "gode" og retfærdige løftesystemer, så at tidlige bidragsydere, som har registreret deres handlinger på kæden, får deres retfærdige belønning. Og af denne grund, eller rettere fordi andre sikkert vil have en mere eller mindre tilsvarende tiltro til folk, så vil der derfor sikkert være mange, der gerne vil være med fra starten --- og bl.a. med til at skabe de gode web 2.0-agtige platforme, som kan fremme kædens brugbarhed (ved bl.a. at udgøre gode interfaces til kæden, og selvfølgelig også pga. divesre funktionaliteter som platformen implementerer oven på kæden). Det er selvfølgelig også vigtigt at understrege, disse ting taget i betragtning, at kæden jo også vil kunne booste donationer til diverse open source-bidrag, som allerede findes så mange af, og som bliver ved med at komme. Og nu med denne teknologi kan man så donere mere direkte til de gode bidrag --- også selv uden at tænke en hel masse over det, men bare ved at formulere nogle overordnede principper, for så kan modtagere af donationsløfterne jo selv tænke de tanker, idet de jo kan handle modtagelsesretten videre (selv med det samme, hvis de vil). ..Og det \emph{er} jo genialt (i.e. det bliver det). ..Uh, kom lige i tanke om: ..Forsikringsselskaberne kan vel sælge risici til donorerne selv, hvorved man så kan annullere risiko helt.. For så kan donor-grupper jo sige, "vi har det godt med, hvordan bedømmelsen af den samlede donationsstørrelse til bidragene inden for område x er ligenu, så derfor vil vi gerne gå med til at fryse beløbet ved at købe den modsatte risiko." Og så er både bidragsydere, der er interesserede i at fryse deres værdipapirer i værdi, og donorer, der er interesseret i det samme, jo glade.. Spørgsmålet er så bare, om dette også kan lade sig gøre, hvis vi kun taler mindre delmængder af bidragsyderne og/eller af donorerne..? ..Hm, tja, men donorerne vil på en måde altid være "forsikrede" i den vished, at hvis deres løfter handler om, at betale lidt af værdien tilbage for den nytte, de har fået fra bidrag på kæden, jamen så behøver de jo alrdig at betale noget, der kan virke som en uretfærdig sum penge, for så kan de jo bare undsige sig at gøre dette, uden at folk vil blive sure på dem. Ja, og for den sags skyld, kan de også altid bare for at skrive forsikringer ind i deres egne løfter, hvilket kunne være ting som, at det ikke må overstige et vist årligt beløb og/eller at uventede forhold i vedkommendes eget liv, som tager midler fra denne, kan undskylde vedkommende fra at skulle donere i sidste ende. Hm, og hvad med forsikringer til bidragsyderne, der kan fryse deres værdipapir-kurser..? Hm, der er vel bare netop at sælge dem, og så kan man altid bare sælge dem til et selskab, som så tager sig lidt ekstra betalt for at købe risikoen med. Ja..

(03.09.21) Jeg kom frem til, at det ikke fungerede alligevel, og dagen efter, mener jeg, det var, kom jeg så frem til, at det alligevel kan holde. Min version af idéen nu har ikke nogen deadlines og/eller nogen central enhed eller mekanisme i det grundlæggende, der skal sørge for at betale pengene til disse deadlines. Min idé nu handler bare om en kæde, hvor folk kan logge deres gode bidrag på, samt logge løfter om fremtidige donationer til gode handlinger og/eller handlinger, der kommer donoren selv til gode. Så simpelt er det\ldots! Jeg vedlægger lige mine brainstorm-noter, hvor jeg kom frem til denne nye version (inklusiv nogle opfølgende noter) som følgende paragrafer. Og ellers har jeg faktisk bare tænkt mig, at jeg vil redegøre færdigt for idéen i den næste sektion, som opsummerer nogle idéer, jeg (nok) gerne vil prøve at fremføre for verden i den kommende tid.

{\slshape
\%(26.08.21) Åh! Jeg var virkeligt klar til at afskrive meget af idéen..!.. Jeg har i øvrigt også gjort nogle andre tanker omkring kundedrevet virksomheder og forbruger-foreninger (inkl. en forbruger-web 2.0+-platform..) og ting i den retning. Men de var nu overvejende ret pessimistiske, ift. hvor jeg var i går formiddags.. Men nu her til aften kom jeg lige i tanke om, at LV-idéen måske kan virke alligevel!.. For man kan måske bare opbygge det sådan, at folk starter med at melde sig på et system (ved brug af tillid ('trust') og certifikater og tilhørende algoritmer osv.), som er beregnet på at konstruere og vedligeholde et lødigt kort (eller graf) over, hvilke bidrag er blevet gjort i fællesskabet og hvem i fællesskabet har gavnet af dem. Og pointen skal så bare være, at man ved at melde sig ind i sådan et fællesskab --- og der kan altså være flere af sådanne fællesskaber logget på samme kæde og kørende uafhængigt af hinanden --- erklærer, at man vil respektere processen, medmindre at der forekommer evidens på korruption af processen (hvilket man så kan erklære og melde sig ud, men så kan alle bare se dette, og vil generelt holde sig fra at stole på en i fremtiden, hvis folk generelt vurdere, at udmeldingen skyldtes selviske grunde), og at man vil sørge for at donere penge til folk, hvis bidrag man ifølge processens resultater har draget gavn af i en eller anden passende grad, som aftales på forhånd (mere eller mindre formelt). Så med andre ord melder man sig ind og siger: "Jeg vil gerne donere penge til folk, der yder bidrag som gavner mig, og jeg stoler på at proces x vil være i stand til at bedømme dette i god nok grad (og ellers må jeg bare gøre indvendinger og melde mig ud og håbe på, at folk vil se tingene fra min side i så fald), og jeg tror også på, at samme proces vil finde frem til fair donationsbeløber, som jeg så altså vil rette mig efter, hvis jeg har midlerne på hånde (og ellers må jeg jo bare erklære, at jeg holder pause fra hele systemet (inkl. andre konkurrerende varianter), og at jeg så betaler beløbet, hvis jeg melder mig ind igen)." Og hvis vi skærer al usikkerheden fra og antager, at systemet holder, og at folk overholder deres løfter, så melder man sig altså ind og siger: "Jeg vil gerne betale for handlinger / donere penge til folk, der har gjort handlinger, der i sidste ende gavner mig, og jeg vil gerne betale/donere et fair beløb." Folk der så yder et bidrag til hele eller dele af fællesskabet tager så selvfølgelig lidt en chance, for det kan jo være, at folk vil undsige sig fra at holde deres løfter. Men fordi folk generelt gerne vil holde sådanne løfter (og generelt altså gerne vil give tilbage, hvis nogen yder for dem), og fordi folk også kan holde øje med, hvem der tilsyneladende er svindlere, og så lukke dem ude af fællesskabet, indtil de går med til at betale, hvad der er fair (i det gennemsnitlige medlems øjne), så tror jeg altså, at det kan fungere! Ja..!! Så kæden i idéen, som den er nu, kommer altså til at bestå af, at folk notere deres bidrag helt generelt ligesom før, men at folk så også hver især melder sig til et fællesskab (som også er erklæret ved at uploade definerende/erklærende tekster til kæden) med en tilhørende proces for at bedømme, hvad individet skylder eller skyldes. Folk kan så melde sig ud og ind af fællesskaber, og kan også endda melde sig til flere på én gang, hvilket dog fra individets synspunkt helst kræver, at fællesskaberne arbejder sammen, så man ikke bliver bonget dobbelt for noget. Og hvis man melder sig ud af et fællesskab kan det så altså komme til at koste for ens omdømme, hvis man ikke gør det på en god måde. Ni-iice! (26.08.21)
\%..Og man kan sagtens bare starte med at melde sig til nogle ret vage fællesskabserklæringer, hvor man så i bund og grund bare erklære, hvilke nogen senere, mere avancerede fællesskabssytemer/processer, man vil melde sig til efterfølgende. På den måde kan man f.eks. melde sig til og love de første minere belønninger, men uden så at blive bundet til selve fællesskabet, så man måske får svært ved at melde sig til bedre udformede fællesskab-processer/-systemer senere. Man kan f.eks. bare melde sig til en fælleserklæring om at: "Jeg vil gerne give så og så meget til de første minere på kæden (der jo er med til at skubbe tingene i gang) alt efter, hvordan kæden udvikler sig, i henhold til sådan og sådan," og så kan man jo altid melde sig til andre erklæringer bagefter, så længe man bare sørger for at lave nogle passende offentlige donationer til minerne. Så ja, man styrer helt selv, hvor formelle og/eller præcise erklæringerne, man melder sig til, skal være, og hvor bindende tilmeldingen skal være osv., og erklæringerne kan sagtens altså bare udformes selv som nogle lidt løse sætninger (med tilsvarende løse løfter i). Jep. :) 




\%(31.08.21) Om LKV-idé: Jeg er faktisk kun blevet overbevist mere om brugbarheden i idéen om min donationsløfte-blockchain her i weekenden. Kæden kunne nemlig også selv sætte gang i arbejde på at udvikle web 2.0-agtige platforme, såsom især et forbruger-SoMe, som kunne spille rigtigt godt sammen med kæden. Og folk kan tilegne sig bedre handelsmæssige positioner i deres liv, hvis de beviser, at de er i stand til at holde løfter, så for den enkelte er der sikkert økonomisk og politisk magt at vinde ved at deltage i kæden og donere til ting. Værd at nævne også, at kæden jo vil kunne blive en god begrundelse til at sænke skatten for almindelige mennesker og lade tilsvarende, skatteagtige penge flyde mere direkte fra folk og til, hvor de synes. (Så vejen til at folk får mere direkte indflydelse over, hvor deres skattepenge går (hvilket mange sikkert vil se positivt på, og resten vil nok mest bare se mere neutralt på det frem for direkte negativt) --- og da det er en løbende proces, vil den jo også være rimeligt reversibel ift., hvis man bare gik over til mere direkte indflydelse fra den ene dag til den anden.) Jeg håber på, at folk vil vælge at se på samlede (og lødige) oversigter over, hvilke bidrag på kæden har bragt dem nytte, når de skal donere, og ikke på mere selviske varianter, hvor man kun ser på bidrag gjort efter, at man selv oprettede sig på kæden. Jeg tror dog, at folk faktisk gerne generelt vil vise denne godhed, og at de derfor gerne vil tilslutte sig (hvilket jo også gøre offentligt, så alle kan jo se, hvad andre tilslutter sig) de mere "gode" og retfærdige løftesystemer, så at tidlige bidragsydere, som har registreret deres handlinger på kæden, får deres retfærdige belønning. Og af denne grund, eller rettere fordi andre sikkert vil have en mere eller mindre tilsvarende tiltro til folk, så vil der derfor sikkert være mange, der gerne vil være med fra starten --- og bl.a. med til at skabe de gode web 2.0-agtige platforme, som kan fremme kædens brugbarhed (ved bl.a. at udgøre gode interfaces til kæden, og selvfølgelig også pga. diverse funktionaliteter som platformen implementerer oven på kæden). Det er selvfølgelig også vigtigt at understrege, disse ting taget i betragtning, at kæden jo også vil kunne booste donationer til diverse open source-bidrag, som allerede findes så mange af, og som bliver ved med at komme. Og nu med denne teknologi kan man så donere mere direkte til de gode bidrag --- også selv uden at tænke en hel masse over det, men bare ved at formulere nogle overordnede principper, for så kan modtagere af donationsløfterne jo selv tænke de tanker, idet de jo kan handle modtagelsesretten videre (selv med det samme, hvis de vil). ..Og det \emph{er} jo genialt (i.e. det bliver det). ..Uh, kom lige i tanke om: ..Forsikringsselskaberne kan vel sælge risici til donorerne selv, hvorved man så kan annullere risiko helt.. For så kan donor-grupper jo sige, "vi har det godt med, hvordan bedømmelsen af den samlede donationsstørrelse til bidragene inden for område x er lige nu, så derfor vil vi gerne gå med til at fryse beløbet ved at købe den modsatte risiko." Og så er både bidragsydere, der er interesserede i at fryse deres værdipapirer i værdi, og donorer, der er interesseret i det samme, jo glade.. Spørgsmålet er så bare, om dette også kan lade sig gøre, hvis vi kun taler mindre delmængder af bidragsyderne og/eller af donorerne..? ..Hm, tja, men donorerne vil på en måde altid være "forsikrede" i den vished, at hvis deres løfter handler om, at betale lidt af værdien tilbage for den nytte, de har fået fra bidrag på kæden, jamen så behøver de jo aldrig at betale noget, der kan virke som en uretfærdig sum penge, for så kan de jo bare undsige sig at gøre dette, uden at folk vil blive sure på dem. Ja, og for den sags skyld, kan de også altid bare for at skrive forsikringer ind i deres egne løfter, hvilket kunne være ting som, at det ikke må overstige et vist årligt beløb og/eller at uventede forhold i vedkommendes eget liv, som tager midler fra denne, kan undskylde vedkommende fra at skulle donere i sidste ende. Hm, og hvad med forsikringer til bidragsyderne, der kan fryse deres værdipapir-kurser..? Hm, der er vel bare netop at sælge dem, og så kan man altid bare sælge dem til et selskab, som så tager sig lidt ekstra betalt for at købe risikoen med. Ja..

}

Som nævnt vil jeg redegøre færdigt for denne idé i den følgende sektion.









\subsection{Opsummering af de idéer, jeg nok vil prøve fremføre til verden nu, som har med emner fra ovenstående sektioner at gøre (03.09.21)}

Jeg tror, som nævnt, jeg vil dele mine idéer op i fire, og så ellers plus det løse. Vi snakker min idé til en semantisk vidensdelingsside, som også kan præsenteres son en wiki-side, hvor titler også kan indeholde prædikater, der beskriver hvordan teksten bør udformes (således at samme overskrift kan fås i mange forskellige varianter), og som i øvrigt også bruger prædikater lave dispositioner til tekster, hvorved konstruktionen af en tekst bliver mere lagdelt designmæssigt. Vi snakker også min idé om en åben web 2.0-agtig platform, hvor brugerne selv står for at programmere og forbedre siden, og hvor folk så kan følge og like'e skabere, der tilføjer gode tiltag til siden. Denne idé spiller så sammen med den første, fordi den samme lagdelte (semantiske) designstruktur også vil være smart (tror jeg) for denne side, eftersom programmør-basen bliver så bred og decentral, og fordi produktet ikke bare skal være ét produkt, men mere a la VSCode, hvor brugerne helst skal kunne få det ligesom de selv vil. Min tredje idé er så min donationsløfte-blockchain, som i bund og grund handler om at fremme en ny tilgang til finansiering, som er mere bruger-/kundedrevet, og hvor gode gerninger kan blive belønnet uanset om yderen lige fik kæmpet sig til de rigtige copy-rettigheder eller ej. Og sidste idé handler så bare lige om en idé til en social platform omhandlende forbrug. Jeg har ikke udviklet så meget på idéen i skrivende stund, men jeg tror den bliver værd lige at nævne uanset hvad. Jeg har i øvrigt nogle idéer, som bl.a.\ handler om debat %, og nogle idéer, som handler 
og om at lave gode anbefalinger til internetindhold og andet til folk, som jeg også lige vil komme ind på, hvor det passer. Da mange af mine idéer her på en eller anden måde har med det semantiske web at gøre, så vil jeg nok også lige komme ind på dette, og måske forklare om, hvordan mine idéer kan være med til at fremme det semantiske web. 

Lad mig faktisk starte med at introducere det semantiske web først. Det vil jeg ikke gøre her, men jeg vil nok gøre det så i den renskrevne version. Så lad os forestille os, at denne paragraf er en tekst (sikkert opdelt i flere paragrafer), der introducere det semantiske web overordnet set, samt beskriver de fire nyttige muligheder, de vil bringe til webbet, som jeg ser det, hvilke jeg har beskrevet under ny-ny-nu-med-ny-tilgang-undersektionen ovenfor. Så langt så godt.

Min vidensdelingsside-idé tager så i høj grad udgangspunkt i det tredje punkt af de nævnte muligheder, som handler om, at folk kan finde frem til den rigtige \emph{udgave} af det, de søger. 
Hvis vi ser på en side som velkendte Wikipedia, så følger artiklerne gerne nogle konventioner om at være velformulerede %... %(og om at være rimeligt kortfattede og præcise...)%...Hm, og er der ikke en risiko for, at der allerede er mulighed for at lave bl.a. længere tekster på wikipedia, men at folk bare ikke har motivation til at gøre det..? ...Nej, det er en masse guidelines, der bl.a. handler om at være så koncis som muligt, og om ikke at inkludere noget som helst, hvor der er flere synspunkter..
og koncise i henhold til, hvad man forventer af en encyklopædi, samt i øvrigt kun at indeholde generelt accepteret viden. Men hvorfor ikke bare indføre disse konventioner som ét prædikat --- eller et sæt af prædikater, men man må altså også gerne kunne samle flere prædikater til ét, så det er nemmere at søge på --- og så give plads til en masse andre varianter af tekster? Vi ved jo, at folk i høj grad har behov for andre typer tekster; de har brug for lærende og introducerende tekster, hvis de skal lære om et nyt emne, og de har brug for artikler, der diskuterer ny viden og ting, vi ikke er nået til enighed omkring endnu, hvis de vil være med i denne proces eller bare er interesseret i lære om de forskellige muligheder? Og hvis så bare har denne horisontale opdeling (via de nævnte tekstprædikater) af de forskellige typer tekster på siden, så folk er klar over, hvilken type tekst de har valgt, hvorfor så ikke bare sigte mod at inkludere, alt hvad brugernes hjerter kan begære? Jo, pladsmangel kunne være et argument, samt at man måske er bange for, at det bliver for uoverskueligt. Angående pladsmangel, så vil jeg komme ind på, hvorfor plads ikke behøver at være en begrænsende faktor, idet siden skal implementeres decentralt (så folk kan bare gemme, hvad de selv er interesserede i, og/eller hvad der er efterspørgsel efter). Og angående uoverskuelighed, så kan man netop bare som læser holder sig til afgrænsende prædikatsæt, der indeholder konventioner om ikke at linke på kryds og tværs til andre prædikatsæt (medmindre det lige er oplagt for det givne emne). 

%Som eksempel ... %Arbejdsopgaver, (spørgsmål og svar?..), nsfw, videoer og humoristisk indhold, socialt indhold. ..Hm, men måske man lige skulle forklare om pointene først.. ..Og eksempler: Utilstrækkeligheden af de konventionelle principper om kildeanvisninger (kort).. 
%Tja, jeg kunne måske lige give give få eksempler så som arbejdsopgaver for at motivere idéen endnu mere, og så enten gå videre til at forklare point og semantisk struktur eller gå videre til at nævne, at jeg også har idéer, som måske kan medføre flere donationer, som sådant ekstra arbejde med at bygge lærende tekster osv. i fællesskabet måske kræver.. ... 

\ldots

(07.09.21) Ah, jeg har faktisk fundet ud af, at jeg kan fokusere mere på, og tage mere udgangspunkt i, ``rating-folksonomies,'' i.e.\ folksonomies, ikke over tags, men over udsagn med tilhørende rating (med knapper så brugeren selv kan bidrage med ét eller nogle få klik). Jeg kom nemlig frem til (i går), at sådanne faktisk udgør en ret simpel idé, som virkelig har mulighed for at blive en stor og udbredt ting på gængse web 2.0-platforme (tror jeg). Og selvom idéen om en udvidet wiki-side også er ret simpel og har stort potentiale i sig selv, så er denne alligevel også lettere at forklare om, hvis man allerede har forklaret, hvordan prædikaterne rate'es af brugerne. Og fordi der også er mere kommercielt kød på rating-folksonomy-idéen, så giver det rigtig god mening at starte med den. Hm, men det er nu egentligt ikke fordi, denne sektion skal have den helt rigtige disposition; min tilgang er først at skrive en sammenhængende tekst over idéerne (i denne sektion), og så kan jeg lave dispositionen (til en mere gennemarbejdet tekst) bagefter. Men lad mig alligevel hoppe over og prøve at beskrive rating-folksonomy-vinklen nu.

Idéen går kort sagt ud på at skifte de gængse tags i folksonomies ud med udsagn, der så kan vurderes (rate'es) af brugerne i henhold til, hvor enige de er med udsagnet samt hvor vigtige de synes udsagnet er for indholdet. Den sidste del er knap så vigtig, men kan bruges som hjælp til at få de mest relevante rating-tags for indholdet vist til brugeren. Udsagnene vises stadig som tags, når brugeren anskuer dem, men forskellen er bare, for det første at disse tags nu indeholder en rating, hvor brugeren selv kan give sin vurdering også, og for det andet skal der nu også være et link til en tilhørende dokumentation for tagget, der beskriver semantikken bag det udsagn, det repræsenterer. Dette er de to vigtigste forskelle. I øvrigt kunne man så også med fordel indføre prescripts til tags'ne, således at de kan grupperes, når brugeren anskuer dem. På denne måde kunne man f.eks.\ indføre et `emne'-prescript, således at alle udsagn-tags, der beskriver hvilke emner, indholdet handler om, grupperes sammen, og man kunne også indføre andre grupper så som `tekst-udformning,' `advarsler' og `vurderingsudsagn' (e.g.\ `humoristisk,' `lærerig' etc.), og hvad man ellers kan finde på. 

Allerede med disse få ting tror jeg, at der er et ret stort potentiale. Ikke mindst fordi gængse folksonomies altså bare ikke fungerer super godt, som de er nu; det er tit, der sniger sig et irrelevant tag ind. Og man kan sådan set bare starte med dette og så eventuelt udvikle systemet mere senere hen. De næste tiltag jeg så vil foreslå handler om at åbne op for, at man kan rate specifikke delelementer af indhold (så som tekst- eller videoudsnit), og også om at indføre mulighed for mere komplicerede aggregater end bare den gennemsnitlige brugervurdering og antallet af svar på evalueringen. Og angående sidstnævnte vil jeg så også foreslå, at brugere kan vurdere hinanden (hvilket kan åbne op for en slags FOAF-credibility assignments, i.e.\ for algoritmer, der inkludere brugernes interne vurderinger af hinanden, når aggregatet skal udregnes) samt oprette en slags grupper i det samlede brugernetværk (som kan bruges til samme formål).

%Det kan også godt være, at jeg skal prøve at få det om produktvurderinger ind her tidligt, men det kan jeg lige se.

Lad mig starte med at forklare om tiltaget, hvor der åbnes op for, at specifikke dele af diverse indholdsobjekter kan vurderes. Dette handler altså om at designe et overlay af annotationer, som brugerne slå til og fra, og hvor brugerne selv kan vælge, hvilke udsagns-tags skal inkluderes. Der er i øvrigt intet til hinder for, at man bruger de samme tag-definitioner, som bruges til at vurdere de fulde indholdsobjekter. For tekster bør annotationerne så kunne vises via tekst-highlights. Også i denne forbindelse synes jeg, at min tilgang bringer noget nyt og brugbart på banen, for nu vil jeg så foreslå, at disse highlights kan afhænge af den pågældende (aggregerede) vurdering. En simpel måde at vise vurderingsaggregatet på kunne jo så være at have en slags superscripts eller tilsvarende, der viser værdierne, men jeg vil også foreslå noget andet, nemlig at det også skal være muligt at få highlight-farven og/eller -gennemsigtigheden til at afhænge af vurderingen. Herved kan man sørge for, at brugere hurtigt kan få overblik over de mest vigtige (pr.\ andre brugeres vurderinger) udsnit, når det kommer til f.eks.\ en advarselsannotation. Dette kunne så være advarsler a la `nsfw' m.m., men det kunne også være advarsler sådan som ``dette udsagn er ikke understøttet af kildehenvisninger'' og ting i den dur, som handler om, hvor pålidelig og/eller hvor kontroversiel kontra alment accepteret udsagnene i teksten er. I øvrigt vil jeg også komme ind på senre, hvordan disse idéer kan bruges til vidensdelingssider (a la Wikipedia), og her kan det også blive brugbart, at man hurtigt kan få et overblik over `tekstudformning'-prædikater, så man kan se, hvor der måske er brug for at blive redigeret i teksten. 

Og lad mig så forklare om bruger-til-bruger-vurderingerne og den del af det. Hvis man ser på gængse idéer inden for området omkring `det semantiske web,' og navnlig indenfor området omkring, hvordan `credibility'/`trust' uddeles i sådan et decentralt netværk, så kan man allerede få et rimeligt godt overblik over, hvordan bruger-til-bruger-vurderinger kan føre til, at hver bruger kan få vist vurderingsaggregater, der passer bedst til dennes antagelser om pålideligheden af andre brugere. Inden for det semantiske web benyttes det, der hedder FOAF (friend of a friend), til at indkapsle bruger-til-bruger-vurderingerne. Der er i øvrigt mange af de idéer, jeg vil præsentere her, som ligger tæt op ad gængse `semantiske web'-tanker, men det vil jeg komme ind på i et senere afsnit her. Nå, men lad mig så beskrive mit take på et sådant system til at fordele tillid og pålidelighed/troværdighed blandt brugere (som altså minder meget om det gængse, men jeg har også lige nogen egne idéer). 

%Hm, måske kunne man sige, at venners venner (men ikke venners venners venner) skal kunne se en rigtige navn, så man forhindrer, at folk snyder sig til flere aliaser uden at vennerne ved det, fordi de bare spørger forskellige i venne-netværket..
\ldots Hm, men måske har jeg faktisk en ret anderledes tilgang til det ift.\ de gængse tanker (omkring FAOF og `trust')\ldots\ Ja. Jeg tror, jeg vil foreslå, et (lidt mere centraliseret) system, hvor brugere har en helt eller delvist officiel profil og også en anonym profil i netværket, hvor den ikke-anonyme profil kan ses enten af alle eller af venner og muligvis venners venner osv.\ (i en grad man selv vælger), men hvor den anonyme profil ikke umiddelbart kan ses af nogen andre. Man har så profilerne hos en central instans, hvilket passende kan være den instans, der styrer den pågældende platform, hvis vi snakker en gængs web 2.0-side. Det kunne dog også være en instans, som muligvis er uafhængig af web 2.0-platformene, men hvor flere forskellige web 2.0-platforme så kan query'e denne instans, så vurderingsdata kan aggregeres og vises på tværs af flere platforme. Uanset hvad, så kræver systemet dog altså, at man lader en central instans varetage sin anonyme data. Og hvad kan man så bruge denne anonyme data til? Jo, man kan så som bruger samtykke til, at dataet kan indgå i visse aggregat-algoritmer, som så kan designes således, at de ikke afslører, hvilke vurderinger hører til hvilke (ikke-anonyme) profiler. Hm, måske burde jeg endda ikke kalde dem for `profiler,' de anonyme af slagsen, for man kunne med fordel helt undlade, at folk for mulighed for at gå ind på de anonyme profiler og se den samlede data for denne. Nej, så vi snakker bare `profiler,' som så kan være helt eller delvist offentlige, og disse profiler kan så have anonymt data, som ikke umiddelbart kan ses, men som altså kan indgå i visse (ikke-afslørende) aggregater, som brugeren har givet samtykke til. Hele pointen er så, at brugere kan tildele hinandens offentlige profiler tillid, blandt andet især tillid til, at vedkommende er en faktisk person og ikke en bot, samt også at vedkommende ikke har flere profiler i netværket (så personers vurderinger ikke er duplikeret i aggregaterne), og at denne tillid så i nogen grad kan indgå, når man forespørger aggregater, der også baseres på det anonyme data i netværket. Så hvis man f.eks.\ ikke har tillid til, at en vis mængde af profiler ikke er bots, og dermed gerne vil have deres vurderinger filtreret fra, så kan man altså også få deres vurderinger filtreret fra aggregater baseret på anonym data. Aggregat-algoritmerne skal så bare designes således, at brugere ikke ved at stille på forskellige parametre for, hvilke profiler skal sorteres fra (eller have deres vurderinger vægtet med en mindre faktor), så på denne måde kan måle den underlæggende datastruktur særligt præcist, og dermed altså slet ikke præcist nok til at kunne sammensætte individuelle vurderinger med individuelle profiler. For at opnå dette kræver det så, at brugere ikke bare kan filtrere vilkårlige profiler fra, når de vælger deres filter (altså når de ``stiller på parametrene'' for at genbruge dette udtryk). Brugere skal altså kun kunne filtrere større grupper fra ad gangen, og hertil vil jeg altså så foreslå, at det bliver en integreret mulighed i systemet, at brugere kan oprette (og muligvis administrere) og tilmelde sig brugergrupper (i.e.\ tilmelde deres profiler til brugergrupper). En `brugergruppe' er så basalt set en protokol til at fordele specielle `trust'-/`credibility'-værdier til folk, som folk så kan få lov --- muligvis først efter at have ansøgt om dette --- til at vægte deres aggregat-forespørgsler med, også altså når det kommer til aggregater over den samlede data inklusiv den anonyme. Hvis folk gerne vil bevare deres anonymitet i høj grad, så skal de så helst ikke melde sig alt for meget ud og ind af brugergrupper, og brugergrupperne bør også have en politik om kun at opdatere sig selv (hvorved `trust'-/`credibility'-værdierne ændrer sig, og muligvis bliver nul for de brugere, der melder sig ud) med mellemrum lange nok til, at der er sket så mange ændringer, enten i form af offentlige ændringer (i.e.\ hvis folk har meldt sig til eller fra, eller hvis de er blevet givet en anden værdi/vægt i gruppen), eller i form af anonyme ændringer (i.e.\ hvis vurderingsdataet har ændret sig), at man ikke udefra kan spore, hvilke profiler første til hvilken ændring, når de har meldt sig til eller fra, eller hvis de er blevet givet en anden `trust'-værdi i gruppen. Så det er altså tanken. I øvrigt har jeg kaldt værdierne for `trust'- og/eller `credibility'-værdier, men dette har jeg bare gjort, fordi disse termer bliver brugt i gængse `semantiske web'-tanker. Det behøver altså ikke være værdier, der signalerer præcis disse prædikater; det kunne f.eks.\ også bare være `relevans' for et givent; f.eks.\ er det værd at høre denne profils mening, når det kommer til det og det emne. Det kunne jo eksempelvis være, at brugeren er berørt af emnet, hvis man nu eksempelvis diskutere et eller andet tiltag. Man kunne sikkert også finde på mange andre prædikater, som kan være værd at bruge i filtre, når vurderingsaggregater skal udregnes og vises --- og jeg har faktisk også nogle andre interessante eksempler, som jeg vil komme ind på lidt senere. %i.e. ML-teknikker og menings-/psykologigrupper.

%..Jeg skal vist også lige forklare om, hvordan folk vurderer hinanden, og det jeg nævnte over sidste paragraf.. ..Ah, men det kan jo komme i forbindelse med næste paragraf.

Og hvordan skal `brugergrupperne' så oprettes og administreres? Tja, som sagt er det jo bare en slags vægtning, som kan få lov at indgå i aggregat-algoritmerne, når en central instans skal udregne aggregater over anonymt data. Så for det første handler det altså om at udforme en algoritme til at uddele værdier/vægte til medlemmerne af gruppen. En vigtig form for algoritme bliver som nævnt en bare til at vurdere, om profilen er en bot eller en faktisk person --- for alle brugergruppe-algoritmer vil jo nok komme til at bygge på sådan en algoritme som deres grundlag, hvorfra de kan bygge videre. Som en del i at skabe grund for sådanne algoritmer, vil det således være smart, hvis man opfordrer brugere til at linke deres profiler med hinanden for folk de kender, så man på den måde opnår en (simpel) FOAF-graf. Man bør altså opfordre folk i netværket til at forbinde med nogen af de folk, man kender, sådan at andre brugere i netværket kan tildele en tillid (til at være en rigtigt (og ikke-duplikeret) person). Og nu kan man måske forstå, hvorfor jeg synes, det er vigtigt at brugere skal kunne afgive deres vurderinger anonymt, for hvis man skal til at bygge applikationer på en teknologi, der kræver at profiler bliver linket til den faktiske person bag, og endda at man sørger for at forbinde med bekendte personer, så ville personer jo pludselig til at være omhyggelige, hvilke vurderinger de giver på platformene, medmindre de vil have alle i deres personlige netværk til at kende alle deres interesser og meninger. Og jo, Facebook er et godt eksempel, hvor mange mennesker har de fint med at fremvise deres interesser og meninger til deres personlige netværk, men her snakker vi også en lukket platform, hvor folk altid er har i mente, når de liker noget, at dette kan ses af andre. Men min forestilling er, at dette netværk skal kunne bredes ud på hele nettet (note: dette skal jeg nok klargøre tidligere i teksten), så brugerne kan bruge samme tillids-algoritmer når det kommer til alverdens vurderinger rundt omkring på nettet. \ldots Ja, jeg bør nok nævne dette tidligere, og nævne lidt om muligheden i at oprette et firma, der udbyder en serverless applikation, hvor udviklere af alverdens sider kan inkludere vinduer ved siden af sidernes indhold --- eller over, hvis man vil have highlight-annotationer m.m. --- hvor ratings fra dette netværk kan vises. Og hvis dette er tilfældet, så er der rigtig mange mennesker, der ikke har lyst til hele tiden at tænke over, om f.eks.\ deres chef eller deres mormor kan se, hvad de vurderer rundt omkring på internettet (e.g.\ hvilke natklubber, man besøger, eller hvad har vi). Og det behøver faktisk ingen gang være fordi, vedkommende har noget at skjule, men rigtig mange mennesker ville alligevel have det langt rarere, hvis de ikke skal tænke over, hvem der kan se det, når de vurderer ting, men bare kan gøre det anonymt. Så derfor tror jeg, at min idé her omkring et netværk, hvor man kan lave FOAF-baserede tillidsfordelings-algoritmer, men hvor svarene bag de aggregerede rating-resultater, og hvem der sagde hvad, stadig holdes anonymt (medmindre man altså tilføjer vurderingen til sin offentlige profil også).  


Nå og lad mig så lige snakke lidt om det her med, at netværket bør bruges på tværs af platforme, og så kan jeg i øvrigt også lige komme ind på et andet hængeparti, nemlig omkring mere semantiske produktvurderinger, hvilket jeg måske også bør komme ind på tidligere. Disse tanker bør altså muligvis komme tidligere i teksten. Jeg har været lidt inde på det allerede, at netværket kunne bruges på tværs af platforme. I forhold til at sælge idéen til gængse web 2.0-platforme, så kan man sige, at en platform, der er hurtigt ude og får oprettet systemet, jo kan tilbyde det som en serverless service til andre platforme. Jeg tænker lidt ligesom at Facebook tilbyder andre udviklere, at de kan få adgang til brugerens Facebook-netværk via\ldots\ Tja, eller her taler vi vel egentligt bare om, at folk kan dele indhold via Facebook, men jeg ved, at der findes andre services, som varetager rating-data, hvor udviklere så kan bruge disse services til at implementere ratings på deres sider\ldots\ Anyway, så idéen kan altså godt nå ud til mange steder på internettet, også selvom den bliver udviklet og igangsat af en eller anden større kommerciel IT-virksomhed\ldots\ %Hm ja, man kan vel gøre to ting... %enten oprette et firma, der selv holder på al data, eller.. bare holde på data fra brugerne, men..? Hm.. Ja, hvad kan man ellers gøre, kan man bare nøjes med at holde netværksdata, men ikke holde selve vurderingerne..? Tja, men det er vel bedre at samme netværk holder alle vurderingerne, og at man så bare kan give certifikater for rating-snapshots... Og så kan man altid bare dele netværket op i flere centraler... Hm, men der må være alternative muligheder også.. ..For det kan jo sagtens være, at profil-netværket (med offentlige profiler, men med anonymitet alligevel) først vil komme til senere.. ..Hm, ja så der er jo muligheder uden at data behøves at kommunikeres på tværs af platforme (andet end rating-udsagns-dokumentationer), og først når man prøver at implementere offentlig-profil-men-anonymt-data-netværket.. ja, så bliver det vel først mere relavent der at gøre det tværgående..? ..Hm, men vel det netop ikke være totalt smart (lidt ligesom at Facebook kan tilbyde), at brugere ikke behøver at oprette en konto?(!..) Jo, det kunne da også være relevant at fokusere på tidligt (og måske lade det være indgangsvinklen til at oprette det tværgående system).. Hm, ville Facebook forresten egentligt ikke være en klar fordelagtig spiller til at implementere dette netværk..? ..Jo, det ville de vel.. Hm, det er bare ikke sikkert, at dette vil være super fremtidsholdbart... ..Hm, problemet er lidt, at Facebooks model handler om at bl.a. at sælge folks data videre. Men man kan jo sige, hvis eferspørgslen er der, så kan man jo bare starte et firma, som folk kan skifte til, hvor de bare kan tage deres FB-data med sig. Så ja, Facebook vil være en klar mulig spiller, og så kan man jo altid prøve at overgå dem. Men ja, det er en god vinkel, den jeg kom frem til her med at begynde at introducere et mere grundigt brugernetværk omkrimg rating-folksonomies'ne, ved at se på, at det jo vil være rart, hvis man ikke hele tiden skal logge sig på, men bare kan give sine ratings overalt på nettet. Uh, og så kommer motivationen for annonymiteten også inden at jeg begynder at forklare om, hvordan man kan opnå det, hvilket jo er smartets.  
Okay, nu har jeg en bedre idé om, hvordan jeg skal omstrukturere dette. Efter at jeg har redegjort for mulighederne med folksonomy-ratings i sig selv (hvor brugere dog kan copy-paste rating-udsagns-dokumentatioenr osv.\ imellem sider), så kan jeg komme ind på, at det jo vil være smart, hvis det gøres mere tværgående, bl.a.\ så man heller ikke skal oprette en profil på hver enkle side, men bare kan logge ind med den samme profil. Hvis jeg så begynder at forklare FOAF-algoritme-mulighederne i at benytte offentlige profiler --- og jeg kan i øvrigt godt kommer ind på her, at Facebook kan blive en oplagt spiller --- så får jeg også motivationen for at indføre anonymitet til systemet, inden jeg begynder at forklare om, hvordan dette kan gøres (hvilket jo vil være smart ;)). Og angående Facebook, så kan jeg så komme ind på faren ved, at lade Facebook styre spillet, men påpage, at man let bør kunne oprette et konkurrerende system, hvis efterspørgslen er til det. 

Ah, og nu (d.\ (08.09.21)) er jeg også lige kommet frem til, at jeg skal have kommentar-ratings og, videre, tilhørende kommentar-filtre med ind i billedet tidligt. (Og før nu havde jeg i øvrigt ikke tænkt så meget på denne del (altså ikke i denne omgang), så det er dejligt, at jeg kom i tanke om dette.) Dette bliver så også en god motivation for at udvide systemet og gøre det mere tværgående samt lægge op til bl.a.\ bruger-til-bruger-vurderinger. 

\ldots

%Okay, så angående mere semantiske produktvurderinger for det første, så handler det om, ...
%
%...
%
%Strukturen på denne tekst er rodet, men vi kan lade som om, at jeg nu har motiveret og forklaret de anonyme vurderinger og tilhørende `brugergrupper' i den rækkefølge. Nu er jeg så nået til at skulle forklare, hvordan man kan lave mere komplicerede aggregater end bare gennemsnit og antal\ldots\ Nå nej, det har jeg berørt lidt med de her brugergruppe-vægtninger\ldots\ %(Hvad er der så mere, jeg skal forklare om her?) ..Eller skal man bare lige føre det videre på psyk-/menings-grupper..?


%(11.09.21) Brain: Jeg er kommet frem til nogle få ting, som ændrer det en anelse. Meget af det er lidt subtilt (og handler mere om, hvad jeg \emph{ikke} behøver at fokusere så meget på..), men en af de vigtigere ting er, at mine 'brugergrupper' måske kan gøres lidt mere simple. Ja, de kan, og bør nok, gøres på en måde, hvor der ikke lægges så meget op til tekniske algoritmer, men mere simple regler for, hvordan vægt kan gives videre fra bruger til bruger i netværket, og altså regler for, hvordan denne vægt propagerer rundt (og også hvornår den f.eks. aftager og sådan..).. Og angående det med anonymitet kan det så introduceres mere simpelt ved at sige, at brugergruppereglerne og vægtene i dem er offentlige, men at individuelle vurderingsværdier ikke behøver at være det. Og så kan jeg nemlig altid bare forklare senere om, hvad man kunne gøre, hvis man vil starte et nyt system, hvor der ikke er nogen central på forhånd, og/eller hvor der kan være flere centraler, der kan startes og udskiftes løbende.. Tanken er så i øvrigt, at brugergruppen skal være frie til at ansøge om at forke, hvorved vægtene sådan set bare sættes til noget nyt, alt efter hvad delgrupperne bliver enige om. Og grupper må så meget gerne fokusere på et emne, således at hvis man som bruger, hvis man gerne vil søge efter meninger fra en gruppe af mennesker med med flere uafhængige prædikater om sig, jamen så kræver det ikke, at der lige findes en officiel brugergruppe med alle de pågældende prædikater; i stedet kan man bare finde hver gruppe med de ufhængige prædikater om sig og så søge på indhold, som alle disse grupper har vurderet højt.. Hm, et eksempel kunne være.. ..Ha, det kan jeg åbenbart ikke lige finde på.. Det må jeg hellere tænke noget mere over så. Men pointen er altså, at man nok kan komme rigtigt, rigtigt langt med noget virkeligt simpelt, som selv langt de fleste ville kunne forstå ret intuitivt. For idéen er nemlig, at det så ikke drives af underliggende algoritmer, men at vægtene propagerer og ændrer sig som følge af brugeres handlinger. ..Ja, man kommer nok nærmest til at kunne se det som, at brugerne uddeler mønter til hinanden.. Så det'.. Og jeg skal vist nok bare fokusere særligt på det at kunne gruppere kommentarer samt efterspørge kommentarer (og/eller links) med nogle faste semantiske udsagn, og så det her med at kunne forstærke denne funktionalitet, samt den mere simple, som bare er at rate selve indholdet, via disse brugergrupper. Og ellers vil jeg bare nævne, at der måske er et potentiale i at begynde at fokusere meget på semantiske sammenligninger.. nå ja, og så det mere simple også, hvilket jo er semantiske udsagn(s-tags/ratings), så man altså opfordrer brugere til at bruge dokumenterede tags af alverdens semantik (ved bl.a. at gøre det rigtigt nemt at rate disse tags, nemlig ved også at have eksplicitte, synlige rating-knapper på tags'ne (og altså have et link til dokumentationen; disse to ting er forskellen fra gængse tags)).. ..Hm, så jeg skal vel bare lige finde de gode eksempler på, hvornår man kan bruge disse tags, når det kommer til anbefalinger (jeg kan nemlig let finde eksempler på de andre ting), og så skal jeg jo også lige finde nogle gode eksempler på, hvad man kan bruge grupperne til (og gerne udover bare "safe-space" (kontra ...) og ekspertviden for området..)... ..Hm, og måske bliver de mere komplicerede anbefalings-/emne-/prædikat-tags også mere brugbare, hvis man netop fremhæver særlige grupper, for så kunne man nemlig lave anmelder-grupper (og man kunne jo hurtigt påpege, at det ikke ville være dumt, hvis man også kunne donere til disse grupper). ..Hm, 'anmeldergrupper' (muligvis semiprofessionelle) kan godt gå hen og blive en ret vigtig ting, og en ret vigtig pointe.. Hm, ikke just dumt.. ..Ja, og så bliver det med at forbinde dokumentation til tags altså bare super vigtigt.. ..Også selv uden særlige anmeldergrupper; det bliver bare vigtigt for anmeldelser med mulighed for mere uddybende/ede tags. Jep, så jeg skal nu bare lige tænke over nogle flere muligheder med (psyk.-)grupperne, og så har jeg en rimelig simpel folksonomy-idé at sælge (hvor jeg så kan føje nogle flere tanker og idéer efterfølgende, og hvilken også gør det (endnu) nemmere at forklare om min wiki-idé (som ellers også allerede er gjort ret simpel)).. ...Tjo, men psyk.-/anbefalings-/menings-grupper kan vel også bare startes med udgangspunkt i nogle personer/profiler, hvor man så giver et emne samt nogle overordnede ord for, hvad der repræsenterer denne meningsgruppe, og så er man vel allerede i gang..?.. ..Ja, det er faktisk så simpelt.. Okay. Jeg fik lige nogle tanker omkring brugerundersøgelser (som muligvis bare hører til forbrugerforeningsidéen, men det kan jeg lige se), som jeg lige vil tænke over, men ellers er det altså bare at få redegjort for disse (simple) nævnte ting.. ... Ah, yes. Jo, der er nemlig værd at have med til denne idé, at brugergrupper jo også kan have som mål at aggregere repræsentative (måske bare for en undermængde af brugerskaren) anmeldelser for produkter, hvorved producenterne (og muligvis konkurrenterne, men så hører det mere til anmeldelse-delen stadigvæk, og ikke til konstruktivt feedback-delen) jo kan sponsorere sådanne grupper, fordi de vil være interesserede i en sådan repræsentativ feedback --- hvilket jo nemlig i øvrigt også kan inkludere ønsker til fremtidige produkter. Så yes. Men så tror jeg også umiddelbart, det er det, for den grundlæggende redegørelse for (folksonomy-)idéen. 
%Hm, og hvad med brugergruppe-algoritmer, der afhænger af andre brugergrupper? ..Tja, det spørgsmål svarer jo lidt bare sige selv: Man bør kunne oprette brugergruppe, hvis endelige vægte kan afhænge af andre brugergruppe-vægte (evt. faktoreret sammen, eller hvad man har lyst til). Jep, cool nok. I øvrigt skal også lige nævne, at mine brugerstyrede, certifikerede aggregat-udregninger også kan implementeres via sådanne 'brugergruppe'-systemer. De handler jo bare om at oprette en brugergruppe med en meget begrænset vægt-propagation, hvis formål er at udregne visse aggregater (af andre ratings og evt. afhængigt af vægte fra andre brugergrupper). 
%(12.09.21) Jeg kom lige på i går, at man jo også kunne lave en seperat SoMe-side omkring debat. Tanken er så at folk her kan aftale at holde (strukturerede) debatter med hinanden (hvorved begge parter kan stige og falde i popolaritet hos andre brugere, altefter om de gør et godt arbejde med at få de relevante pointer igennem (og hvad ellers folk går op i; vi snakker bare ren og skær popularitet (som kan måles nogenlunde i antal følgere og likes))). Folk kan så oprette individuelle profiler, men bør også kunne oprette grupper, så grupper kan arbejde sammen om at debatere hinanden. Jeg tror denne idé har kommercielt potentiele som en SoMe-/web 2.0-platform. Og den vil også bringe andet godt med sig, fordi debatterne vil fremhæve emner og pointer, og også fremhæve mennesker med flair for at diskutere/analysere emner samt træne både deltagere og publikum til debatterne i rationelle analyser af emner. Og debatterne skal altså selvfølgelig have mulighed for at være struktureret ikke-lineært og med brug af tekstprædikater og sådan (hvilket jo især er smart, når flere arbejder sammen om at skrive en argumenterende tekst). Og så bør jeg i øvrigt også fremhæve min pointe om, at selvom diskussionsgrafer (mod-/med-argument på mod-/med-argument på mod-/med-argument...) kan være gavnlige, så er det bedst, hvis man kan ende ud med at forme nogle sammenhængende tekster, der redegør diskussionen generelt samt for hver enkelt synspunkt (af de relevante, men vil inkludere), så der altså bliver en sammenhængende tekst fra hvert relevant synspunkt af (og som altså prøver at "overbevise" læseren om, at grundsætningerne fra det pågældende synspunkt er sande). Sådanne sammenfattende tekster bør selvfølgelig helst addressere så mange af de vigtige pointer fra de konkurrerende synspunkter som muligt, men altså uden at det løber helt løbsk i en argument-på-argument-på-argument-graf. 


\subsubsection[Forfra]{Jeg starter lige forfra igen}
(11.09.21) Okay, jeg tror, jeg vil prøve at skrive det lidt forfra igen, for jeg tror nemlig ret hurtigt, man kan redegøre for det mest vigtigste ved mine tanker ret kort i virkeligheden, og uden at komme ind på særligt mange tekniske ting først. 

%%(12.09.21):
%Mine første forslag til, hvordan man kan forbedre folksonomy-systemer på internettet, kommer så til at gå på at få en ny form for tags, hvor der for det første er knyttet en dokumentation til tagget, der formulere et udsagn, og hvor tags kan rates direkte af brugerne, når de ser dem tilknyttet et objekt. Brugere bør altså kunne trykke på tagget og få en drop-ud-menu, hvor de kan afgive svar til to ratings. Den ene rating handler om, hvorvidt og/eller hvor godt udsagnet passer for den pågældende ressource (i.e.\ det pågældende ``indholdsobjekt,'' som jeg før har kaldt det her), og den anden rating handler om, hvor relevant dette udsagn er for ressourcen --- eller rettere hvor relevant det er for brugere, der iagttager ressourcen. Og hvordan finder man så ud af, hvilke tags hører til hvilke dokumentationer, og hvordan undgår man forvirring? %...Hm, jeg tænkte lige en nem løsning for folk, der har en konto, men hvad med folk, der ikke er logget ind på en konto?.. ..Hm, men dette bliver da ikke noget problem rigtigt.. Hm, kunne man ikke bare gøre det sådan, at brugerne kan stemme et ikon ind for tagget..? ..Hm, og hvis det alligevel sker med en central administrator, så kan denne jo bare kontrollere ikonerne, så det ligesom bliver en anmodning-afstemning-protokol.. ..Ah, man skal jo alligevel være logget ind (sandsynligvis) for at kunne afgive en stemme, og det er kun her, det rigtigt bliver relevant, så all good. 
%%Jo, man sørger bare for at forskellige tags med identiske tag-navne (men med forskellige udsagnsdokumentationer) for et superscript og/eller et andet ikon, i hvert fald hvis de ikke er den mest populære version. Så den pågældende platform, der indfører denne type folksonomy-system, skal altså holde øje med populariteten af tags'ne, hvilket altså vil kunne måles nogenlunde på deres generelle 'relevans'-ratings, og skal så %... Hov, er det ikke bedre, hvis forfatteren og/eller brugerne generelt for besked om, at der er en genganger, og at de bør ændre tag-navnet på en eller flere af gengangerne, så de kan kendes fra hinanden. ..Hm, og det ville da ikke være en dum idé, hvis der også kan vælges et ikon til tagget.. Det kan man jo overveje..
%Jo, her må den platformen, der implementerer folksonomy-systemet bare være vakse og sende meddelelser til forfatteren/erne til et tag, og/eller til hele brugerskaren, om at to populære tags har samme tag-navne (men forskellige udsagnsdefinitioner), og bede dem om at ændre i en eller flere af dem, så man kan adskille dem fra hinanden. I øvrigt kunne man overveje, om man vil give mulighed for, at brugere kan vælge små ikoner til tags'ne, både for at kunne genkende dem bedre og adskille dem fra hinanden, men også muligvis bare hvis dette forbedrer udseendet på platformen. Samtidigt vil jeg dog også foreslå, at man for brugere, der er logget ind, også sørger for at gøre tags, hvis definition brugeren endnu ikke har gennemgået --- og/eller sat flueben ud for --- halvgennemsigtige, så brugere kan se, at de skal læse dokumentationen, før de selv rater tagget, og før de fortolker resultatet af de tilhørende ratings. 
%
%\ldots Nå ja, jeg bør faktisk introducere idéen om at have definerende dokumentationer til tags mere selvstændigt, og så følger det nemlig ret meget heraf, at man selvfølgelig så bør rate relevansen og korrektheden %..Hm, skal man virkeligt dele det op i to..?
%af sådanne tags. For det duer jo ikke, hvis brugere skal ind og skrive dokumentationerne om selv\ldots\ Hm, tja\ldots\ Tja, whatever. For mig virker det oplagt, at tags skal kunne vurderes med enkelte klik fra brugere, der iagttager ressourcen. Hm, og jeg ved ærligt talt ikke, hvorfor jeg ikke synes, jeg har set det før på den måde\ldots\ \ldots Hm, måske er det bare fordi, nuværende tags ligesom ikke behøver særlig stor grad af præcision\ldots\ Nå, det kan jeg jo eventuelt lige tænke over. Noget andet, jeg skal tænke over, er, om det virkeligt er nødvendigt at dele ratings'ne over i to; skal man virkeligt inkludere `relevans'-ratingen? ...
%
%
%Og hvad får man så ud af at indføre definerende dokumentationer til tags? Jo, der er mange muligheder, hvor et sådant system kan forbedre brugbarheden og brugeroplevelsen af platformen. ...



%(12.09.21):
%Mine første forslag til, hvordan man kan forbedre folksonomy-systemer på internettet, kommer så til at gå på ...%... Hm, det kunne være, at jeg skulle fokusere lidt mere på kommentar-tags/-grupperinger (hvilket i øvrigt senere også kan blive til annotations-ratings; jeg har ikke snakket om det i noget tid, men annotationer er jo den anden ting man kan bidrage med til en ressource, og det kan jeg altså så komme ind på efterfølgende...).. ..Hm, jeg føler, jeg har en ret god, simpel idé, man hvordan får jeg lige vinkelt den, for den afhænger måske lidt af flere forskellige principper, der skal spille sammen..?.. ...Hm, okay, i første omgang kommer brugbarheden nok i høj grad i at bunde i, at brugere nu kan efterspørge specifikke typer af kommentarer (eller annotationer) omkring ressourcer.. (ikke?).. ..Og at man også ret nemt kan gøre efterspørgslen af visse relevante kommentartyper, der typisk er relevante, så de lidt efterspørger sig selv.. Hm, og nu kom jeg så til at tænke på, om man skulle have det sådan, at profiler kan få point relateret til et specifikt tags, således af disse point fortæller: "vedkommende er god til at besvare på dette spørgsmål / give rette kommentarer af denne type." Ja, det kunne nok være meget smart.. ..Hm, for selvom mere uddybede anmeldelses-tags i sig selv kan blive vigtigt også uden brugergrupperne, så kan det nok godt være, at man helst skal bruge brugergrupperne, og altså nogle specielle profiler, der specialiserer sig i at finde på anmeldelserne, før at der rigtigt kommer gang i det.. hm, men nu sagde jeg jo "profiler;" dette kan jo godt bare være individer, så..? Ja.. ...Hm, jeg tænkte over, om man skulle have særskilte grafer, som diverse brugergrupper kan vedligeholde, og som implementerer en måde at vælge, hvilke faste tags skal vises for diverse ressourcer..?.. ..Hm, grunden til, at jeg tænkte på det var, at jeg overvejede, hvordan man frem-rater nye anmeldelses-tags... Hm, men er det ikke lige præcis, det man bør gøre..!..? At sørge for at have tag-grafer, så tags som sagt kan blive foreslået automatisk, og så brugerskaren kan bruge tid på at stemme tags ind og ud for diverse ressource-typer (og altså bl.a. altefter tidligere tags). ..Jo. Det vil klart være det bedste, og det giver også god mening at få dette delsystem op at køre tidligt i udviklingen. ..Og så snakker vi altså både relevante ressource-rating, samt relevante tags man ønsker fra kommentarerne (og annotationerne, når man når dertil). ..Og så kunne man måske bare indføre profil-tag-pointene lidt i et underliggende lag, hvorved platformen altså gerne må vise kommentarer fra en profil tidligt i pågældende kommentarfelt, også selvom kommentaren ikke har fået så mange vurderingssvar endnu.. ..Og det er faktisk også fedt, at idéen omkring en tag-graf er en ret selvstændig idé, så jeg i øvrigt også får et godt tidligt holdepunkt i idéen.. ..Når nu jeg siger, at brugergrupper vedligeholder tag-graferne, så sker dette i et lag helt over, den del af systemet, hvor brugergrupper kan sættes som subscript på et tag for at vægte ratingresultatet anderledes. Brugergrupperne kan dog sagtens genbruges både til at vægte tag-graf-vedligeholdelsesafstemninger og til at vægte vurderinger, men eksempelvis er brugergrupper helt frie til at fremstemme tags i grafen, som har subscripts fra andre brugergrupper (eller som er fra deres egen eller som er neutrale for den sags skyld); det kan variere fra tag til tag i grafen, og brugergruppen er på ingen måde tvunget til at favorisere sig selv. ..Ah, jeg tror faktisk, denne idé var ret vigtig. 
%Jeg skal lige overveje, hvordan man kommer godt i gang med graferne, uden at skulle i gang med hele brugergruppe-systemet først. Jeg tænker, at det godt kan bæres af individer i starten, for det handler bare om lige at skabe lidt variation.. Hm, men jeg skal altså lige tænke over, hvordan dette skal foregå... Hm, jeg tror sådanset bare, at man skal starte med en version af mine simple 'brugergrupper,' hvor vægt-mønterne så bare kan starte hos nogle få personer.. ..Hm, men bliver det ikke et problem, at det måske bliver svært at finansiere engagementet, det kræver, hvis man skal være aktiv semantisk kommentør.. kommentator..? ..Jo, men man det kan måske også være mere bredt drevet selv i starten.. ...Hm, men platformen kunne jo eventuelt også sætte skub i det ved at brugere kan samle point, og ved at platformen så lader reklame-penge gå til kommentatorerne/annotatorerne.. 
%Hey, jeg har faktisk nogle fine eksempler på anmeldelses-tags, hvor den almindelige brugerskare altså sagtens kan fremføre sådanne. Så jeg kan altså sælge pointen om bare at have tags med semantik-dokumentationer..! Noget andet er, at jeg godt kan bringe ML-udregninger på banen rimeligt tidligt også. Og noget tredje er, at når annotationerne kommer på banen, så kan dette jo blive en vej til triplet-delen af det semantiske web..! Det vil jo være skønt, hvis der kan blive sådan en god vej til de mest gængse sem-web-visioner også.:) Så altså alt i alt ret fedt.:) Nå ja, og lidt appropos at bruge ML, så kan simple reaktions-tags også hurtigt blive mere gavnlige, når man for gang i 'brugergrupperne.' ..Ah, jeg tror altså, det var en rigtig god idé, den med at starte med at fokusere på, at tags'ne skal kunne sættes op i en graf..! Jeg ved i øvrigt ikke, om jeg har nævnt det, men der er så bl.a. meningen, at emne-tags skal kunne arves fra forælder-emner.. 

%(13.09.21) Okay, der er sket flere ting. Jeg blev helt i tvivl om denne folksonomy-idé/-vinkel i går aftes. Og det er jeg stadigvæk, men måske jeg bare skal fokusere lidt mere tilbage på tags-med-ratings, så det ikke bare er enten-eller, men at udsagn kan have en korrekthedsgrad, så det dermed kan blive meget mere nuanceret allerede der.. Jeg fik forresten ikke nævnt, hvad jeg tænkte på i går, nemlig at jeg altså tror på, at stjerne-rating har været en dårlig idé; en dårlig konvention for vores ratings rundt omkring. Jeg tror nemlig, at man folk vil have meget nemmere ved at trykke på et punkt på en linje med en slider end på en stjerne, for så kommer man jo altid ud i at tænke over, "uh, skal jeg nu give fire og ikke tre stjerne," eller "jeg vil gerne give tæt på fem, men jeg vil også gerne påpege små fejl..." Så jeg tror altså faktisk (ikke at jeg på nogen måde er ekspert, så jeg kan jo \emph{sagtens} tage fejl), at vi har skudt os selv lidt i fødderne ved at have fremført stjerne-ratings som en almindelig konvention. Nå, men udover dette, så.. nå ja, så er der også lige det med at prøve at indføre konventioner om at bruge sammenligninger i langt højere grad (og her snakker vi altså når det kommer til specifikke tags/prædikater). Men ellers bliver mit system vel egentligt bare primært et system for at brugerne kan "bestille" annotationer til ressourcer.. Bliver det ikke den primære ting i det..?.. Hm, det vil jeg så tænke over nu, men grunden til, at jeg allerede skriver her igen, er, at jeg så lige kom på en KV-idé, som faktisk \emph{har} en valuta a la BitCoin osv., som folk kan handle med og spare op af (og altså "investere i")..!! Og da det faktisk er en idé omkring en etos-baseret kæde med løfter på (og ja, jeg har haft en tilsvarende idé førhen, men nu tror jeg altså igen, at en sådan type kæde kan lade sig gøre (og kan opnå popularitet)), så spiller idéen endda rigtigt godt sammen med min donations-løfte-kæde..!! Så det er jeg lidt spændt på.:) Så nu er jeg både spændt på en lidt negativ måde, på om der overhovedet er særligt meget kød på min folksonomy-idé, og så er jeg spændt på en positiv måde, på om min nye idé til en løfte/etos-KV holder, som jeg har på fornemmelsen nu. Så ja, lad mig lige tænke lidt over det hele...
%Ah, nu er jeg lidt kommet frem til så småt, at der nok er kød på idéen, men kun i sådan en grad, at jeg kan nævne den som en måske-kunne-dette-give-pote-at-fremføre-idé.. Og så tror jeg lidt bare, at jeg kan fokusere på, "hey, lad os bruge at indføre rating-folksonomies (gerne med sliders), så folk kan begynde at vurdere ting med større nuance." Og så kan jeg også nævne, at tags-baserede / semantiske sammenligninger måske også kunne være en god idé at udbrede ligeledes. Og ellers er resten af idéen så ret meget bare det, at sørge for at brugere kan efterspørge annotationer (og så kan vi faktsik inkludere 'kommentarer' som en slags annotationer) på vise typer objekter (og ja, her kan måske bruge idéen om at opstille det i en graf, som jeg har tænkt mig). Og hvis alt dette giver mening, så kan det altså nok også forbedre systemet yderligere, hvis man indfører mulighed for at oprette 'brugergrupper,' i.e. FOAF-agtige, tillidsfordelings-agtige systemer, som kan bruges til at dele brugerne op og også til at danne en slags moderatorgrupper til at varetage bestemte udgaver af bl.a. tag-graferne.. Ja, og så tænker jeg umiddelbart faktisk stadigt at starte med denne idé, også selvom den lidt er blevet til en måske-idé, for den giver stadig en meget god anledning til at introducere/repetere 'brugergrupper' (eller tillids-/meningsvægt-fordelingsalgoritmer).. Så det er mine umiddelbare tanker, som jeg nu fortsat vil tænke en smule videre over. Angående min blockchain-/kryptovaluta-idé så tror jeg kun endnu mere på idéen i skrivende stund..!:) ..Den har bare alle brikkerne, føles det nærmest som om.. Også inklusiv at den lægger lidt op til min "lykke-valuta" (i idéens seneste udgave), hvorved den dermed også lægger op til en mulig bedre fremtid, hvilket bestemt er en ret god ting (hvis folk ellers altså kan overbevises nogenlunde om, at min donations-løfte-kæde-idé har et godt potentiale). Så ja, sikke rart denne dag udviklede sig, især ift. hvor den ligesom startede. 
%(14.09.21) Ja, jeg tror altså, det holder på denne måde.:) Og jeg har endnu ikke fundet noget i vejen med min nye KV-idé heller. Jeg skal forresten lige huske, at det stadig helt klart er værd at nævne, at man kan gruppere kommentarer, især f.eks. rettelse-kommentarer. 



(14.09.21) \ldots Og nu har jeg lige tænkt lidt mere over tingene, og er bl.a.\ kommet frem til, at jeg skal sælge min folksonomy-idé på en lidt mere low key måde, hvor jeg altså bare kommer med nogle forslag, som \emph{måske} kan forøge brugbarheden og brugeroplevelsen, hvis de implementeres på web 2.0-platforme. Jeg tror dog stadig, at jeg vil forklare om disse idéer for min wiki-idé, for det introducere ligesom det med at rate prædikater om ressourcer i et brugerfællesskab. Og forstå mig ret, det kan sagtens være, at der er et gennemslagspotentiale i disse idéer, men jeg skal bare ikke antage eller konkludere, at der er det, i min tekst.

Jeg er også lige kommet frem til en kryptovaluta-idé, som jeg tror holder, hvor der altså er en valuta på kæden ligesom for BitCoin osv., og hvor folk kan handle med og ``investere'' i disse ligesom for BitCoin m.m. Idéen er sådan set bare en samling af tidligere idéer, men nu tror jeg altså på, at denne samling af idéer holder! Den er dog meget ny, skal det siges, så det kan sagtens nå at gå i vasken endnu. Men jeg har altså en god fornemmelse med idéen nu. Og noget godt ved denne KV-idé er også, at den spiller godt sammen med min donations-løfte-kæde, hvilket kun gør tingene bedre. Jeg tror sådan set bare, jeg vil redegøre for disse KV-/blockchain-idéer her senere i denne ``opsummerings-sektion,'' og så må jeg bare lige indsætte referencer sidst i mine KV-sektioner ovenfor.


%%%Introduktion og eksempler på tags:

Okay, så mine forslag til, hvordan man kan forbedre folksonomy-systemer på internettet, kommer nu faktisk igen til at fokusere meget på at tilknytte ratings til folksonomy-tags i første omgang. Den store pointe er så, at brugere ikke bare kan være interesseret i at se, hvilke udsagn gælder om en ressource, e.g.\ hvilke emner den tilhører, eller advarsler bør rejses for den osv., men vil også kunne være interesserede i at se, i hvor høj \emph{grad} udsagn gælder om ressourcen. Idéen er altså så at tilknytte en stjerne-rating eller en slider-rating (jeg bør i øvrigt lige finde ud af, hvad de konventionelle navne er for sådanne) til alle tags, så brugere kan meddele i hvor høj grad tag-udsagnet passer. Tanken er dermed så, at brugere ikke bare ser den normale rating, der følger med ressourcer, der typisk har det implicitte udsagn, ``undertegnede synes godt om ressourcen,'' men altså ser en hel række ratings, hver med forskellige andre udsagn tilknyttet sig. Et eksempel kunne være NSFW (not safe for work), som typisk forstås som betegnende, at ressourcen indeholder seksuelt eksplicit materiele, enten i form af tekst eller billeder. Men hvor går grænsen for, hvornår noget er ``safe for work'' eller ej? Er en ressource NSFW, hvis den bare indeholder en lille vittighed med seksuelle undertoner, hvis den indeholder en vittighed om noget direkte sex-relateret, men stadig rimeligt uskyldigt, eller skal vi ud i noget mere slibrigt (for at bruge et måske lidt gammelt udtryk) og/eller grafisk? Jeg synes, jeg har set NSFW bruges i alle disse tilfælde (på Reddit og på 9gag). Jeg har endda set det brugt om videoer, der ikke indeholdte andet end velkendte bandeord, som baserer sig på noget seksuelt (så som ``F-ordet''). Men hvis vi virkeligt kan have alle disse tilfælde med, så er det jo lige pludseligt ikke et særligt brugbart tag, for hvis man så gerne vil undgå meget eksplicit indhold, så får man sorteret alt for mange mere uskyldige ting fra også, og man kan heller ikke bruge det til at sortere alt indhold fra, der er den mindste smule NSFW, for i langt størstedelen af tilfældende vil det mere uskyldige indhold slet ikke opnå dette tag. Men hvis vi i stedet har en rating tilknyttet alle sådanne tags, så folk kan se, i hvor høj \emph{grad} andre brugere (gennemsnitligt) har vurderet taggets udsagn til at gælde, så får man med det samme en meget bedre idé om, hvad man kan forvente af indholdet. 

Her kommer lige nogle flere eksempler, og så kan jeg se på, hvilke nogen jeg skal inkludere her, og hvilke nogen jeg bare kan nævne efterfølgende (inkl.\ NSFW-eksemplet). Et andet godt eksempel kunne være ift.\ hvor uhyggelig en film er, hvis vi f.eks.\ snakker en streaming-platform. Jeg skal forresten i det hele taget huske på ikke kun at snakke om web 2.0-platforme, for det kan jo også være streamings-platforme eller produkt-platforme (Steam, Amazon\ldots), eller hvad man kalder det. Nå, men angående uhyggelige film, så kan ``gyser'' jo også dække over mange ting. Men hvis vi i stedet går over til at tilknytte ratings til vores tags, så kan man jo lige netop hurtigt få indblik i (med det samme imens man browser), \emph{hvor} uhyggelig/ubehagelig filmen er. Dette eksempel var også lidt over i kategorien `advarsler.' Jeg skal også komme nogle eksempler, hvor genren kan bestemmes mere nøjagtigt, og også nogle eksempler omkring produktvurdering. I går var jeg inde på Steam og så Age of Empires (AoE) i et feed over spil på store-siden. Og her lagde jeg mærske til, at spillet var kategoriseret, via genre-tags, som ``city builder.'' Jo, det kan i og for sig godt sige; folk der nyder `city builder'-spil kan muligvis også nyde AoE af samme grund. Men et noget vigtige tag for spillet er klart RTS (real time strategy). Så her er et eksempel på, at hvis man bare får vist alle populære tags, og dermed i bedste fald alle ``gyldige'' tags, så giver det ikke nær så godt et billede, som hvis man havde ratings tilknyttet tags'ne. Hvis Steam brugte rating-tags, så ville man hurtigt kunne se, at genreren overvejende var RTS og kun har en lille del af `city builder' over sig. Og hvis man så sammenlignede med et `city builder'-spil, med få RTS-elementer i sig, så ville man kunne se forskel på de to genre-kategoriseringer, i modsætning til hvad man kan nu.\footnote{Måske er rækkefølgen betydende for populariteten, men i så fald er dette jo heller ikke vildt sigende.} Jeg har bemærket, at Netflix er begyndt at benytte tags, der beskriver et prædikat om filmene, såsom `slick' eller `visuelt imponerende,' i stedet for bare tags, der beskriver, hvad vi typisk forbinder med genrer. Her har vi så også et godt eksempel, hvor ratings tilknyttet tags'ne ville være smart. Her kunne det således være smart for brugeren, hvis denne kunne se \emph{hvor} ``visuelt imponerende'' filmen er vurderet til at være (af andre brugere). %..(Jeg vil i øvrigt komme ind på senere, at jeg tror, det kunne være en god idé, at opfordre folk til at sammenligne film på baggrund af sådanne tag-prædikater.)
Angående produkter så vil jeg rigtigt gerne nævne et eksempel om holdbarhed. For dette er jo ofte virkeligt en vigtig ting, når man skal købe et produkt, og det er nemlig også noget, der tit kan skjule sig ret godt, hvis man iagttager folks feedback i form af vurderingssvar og kommentarer. For mange gange vil folk jo anmelde et produkt kort tid efter købet og give deres stjerner da. Men hvis man i stedet havde en specifik tag-rating, der handlede om `holdbarhed' --- eller at man måske havde flere så som `holdbarhed inden for 1 år,' `holdbarhed inden for 3 år,' `holdbarhed inden for\ldots' --- så ville brugeren pludselig meget nemt og hurtigt få adgang til denne information, når denne browser produkterne. %Dette eksempel lægger i øvrigt lidt op til også at komme ind på semantisk gruppering af kommentarer (især når det kommer til sådanne anmeldelser).
Hvis jeg finder på nogle andre gode eksempler, så vil jeg lige skrive dem her. Nå jo, jeg burde også have et eksempel omkring nogle tags, hvor man får noget ud af muligheden for at uddybe tags'ne mere. Hm\ldots\ 
*(Jeg tror endelig, jeg har et godt eksempel, hvor man rigtigt har brug for tag-dokumentation. Det gør man jo især, når man prøver at oprette et tag for et lidt nyt, ukendt koncept. Så eksempler, hvor man har et koncept (mere eller mindre vagt) i tankerne og gerne vil udtrykke det som et tag, er gode, for så kan man bestemt ikke bare klare sig med en kort overskrift. Og jeg vil så også gerne komme med et eksempel, hvor man netop kun har en vag idé selv om, hvad konceptet er, og derfor måske bare beskriver det ud fra eksisterende ressourcer (f.eks.\ film) --- og måske bl.a.\ beskriver konceptet ud fra et ``vibe'' i de ressourcer. Og det gode er så, at sådanne tags nu altid kan forklares bedre i senere tags, og så kan de gamle tags blive ``absorberet'' i disse, ligesom.)




%%%Hvorfor folk vil benytte muligheden nok til at rate tags til at systemet bliver værdifuldt (altså selv for brugere der ikke selv rater..):

Brugbarheden af dette system afhænger så selvfølgelig af, at folk vil være villige til at afgive svar for tag-ratings'ne, når de har set/læst en ressource, men det tror jeg også, at mange vil. Hvis vi nemlig ser på det hav af kommentarer, der tit følger med ressourcer på web 2.0-platforme, så handler mange af den ofte om at give udtryk for menings-feedback omkring selve ressourcen. Så mange folk har helt klart en energi til at ville vurdere det indhold, de ``forbruger,'' hvis vi kan sige det. Og hvis man nu lige pludselig kan give en ret nuanceret vurdering af indholdet på ved nogle klik på tag-ratings, så tror jeg bestemt, at mange folk vil benytte sig af denne mulighed. 
Jeg tror faktisk også, at man ville opdage, hvis man undersøgte det, at mange brugere føler sig begrænsede af den binære rating-knap, der typisk er, når de har mulighed for at rate en ressource. Tit ved man måske ikke, om man skal give tommelfinger op eller ned, fordi der er flere faktorer, der spiller ind i ens vurdering med modsatrettede kræfter. Her tror jeg så, at hvis brugeren har muligheden for at give feedback på hver af de modsatrettede kræfter for sig, så vil det øge lysten til at give vurderingssvar. Jeg tror endda faktisk også det i mange tilfælde vil gøre det nemmere og hurtigere at beslutte sig for et svar med disse muligheder, frem for hvis der kun er én rating tilknyttet ressourcen. 

%Samtidigt må man også konstatere, at folk i høj grad er villige til at give kommentarer til ressourcer på internettet. Antallet af kommentarer kan selvfølgelig variere meget, alt efter hvor populært indholdet (og lad mig forresten bare lade ``indhold'' betegne produkter også, hvis nu ressourcen f.eks.\ er produktoversigt) er, men ikke desto mindre sker det ofte rundt omkring på internettet, og kommentarene kan måles i tusinder. Nogle af disse kommentarer vil godt nok typisk være en del af en tråd, hvor brugere svarer på hinandens kommentarer, men som regel vil en stor del af kommentarerne handle om en reaktion på selve ressourcen. Hver af sådanne kommentarer er jo et udtryk for, at en bruger har haft et ønske om at bidrage noget til samtalen omkring ressourcen. Dette kan enten være i form af kommentarer som kommer med svar, modsvar eller rettelser til dele af indholdet i ressourcen, men i høj grad vil typisk også bare være i form af ... %Hm, dette kommer jo til at lægge gevaldigt op til kommentar-semantik-tags (ja, pointen giver næsten kun mening, når denne del er på plads)...


%...
Jeg kan også lige pointere, at jeg tit synes, man støder på eksempler på internettet, hvor (folksonomy-)tags eller (taxonomy-)kategoriseringer er gjort forkert eller i det mindste gjort ringe. Hm, jeg vil faktisk ikke nævne nogle eksempler, men måske læseren har stødt ind i det samme? Jeg tror på, at præcisionen omkring kategoriseringer vil blive meget større med denne måde at gøre det på. Dermed tror jeg også, at nærmest al kategorisering i fremtiden, hvis dette system udbredes, vil gøres på en brugerdrevet måde. Jeg vil derfor foreslå, at vedkommende, der uploader ressourcen, bare får en større vægt, når det kommer til at kategorisere ressourcen. Jeg vil komme ind på et udvidet system, hvor brugere kan have forskellige vægte i forskellige sammenhænge senere, men det kan sammenlignes med, at vedkommendes stemme får lov at tælle som $x$ antal personer. Man kan endda implementere det sådan, at disse ekstra-stemmer bliver taget væk igen efter noget tid, hvis synes dette giver et mere balanceret slut-resultat. På denne måde kan man altså implementere et system, hvor uploaderen stadig kan sende sin ressource i retning mod den rigtige delmængde af brugere (ift.\ de emner, de er interesserede i), men man har stadig i sidste ende ikke større magt over kategoriseringen end alle andre. Jeg kan nok forklare dette kortere, måske også hvis jeg venter med at nævne det.


%...
%Og nu når vi lige snakker om, hvad brugere kan have af årsager for ikke at afgive vurderingssvar, så tror jeg også rigtigt tit, det afhænge af, at man måske ikke tør at fortælle feed-sorteringsalgoritmen, at man godt kan lide en ressource, hvis platformen har sådan en, såsom f.eks.\ YouTube har. Jeg forestiller mig således, at mange YouTube-brugere som eksempel er bange for at like videoer, fordi de ved, at deres feed så vil blive fyldt op af relaterede ting, hvad brugeren måske ikke er interesseret i. Her tror jeg så også, at et system som det, jeg foreslår her, vil komme til gavn, for hermed kan brugere bedre tilkendegive, lige præcis hvad de godt kan lide ved en ressource. For hvis man f.eks. vurderer om en YouTube-video, hvis vi forestiller os, at YouTube indfører systemet, og tilkendegiver, at man godt kan lide ... 




%%%Brugerflade-kommentarer:

Angående hvilke tags, der så skal vises ved siden af (eller under eller over) en ressource, så kan dette jo bl.a.\ afhænge af, hvilke tags i forvejen har fået flest svar på sig for ressourcen, og også muligvis af, hvilke tags brugeren plejer at have lyst til at give svar til (hvis brugeren altså er logget ind på en konto). Jeg forestiller mig således en lille menu af tags med tilhørende ratings ved siden eller under ressource-boksen, hvor brugeren kan trykke tags og få en drop-down-menu, hvor de bl.a.\ kan afgive svar til tag-rating'en med et enkelt klik. I øvrigt mener jeg også, at der bør være et link til en dokumentation til tagget, for jeg mener nemlig, at det ville være gavnligt, hvis alle tags i et sådant system som dette har en uddybende dokumentation tilknyttet sig, som præciserer semantikken bag det udsagn, de repræsenterer. Og angående hele tag-oversigten, så forestiller jeg mig, at der også har skal være en mulighed for at få en drop-down-menu og/eller at udvide tag-menuen, så brugeren dermed bare kan få vist de mest relevante (for ressourcen og for brugeren selv) først, og så ellers efterfølgende kan udvide menuen for at få vist flere mulige tags. 



Okay, så nu bør jeg have forklaret det grundlæggende nogenlunde, på nær at jeg også lige har en kommentar omkring stjerne-ratings. Den kommer her. Jeg er tilbøjelig til at tro, at vi har skudt os selv lidt i fødderne ved at have udbredt 5-stjerne-ratings så meget som vi har. En disclaimer er selvfølgelig, at jeg slet ikke er ekspert på området, og at jeg heller ikke har skyggen af empiri til at understøtte denne påstand. Men jeg kan nu ikke lade være med at tro, at vi har ramt lidt ved siden af her. Fem (eller seks) stjerner er simpelthen alt for få. Tanken kan selvfølgelig være, at det er nemmere at sammenligne ting, hvis ratingen er så diskret, men det passer ikke rigtigt; folk vil generelt gerne så kunne se et procenttal eller et kommatal ved siden af stjernerne\ldots\ Og hvis tanken er, at så er det nemmere for brugeren at beslutte sig for, hvad der skal gives, så tror jeg nærmere effekten er modsat. Det er så tit, at man kommer i tvivl om, ``skal den nu have tre, eller skal den have fire stjerner,'' eller ``har den virkeligt også fortjent at få alle fem stjerner, når nu der er de her fejl.'' Og dette mener jeg faktisk (igen uden dog at have noget empiri for det), kan være et anseeligt problem, der kan afholde brugere fra at vurdere ting, når de har set/læst/købt det; fordi denne tvivl, der kan opstå, kan trække tiden lidt ud og dermed efterlade brugeren med en ikke-så-god brugeroplevelse af at vurdere tingen. Og denne faktor i brugeroplevelsen kan altså i så fald dæmpe lysten til at give vurderinger i fremtiden. Jeg vil i stedet foreslå, at vi går over til mere at bruge sliders, når folk skal vurdere ting. For hvis man gerne vil beholde stjerner, fordi man så kan annotere ratingen med lidt semantik omkring, hvad de forskellige svargrader bør repræsentere, jamen så kan man jo altid bare vise stjernerne over slideren og så fylde dem op med farve --- enten med diskrete skridt i opfyldningen eller med en mere kontinuer farveopfyldning --- når brugeren justere slideren. Og så skal det bare være helt klart for brugeren, at præcisionen stadig ikke er vigtig, også selvom svaret nu medfører en mere præcis værdi i databasen. Hvis man gerne vil sørge for virkeligt at få denne pointe igennem og signalere til brugeren, at denne ikke behøver at bekymre sig om pixel-præcision, når vedkommende justerer slideren, så kan man bare lave en lidt bred slider med en lidt diffus midte. Hm, og når nu vi også har forslaget på bordet om at beholde stjernerne, så kunne man jo eventuelt lægge dem ovenpå slideren og gøre selve slideren til et lysskær med halvdiffuse kanter, således at man får de overliggende stjerner til at ``lyse op,'' når man så at sige flytter lyset neden under dem. Lyset skal dog ikke være totalt diffust i så fald, for brugeren skal stadig få et klart indtryk af, at lysskæret er en slider, som man kan tage og rykke i. Desuden må der meget gerne stadig være et procenttal (eller et kommatal, hvis man altid bruger et fast antal stjerner) for enden af slideren, og det må i øvrigt gerne være muligt for brugeren at trykke på dette tal og justere dette via keyboard-input (hvorved slideren så rykkes af sig selv, når et tal inputtes). (Og man skal selvfølgelig også kunne klikke på et punkt på slideren, hvorved slideren rykker derhen, i stedet for altid at skulle trække i slideren, det er klart.) Så ja, det var altså bare lige en lille kommentar til, hvad der måske kunne få oplevelsen af at give vurderingssvar til at føles nemmere og hurtigere. Jeg ved ikke selv, om det ville være bedst at have stjernerne over slideren og fylde dem op med farve alt efter sliderens placering (enten kontinuert eller diskret (i.e.\ i trin)), eller om man skulle lægge dem ovenover slideren som foreslået, så det må man jo bare prøve ad at se på. \ldots Man kan dog sige, at hvis man vælger første mulighed er man mere sikker, for så kan brugere, der synes godt om at benytte stjerne-ratings (og dem skal der jo nok findes), bare klikke på selve stjernerne i stedet. *(Jeg har forresten besluttet, og jeg ikke skal skrive det på denne måde, for jeg \emph{kan} jo ikke vide, om sliders vil være bedre end (selv kun fem) stjerner, så det er nok langt bedre, hvis jeg bare nævner og forklarer det som min egen (forventede) præference.)

Jeg kom i øvrigt også lige på en anden ting: Mange steder vises tags jo som en slags piller med skrift indeni. En måde man så hurtigt kunne vise ratingen for tags'ne, kunne jo så være at fylde hver tag-pille delvist op (fra venstre til højre) med farve (som så er en mørkere og tydeligere nuance end for den resterende del af pillen i højre side) svarende til procenttallet fra rating-resultatet. Og man kunne så endvidere lade farven ændre sig alt efter, hvor stor procenttallet er, hvilket nok især er en god idé, da pillerne jo alt andet end lige kan have forskellige længder, og dette vil således hjælpe med at sammenligne procenttallene på tværs af piller (uden at klikke på dem og se en mere uddybende version af ratingen) på trods af dette forhold. Dette gør dog godt nok, at man så ikke kan bruge farver til at gøre de specifikke tags mere genkendelige, men her kunne man jo med fordel så sørge for, at der altid er et lille ikon følgende med til tagget ude i venstre side af det. 



En anden lille ting, der handler om brugerfladen, som jeg vil nævne, er, at de foreslåede tags kunne afhænge af, hvilke andre tags er populære. Jeg nævnte jo, at de viste tags ikke bare kunne være sorterede ud fra den opmærksomhed, tagget har fået fra andre brugere, i tag-menuen, men også kunne afhænge af, hvilke tags brugeren selv plejer at være interesseret i. Hm, jeg burde nok bare nævne dette i forlængelse af det andet, men jeg vil altså også nævne (og nu bliver det bare her), at platformen også kunne give inkludere tags i menuen, hvis interessen omkring et tag, som allerede har fået opmærksomhed, ofte er korreleret med interesse i et andet tag. Hvis f.eks.\ ressourcen har fået et tag, der siger ``omhandler et produkt,'' så kunne platformen måske foreslå et tag såsom ``brugertilfredshed af nævnte produkter i ressourcen'' eller ``indeholder betalt promovering,'' eller hvad vi ellers kunne finde på. Eller hvis ressourcen er blevet vurderet til at være inden for genren, `humor,' så kunne platformen måske automatisk foreslå tags omhandlende typer af humor og/eller tags omkring graden af, hvor sjovt indholdet er ifølge de brugerne, der har afgivet stemme. 





%%%Flere eksempler (bl.a. eksempler der benytter specielle ting omkring brugerfladen):

Jeg vil også gerne pointere, at jeg tror, systemet vil åbne op for, at brugerne vil finde endnu flere nyttige tags a la `slick' og `visuelt imponerende.' %... %Eksempler, der bl.a. viser, at det kan være smart med uddybende tag-dokumentation.
\ldots *[Det var meningen, jeg ville komme med nogle flere eksempler her. Hvis det er nødvendigt med flere, så tror jeg hellere bare, jeg vil overveje det nærmere, når jeg nu her skal skrive en renere version af denne tekst, men jeg kan da lige nævne noget, jeg har tænkt over: Jeg har tænkt over et (eller flere) eksempel/ler, hvor man f.eks.\ siger, ``denne film har et Wes Anderson-vibe over sig,'' eller ``denne serie har samme vibes som den og den serie i sig.'' På den måde kan der være nogle vage koncepter, som brugerne bestemt er interesseret i at have tags for, men som de måske ikke kan udtrykke præcist --- og som bestemt ikke har et velkendt term for sig (endnu, om ikke andet).]


%Der er bestemt andre gode eksempler end bare NSFW-vurderinger på, hvor ratings tilknyttet tags kan være brugbart. \ldots Nu har jeg nævnt nogle flere eksempler i det ovenstående, men man kunne nemlig også nævne et begrænset antal deroppe, og så nævne nogle flere her. %Hvis vi endda bare bliver ved emnet omkring advarsels-tags, så kan man også nævne ...





%%%Udvidelser:

Cool. Jeg håber, at jeg ved dette punkt i teksten har nået at sælge idéen godt omkring, hvorfor at rating-folksonomies vil være gode. %Jeg mangler dog lige nogle ting i det ovenstående i skrivende stund, nemlig at give eksempler, hvor en uddybende tekst er smart, samt at forklare om, hvorfor man bedre kan styre sin feed-algoritme med sådan et system som dette..
I det følgende vil jeg så beskrive nogle eventuelle udvidelser til denne idé.



%Jeg har også en idé til noget andet, man kunne gøre, som måske kunne gavne anmeldelser, og det er, hvis man giver mulighed for, at brugere kan lave sammenligninger af ressourcer på bagrund af specifikke tags, og opfordrer dem til at gøre dette. Så hvis vi f.eks.\ tager et eksempel, hvor ... %Lad mig lige finde et godt eksempel først..
%Hm, måske skal dette stå efter annotationer, for skal man mon ikke bruge dem, hvis man skal implementere det, helst?..
%(18.09.21) Nu, hvor jeg lige har fundet ud af, hvordan jeg vil have "specifikke tags," sorterings-/filter-menu, prioritets-slider og kurve-indstillinger (og tag-browsing-(semi-di-)graf), og altså er nået så langt, så bør jeg lige genoverveje dette emne; hvordan skal man så lægge op til, at brugere kan komme til at lave semantiske sammenligninger? ..Hm, det må vel næsten så være noget med at gå ind på det relevante tag og/eller en af de relevante ressourcer og så.. Hm.. Tjo, kunne det ikke være at navigere til det relevante tag og til en indstilling, hvor ét eller begge af de relevante ressourcer er valgt, og så.. Hm.. ..Ah, det handler vel nærmest om at lave betingede/relative ratings.. Ja, og hvordan skal man så gøre dette.. Jo, det er vel netop at gå ind over en oversigt under tagget, hvor menu-felter med ressourcer vises for hvert udsnit af rating scoren, hvor ressourcerne så altså er eksempler på, hvad der ligger i de forskellige score-intervaller. Og når brugeren så vælger specifikke ressourcer, hvilket også gerne må kunne vælges via URL'en, så man kan navigere direkte til denne indstilling fra andre lokationer af, så skal disse ressourcer altså vises blandt eksemplerne, og man skal kunne se dem og deres score klart (muligvis ved også at kopiere dem ud i en særlig boks og vise dem i stort format).. Og her skal man så altså (muligvis; hvis man vil have denne feature) kunne lave relative ratings, hvor den effektive rating fra brugeren altså praktisk set kommer til at afhænge af, hvordan, og særligt hvor tæt, de to scorer ligger i forvejen. ..Hm ja, det er da en idé, men måske er det dog lidt meget for så lille en ting.. Hm, jeg ville jo egentligt oprindeligt også gerne have denne mulighed bl.a. så brugerne hermed også især kan anbefale lignende ting, ved at sige, at de har ca. den samme grad af denne semantik.. Tja, måske skulle man bare foreslå idéen om at få en oversigt over taggets ressourcer-eksempler, og hvor man rigtigt nok kan søge på og udvælge specifikke ressourcer og sammenligne dem, men om man ligefrem skal nævne det her med relative/betingede vurderinger..? ... Okay, jeg tror faktisk, at jeg vil nævne/foreslå relative point, men ellers er det store take away faktisk, at der skal være en menu, hvor man ikke bare har sorteret alle relevante ressourcer efter rækkefølge i form af et feed, men hvor man i stedet ser en hel akse, og hvor der så er illustrerende eksempler for hvert interval på denne akse. Og denne oversigt kan så også godt være multidimensionel. Jeg har nu nogle foreslag til, hvordan dette måske kan lade sig gode. Men nu er jeg så til gengæld nået til et nyt spørgsmål om, hvordan man egentligt helt præcis skal fortolke de forskellige rating-værdier i første omgang. For alt andet end lige så vil alle scorer jo nok være relativt definerede til hinanden i en eller anden forstand.. Men hvilke retningslinjer skal man så bruge..? Hm, måske kunne man jo udvægle repræsentative ressourcer til at definere semantikken bag forskellige punkter på ratingen, men så skal man jo til at stemme om disse og sådan.. Hm, hvad er der af muligheder?...
%Tja, det må jo næsten være en del af semantikken bag tagget (og altså dets dokumentation). Nu har jeg så tænkt lidt, og måske man kunne bruge en konvention om ikke at bruge specifikke ressourcer som sammenligningspunkter i tag-dokumentationen, men så i stedet have en særlig afstemning, hvor brugere kan indstemme gode og illustrative eksempler for ratingen. Disse eksempler vil så ikke være fastlagte på aksen, men vil bare komme op som eksempler, når brugeren iagttager tag-dokumentationen --- og måske også når brugeren skal afgive en rating.. Hm, man kunne jo eventuelt vise eksemplerne øverst, når dokumentationen vises, og så give brugeren mulighed for at afgive sit vurderingssvar på selvsamme akse, som disse eksempler vises.. ..Ah ja, det må da næsten være sådan, man skal sige det.?:) ..Og nu vil jeg så lade disse overvejelser og idéer komme efter, at jeg har redegjort for feed-filtre m.m., for det bliver nok nemmest sådan.



%%%%%Kommentar-ratings og -grupperinger..:

Den næste mulige udvidelse, jeg vil foreslå til systemet, er, at man kunne tilføje tag-ratings til kommentarer. På de platforme, vi snakker om, vil brugere jo ofte have mulighed for at give kommentarer til ressourcen. Brugere vil også her typisk have mulighed for at afgive en vurdering, om de kan lide kommentaren eller ej. Men hvad hvis man er uenig med en kommentar, men synes det er en vigtig snak at bringe på banen, som kommentaren lægger op til, og derfor ønsker at flere vil se og forholde sig til kommentaren? Skal man så sige ``kan lide'' eller ``kan ikke lide,'' eller hvad? På Reddit er det endda sådan (så vidt jeg lige ved), at hvis man nedvurderer en kommentar, så ryger den i sidste ende længere ned i listen, hvilket netop gør, at færre folk vil forholde sig til den. Så her har vi altså et eksempel, hvor det muligvis også vil være gavnligt, hvis man også kan give kommentarer flere og mere nuancerede tags. 

En for mig endnu vigtigere ting, når vi snakker om kommentar-tags, er dog, at åbne op for muligheden for at gruppere kommentarer efter indholdssemantik. Lad os som eksempel se på en hypotetisk YouTube-video med nogen, der forklarer om et eller andet emne. Og lad os sige, at vedkommende kommer til at sige noget, der er forkert i løbet af videoen. På trods af hvad videoen ellers indeholder af interessante ting man kan reagere og kommentere på, vil kommentar-sektionen nu typisk blive fuld af kommentarer, der alle pointere denne ene fejl. Dette fænomen er ikke specielt gavnligt for læseren, for medmindre der kan herske tvivl om, hvad der er korrekt eller ej, så vil én kommentar være nok. Man vil som læser ikke være interesseret i at læse den samme rettelse igen og igen. %Men lad os så forestille os,
Men tænk, hvis der var en måde at gruppere alle disse kommentarer, så man som læser bare kunne læse én og så sortere de andre fra kommentar-feedet. Mon ikke dette kunne forbedre brugeroplevelsen? Vi kunne også finde på andre eksempler, som lægger sig op ad denne, såsom folk der klager over lyden eller videokvaliteten, eller andre klager for den sags skyld. Jo, det ville jo nok være meget dejligt, hvis man kunne gruppere sådanne kommentarer på en eller anden måde og sortere dem fra i resten af feedet (især hvis de ikke indeholder andet semantisk indhold), men hvad der kunne gøre det hele endnu bedre, var hvis man også kunne oprette det pågældende omdiskuterede udsagn (e.g.\ ``er påstand $x$ fra videoen nu også sand'' eller ``var lyden for høj på det og det tidspunkt'') som et tag til ressourcen (i.e.\ til videoen, hvis vi snakker YouTube-eksemplerne), hvorved folk så kan stemme om korrektheden af dette udsagn. 

Nå, og så kan vi jo spørge: Kan det lade sig gøre, at implementere et system, som giver mulighed dette? Ja, det tror jeg, og jeg tror at en måde at opnå dette på, er ved at indføre højere-ordens-tags til systemet, hvilket vil sige at indføre muligheden for at oprette tags, der kan tage parameterinput, og hvor dette input %bl.a.\ 
så kan være andre tags. På denne måde kan man så bl.a.\ oprette et højere-ordens-tag, der siger ``$\lambda x.$denne kommentar handler primært kun om korrektheden af $x$,'' og så kan man i vores tilfælde f.eks.\ indsætte den omtalte påstand fra videoen eller en påstand om at ``lyden var for høj på det og det tidspunkt.'' 
Idéen er så, at brugerfladen skal understøtte sådanne højere-ordens-tags, ved at der i første omgang er en særlig del af tag-menuen, som er beregnet til tags med definerende udsagn bag, der retter sig specifikt mod selve ressourcen, f.eks.\ ved at omhandle specifikke dele af indholdet. Disse tags kan så være fuldstændigt nye tags, oprettet specifikt som feedback til ressourcen, eller de kan være sammensat af højere-ordens-tags med input i form af sådanne specifikke tags. %(Jeg gider ikke lige at formulere dette bedre (og sådan går det tit; det er ikke meningen, at denne tekst skal være velskrevet).)
Og ydermere skal de foreslåede tags nede i kommentarerne så gerne kunne afhænge af, hvad de mest populære specifikke tags er, og rigtigt gerne på en måde, så brugeren er få klik fra at kunne oprette sammensatte tags (i.e.\ højere-ordens-tag med input-tags sat ind) med input taget fra de mest populære `specifikke tags,' lad os kalde dem det. Så hvis man f.eks.\ gerne vil igangsætte en proces, der gruppere visse rettelseskommentarer under et tag, så rettelsesudsagnet dermed kan vurderes, og så de gentagne kommentarer kan sorteres fra i kommentar-feedet, så starter man altså bare med at oprette et specifikt tag omkring rettelsesudsagnet. I takt med at dette tag så opvurderes, vil det så blive muligt hurtigt at klikke sig frem til at rate'e ``denne kommentar handler primært kun om korrektheden af $x$,'' hvor $x$ nu har fået rettelsesudsagnet indsat på sit sted. Og i takt med at flere og flere kommentarer for aktivitet omkring dette sammensatte tag, kan platformens algoritme så også bare gøre det til en af de første tag-forslag, når brugerne vil vurdere kommentarerne. I øvrigt er det så også kun naturligt, at det pågældende specifikke tag også får endnu højere prioritet i tag-(del)menuen, jo flere kommentarer der har vurderingsaktivitet omkring tagget. På denne måde kan man altså få et system, hvor kritisk og konstruktivt feedback fra kommentarsektionen kan løftes op og blive til en rating i tag-menuen. 

Men hvad så med ønsket om at frasortere gentagne kommentarer fra kommentar-feedet? Hvordan i al verden får man implementeret det? Jo, en hurtig mulighed, hvis vi kun ser på vores specifikke eksempel kunne jo være, hvis platformen fremhævede  bl.a.\ ``$\lambda x.$denne kommentar handler primært om korrektheden af $x$,''-højere-ordens-tagget, samt f.eks.\ et tilsvarende tag, der sagde ``$\lambda x.$denne kommentar handler primært om emnet $x$.'' Platformen kunne så sørge for at fremhæve de mest populære $x$'er (altså tag-input) i disse specielle højere-ordens-tags over kommentarfeltet og muligvis tilføje en slider til dem, som brugeren kan stille på, hvorved frekvensen af kommentarer indenfor pågældende emnet / omkring pågældende udsagn justeres i kommentar-feedet. Dette kunne være én mulighed. En anden mulighed kunne være at lave ét særligt højere-ordens-tag, som simpelthen siger ``please group under tag $x$,'' hvorved det så er underforstået, at brugeren beder platformen om, for det første at oprette $x$, som et feed-sorterings-tag, og for det andet så om at lade kommentaren blive associeret med dette tag (hvorved brugerne altså igen får omtalte slider at stille på osv.). En tredje mulighed er at lade brugerne af platformen stemme om helt generelt og overordnet, hvilke nogle højere-ordens-tags skal fremhæves og få denne funktionalitet omkring sig, så det altså kan bruges til at skabe feed-sorterings-tags. Og denne beslutningsproces kan selvfølgelig også godt være delvist kontrolleret / modereret af platformen selv. %...%Er der flere ting, man kunne gøre, eller flere ting, man kunne ønske sig a la dette? ... Ah vent, nu ved jeg det. Man skal selvfølgelig bare have et ekstra menu-punkt --- som muligvis kan være nede over kommentarfeltet --- med alle kommentar-tags'ne, og så er pointen, at man i denne (del)menu for de prioritets sliders, jeg har nævnt her (helt helt i bund betyder, "frafiltrer alt med positiv rating for dette tag" --- måske dog med en lille positiv margin på nogle stemmer), og i øvrigt også, at man så kan erklære, at man ønsker flere kommentarer der passer på pågældende tag (hvormed man så bl.a. kommer til at kunne ønske svar, kilder, links osv.). ...Hm, og jeg kan jo lige se på her, om der ellers er andet, man skal kunne signalere omkring kommentar-tags.. Hm, en indskudt bemærkning er, at jeg lige kan nævne, at disse filter-tags self. også gerne må kunne tilpasses den individuelle bruger (når denne er logget ind).. ..Hm, hvis man så vil over i et system, der lægger op til endnu mere struktureret argumentation, så handler det jo så bare om, at betragte kommentarer som selvstændige ressourcer også (hvis man ikke allerede gør dette).. Hm, ja, men ellers kunne man måske alternativt også udvide systemet, så at.. ja, eller man skal vel egentligt bare have højere-ordens-tag a la "argument for udsagn x" og "modargument til udsagn x", som så også kan fremstemmes blandt kommentar-tags'ne, og hvor man så altså også kan efterspørge det. "Kilde til udsagn x" kan også være et andet godt eksempel.. Hm, men så vil det bare være godt, hvis man kan få lidt en overgang til et argument-system.. Hm.. ja, det ville faktisk være meget godt (for diskussionerne kan jo stadig godt være meget ressource-specifikke umiddelbart..).. ..Hm, og måske man i det hele taget skal tænke lidt over semantiske klassifikationer af argumenter.. ..Hm, men kan det ikke bare komme med et mere og mere udarbejdet tag-system, og så kan gamle kommentarer jo altid bare opdateres med nye, mere semantisk præcise tags. Og så handler det jo lidt bare om at kunne referere til kommentarer på tværs af ressourcer (og altså på tværs af forskellige ressourcers kommentarfelter), ikke..? Hm.. ..Er der mon så en måde at hive dataen fra kommentar-ønske-sliders'ne med sig..? ..Hm, men vent, skal man ikke bare sørge for, at specifikke tags kan hives ud som sin egen ressource, hvor man så også kan koble flere (tilsvarende) tags på fra andre ressourcer..?!.. Hm, og hvordan kan det lade sig gøre..? ..Tjo, men man skal da bare have.. Ah, vent kunne man gøre det via dokumentationerne, så tags kan have flere.. hm, eller måske snarer, at tag-dokumentationer kan få tilknyttet relationer i mellem sig, der siger, at dokumentationerne har samme semantik, men hvor vi altså også taler om en rating, så dokumentationer, der ikke helt er semantisk ens, så bare ikke vil få fuld rating-score for den pågældende relation. Ja, og så kunne man også rate selve dokumentationer efter popularitet, således at mere koncise og velformulerede dokumentationer kan få flere point end andre. På denne måde kan man så denne nærmest en di-graf, hvor hver dokumentation så kan få en sti til en mere koncis og/eller velformuleret udgave, hvis sådan en eksisterer. Og når man så browser en dokumentation.. eller hvorfor ikke bare sige "tag," og altså inkludere "titlen" på dokumentationen som en del af helheden, når man f.eks. sammenligner dokumentationer (og nu også tag-titler).? Jo. Og ja, når man så browser tags, så bør man kunne få et feed både af ressourcer/kommentarer, som er rate'et til at passe på tagget, men også ressourcer/kommentarer fra søskende-tags, der er vurderet til at have semantik meget lig tagget selv. Nice nok, altså.! :) 
\ldots

Ah nej, nu ved jeg det. Man skal bare have en ny tag-menu% --- muligvis under de to andre tag-delmenuer eller muligvis lige over kommentarfeltet --- 
, som meget vel kan placeres lige over kommentarfeltet, som indeholder `kommentar-tags,' og hvor de normale ratings er erstattet med sliders, der kan bruges til at justere sorteringen af kommentarerne. \ldots Hm ja, og ift.\ til det, jeg lige skrev, er pointen jo egentligt bare, at der ikke er nogen grund til at prøve at holde nogen ellers populære tags fra kommentarerne ude fra sorterings-menuen (kan vi også kalde den); alle tags, der kan være relevante for kommentarerne, bør også kunne være relevante for sorteringen. Man kunne så have det sådan, at brugerne først skal klikke på tagget i sorterings-/filter-menuen, før det kan tages i brug, hvorved det bliver pinned i toppen/starten af menuen, og hvorefter man så kan justere filter-/sorterings-slideren. Herved kan platformen nemlig så opsamle data omkring, hvilke former for tags er populære at sortere ud fra og kan dermed begynde at foreslå dem oftere for brugerne. Og igen vil det selvfølgelig være smart, hvis platformen også kan huske brugerens valg (hvis denne ønsker dette) og tilpasse sorteringsmenuen, så den passer bedst muligt på brugerens vaner --- og gerne også på en måde, at visse filter-indstillinger beholdes, når brugeren browser andre ressourcer. Dette kunne jo f.eks.\ være filtrering baseret på NSFW-tags og/eller andre ``advarsels-tags.'' Ja, måske skulle man bare gøre det, så at brugeren kan selektere en valgmulighed for ethvert tag, som gør at tagget beholder sin slider-placering i sorterings-/filter-menuen, når brugeren browser andre ressourcer på tværs af platformen.

Og når det så kommer til højere-ordens-tags med input i form af de ``specifikke tags,'' så kan platformen jo bare registrere, hvor ofte en frekvent brug af et givent højere-ordens-tag i kommentarerne (altså hvis tagget får meget opmærksomhed i kommentarfeltet) fører til, at brugere vælger at sortere kommentarerne ud fra dette tag, og hvis dette sker ofte, så kan de pågældende tags foreslås tidligt i sorterings-tag-menuen. Så hvis brugerbasen f.eks.\ indfører et tag, der siger, ``please group under tag $x$,'' og det så registreres at sammensatte tags på denne form ofte bliver brugt i brugeres sorteringer, hvis det får opmærksomhed i kommentarfeltet, så kan platformen altså måske blive hurtig til at foreslå sammensatte tags af denne form oppe i sorterings-menuen. \ldots Tja, eller også kan man selvfølgelig også bare foreslå ud fra, hvad andre brugere har valgt tidligere for samme ressource, men det kan man jo bare overveje og se, hvad der giver mening. 

Bemærk, at tags tilknyttet kommentarer i dette system aldrig bare er enten sat eller ikke sat for en kommentar, men altid vil have en rating-værdi tilknyttet sig. Så hvis man f.eks. gerne vil sortere NSFW-materiale lidt groft fra, så kan man jo sætte slideren for NSFW-tagget i den lave ende. Herved bør det så gælde, at alle kommentarer med en vis score for NSFW-tagget bliver sorteret fra; og hvis slideren sættes helt i bund bør alle kommentarer med en positiv rating --- eller måske med en rating højere end en lille positiv margin --- sorteres fra. Hm, man kunne måske også i visse tilfælde have lyst til at justere frekvensen af tags med det pågældende tag\ldots\ Ja, og dette kan så i udgangspunktet kobles sammen med samme slider, men i princippet bør man jo faktisk have to sliders; én for at filtrere kommentarer fra ud fra en vis tærskel og én til at op- eller nedjustere frekvensen af kommentarer med det pågældende tag. \ldots Hm, eller også kunne man altid vise to sliders og så dele det op, så at første slider bestemmer tærsklen for, hvornår en%...%Hm, nej det bliver vist lidt mærkeligt.. ..Hm, i virkeligheden bør man jo næsten kunne justere en hel del sliders, hvis man virkeligt vil styre feed-algoritmen selv.. Hm.. ..Okay, det giver nok god mening først at have en maks-tærskel.. Og derefter en frekvens-parameter.. ..Ja, hvis udgangspunkt er i midten af slider-baren.. ..(Men som muligvis kan følge med nedad, hvis man justerer maksværdien ned).. Ja, og maksimum-slideren bør starte i top som udgangspunkt.. Og hvordan kan man så ellers inddele frekvensen..? ..Det kan man jo så med en skew-parameter, og er det så ikke bare det? En maksimumgrænse, en frekvens og en skew-p.. hm, men hvad gør sidstnævnte egentligt.. hm, den er vel til for, hvis man prioriterer ressourcerne/kommentarerne ud fra andre tags/rating (plus eventuelt andre ting) også..(?).. ..Ah ja, så det er faktisk en prioritets-parameter i virkeligheden..! Så denne parameter fortæller altså, hvor vigtig dette tag bør være ift.\ hvad der vises øverst i feedet (og altså på hele rækkefølgen af det). ..Ah og maks-slideren skal så faktisk have to sliders, så der også bliver en minimumsslider, og så man altså kan præcisere et interval.. Hm, enten det, eller også skal man bare kunne toggle imellem en min- og en maks-slider --- det kan godt være, at det giver mere mening.. Tja, eller der kan være (sjældne) forhold, hvor man godt kunne tænke sig begrænsninger i begge ender, men hvorfor så ikke bare tilføje dette som en tredje toggle-mulighed? Ok. Det virker altså umiddelbart ret nice, det her.. ..Hm, men hvordan kommer frekvens-parameteren så til at fungere..? ..Uh, forresten skal prioritets-slideren også kunne være negativ.. ..Hm, men kan man ikke klare sig bare med "prioritet"..? ..Ah, vent kunne man ikke have prioritet-parameter, og så eventuelt en slags maks-frekvens-parameter..? ..Så man ligesom eksempelvis kan sige, "vis mig kun (omkring) én af denne rettelse" eller "vis mig ikke mere en et par NSFW/ubehagelige ting ad gangen, når jeg scroller ned i mit feed".. ..Hm, ja så en parameter til ligesom at erklære, at der helst gerne må være et vist mellemrum mellem ressourcer/kommentarer af pågældende type (hvilket jo i sig selv aldrig er helt præcist, så man kan heller aldrig forvente hel præcision med denne indstilling), og så skal man måske bare gøre det så, brugeren kan vælge, om første ressource af typen gerne må komme i starten af feedet, eller om denne gerne først skal komme efter at interval-længden er gået.. Hm, eller måske bliver det faktisk lidt mærkeligt.. ..Ja, måske den ting mere skal være en sag for prioritets-parameteren.. Ja, så denne trejde parameter handler bare om at sige, at hvis en ressource af pågældende type bliver vist i feedet, så må prioritets-parameteren for denne type gerne falde effektivt set i det efterfølgende feed --- og ja, den kan nok godt bare falde hver gang, så frekvensen bare ligesom bliver eksponentielt aftagende i princippet, det virker fint. ..Ja, det lyder da meget godt altsammen. Og så kan den tredje parameter bare være en lidt advanceret parameter (hvor brugeren lige skal trykke en ekstra gang og/eller gå ind et specielt sted, for at få den vist), og i øvrigt kan det så også være en parameter, man godt bare kan vente med at implementere indtil det lige passer sig. ..Hm, men vent, vil der ikke være en del situationer, f.eks.\ når det kommer til prædikater omkring film/spil/produkter, hvor man gerne vil have vist ting inden for et specifikt interval, men altså hvor det dog godt må overskride intervallet? Hm jo.. I princippet kunne man så bare tilføje en parameter, der visker grænserne for intervalbegrænsningerne lidt ud, men der må være mere intuitive muligheder.. Hm, kunne man gøre noget med, at.. Ah, det er så her, der kan nå op at blive en del parametre, for i princippet kan man sige, at dette handler om at skabe en ny parameter på baggrund af en gammel, og det kan jo så netop gøres ud fra gennemsnit, kurvebredde, skewness, cut-offs og sådanne parametre.. Hm.. ..Tja, men er det egentligt ikke bare det: Skal der ikke bare være en (pop-ud-, muligvis) menu, hvor man kan justere en kurve ud fra sådanne parametre, og så skal man bare have en prioritets-slider, som også kan gå i negativ.. Og er det så bare det..? ..Og brugeren skal så bare have mulighed for at gemme sine egne indstillinger, både for hvert enkle tag, men også for samlinger af tags, så brugeren nemt kan skifte mellem forskellige feed-præferencer. Nice. Gode tilføjelser til idéen.. ..Ah, og kurven skal kunne gå negativt også, så man kan lave cut-off på feedet, samtidigt med at man måske faktisk efterspørge tagget for et andet interval af ratingen.! 
\ldots\ 

Okay, nu har jeg tænkt lidt mere, og kommet frem til en tilsyneladende meget god løsning. I første omgang skal brugeren kunne indlægge en funktion over rating-aksen. Funktionsværdien bliver så den værdi, som feed-sorteringen sker ud fra. Derudover skal der så være en ``prioritets-parameter,'' som jeg indtil videre kalder den, der afgør i hvor høj grad ressourcer skal op- eller nedprioriteres ud fra førnævnte funktionsværdi. Hvis funktionsværdien er helt i bund (hvilket enten kan være ved nul eller ved en nedre negativ grænse, alt efter hvad der virker mest intuitivt), så filtreres ressourcer med scorer inden for pågældende interval selvfølgelig bare helt fra. Og angående hvad man gør, hvis der er meget få, der har vurderet en ressource, så må man næsten bare antage, at der vil være brugere, der ikke har lavet\ldots\ Uh nej, vent! Måske kunne man bare give mulighed for, at brugeren sammen med sin funktion også kan sætte et start-offset, således at alle ressourcer bliver bedømt, som om de havde $n$ antal stemmer i en vis retning. Dette kan så bruges enten til at give ekstra snor til ressourcer (/kommentarer), der ikke har fået så mange vurderingssvar, eller til det modsatte, således at man måske vælger, at man ikke vil se ressourcer/kommentarer, før de har fået en vis mængde positiv opmærksomhed. Yes, det virker som en god og simpel løsning! Cool. Det næste jeg vil nævne er så, at man muligvis (på et tidspunkt) kan indføre en tredje parameter også, som basalt set lader den effektive ``prioritetsscore'' for pågældende tag falde for alle objekter for hver gang en ressource/kommentar vises i brugerens feed. Og her skal det i øvrigt bemærkes, at brugere nemlig godt kan lave samlede sorterings-filtre/-algoritmer ved at sortere ud fra flere tags på en gang. \ldots Tja, eller det gav vel egentligt lidt sig selv, men fint nok. Ja, og selvom der kun benyttes ét tag i filteret, så kan denne tredje filtermulighed stadig være relevant, så never mind. Men pointen er så, at man f.eks.\ hvad angår de førnævnte rettelses-kommentarer jo godt kan have lyst til kun at se én udgave indenfor et vist emne (hvorved aftagnings-parameteren kan sættes højt) og så ikke se andre derfra. Eller måske vil man bare sørge for at der altid kun er en begrænset mænge ressourcer med visse tags tilknyttet sig i ens feed. Dette kunne f.eks.\ være, hvis man gerne vil begrænse mængden af ``ubehageligt indhold'' i sit feed, eller hvis man vil begrænse indhold inden for et særligt emne, man ellers hurtigt kan få nok af.

Angående funktionen over rating-aksen kan man sikkert komme langt, hvis brugerne bare kan indstille et minimumspunkt og et maksimumspunkt, et middelpunkt hvor kurven er højest, en højde/tykkelse af kurven (i.e.\ hvor bred kontra snæver skal kurven være (uagtet minimums-/maksimums-cutoffs'ne)), samt muligvis en parameter, der giver kurven en skewness, hvis man vil gå så langt. Jeg mener så at alt dette bør kunne indstilles ved at brugeren ``trækker og slipper'' forskellige punkter på en 2D-graf --- gerne også hvor der er parameterfelter under, hvor man kan aflæse værdierne af de nævnte parametre og også indsætte tal direkte her i stedet for at ``trække og slippe'' på grafkurven.  

Det kan godt være, at jeg lige skal genoverveje, hvor vigtigt det bliver med de ``specifikke tags.'' Måske kommer det lidt til at give sig selv, hvis man bare lige sørger for at indføre en måde, at brugere let kan referere til den ressource, de opretter tagget i forbindelse med\ldots\ Tja, eller nu hvor jeg lige tænker over det, så skal tag-dokumentationer måske bare kunne inkludere et vilkårligt antal referencer --- gerne a la gængse hyperlinks, hvor man kan tage et lille tekstudsnit, der er en direkte reference til ressourcen og så tilknytte et link til denne tekst --- og hvor tagget så bare dermed automatisk kommer til at indgå i ressourcens ``specifikke tag''-menu, eller hvad vi nu skal kalde den. Ja, og så bør det bare være rigtigt nemt for brugere at indsætte links i tag-dokumentationerne, især hvis de opretter tagget i forbindelse med en specifik ressource (hvor de altså så har navigeret til tag-redigerings-applikationen direkte fra denne ressource). Ja, det lyder fornuftigt. 


Okay, så langt, så godt. Nu bør jeg så komme med nogle illustrerende eksempler, der lige kan vende læseren lidt til tanken om pludseligt at kunne styre både kommentar- og ressource feeds på diverse platforme (der implementerer sådan et system) selv og, tror jeg, endda ret effektivt. Ah, og så bør jeg også komme ind på i denne forbindelse, hvor brugere skal kunne gemme disse feed-sorterings-indstillinger, både i form af indstillinger for enkelte tags og for en gruppe af tags på én gang. Den enkelte bruger skal herved have mulighed for hurtigt at skifte mellem forskellige feed-indstillinger, og brugerne skal desuden have mulighed for at dele deres feed-indstillinger med hinanden og for at offentliggøre dem til hele fællesskabet. 


%"Husk at nævne, at tags ikke her er binært sat, når vi snakker filtre.."
%"Ineffektive gængse korrelationer og YT-klik-eksemplet..."

\ldots Det kan forresten godt være, at jeg skal omstrukturere det ovenstående lidt, for det kan bl.a.\ godt være, at det giver bedst mening at snakke om sortering lidt tidligere i teksten. Det vil jeg lige se på. Men jeg kommer nok til at genskrive dette i endnu en løs opsummering, inden jeg går i gang med at lave den mere velstrukturerede (og i sidste ende mere renskrevne) tekst, så no worries. 


Lad os sige, at jeg har gjort det nu, jeg snakkede om i forrige paragraf, i.e.\ at jeg har givet nogle eksempler og måske forklaret lidt mere om at bruge tags til at sortere både kommentar-feeds og også ressource-feeds generelt. 
Det næste bliver så at forklare lidt om tag-dokumentationer og også om, hvordan tags'ne må kunne ordnes i en graf med relationer imellem sig, alt efter hvor tæt de ligger på hinanden semantisk. \ldots
%...
%
%... 
%Forklar her om at kunne efterspørge ting, bl.a.\ kilder og argumenter/modargumenter... 

(20.09.21) Okay, holdt, stop! Nu har jeg nogle flere idéer, og det ender altså med, at jeg skal skrive det ovenstående en hel del om (strukturmæssigt). En hel stor ny idé er, at man muligvis kan klare alt dette med at gruppere f.eks.\ rettelseskommentarer og andre typer kommentarer ved at indføre et eller flere højere-ordens-tags a la ``denne kommentar er bedre redegjort for i kommentar $x$.'' Og i øvrigt skal afstemninger (/ratings) om udsagn omkring en ressource nok bare ske nede i kommentarfeltet, men hvor man stadig så har en tag-menu, hvor man kan se populære kommentar-semantik-tags. \ldots Hm, men før man så brug for dette? \ldots Tjo ja, det ville jo være et meget dejligt design, hvis man har en mere vandret liste af emner lige over kommentarfeltet, alle beskrevet med relativt korte overskrifter (så altså nemlig i form af en slags tags, kan man sige)\ldots\ Ah, men kunne man så ikke have højere-ordens-tag a la ``denne kommentar hører til kommentaremne $x$,'' eller endnu bedre: ``denne kommentar hører til kommentaremne [$x$, $x.y$, $x.y.z$, \ldots]'' (hvorved man altså også har mulighed for endda at præcisere underemner)!? Det lyder da som en god idé\ldots! Okay, cool, og så skal jeg lige overveje, om ikke man både skal have tags a la ``denne kommentar er \emph{bedre} redegjort for i kommentar $x$'' og a la ``denne kommentar er redegjort for i kommentar $x$'' (eller om man bare skal nøjes med ikke-relative (absolutte) ratings)\ldots\ Ja, det må jeg bare lige tænke over, og ellers så er en anden ny idé, at man også måske bør sigte imod at kunne have argumentstruktur-udsagn, så man f.eks.\ kan sige: ``Denne kommentar antager korrektheden af udsagn $x$.'' Dette kan være særligt vigtigt ift.\ at gøre det synligt, når kommentarer er potentielt vildledende. Derudover ville det måske heller ikke være dumt med udsagn a la ``denne kommentar konkluderer $x$, hvis udsagn $y$, $z$, \ldots\ er sand(e).'' Og nu her i morges kom jeg så på, at disse ting måske kunne implementeres sammen med, hvad jeg har kaldt ``automatiske point,'' så man kan aggregere korrektheds- og/eller faresignaler osv. :) Så ja, der er altså lige et par ting, jeg skal tænke over nu her. \ldots\ Ah, og nu kom jeg også lige på, at tags også skal kunne forbindes med tag a la ``dette tag kan (måske med fordel) erstattes med tags $x$, $y$, \ldots (tilsammen).'' Nice nok, må man sige.

%(20.09.21) Brainstorm fortsat fra oven stående paragraf (sluttende med "skal tænke over nu her. \ldots\ "): ..Ah, og man selvfølgelig også gøre det samme med tags, hvor man så også siger "dette tag kan (måske med fordel) erstattes med tags x, y, ..." og lignende. Cool, det tror jeg lige også, jeg vil nævne ovenfor.. Sådan. ...Hm, jeg tror nok, at automatiske point kan gemmes som en senere tilføjelse, muligvis lidt sammen med pointen om at bruge "compound-tekster/-ressourcer," eller hvad jeg skal kalde det.. ..Ja, så første del af folksonomy-idéen er det her jeg har nævnt omkring fordelene ved rating-tags, og hertil bør jeg så også nævne det med at kunne navigere til en side, hvor man kan få (multidimensionelle) oversigter over, hvordan forskellige ressourcer kan sammenlinges på forskellige punkter. Herefter har jeg så nogle idéer, der handler om at gruppere kommentarer under kommentar-emne-tags, samt at skjule genganger-kommentarer og kun vise de bedste versioner. Og når dette er forklaret, så kan man passende forklare, hvordan et tilsvarende system kan bruges til at gruppere tags og fjerne gengangere. ..Og det er vel nærmest det, inden vi så når videre til de næste ting, som jeg heller ikke er nået til i det ovenstående, er det ikke?.. ..Tja, plus det løse, selvfølgelig, og ikke mindst skal man jo også lige huske at diskutere feed-sortering også.. selvom dette nu faktisk også nok godt kan komme lidt senere i det hele.. ..Nå nej, det med at kunne søge efter ressourcer med ratings indenfor specifikke intervaller, og ikke bare kunne søge efter mest/mindst muligt, kan jo være meget vigtigt.. Hm, jo, det er vigtigt, men det kan nu stadig godt nævnes senere i teksten (for det er ikke så forståelsesmæssigt vigtigt, og oftest vil man sikkert kunne klare sig fint bare med prioritets-parameterne (og altså en triviel funktion a la f(x)=x)).. ..Og jeg har visst ikke nævnt det, men nu mener jeg nemlig også, at ratings omkring udsagn, der handler ("specifikt") om ressourcen, nu bare skal formuleres som en kommentar, og så kan ratingen ske dernede. Og hvis man altså indføre en (rimelig) vandret kommentar-emne-tag-menu over kommentarfeltet, hvor de mest relevante kommentar-mener kan vises tidligt i listen, så opnår man jo alligevel den samme overskuelighed, som da vi snakkede specifik-udsagns-tags ved siden af ressourcen. ..Hm, skal man kunne linke ressource-tags og kommentar-tags, og i hvor høj grad skal man kunne udplukke udsnit fra kommentarer at rate i de tidlige stadier?.. ..Ja, man skal kunne linke kommentar-tags til (højere-ordens, eller rettere sammensatte) ressource-tags, for man skal jo bl.a. gerne kunne op-rate en advarsel, som så kan linke til et kommentar-emne, hvor dette forhold er forklaret. Så ja, det ville være godt. ..Hm, angående udsnit så tror jeg nu, man kan komme meget langt ved bare at kopiere udsnit over og tilføje dem som nye kommentarer, hvorved de så kan rates der. Og så kan man jo med tiden få det sådan, at man kan sige "denne kommentars korrekthed afhænger af x, y og ..." og ting i den dur. Så ja, det bliver faktisk ikke særligt vigtigt, om man kan udplukke udsnit fra kommentartekster, eller om man bare skal kopiere dem over i nye kommentarer. Fint.. ..Husk forresten det med, at tags bare skal have lov til at referere frit til flere forskellige ressourcer (inkliusiv-kommentarer.. og også andre tags og kommentar-emne-tags..!), hvorved ratingen, hvis den får nok opmærksomhed, så kan vises iblandt ressource-tags'ne. Husk også, at 'brugergrupper' gerne må komme lidt tidligere i teksten, end hvad jeg har tænkt indtil nu her for nyligt. Men ja, ellers tror jeg da næsten, at det bare er det, der skal være den nye struktur på denne del af idé-redegørelsen.. (og altså med mine seneste opdateringer til idéen).. 




%Hm, "specifikke tags" bliver vel sådan set bare tags, så, der refererer til pågældende ressource..(?).. ..Eller er det rigtigt nok bare alle tags, der handler om specifikke ting omkring eller nævnt i / indgående i indholdet..? Tja...
%Hvad var det nu, jeg tænkte på ellers?.. 
%Jeg skal i øvrigt også huske at diskutere, hvordan man så kan relatere to tags til hinanden (hvis de f.eks. har forskellige værdi-til-fortolkning-skalaer..).. 
%Åh, og husk også det med at man kan bruge tags til at efterspørge ting!
%Hm, jeg ved ikke, hvad jeg tænkte på, men måske var det bare omkring behovet for de "specifikke tags"... '(og/eller det om de så bare skulle referere til resourcen mere.. (måske, men ved ikke..))
%Jeg skal også lige overveje, hvilke mellemtrin der kan være til alt det her.. 

%Indskudt bemærkning: Hvor er jeg bare glad for denne folksonomy-vinkel-idé..! Hvor er det fedt med sådan en vinkel, hvor det tilsyneladende bare kan startes (det semantiske web --- eller i hvert fald den del, jeg er mest interesseret i) via tilføjelser til gængse systemer..! For der vil jo altid være en kæmpe netværkseffekt, når vi snakker sådan nogle ting, men her kan man så bare benytte netværk, der allerede eksisterer..!:) Og jeg kan stadig ikke fatte, at jeg nu --- og det ser det stadig ud til (at det holder)! --- har en KV-idé, som altså er meget tæt på de gængse systemer, og som jeg tror på, kan udkonkurrere dem (og altså på deres egen banehalvdel, så at sige; vi snakker ikke en KV ligesom mine lykke-KV-idéer (som jeg dog med min seneste version er omkring ligeså spændt på..!), der er inden for en anden niche, og som kan sameksistere med de mere "hårde" KV-kæder ret uafhængigt)..! :)


%Uh, og frasorteringen kan måske implementeres med et højere-ordens tag, der siger "please hide comment under semantic tag x." (tjek)
%Husk at diskutere, om folk vil benytte sig af at vurdere og inddele kommentarerne i høj nok grad til at systemet er det værd.
%Og nævn, at dokumentationen selvfølgelig ikke behøver at være særlig præcis for specifikke tags; det kan den i stedet være for de mere globale tags.
%"Tag-skabeloner"..
%At browse tags og tag-relationer i en (næsten di-)graf..
%Husk at nævne, at tags ikke her er binært sat, når vi snakker filtre..
%Ineffektive gængse korrelationer og YT-klik-eksemplet...
%Indstemme eksempler for dokumentationen..
%Husk at sige, at folk skal kunne offentliggøre filterindstillinger.
%Og ja, husk self. også det med at gemme sine egne indstillinger..




%Kommentar-tags..
%Hm, hvad gør vi egentligt her, snakker vi mon i første omgang, at man indføre tags om ressourcen, som så åbner op for kommentar-tags.. Hm, eller kommer dette ikke bare naturligt, hvis man lader kommentar-tag-forslag kunne afhænge af forælder-ressource-taggene?.. Og så kunne man måske bare oprette særlige tags specifikt til det formål at ændre kommentar-tag-foreslagne.. Hm, men så skal systemet vel gerne være mere åbent (ligesom det jeg tænkte med at have en særlig tag-graf ligefrem), så brugerne selv kan implementere dette, ikke?.. Hm... ..Ah, nu ved jeg det. Jo, men kan sagtens gøre det ret simpelt, og uden at man behøver at referere til de mere decentrale/brugerstyrede udvidelser (og særligt til tag-grafer). Jep, nice nok. ..Hm, og angående det at samle/sammenfatte reaktioner på indholdet, skulle man så gøre noget med afledte tags..? Hm, og så kan det måske godt være, at det bliver omvendt, sådan kommentar-tagget bliver "udtrykker enighed med udsagnet x," hvor x så kan være et udsagn forbundet med et tidligere defineret tag.. ..Ja, god idé, og så bliver der altså bare lidt graf- (og nærmere træ-)struktur over tags'ne i tag-menuen, selv i denne ellers ret simple idé, men det er jo også fint. ..Ja, og det er jo ikke fordi, brugerne så skal til at strukturere en hel masse nødvendigvis, for det handler bare om, at indføre tag-typer, der har et input, hvor dette input så enten bare kan være ny skrift eller tidligere tags. Og hvis inputtet er et tag, jamen så må dette jo bare naturligvis medføre en sammenhæng i tag-menuen, gerne i form at at det ene tag for det andet som barn i en træ-graf, men man kan også designe sammenhængen på andre måder. Cool cool, så altså mulighed for højere-ordens-tags, kan man vel nærmest sige. ..Man skal også kunne inputte ny skrift og så markere, at der gerne skal komme et nyt-barne tag i tag-menuen (hvis altså folk giver dette tag opmærksomhed og ikke bare vælger en alternativ version uden dette markat).. ..Det bliver så selvfølgelig også med højere-ordens-tags, men efterspørger kommentarer med, for dette kan man jo gøre med et "gode kommentarer omkring emnet x efterspørges"-tag. 






%%%%%'Brugergrupper'...:

%Husk også smagsdommere/anmeldergrupper..




%%%%%Mere brugerstyret..:


%%%%%Mere tværgående og muligvis decentralt.. (måske bare kort):





%%%%%Annonymitet og brugerdrevet ML.. :





%%%%%Annotationer



%%%Sammenligninger med sem-web:




%%%Nævne måske web 3.1 (p-modeller)..:




%%%Kom ind på "wiki-idé" ved at foreslå mit kollaborationsparadigme allerede her (og så kan wiki-idé-parten bare lige redegøre for, hvordan man også måske kunne fremføre idéen startende med sig selv): *...Nej vent, måske ikke alligevel; det kan godt bare være i sin egen sektion.. *Måske skulle jeg lave en ny sektion, men centrere teksten omkring teknikken i at dele værker op i semantiske bidder, når man samarbejder mange om et værk. Og så kan jeg relatere dette til en tilføjelse til folksonomy-idéen og alt det, og jeg kan snakke om muligheden for at starte en ny form for wiki, og altså endda muligvis før folksonomy-tingene fremkommer. *Nah, jeg kan godt nævne det i folksonomy-sektionen også (og muligvis faktisk så bare blande wiki-idéen ind i denne..(?)).








%Slider vs. stjerner. (tjek)
%Produktvurderinger..(tjek for nu, men kunne godt bruge et eksempel eller to mere)
%Ineffektiviteten af bare at bruge korrelationer (i.e. dårlige anbefalinger og "pas på, hvad du klikker på, og hvad du liker"). 
%Opfordring til at sammenlinge ressourcer på baggrund af et tag. (i gang)
%Tags til kommentarer... (eksempel: rettelseskommentarer.. *og anmeldelses-kommentarer..)
%Tag-relevans kunne også afhænge af tidligere fremstemte tags.. (tjek)
%Forespørgsel, fast og/eller fremstemt, af annotationer..
%Annonymitet og brugerdrevet ML.. 
%Sammenligninger med sem-web (og vejen til sem-web)..
%Nævne måske web 3.1..






\subsubsection[Forfra igen]{Jeg starter lige forfra igen igen}
%Kopieret ovenfra:
%"Brainstorm fortsat fra oven stående paragraf (sluttende med "skal tænke over nu her. \ldots\ "): ..Ah, og man selvfølgelig også gøre det samme med tags, hvor man så også siger "dette tag kan (måske med fordel) erstattes med tags x, y, ..." og lignende. Cool, det tror jeg lige også, jeg vil nævne ovenfor.. Sådan. ...Hm, jeg tror nok, at automatiske point kan gemmes som en senere tilføjelse, muligvis lidt sammen med pointen om at bruge "compound-tekster/-ressourcer," eller hvad jeg skal kalde det.. ..Ja, så første del af folksonomy-idéen er det her jeg har nævnt omkring fordelene ved rating-tags, og hertil bør jeg så også nævne det med at kunne navigere til en side, hvor man kan få (multidimensionelle) oversigter over, hvordan forskellige ressourcer kan sammenlinges på forskellige punkter. Herefter har jeg så nogle idéer, der handler om at gruppere kommentarer under kommentar-emne-tags, samt at skjule genganger-kommentarer og kun vise de bedste versioner. Og når dette er forklaret, så kan man passende forklare, hvordan et tilsvarende system kan bruges til at gruppere tags og fjerne gengangere. ..Og det er vel nærmest det, inden vi så når videre til de næste ting, som jeg heller ikke er nået til i det ovenstående, er det ikke?.. ..Tja, plus det løse, selvfølgelig, og ikke mindst skal man jo også lige huske at diskutere feed-sortering også.. selvom dette nu faktisk også nok godt kan komme lidt senere i det hele.. ..Nå nej, det med at kunne søge efter ressourcer med ratings indenfor specifikke intervaller, og ikke bare kunne søge efter mest/mindst muligt, kan jo være meget vigtigt.. Hm, jo, det er vigtigt, men det kan nu stadig godt nævnes senere i teksten (for det er ikke så forståelsesmæssigt vigtigt, og oftest vil man sikkert kunne klare sig fint bare med prioritets-parameterne (og altså en triviel funktion a la f(x)=x)).. ..Og jeg har visst ikke nævnt det, men nu mener jeg nemlig også, at ratings omkring udsagn, der handler ("specifikt") om ressourcen, nu bare skal formuleres som en kommentar, og så kan ratingen ske dernede. Og hvis man altså indføre en (rimelig) vandret kommentar-emne-tag-menu over kommentarfeltet, hvor de mest relevante kommentar-mener kan vises tidligt i listen, så opnår man jo alligevel den samme overskuelighed, som da vi snakkede specifik-udsagns-tags ved siden af ressourcen. ..Hm, skal man kunne linke ressource-tags og kommentar-tags, og i hvor høj grad skal man kunne udplukke udsnit fra kommentarer at rate i de tidlige stadier?.. ..Ja, man skal kunne linke kommentar-tags til (højere-ordens, eller rettere sammensatte) ressource-tags, for man skal jo bl.a. gerne kunne op-rate en advarsel, som så kan linke til et kommentar-emne, hvor dette forhold er forklaret. Så ja, det ville være godt. ..Hm, angående udsnit så tror jeg nu, man kan komme meget langt ved bare at kopiere udsnit over og tilføje dem som nye kommentarer, hvorved de så kan rates der. Og så kan man jo med tiden få det sådan, at man kan sige "denne kommentars korrekthed afhænger af x, y og ..." og ting i den dur. Så ja, det bliver faktisk ikke særligt vigtigt, om man kan udplukke udsnit fra kommentartekster, eller om man bare skal kopiere dem over i nye kommentarer. Fint.. ..Husk forresten det med, at tags bare skal have lov til at referere frit til flere forskellige ressourcer (inkliusiv-kommentarer.. og også andre tags og kommentar-emne-tags..!), hvorved ratingen, hvis den får nok opmærksomhed, så kan vises iblandt ressource-tags'ne. Husk også, at 'brugergrupper' gerne må komme lidt tidligere i teksten, end hvad jeg har tænkt indtil nu her for nyligt. Men ja, ellers tror jeg da næsten, at det bare er det, der skal være den nye struktur på denne del af idé-redegørelsen.. (og altså med mine seneste opdateringer til idéen).."

%(20.09.21) Uh, nu kom jeg lige på en ny ting: Man kan også bruge sammensatte tekster til at lave rapportstrukturer! Man kan altså lave en skabelon for, hvilke nogle ting skal besvares i en rapport, når det kommer til et givent emne. Og så kan folk herved ret automatisk bedømme, hvis alle udsagn i protokollen er besvaret tilstrækkeligt, om konklusionen så holder. :) 

%(21.09.21) Jeg tror, jeg vil forklare om at sammensmelte tags, før jeg forklarer om at sammensmelte kommentarerne. Men jeg kan måske godt nævne grupperingstags til at ordne kommentarfeltet, før jeg nævner, at man også kan gruppere tags'ne på tilsvarende vis (hvad jeg nu tænker).. Så hvad bliver dispositionen?.. Forklar om rating-tags (gider jeg ikke at gentage her), forklar om dokumentation med eksempler på rating-skalaen, forklar om "multidimensionel" oversigt, forklar om tags med flere input *(og om "tag-skabeloner," som jeg nok vil kalde dem nu --- i.e. tags med variabelt input.. ja, og jeg kunne også kalde dem variable tags, men tag-skabeloner virker nu også meget fint), forklar om tag-relationer, som jo faktisk er en form for højere-ordens-tags, der tager tags som input. Forklar så om visse mulige måder til at sætte helt eller delvist lighedstegn --- måske med en retning oven i købet --- mellem tags og/eller mellem grupper af tags, så tags på denne måde kan "opdateres," så at sige. *(Nævn også ting omkring feed-sortering (før eller efter følgende punkt om 'brugergrupper').) ..Hm, det er lige før, man så skulle nævne 'brugergrupper' allerede her (før det bør vistnok kunne nævnes ret kort).. Det tror jeg næsten, jeg vil gøre, for alt det med kommentarerne kan i princippet implementeres, hvis siden tillader tekst-ressourcer; det er kun hvis siden allerede har et livligt kommentar-/anmeldelses-system, at det bliver vigtigt at se på dette emne separat. Hm, skal jeg så også nævne det at gruppere tags i tag-menuen (som faktisk er en ny idé fra i går aftes, selvom jeg har snakket om at "gruppere tags" i det ovenstående (men her tænkte jeg nemlig mere ift. at absorbere/fjerne gengangere)) allerede? Ja, det er nok en god idé, for det kan nævnes rimeligt kort, og så er hele denne del af systemet (nemlig omkring ressource-tags) redegjort for.. (Er den ikke?..) ..Jo, det er det muligvis. Og inden man så går over til kommentarer, så kan man lige nævne, hvordan kommentarer nu i princippet allerede kan implementeres via ressource-tags'ne, hvis systemet ellers kan inkludere tekster som ressourcer, og hvis man så lige kan sørge for, at sådan tekst kan vises i forbindelse med et tag. Men dette er kun i teorien. I praksis bør man selvfølgelig bare bruge det kommentar-system, der allerede er implementeret --- og så kan man altid konvertere til et system, hvor kommentar-teksterne hives ud og bliver opgraderet til førsteklasses-ressourcer, hvis dette bliver bedre på et tidspunkt. Men indtil da kan vi passende se på, hvad man kan gøre for kommentarerne --- hvilket så bare nu meget kommer til at gentage pointerne fra ressource-tags'ne. ..Nå jo, jeg har så ikke diskuteret feed-sorteringer endnu.. Hm, det kunne jeg jo gøre lige før eller efter (umiddelbart hælder jeg dog til før), jeg nævner 'brugergrupperne'.. Cool, er det så det..? Jo, plus det løse.. Jeg bør forresten også huske at nævne, at tag-skabeloner også bør kunne rates ind i grupper. Hm, måske bør jeg vente med at illustrere særligt meget omkring tag-menuen, indtil jeg når til kommentarerne; lad mig bare nævne det lidt kort i forbindelse med ressource-tags'ne.. Nå ja, og nævn at ressource-tags jo stadig dog gerne må kunne referere direkte til en kommentar. Ok, cool.:) *(Jeg skal også huske de automatiske point.. Det kan måske komme efter 'brugergrupper' og inden kommentarerne..) *(Og jeg skal også huske det med at diskutere, om brugere vil rate nok til at systemet bliver værdifuldt..)


Jeg gider ikke at gentage forklaringerne på, hvorfor rating-tags er smarte, her. Så lad os lade som om, jeg har redegjort for dette allerede (hvis vi ser teksten i denne undersektion som et brainstorm-udkast til den tekst, jeg gerne vil skrive).

Jeg vil så gerne som noget af det første efter dette forklare lidt om dokumentationerne til tags'ne. 
Jeg vil nemlig (som måske er nævnt) foreslå, at tags'ne i dette system alle har en dokumentation med sig, der sammen med deres tag-overskrift definerer tagget, således at to forskellige tags faktisk godt kan have samme overskrifter. Dette kan umiddelbart skabe noget forvirring, men jeg vil så foreslå, at for hvert nyt tag, som en bruger uploader og anmoder om godkendelse for, så vælger platformen et symbol, og også gerne en farve, der kan omkranse tagget, hvis designet tillader dette. Platformens opgave er så at sørge for, at ingen tags har samme symboler, og slet ikke hvis de har samme eller lignende overskrifter. Platformen må gerne opdatere disse symboler og farver, men kun på en måde, så de pågældende tags ikke så kan forvirres med andre tidligere tags --- og selvfølgelig heller ikke med andre samtidige tags. Dette kan måske virke som en stor opgave, men i praksis kan platformen bare for det første generere symbolerne tilfældigt, indtil brugerne kommer med en specifik forespørgsel (så platformen behøver altså ikke at sætte folk til at finde på symboler selv), og samtidigt kan platformen også bare implementere et system, hvor brugerne selv kan rapportere, hvis to forskellige tags let kan forvirres med hinanden, og hvor de så også selv kan indsende forslag til (og måske stemme om) ændringer, der eliminerer forvirringsmulighederne. Platformen behøver i så fald bare lige at sætte nogen til at gennemgå godkende disse forslag --- hvilket man endda også kan sætte moderator-brugere til, hvis man virkeligt vil spare på ressourcerne.

Og hvad skal tag-dokumentationerne så indeholde? De skal såmænd indeholde en forklarende tekst, der definere brugen af tagget. Hvis vi f.eks.\ ser på NSFW-tagget, så bør det forklares, hvordan dette ikke bør forvirres med NSFL (not safe for life), eller for den sags skyld med andre tags, man har indført til at opdele disse advarselskategorier --- så som ``NSFW swearing'' eller ``foul language'' og hvad man ellers har af mulige advarsels-tags. Herved kan vi altså eliminere en masse forvirring omkring tags, der ellers ofte hersker på nuværende internetplatforme. Jeg bør i øvrigt også allerede have nævnt et eksempel, hvor man bliver nødt til at bruge tag-dokumentation for at forklare tagget ordenligt (nemlig hvis man prøver at definere et nyt koncept, som altså ikke er velkendt nok til, at man bare kan beskrive det med en hurtig overskrift). (Ellers bør jeg nævne eksemplerne her.) Så med uddybende tag-dokumentationer (/-definitioner, kunne man også kalde dem), kan brugere altså også lettere skabe nye tags, som andre brugere ikke har set før, og stadigt gøre konceptet forståeligt. 

Ok, men det er ikke hele historien, for i dette system skal tags jo også rates, og en stor del af hele pointen med at indføre et sådant system er, at brugere så også bliver i stand til at få gavn af tags, der ikke bare enten er sande eller usande, men som kan passe til en vis \emph{grad} på ressourcen. Og derfor er det så også vigtigt, for alle sådanne tags, der ikke bare er enten sande eller falske, at inkludere dokumentation for, hvordan rating-skalaen skal fortolkes. Hertil vil jeg så foreslå, at forfatteren angiver skriftlige eksempler langs hele rating-baren (med passende jævne intervaller; muligvis med i omegnen af fire--seks forklaringseksempler). Jeg vil foreslå at man dog prøver ikke at inkludere specifikke ressourcer i disse eksempler, hvis det er muligt, for disse ressourcer er jo en del af, hvad der skal rates, og man kan jo netop ikke forvente, at alle brugere ser ens på pågældende ressourcer. Så hvis man skal komme med eksempler, så bør man måske i det mindste give flere ad gangen, og så sige det lidt løst, at ``disse eksempler bør nok befinde sig heromkring på rating-aksen (pr.\ forfatterens hensigt med skalaen for det første, men selvfølgelig også pr.\ forfatterens vurdering af pågældende eksempler).'' Man kan selvsagt designe det på flere måder, men jeg forstiller mig så en dokumentations-side, hvor der øverst er en (vandret) rating-akse, og hvor selve dokumentationsteksten så er nedenunder denne. På rating-aksen forestiller jeg mig så, at der kan være punkter på, som forfatteren har sat, hvilke læseren kan klikke på / holde musen over, hvorved brugeren så bør få en pop-ud-boks, med eksempelteksten for pågældende punkt. Ydermere vil jeg også forslå, at man lige ovenover denne aksen giver plads til en håndfuld ressource-thumbnails, som så kan vises i form af ``pegs'' (eller hvad man kalder dem konventionelt set; jeg tænker a la den type pegs, som bl.a.\ Google Maps bruger (men selvfølgelig ikke nødvendigvis med samme (runde og glatte) design som for dette eksempel)) med pilene / de spidse ender ned på rating-aksen. Tanken er så, at platformen kunne have en algoritme, der finder et passende udsnit af ressourcer, for hvilke pågældende tag er ratet som ``relevant for ressourcen,'' og indsætter dem på aksen som nuværende eksempler på, hvordan brugerskaren bruger tagget. Dette er altså mine forslag til designet og brugen af dokumentationerne.


Fordi tags'ne i dette system kommer til at kunne sige noget forskelligt om en ressource alt efter, hvor meget ratingen er på, så vil det derfor også blive gavnligt, hvis folk ikke bare kan søge på ressourcer ud fra, hvilke har den højeste rating-score for ét eller flere tags, men også kan søge efter ressourcer, der matcher en eller flere specifikke tag-værdi(er) bedst muligt. 
Lad os således forstille os et feed af ressourcer på platformen. Dette kunne passende være hoved-feedet på platformen. I princippet kunne der også være tale om et feed (hvis man kan kalde det det i denne sammenhæng også), der kommer fra en søgning. Dog vil dette nok sjældent blive ligeså relevant i praksis, for hvis man gerne vil søge på ressourcer via tags, så vil det nok typisk være fordi, man leder efter nye ressourcer, man ikke kender på forhånd, og ikke fordi man leder efter en specifik ressource\ldots\ Tja, eller på den anden side så kan jeg faktisk tænke mig et par eksempler, hvor det ville give mening at kombinere søgeord med tag-rating-søgninger. Det kunne nemlig enten være, at man leder efter noget specifikt, som man dog ikke kan huske navnet på, eller det kunne også være, hvis man er interesseret i et vist emne (passende til ens søgeord), men man gerne vil sørge for at filtrere visse typer indhold fra. Dette kunne jo enten være indhold med advarselstags a la NSFW osv., eller det kunne også være, hvis man ikke er interesseret i et specifikt underemne af, hvad man søger på --- eksempelvis hvis man gerne vil søge på platformspil (og af en eller anden grund bruger søgeord og ikke tags til at søge på dette), men hvor man ikke er interesseret i Maga Man-agtige platformspil. Anyway, lad os nu forestille os sådan et feed. Det kunne så være, at man f.eks.\ ledte efter et godt adventurespil, og gerne med en del vægt på puzzle-elementer, men uden at genren går helt over i puzzle-kategorien (som den primære kategori)\ldots\ Hm, eller lad os sige, at man lige har gennemført Half-Life 2 endnu en gang, og godt kunne tænke sig at se, om ikke der er andre gode action-spil (og måske særligt FPS-spil), hvor action-sekvenserne er brudt op jævnligt af puzzle-agtige segmenter. \ldots Tja, jeg bør faktisk komme med en hel del eksempler; måske skulle jeg således lave en appendiks-sektion, hvor jeg kommer med en hel del flere eksempler. Og her kan jeg så bare komme med et enkelt eller to (nok hellere to), og så referere til appendikssektionen for flere eksempler. Nå, men det korte af det lange er, at sådanne former for søgninger nu kan lade sig gøre, hvis ressourcerne får rimeligt rammende tag-ratings, der beskriver hvor meget de hører til et emne og til et andet. Det er lige før, jeg også burde nævne noget om under-tags (som i underkategori-tags) her, men jeg kunne også putte en nål i det, og vente til jeg har forklaret om tag-relationer\ldots? For det giver jo mening, hvis jeg vil nævne noget som ``Mega Man-agtige platformspil,'' som jo vil være en oplagt underkategori til ``platformspil.'' \ldots Men lad mig putte en nål i det for nu og gå videre (til næste punkt, som er koordinatsystem-oversigter *(Nej, det næste punkt bliver feed-sortering)). %, og så tror jeg faktisk, jeg forklarer om tag-relationer m.m. og om kommentargruppering inden da\ldots\ )). 

*Nå nej, først bør jeg måske lige knytte nogle flere kommentarer til feed-(søgnings-)algoritmer. Ja, jeg kan lige nævne, at for søgninger centreret omkring punkter på en eller flere tags-rating-barer, så kunne man måske også give mulighed for det første for at vælge en parameter for, hvor hurtigt relevans-scoren skal aftage, når vi bevæger os væk fra det valgte punkt. Dette er især vigtigt, når der er flere tags i spil i søgningen. Her kan det nemlig være, at man gerne vil søge meget snævert omkring én tag-rating, men ikke er ligeså bekymret over, hvor nøjagtigt de andre tags passer på det valgte midtpunkt for søgningen. Måske ville det give mening, hvis man også kunne vælge at dele det op, så man valgte en aftagnings-parameter i hver retning, men så snakker vi selvfølgelig en avanceret søgeindstilling. Det kunne også meget vel blive gavnligt, hvis man kan vælge et minimums- og maksimumspunkt for søgeintervallet for en tag-rating, så kan sørge for helt at undgå ressourcer, der har en tag-rating under eller over en vis / visse værdi(er). 

%*%Uh, og jeg skal jo også lige komme ind på at gemme og dele indstillinger, og så skal jeg også komme ind på (kort), at brugerstyrede feed-algoritmer bare er ftw.
*Jeg skal vist lige huske også at komme lidt ind på, at brugerne jo skal kunne gemme deres feed-algoritme-indstillinger, og at de så også gerne endda skal kunne offentliggøre dem og/eller dele dem med andre. Måske kan jeg også sige noget mere om, hvorfor brugerstyrede feed-algoritmer er nice, men det kan jeg lige se på.


Og når vi så snakker om at få vist ressourcer på akser, så kan jeg passende også lige nævne en anden ting, jeg tror, vil være gavnligt for sådan en platform. Jeg vil nemlig gerne foreslå, 
%Skal kunne klikke og få vist et feed med nærmest-relaterede ressourcer. Husk i øvrigt også at nævne det med, at det kan hjælpe folks motivation (og deres effektivitet/præcision) for at rate ressourcer, når tingene vises sådant ved siden af hinanden på en akse..
at brugerne får mulighed for at navigere til en separat side, hvor ressourcerne ikke vises i form af et lineært feed med ressourcer fra højest til lavest relevans ift.\ søgekriterierne (og den underliggende feed-agoritme, hvis nu platformen kun har begrænset gennemsigtighed og åbenhed, når det kommer til de brugte algoritmer), %feed, hvor de ressourcer med den højeste score ift.\ den valgte (og den underliggende, hvis platformen kun har begrænset gennemsigtighed og åbenhed, når det kommer til de brugte algoritmer) feed-sorterings-algoritme bliver vist øverst og/eller først i feedet, 
men i stedet vises i et koordinatsystem med tag-rating-resultater langs koordinatakserne. 
Dette kan være gavnligt, hvis man gerne vil have en oversigt over de forskellige muligheder, når det kommer til genre/kategori. Hvis man således f.eks.\ har to tags på sinde, lad os sige\ldots ``historisk korrekthed'' og ``spændingsfyldt,'' hvis vi ser på et eksempel med film --- ja, og det kunne eksempelvis være, hvis man er to personer, der hver især hælder mest til det ene prædikat og er mere ligeglad med det andet --- og man gerne vil have et overblik over, hvilke muligheder, der er. Så kunne sådan et koordinatsystem være smart, hvor man kan få plottet relevante ressource-thumbnails ind i koordinatsystemet. Her kunne relevans så afhænge af generel popularitet på siden og/eller af, hvad platformens algoritmer tror, er relevant for brugeren (/brugerkontoen), og man kunne måske i øvrigt også sørge for at allerede sete film (nu vi altså snakker om film) også bliver vist i et vist antal, så brugeren har noget kendt at sammenligne med i koordinatsystemet. Min tanke er så at brugere skal kunne klikke på et område af koordinatsystemet, hvorved brugeren så skal få et feed med tag-søgekriterier centreret omkring de pågældende rating-værdier. Således kan man nemlig nøjes med kun at vise passende eksempler i koordinatsystemet, og hvis brugeren så vil se flere eksempler omkring at punkt, kan denne altså klikke omtalte søgnings-feed frem. 
Jeg kan altså forestille mig, at denne mulighed vil være gavnlig. 

En ting jeg så lige bør forholde mig til, er jo så, hvad man gør, hvis man gerne vil se en oversigt over flere end bare to tags. Her er vi jo som mennesker begrænset af, at vi kun kan se, helst kun to, og maksimalt kun tre dimensioner i et koordinatsystem. Allerede når vi når til tredimensionelle koordinatsystemer kan det blive uoverskueligt, og flere dimensioner end dette er stort set umuligt at præsentere på en overskuelig vis for brugeren. Jeg tror dog, at to koordinatakser ad gangen er rigeligt, også selvom man har flere en to tags i spil. Så kan man nemlig bare holde de resterende tags konstante (forstået på den måde at de viste ressourcer tages ud fra en feed-algoritme, hvor tag-rating-værdierne er centreret nogenlunde omkring et punkt) og kun have to ad gangen, der variere i koordinatsystemet. Jeg forestiller mig så, at brugeren så bare skal have mulighed for enten at klikke på en række eller en kolonne (hvis altså koordinatsystemet er opdelt i et grid --- og ellers må man bare vælge vandrette eller lodrette linjer i stedet). Når brugeren så har valgt sådan en række/kolonne/linje, så skal brugeren kunne klikke på et af de ikke-aktive tags, man ser på, hvorved dette tag så skal erstatte det forrige tag på henholdsvis den lodrette eller den vandrette akse (alt efter om man valgte en vandret række/linje eller en lodret kolonne/linje). Jeg ved godt, at det er lidt rodet beskrevet, men jeg håber, det giver nogenlunde mening. Og med dette tror jeg altså, at brugere ret effektivt kan komme til at få oversigt over, hvad der i princippet er et multidimensionelt koordinatsystem. 

Man kan så diskutere, hvor meget man skal gå op i, at brugere kan præcisere søge-kriterierne for de ``konstante tags,'' man har i spil\ldots\ Ja, jo, men nu hvor jeg faktisk har nævnt nogle ting omkring den mulige avancerede søgning for tags, så kan man bare nævne her, at de ``konstante tags'' netop bare kan have valgt faste søgekriterier for sig (så de altså er centreret om et vist punkt, men hvor brugeren desuden også kan justere på, hvor snævert resultaterne skal ligge omkring søge-midtpunktet for tagget, og også kan vælge minimums- og maksimums-punkter for søgningen). 

En yderlige stor gavn, jeg tror platformen ville få af at inkludere sådan en koordinat-oversigt-applikation, og hvorfor jeg rigtigt gerne ville nævne denne lille idé, kommer sig så af, at det nok vil øge folks interesse i at rate ressourcerne gevaldigt. Jeg tror således, %..Hm, måske skulle man alligevel have relative vurderinger, for dette er da også meget mere annonymt for brugeren..!..? ..Hm, men det er nu bare lidt svært at implementere, for relativ forskel kan jo ikke hverken bare måles i en procentforskel eller i en faktor; afstanden folk vil give til to ressourcer afhænger af, hvor de i forvejen er henne på kurven. Så hvad kan man overhovedet gøre..? ..Nå, det vil jeg lige vende tilbage til senere, for de skulle blive godt vejr i dag; så kan jeg tænke over det der.. ...Ah, nej. Never mind omkring relative tags, for det bliver heller ikke nær så vigtigt, når ratingerne nu er mere nuanceret; så skal man ikke tænke over \emph{hvad} man liker..!(..!!) 
at sådan en oversigt, hvor folk kan holde godt øje med, hvordan forskellige ressourcer rangerer ift.\ hinanden og forskellige tag-udsagn, vil motivere mange meget for at afgive stemmer, for dette kunne jo f.eks.\ være, hvis man er uenig i, at én ressource rangerer over en vis anden for et tag. Herved kan man jo så nemt blive motiveret til lige at give en eller to hurtige stemmer, som hjælper til at ``rette'' denne rangering. 


%At sammenligning motiverer.


Bemærk (hvis jeg ikke har nævnt det), at forslaget omkring en koordinat-oversigt-applikation selvfølgelig ikke hører til det grundlæggende i idéen, men hører til noget, man muligvis kan implementere efterfølgende i takt med at tag-rating-systemet på platformen udvikles. For selvom idéen måske godt kan fungere fint bare med de helt grundlæggende tiltag, så kan det jo ikke skade at komme med flere idéer til, hvordan systemet kan udvikles.

%%%%%%Feed-sortering allerede her? (..Ja, og så muligvis faktisk brugergrupper lige efter..) *..Hm, eller måske skulle jeg faktisk lige skrive om feed-sortering først, før koordinatsystem-siden.. ..Nu har jeg gjort det sådan (altså feed-sortering først).. (tjek; ovenfor)
%
%%%%%%Måske: Brugergrupper. ..Ja. Cool, og så fortsætter jeg bare nedenfor med tag-input, højere-ordens-tag og tag-ralationer (inkl. nu under-/over-tag-relationer) osv. *(Nej, nu venter jeg igen til senere med dette punkt.)


Det næste, jeg vil foreslå, er %at tag-dokumentationer får mulighed for at indeholde links til...
``tag-skabeloner,'' %'' samt ``højere-ordens-tags'' og ``tag-relationer.'' Pointen her er for det første, at det måske kunne være gavnligt at tilføje muligheden for at lave tag-skabeloner, 
som %altså
er en slags variable tags, %der skal tilføjes en vis mængde input, før de bliver til gyldige tags. Det er altså en måde at genbruge dokumentationstekst på, således at brugere kun behøver at inputte visse %..Ja, det bliver bestemt ikke den kønneste tekst den her (min hjerne kører åbenbart lidt langsomt), men det gør jo ikke noget.. ..Ah, fuks det, jeg tager lige en tidlig middagspause, og så kan jeg lige tænke over de her ting (og altså også det jeg nævnte før om relative vurderinger).. ...Nu har jeg den vist..
der tager en mængde input og ``returnerer'' et normalt tag. Det er altså bl.a.\ en måde at genbruge dokumentationstekst på til flere tags. Input kunne således i første omgang bare være tekst, som fuldender dokumentationen. Et eksempel kunne være en tag-skabelon på formen ``$\lambda x.$kommentaren, $x$, er en vigtig rettelse til noget, der blev sagt i denne ressource.'' Det kunne også handle om ``et vigtigt modargument'' eller ``en relevant kilde'' osv. Udover at det så bliver lettere for brugere at udforme specifikke tags, fordi de kan benytte en fast skabelon, så kan brugen af tag-skabeloner også medføre en anden fordel. Platformen kunne således sørge for, %at det kan registreres hvilke skabeloner har det med at blive populære...%Hm, skal man nu også foreslå dette, for jeg kommer jo alligevel til at foreslå noget lidt bedre, og vil det virkeligt være særligt praktisk; populære tags skal jo nok blive vist i sidste ende alligevel.. Ja, nej; så længe man bare får mulighed for at søge på tag-skabeloner, så... Ah, men det er da bare netop den ting, jeg skal foreslå her, så.
at brugere kan søge på tags dannet ud fra en specifik tag-skabelon i tag-menuen. Så hvis en bruger synes, at der blev hævdet nogle utrolige og/eller mærkelige ting i ressourcen (hvis ressourcen er en video eller tekst, der indeholder påstande), så kan brugeren altså så sage specifikt på ``rettelse''-skabelonen, eller måske på ``kilde''-skabelonen bagefter, og finde frem til en liste af de mest op-ratede rettelser eller kilder. 

Så langt, så godt, men det kunne også i så fald være smart, hvis tag-skabeloner kunne tage andre ressourcer som input. Her kunne vi f.eks.\ tænke os et tag, der sagde, ``$\lambda x.$denne ressource indeholder spoilers for ressourcen, $x$,'' eller ``$\lambda x.$denne ressource skal helst ses/opleves efter ressourcen, $x$, er set/oplevet.'' Det kunne også, hvis vi f.eks.\ tænker på videospil, være ``$\lambda x.$dette spil og spillet, $x$, indeholder lignende segmenter (eksempelvis puslespil eller logiske gåder), der kan medføre, at underholdningsværdien falder for et af disse spil, hvis man har spillet det andet spil først (og især hvis det er for nyligt).'' Med disse eksempler bør det heller ikke være svært at se, at det så vil være gavnligt, hvis sådanne tags, der indeholder referencer til andre ressourcer, også får tilknyttet et link til samme, når man iagttager tagget (så man f.eks.\ kan navigere direkte til den ressource, man eftersigende bør se/opleve før den pågældende ressource osv.). Ydermere tror jeg også, det ville være smart, hvis tags, hvor ressourcer er givet som input, også bliver vist i tag-menuen for disse, og altså ikke kan bliver vist for den ressourcer, der figurerer som subjektet for taggets udsagn. Her er mit tredje eksempel omkring spil, hvor subjekt og input-objekt er ligeværdige, jo et godt eksempel på, at dette kan være gavnligt. 

Lad mig også lige komme med nogle eksempler, hvor der bruges mere end ét input\ldots\ 
*[Hm, nu er jeg så tæt på at færdiggøre hele dette notesæt, så jeg tror bare jeg vil vente med at overveje sådanne eksempler, til når jeg nu skal til at skrive en renere version af denne tekst (altså omkring mine rating-folksonomies).]

Det næste jeg vil foreslå kræver måske lidt mere arbejde at implementere end mange af disse andre ting, men jeg tror bestemt at fordelen ved at få idéen op at køre vil være tilsvarende stor. Jeg vil foreslå% , hvad vi kan kalde ``højere-ordens-tags,'' i.e.\ tags, der tager andre tags som input.
en slags ``tag-relationer,'' som i princippet kommer til at fungere meget ligesom tag-skabeloner, bare hvor dele af inputtet simpelthen er andre tags. Bemærk for det første, at alle tags kan ses som et prædikat omkring subjekts-ressourcen og/eller som en relation mellem ressourcer, hvis tag-dokumentationen referere til flere ressourcer (eksempelvis hvis der er tale om en tag-skabelon med tilførte ressource-input). Det er kun fordi idéen tager udgangspunkt i en udvidelse til folksonomy-systemer, at vi kalder dem ``tags'' her. Ellers kunne vi ligeså godt have kaldt dem prædikater/relationer. Og nu vil jeg altså foreslå, at systemet udvides til at inkludere, hvad der svarer til højere-ordens-relationer, nemlig relationer imellem tags. Og grunden til, at jeg gerne vil foreslå dette, er, bl.a.\ så man kan få mulighed for at relatere tags til hinanden, hvis de har lignende semantik, og så man på denne måde også effektivt set kan komme til at ``opdatere'' tags til bedre versioner. Dette kan enten handle om at opdatere et tags dokumentation til en mere koncise og/eller bedre beskrevet udgave, eller det kunne også f.eks.\ være, hvis man fandt ud af, at det var smart at opsplitte et tag i flere dele, eller hvis man fandt en mere intuitiv og/eller mere præcis opdeling af semantiske principper i form af en gruppe tags, som man så gerne vil erstatte en tidligere gruppe tags med. Jeg bør give nogle eksempler på dette, men lad mig før da lige forklare, hvordan tag-relationer skal bruges til at opnå denne mulighed.

En måde, jeg tror, man kan implementere dette på, er, hvis man på platformen indfører specielle tag-relationer (/højere-ordensrelationer) med betydningerne: ``$\lambda x,y.$Dette tag, $x$, forklares bedre og bør erstattes af tagget, $y$'' og (for flere tags) ``$\lambda x_1, x_2, \ldots, y_1, y_2, \ldots\,.$denne gruppe af tags, $x_1, x_2, \ldots$, bør erstattes af tag-gruppen, $y_1, y_2, \ldots$'' (som selvfølgelig også indeholder første skabelon, hvis kun tager ét $x$ og ét $y$) samt ``$\lambda x,y.$dette tag, $x$, og tagget, $y$, har tilsvarende betydning,'' ``$\lambda x_1, x_2, \ldots, y_1, y_2, \ldots\,.$denne gruppe af tags, $x_1, x_2, \ldots$, og tag-gruppen, $y_1, y_2, \ldots$, dækker tilsammen over det samme''\ldots\ Ja, eller man kunne også bare have ``$\lambda x, y_1, y_2, \ldots\,.$betydningen af dette tag, $x$, bliver dækket af tag-gruppen, $y_1, y_2, \ldots$,'' i stedet for sidstnævnte tag-relations-skabelon. Og det er vel det, er det ikke? \ldots Ja, bortset fra at man så også muligvis kunne erstatte min anden skabelonssætning ovenfor her, med ``$\lambda x, y_1, y_2, \ldots\,.$betydningen af dette tag, $x$, bliver dækket på en bedre måde af tag-gruppen, $y_1, y_2, \ldots$, og bør erstattes til fordel for brugen af disse tags.'' Bemærk, at jeg lige har valgt at gøre alle subjekterne eksplicitte i sætningerne i modsætning til, hvad jeg ellers har gjort ovenfor. 

På denne måde kan vi bygge en hel graf over platformens tags, hvor kanterne repræsenterer, når to tags har lignende semantik, og hvor graden er, hvor meget deres semantik ligner så bør komme til udtryk ved, hvor højt disse tag-relationer (der er repræsenteret af kanterne) er vurderet af brugerne. 
%Det er så meningen, at platformen skal holde øje med denne graf og
Hvis så et tag eller en gruppe af tags i høj grad bliver vurderet til at burde erstatte en anden gruppe tags, så bør platformen så have en måde at hjælpe denne proces på vej. Dette kan som minimum bare være at sørge for, at 
%Ah, nu har jeg det. Frivillige brugere, der gerne vil prøve at fremføre en tag-gruppe frem for en anden, kan udarbejde transformationsskemaer. Ofte behøver man sikkert kun ét, men der kun være tilfælde, hvor der kan være tvetydigheder, der går tabt i overgangen, og hvor folk derfor kan have forskellige meninger. Men pointen er alts, at brugere, der har rate'et ting fra den gamle tag-gruppe, nu bare kan blive spurgt, om de vil tranformere over, og så kan de vælge det transformationsskema, der passer til deres fortolkning af de gamle tags.:)
som mange brugere har stemt for, bør udgå, får et mærke, der siger ``deprecated,'' og giver et link til erstatningerne, så folk kan begynde at bruge disse i stedet.
Men man kan også gøre mere end dette. Det ville nemlig være smart, hvis man på en eller anden måde, kunne få brugerne til at overføre deres gamle vurderingssvar til de nye, opdaterede tags. Her kunne man selvfølgelig bare spørge brugerne, om de ikke vil gøre dette, og måske implementere en måde, hvor de hurtigt kan gennemgå alle de ressourcer, de tidligere har vurderet med tags fra den gamle gruppe, og hvor de nye tags også er vist, så brugerne kan klikke deres vurderingssvar over på disse. Det kan godt være, at dette bare bliver løsningen i praksis, men jeg har også et andet forslag, der måske kan være værd at indføre. Hvis nu frivillige brugere havde muligheden for at lave ``transformationsskemaer,'' så brugere bare kan overføre alle deres tidligere rating til den nye gruppe med et enkelt klik, så ville dette jo lette processen. Især hvis der nu bliver mange små opdateringer til tag-dokumentationerne på platformen, og især hvis semantikken ikke ændrer sig så meget for hver gang (hvis den overhovedet ændrer sig). Transformationsskemaer er så bare, som navnet antyder, opskrifter på, hvordan vurderingssvar automatisk skal overføres fra én tag-gruppe til en anden. Hvis brugeren så er enig med, hvordan transformationsskemaet har fortolket de gamle tags (som jo netop sagtens kan være tvetydige til at begynde med, så brugere kan sagtens have haft forskellige fortolkninger af dem), så kan denne godkende dette, og overførslen kan ske automatisk. Der kan så være tilfælde, hvor det er svært at overføre de gamle tags til de nye, måske fordi de gamle netop var tvetydige. Det kan så være, at nogen brugeres brug af de gamle tags kan overføres direkte på et nyt tag, men ellers må man jo bare blive nødsaget til at overføre vurderingerne manuelt (hvis man gider som bruger). Det kan også være, at et gammelt tag simpelthen havde en semantik, der blandede flere forskellige koncepter, og at disse koncepter nu er opdelte i forskellige tags i den nye grupper. Her er der så heller ikke andet for end at spørge brugerne, om de vil opdatere deres vurderinger til de nye tags manuelt. Dog vil sådanne semantik-opsplittende tag-opdateringer muligvis nok blive relativt sjældne ift.\ de mere simple tag-dokumentations-opdateringer, så selvom man ikke vil kunne bruge dem i alle tilfælde, så kan det altså måske stadig være det værd at implementere transformationsskemaerne på platformen.

Der skal selvfølgelig være en separat side/applikation, hvor brugere kan få overblik over taggene på platformen og deres relationer, og hvor brugere også nemlig så kan vurdere tag-relationer samt oprette nye imellem tags. Hvordan designet af denne side/applikation skal være, har jeg ikke så meget at sige om. Det lyder selvfølgelig meget cool umiddelbart, hvis man kunne få vist en graf visuelt, og få vist kanterne mellem grafknuderne (som så er selve tags'ne), men man kunne også sagtens bruge noget, der er mere nede på jorden, hvor hver tag har en side, og hvor kanterne/relationerne til andre tags bare listes på denne side (og hvor man altså kan følge et link til de pågældende tags samt vurdere den pågældende tag-relation). 
%At se selve grafen..


*Jeg kunne også overveje, om jeg skulle nævne muligheden for at lave ``under-tags'', hvilke f.eks. kan bruges til at implementere muligheden for underkategorier med. Meningen er så for det første, at en rating af et ``under-tag'' skal føre til en rating af det(/de) pågældende ``over-tag(s)'' med en eller anden forskrift. Apropos kunne man også bare nævne muligheden for generelt at kunne indføre afhængigheder imellem tags, så brugerne, hvis de samtykker til afhængighederne, vil få overført deres ratings til relaterede tags (med den givne forskrift), når de rater et tag, der har afhængigheder til andre tags. Ja, så disse muligheder er der. Og den anden ting, som så ligger bag idéen om ``under-tags,'' er at over- eller forælder-tags'ne så kan vises i tag-menuen, men hvor der så udover de normale funktioner også er en pop-ud-menu, hvor man kan se ratings'ne for undertags'ne. Dette kunne f.eks. være, hvis man klassificerede et spil som et `skydespil,' og at brugerne så kan holde over (eller klikke på) dette tag for så at se tags (nemlig ``under-tags''), der repræsenterer underkategorier af kategorien, `skydespil.'



%Tag-grupperinger (kort).
Det sidste, jeg vil foreslå omkring tag-relationer, kommer så her. Det handler om at bruge tag-relationer til at gruppere tags i tag-menuen. Det vil nemlig sikkert være meget smart, hvis tag-menuen kan opdeles lidt, så at ``advarsels-tags'' måske kan vises i én del-menu, og at ``emne-tags'' kan vises i en anden menu. Her kunne man nok komme langt med bare at lade platformen designe, hvilke nogle opdelinger, der skal/kan være i tag-menuen, og også lade denne stå for at kategorisere de forskellige tags. Men det kunne altså også være en mulighed, hvis man lod brugerne selv lave disse kategoriseringer ved at give dem mulighed for at indføre tag-kategorier (i form af tag-prædikater) og for så selv at stemme om, hvilke tags skal høre til hvilke kategorier. Og så vil den enkelte brugere selvfølgelig også bare selv kunne vælge, hvilke af disse kategorier skal optræde i tag-menuen. Fordelen ved at gøre dette mere brugerdrevet er, at der måske herved vil opstå flere muligheder, som den lille gruppe af udviklere for platformen måske ikke ville komme i tanke om og/eller måske ikke har tid til at udvikle (og de har måske heller ikke tid til at overvåge brugeraktiviteten så grundigt). Jeg tænkte også på, at ``rettelses-tags'' og ``kilde-tags'' osv.\ kunne blive nyttige kategorier, og med alle sådanne forskellige muligheder bliver det måske bedst, hvis brugerne selv kan drive den udvikling. 
Men ja, meget mere har jeg dog ikke at sige om dette, når det kommer til ressource-tags'ne. Når vi kommer til at snakke om kommentarer i det følgende *(lige efter dette), så bliver der dog mere kød på emnet, hvis vi betragter, hvad man kan opnå ved at gruppere kommentarer i forskellige menu-sektioner.

%%%%%Hm, bør jeg ikke bare gennemgå kommentar-idéerne her? Vil det ikke være mest oplagt? ..Nå jo, det var jo også planen, som den var nu her. Okay, det gør jeg.

Jeg har nemlig slet ikke nævnt, hvad man kan gøre med de kommentarer og/eller anmeldelser, der typisk vil være på de platforme, vi snakker om her. Kommentarer/anmeldelser er jo også en slags ressourcer i sig selv, og de kan ligeså vel indeholde udsagn, man kan vurdere og diskutere, og de kan indeholde NSFW-indhold eller spoilers osv. --- nærmest alt hvad de ``normale ressourcer'' på en sådan platform (vi kan kalde dem førsteklasses-ressourcer) kan indeholde. Mange nuværende platforme vil netop også sørge for, at brugere kan vurdere kommentarer, og jeg tror endda, hvis ikke jeg tager fejl (jeg tænker på Reddit), at nogle platforme også tillader, at kommentarer kan gives tags (men jeg er nu ikke helt sikker på stående fod --- det må jeg lige tjekke op på). Jeg vil så foreslå, at kommentarerne på den type platform, som vi snakker om her, hvor man har indført folksonomy-rating og muligvis flere af mine forslag, også skal kunne gives rating-tags ligesom førsteklasses-ressourcerne. Og noget, jeg som nævnt også tror ville batte rigtigt meget, specielt for kommentarerne, er, hvis man også\ldots\ %Hm, nå nej, det er jo en forskel her. Da vi snakkede om ressource-tag-grupperinger, da snakkede vi om faste kategorier, men her med kommentarerne snakker vi i stedet om mere specifikke kategorier, opfundet på stedet, som handler om kommentarernes semantik.. Hm, men relaterer det sig så dog alligevel til tag-relationerne..? 
Nå nej, det jeg har i sinde omkring at gruppere kommentarer relaterer sig faktisk nærmest mere til, hvad jeg har forklaret om at opdatere tags\ldots\ Ja, så man kan ikke helt sammenligne det, og det er derfor også ekstra vigtigt, at jeg diskutere, hvad man kan gøre med kommentarerne her for sig.

Men jeg kan så i første omgang nemlig nævne, at ved at kunne give kommentarer (rating-)tag ad libitum, så kan man bl.a.\ indføre noget tilsvarende det, jeg nævnte her for ressource-tag-menuen, hvor man således for kommentarfeltet kan oprette forskellige\ldots\ blade, vil det jo nok være, da kommentarfeltet typisk ikke vil være delt over, men vil være ét felt med en lodret liste af kommentarer. Så jeg forestiller mig altså, at kommentar-kategorier kan implementeres designmæssigt ved, at der kommer blade i toppen er kommentar-feltet, hvor man så kan skifte imellem de forskellige kommentar-kategorier. Og jeg forestiller mig så nemlig også, at disse kommentar-kategorier kan dannes på samme måde, som hvad jeg beskrev for tag-kategorierne. Et godt eksempel på en kommentar-kategori kunne være ``rettelser.'' Ofte sker det nemlig (f.eks.\ på YouTube), at når folk har rettelser til indholdet, så kan kommentarfeltet hurtigt fyldes op af disse rettelser, og i reglen vil der så her være tale om en masse gentagelser af samme rettelse. Dette gør kommentarfeltet mindre brugbart for brugeren, da det jo oftest kun vil være relevant for brugeren at læse hver enkelt rettelse højest én gang. Ved dog at kunne opdele kommentar-feltet i flere blade, så alle rettelser kommer ind under samme blad, så kan brugeren bare læse lidt fra dette blad, indtil vedkommende begynder at støde på gentagelser, og kan herefter gå over til andre blade, der ikke indeholder kommentarer (hvilket muligvis kunne være et blad (i.e.\ en kategori), der simpelthen ikke gør andet end at skille rettelses-kommentarer fra).

Så langt, så godt, men jeg vil også forslå en anden ting, man kunne gøre. Jeg vil foreslå, at man indførte en specielle kommentar-relationer, hvilket med ovenstående nomenklatur bare svarer til, hvad vi har kaldt tag-skabeloner, i hvert fald hvis vi nu bare udvider tags'ne, så de også kan tage kommentarer som subjekter og/eller som input (når vi snakker tag-skabeloner). Disse specielle kommentar-relationer skal så henholdsvis have semantikken, ``$\lambda x, y.$denne kommentar, $x$, har samme semantik som kommentaren, $y$'' og ``$\lambda x, y.$denne kommentar, $x$, er bedre forklaret i kommentar, $y$, og bør skjules under denne.'' Man kan måske endda også gøre ligesom for tag-relationerne og også indføre kommentar-relationer, der udvælger en gruppe af kommentarer, og så siger, at ``$x$ forklares bedre af og bør skules under'' denne gruppe. Pointen er så at bruge dette til at eliminere genganger-kommentarer ved at ``skjule dem under'' kommentarer, der forklarer tingen bedre. Og med dette forestiller jeg mig så bare, at pågældende kommentarer fjernes fra kommentar-feedet (uanset hvilket kategori-blad, man har valgt), og at brugere så bare kan se på kommentarer, når de har andre kommentarer ``skjult under sig,'' og så bare kan trykke på pågældende kommentarer for at få de ellers skjulte kommentarer vist. 

Lad mig lige bemærke, at selvom jeg godt nok nævnte, at man kunne have rettelser og kilder oppe i ressource-tag-menuen, så mener jeg dog, at det nok vil være mest praktisk, at fokusere på bl.a.\ rettelser og kilder i form af kommentarer. Dermed ikke sagt, at ressource-tag-menuen ikke godt kan indeholde de nævnte tags, men så skal det primært være for ting, der virkeligt har høj relevans for ressourcen. Ligeså snart vi når ned til bare et lidt lavere relevans-niveau, vil der for mange ressourcer hurtigt komme alt for mange ting, til at der er plads til det i tag-menuen --- som nemlig alt andet end lige bør designes mere kompakt end kommentar-feltet, der derimod kan være vilkårligt langt i praksis. Selvfølgelig kan man bare ændre, hvad man forstår ved et ``kommentarfelt'' og en ``tag-menu,'' og man kan også sagtens bytte om på deres udformning og/eller inkludere et salgs over-blad i kommentar-feltet, så man kan skifte imellem at se tags og kommentarer i feltet. Så ja, man kan gøre mange ting, men hvis man designer det på en oplagt måde, hvor tag-menuen er af begrænset omfang og befinder sig tæt på ressource-boksen, og hvor kommentarfeltet er nedenunder og kan strække sig vilkårligt langt, så giver det altså bestemt god mening at udnytte den plads der er i kommentarfeltet til f.eks. at have rettelser og kilder osv. 

Men det er altså klart at grænserne kan blive lidt mudrede imellem tags og imellem kommentarer, hvis vi både måske går med til, at kommentarer kan behandles som førsteklasses-ressourcer (og derved kan gives de samme tags ad libitum og kan få deres egne sider --- hvor man så igen kan kommentere på kommentarerne (i form af en såkaldt kommentar-tråd)) og/eller hvis man som nævnt giver mulighed for at tag-skabeloner kan tage vilkårlige tekster som input, som brugerne selv kan udforme til lejligheden. Det er således ikke svært at se, at kommentarer i princippet slet ikke behøver at være en separat ting til at begynde med, men at de bare kan implementeres via tag-skabeloner med tekst-input. Jeg tror dog ikke, at dette vil være det praktiske at gøre. Mange platforme vil allerede have et system omkring kommentarer, og brugere er desuden også vant til, hvordan dette fungere, så jeg ser ingen grund til at prøve at implementere kommentarer på en helt ny måde. Og selv hvis man synes, det ville være ``renere'' på en måde, eller noget i den stil, hvis man indførte kommentarer via tag-skabelonerne, så må man dog bare sige, at man altid ville kunne lave en transformation til dette ``mere rene'' system til hver en tid --- for dette handler jo i bund og grund bare om at gøre semantikken eksplicit omkring, hvad en kommentar er for noget, i stedet for at det bare er underforstået. \ldots Hm, ja, jeg ved jo godt, at dette nok ikke er særligt forståeligt for nogen (og jeg føler mig også selv en anelse på dybt vand), og det er også bestemt oppe i luften, om jeg overhovedet skal have nogen af disse bemærkninger med (bedre formuleret eller ej) i sidste ende, men indtil videre tror jeg lige, jeg venter med at udkommentere denne paragraf (og den forrige for den sags skyld)\ldots 

Hm, jeg tror altså bare, jeg vil vende tilbage til disse punkter senere, og lade dem stå som de er for nu. Jeg tænkte dog, at jeg gerne ville have noget mere om diskussion og argumentation med\ldots\ 


%Ah, kunne man ikke bare foreslå, at tag-skabeloner med tekst-input også altid bliver oprettet som kommentarer, når de dannes?! Og så selvfølgelig også bare med en automatisk tilhørende kategori (speciel for tag-skabelonen)..!!(?) ..Jo, sgu; hvorfor skille det ad? Så kommentarer kan bare være tekster med relation til den pågældende ressource generelt.. Ja, fint at blande det sådan (så vilkårlige kommentarer også kan eleveres til tag-menuen, hvis de bliver populære nok.)!.. Hm, men hvad så egentligt med at filtrere kommentarer fra (pga. advarsler) i tag-menuen..? ..Hm, er det bare at bruge højere-ordens-tags (altså prædikater) endnu en gang, så brugeren kan filtrere tags fra..? Tja, det lyder jo lidt indviklet, men man kan jo sagtens.. (Vi er jo allerede i et HOL-system..) ...Hm, eller også skulle man bare helt lade være med at blande det, det er også en mulighed..(?) ..Eller man kunne bare lade være med at vise tekster direkte i tag-menuen.. ah, men gøre så man kan klikke på dem og se teksten ligesom en kommentar, hvorved man så kan bruge de samme filter-indstillinger som nede i kommentarfeltet til at sløre kommentaren (hvilket jeg måske burde komme ind på, btw..(?)).. ..Ja, og i starten kunne man så bare have et link til kommentaren (og altså uden at vise den).. ..Hm, men jeg har da også næsten helt glemt det med at vurdere relevansen for et tag, har jeg ikke..?! Kunne dette ikke netop ske i form af tag-prædikater..?? ..Jo! (Og vi snakker altså relevansen for den specifikke ressource.) Nice..! ..Og angående om man skal blande ressource-tags-med-tekst og kommentarerne, så kan jeg bare nævne, at man kan gøre flere ting, og så give det som et forslag, at man blander det.. Ja, cool. ..Hm, mon ikke også, forslaget om at blande det, gør systemet mere åbent, hva? ..For så er det på en måde let (tror jeg) også at gå den anden vej og løfte kommentarer op som tags (og dermed altså som førsteklasses-borgere i dette semantiske HOL-system). Jeg tror nu ikke umiddelbart, at jeg ligefrem vil nævne dette, men det er stadig en god grund til at foreslå, at man blander tags og kommentarer således. 

\ldots\ Okay, jeg skal lige lappe denne tekst lidt ved at skrive, at jeg nu er kommet frem til, at jeg nok også bør slå et slag for tag-prædikater. Dette kan nemlig bl.a.\ bruges til at vurdere relevansen af et tag for et emne. Et tag kan jo sagtens være super relevant for en ressource også selvom at ratingen ikke er ved et ekstremt punkt på aksen. Her er det så, at man kunne se på antallet af vurderingsvar afgivet, hvilket rigtigt nok typisk vil være nogenlunde korreleret med relevansen, men slet ikke 100 \%. Det kan f.eks.\ være at nogle brugere gennemgår alle ressourcer i en række og rater alle de samme tags --- hvilket faktisk jo er rigtig god stil, for det eliminerer også den korrelation, jeg kommer til at snakke om (eller rettere har gjort, men det er lige meget) nedenfor omkring emnet 'brugergrupper.' Og i det hele taget kan man godt forvente, at visse brugere kan have en række faste tags, som de altid bedømmer, og dette vil endda være en god adfærd, som man bør fremhæve og tilskynde. Så en separat rating for hvert tag omkring, hvor relevant tagget er, ville ikke være dumt --- og især ikke, når vi også får "brugergrupper" oveni, som kan hjælpe med at vurdere relevansen\ldots Nå, men skal man så gøre relevans-bonus-ratingen særlig, eller skal man åbne op for vilkårlige bonus-ratings (altså i form af tag-prædikater)\ldots? Ah, men tags, der indeholder tekster (der kan handle specifikt om ressourcen) er jo et godt eksempel på, at tags selv kan have brug for advarselslamper, og således giver det altså god mening at foreslå, at tags kan tilføjes prædikater, som så kan rates i sig selv. Disse prædikater kan så vises som små mærkater på tagsne, og hvilke mærkater der vises, kan så afhænge af deres rating såvel som mærkaternes generelle popularitet. Brugeren skal så kunne trykke på de specifikke mærkater og få vist ratingen, hvor de selvfølgelig selv kan give input, og de skal også kunne klikke en hel pop-ud-menu ud --- og gerne med et "vis flere"-link i nederst --- hvor de kan se flere mærkater/tag-prædikater og kan rate dem hver især. 

%Følgende paragraf starter som en brainstorm (jeg har ikke tænkt på taxonomy rigtigt, før jeg skrev det her i det følgende), og så må jeg se om jeg beholder resultatet...
Ok, og jeg vil så også lige foreslå, at der bliver en taxonomy for kommentarer, som implementeres via og/eller i sammenspil med tag-skabeloner med (ét) tekstinput. Så brugere vælger altså en kategori (inkl.\ ``ingen valgt kategori'') for hver kommentar de uploader, og denne kategori skal så svare til en tag-skabelon af pågældende type. På denne måde kan de mest relevante kommentarer for ressourcen nu (ret automatisk) kunne komme til at blive vist i tag-menuen, hvis der ellers er plads at give af i denne. Og på den måde bliver kommentarer en førsteklassesborger i dette semantiske HOL-system, hvor det så bare er en konvention at vise dem i et særligt felt under den relaterede ressource --- og nærmest som noget ``mindreværdigt'' ift.\ ressourcerne, men i virkeligheden er de altså førsteklasseborgere i systemet. \ldots Hm ja, det er da værd at foreslå, at man kan gøre tingene på denne måde. 
Brugere behøver dog slet ikke at bruge taksonomien i sorteringen af kommentarbladende. Men det ville dog være smart, hvis man kunne lade taksonomi-kategoriseringen tælle op i sorterings-relevans-scoren for et kommentarblads sorteringsindstillinger\ldots\ Ja, og her kan man f.eks.\ foreslå en implementation, hvor sorteringsindstillingerne bare kan give et relevans-point-offset for hver taksonomi-kategori. 



*Hov, fik jeg nævnt, at man også skal kunne forespørge tags og kommentarer?\ldots\ Tja, det har jeg vel nok et sted, men ellers kan jeg også bare lige sige her, at tag-prædikaterne\ldots\ Nå nej, det må næsten være tag-skabeloner, man så op-rater\ldots? Af ja, for det svarer så netop til at op-vurdere en kategori! Fint. Så folk skal faktisk kunne vurdere\ldots\ Ja, men for en renere semantiks skyld, bør man så oprette et tag-prædikat, eller rettere et tag-skabelons-prædikat, der tager en tag-skabelon som input (som så kan repræsentere en kommentar-kategori/-type) og så siger ``i konteksten af ressourcen, $x$, så er kommentarer, som er opfylder prædikatet repræsenteret ved ``tag-skabelon'' (eller hvad man lige kalder det i sammenhængen; man kunne også bare kalde det en `relation,' eller et `prædikat,' hvor subjektet så implicit er selve den givne ressource), $y$, relevante at få vist for mange brugere og bør derfor efterspørges, hvis de ikke allerede er blevet udarbejdet og uploadet.'' \ldots Hm, dette vil jo være den ret semantiske måde at gøre det på, men måske bliver dette dog en smule indviklet alt i alt\ldots\ Ah, men man kan implementere det på en simpel måde! Man kan bare vise ``kategorierne'' eksklusivt over kommentarfeltet (og hvor jeg altså ligesom tænker dem som en række blade, man kan klikke på, hvorved man så kun får vist kommentarer, der hører til den kategori (ifølge en bruger-justeret algoritme for at beslutte dette tilhørsforhold)), og så kan disse kategori-blade bare få deres egen lille rating. Kategorier, der så er særligt populær for en specifik given ressource kan så selvfølgelig vises med prioritet i blad-rækken, og generelt kan en underliggende algoritme på platformen også regne på, hvilke kategori-blade plejer at være populære for forskellige ressource-kategorier, og så foreslå disse blade ofte for brugerne her, også selvom de måske endnu ikke er blevet vurderet for den specifikke ressource. Nice, det lyder da som en meget god løsning.

*Måden platformen kan forslå kategorier (og tags osv.), når den kan se, at de generelt er ret populære for emnet, kunne være ved at give dem et rating-offset (svarende til at platformen selv har givet nogle vurderinger til katagorien/tagget til at starte med). Endvidere kunne man endda måske gøre det, så at det faktisk er brugerne, der stemmer om og vurdere, hvad disse offsets bør være for forskellige tags, kommentarkategorier osv., når det kommer til en given ressourcekategori. 


%..Vi kan lige bemærke... *("Jeg ved godt, jeg har talt om rettelser i forbindelse med tag-menuen...")
%Husk at tangere argumentation og sådan i kommentar-paragraferne også.. %..Hm, det kan jeg måske netop gøre i forbindelse med, at jeg diskuterer, hvorfor kommentar-sektioner er vigtige... Hm.. 


%Husk: Jeg vil som nævnt gerne have at diskussioner, argumentation lige bliver nævnt her omkring kommentarerne.


%%%%%Annotationer kort.
Inden jeg går videre til noget helt nyt, vil jeg også lige hurtigt komme ind på muligheden for at tilføje ``annotationer'' til systemet. Vi har jo snakket om, hvordan kommentarer ikke bare kan være reaktioner på indholdet i en ressource, men også kan være bidrag til ressourcen i form af kritisk feedback eller tilføjelse af kilder osv. %(og jeg bør altså også i øvrigt gerne have nævnt noget mere om at diskutere udsagn indeholdt i ressourcen). ...
Man kunne sågar også gå et skridt videre og indføre muligheden for som bruger at tilføje annotationer til en ressource. Dette kunne bl.a.\ være i form af tekst-highlights med tilknyttet kommentar, hvis vi taler tekst, og hvis vi taler videoer, kunne det være i form af at opdele videoen i udsnit med emnetitler på (som man ser det flere steder nu, bl.a.\ på YouTube, hvor skaberne kan inddele videoen i segmenter). *(Uh, og det kunne også være at give og rette undertekster til videoer som en vigtig mulighed.) I et mere avanceret system (og altså måske lidt længere ude i fremtiden) kunne man endda også give mulighed for at tilføje rettelser, i tekst eller videolyd, billede (f.eks.\ censur-felter), osv. For en tekst kunne dette så være i form af såkaldte difs til teksten. Her kan brugere så selvfølgelig sætte en tæskel for, hvornår rettelser skal træde i kraft, i.e.\ hvor mange skal have vurderet rettelsen positivt eksempelvis. Meget af idéen giver nu sig selv herfra, men jeg bør lige tilføje noget omkring tekst-highlights og tekst-annotationer. Jeg forestiller mig nemlig, at man kunne gøre det, så at highlight-farven kan afhænge af rating-graden. Dette kan især være nyttigt, hvis man gerne vil påpege mulige fejl og mangler i en tekst; så kan en læser hurtigt få et overblik over, hvilke dele af en tekst, der er mest enighed omkring, og hvilke dele er mere usikre, kontroversielle, ikke-gennemgået, ikke-underbygget (med eksplicitte kildehenvisninger) osv. Nå ja, og grunden til, at jeg nævner flere ting her, er selvfølgelig, at man kunne have flere forskellige highlight-annotationer omkring flere forskellige ting. Angående ``ikke-gennemgået'' så kan man f.eks.\ sammen i fællesskabet holde styr på, hvor gennemgået (og altså hvor ``peer-review'et'') de individuelle tekstudsnit er af fællesskabet. Og ja, det kunne også være en highlight-annotationstype, der talte om, hvor korrekt-og-velskabt helt overordnet tekstudsnittene er (så altså lidt en aggregeret score over mere specifikke highlight). (Jeg tror jeg vil stoppe her, men jeg burde sørge for lige at dvæle ved emner så som diskussion/argumentation her.) Og angående f.eks.\ NSFW, så kunne man også bruge teksthighlights her, fordi man så kan vælge en farve-skala, der hurtigt går imod hel sort for highlightet, når ratingen bliver positiv nok. I øvrigt kunne man overveje noget tilsvarende for billeder, og muligvis overveje en ``highlight'' i form af en bluring af billedet.  


%*Jeg kunne muligvis starte på en sammenligning til det semantiske web her, og så færdiggøre emnet, efter at jeg har talt om `brugergrupper'\ldots? Ja, det er måske meget fint. Dette kunne så bare være en kort kommentar, der siger: Bemærk at dette system kan blive ret sammenligneligt med ``det semantiske web,'' for dem der kender dette begreb --- og eller kan man læse om det her og her: *kilder.* Dette er ikke helt tilfældigt, da denne idé faktisk udspringer af, at jeg har gået og overvejet, hvordan man mon kunne fremføre det semantiske web (da det ser ud til at have ligget ret stille længe). En del af det semantiske web handler om, at kunne søge effektivt på ting (giv et eksempel). Men der er også andre gode udsigter / muligheder med det semantiske web; ting jeg faktisk er en anelse mere interesseret i. Det kan nemlig også bruges til at samle og organisere nettet, når man browser det, så man let kan finde frem til alle de relevante kommentarer, rating (skal jeg jo nævne), kilder, relaterede links osv., når man iagttager en ressource på internettet. Jeg tror således på at min fremførslen af sådanne folksonomy-systemer, som dem jeg beskriver her, kan blive en vigtig udvikling, der netop kan føre alt dette med sig. \ldots Tja, eller nu får jeg faktisk lyst til bare at tale om dette til sidst i sektionen, men lad mig bare fortsætte kort her alligevel: ...Nej, jeg gemmer det.



%\newpage
%%%Brugergrupper:

Det næste jeg vil foreslå er, at man kan oprette det, jeg indtil videre bare kalder ``brugergrupper,'' på platformen, som så er helt eller kun delvist lukkede grupper, der kan fungere som anmelder-/menings-grupper og/eller som moderator-grupper. Pointen er så, at brugere kommer til at kunne benytte disse gruppers rating-svar, hvis de har lyst, så de ikke kun kan få vist den gennemsnitlige rating på tværs af hele brugerskaren som det eneste, men også kan valge at se aggregerede ratings over specifikke mængder af brugere, og måske med forskellige vægte på deres rating-svar. %grupper og/eller... %Hm, dette lægger jo meget op til sem-web overvejelser.. Er det mon lige før, at man skulle forklare om.. nogle af de andre ting først.. ..? ..Hm, ja, måske skulle man bare have feed-sortering, koordinatsystem-app og så brugergrupper til sidst; efter tag-relationer m.m. og endda efter kommentar-grupperinger..? ..Ah, lad mig stadig nævne feed-sortering der, tidligt.. Hm, men burde jeg så ikke bare beholde koordinat-app-teksten, hvor den er, og så bare udskyde brugergrupper lidt..? ..Jo, lad mig bare sige det for nu.
Lad f.eks.\ sige at nogle brugere melder sig på banen til at administrere en brugergruppe af moderatorer, hvis opgave samlet set er at sørge for at gennemgå en vis mængde af platformens ressourcer (med en grundighed de selv erklærer) og vurdere, om de eksempelvis er NSFW eller ej og/eller andre ting. Hvis så denne gruppe kan opnå medlemmer og/eller tidsressourcer nok til at gennemgå den lovede mængde af ressourcer grundigt nok, og hvis brugergruppen er gode nok til at holde spammers og trolls ude af sig, så kan brugere sikkert med fordel vælge at se denne gruppes svar på de specifikke ratings, i stedet for bare at se på gennemsnittet på tværs af platformen. 
Et andet eksempel kunne være en anmeldelses-/smagsdommer-/kritikker-gruppe, der giver sig selv som opgave at vurdere en anseelig mængde af ressourcer inden for et emne og/eller en kategori. Sådanne grupper kunne være sammensat på mange forskellige måder. Der kunne f.eks.\ være tale om en slags ``eksperter på området,'' eller bare folk med særlig erfaring og/eller interesse på området, og som har tid og lyst til at lave seriøse anmeldelser/vurderinger. Det kunne også alternativt være en gruppe, der løbende prøver at måle sig med stikprøver af diverse vurderinger fra den brede brugerskare, og som prøver at justere sine egen gruppe (evt.\ bare ved at ændre på ``vægtene'' i gruppen, men det kommer jeg tilbage til om lidt), så den passer bedst muligt med de gennemsnitlige holdninger. %Hvis der så samtidigt i sådan en brugergruppe er et vist (internt) krav om, at man aktivt skal vurdere ressourcer for at være med, ...%Hm, denne idé er da forresten slet ikke særligt god. For det vil stort set altid være næsten ligeså godt, hvis ikke bedre, bare at bruge den tværgående vurdering.. Hm, var der nu noget at komme efter i denne idé, eller har jeg bare ikke lige tænkt mig om nok?.. Ja, nej, der var vist ikke noget ellers i den tanke (så jeg udkommenterer bare).. %Hm, eller..? ...Ah jo, det kan give mening..! For der kan jo være et problem med, at mængden af folk, der vurderer en vis ressource, kan være korreleret med en vis holdning til ressourcen (eksempelvis at synes godt om den). Ja, og samtidigt kan man så netop også undgå folk, der vurderer, fordi de har et stake i ressourcen (og en bias dermed). Yes, ok. Indkommenterer igen..
Grunden til at dette kan være smart fremfor bare at se på den gennemsnitlige vurdering på tværs af brugerskaren, kan bl.a.\ være fordi, folks holdninger til en vis ressource godt i nogen tilfælde kan være korrelerede med sandsynligheden for, at de vælger at afgive en vurdering til ressourcen. Dette kan f.eks.\ forekomme, hvis fans af en genre også har det med at vurdere ressourcer fra denne genre oftere end resten af brugerskaren. Desuden kan det også forekomme, at folk med stake i en eller flere ressourcers popularitet og/eller ratings (positivt eller også negativt for den sags skyld), vil sørge for at vurdere alle disse, og således kan der også komme et bias i ressourcerens gennemsnitlige vurdering. Det kan sågar forekomme visse steder, at folk med stake opretter flere konti specifikt med det formål at ændre en vurdering i en retning væk fra den egentlige gennemsnitlige holdning. Alt sådan nogle ting kan man i princippet opnå at kunne undgå, hvis man får fremført brugergrupper, hvor medlemmerne inkluderes, så de er nogenlunde repræsentative for resten af brugerskaren, og hvor medlemmerne sætter sig for alle at vurdere ressourcer så meget som muligt i folk, så man altså ikke får omtalte korrelationer (imellem interesse/stake og hvorvidt man afgiver vurdering eller ej). For at sådanne typer af brugergrupper fungerer kræver det selvfølgelig, at de ikke prøver at være repræsentative for gennemsnittet af vurderingerne, og altså ikke måler sig op imod disse, for så kan de samme korrelationer jo bare snige sig ind alligevel. I stedet bør man altså enten bare optage medlemmerne på en tilfældig måde (hvorved der dog kan komme korrelationer ift., hvem der vælger at tilslutte sig sådan en brugergruppe), men ellers må man finde en eller anden god måde at lave stikprøver iblandt brugerskaren\ldots\ Tja, og her kan der jo komme lignende korrelationer, nemlig ift.\ hvem der vælger at deltage i stikprøven, men sådan er det jo bare. Man kan sikkert alligevel komme ret langt med de to nævnte fremgangsmåder, for korrelationer imellem lysten til at deltage og så interesser og meninger vil alligevel sikkert have langt mindre betydning, end de korrelationer man ellers kan forvente (at der er en risiko for), hvis man bare ser på de gennemsnitlige vurderinger.
%.."For at en sådan type BG fungerer, kræver det så... (stikprøve uden korrelation..)" (tjek)

Brugergrupper kan også stå for at komme med kommentarer og annotationer (hvis man har sådanne). Blandt andet kunne de stå for at tilføje kilder til tekster. Gruppen kunne også repræsentere en befolkningsgruppe/faggruppe/interessegruppe i en eller anden forstand og så stå for at give gruppens samlede mening omkring ting. Dette kunne være alt fra politiske holdninger i en debat til meninger omkring videnskabelige debatter til meninger omkring visse produkter og/eller deres indvirkning på omgivelserne (hvilket f.eks.\ kunne være en mijlø-bevidst gruppe eller en dyrevenlig gruppe og alt sådan noget). Og ja, grupper kan selvfølgelig også komme med sammenhængende anmeldelser og/eller rapporter omkring ressourcer (inkl.\ ressourcer der refererer til produkter). 

%Kan også komme med kommentarer. %..Og kan komme med links og kilder og meninger, rapporter (nævn dette!:)). *(Jeg vil ikke nævne "rapporter" endnu..)

Nå, og angående det mere tekniske omkring, hvordan brugergrupperne skal dannes og vedligeholdes, og hvordan ``stemmevægtene'' skal fordeles i gruppen, så handler det i første omgang om, at en gruppe brugere anmoder platformen om at oprette en `brugergruppe' i systemet, hvor denne gruppe er administratorer. Hver administrator får så en vis administrator-værdi i henhold til, hvad man ansøger om. Denne behøver slet ikke at være ligeligt fordelt. For eksempel kunne man have en person, der gerne vil administrere en gruppe på egen hånd samt en række andre brugere, der bare gerne vil støtte førstnævntes initiativ, fordi de tror på det. Når nu `brugergruppen' så er oprettet og administratorpointene fordelt, så kan der herfra for det første eventuelt være en måde, at ændre omtalte fordeling på og således også give administrator-point videre til andre brugere. Dette kan eventuelt gøres simpelthen ved at administratorer kan give deres administrator point væk. Man kunne også eventuelt tilføje, at andre administratorer (ved afstemning) kan vetoe enhver overførsel af administrationspoint (hvis de nu ikke har nær samme tillid til den potentielle modtager, som giver-administratoren har). Man kan selvfølgelig også finde på flere mulige systemer. Det næste er så, at administratorerne kan uddele stemme-vægte til andre brugere (som ikke behøver selv at have nogen administrator-point). Nå ja, jeg skal også lige have nævnt, at `brugergruppen' også skal have en tilknyttet funktionshjemmel, når den oprettes, og som administratorerne dog skal kunne justere lidt i løbende, hvis det bliver nødvendigt. Denne hjemmel skal bl.a.\ forklare, hvad formålet er med brugergruppen, og hvad medlemmer bør indvillige i at foretage sig, når de bliver medlemmer. Nå, tilbage til `stemmevægtene:' Her kan der også være flere måder at gøre det på, når man skal implementere, hvordan administratorerne kan fordele dem. Her kunne man jo gøre noget tilsvarende med, hvad jeg foreslog for administratorpointene, men problemet bliver så bare, at aktive administratorer, der gennemgår og optager mange brugere, så bliver nødt til at give disse brugere mindre stemmevægte, hvilket ikke nødvendigvis er smart, for de andre administratorer kan jo bare være mere dovne. Hm, man kunne så netop også i stedet gøre noget med, at administratorerne kan uddele et vist antal vægt-point om måneden eller noget i den stil\ldots\ Ja, jo, og så skal man vel bare kunne vælge at gøre vægte negative, så man enten kan uddele vægt-pointene med positivt eller negativt fortegn (hvor resulterende vægte dog aldrig kan blive negative, også selvom brugeren har et negativt antal vægt-\emph{point}). Ja, det lyder ret fornuftigt. På denne måde kan administratorerne nok fint styre brugergruppen tilsammen. Ja, så jeg vil i hvert fald foreslå denne løsning. De uddelte vægt-point resulterer så i førnævnte ``stemmevægte'' (ud fra en fast forskrift). Den almindelige brugers vil så have et stemmevægt-offset bestemt af brugergruppe administratorerne (hvilket altså afgør, hvor meget man vil inkludere den gennemsnitlige holding i rating-resultaterne). Dette offset kan f.eks.\ (og vil sikkert ofte) bare sættes til 0. Og hvis brugere får negative point, så skal stemmevægten altså gå mod 0 (hvis offsettet er positivt, og ellers er den jo bare 0) --- f.eks. således at den bliver lig 0 efter en vis tærskel. Og ellers er pointen med stemmevægtene helt generelt, at de giver en faktor for, hvor meget vedkommens stemme tæller med i ratings, når det kommer til brugergruppens rating-resultat. Og brugere skal altså så kunne skifte imellem flere brugergruppe, inklusiv den basale, hvor alle brugeres vægte bare er den samme, og også inklusiv vægtinger administreret af selve platformen (hvis den nu allerede har et system til at filtrere spammere fra), når de iagttager ratings på platformen. Og brugerne skal selvfølgelig kunne ændre på, hvilke typer af disse rating, i.e.\ hvilke `brugergrupper' er sat som standard mulighederne, og ellers skal de også bare kunne skifte valget for specifikke tags, de iagttager, imens de browser platformen (hvor denne ændring så bare er midlertidig, imens de iagttager ressourcen). Så ja, dette burde beskrive idéen om `brugergrupper' meget godt. Jeg kan vende tilbage, hvis jeg har glemt noget. 

*Jeg har nu indset, at brugergrupper også skal have mulighed for at have tidsafhængige vægte, både på sig selv, på brugere generelt (specifikke og samlet set) og på andre brugergrupper (for brugergrupper skal jo kunne bygge videre på andre, tidligere oprettede brugergrupper)! Dette kan både f.eks.\ bruges til, hvis nu ens tillid til en anden brugergruppe er tidsafhængig; hvis man f.eks.\ ikke kan lide, hvordan den har udviklet sig, eller hvis man ikke kan lide, hvordan den var førhen. Det kan også bruges, hvis man gerne vil vurdere noget, hvis kvalitet kan ændre sig over tid. Nice.




%Ah, man bør også kunne have "under-tags" (som i "underemner" for tags)..! *Hm, dette har jeg vist ikke skrevet om, har jeg?.. *Okay, nu har jeg bare nævnt tanken kort i det ovenstående. 




%%%Mere tværgående og muligvis decentralt.. (måske bare kort):
Nå, nu kommer jeg så til spørgsmålet om, hvilken ``platform'' det er, vi går og taler om; skal det være én platform eller mange, og, særligt, skal det i virkeligheden hellere være et system, man kan bruge på tværs af hjemmesider og platforme (hvor man så kan genbruge sine konti og indstillinger på tværs af disse)? Jo, der kunne såmænd være tale om én eller flere eksisterende platforme, der bare indføre et system a la noget af det, jeg har beskrevet, mere eller mindre uafhængigt af hinanden. Det kunne selvfølgelig også være en helt ny platform, man designer til formålet. Denne kan så være så bred som det skal være, og altså kunne inkludere så mange forskellige former for ressourcer som muligt. Men noget andet, der også kunne være ret smart, var, hvis man kunne have det lidt som en ``serverless'' (tror jeg, man kalder det) applikation, der kan benyttes af hjemmesider på tværs af nettet --- meget ligesom at man nu f.eks.\ ofte kan forbinde til Facebook (og dele og like'e ting) og andre SoMe-platforme på tværs af webbet. Især for hjemmesider, hvor man kan købe produkter, vil sådan en mulighed være rigtigt smart. Så kan man give sine vurderinger til folksonomy-netværket i stedet for bare direkte til den pågældende side, og, endnu mere vigtigt, man kan så bare logge ind i dette system i stedet for at skulle oprette en ny konto (og huske dem) for hver produktside, man besøger. Og det samme gælder i øvrigt for andre sider, der f.eks.\ bare viser billeder, videoer og/eller andet indhold. Her vil det også være smart, hvis man bare kan logge ind på en konti i folksonomy-netværk-systemet og give sine vurderinger via dette. Ydermere ville det så være overordentligt smart, hvis man altså så også kan hente ratings fra systemet og få vist på den pågældende tredjepartshjemmeside --- eksempelvis i form af den vante ``tag-menu,'' vi har snakket om (og man kunne sågar måske også få mulighed for at hente kommentarer m.m.). 
Så der er altså nogle forskellige muligheder her. Man kan jo også altid bare have en bred platform og bruge denne, og så bare tilføje links til produkter til de hjemmesider/platforme, hvor man kan købe dem. Desuden kan man, hvis vi forestiller os et system, der giver andre hjemmesider mulighed for at hente og vise tag-ratings'ne, selvfølgelig også have det som en blanding, hvor man altså både kan browse det alsidige system af ressourcer og rating på en passende hjemmeside/platform, men hvor denne platform så altså bare også giver andre hjemmesider adgang til ratings'ne m.m. 

*Jo, lad mig lige hurtigt understrege, at jeg tror, der er et \emph{stort} potentiale i en tværgående platform --- især endda hvis den er ret open source, det tror jeg kunne blive rigtigt godt. Og ja, selvfølgelig også især hvis ingen af de gængse platforme tager teten op og melder sig på banen. I så fald ville der være stor grobund for en ny platform, som virkeligt kunne vokse sig stor (og blive meget brugt og besøgt).  

%%%%Annonymitet..:
%Og når vi så snakker om platforme, der er alsidige, frem for bare at have en masse platforme til forskellige formål, så vil det så selvfølgelig blive ret vigtigt, hvis folk, har... %Vent, det giver da sig selv..(?) Ja, folk skal kunne have flere konti, og de skal kunne have konti, hvor deres identitet er skjult; selvfølgelig.. Hm, jeg kom jo nok fra et sted før, hvor jeg tænkte mere på mere åbne udgaver.. Ja, jeg kom fra decentrale systemer, det er derfor!.. For her skal man nemlig over en stor byrde i starten omkring at sortere spam og bots fra osv. Men det skal man ellers ikke rigtigt.. Men ja, så hvis man starter det decentralt, så kan man blive ret afhængig af, at brugere opretter offentlige konti, og så bliver dette omkring "annonymitet" noget, som man så skal huske at tilføje. Men det skal jeg altså ikke nu, det er der ikke behov for (vel?).. Ah, men vent, vi snakker jo annonymitet i brugergrupper, gør vi ikke.!..? Hm.. ..Jo, og noget med at gøre de enkelte stemmer annonyme for brugergrupper og kun vise det endelige aggregat.. Men er der behov for dette nu rigtigt..? ..Hm, men det giver nærmset lidt sig selv uanset hvad, for de bliver jo alligevel en ting man skal forholde sig til: Skal brugerne bare kunne se aggragetet, eller skal de også kunne gå ind på andre brugere og se deres samlede vurderinger. Og hvis man nu kan, så er det jo oplagt i det mindste så at give mulighed for, at brugere kan skjule visse af deres vurderinger. Ja, så det er lige før jeg slet ikke behøver at nævne dette; ingen gang som bare en kort kommentar (enkelt sætning eller to)..(?) Nej, det er vist fint bare ikke at have det med.


%%%Brugerdrevet ML..: *(Nej, ikke her alligevel.)

%Hm, og hvad så med brugerdrevet ML; er der noget, jeg så ikke behøver at komme ind på her? Og hvad skal jeg skrive om i det hele taget? Hm, det kan jeg faktisk lige tænke over.. For måske kan det bare ligge i forlængelse af 'brugergrupper' og så måske referere tilbage til de lidt vage tags..?.. ..Hm, lad mig bare skrive det efter afsnittene om sem-web uanset hvad, ikke..? Jo, lad mig gøre det sådan. I øvrigt \emph{kan} jeg faktisk nævne annonymitet i denne forbindelse, for ML-teknikken kræver jo netop, at man har data på brugeres vurderinger. 




%%%Sammenligninger med sem-web:
%\newpage
Bemærk at dette system kan blive ret sammenligneligt med ``det semantiske web,'' for dem der kender dette begreb --- og eller kan man læse om det her og her: *kilder.* Dette er ikke helt tilfældigt, da denne idé faktisk udspringer af, at jeg har gået og overvejet, hvordan man mon kunne fremføre det semantiske web (da det ser ud til at have ligget ret stille længe). En del af det semantiske web handler om, at kunne søge effektivt på ting (giv et eksempel). Men der er også andre gode udsigter / muligheder med det semantiske web; ting jeg faktisk er en anelse mere interesseret i. Det kan nemlig også bruges til at samle og organisere nettet, når man browser det, så man let kan finde frem til alle de relevante kommentarer, rating (skal jeg jo nævne), kilder, relaterede links osv., når man iagttager en ressource på internettet. Jeg tror således på at min fremførslen af sådanne folksonomy-systemer, som dem jeg beskriver her, kan blive en vigtig udvikling, der netop kan føre alt dette med sig. %Nævn i hvert fald, hvordan forbedringer af søgnings-brugbarheden kan komme efterfølgende (for det kan det vel..?), og nævn også mine brugergruppe-idéer kontra simpel FOAF og "tillidsfordelingsalgoritmer".. 
Kan det så også i sidste ende føre forbedret søgning på internettet med sig? Ja, hvis brugernetværket finder det det værd, at begynde at oprette mere basale tag-skabeloner, der kan repræsentere diverse ``triplet-relationer,'' som det (konventionelle) semantiske web bygges på, og begynde at uploade ressourcer, der repræsenterer simple termer (og som et eksempel her kan jeg gentage termerne fra eksemplet ovenfor), og så begynde at rate diverse sammenhænge imellem disse termer. Og jeg tror, at dette godt kan være rimeligt sandsynligt, at folk vil begynde på det, hvis de kan se og mærke, at dette arbejde kan lette søgninger generelt, så ja, jeg tror på, at folksonomy-systemer i sidste ende også kan føre resten af det semantiske web med sig, hvis ikke noget andet gør det i mellemtiden. 

Og angående, hvordan det ovenfor beskrevne system relaterer til det semantiske web, så vil jeg også godt lige nævne, at mine tanker om ``brugergrupperne'' muligvis kan erstatte til en hvis grad, og ellers blive en forløber til, de mere gængse beskrevne metoder, som er tænkt omkring at fordele ``tillid'' i netværk på det semantiske web. Og disse gængse teorier fokuserer nemlig særligt på, så vidt jeg kan se, at fordele ``tillid'' --- hvilket jo giver god mening, men man kan nu godt finde på eksempler, hvor det kan være godt at fordele andet end `tillid,' når man vil høre folks meninger, f.eks.\ kunne der være tilfælde, hvor nogen er mere berørt af spørgsmålet, og hvor deres mening derfor må tælle mere, hvis man spørgsmålet nu f.eks.\ handler om at nå til enighed om en beslutning. Men dette kommer nu ikke så meget pointen ved. Den store forskel, så vidt jeg lige ved (og jeg kan jo tage fejl), ligger mere i, mine 'brugergrupper' er en anelse mere centraliserede end de gængse metoder, der bruger FOAF-netværker og så lave tillidsfordelingsalgoritmer via disse. Sidstnævnte fungerer nemlig med end-useren i centrum og tager udgangspunkt i dennes knude og FOAF-netværket. Dette er \emph{virkeligt} en god ting på mange måder, og jeg er glad for, at normen (hvis ikke jeg tager fejl) er at sigte imod dette! Men jeg tror nu alligevel, at det er ret vigtigt for brugerne også at have nogle gode fælles referencepunkter, så at alle spørgsmål ikke bare afhænger af, hvem der spørger, men hvor brugerne kan vide, at de får de samme svar, som alle andre, der bruger samme indstilling. Så selvom mine `brugergrupper' sandeligt kan komme til (potentielt) at blive forløberne for de mere avancerede og mere decentrale algoritmer, så tror jeg faktisk også det vil blive ved med at være værd at holde fast i dem. Om man så holder fast i lige de former for `brugergruppe'-algoritmer, som jeg har kommet med forslag til her, eller om man vil bruge nogen andre til den tid, det må tiden jo vise. Jeg vil dog lige påpege, angående den løsning, som jeg har lagt op til her, at jeg tror, der bliver en stor værdi i at have sådanne meget simple algoritmer, som alle (stort set) kan forstå, hvordan fungerer (og hvor det altså er rigtigt nemt og hurtigt at læse sig til, hvordan de fungerer). Dette kan virkeligt være en værdi i sig selv, også selvom mere avancerede algoritmer kan blive mere ``brugbare'' i princippet (for man må jo ikke bare tænke på ``brugbarheden'' men også på ``brugeroplevelsen'').  

%%%Nævne måske web 3.1 (p-modeller)..:
Okay, herfra er det så bare nogle ekstra-ting, jeg vil nævne. For det første, nu hvor vi lige har snakket om det semantiske web, eller ``web 3.0,'' som det også kaldes, så kunne man måske for det første se mine idéer lidt som et ``web 2.1'' (eller ``2.5,'' hvis det giver mere mening, men nu vil jeg bare holde mig til ``$x$.1'' (det tror jeg nu, giver bedst mening pr.\ konventionerne)). Jeg kan jo om ikke andet håbe, at det bliver sådan, og at folk vil betragte det som sådant. Men hvad vigtigere er, så vil jeg gerne nu nævne, hvad jeg forestiller mig kunne blive, hvad man kunne (og måske kommer til) at betegne som web 3.1. Jeg tror nemlig på, at der bliver et ret anseeligt mål, som vi kan prøve at opnå, efter vi når til web 3.0, og jeg tror ligeledes, at det med tiden vil blive tydeligt, netop at dette skridt er så vigtigt. Jeg tror nemlig, at et stort næste skridt vil være at opnå et web, hvor %brugere 
%(..og det er ikke sikkert, at jeg vil have denne paragraf med, men nu færdiggør jeg den bare her, og så må vi se, hvor meget den giver mening..)
%kan stille et vilkårligt spørgsmål og få (hvis spørgsmålet altså kan forstås af sem-webbet) et svar med en tilknyttet sandsynlighed. `Sandsynlighed' er altså nøgleordet her for mig. Et sådant system vil kræve...
alle relationer i webbet kan gives en sandsynlighed (alt efter, hvem der spørger, eller rettere hvilke aksiomer/antagelser vælges for søgningen), og hvor brugere altså kan stille et vilkårligt spørgsmål og få (hvis spørgsmålet altså kan forstås af sem-webbet) et svar med en tilknyttet sandsynlighed. Okay, dette er jo egentligt ikke så svært at opnå i sig selv, for man kan jo bare udvikle en eller anden algoritme, som kan give dens bedste bud på sandsynligheder --- og man kan i princippet ikke fejle, hvis bare man er fedtet nok med sine sandsynligheder, for i princippet kunne man jo bare lave en naiv, men ikke ukorrekt, algoritme, der bare gav alle sandsynligheder 50 \% (hvilket jo på en måde er sandt, hvis algoritmen nu ikke har adgang til noget data om verden). Det store arbejde kommer altså i stedet simpelthen til at ligge i at gøre disse algoritmer mere og mere forudsigende (i.e.\ få presset sandsynlighederne mere imod ekstremerne, imens de dog stadig er korrekte og altså passer med det observerede (og kan klare at blive testet i forsøg)). Her bliver det så ikke nok bare at have tillidsfordelingsalgoritmer; for det første er det ikke tilstrækkeligt med et system, der bare bygger på folks tidligere svar, det er klart. Vi snakker jo ``det semantiske web,'' så visionen bør helt klart indeholde en vis grad af at maskinerne kan ``forstå'' spørgsmål, ikke bare fordi de kender dem og har hørt dem før, men fordi de kan udlede meningen af dem. Og denne tanke omkring ``web 3.1'' er så, at de (altså web-serverne) så ydermere indeholder teorier og afledte modeller (hvor f.eks.\ visse variable søge-antagelser kan sættes), hvor de så, ikke bare kan finde et svar, men kan finde et svar og en sandsynlighed på, at svaret er sandt. Og ja, jeg tror så, at der kommer til at ligge rigtig meget interessant arbejde i, at få dette til at fungere mere og mere optimalt, så maskinerne ligesom kan bruge al den viden vi mennesker tilsammen har om vores verden, til effektivt at opbygge sandsynlighedsmodeller (og -teorier (i.e.\ med variable antagelser)) over denne viden. (Og jeg gider så ikke at gentage det her, men jeg kan jo lige referere op til afsnittet om \textbf{Prædiktive ontologier} (som nok burde hedde ``prædiktive modeller'' i stedet\ldots) ovenfor (\ldots og måske har jeg også skrevet om det andre steder\ldots\ det kan jeg ikke lige huske helt), hvor jeg har nævnt nogle småting om, hvad dette arbejde kunne indebære.)

**(06.11.21)\footnote{Her lader jeg `**' betegne, at paragrafen er indsat efter at jeg egentligt har afsluttet notesættet.} Hov, jeg synes, jeg mangler at nævne brugernes aksiomer. Hele pointen med et prædiktivt semantisk web er nemlig, at brugerne kan komme ind med deres egne aksiomsæt, der giver opskriften på, hvordan sandsynlighederne kan beregnes, og at webbet så kan finde ud af at konvertere en forespørgsel samt et tilknyttet aksiomsæt til et svar med tilknyttede sandsynligheder. 


%(05.10.21) Jeg var ved at skrive om min ``wiki-side-idé'' i en ny undersektion herunder (nu udkommenteret her og i stedet flyttet til mit ``lille appendix'' nedenfor), men nu har jeg besluttet, at idéerne hører bedre hjemme som en forlængelse af denne ``folksonomy-idé.'' Så de tanker og idéer kommer altså bare her i det følgende i stedet. Og så har jeg også lige et muligt hængeparti i denne sektion omkring mine tanker om ``brugerdrevet ML,'' men dem kan jeg så lige skrive (muligvis bare helt kort) om til sidst her (inden jeg så går videre til først debat-SoMe-idé og derefter blockchain-idéer).
%
%Idéen går helt kort ud på... %Hm, nej vent, skal jeg ikke alligevel forklare det i en ny sektion..? Man kan jo også sige, at debat-SoMe-idéen også bliver en kort sektion.. Og jeg vil måske også dvæle lidt ved WoApps-idéen... ..Jo, lad mig stadig gøre det i sin egen sektion alligevel. ..Og så kan jeg altid se på bagefter, hvordan den endelige sektionsstruktur skal blive i min renskrevne tekst (det har jeg jo i det hele taget ikke rigtigt gået op i her).




*Lad mig lige formulere følgende omkring det semantiske web igen her: Det semantiske web indebærer en fremtid, hvor linked data for det første gør, at man bedre kan query'e søgemaskiner og finde præcis, hvad man leder efter, men også hvor samme linked data kan forbedre brugbarheden og oplevelsen af at browse ting på webbet, fordi alting er mere forbundet. Min pointe er så, at sidstnævnte nok er en realitet, som lettere og hurtigere kan opnås end de mere ``intelligente'' søgemaskiner. Jeg mener således, at det at bruge linked data til at skabe en bedre og mere brugbar browsing-brugeroplevelse kan udgøre et rigtigt fornuftigt første skridt, så at sige, imod det semantiske web. Når jeg siger ``første skridt,'' så skal det dog siges, at det selvfølgelig ikke vil ske som en pludselig overgang; det er noget, der vil udvikles og forbedres løbende, og i øvrigt kan udviklingen af smartere søgemaskiner (også ved brug af linked data) også sagtens ske sideløbende med. Pointen er bare, at forbedringen af browser-oplevelsen er noget, vi vil kunne nå frem til hurtigere, og at dette så kan åbne op for, at linked data bliver meget mere anvendt på webbet, hvilket så videre kan booste den efterfølgende udvikling af smarte, semantiske søgemaskiner. Så jeg mener altså, at det kan være værd at lægge fokusset meget på linked data til at forbedre browsing-oplevelsen som det første i sem-web-fællesskabet. I øvrigt kan jeg så også lige slå fast, at jeg jo mener, at min rating-folksonomy-idé kan være en af de ting, der sætter gang i den browsing-relaterede del af visionen, fordi man jo via rating-tags'ne kan begynde at få et system, hvor folk nemt kan tilknytte relevante links til ressourcer, og på en måde hvor man endda både undgår det potentielle problem med spam, fordi folk jo kun vil rate'e de gode links til tops, og hvor brugerne meget nemt vil kunne gøre brug af disse links, fordi linksene så automatisk vil poppe frem i brugerfladen, og uden at brugerne altså selv skal gå ind og klikke og taste for at finde frem til sådanne høj-relevante links (for hvis de er relevante nok, vil de selv propagere frem til toppen af feedet i den relevante menu).  




%%%Brugerdrevet ML..:
En af de sidste ting, jeg vil skrive om (afhængigt af, om jeg lige finder på flere små ting, jeg vil nævne), er så mine tanker omkring, hvad jeg har kaldt ``brugerdrevet ML'' (machine learning) i mine kommentar-noter, men som egentligt bare går ud på, at give brugerne adgang til (gerne anonymiseret) statistisk data omkring brugeres generelle præferencer i de samlede fællesskab, og altså særligt omkring de korrelationer, der kan være for de præferencer. Bemærk dog, at et komplet kendskab til alle korrelationer med høj nok præcision kan bryde anonymiteten for visse brugere, så selv vis brugerskaren bare får offentliggjort korrelationerne (og ikke den direkte data), så skal det stadig igennem et filter / en maske først, hvis man skal undgå anonymitetsbrud. I ML bruger man jo meget en teknik, hvor man finder frem til korrelationer i data, og det er altså dette, jeg mener, at brugerskaren bør få adgang til selv at gøre. Dette kan nemlig være en kæmpe hjælp %til at %...%Finde frem til præference-brugergrupper, finde frem til koncepter.. ..koncepter der kan udpege en samenhæng i kvalitet/type/art for ressourcer.. eller altså i folk selv.. ja.. 
til især at finde frem til nogen af de mere abstrakte sammenhænge der kan være imellem ressourcer, som kan være svære helt at sætte fingeren på. Ved at kunne analysere korrelationsdata kan engagerede brugerne så bruge dette til at finde frem forslag til nye tags, som beskriver koncepter omkring ressourcer, man måske ellers ikke vil få øje på eller tænke sig til selv. 

Det kan også være koncepter omkring vores psykologi som brugere, hvor man så kan bruge data til at hjælpe med at udpege tendenser i folk, man måske ellers ikke ville få øje på. Jeg har vist ikke snakket så meget om tags til at bedømme brugerne selv, men dette kunne jo være en mulig tilføjelse til brugergrupperne, nemlig at folk kan bedømme sig selv og hinanden med tags, hvorved dette i sidste ende kan komme til at indgå i vægtene. Dette vil faktisk kunne blive ret gavnligt, hvis man bare kan sørge for, at man gøre det rimeligt anonymiseret, så folk bl.a.\ ikke kan gå ind og se, hvad andre brugere mener om andre brugere. For hvis man i det hele taget kan dele, ikke bare ressourcer, men også brugere effektivt op i kategorier, så kan brugerkategorierne indgå som en ekstra (mulig) nyttig parameter i ens søgekriterier (ved at sige: ``vis i højere grad ting, der er populære for brugergruppe $x$,'' hvor brugergruppe $x$ så altså kan være designet ud fra en opdaget korrelation imellem brugeres præferencer). Hm ja, så pointen med denne paragraf er bare, at man også kan bruge korrelationsdata til at udpege og danne nyttige brugergrupper, man måske ellers ikke ville få øje på, samt at en mulig udvidelse til brugergruppe-teknologien er, hvor brugere også kan bedømme sig selv, og måske hinanden, med tags (hvor resultatet så kan indgå automatisk i den endelige vægtfordeling).

I øvrigt tror jeg, man kan tage låget lidt af denne ML-teknik og gøre den mere forståelig og tilgængelig for almindelige brugere, der ikke kender til linear algebra, hvis man sørger for at korrelationsvektorerne kan visualiseres som histogram-kurver over en mængde af relevante tags (hvor man så kan prøve at sætte de mest korrelerede tags sammen, så kurven bliver mest muligt blød og klokkeformet (i hvert fald for den nulte korrelationsvektor, hvis man ortogonaliserer dem)). I øvrigt er det ikke sikkert, at man her nødvendigvis har lyst til at finde frem til ortogonale korrelationsvektorer, og det er ingen gang sikkert, at man har lyst til at opnå et lineært uafhængigt sæt. For man kan jo sagtens forvente, at de \emph{virkelige} korrelationer (altså de psykologiske faktorer i folk, der giver anledning til korrelationerne) ikke er hverken ortogonale (ift.\ det pågældende tag-sæt) eller lineært uafhængige (der kan jo således godt være flere virkelige psykologiske korrelationer end der er tags i sættet). Virkelige korrelationer er pr.\ definition --- for jeg definerer dem som signifikante korrelationer, der skyldes en vis konkret og intuitiv (hvis man først kender og kan forstå den) parameter i menneskers psykologi, hvor vi har det med at variere meget --- mere intuitive, og derfor kan det være meget gavnligt, at holde sig til disse ``virkelige'' korrelationer, for så kan folk forstå dem intuitivt. Hm, dette kræver vist et eksempel. Lad os lege, at visse mennesker med et vist gen eller en vis barndomsoplevelse er mere glade for rød end andre, og at et andet gen eller en anden barndomsoplevelse gør visse mennesker glade for bær frem for frugter (et rent hypotetisk eksempel). Lad os også sige at de fleste bær er røde. Så vil en normal ML-teknik, hvor man finde korrelationsvektorer, især fordi den altid resulterer i ortogonale vektorer, ikke finde frem til lige præcis disse (``virkelige'') korrelationsvektorer, fordi de så ikke vil være ortogonale. I stedet vil den finde frem til mere obskure og mindre intuitive linearkombinationer af disse, som så er ortogonale. Når mennesker så skal analysere og forstå korrelationerne, så vil de altså virke mere obskure, end hvis man havde dem på den ``virkelige'' form. Og min pointe er så bare, hvis man har mulighed for ikke bare at smide automatiske algoritmer efter dataet, men har mulighed for at finde frem til de ``virkelige'' korrelationer via menneskeligt arbejde, jamen så er resultatet kun mere brugbart; der er bestemt ingen grund til at foretrække hverken ortogonale eller bare lineært uafhængige sæt af korrelationsvektorer frem for mere intuitive korrelationsvektorer (og som bunder i en mere konkret ting i virkeligheden). 


%"Når jeg nævner "brugerdrevet ML," så handler det bare om, at jeg tror, det ville være gavnligt, hvis brugerne selv fik adgang til (måske annonymiseret) data omkring, hvilke korrelationer, der er (i brugerholdninger osv.), og så simpelthen kan sætte deres hjerne på at finde forslag til, hvilke \emph{koncepter} disse korrelationer kunne skyldes. Ikke nok med, at man måske bedre kan finde frem til de egentlige underlæggende (virkelige) korrelationer (for de virkelige korrelationer behøver jo slet ikke være ortogonale, og behøver ingen gang at være lineært uafhængige), men man vil også bedre som bruger så kunne gætte sig til, hvor man selv lægger (hvis man har passende begreber for, hvad korrelationerne kommer af). Og ja, jeg tror også bare man vil lære en masse som fællesskab af at finde frem til korrelationer, da det jo siger meget om os og om vores psykologi. (Ikke at jeg behøver at inkludere alt dette i teksten; det er bare lige for at gøre rede for mine tanker. Jeg tror faktisk, at jeg vil gøre teksten om brugerdrevet ML-teknik (i.e.\ teknikken om at finde korrelationsvektorer) ret kort..)."




%%%Evt. appendiks:

%Hm, jeg har jo egentligt ikke snakket om, at brugere skal vurdere hinanden.. Nå nej, men det er jo netop alt det, jeg vil spare ved netop bare at holde mig til de langt mere simple 'brugergrupper.'..




\subsubsection[Wiki-side m.m.]{Videre til tanker omkring wiki-side m.m. (så dette er altså en undersektion til ``forfra igen igen''-sektionen ovenfor)}
Når jeg kigger tilbage, ser det ud til, at det er starten af ``\textbf{Ny tilgang\ldots}''-sektionen samt starten er første ``\textbf{Ny ny tilgang\ldots}''-undersektion (hvis man ikke tæller mine udkommenterede noter med\ldots), som opsummere tankerne indtil videre\ldots\ Ah, der bør næsten også være andre steder\ldots\ Nå, men lad mig bare opsummere dem igen her, så alligevel.

Hvis vi tillader os selv at bygge videre fra forrige idé, så kan man sige, at denne idé handler om en teknik til at samarbejde om at redigere tekster --- eller andre ressourcer --- i et stort fællesskab. Idéen gør så i høj grad også brug af, at brugerene skal rate tester/tekstudsnit (/ressourcer/ressource-udsnit) med semantiske prædikater og relationer (svarende til at rate ``tags'' folksonomy-idéen). Og idéen kan så ligeledes også med stor fordel gøre brug af min `brugergruppe'-idé (såvel som andre algoritmer til at fordele stemme- og/eller tillidsvægte til brugere i netværket). Udgangspunktet for denne idé er bare et andet; den handler om vidensdeling frem for folksonomies og brugeranmeldelser/-vurderinger. Jeg kunne derfor også have startet med denne idé, og så have introduceret mine idéer omkring at system, hvor brugere i høj grad giver vurderinger af prædikater/relationer (om end vi tænker på dem som ``tags'' eller ej), og om stemmevægt-fordeling og ``brugergrupper'' i denne sammenhæng. Og hvis jeg så forklarede om folksonomy-idéen bagefter, så kunne jeg altså have skåret hjørner her i stedet for, som nu, ved denne idé. Men nu har jeg altså valgt denne rækkefølge, og så vil jeg altså ikke gentage de samme pointer her, som jeg allerede har forklaret om i ovenstående sektion.

\ldots

(05.10.21) Nå, nu har idéen, eller måske rettere min tilgang til idéen, ændret sig en anelse. Med min nye tilgang kan jeg introducere denne vidensdelingsidé mere i forlængelse af folksonomy-idéen. Jeg er dog ikke helt sikker på, hvordan jeg deler det op i sektioner, så nu lader jeg det bare lige blive i denne sektion for sig. Jeg havde forresten noget mere tekst til denne sektion, men det har jeg bare udkommenteret i denne omgang (i stedet for at lave endnu en ``jeg starter forfra igen''-sektion). Jeg har i øvrigt også kopieret teksten ned i min ``lille appendix''-sektion nederst i dette (``general notes''-)notesæt (\ldots selvom man nok alligevel skal læse de udkommenterede noter undervejs, hvis man vil forstå min tankebane bedre\ldots). Nå, min nuværende udgave af min vidensdelingsidé kommer så her.

Når vi ser på, hvad ``ressourcerne'' i folksonomy-idéen kan være, så kan det jo bl.a.\ være tekster. Dette kunne f.eks.\ være forklarende tekster såsom Wikipedia-agtige artikler eller andre artikler. Det kunne sågar være hele lærebogstekster for den sags skyld. Nu vil jeg så beskrive, hvad jeg tror, man kan opnå ved at bruge rating-tags til sådanne tekster, og vil i den forbindelse også slå et slag for, at man begynder at bruge tag-rating til vidensdelingssider også. 

Ved at benytte rating-folksonomies til vidensdelingssider (hvilket enten kunne være eksisterende sider, det kunne være en ny vidensdelingsside oprettet med det formål, eller det kunne implementeres som en del-applikation til en alsidig version af de folksonomy-platforme, jeg har forslået ovenfor), kan man nemlig opnå meget større alsidighed på siden ved at have mange flere forskellige typer af artikler og andre tekster, men uden at der går uorden i det hele, fordi man sørger for at have det hele ordnet via tag-kategorierne. 

Eksempler på hvilke ``tekstprædikater'' man kunne gøre brug af på sådan en side%, ...%(kommer så her...)
%Husk at nævne forskellige standarder, hvis ikke ligefrem "rapporter" og det.. *Nej, det kan jeg måske godt nævne lidt senre i stedet.. Jeg kan i hvert fald godt uddybe det mere senere i teksten..
\ldots Ja, jeg har jo skrevet om sådanne eksempler før ovenfor, så det gider jeg ikke at gruble mere over her. Men det kunne altså bl.a.\ være hvor ``letlæselige'' tekster er, hvor ``uddybende'' de er, hvor ``velstrukturerede'' de er, hvor godt opgaver og faktabokse er tilføjet teksten, og hvor passende disse er i sammenhængen, hvor ``velunderbygget'' teksten er og hvor kontroversiel den er, kontra hvor meget den holder sig til områder og udsagn, der er meget enighed omkring. Og så videre osv.\ osv. Nå ja, og jeg kan også lige nævne, at de kan handle om, hvilke sprog de er skrevet på, samt hvilke forskellige konventioner, man bruger frem for andre. 

Jeg har så også noget særligt at sige om de links, der kan være indeholdt i en artikel/tekst. Her mener jeg nemlig også, man kan gøre brug af tag-ratingerne på en smart måde. I stedet for at give et link til en specifik artikel kan man nemlig opskrive et link i form af en række prædikater om, hvad den pågældende artikel skal indeholde (og eventuelt hvor ``dybdegående'' den skal være osv.\ osv.). Når brugeren så følger linket, bør denne så kunne vælge at gå direkte til den artikel med de bedst mulige ratings, både i forhold til relevans for de listede prædikater og også bare overordnede rating, eller i første omgang at få vist en liste over de bedste bud af artikler, der matcher linket. Hvordan denne relevans-score udregnes bør selvfølgelig kunne indstilles og justeres af brugeren selv. For eksempel bør brugere således kunne vælge hvilke sprog for artiklerne de kan være interesserede i, samt hvilke nogle artikeltyper, de er mest interesserede i, og hvilke ``advarselslamper'' de gerne vil være fri for. 

Den umiddelbare fordel ved dette er, at brugere så kan tilpasse artiklerne efter deres behov. Men en anden stor fordel, som jeg rigtigt gerne vil fremhæve, er også, at det så kan blive nemmere at samarbejde i fællesskabet om at udarbejde artikler. Jeg vil nemlig foreslå, at man så indfører en konvention/tradition i fællesskabet om at fokusere meget på at lave, hvad vi kan kalde ``dispositions-artikler'' (eller -tekster). Dette er nærmere bestemt artikler, hvis formål er at opskrive en vej i form af en række artikel-emner, som har til hensigt at gøre læsere kyndig på et område givet at de starter på et vist fagligt begyndelsespunkt. Lad mig forklare det bedre\ldots\ Disse artikler skal altså formulere en slags disposition for, hvad et hypotetisk værk (f.eks.\ en lærebog) skulle indeholde, hvis værket skulle løfte læserens viden / faglige kyndighed fra ét givent niveau til et andet. Artiklerne skal i øvrigt så også selv indeholde en beskrivelse af, hvad der forventes af læserens udgangspunkt, samt en beskrivelse af, hvilken kyndighed læseren kan forvente efterfølgende. Angående det sidstnævnte bør dispositions-artiklerne så også gerne kunne vedtage en minimumsgrænse for, hvor ``uddybende'' de indeholdte artikler bør være vurderet til at være, før at udsagnet om, hvad man opnår gælder, som det er skrevet. 

Der er nemlig så meget viden på internettet og så mange engagerede mennesker, der har lyst til at lære fra sig, så tiden er ved at være moden til, at vi begynder at arbejde sammen på at strukturere al den viden. Som det er nu kan man stort set slå alle atomare facts slås op på internettet, hvis man allerede har lært det før og ved, hvad man søger efter. Dette har i hvert fald været min erfaring med de områder, jeg har bevæget mig indenfor. Man kan godt tage online kurser, men de er som regel closed source. Vi mangler, mener jeg, at tage det næste skridt og begynde at lægge et fælles arbejde i lærings-stier/dispositioner/hvad vi kan kalde det. Dette arbejde, som jeg forestiller mig det, vil komme til at bestå i at lave en hierarkisk graf over lærings-/vidensstadier, som kan udbygges mere og mere løbende af fællesskabet. Vi bruger allerede implicit konceptet, hver eneste gang vi skal skrive en faglig tekst. Disse vil altid have en vis antagelse om, hvilket fagligt stadie læseren er på inden for fagområdet, så man ikke skal uddybe hver enkelt lille ting i teksten. Den graf (i.e.\ en struktur af knudepunkter med kanter imellem sig --- jeg kan jo ikke antage, at læserne til denne tekst ved, hvad jeg referere til med det samme, når jeg skriver ``graf''), jeg har i tankerne, handler bare om at formalisere dette. 
%(Uh, det bliver faktisk rigtigt godt, det her med at fokusere på disse tanker --- som btw er ret nye.) *(..!:)) 
I første omgang bør grafen altså %..Hm, lad mig lige se, bliver det noget problem at stadierne kan overlappe.. Nej.. Nej.
bestå af knudepunkter, der repræsenterer vidensstadier. Et vidensstadie skal ikke repræsentere en persones samlede viden om alt, men skal kun repræsentere en viden inden for et område. Grafen bør så bl.a.\ have visse rettede kanter, der siger, at knude B indeholder al kundskaben fra knude A (og mere til, hvis ikke de er identiske). Når man skal definere et nyt vidensstadie, skal man derfor bare oprette kanter til alle de videnstadier, der bygges ovenpå, og så skrive en definitionstekst, der også fastslår disse tidligere videnstadier, og så også indeholder en forklaring om, hvad man ellers skal vide/kunne for at være på det, ja, jeg burde vel kalde det et `kyndighed-/vidensstadie,' hvis det skal være mere omfattende. Ja, ok. Der vil så naturligvis kunne tegnes en hvis horisontal inddeling af grafen, hvor man deler knuderne lidt op i fagområder, men denne inddeling behøver altså ikke at være eksplicit; den kommer af sig selv på denne måde. Det interessante kommer så imellem disse vidensstadier: Hvordan kommer man fra et stadie og op til et andet? Det kan man jo gøre med et kursus og/eller en lærebog. Lad mig egentligt bare kalde det `kurser.' \ldots Ah ja, så jeg kan hermed også droppe at kalde det ``dispositioner.'' Nu kalder vi det `kurser.' Et kursus vil som regel bestå af flere dele, som på en eller anden måde relaterer sig til hinanden, som gør at det giver god mening at lære det hele på én gang. I princippet kunne man så tage hvert punkt i kursets forløb og sige, at det er et selvstændigt vidensstadie. Men man bør dog hellere, mener jeg, sørger for at sådanne intermediate-stadier for sin egen knudetype i grafen. Med denne adskillelse kan fællesskabet nemlig bedre fokusere på, hvad der er gode vidensstadier at arbejde med, når man f.eks.\ skal skrive nye fagtekster referere til, hvad er forventet af læseren, og så lade det ligge i et lag for sig, hvordan man så kommer fra videnstadie til videnstadie. Men det er dog ikke dumt også at tegne disse intermediate-stadier som knuder i grafen også, bl.a.\ fordi forskellige kurser kan bruge nogle af de samme mellemstadier, hvis de f.eks.\ starter fra samme stadie og divergerer på et tidspunkt, enten for at gå til to forskellige slutstadier eller for at samles igen på et tidspunkt og altså derfor bare udgøre forskellige rækkefølger af et kursus. 

Uanset om man deler kurserne op i stadier eller ej, og om man altså enten tegner en sti med en eller flere små knuder i sig, eller om stien bare består af én kant, så skal alle kanterne have en tilknyttet kursustekst, og alle eventuelle mellemstadier bør selvfølgelig også defineres med en forklarende tekst (ligesom at (de store) videns-/kundskabsstadier skulle det). Og så når vi altså det helt essentielle for graferne, nemlig selve kursusteksterne. \ldots Hm, det bliver måske lidt rodet med denne vinkel. Måske burde jeg forklare om kursusteksterne først, som nemlig godt i sig selv kan indeholde mellemstadier, og så bagefter forklare, hvordan man så kan sørge for, at kurserne kan tegnes op i grafen i form af eventuelle mellemstadie-knuder og (rettede) kanter imellem dem. Ok, det vil jeg gøre til den tid. Nå, men kursusteksterne er så enten mine ``dispositioner'' fra ovenfor. Hvis der så er tale om naturlige mellemstadier i kurset vil denne disposition altså så indeholde flere kursustekster, samt en forklaring på, hvad er opnået efter hver del (hvilket så altså udgør vores mellemstadier), og ellers kan der altså også være tale om bare en monadisk disposition, som kun referere til én artikel (selvfølgelig ikke direkte, men via prædikater i stedet), og som derfor ikke har nogen mellemstadier. Ja, og dette skal jeg nok få skrevet mindre rodet, når vi når dertil. 

Selve de enkelte kursusartikler skal så selvfølgelig typisk ikke bare være små Wikipedia-agtige artikler, men skal være mere uddybende og forklarende (i stedet for bare opsummerende), og skal gerne indeholde opgaver også. 

Hm, man bør vel næsten også kunne have forskellige niveauer til det samme vidensstadie\ldots? \ldots Ja, kunne man ikke lave en fast rang-skala, der kan gå fra ``har et nogenlunde kendskab til / overblik over området indeholdt i pågældende videnstadie'' og til ``er ekspert indenfor for videnstadie-området; kan løse opgaver omkring det effektivt, og kan endda lære fra sig af emnet.'' Så man kunne altså således have en rangstige fra 1 til 10, og så kunne kurserne også benytte denne skala til at præcisere niveauet nærmere. Et kursus der f.eks.\ går fra højt niveau til højt niveau vil så typisk benytte prædikater om, at artiklerne skal være meget uddybende og skal indeholde svære opgaver som en del af opgavesættet, imens kurser, der f.eks.\ går fra højt niveau til lavt niveau, ofte vil kunne klare sig med mindre uddybende artikler og mere overordnede og lette opgaver. Ja, det kunne altså være rimeligt cool sådan.

Brugere skal så kunne op- og nedvurdere alle disse dele af grafen, i.e.\ videns-/kundskabsstadierne og kurserne imellem dem --- og også nævnte kanter imellem vidensstadier, der beskriver, om der er overlap, og på den måde kan de mest populære og brugbare stadier og kurser vinde frem i grafen. Samtidigt kan brugerne jo op- og nedvurdere selve de enkelte artikler (samt diverse dif-rettelser til disse (se emnet ``annotationer,'' som jeg godt nok muligvis bør have uddybet noget mere\ldots)), som dermed også løbende kan udskiftes og forbedres for kurserne. 
Bemærk, at samme kursus også herved f.eks.\ kan implementeres i forskellige sprog, og i det hele taget kan tilpasses læseren på den ene og den anden måde, fordi kursusartiklerne bare skal overholde visse prædikater om deres indhold; selve formen på artiklerne kan sagtens holdes variabel. 

*Her skal jeg også lige nævne det smarte ved at have kursers disposition og så de individuelle kursusartikler opdelt i lag på denne måde, så de kan vurderes uafhængigt af hinanden. %Hm, lad mig lige tænke igen over det her med, om ikke det vil være godt og/eller nødvendigt, hvis kursusdispositionerne også så kan gives en score, som så kan afhænge af, om de forespurgte artikler er til rådighed.. Hm, men man kan jo rigtignok bare.. Tja, enten kan man jo bare arbejde med en standardudregning (bruger-justerbar self.) for, om en disposition er udfyldt-opfuldt, eller også kunne man bare lave et rating-tag, der siger, hvor udfyldt-opfyldt dispositionen er, og det må sgu næsten også være fint. Ja.. Simpelthen: Der bør bare være to ratings, og så er det fint. 


Så langt, så godt. Jeg vil nu videre påpege, at dette system ikke kun behøver at indeholde vidensstadier med konventionel, faglig viden, men bl.a.\ også kan indeholde ting som `forståelse af et vist synspunkt.' Dette kunne f.eks.\ være hvis man havde en eller anden debat, videnskabelig eller ej, og man aldrig kommer videre, fordi parterne måske taler forbi hinanden og ikke forstår hinanden. Noget der så kunne være gavnligt, er, hvis man så bruger tid på at udarbejde et kursus, simpelthen, i hvordan et givent synspunkt skal forstås. Kursets formål skal altså så være at gøre læseren/deltageren klog på, hvad pågældende part mener, bl.a.\ ved som et mål at kunne svare på, hvad parten ville sige til det og det spørgsmål. Et sådant arbejde vil også hjælpe med at finde vigtige skel i en såkaldt ``lejr,'' i.e.\ en tilsyneladende gruppe af samme overbevisninger, men hvor pointen altså er, at denne enighed altså godt lidt kan være en illusion. Dette kan nemlig også virkeligt ofte være en hindring for at to grupper kan mødes, fordi de bliver ved med hver især at få øje på ydrepunkterne i den anden lejr, og på den måde bliver ved med at skude den anden lejr i skoene, at deres holdninger (eller væremåde apropos) er langt mere radikale end de egentligt er. Ved derimod at finde frem til skellene inden for en gruppe, kan man således langt bedre nå frem til de gode og reelle diskussioner --- også internt i gruppen for den sags skyld --- og kan undgå rigtig meget forvirring og mudder og øv i diskussionerne. Og måden at opnå dette på kan altså være ved at behandle synspunktet lidt som et videnstadie, som jeg har beskrevet det her, bare hvor læseren/kursusdeltageren ikke behøver selv at blive enig i pointerne, men bare får kendskab til, hvad folk af pågældende ``synspunkt'' (som altså defineres som en lidt bred samling af meninger) mener og tror, samt hvilke udgangspunkter/bevæggrunde de kan have for at mene og tro det, de gør. Her vil det så selvfølgelig være mest relevant for brugere af selve synspunktet at være med til at udarbejde videnstadie-definitionerne og tilhørende kurser, og som tommelfingerregel bør man i hvert fald nok vedtage, at folk af synspunktet i det mindste bør være grundigt repræsenteret i arbejdet/redigeringen af kurser/stadie-definitioner, før at pågældende kan tages ordentligt seriøst af fællesskabet. 



%..At forstå synspunkter.. (tjek :))
%Det at kunne ændre.. tjek..
%Ah, mine "tre punkter" handlede alle sammen om programmering, så.. på når rating-highlights..

Hvis vi nu bevæger os lidt videre, så bør jeg nok lige komme lidt mere ind på emnet omkring ``annotationer'' og hvad man kan bruge dem til. Det er nemlig meget relevante, når nu vi snakker tekst-ressourcer. \ldots Ah og dog, jeg har faktisk nævnt det vigtigste ting ovenfor. Og det med dif-rettelser er også forståeligt, hvis man kender til sådanne på forhånd. Så nu kan jeg enten sige nogle ord om programmering og om ``web of programs/applications,'' eller hvad vi skal kalde det, eller jeg kan fortsætte omkring debat\ldots\ Lad mig gøre førstnævnte og vende tilbage til `debat' i en (her) ny sektion med udgangspunkt i min debat-SoMe-idé.

%Ej ja, hvor er det altså bare fedt med den her vinkel på det! Det gør det dejligt simpelt at forklare..!:D 

Nå ja, jeg kan dog lige bemærke, inden vi går videre, at kursusartikler selvfølgelig godt kan referere til ting uden for pensum. De skal ligesom bare ikke antage at læseren skal kunne noget, der ligger uden for dette. \ldots Hm, jeg har det som om, der også var noget andet, jeg skulle nævne\ldots\ Nå, hvis jeg kommer i tanke om dette, eller om andre ting, jeg bør nævne, så kan jeg indsætte det i en paragraf efter denne.

Nå ja, jeg har helt glemt ``fold-ud-links,'' som jeg har kaldt det (i noter i kommentarerne (altså udkommenteret)). Det var vist ikke det, jeg tænkte på, for det var vist ikke noget særligt stort. Men det er dette. Jeg vil gerne nævne, at %...%Hm, men nu er mine fold-ud-noter jo netop ikke særligt vigtige længere.. Jeg kom jo lidt frem til, at det kun var ved programmering og ved små nok artikler, som stadig indeholdte selvstændige sektioner, at det kunne være værd at gøre.. ..Hm, skal jeg så nævne det her eller i forbindelse med programmering? ..Lad mig bare lige nævne det her også.
visse typer af artikler, især vis de er relativt korte og overfladiske/opsummerende ift.\ emnet --- eksempelvis såsom Wikipedia-artikler --- godt kan indeholde ret selvstændige sektioner, men hvor den samlede tekst alligevel er lille nok til, at det er nemmest, hvis læseren kan se den hele på en gang, og at denne ikke skal klikke på en masse links for at læse sektionerne, især ikke hvis disse links leder væk fra siden. Noget man kunne gøre her er så at bruge, hvad jeg kalder ``fold-ud-links,'' hvilket altså er links, som forfatteren erklæres gerne må ``foldes ud'' automatisk, hvorved linket så at sige erstattes med den sektionstekst, der har den bedste rating ift.\ læserens indstillinger. På den måde kan sådanne artikler nemlig også udnytte uafhængigheden af dens sektioner, således at disse kan udskiftes og opdateres uafhængigt af hinanden, og ikke mindst uafhængigt af dispositionens/kursets overordnede vurdering, men altså stadig uden at læseren bliver tvunget til at klikke på en masse links for at læse artiklen. Hm, dette peger i øvrigt også på en ting, jeg lidt har manglet at forklare, nemlig det her med det smarte i at ratingen af kurser/værker bliver lagdelt: Brugere kan rate kursets disposition for sig, samtidigt med at de indsatte artikler kan rates hver for sig. Lad mig lige sætte det ind som en bemærkning ovenfor også\ldots\ 
Okay. 
Det vil så selvfølgelig ikke være dumt, hvis ``linket'' i form af prædikat-rækken stadig så bliver beholdt over den udfoldede sektion, så læseren alligevel kan klikke på det og få mulighed for at skifte til andre versioner af teksten. Der er i det hele taget flere forskellige designvalg, man kan bruge her --- man kan f.eks.\ også vælge om sektionsteksterne skal skjules først og så folde sig ud, først når brugeren trykker på sektionstitlen, eller om den skal foldes ud fra start af sig selv --- men det kan man jo alt sammen bare designe, som man vil. 

Noget andet, jeg kom i tanke om, imens jeg skrev ovenstående paragraf (og indsatte en bemærkning ovenfor), er, at kursers overordnede dispositionsdefineringer (hvilket altså netop af selve kursus-teksten; de indsatte artikler er jo selvstændige og uafhængige (og kan således f.eks.\ genbruges i andre kurser også)) bør vurderes med mindst to ratings: For det første skal et kursus jo vurderes ud fra, hvor godt det er givet at dets dele, altså ``kursusartiklerne,'' bliver udarbejdet, og desuden må der så også gerne være en vurdering, der siger hvor vidt disse kursus-artikler nu også \emph{er} blevet udarbejdet på det givne tidspunkt. \ldots Uh, til denne rating kunne man måske i øvrigt med fordel gøre brug af en speciel ``brugergruppe,'' hvis formål er at give vurderinger\ldots\ Nå nej, brugergrupper giver jo konti vægte, ikke specifikke vurderingssvar\ldots\ Hm\ldots\ Nå, never mind; det var egentligt også nok en bedre idé bare at bruge en automatisk udregning for, hvor udfyldt-opfyldt dispositionerne er (denne udregning behøver jo ikke at være særlig kompliceret --- slet ikke ligeså kompliceret, som jeg før har tænkt)\ldots\ Hm, men lad mig lige vende tilbage til dette emne igen senere, og så bare tilføje, hvad jeg kommer frem til i en (indskudt) paragraf lige her nedenfor.

Nå ja, og jeg bør netop også huske at nævne det med, at kursus-artikler jo kan genbruges i flere forskellige kurser. 

%(07.10.21):
Ah, og jeg bør forresten også lige sørge for at dykke lidt mere ned i, hvad man kan med tekstprædikaterne, og at man kan bruge dem til at opskrive alle mulige standarder for artiklerne. En standard kan så bestå af flere prædikater og brugere kan så vurdere disse prædikater individuelt for artiklerne. Prædiketerne kan så indgå i kursusdispositionerne (altså i kursusartikel-linkene), hvorved de så kan blive en slags tjeklister, for andre brugere der gerne vil udforme en artikel så den passer til pågældende disposition. Alternativt kan brugere selv påføre ekstra (form-)prædikater til kursusartiklerne\ldots\ Hm, hvad var min pointe med det her; jeg synes ikke jeg kommer til noget, jeg ikke allerede har forklaret\ldots? Ah\ldots\ Okay, jeg kan bare lige nævne, at det jo så ville være passende, hvis folk så også udarbejdede tekstskabeloner til de populære tekststandarder, så det er nemmere for forfattere at opfylde kravene, når de udarbejder teksten. Ja, og det er vel bare det, jeg lige kan sige.

Ah, angående det med både at vurdere selve kursusdispositionerne og så hvor udfyldte, de er, så tror jeg faktisk, at man simpelthen bør foreslå for ``brugergrupperne,'' der også kan sættes en tidsafhængig kurve for tilliden/troværdigheden. For hvis en brugergruppe f.eks.\ vil benytte en anden brugergruppe, jamen så kan sidstnævnte jo godt ændre sig med tiden. Og så vil det jo være smart, hvis man kan sætte en voksende eller aftagende kurve for, hvor højt deres stemmer skal vægtes ift., hvornår de er givet. Så det vil jeg lige indsætte ovenfor. Og så bliver det herved pludselig rigtigt nemt at implementere en udfyldt-opfyldt rating, for så kan man bare enten benytte særlige brugergrupper til formålet, der vægter sig selv med en kurve, der aftager, hvis man går bagud i tid, eller, endnu bedre, man kan bruge en separat brugergruppe, der har en bagud-aftagende kurve for alle og så bare at blande\ldots\ Nå nej, det afhænger af, om det er muligt for end-useren at blande vægte for forskellige brugergrupper (i.e.\ gange dem sammen)\ldots\ Hvor det måske så bare er nemmere, at bruge førstnævnte løsning. Ja, førstnævnte løsning var god nok. \ldots Så nu har jeg også indsat en lille paragraf om det ovenfor (ved `brugergruppe'-emnet).

Jeg bør også lige vende tilbage her, inden jeg går videre til programmering, til highlight-annotationer (som kan ændre styrke alt efter ratingen) og lige gentage (hvis ikke jeg simpelthen bare udskyder det til her (så skal jeg bare vende tilbage og snakke om video-annotationer her i stedet\ldots)), hvordan det fungerer, og hvordan man således kan annotere og rette tekster løbende. Her kan jeg så nævne, at man eksempelvis med fordel kan bruge en ``forståeligheds''-highlight-annotation, så man kan få overblik over, hvad der glider rent ind, og hvad man måske kunne forklare bedre. Og jeg kan også lige nævne, at folk jo så også kan smide link-annotationer ind til de steder, hvor der er refereret til noget, som det måske ikke er alle læsere, der forstår, hvad er. Og ja, selvfølgelig kan man altså også rette alle mulige fejl i teksten. 

*Det kan i øvrigt godt være, at jeg også lige skal sørge for at knytte lidt flere ord til, hvordan de personligt foretrukne prædikater kan indsættes i linkene, inkl.\ fold-ud-linkene. Bare så jeg lige husker det. 



%%%Programmering:
\phantom{\\}\noindent
\textbf{Programmering}\\
Nu har jeg dykket lidt ned i, hvad man kan gøre med tekster og et system med prædikat-/tag-ratings. Nu vil jeg så komme ind på programmer. 

%Jeg kan måske starte med overordnet at sige, hvor fedt det ville være, hvis.. ..Tja.. %Hm, jeg vil jo om ikke andet gerne sige noget overordnet om, at selve platformen måske kunne være open source og deles via sig selv.. Tja, eller hvad..? Der er jo ingen grund til at "genopfinde" open source.. ..Tja, men der kan jo ligge noget i, at man kan få et bedre overblik og/eller søge nemmere i programmer med rating-tags..(?).. ..Tjo, men det er nok lidt for avanceret; det bliver først, når vi når mere ind i web 3.0.. ..Det vil jeg summe lidt over til i morgen.
%(07.10.21) Okay, jeg starter bare med at fortælle om de tre ting, og så kan jeg nævne, at det kunne være sejt, hvis platformen så kunne danne ramme for et open source-fællesskab, og inkl.\ om sig selv. ..Hm, skal jeg egentligt nævne highlight-ratings noget mere ovenfor..? ..Ja. ..Gjort.

Jeg vil således foreslå, at open source-programmer også bliver en del af platformen (hvis altså ikke man vil oprette en særlig platform/side netop til dette, men som altså har de samme grundlæggende egenskaber, som jeg har beskrevet ovenfor, i.e.\ med tag-ratings, brugergrupper og highlight-annotationer osv.), så man kan gøre brug af både tag-ratings, hvor man altså kan rate helt specifikke kvaliteter ved kode, og ikke mindst så man kan gøre brug af ratede annotationer, inkl.\ ``highlight-annotationer,'' som jeg kalder dem.  

Angående highlight-annotationer kunne man så sikkert med fordel fokusere meget i fællesskabet på, hvor meget tillid er til, at de enkelte kodeudsnit virker efter hensigten. En sådan annotation bør altså tage et ekstra input, som redegør for, hvad hensigten er. Her kan man så dele ratingen op i to, så brugerne både vurderer, hvilken redegørelse passer bedst på formålet med kodeudsnittet, og altså også generelt hvor godt de passer, og derefter vurderer, hvor godt udsnittet lever op til dets formål. Målet er så, at hele kodebasen skal få så grønne (hvis det er den farve, man vælger) highlights som muligt (hvor meget energi og tid, man vil bruge på det, taget i betragtning) over det hele, og der hvor der så er knapt så grøn kan vil så netop typsik være der, man skal lede efter potentielle bugs. 

%*Jeg skal lige huske at præcisere, at annotationerne jo godt kan være i flere lag; man kan både annotere en hel funktion eller klasse (eller et helt program eller bibliotek for den sags skyld) ad gangen, samtidigt med at man også annoterer alle dets moduler hver for sig (hvis egne moduler hver især så også kan annoteres osv.). Når man så iagttager en hvis annotation (med et vist prædikat), så ville det jo derfor nok være en god idé, hvis man så kan scrolle imellem at iagttage forskellige niveauer. ... %Hm, jeg har allerede en hurtig, simpel løsning her, hvor man bare sørger for at fjerne mindre populære felter, hvis der er overlap, men hvad blev der egentligt af SCID-delen af idéen..? ..Ah, men det kan jeg jo nævne ifm. intentionel programmering-delen, og så kan jeg også bare vente med at uddybbe dette til der. 


Jeg ved ikke rigtigt, hvor meget eksisterende kollaborationsværktøjer går op i denne del, nemlig positive vurderinger af kode, og ikke bare ``negative'' (som jo dog er mindst lige så konstruktive) annotationer, der peger på fejl og mangler specifikt. Det kan også sagtens være, at det netop først er, når man virkeligt bliver mange omkring den samme kodebase (som eksempelvis i et stort open source-fællesskab på tværs af internettet), at en sådan form for fuldstændig gennemgang af koden (hvor man altså sørger for at annotere og vurdere al kode samlet set, også det man tror, virker) kan give god mening og vil være det værd. Det kan selvfølgelig også være, at jeg tager fejl, og at det heller ikke vil være tiden værd i sådan et stort fællesskab, men det tror jeg nu altså, det vil. 

Og ja, hvis det viser sig at blive en succes for et sådant bredt open source-fællesskab, så kan man jo så derefter så på, om det ville give mening at indføre samme konventioner for kollaborationsværktøjer, der kan bruges i closed source-fællesskaber/-firmaer. 


Bemærk at det med tiden så også kan blive nemmere at søge på kode-moduler, man står og skal bruge til et projekt, fordi de så alle er beskrevet med, hvad deres formål er. Dette vil så især gøre sig gældende, når først vi begynder at nå hen til et web 3.0, fordi disse formålsbeskrivelser så kan gøres semantiske, så søgemaskiner kan søge mere og mere præcist efter det. 
Man kan i øvrigt også hjælpe denne udvikling på vej på selve platformen ved at indføre kategorier for kodesemantik, hvilke man så kan indføre flere og flere af med tiden (bl.a.\ i form af underkategorier), samtidigt med at man så præciserer kategorierne mere og mere. Dette er jo i det hele taget grunden til, at jeg tror, mine idéer her kan være med til at fremme det semantiske web, nemlig fordi jeg mener, at hvis vi får mere og mere gang i tags, og særligt i kategoriserende tags, til ressourcer på internettet (hvad end det sker på en specifik platform, eller om det sker mere bredt), så kan disse kategorier også med tiden blive mere og mere semantisk præcise. Det kan så ske at søgemaskiner vil kunne bruge disse mere og mere præcise kategoriseringer direkte til at forbedre søgningen, og ellers kan det muligvis også ske, at det bare til sidst vil give god mening for fællesskabet, at formulere kategorierne i mere og mere semantisk formelt sprog, så søgemaskiner kan bruge det. Men selv hvis ingen af disse ting bliver tilfældet på den kortere bane, så vil ``søgning på nettet'' alligevel forbedres uanset hvad. For ``søgning'' behøver jo ikke kun at indebære, at man taster en forespørgsel ind i et søgefelt; søgning kan jo også være, at man selv navigerer hen til, hvad man leder efter, eksempelvis ved først at søge efter en overkategori og så følge links til underkategorier --- og måske kombinerer det med andre tags også undervejs --- indtil man til sidst indsnævrer emnet til, hvad det er, man søger på. 


Så langt, så godt. Det sidste, jeg så også vil nævne omkring programmer og programmering, er, at man måske også kan bruge ratings til at opdele programdesignet i lag ligesom i ``intentionel programmering,'' som jeg forstår begrebet. Her vil omtalte formålsprædikater så være dem, som designerne kan opskrive som det første, inden man går i gang med at implementere kodemodulerne. På denne måde kan man opdele kodebasen i lag, lidt ligesom vi så det for ``kurserne'' ovenfor, hvor man altså starter med en overordnet struktur (a la en ``disposition'') og så skriver med prædikater, hvad de forskellige moduler i den struktur skal indeholde. Og hver modul kan så selvfølgelig også selv være opdelt på denne måde. Nu kommer vi så også til endnu en ting, man kan bruge ``fold-ud-linkene'' til: Ved at indsætte kodemodul-prædikaterne lige der i den omkringliggende kode, hvor den pågældende kode skal indsættes, hvis de skal implementeres, og ved altså at notere, at de genre skal foldes ud automatisk, så vil læseren af strukturdesign-filen også kunne se de mulige specifikke implementeringer af denne overordnede struktur/disposition, i takt med at fællesskabet får kodet implementeringer af pågældende moduler. Og det endelige resultat kan så endda altså blive en sammenhængende kode, der implementere det overordnede formål med programmet, og som simpelthen kan copy-pastes direkte, når først alt er foldet ud. 

Så ja, jeg vil altså forslå en sådan lagdelt design struktur for programmerne også, tilsvarende det jeg foreslog for de forklarende tekster / kurserne. Forskellen er bare, at semantikken nu er en formel program semantik og ikke en forklarende semantik. Og hvis jeg så lige endeligt skal sammenligne med det gængse ``intentionelle programmerings paradigme,'' så ligger der nok også en forskel i, at denne udgave er decentral, og at implementationer ikke bare tilføjes direkte til deres (intentionelle) formålsdefinitioner af folk med privilegierettigheder til det, men at de i stedet rates ind af fællesskabet (på en decentral måde, hvor end-useren i sidste ende bestemmer, hvordan andre folks stemmer skal vægtes). 


%Og jeg kan så heller ikke dy mig for lige at nævne, at det jo kunne være ret sejt, hvis dette i fremtiden ligefrem kunne føre til 
%%Okay, lad mig lige brainstorme kort over denne sidste del. Det kunne være sejt med open source biblioteker, hvor man kan ændre i skifte (abtrakte) prædikater ude fra, og at dette så automatisk kan føre til kaskadende ændringer i koden, så den passer til brugerens præferencer. Ja..
%fleksible (open source) kodebaser, hvor man kan ændre på nogle udvendige prædikater, og at dette så kunne medføre an kaskade af ændringer i kodebasen, fordi %..Hm, skal jeg så egentligt dykke lidt ned i, hvordan prædikater kan.. propagere ned igennem dispositioner.. Hm, ja det må jeg vel lige se på.. ..Tja, men det bliver vel for avanceret; skal man ikke bare nøjes med, som jeg tænkte det for kurserne, at ændrede prædikater kan føre til ændrede dispositioner (men at prædikaterne ikke overføres), og så at brugeren dog kan indstille nogle overordnede præferencer, der godt kan ændre på hvilke (under-)dispositioner og tekster bliver valgt rundt omkring..? ..Ja, og så er det måske lidt for meget at drømme om, at denne form for intentionel programmering ligefrem kan ændre på, hvor nemt det bliver at gøre kodebaser mere fleksible; man må vel nok antage, at sådan fleksibilitet skal implementeres bevidst og manuelt..? ..Jo, det skal det vel.. ..Ja, jo, selvfølgelig.. Okay, never mind, så, men hvad skal jeg så skrive om dette?.. ..Tja, nok ikke noget, men ellers kan jeg jo bare lige tænke lidt over det, og så vende tilbage, hvis jeg finder på noget, der er værd at nævne.

%Og til sidst vil jeg bare lige nævne det ... %Hm, men giver det ikke også kun mening rigtigt at nævne det med at bootstrappe sig selv som platform, hvis jeg kunne pege på nogle lidt mere konkrete fordele ved at bruge platformen til open source programmering?.. Lige nu er mine pointer vist lidt for få og lidt for btw-agtige til, at det rigtigt giver mening at brygge videre på den vision, er de ikke..? ..Jo, og om ikke andet, så er idéen alligevel også trivielt oplagt; hvis man har en open source platform, hvorpå man kan dele open source kode, jamen hvorfor så ikke inkludere platformens egen kodebase i dette? Så ja, jeg vil nok ikke nævne det her, heller ikke. 

Jeg troede jeg havde nogle flere ting at sige om dette emne, men det er ikke sikkert, jeg har det alligevel. Jeg vil derfor gå videre til de næste emner, startende med mine idéer til en SoMe-platform omkring debatter. 



*(Tja, jeg skal måske bare lige sikre mig, at jeg får nævnt, hvor vigtigt det kan være, når brugere selv får lov at modde (lave mods til) en platform. For så kan brugernes præferencer bare nemmere blive mødt, også når nye ønsker opstår undervejs i brugen (så længe brugerfællesskabet bare er stort nok, så der er rigeligt med engagement og ressourcer at trække på). Jeg håber således også virkeligt på en open source-implementation af min rating-folksonomy-idé.)



%Husk også "bootstrapped"/"HTML..."..

%..Det kan godt være, at man allerede har nogle lidt tilsvarende muligheder med collaborations-programmer, men.. jeg kunne jo foreslå, at man så på, om det ville give mening for closed source fællesskaber/firmaer også.. (tjek)


%"rating-afhængige highlights, positive *(highlight-)annotationer, og intentionel programmering med ratings"





%%%Debat-SoMe-idéer:
\subsubsection[Debat]{Debat-SoMe-platform}
%(Fra brainstorm nedenfor:)
%"[...] Man kan også lave debatter, hvor hver påstand kan tilknyttes argumenter og modargumeter, lidt ligesom Kialo eller hvad, annotations-W3C-gruppen tænker, hvilket bestemt er gavnligt, men dog kan blive ret rodet i sig selv. Men jo, det er bestemt vigtigt med sådanne argumentgrafer, men de bliver først rigtigt gavnlige, hvis man altid sørger for at udforme sammenhængende tekster fra hvert relevante antagelses-(aksiom-)sæt/grundsynpunkt og/eller mere specifikke antagelser, der bedst muligt forklarer og redegører for alle relevante fakta for diskussionen (og selvfølgelig også dem som "modstanderne" har pointeret!), og diskuterer, hvor sandsynlige disse omstændigheder er ud fra de pågældende grundsynspunkter/grundantagelser og/eller specifikke antagelser omkring det diskuterede emne. Disse tekster behøver ikke at være udformet nødvendigvis rent af folk, der \emph{har} disse synspunkter/antagelser, men de \emph{skal} være skrevet nærmest som om de var, og de skal altså ligesom kunne godkendes af folk med det synspunkt. En sådan teksts mål er så, at redegøre for alle relevante argumenter og modargumenter og altså at give en overordnet sandsynlighed for, at fakta omkring det diskuterede emne er som det er --- og altså med andre ord sandsynligheden for, at fakta ville blive som det er eller "værre" (mindre sandsynligt), men man genafspillede tidslinjen på ny og ændrede tilfældige ting (så at tidslinjen ændres pr. kaosteori), så at pågældende fakta ligesom skulle genereres på ny, så at sige. Når andre så skal vurdere, hvor god sådan en redegørelse er, så kan de i første omgang kigge på, om alle relevante argumenter/modargumenter og alle relevante fakta er medtaget (og hvis ikke må man vurdere rapporten stærkt kritisabel), og derefter om logikken holder vand (og at konklussionerne altså ikke er helt ude i skoven --- og desuden at man ikke har sprunget let henover skridt i analysen, og ikke har prøvet at feje noget modargument til side uden at forholde sig ordentligt til det). Hvis alt dette er ok (hvad det gerne skal være; ellers må man skrive rapporten om igen), så kan man så i sidste ende se på konklussionen omkring sandsynlighederne, og hvis disse er helt vildt små kan man så konkludere at antagelserne er forkerte (men at rapporten stadig er god nok; rapporten er vigtigt også selvom man afviser hypotesen). Ok. [...]"
%
%Og så:
%"Okay, det næste er så, at jeg vil foreslå en debat-SoMe-platform. Denne kan bygges lidt ud af wiki-siden og/eller af folksonomy-idé-platformene samt de brugergrupper, der vil opstå her, eller den kan bygges fra grunden af.. Nå ja, eller den kunne også bygges ovenpå (og af) eksisterende SoMe-platforme. Denne platform skal handle om, at brugere kan tilmelde sig forskellige (selvstyrende) brugergrupper, som repræsenterer befolkningsgrupper, grundholdninger/grundantagelser(/grundsynspunkter) (inklusiv grupper med politiske orienteringer osv.), interessegrupper (inkl. faggrupper osv.), og hvad man ellers kan finde på af grupper med visse alignments. Disse grupper kan have bredt optag, eller have mere kontrolleret optag. De kunne sågar begrænse optaget til "eksperter" eller "interlektuelle" og/eller andre mere elitære ting (og ikke noget galt i dette!), så at medlemmerne altså skal bevise, at de har visse kundskaber/kvalifikationer for at måtte deltage. Og hvad gør man så i disse grupper? Jo man diskuterer for det første, og man udformer også de argument-rapporter, som jeg snakkede om her lidt tidligere i denne tekst. Nr. 2 punkt her gider jeg ikke at snakke mere om nu. Det.. ah, vent! Der er også et vigtigt punkt imellem disse, og det er nemlig, at de vædder med hianden! Så ja, de diskuterer for det første frem og tilbage på platformen om hvad som helst. Der er ikke nogen særlige krav til disse diskussioner, men man kan selvfølgelig med fordel prøve at strukturere dem lidt, så det ikke bare er lineære jeg-siger-du-siger-diskussioner, men at de opbygges i en grafstruktur med argumenter og modsvar osv. Og dette bør altså, som jeg ser det, bare foregå rimeligt frit, uden at folk kan trække deres gruppes/gruppers omdømme ned, hvis de lige for skrevet en påstand lidt hurtigt, uden at tænke sig ordentligt om (og også uden at spørge resten af gruppen til råds; det behøves ikke her). Det bør også være sådan, at brugere kan markere, at de ikke lægger noget bag argumentet, men i princippet leger djævlens advokat (også selvom de dog ikke \emph{behøver} at være uenige med det de selv skriver, når de markerer dette), når de udformer et argument som "det kunne have lydt (fra en person fra en passende overbevisning eller med et passende bias til at sige sådan)." Sådan aktivitet bør endda værdsættes af det samlede fællesskab; det er kun godt at få så mange brugbare argumenter på bordet som muligt, inden man går i gang med at analysere, og der er ingen grund til at de \emph{skal} komme fra folk, der også selv mener at påstandene er sande. Det er kun sundt med en pragmatisk kultur, hvor folk er gode til at diskutere tingene fra forskellige vinkler, og endda diskutere "imod sig selv," så at sige. Nå, men nu kommer den store pointe så. Det skal stadig forventes, at brugergrupperne løbende sørger for at erklære sig enige og uenige med påstande, og erklære sig eneige eller uenige med argumeter (i.e. i at argumeterne holder). Andre brugergrupper, der så erklærer sig af modsat holdning til et argument, skal så have mulighed for at udforde den anden brugergruppe, og sige: "Vil I virkeligt gerne stå inde for denne påstand / dette argument, og er I klar til at vædde "omdømme-point" (eller hvad jeg har kaldt "street credits" i mine papirnoter, men hvad man måske kunne kalde "reputation credits/points" på engelsk..) på denne påstand?" Idéen er så, at alle grupper til hver en tid har uendeligt/vilkårligt mange "omdømme-point" at vædde med, men at der på hovedsiden af platformen holdes øje med for hver måned, hvor mange point/credits de forskellige grupper har vundet/tabt samlet set. (På denne måde vil "værdien" af disse credits også bare være relativ til, hvad der normalt vindes og tabes på en måned.. Ah, eller man kunne godt kontrollere værdien, ved at sige fremhæve det, når en gruppe når over eller under en vis mængde credits tabt eller vundet på en måned; på den måde vil "værdien" ikke flukturere arbitrært meget, men der vil ligesom være sat en vis forventning til den til at starte med.) Men ja, og det er sådan set bare det. Så er der lige det spørgsmål om, hvem der skal bedømme, når en påstand kan afvises eller påvises. Her skal det nævnes, vi vi her ofte snakker om påstande, der giver en specifik forudsigelse om fremtiden (eller om, hvad man senere finder ud af om nutiden, men det kan vi jo også bare kalde "fremtiden," da dette jo så handler om fremtidig viden). Så meningen er altså at brugergrupper særligt skal udfordre hinanden på ting, som kan formuleres som en forudsigelse for fremtiden. De to væddende grupper skal så bare vælge andre grupper som dommere, som afgør væddemålet, når på et tidspunkt at nok fakta har set dagens lys til, at man mener, man kan afvise eller påvise forudsigelsen. Hvis der på en eller anden måde sker noget ulødigt med denne dommerproces, jamen så er det bare op til brugerskaren som helhel om at udskælde den pågældende dom, og notere hver især, at måned n i år x var der et ulødigt resultat for et væddemål imellem gruppe a og gruppe b, så deres gevindst/tab for denne måned, skal altså tages med en asterisk. Cool. Og det er sådan set det, der er idéen. Den lyder ret simpel, men jeg tror virkeligt, at der vil kunne blive en \emph{kæmpe} energi omkring sådan en SoMe-side som denne. ..Ah vent, det bør også i øvrigt fremgå, \emph{hvilke} andre brugergrupper, man har tabt/vundet penge til/fra, der der jo nemt kan blive meget forskellig prestige ift. dette. Desuden skal jeg også lige nævne, at en brugergruppe internt skal fungere ligesom de brugergrupper jeg vil (har i skrivende stund ikke gjort det endnu i teksten ovenfor) skrive om i teksten ovenfor, hvor der altså for det første kan være forskellige vægte til folk stemmer (altså hvor meget deres stemme vejer i beslutninger/vurderinger) og også forskellige moderator-/administrationsniveauer af gruppen (ift. hvem der skal optages og udsmides af gruppen, og hvordan omtalte stemmevægte skal fordeles)."


%Idéen går ...
%Ovenfor har vi snakket om, hvordan det, jeg har kaldt ``kurser,'' ikke behøver bare at handle om accepteret viden, men også kan handle om at forstå forskellige synspunkter til områder, hvor der er forskellige meninger. Processen om at udarbejde sådanne tekster, og diskussionerne dette kan medføre, kan som nævnt føre til en bedre forståelse internt i en gruppe en ``lejr,'' om man vil) for, hvor der måske er delte synspunkter, men hvad med diskussioner på tværs af grupper med forskellige synspunkter? Sådanne diskussioner er jo også stærkt nødvendigt, før man overhovedet kan nå frem til, hvad man selv mener, og hvad man kan være uenige om i første omgang. ...
Dette er en idé til en social medie-platform, der fokuserer på diskussion og debat. Det, jeg nævnte ovenfor omkring at udarbejde tekster (/``kurser'') til at indføre læserne i et synspunkt, er i sig selv, mener jeg, en virkeligt vigtig ting, der kan forbedre diskussionsprocessen på internettet. Men det giver først mening at bruge tid og energi på at udforme sådanne tekster, når man allerede er nået til en vis stilstand i diskussionen --- ikke fuldstændig stilstand, men nok til at parterne kan have en god idé om hver især, hvad udefrakommende skal forstå hvis de vil indføres i og/eller omvendes til deres synspunkter. Forud for denne proces bør der jo på en eller anden måde foregå fælles diskussion, så man kan få nogenlunde overblik over de relevante aspekter af diskussionen, og også over hvordan synspunkterne fordeler sig nogenlunde (så man har en idé om, hvor mange tekster, man skal gå i gang med i det samlede fællesskab).

%Her kan man selvfølgelig bare bruge de eksisterende platforme, men hvis vi tænker på Twitter og Facebook og sider i den stil, så er bliver debatten meget ustruktureret, og det bliver derfor meget svært at få et fælles overblik over argumenterne uden brug af nogen eksterne midler til at holde oversigt over diskussionerne. ...

\ldots\ Hm, denne idé bliver egentligt nok meget i forlængelse af min kursus-/wiki-idé, så jeg tror\ldots\ Hm, jeg tror faktisk, at jeg muligvis vil vente med at forklare om alt det med synspunkt-kurser til denne sektion, så at alt det, der handler om diskussion og debat, bare kommer med som indledning til denne sektion. Og jeg tror faktisk muligvis så, at jeg også vil vente med at udgive denne sektion, så den ikke er med i første udgivelse. For selvom idéerne i princippet er neutrale, så er der nok mange, der alligevel vil se dem, eller om ikke andet bare føle dem (underbevidst), som politiske. Og det vil jeg nemlig måske gerne prøve at undgå med min første udgivelse; så det bare kommer til at handle om idéer, man kan tjene penge på, man forbedre brugeres oplevelse-brugbarhed med, og/eller som kan være interessante videnskabeligt (og altså bidrage til videnskaben). Det eneste er så bare lige, om jeg nu så også skal overveje, hvor meget jeg skal have med omkring kilde\ldots\ Nej, never mind. Jeg kan sagtens forklare om, hvordan mit folksonomy-system kan gøre det nemmere at efterspørge og finde kilder, uden at det kommer til at føles som særligt ``politisk'' for særlig mange. Men ja, så for at vende tilbage til denne sektion igen, så skal man altså forestille sig her, at jeg lige har forklaret om synspunkt-kurserne og alt det.

Hvis ikke jeg har nævnt dette, så er det værd at bemærke, at ``synspunkt-grupper,'' hvis vi skal kalde dem det (altså de debatmæssige ``lejre'' så at sige), jo med fordel kan oprette ``brugergrupper'' (hvilket jeg jo bruger til at betegne en proces til at fordele stemme-vægtninger til brugere, administreret af nogle centrale administrator-brugere, der hævder at følge en vis hjemmel / et vist manifest, når de fordeler vægtene), hvis funktion er at være repræsentativ for et vist ``synspunkt'' (som med fordel kan defineres lidt vagt). Med andre ord kan man altså oprette brugergrupper, der forsøger at give brugere størst mulig vægt ud fra, hvor godt de argumentere \emph{for} (ikke imod) det givne synspunkt. Naturligvis bør et kursus, der forsøger at indføre læseren i et vist synspunkt, så gerne være vurderet godt af den brugergruppe, der repræsenterer synspunktet bedst muligt, hvis sådan en findes. Hvis ikke et synspunkt-kursus er vurderet særligt højt af den brugergruppe, der bedst muligt repræsenterer synspunktet, så er kurset ikke særligt meget værd (og det bør alle ærlige parter ligesom være enige i). 

Hvis vi så tager afset i disse synspunkt-brugergrupper, så ville det jo være oplagt, at disse grupper også jævnligt debatterer hinanden omkring emner og forhold, de kan være uenige omkring. Hertil vil jeg gerne %for det første foreslå ... %Hm.. Hm, kunne man foreslå brugergruppe-vægtede diskussionsgrafer..? ..Hm, men kommer der.. ...Måske kunne man lave en blanding, så man afholder en diskussion, som godt kan læses i kronologisk rækkefølge.. Hm, måske bliver det lidt rodet, jeg tænker altså, at der både kunne være en mere kronologisk, og så en graf-baseret måde at stille diskussionen op på.. ..Hey, hvad med at man bare kræver, at de diskuterende parter for hver "tur," ligesom, hele tiden skal sørge for også at give et direkte svar, der opsummerer argumenterne imod det forrige svar..!.. Så man altså hele tiden ligesom sørger for at lave delkonklussioner imens man diskuterer, hvor man opsummerer, hvad der er ens samlede svar på modpartens tidligere svar.. Og jeg tænker så noget mere ift., hvad man gør med forgreningerne og med nye rødder, man det kan jeg lige vende tilbage til.. ..Hm, og kan der også være dommere og point og sådan noget undervejs..?.. ..Ah, men man kan jo måske bare sætte alle reglerne ekstart. Diskussionerne kommer så til at involvere mindst to diskutterende parter i form af brugergrupper samt et vilkårligt antal dommer-brugergrupper, som de diskuterende parter kan have sagt god for eller ej, og som så står for at give point undervejs samt sørge for at de diskuterende parter holder sig til visse regler og retningslinjer, som de har indvilliget i. ..Lyder umiddelbart nice..! Og så vil jeg altså foreslå, at man sørger for i disse diskussioner ikke bare at blive ved med danne nye forgreninger, men for hver igangværende diskussion (inkl. undergrenene til en vis grad) hele tiden sørger for at bygge på hovedtråden, hvor man skiftes til at opsummere, hvad man føler, man har besvaret/konkluderet med de seneste posts/opdateringer (til diverse deldiskussioner i diskussionsgrafen), og hvor man så anser diskussionen for at befinde sig nu. Og ja, jeg kunne så godt foreslå, at man ligesom går på tur og skiftes med at poste opdateringer i de to diskuterende grupper. Hm, måske bør man så have et begrænset antal "overordnede tråde" (inklusiv hovedtråden/ene, hvor man startede fra) i gang på samme tid, som så altså defineres ved, at de diskuterende parter for hver tur, når de oploader posts til undergrene af denne tråd --- eller hvis de bare vil nøjes med at oploade et svar direkte på modpartens forrige konklusion --- altså forventes at oploade en konklusion på, hvad de har tilføjet, og hvor de mener diskussionen så er nu (f.eks. ved at pege på, hvad modstanderen mangler at besvare i deres øjne og/eller hvilke konklussioner man nu bør kunne drage af argumenterne). Så alle disse "overordnede tråde," eller hvad vi skal kalde dem, forventes altså, hvis de er aktive, at vokse med mindst én post pr. "tur" (som altså så bør udgøre en slags delkonklussion). Og når jeg her altså så taler om en "tråd," så mener jeg således en slag "hovedåre" i en graf. Vi kan således se en diskussion som en graf, hvor kanterne altså så bl.a. kan repræsentere "x er et modargument til y" eller "x er et argument for y," ligesom i gængse "argumentationsgrafer," men hvor der så bare i disse grafer også findes en særlig kant beregnet til disse "hovedårer" i grafen, som repræsenterer "x er en delkonklussion over de svar, der blev postet i denne tur, som svar til de post, som y har til hensigt at udgøre en delkonklussion over." ..Nice nok.. Jeg tror nemlig, det er ret vigtigt, at man hele tiden sørger for at lave sådanne delkonklussioner, hvis man skal indgå i sådanne ellers ikke-lineære diskussioner. Diskussioner skal selvfølgelig så kunne vises "konologisk," ved at brugerne kan bladre frem og tilbage i tid og så de forskellige stadier af diskussionsgrafen. Jep, tror altså, det er ret nice, det her..! 
%(15.10.21) Hm, jeg bør også lige nævne, at dommere gerne skal "straf-point," hvis de synes at del-konklussionerne tager munden for fuld for tidligt, og/eller hvis de generelt bare er for gentagende; det skal helst være sådan, at læserne kan få noget ud af at læse "hovedåren" af delkonklussioner i diskussioner, og altså har lyst til dette. Så det nytter ikke noget, at delkonklussionerne hele tiden konkludere for overordnet på tingene, og at de derved også så hele tiden kommer til at gentage sig selv.
foreslå, at man afholder formelle diskussioner i form af argumentationsgrafer, som to (eller flere) brugergrupper så går med til at uploade argumenter til på tur. Sådanne argumentationsgrafer er altså væsentligt anderledes tekstobjekter, end hvordan ``kurserne'' kommer til at være. Idéer og teknologi omkring ``argumentationsgrafer'' er bestemt ikke noget nyt, som jeg har fundet på. Hvis man søger på ``Argument Web'' på internettet, vil jeg mene, at man hurtigt vil kunne komme frem til, hvad jeg snakker om. Som sagt indebærer min idé her så, at de involverede parter (som altså kan have form af mine ``brugergrupper,'' som jeg introducerede ovenfor i forbindelse med min rating-folksonomy-idé) i et eller andet omfang sørger for skiftes til at uploade argumenter til den pågældende argumentationsgraf. Noget særligt, jeg så vil foreslå til disse argumentationsgrafer, som sikkert ikke er en gængs idé, når det kommer til de former for argumentationsgrafer, der har været udviklet hidtil, og som er grunden til, at jeg gerne vil have at parterne går på skift mere eller mindre, er, at der skal være en ``hovedåre,'' som jeg kalder det, af delkonklusioner i hver af de ``større diskussioner,'' som udgør argumentationsgrafen. Og før jeg forklarer, hvad sådan en ``hovedåre af delkonklusioner'' indebære, så lad mig forklare, hvad jeg mener med ``større diskussioner, som udgør argumentationsgrafen.'' Når to (eller flere) parter aftaler at indgå over for hinanden i en sådan formel diskussion, så vil der som regel være én, eller i hvert fald et fåtal, af overordnede spørgsmål/udsagn, man gerne vil diskutere. Dette kan vi se som ``hovedsætningerne'' i diskussionen. En diskussion kan så afføde en masse mindre diskussioner, hvorfor vi kalder det en argumentations\emph{graf}, og ikke bare en ``argumentationstråd'' eller lignende. Nogle af de affødte diskussion kan dog godt være så store, at de kan føles næsten ligeså vigtige som hovedsætningerne. Vi kan se sådanne som en slags ``lemmaer'' i diskussionen. Og derudover vil der så også være andre små diskussioner, som vi bare kan se som ``propositioner,'' hvis vi skal blive i matematiksammenligningen. Min pointe, som jeg altså kunne forestille mig er ny (men jeg ved det dog ikke; så grundigt har jeg nu ikke studeret de gængse forslag til argumentationsgrafer), er så, at alle sådanne diskussioner, hvis de skal kunne læses og forstås bagefter uden alt for meget besvær, og altså uden at det hele bare bliver en rodet bunke af argumenter på argumenter, bør have en passende mængde delkonklusioner undervejs, især når det kommer til ``hovedsætningerne'' og til ``lemmaerne.'' Og for dette vil jeg altså foreslå, at der bliver en konvention for at have en form for ``hovedåre'' i disse diskussioner, hvor man løbende laver oversigter over status på alle de mindre deldiskussioner, samt bringer de ubesvarede og/eller kontroversielle punkter (imellem de to grupper) op fra dybet og frem i lyset, så at sige, ved altså at fremhæve dem oppe i denne ``hovedåre.'' Så ``hovedårerne,'' eller hvad vi skal kalde dem, er altså kronologiske rapporter over diskussionernes stadier, og det er så nemlig meningen, at disse statusrapporter, disse ``delkonklusioner,'' for så vidt muligt skal kunne læses af brugere, inkl.\ udefrakommende, og give dem en god oversigt og forståelse af diskussionen forløb. Disse delkonklusioner bør så selvfølgelig meget gerne referere direkte til de deldiskussioner de (del-)konkluderer på, så at læseren får de relevante referencer i form af link, denne kan klikke på og følge. 

For at der kan være sådanne ``hovedårer'' i diskussionerne er det så selvfølgelig vigtigt at have en speciel kant iblandt byggestenene til argumentationsgrafen, som siger: ``Delkonklusion $x$ er en opsummering (til tiden, $t_x$) af de deldiskussioner der har været siden delkonklusion $y$ (og altså siden $t_y$).'' Jeg har puttet nogle parentetiske bemærkninger ind omkring tiderne $t_x$ og $t_y$, som altså betegner det timestamp, som man med fordel kunne knytte til hver af disse ``delkonklusioner.'' 

Nu snakkede jeg godt nok om, at de diskuterende parter kunne skiftes til at uploade argumenter, og hvorved det så nemlig også var tanken, at de afsluttede hver tur ved at give passende delkonklusioner, hvor det er relevant, men alternativt kunne man altså også bare gøre det sådan, at man opdeler diskussionen, ikke i ture, men i runder, hvor begge (eller alle) parter kan uploade argumenter i løbet af runden, og hvor begge parter så forventes at give deres version af de delkonklusioner, der er relevante for den aktivitet, der foregik i runden. På denne facon kunne ``hovedårerne'' så passende bestå af op til to delkonklusioner i hvert ``link'' i kæden/åren, så læseren, der følger hovedåren, for hver runde kan læse delkonklusionen fra hver part i stedet for at læse en række af delkonklusioner, der (mod-)svarer hinanden. Jeg kan ikke selv afgøre, hvilken én af disse idéer, der er bedst; der er bare noget, man må finde ud af ved at prøve sig frem, tænker jeg.

Angående de helt små diskussioner, som man ikke betragter som havende ``lemma-status,'' så kan man se i hvor høj grad, det giver mening at have konvention for også at lave delkonklusioner for disse. Det vil dog, uanset hvad, sikkert ikke være dumt, hvis konventionen er altid at afslutte sådanne deldiskussioner med op til to (eller hvor mange parter nu end er involveret) udlægninger af en konklusion over hver af disse. 


Så langt, så godt. Så vil jeg også gerne foreslå noget mere hertil, og det er at tredjeparts-brugergrupper skal kunne meldes på sådanne formelle diskussioner som en slags dommere over diskussionen. (Hm, jeg bør forresten faktisk næsten kalde mine ``brugergrupper'' for ``vægtede brugergrupper;'' det virker faktisk allerede rimeligt rammende, bare med denne lille justering.) Formålet med sådanne ``dommer-brugergrupper'' skal så være, ikke at afgøre, hvem der ``har ret,'' og hvem der ``ikke har,'' men skal i stedet være at prøve at få de diskuterende parter til at overholde nogle regler og retningslinjer, som parterne har samtykket til ved diskussionens start. Der kan nemlig sagtens blive mange forhold, hvor parterne kan gøre ikke-konstruktive ting, som ikke kan registreres automatisk af den platform, man benytter til den formelle diskussion, men som ikke desto mindre er værd at forhindre. Og min idé til at løse sådanne problemer, som ellers kan føre til ikke-konstruktive (og dermed heller ikke ligeså værdifulde for de efterfølgende læsere) diskussioner, er så simpelthen, at parterne formelt samtykker til en række retningslinjer og regler, og så eventuelt også udpeger specifikke ``dommer-grupper,'' som de hver især godkender til at håndhæve disse. De vil så bestemt være ønskværdigt, hvis begge parter kan gå med til at godkende visse samme dommer-grupper, således at alle involverede kan forvente, at de diskuterende parter vil gå med til at rette sig efter pågældende dommergruppers bedømmelser. Jeg skriver dommergrupper her i flertal, for det gør bestemt ikke noget, at der er flere, som de begge kan godkende, men hvis bare de begge kan godkende én dommergruppe, så er det også fint nok. Dommergruppens funktion bliver altså så løbende at komme med domme, hvis nogen af grupperne ikke holder sig til retningslinjerne (efter dommergruppens vurdering af sagen). Man kan jo så gøre det sådan, at dommergruppen således kan give straf-point, hvilket så skal kunne ses i det endelige resultat, når både parterne og udefrakommende evaluerer på resultatet. En diskuterende gruppe er så selvfølgelig motiveret i et vist omfang til ikke at få strafpoint på sig, for så vil deres anseelse dale i øjnene af det samlede fællesskab, og dermed vil det så også blive sværere og sværere for gruppen, at finde interesserede diskussionspartnere i fremtiden. 

Det er selvfølgelig ikke kun de dommergrupper, som parterne har udvalgt, der kan evaluere diskussionerne med point og strafpoint; diskussionerne er offentlige og enhver gruppe kan vælge at udarbejde deres evaluering af en diskussion, og dette kan desuden sagtens gøres efterfølgende til hver en tid. Men selvfølgelig er det altså smart specifikt at have dommergrupper til en diskussion, som begge parter har sagt god for, og som løbende evaluerer diskussionen, for det kan så altså netop være en måde, at få de diskuterende parter til at overholde et sæt af konstruktive retningslinjer. 

Cool, var det ikke nærmest det, jeg ville sige så? Nå nej, selvfølgelig bør jeg så lige samle op ift.\ arbejdet med at udarbejde ``kurser,'' og sige at efter sådanne formelle diskussioner er afholdt, så vil det alt andet end lige, medmindre at den pågældende brugergrupper overgiver sig helt og aldeles, være forventeligt, at hver diskuterende part af en diskussion (men også gerne andre parter), sætter sig for med udgangspunkt i diskussionen at udarbejde et mere sammenhængende ``kursus,'' som så skal indføre læseren i de holdninger, brugergruppen har omkring det diskuterende emne. Okay, selvfølgelig kan det også være respektabelt, hvis en brugergruppe vælger at udskyde dette arbejde lidt, indtil nogle flere diskussioner er gjort --- især selvfølgelig også, hvis pågældende gerne vil udarbejde et kursus, der omfatter flere diskussioner på én gang --- men pointen er altså, at energien omkring disse formelle diskussioner, hvis den skal bruges så konstruktivt som muligt, helst i sidste ende bør rettes imod at udforme de ovenfor beskrevne `synspunkt-kurser,'' så interesserede brugere i sidste ende kan få adgang til velstrukturerede og koncise tekster, der indfører læseren i gruppens synspunkt omkring et emne. Cool.

Ah, og der var så godt nok også nogle flere idéer, jeg også ville nævne. %(Nemlig om street credits osv.) Dem kan jeg lige summe lidt over, inden jeg skriver videre her.. (Og måske vil jeg lige skrive nogle ord nedenfor om min donationskæde-idé inden da, men det ser jeg lige..)
Ja, følgende er faktisk en ret vigtig idé, synes jeg. Den er i princippet uafhængig af min ovenstående idé her til en variant af formelle diskussioner, men kan bestemt også bruges i forlængelse af disse. Idéen er at sådanne brugergrupper, der altså repræsenterer et vist synspunkt (og/eller en vis befolknings-, fag- eller interessegruppe), melder sig til en speciel platform, hvis formål er at opstille et system, hvor grupperne kan vædde med hinanden om udsagn --- ikke om penge, men bare om symbolske point, med hvilke de så kan bruge til at måle sig med hinanden. Denne platform kan i princippet altså være uafhængig af hvilken end platform, man bruger til at opbygge og evaluere formelle diskussioner på (a la dem jeg lige har beskrevet), men det ville nu ikke være dumt med en samlet platform med alle disse funktioner. Pointen er nemlig, at væddemålene gerne skal kunne tage udgangspunkt i citater kommende fra medlemmer fra en brugergruppe. Dette kan altså være citater fra enhver SoMe-platform, hvor medlemmer fra grupperne er repræsenteret, men ja, specielt kan de altså også komme fra de ovenfor beskrevne formelle diskussioner. En gruppe skal så kunne gennemgå en proces, hvor de overvejer, om de vil udfordre en anden gruppe på et citat/udsagn fra et eller flere medlemmer, og hvis de vælger at gøre dette, skal platformen altså så kunne slå udfordringen op formelt. Den udfordrede gruppe får så muligheden for at bakke ud eller at stå ved udsagnet, og hvis sidstnævnte vælges, indleder gruppen nu en forhandling om, hvor mange af de nævnte ``symbolske point'' skal væddes om. For at et udsagn overhovedet bør betragtes som et udsagn, der \emph{kan} væddes om, skal det så gerne have karakter af et udsagn, der kan efterprøves inden for en realistisk tidsramme (så altså helst ikke alt for mange år), eller som man automatisk vil få svar på med tiden, uden at det kræver en aktiv undersøgelse. Så pointen med alt dette er altså, at hvis en gruppes medlemmer kommer med udsagn, som til en vis grad faktisk kan besvares inden for en overskuelig fremtid, jamen så kan en rivaliserende gruppe via denne platform altså udfordre førstnævnte til at bekræfte, at de kan stå inde for udsagnet og de derved er parate til at lægge deres ære på spil (eller lidt af deres ære, rettere) ved at indgå et officielt væddemål. 

På samme måde som for de ``formelle diskussioner'' bør brugergrupperne herved også sørger for at blive enige og udpege en (eller flere) dommergruppe(r), som så kan evaluere væddemålets udfald endeligt. Her er der bare en lidt mere konkret magt til den/de udvalgte dommergruppe(r), fordi deres evaluering så vil medføre en overførsel af symbolske point på platformen. Tanken er at platformen, gerne på dens forside, har en oversigt over alle sådanne pointtransaktioner. Det skal så ikke forstås som om, at hver gruppe har en egentligt pointbeholdning; de har ikke nogen ``konto'' med symbolske point på, hvis beholdning vokser eller falder med hvert væddemål. I stedet er man bare interesseret i at fremhæve, hvor mange point en gruppe har tabt eller vundet til andre grupper i et vist tidsrum (f.eks.\ på den seneste måned eller i løbet af det seneste år). For målet er som sagt bare, at grupperne herved kan måle sig med hinanden og vinde eller tabe ære, så at sige, i andre gruppers og i det samlede fællesskabs øjne. Så hvis bare man altså kan få en oversigt over, hvor mange point en gruppe har vundet og/eller tabt i forskellige tidsintervaller og til hvilke grupper --- og også med hvilke grupper som dommere over væddemålet --- så er idéen altså hjemme, som jeg ser det. Der kan så selvfølgelig komme til at lægge mange tanker i, hvordan man designer sådan en oversigt, så den giver det mest brugbare og interessante billede over væddemålene for brugerne. For eksempel kan det jo blive ret vigtigt, at man netop kan filtrere oversigten, så man skiller grupper fra, hvilket eksempelvis kunne være, hvis der var en slags yolo-grupper og/eller troll-grupper, der bare bliver ved og ved med at tabe. Herved vil det så ikke være nær så prestigefuldt at vinde væddemål over disse grupper som at vinde point fra veletablerede grupper. I øvrigt kan det også ske, at dommergrupper laver evalueringer, som ender med at blive kontroversielle i fællesskabet, så måske kunne man derfor også finde en måde, at notere sådanne kontroverser på. Så ja, alt i alt kan der altså være mange ting at tænke over her, men man ville dog, tror jeg, kunne komme ret langt med et rimeligt simpelt design, og så kan man altid bare forbedre det over tid. 

Hvornår skal en dommergruppe så erklære, at et væddemål er vundet af den ene part? Kan alle ``fakta'' og alle konklusioner ikke altid tages op og diskuteres på ny? Jo, men tanken er så bare, at de væddende parter forventes at gøre sig ret umage med selv at formulere, hvornår den ene part har vundet, og hvornår den anden part har --- og ja, hvornår væddemålet må erklæres uafgjort, hvis altså gerne vil undgå, at det er åbent for altid. Hvis man således har i sinde at vædde om et spørgsmål, der ikke kan efterprøves helt, men hvor de to teorier måske alligevel har klart forskellige sandsynligheder for forskellige udfald, jamen så kan man bare nøjes med at vædde om de udfald. Herved får man altså ikke nødvendigvis opklaret det specifikke spørgsmål 100 \% ved væddemålets afslutning, men man får stadig en vis (om end måske lille) indikator på, hvem der nok havde ret. Og når man så netop har sådan en platform, hvor de samme grupper kan blive ved med at lave sådanne nogle væddemål, så vil man alligevel med tiden, også selvom man i hvert specifikke væddemål ikke får et klart svar, få tegnet et mere og mere klart billede af, hvilken gruppe er bedst til at komme med forudsigelser omkring emnet. Hvis f.eks.\ en gruppe har det med virkeligt at dominere andre grupper, når det kommer til et vist emne, jo, men så er det nok en rigtig god idé lytte til deres underliggende teorier og synspunkter for emnet. 


%Ah, jeg bør lige overveje, om jeg ikke også kan finde på noget til at fremhæve og værdsætte i fællesskabet, når diskuterende parter finder frem til noget, de er enige om.. (..Hm, og i øvrigt kunne man måske også prøve at lægge mere direkte op til i de formelle diskussioner, at man skal prøve at finde frem til potentielle væddemål (altså punkter, hvor emperi kan pege i den ene eller den anden retning)..) ...Hm, man kunne msåke bare lægge op til, at dommerne også gerne skal bedømme diskussioner omkring, hvor god parterne er til at finde frem tid enighedspunkter, samt hvor gode de er til at finde potentielle emperi-forudsigelser (som man jo så (passende) kan vædde om evt.). 

Jeg har nævnt, at det nok ikke vil være dumt med en fælles platform til at afholde både væddemålene og de formelle diskussioner. I øvrigt ville det nok også være fordelagtigt (tror jeg bestemt), hvis man som en del af retningslinjerne til formelle diskussioner sørger for at fokusere meget på, om man kan finde frem til udsagn, som kan efterprøves til en hvis grad med empiri (eller rettere som altså har forskellige empiriske forudsigelser), og som man jo så oplagt også kan starte væddemål på baggrund af efterfølgende. Et sådant fokus vil nemlig hjælpe i høj grad, tror jeg, på at gøre diskussionerne mere konstruktive. 

En anden ting, der også kunne være værd at prøve at fokusere særligt på, og som man altså med fordel kunne inkludere i formelle diskussioners retningslinjer, er, at man som diskuterende parter ikke bare prøver at nå frem til punkter, man er uenige i, og så fremhæve disse, men at man også i høj grad prøver at finde frem til og fremhæve punkter, man finder ud af, at man er enige i under diskussionen. For tit vil to diskuterende parter, hvis diskussionen afholdes på en god måde, finde frem til, tror jeg, en masse punkter, hvor det viser sig, at de er enige, men hvor dette ikke var noget, parterne vidste på forhånd (generelt, hvis vi altså snakker grupper). Og ja, noget siger mig, at det bestemt bliver værd virkeligt at fremhæve, når to parter finder frem til punkter, de er enige om. For dette vil jo bestemt også være gavnligt at læse om efterfølgende (f.eks.\ som udefrakommende), i stedet for kun at læse om de diskussionspunkter, hvorom der var uenighed. Hvis dommergrupperne så altså mener at parterne er gode eller dårlige til at gøre disse ting, så bør de kunne give specielle point i evalueringen, så fællesskabet kan se, hvordan parterne opførte sig (når de f.eks.\ skal beslutte om de vil læse, og måske arbejde videre på, diskussionen). Det vil derfor sikkert være smart, hvis man sammen med retningslinjer også kan opfinde nye (brugerlavede) pointtyper på platformen, således at evalueringerne kan begynde at bruge sådanne point til at signalere forskellige ting. 


Så jeg har altså alt i alt tre idéer i store træk, som jeg tror kan forbedre brugbarheden og brugeroplevelser af internetdebatter gevaldigt. For det første mener jeg, at arbejde på simpelthen at udforme ``kurser,'' der skal indføre læseren i et synspunkt, vil være en rigtig vigtig ting at fokusere på som samlet fællesskab. Dernæst mener jeg, at formelle diskussioner, hvor man udarbejder argumentationsgrafer i grupper, lidt som et slags spil, kunne være en rigtig god idé, men især altså, hvis man dog sørger for, at argumentationsgraferne så ikke bare for en struktur af argument på modargument på argument på modargument osv. Jeg tror nemlig på, at sådanne argumentationsgrafer bliver for uoverskuelige at følge med i og at læse efterfølgende. Så mit forslag er altså, at man sørger for at holde en ``hovedåre,'' som jeg har kaldt det her, af delkonklusioner, der hele tiden meddeler om status af diskussionen og bringer de relevante åbne spørgsmål ``op til overfladen'' (så man ikke skal dykke helt ind i undergrenene af grafen for at finde dem). Og sidst men ikke mindst mener jeg altså, at en platform, hvor synspunkt-grupper udfordre hinanden kan vædde om empiriske udfald (hvis deres teorier hver især forudser forskellige udfaldssandsynligheder), vil være en kæmpe stor ting. Sådanne væddemål vil nemlig i sig selv være rigtigt interessante, og jeg tror endvidere også bare, at dette så vil sætte ekstra fokus generelt, når det kommer til internetdiskussioner, på at nå frem til sådanne empiriske punkter, hvilket vil gøre diskussioner så meget mere konstruktive. Og i øvrigt kan man så også med mit forslag til retningslinjer i de formelle diskussioner prøve at forøge fokusset på sådanne ting som dette, der gør diskussioner mere konstruktive. Og en anden af sådanne ting, som kan skabe mere konstruktive diskussioner, er i øvrigt, mener jeg, hvis man også prøver at indføre et større fokus på at finde frem til og fremhæve punkter, hvor det viser sig, at man er enige (og hvor dette altså ikke nødvendigvis var forventet). Så ja, det er altså mine idéer, jeg vil komme med her omkring diskussion og debat på internettet. 


\ldots\ Ah, og lige en ting mere: Nu har jeg jeg skrevet om, at man i diskussioner bør fokusere på, at nå frem til punkter, men kan efterprøve (og evt.\ vædde om). Men det skal så også nævnes, at det selvfølgelig er ligeså vigtigt at finde frem til punkter, der allerede kan besvares med empiri. Pointen er bare at få de diskuterende parter til i høj grad at overveje, hvordan man får de diskuterede udsagn forbundet til udsagn, der i en eller anden grad kan testes, eller som er blevet testet. Så hvis ``retningslinjerne'' indeholder en protokol for at uddele point til grupperne alt efter, hvor gode de er til at forbinde det diskuterede til noget, man kan efterprøve, så bør pointene altså ikke skelne imellem om tingen allerede er afprøvet eller ej. (For det er jo mindst ligeså konstruktivt, hvis den ene part (generelt) ikke var klar over empiri, der usandsynliggør deres teori, og at man så finder frem til dette i diskussionen, som hvis man finder frem til et punkt, hvor der bare mangler empiri.)

Og en sidste lille ting, jeg i øvrigt også kunne nævne, er at diskussioner på f.eks.\ Twitter ofte fanger folks interesse i høj grad, hvis store/kendte individuelle profiler er involveret. Jeg snakker jo her særligt om grupper (fordi diskussionerne alligevel hurtigt kan blive for store til, at det giver mening for enkeltpersoner at påtage sig hele opgaven i rollen som den ene diskuterende part), men man kan jo sagtens have grupper, hvori der optræder enkelte store profiler (som så får meget at skulle have sagt i gruppen). Herved kan man altså også stadig trække på meget af den hype, der kan være omkring visse profiler, også selvom man på platformen lægger mere op til lidt større diskussioner og til at de diskuterende parter er grupper. En gruppe kan jo således sagtens bestå af f.eks.\ en kendt profil, samt assisterende personer (hvilke der godt kan være mange af; man kan jo bare give den kendte profil en rigtig høj stemme-vægt i gruppen (idet vi jo snakker mine ``vægtede brugergrupper'')), som kan tage sig af meget af fodarbejdet i diskussionen. Bemærk i øvrigt i denne forbindelse, at det ikke gør noget, hvis et gruppemedlem siger noget forkert, eller noget, der ikke stemmer med resten af gruppen holdninger, eller hvis bare medlemmet misser et godt modargument og/eller ikke får formuleret et argument særligt godt. Selv hvis de formelle diskussionsgrafer er designet sådan, at parterne ikke kan slette deres svar (medmindre det indeholder følsomt data eller noget), så vil det uanset hvad altid være sådan, at parterne kan rette og forbedre deres argumenter undervejs; ellers vil der være noget helt galt med designet. Det må endda gerne være sådan, at parterne også har mindst én runde til at trække påstande og/eller formuleringer tilbage, eksempelvis hvis nu dommerne gør opmærksom på, at udtalelsen bryder retningslinjerne på en eller anden måde.  

\ldots\ Nå ja, og jeg bør jo også lige nævne, at selvom jeg ikke kan lade være med at tænke meget i open source, så tror jeg altså også, at der ville være et kæmpe (kommercielt) potentiale i at starte sådan en platform som en mere gængs (og closed source) SoMe-platform, især hvis der ikke kommer vind i sejlene nok for open source-projekter omkring det (måske fordi den kommercielle gevinst netop ikke er der der). 

*Ah, jeg kan også lige hurtigt nævne, at selvom jeg her har lagt meget op til, at diskussioner skal foregå som en slags spil, hvor parterne kæmper imod hinanden (bl.a.\ for sådanne forhold kan skabe ekstra interesse og spænding i og omkring fællesskabet), så synes jeg dog generelt, at en god debatkultur også i høj grad har plads til, at folk kan argumentere frit som ``djævlens advokater,'' så at sige. Det er kun godt, hvis diskussionsbidrag, hvor man udarbejder argumenter, som man ikke nødvendigvis er enig med selv, kun er værdsatte i fællesskabet; det er jo kun gode bidrag, og hvis de nu ikke så ender med, at repræsentere særligt godt, hvad en tilhænger af argumentet ville sige, jamen så har vedkommende jo bare mulighed for at komme med rettelser til argumentet og udforme det bedre. Man kan også sagtens have dette som en værdsat del af de formelle diskussioner, og endda måske have det med i retningslinjerne, at folk meget gerne må komme med modargumenter imod sig selv. Dette kan f.eks.\ være brugbart, hvis den ene gruppe er meget mindre end den anden; så kan sidstnævnte jo godt hjælpe førstnævnte med en masse fodarbejde i diskussionen. Man skal selvfølgelig bare sørge for, at en parts egne formuleringer og rettelser til argumenter aldrig drukner i, hvad modparten har bidraget med i form af modargumenter imod sig selv. Alle diskuterende parter skal altid føle, at deres argumenter bliver hørt som det, de er, og ikke bliver forvrængede, f.eks.\ hvis det er modparten, der har formuleret dem. 


*En sidste ting (sikkert), jeg også lige bør nævne, er, at en god måde hvorpå at få synspunkter til at konkurrere imod hinanden på formel vis, særligt når man allerede hver især har udarbejde ``synspunkt-kurser,'' er at lave væddemål omkring, hvad upartiske og/eller uvidende (og for så vidt muligt unbiased) mennesker vil synes bedst om, hvis de læser begge / hver af de modstående synspunkt-kurser grundigt igennem og tager en beslutning. Det vil faktisk ikke være helt dumt, hvis dette gøres til et konventionelt mål med synspunkt-kurserne, nemlig at de skal prøve at slå de konkurrerende synspunkter i sådan en formel udfordring. Tricket bliver så selvfølgelig at finde en passende (villig) mængde mennesker, som kan siges at være unbaised og uvidende nok fra start. Men det gør heller ikke noget, at man ikke kan udføre absolut perfekte forsøg på denne måde. I samme ånd som for andre typer af de nævnte væddemål, så skal hver part bare selv beslutte sig for, hvor meget de vil satse (af symbolske (æres-)point) på væddemålet, så hvis ikke man har tillid til forsøgspersonerne, så kan man bare sørge for at vædde så meget desto mindre (eller man kan jo også så bare kræve, at sammensætningen af forsøgspersonerne ændres, før man går med til at vædde). 


\subsubsection[Blockchain]{Tanker og idéer om blockchain-teknologi}
**\textit{(17.02.22) Nå, den idé, jeg ender med at konkludere denne undersektion med, holder vist ikke alligevel. Se noterne ude i kommentarerne under ``Draft to first publication''$\to$``An attack vector on PoW chains and how to defend against it,'' og særligt måske i undersektionen, ``Trying to outspend the attackers...,'' for en forklaring af, hvorfor det ikke holder alligevel.\,.}\\

%%%%Problemer med PoW (måske; ellers efter PoPP) *(Jo, måske vil jeg gerne starte med det..) *(nej..):
%Jeg vil gerne foreslå en ny form for blockchain, som jeg tror kan konkurrere med PoW- og PoS-blokkæder. 
%%Inden da vil jeg dog gerne lige starte med at pege på nogle problemer med disse gængse kæder. Jeg vil fokusere primært på PoW-kæder, og så vil jeg efterfølgende lige snakke kort om, hvor PoS-kæder har plads til forbedring.
%
%%Det store problem med PoW-kæder, som jeg ser det, er at transaktionerne på kæden i samlede beløb som regel vil overstige de omkostninger, der er forbundet ved at mine blokke til kæden, væsentligt. Og det kan selvfølgelig heller ikke være anderledes, for hvis omkostningerne ved at udføre transaktionerne generelt bliver for store relativt til...
%
%%%%%Proof of.. Public Past(..?):
%%Jeg har ikke besluttet et navn endnu for, hvad man kunne kalde denne kædetype, men jeg kom lige på navnet, Proof of Public Past, nu her (d. (08.10.21) i dag btw). Idéen går i al sin enkelthed ud på simpelthen at droppe alt, hvad der hedder mining. Hvis man ser på, hvad de gængse blockchains forsøger at opnå, så er det jo som bekendt bare at implementere en offentlig historik over transaktioner og smart-kontrakter, hvor fællesskabet...


\ldots

(11.10.21) Okay, min plan har ændret sig lidt nu. Jeg har en idé til en (måske ny) form for angreb på PoW-kæder med et tilhørende forsvar til, som jeg har tænkt over (bl.a.) de sidste to dage. Og jeg tror sgu egentligt muligvis, jeg bare vil nøjes med at dele dette, og så ellers bare (måske) min donationskæde-idé. Jeg havde ellers lidt planlagt at argumentere for, hvorfor et kæde-system, hvor ``mining'en'' bare foregår ved at forskellige interesserede tredjeparter, hvilke nye transaktioner bliver uploadet til de åbne databaser, de hver især administrerer, og hvor disse bare snakker sammen og sørger for at aflyse transaktioner, der er i kapløb med andre transaktioner hos de andre parter. Jeg kan mærke nu her, hvor jeg skriver det, at dette åbner nogle spørgsmål, og at der også er flere måder at gøre dette på, men pointen er netop, at jeg \emph{ikke} vil gå videre med denne idé. %Jo, den kan lade sig gøre (selvfølgelig kan den det), og jo, den vil i princippet (på papiret) fungere bedre end gængse PoW-kæder, hvor man alligevel benytter tillid til tredjeparter (som nemlig også kan være åbne grupper), som det er nu; man kommer aldrig udenom, at... nå nej, det kan man jo godt argumentere for, at man ikke gør i gængs PoW, og det var så dog her, at jeg troede, at den eneste løsning til mit angreb var, hvis man så alligevel begyndte at gå over til at afvise alle kæde-forks, der ikke har været offentlige imens de blev udarbejdet.. var det ikke sådan; det var det måske ikke..? ..Nå nej, det er da egentligt også noget vås: Jo, da jeg færdiggjorde min brainstorm nedenfor (startet før og sluttet efter d. 25/9), der troede jeg, at "afvis alle kæder, der ikke har været offentlige" var en løsning, men det er det jo faktisk ikke alligevel. For at løse det skal der være en klar konsensus om, hvornår en kæde har været offentlig længe nok til, at man godkender den som endegyldig, så man aldrig efter dette vil benytte nogen andre forks i fællesskabet. Og ja, problemet er så, hvordan skal denne konsensus-proces foregå, og hvordan bestemmer man overhovedet dette i et decentralt.. ja, det bliver på en måde lidt proof of stake.. Hm, okay, det kom godt nok bag på mig, hvor rodet denne udredning af disse tanker blev nu her, men sådan går det jo.. Nå, men konklusionen bliver jo det samme uanset, om jeg kan komme frem til en eller anden løsning, hvor man kende forskel i et decentralt fællesskab på, hvilke forks har rent mel i posen, og hvilke forks er kunstige og en del af et angreb af en eller anden art.. Hm, tja, en løsning kunne vel bare være, at formulere noget så vagt som dette, og så bare sørge for, at der også er incitament til at mine kæder i en vis grad i systemet, så fællesskabet ikke skal overvåge dataen og hinanden konstant og med skarpe øjne, men hvor man kan læne sig lidt op af PoW-principperne også.. Anyway, uanset, om jeg vil kunne finde på en løsning, der giver lidt mening her, så er det altså lige meget; det bliver ikke interessant nok til at dele, og alt andet end lige har jeg endda også, hvad jeg skal bruge i min angrebsidé, så jeg behøver det heller ikke. 
%...Jo, man kan sagtens lave nogle KV-systemer, der fungerer ovenpå åbne databaser, som bare skal følge nogle overordnede principper, og som kan udskiftes løbende (og som altså styres decentralt og som kan kopiere og køres af alle), man så ryger al mystikken bag KV'en, og så bliver den intet værd (for gængse KV'er bygger alligevel i sidste ende ret meget på mystikken bag dem og den medfølgende hype-faktor derved).. Så ja, lad mig ikke tænke mere over de tanker --- ikke for denne omgang i hvert fald. 

Så ja, jeg vil altså redegøre for en idé til et angreb på PoW-kæder og for et forsvar imod det. 
Grunden til at det ikke ikke rigtigt kan betale sig, at lave angreb, hvor man forker den eksisterende kæde og lægger arbejde nok i forken til at overhale denne, er, at det beløb, man kan vinde på ikke er stort nok til, at arbejdet kan betale sig. Medmindre en angriber ejer en stor nok del af kædens værdi og kan sælge denne værdi på et kort tidsrum, vil denne ikke kunne vinde nok ved arbejdet i at reversere kæden til, at det er arbejdet værd. Er store kæder såsom Bitcoin så sikre over for angreb, der prøver at reversere transaktioner? Ikke nødvendigvis; ikke hvis mange nok møntejere (f.eks.\ BTC-ejere) kan gå sammen om angrebet til at de samlet set for nok at vinde ved arbejdet, og så de derved også er mange nok til at betale for angrebet.

Det er altså her, jeg tror at gængse PoW-kæder har en umiddelbar sårbarhed. Selv hvis ingen enkelt aktør nogensinde kommer til at eje nok af den samlede møntværdi til at et angreb vil være muligt, så kan flere aktører dog stadig i princippet gå sammen om at udføre et angreb. Og vil et sådant angreb så kunne betale sig i princippet? Det er muligt, for hvis vi kigger på f.eks.\ BitCoin, så er værdien af de samlede transaktioner, så vidt jeg lige kan se, i omegnen af tusind gange så stor som udbyttet minerne får i samme tidsrum. Og værdiforskellen kan endda også svinge meget i tid, så en sammensværgelse vil sagtens kunne sælge op imod 100 eller endda 1000 gange, hvad der mines for i en periode til at betalingerne godkendes. %, og angriberne vil så have op imod 100 til 1000 gange så mange dage til at udføre angrebet indtil... Hm, nej.. Hvad er den mere præcise udregning? Hvor mange dage for at en betiling går igennem? ..Udregningen er vel gevinst = overførte beløb - mining løn for at betaling godkendes * (1 + 1/(x-1)), hvor x er den hastighed, angrebs-forken kan opnå relativt til den normale mining-hastighed. Og denne udregning er altså, hvis man venter med at sætte angrebet i gang til efter at overførslerne er godkendte. ..Hm, så hvis man kunne købe sig til arbitrært høj x, så kan angrebet faktisk i princippet betale sig, hvis bare man har mere end to gange så mange penge i de samlede overførsler, som hvad mining-lønnen var på det samme tidspunkt. Ok.
Dette bør være rigeligt, for som jeg lige kan regne mig frem til, så er den nedre grænse, ift.\ hvornår et angreb kan betale sig, på $(1 + 1/(x-1))$ gange så mange penge, man skal have overført i forhold til mining-lønnen på det tidsrum, hvor transaktionerne godkendes. Dette er, hvis man venter på at overførslerne er godkendt, inden man igangsætter angrebet. I udtrykket betegner $x$ den relative hashrate, som angriberne er i stand til at opnå sammenlignet med den normale hashrate (som de altså skal overhale). Så hvis man kan betale sig til en stor nok hashrate (hvilket man kan med en antagelse om, at man alt i alt betaler mere for samlet antal hashes og ikke så meget for, hvad selve hastigheden er, hvilket som sådan umiddelbart virker realistisk nok), så skal man altså i princippet bare op på over dobbelt så meget værdi overført, som minernes omkostninger var i samme periode, før et angreb kan betale sig. 

Men hvordan skal de potentielle angribere så finde frem til hinanden og aftale et angreb, uden at det bliver opsnappet af resten af fællesskabet? Jo, men har jo lige nævnt, at der som regel overføres tusind gange så mange BTC (hvis vi ser på dette eksempel) som hvad, det koster minerne for samme periode, så i princippet kunne man jo bare lave en rundspørge, efter at man har solgt en masse BTC, og så høre, om andre nylige BTC-sælgere kunne være interesseret i at igangsætte et angreb. Og hvis man kan gøre dette anonymt, så koster det ingen gang folk noget, at igangsætte og svare ja til denne rundspørge. 

Hvad ville angribere så skulle gøre for at iværksætte sådan et angreb? Skal man tage telefonen og spørge rundt til folk? Skal man sende åbne breve (i form af anonyme e-mails) rundt i fællesskabet? Tja, man kan faktisk noget meget mere simpelt. Man kan oprette en smart-kontrakt fra en separat (anonym) pung, hvor man har beholdt en lille del af sin BTC-formue (og har altså solgt resten af formuen fra andre punge). Man kan så vente et lille (tilfældigt) tidsrum, inden man opretter denne smart-kontrakt, så folk ikke kan forbinde den til de handler, man lige har gjort. I smart-kontrakten skriver man så (hvis vi oversætter til naturligt sprog), at hvis nok underskriver sig samme kontrakt, så vil begynde at overføre BTC som løn for at mine blokke til en side-channel i form af en fork til den nuværende hovedkæde til de pågældende minere af nye blokke til forken. Antallet af underskrifter, før kontrakten går i gang, bør så svare til, hvad der skal til for at gøre angrebet rentabelt, plus hvad man eller tror, man kan nå op på, så angrebet bliver endnu mere rentabelt. Bemærk, at smart-kontrakterne altså skal overføre penge på den nuværende hovedkæde til de deltagende minere, sådan at man sikre sig, at det er win-win for disse at deltage i det --- i hvert fald hvis vi for nu antager, at den fremtidige mining-løn ikke falder som følger af angrebet. Kontrakterne kan implementeres ved at belønne minere, der sender deres fork-blokke til de normale minere (hvilket kan inkludere dem selv) indkapslet i transaktioner, og så beder disse (med en lille fee som sædvanligt) at inkludere disse transaktioner. Smart-kontrakterne behøver altså ikke at belønne blokke på den nuværende hovedkæde, som indeholder blokke fra angrebs-forken, men skal bare belønne mineren, der har minet fork-blokken (hvis pung altså skal findes på både den nuværende kæde og på angrebs-forken). Pointen er så at lønnen for at mine fork-blokke herved skal komme til at overstige lønnen for at mine de regulære blokke, således at forken kommer til at vokse hurtigere. Og hvad sker der så med minernes penge, når forken endeligt overhaler den regulære kæde, og dermed overtager sidstnævntes position som den ``rigtige'' fork? De vil jo så miste al den værdi, de har optjent på den oprindelige fork\ldots\ Hm\ldots\ %Hm, lad mig egentligt lige se her, for det er jo ikke sikkert, at angrebs-fork-minerne kan sælge deres belønnings-BTC, imens angrebet står på, så falder angrebet ikke lidt sammen så..? ..Hm, man kunne jo godt nok indlede angrebet noget tidligere med nogle langvarige kontrakter om at overføre de samme penge til de samme punge, hvis man selv på et tidspunkt har deltaget i en endelig underskrift af et igangsat angreb, og hvis målet for dette angreb kan ses at være opnået.. Og man kan så endda kryptere smart-kontrakten, så andre ikke helt så nemt, kan se, hvad man præcis har gang, og også som et ekstra lag, der gør at kontrakten kun udløses, når man selv er klar til det.. (Tja, det sidstnævnte behøves egentligt ikke; det er nok at man bare skal have skrevet under på sidste del i at igangsætte et angreb.) Men ja, man kan kryptere. Dette vil selvfølgelig bare virke lusket, så folk vil så dog stadig kunne se, når der potentielt er et angreb under opsejling.. ...Okay, man skal vist, som jeg kan se, alt andet end lige lade op til sådant et angreb så, fordi man så skal sætte nogle indledende smart-kontrakter, inden at pengene bliver overført, som man prøver at respend'e. Men ja, så det er jo egentligt meget godt alt i alt; så skal fællesskabet bare vide, at angrebet er en mulighed, og hvis der så ser ud til på et tidspunkt at blive mange penge låst til obskure kontrakter (i.e.\ kypterede kontrakter (hvilket man nemlig godt kan lave)), så bør man så sørge for ikke at overføre særligt mange penge. Hm, og kontraktudstederne kan sætte en lang frist på kontrakterne, men så låser de også bare deres egne penge så længe.. Så ja, der er ligesom den måde, at opdage, at et angreb er under opsejling, på, og herved kan man altså tage sig sine forholdsregler, hvis bare man kender til muligheden for angrebet. ..Cool..(?) ..Hm, men angrebet afhænger så lidt af, at det både er kendt og ikke er kendt..?.. ...Ah, man kan faktisk også som angriber låse den indledende BTC til et lightning-netværk, og så kan en del af smart-kontrakterne bare være, at man også skal kunne se.. nå nej, smart-kontrakten vil jo annulleres igen i sidste ende, så den kan man ikke love noget med for forkens vedkommende.. ..Tjo, men derfor kan man vel godt bruge de nye kontrakter alligevel.. Nej.. Ah, jo, for minerene kan jo være ligeglade med, om angrebet dør ud (for dem er det jo win-win). Så jo, man kan godt gøre det en del af kontrakterne, at der også løbende skal overføres tilsvarende mængder penge fra lightning-netværket, for at smart-kontrakterne kan blive ved med at lønne sig, og hvis ikke angriber bare mister al værdi låst til smart-kontrakten. Så ja, man kan bare starte med at låse penge til et lightning-netværk, og så kan man derefter overraske kæden med en rundspørge om at starte et angreb. Så er det bare spørgsmålet, om ikke lightning-netværket bare vil tage sig selv ned i angrebet? På den anden side så kræver det jo bare, at alle angrebs-minerne og alle angribs-stake-holderne udgør et lukket netværk, så de bare selv kan opretholde lightning-netværket.. ..Ah, og jeg har endda gået og tænkt på, at lightning-netværk består af en masse én-til-én-kanaler sat sammen, men man kan jo sagtens lave lightning-netværk, hvor alle kan sende til alle.(!) Så det bliver altså ikke noget problem at finde et l-netværk, som inkluderer nok mennesker, og som har en lang nok tid åben til, at angriberne kan overføre pengne til minerne her. Hm, og pengene kan så herved bare nøjes med at blive overført her, således at mængden af penge låst til smart-kontrakterne bare skal være nok til at sørge for, at angriberne holder deres løfter om at overføre penge til minerne løbende --- bare sådan at alle parter i angrebet lige har en ekstra vished for, at folk har intensioner om at gennemføre det. ..Hm, en modgift imod angrebet ville vel så være at indføre en konvension om ikke at bruge l-netværker, som ikke kan lukkes ned i tilfælde af, at et angreb af denne type påbegyndes?.. ...Hm, man kan faktisk også angribe eksisterende LNs i samme omgang, for disse kræver nemlig, at parterne er online. Man kan så bare kræve i smart-kontrakterne, at der kun mines blokke, hvor bestemte overførsler indgår, og hvor man så.. Hey, man kan da bare sørge for, at smart-kontrakterne indeholder en opskrift for, hvad angrebs-fork-blokkene skal indeholde, hvori der så skal gælde, at angrebs-stake-holderne skal overføre de samme penge løbende, som gøres på den nuværende ("uskyldige"..) kæde.. Uh, og dette kan endda bare være miners anvar at.. Hm, tja.. Man kunne måske også hellere bare sørger for, at der som en del af kravene for angrebs-forken, skal oprettes og underskrives smart-kontrakter i starten af den nye fork. Angrebs-minerne \emph{skal} altså så inkludere disse kontrakter i de minede blokke, for at angrebs-forken kan erklæres gyldig. Og når disse smart-kontrakter er oprettet, så vil samme angrebs-mining-løn gives på begge forks, og således vil det være ligegyldigt for minerne, hvis angrebs-forken overhaler den uskyldige fork (medmindre de selvfølgelig selv er ofre for angrebet --- og selvfølglig hvis vi fortsætter med antagelsen om, at fremtidig mining-løn ikke vil falde og/eller at minerne ikke er afhængige af denne indkomst). Men ja, nu kan jeg jo så endda se, at man også kan udvide angrebet, så det ikke bare retter sig imod folk, der har fået BTC overført kort forinden angrebet, men også kan rette sig imod alle de LNs, som angriberne (både BTC-sælgere og angrebs-minere; alle der underskriver angrebsaftalen) er med i. Herved kan alle angribere altså opnå al værdi, på alle de LN-kanaler (og eksisterende LNs består forresten faktisk af én-til-én, som jeg tænkte), de har åbne ved forkens begyndelse, så længe kanalerne altså bare ikke går til andre angibere (hvorved man jo bare bør lade dem være som en del af angrebsaftalen). Så mit angreb behøver altså ikke nogen speciel (og lumsk) adfærd, før at rundspørgen igensættes, og angrebet påbegyndes, og desuden er der endda større gevinster at vinde, hvis man også retter angrebet mod LNs. Nå, men det store spørgsmål er så nu: Holder mit forsvar så? Det var jo meningen, jeg ville bruge LNs som en del af forsvaret.. Hm, det går jo nok ikke, hvis man bruger den samme teknologi, hvor kanalerne kan opsiges, for de vil jo så netop være sårbare over for sådanne aftalte fork-angreb alligevel.. ..Det skal ligesom være et netværk opbygget med længerevarende kontrakter, som ikke kan afbrydes andet end af alle parter på én gang..  
%(13.10.21) Okay, nu har jeg tænkt videre over det i går aftes og også lidt her til morgen, og der er faktisk ikke nogen måde at gardere sine kryptopenge imod et sådant angreb --- ikke medmindre man vil lade sine penge styres fuldstændigt af en TTP, og så ryger hele idéen jo lidt. Men der er stadig det forsvar, der hedder at ofrene for angrebet bare opretter nogle mod-kontrakter, der overbyder angriberne. En sådan kamp om minerne vil presse mining-lønnen op, hvilket faktisk også udgøre en ekstra "sårbarhed," fordi minerne nu kan have motiv til at starte et sådant angreb; hvis de kan gøre det anonymt, og hvis den uskyldige kæde ender med at vinde, så.. tja, eller KV'ens kurs vil nok falde uanset hvad i så fald, men stadigvæk.. Men ja, så der kan altså være den afskrækkelse for angribernes synspunkt: Hvis fællesskabet bare har tid nok til at forberede forsvarsprocedurer mod angrebet, så kan de sætte dem i værk med det samme, og så vil angrebet bare foresage, at minerne får penge fra alle parter, og sikkert også at KV-kursen falder, begge ting hvilke ikke er fordelagtige for angriberne.. Tja, medmindre målet er at sænke kursen for at vinde på short-positioner.. Hm.. Hm, det er lige før, jeg helt bør undgå at nævne shorts, så der ikke går panik i kæderne.. ..Uh, men min anden løsning kan også være afskrækkende for et angreb, for en (semantisk) fork kan \emph{godt} laves, imens et angreb er ved at finde sted. Denne løsning er at indføre en semantik omkring valutakurserne i KV-fællesskaber, der simpelthen siger: "Forks, der har overhalet en anden fork, der ellers blev betragtet som den gyldige, og hvorpå tranaktioner blev betragtet som clearet, som led i et angreb, hvor minere er blevet betalt for at reversere kæden for at give angribere penge tilbage og/eller for at tage kontrol over udfaldet af smart-kontrakter, bør ses af fællesskabet som invalide forks, og bør ikke anses for at bære nogen værdi på sig af det brede fællesskab." Med andre ord siger man altså, at alle forks, som ville have været udgået, hvis ikke det var for åbenlyse angreb, de bør betragtes som ugyldige. Og det store spørgsmål kommer så: Hvem bestemmer, hvornår en fork har været en del af et åbentlyst angreb (efter denne semantiske hard fork er vedtaget)? Det gør folk i princippet hver især i fællesskabet; det gør alle de handlende parter. Og så er den ikke længere. Dette vil så i princippet åbne op for angreb, hvor man prøver at ballancere på grænsen imellem, hvad folk i fællesskabet registrerer som et "åbenlyst angreb," og hvad de ikke gør. Ved altså at orkestrere et ikke-helt-men-næsten-åbenlyst angreb, hvor man kan ramme et punkt, hvor der bliver delte meninger i fællesskabet, så kan man altså fremtvinge en fork af kæden. Og hvad kan man så bruge dette til (jeg håber lidt at konkludere, at man ikke rigtigt kan opnå noget ved sådan et angreb..) som angriber?.. ..Hm, ja for hvis man laver et ikke-åbentlyst angreb uden den semantiske hard fork, så vil der alligevel blive et tidpunkt.. hm, hvor fællesskabet skal "beslutte".. hm.. Tja, problemet bliver vel, når minerne har valgt den luskede fork, og at der så er uenighed i fællesskabet, om den ligesom er lusket nok, til at kunne ses som et "åbenlyst angreb".. ..Hm, men måske er det netop minerne, der skal.. tage beslutningen.. hvis man altså kan formode, at de samlet set vil være interesseret i så høj en kurs som muligt.. ..Hvad de jo typisk vil samlet set. ..Tja, så skal man ikke bare sørge for, at alle miners kan give deres stemme til kende, alle stake-holdere kan og alle folk generelt kan selvfølgelig også, og ja, så forker kæden bare, og hvis de handlende i fællesskabet splittes på de to forks, så må man vel i sidste ende bare lade minerne bestemme.. Hm.. ..Hm, det nemmeste ville næsten være bare at lave en PoS-afstemning (som bare stadig kan gøres off chain i princippet; man behøver ikke at udarbejde en eller anden vild on-chain-protokol) i sådanne tilfælde, hvor fællesskabet bliver splittet.. ..Hm, men vil der ikke så være angreb, som stake-holderne kan gøre, hvor de så kan vinde ved at vælge den luskede kæde..? ..Ah kan man ikke bare sige, at når fællesskabet er splittet, så laver man bare en analyse og prøver at finde frem til, hvilket valg, der vil resultere i den højeste kurs. Hertil kan det så siges, at jo længere tid en fork af de to har været førende, jo grimmere vil det se ud, hvis man beslutter at skifte til den korte. Og minerne vil som udgangspunkt også være interesseret i at vælge den, der resulterer i den højeste kurs, så tingene løser sig altså nok: Ved tilfælde af en "lusket fork," der overhaler en måske knap så lusket fork, så gøres der først opmærksomhed på dette i fællesskabet. Alle stikker så fingeren i jorden for at se, om der kan være nok "luskethed" til, at folk vil være interesserede i potentielt at lave en afstemning, der kan føre til at man skifter til den kortere fork. Når et fohold er erklæret af fællesskabet til at være "værd at tage til efterretning," så kan minerne begynde at afgive deres stemmer ved at alle minere skifter til den af de to forks, de tror mest på (hvilket så altså godt kan være den korteste af dem). Imens dette står på diskuterer fællesskabet, hvilket valg, der vil føre til den højeste kurs. Hvis et klart valg træffes kan fællesskabet så godt trumfe deres holdning igennem ved simpelthen ikke at være interesserede i at købe minede mønter fra den fork, der ikke anser for gyldig. Men hvis ingen klar beslutning bliver taget af fællesskabet, så vil beslutningen altså efter lidt tid blive taget af minerne (som dog også har haft chancen for at skifte til den mindre luskede (og måske altså også mindre i længde) kæde), for jo længere spænd der bliver imellem de to forks, uden at fællesskabet træffer en klar beslutning, og jo flere minere der svinger over til samme side, desto klarer bliver spørgsmålet også, for i sidste ende vil det så klart ikke være til fordel for kursen, hvis man som fællesskab forcerer et skift. Bemærk, at der som udgangspunkt ikke benyttes nogen PoS-protokoller; der er som udgangspunkt slet ikke peget på hverken nogen central enhed eller på nogen gruppe til at træffe valget om "hvad der vil være bedst for kursen." Det hele handler i sidste ende bare om, hvordan fællesskabet fortolker spørgsmålet om, hvad der vil være "bedst for kursen." Og det skal så være indforstået i denne semantiske hard fork (hvis man kan kalde den det, for den vil jo i sidste ende altid blive lig den originale fork; det vil kun være midlertidigt, at den samentiske "hard fork" kan være anderledes end kæden med den originale fortolkning), at man kun har lov til at forcere et skift i fællesskabet, hvis det sker som led i at udrede, hvad en delmængde af fællesskabet ser som "et åbenlyst angreb." ..Cool.. ..Og ja, jeg kan så også komme med et yderligere forslag til en semantisk fork, hvor BTC-værdien omfortolkes ved at implementere en dæmpning af værdien for mønter minet efter den semantiske fork, hvilken man så kan klade en "grøn fork" eller en "klimavenlig fork" eller noget i den stil. Tanken er altså, at man bruger det, at minernes job nu ikke er helt så vigtigt for kæden; minere er stadig vigtige, men der behøver ikke at være de samme store summer involveret for at holde kæden i live (for det var alligevel ikke nok til at forhindre angreb (af min type) og nu har man altså andre midler til at forhindre dem). Man kan altså nå klare sig med mindre hash power på kæden og kan dermed igangsætte sådanne "grønne (semantiske) forks." Folk kan så vælge, om de vil handle ud fra værdiudregningen gjort i den "grønne fork," eller om de vil værdsætte mønterne ud fra den originale fortolkning; det er altså de handlende der bestemmer, hvilken er de to værdier bedst vil matche den fremtidige værdi af KV-mønterne. Okay, lad mig lige summe lidt over de her tanker og idéer. ... Ej, det virker altså umiddelbart legit.. Så fedt, at jeg nu føler jeg har, hvad der bør betegnes som et godt og fornuftigt forsvar imod angrebet. Det kan så være, at nogle folk vil se det her med, at man kan tage et lille skridt videre og gå over til en "grøn" semantisk fork af valutaen, som noget der kan være en "slippery slope" imod noget, man måske ikke ønsker, især hvis man er miner. Men dette faktum kan så netop også være endnu en ting, der gør, at folk ikke behøver at gå i panik, for minere er jo en del af angrebet, og hvis de ser det som en "slippery slope," så vil de jo nok ikke gå med til angrebet. Og ja, hvis minere viser, at de slet ikke er til at stole på alligevel, jamen så \emph{vil} det nok være en "slippery slope" imod en "grøn fork," for så vil folk jo ikke have særligt meget til overs for miner-fællesskabet alligevel. Disse argumenter vil altså nok gøre, at mange minere, der har en god indkomst ved at mine, som de ikke vil risikere, ikke vil have lyst til at gå med i et angreb. Og så skal angriber-stake-holderne altså lede efter nye minere, hvis de vil starte et angreb, og så kan angrebet altså ikke gøres nær så hurtigt og nemt som ellers. Så ja, dette er altså et ekstra argument, hvorfor der ingen grund til panik er; hvis bare jeg for delt alle forsvarsmulighederne med folk, inden at andre finder på samme angreb og derved kan tage fællesskabet på sengen. Og de andre argumenter er altså, at de bare vil føre til en kamp mellem angribere og ofre, hvilket så gør alle til ofre, og dette er endda kun, hvis altså ikke bare fællesskabet siger: "Hov, der er et angreb på vej, så lad os blive enige om at søsætte en ny semantisk fortolkning af systemet og af værdierne på det (i henhold til det, jeg har skrevet lige her ovenfor), sådan at angriberne ikke får noget ud af det, og så at vi bare kan fortsætte med den "ligitime fork," også selvom angriberne skulle gå hen og overhale denne midlertidigt." Fedt fedt fedt; hvor er det nice! 


%(14.10.21):
Okay, der var et hul i min idé, som jeg vist nu har lappet. Det har så vist sig, at med min nye angrebsstrategi bliver det faktisk sværere at forsvare sig imod i princippet --- men ikke nødvendigvis i praksis; det hele er ikke nødvendigvis så logisk og ligetil. Men jeg kommer tilbage til forsvarsmulighederne. Løsningen til at sørge for, at minerne bliver betalt i begge forks, så de ikke mister noget ved overhalingen (hvis de altså ikke selv også er ofre i angrebet, i.e.\ hvis de altså ikke selv for nyligt har forøget deres BTC-beholdning), er ret simpel. Smart-kontrakterne der udlover dusører (på den originale fork) til minere, der leverer blokke til angrebs-forken, skal også kræve af denne fork, at der bliver oprettet tilsvarende smart-kontrakter i starten af angrebs-forken, således at de samme dusører også bliver uddelt her. Ja, man kan jo sikkert endda udforme en (multi-signature) smart-kontrakt som bare kan indsættes på begge forks med samme effekt. Dette bør være muligt, for jeg tror sagtens man kan lave kontrakter, hvis gyldighed ikke afhænger af, at den bliver uploaded til kæden på et meget snævert tidsinterval. Og hvis underskrivere skal angive en tid for underskriften (og hvis en kontrakt så ikke er gyldig før underskriftstidspunktet), så kan de jo altid bare lyve og lade som om, at kontrakten blev underskrevet på et tidligere tidspunkt. Hm, man kunne så kræve i kontrakten/kontrakterne, at samme bliver uploadet til angrebs-forken, som jeg var ved at foreslå, men på den anden side kan man også bare lade det være op til minerne, at sørge for at den bliver inkluderet der også. For minerne vil jo så selv være interesseret i at få gjort dette, inden at der er fare for, at angrebs-forken overhaler den originale fork.

Så det er altså angrebet. Det er så også gået op for mig i mellemtiden (i forgårs (d.\ 12/10), hvor jeg afsluttede forrige paragraf med ``Hm\ldots'' og tog en tænkepause), at lightning-netværk (LNs) også er sårbare over for dette angreb. 
For i sådan et angreb vil angrebs-forken være dikteret af de smart-kontrakter, der udlover dusører for den. Samme smart-kontrakter kan derfor godt også stille krav til, hvad de minede blokke må indeholde og ikke indeholde. Dette gør alle kontrakter sårbare, der kan afsluttes effektivt ved, at parterne bare bekendtgør, hvad resultatet var, og hvor der så er en indsigelsesperiode, hvis den ene part lyver, hvilket i hvert fald bl.a.\ inkluderer kanaler i LNs. Hermed kan alle parter, der har åbne LN-kanaler, nu blive potentielle angrebs-stake-holdere, hvis man altså kan sørge for, at dusør-smart-kontrakterne indeholder en protokol, hvor alle underskrevne angribere får lov at igangsætte en terminering af de kanaler, de ønsker --- \emph{hvis} disse kanaler altså ikke deles med andre angribere --- men hvor blokke dog ikke må indeholde de efterfølgende indsigelser. Hvis angrebet så varer længere end indsigelsesperioden kan alle angribere med kontrakter af denne type altså stjæle al værdi på disse. Og dette gør altså angrebet endnu mere seriøst --- ikke mindst også fordi sådanne kontrakter er ret vigtige for layer 2-løsninger, hvilke igen er ret vigtige for brugbarheden af kæder såsom Bitcoin. 


Jeg har ikke noget konventionelt forsvar til dette angreb, som ikke bare kræver, at ofrene samt sympatisører (inkl.\ stake-holdere i kæden) simpelthen udspenderer angriberne. Denne løsning vil kræve ressourcer af alle parter, så den er ikke ideel. Men til gengæld \emph{kan} den sagtens være afskrækkende for angreb, og måske endda ret effektivt. Det kræver bare, at man i fællesskabet gør sådanne modangreb klar, så man ikke bliver taget på sengen af angribere, og så der altså ikke går for længe inden man får stablet modangrebet på benene (for dette vil koste mere, jo længere tid der går, før man reagerer). Selvom dette forsvar er ideelt i sig selv, så vil det alligevel være en virkeligt stor dæmper på fristen til at igangsætte et angreb. For hvis fællesskabet således tager sine forbehold i tide (og ikke bliver taget på sengen), så vil angrebet altså bare svare til, at man som angriber ofre penge på, at andre på kæden bliver nødsaget til at ofre ligeså mange penge, hvilket al sammen går til minerne\ldots\ Hm ja, selvfølgelig kunne de angribende så være minere selv\ldots\ Ja, det er faktisk et problem. Ellers er der nemlig som sagt ikke så meget at komme efter som angriber (hvis man altså kan se at fællesskabet er klar til et modangreb --- og selv hvis modangrebet primært finansieres af de ofrene selv --- selvom der nu helt sikkert vil være andre interesserede parter, for kursen vil alligevel tage væsentlig mere skade af et succesfuldt angreb, end hvad den vil med et (åndsvagt) ``det gør ligeså ondt på jer, som det gør på os''-angreb). Men ja, igen: Hvis angrebet finansieres af minere, så er det dog en lidt anden snak. Men i det mindste er det så sådan, at mit egentlige forsvar, som jeg vil foreslå, og som altså godt nok er en lille smule radikalt, men til gengæld effektivt, faktisk kan ende med at føre til en reducering i mining-lønnen, og dette er endda især en fare, hvis fællesskabet føler mistanke til, at minerne generelt ikke fortjener nogen tillid til at hjælpe med at forhindre sådanne angreb: Hvad skal man så bruge al den hashing til, hvis det alligevel er rimeligt spildt i sidste ende? Så hvis de eneste, der rigtigt kan have gavn af et angreb af min type her, er minere, så vil kæde-fællesskabet generelt bare have væsentligt færre grubler ved at implementere et halv-radikalt forsvar, som oven i købet kan reducerer mineres løn. 

Okay, og nu til min (halv-radikale) idé til, hvordan man mere effektivt kan forsvare sig imod sådanne angreb. Forsvaret består i, hvad jeg kalder en ``semantisk fork'' af kæden. Det svarer lidt til en såkaldt hard fork, men hvor en hard fork typisk vil få en kæde til at divergere til to forks, som aldrig forenes igen, så %vil en semantisk fork, 
kan en semantisk fork, 
som jeg har i tankerne, %typisk kun, 
sagtens bare, 
hvis størstedelen af fællesskabet går med på den, i perioder give et andet svar på, %hvad gælder om kæden og dens værdifordeling, end hvad den originale fortolkning af kæden vil sige. ..Hov nej, det passer faktisk ikke helt generelt..
hvilken fork af kæden er den ``gyldige.'' Pointen er nemlig, at man som eksempel kan vedtage i kædefællesskabet, at visse forks nu er ugyldige. Man bliver altså enige om et ekstra lag af restriktioner, som blokkene i kæden (tilsammen) skal opfylde. Hvis man så er mange nok i fællesskabet, der går med til og handler efter disse nye restriktioner, så vil det dermed heller ikke blive hver at investere i og/eller mine blokke til en fork, der ikke opfylder restriktionerne, også selvom denne måske ellers er den længste på det pågældende tidspunkt. Og hvis der således ikke vil være efterspørgsel efter mønter på den fork, der er ugyldig efter de nye antagelser, så vil en anden fork med tiden alligevel overhale den ``ugyldige'' fork, hvorved denne så også vil blive den gyldige fork selv efter den gamle fortolkning af kæden. Sådanne restriktioner\ldots\ Ah vent, dette er faktisk hvad man kalder en soft fork, det jeg beskriver her. Ok, så en \emph{semantisk} soft fork, er så bare et udtryk for, når bonus-restriktionerne i en sådan soft fork ikke implementeres alene ved at opdatere node-softwaren, men at soft-forken også implementeres ved at begivenheder på kæden kan udløse, at node-ejerne (inkl.\ minere) bliver promptet om at tage en beslutning mellem to (eller flere) forks af kæden. Folk kan så træffe forskellige valg, men det vil ikke være klogt at committe til dette valg, hvis man kan se at de fleste andre, og særligt køberne af mønterne, er mere interesseret i den anden fork. Og pointen med en semantisk soft fork er så, at der i definitionen af soft-forken kan erklæres en formel, men ikke nødvendigvis maskinforståelig, tekst, hvad bonus-restriktionerne skal være. Hvis majoriteten i fællesskabet så erklærer sig enig og engageret i soft-forken, og hvis de også viser sig at handle derefter, når sådanne prompts opstår, så vil en sådan ``semantisk soft fork'' altså lykkes. Bemærk, at node-softwaren jo så stadig skal opdateres, så at ejerne kan få og reagere på sådanne begivenheder. Forskellen er bare, at noden her kan få brug for ejeren til at træffe beslutningen, når man skal vurdere om en fork, der potentielt er kandidat til den gyldige fork, også er gyldig eller ej ud fra de nye restriktioner.


Hm, nu bliver jeg i tvivl om man overhovedet vil se denne idé som ``radikal'' i nogen særlig grad; det er jo bare en soft fork ret meget som andre soft forks, bare med den ene forskel, at node-ejerne kan være tvunget til at give input til noden, hvis denne gerne vil sikre sig, at noden hurtigst muligt følger trop med den formentlige konsensus, der vil indfinde sig. \ldots Hm, nå ja, og som jeg har skrevet i nogle kommentarnoter her tidligere i dag, så \emph{behøver} soft-forken ingen gang at være ``semantisk!'' Den kan faktisk godt implementeres, så noderne (eller `knuderne,' om man vil) selv kan træffe valgene automatisk! Så handler det bare om, at kæde-fællesskabet forbereder sig på potentielle angreb af den type, jeg har beskrevet, hvad vi forresten kunne kalde dusørkontrakt-drevne respend-angreb (\ldots eller måske ``chain reversing''-angreb, eller hvad man nu end konventionelt kalder sådanne angreb), og dermed klargøre en soft fork-software, der kan sørge for at frasortere angrebskæden\ldots\ Hm, hvilket man jo i princippet kan rette helt specielt imod de specifikke angrebs-forks\ldots\ Ja, hvorfor ikke? Der er jo stadig ikke nogen central beslutning, og der er bestemt heller ikke noget fordækt ved det, fordi man jo bare specifikt afværger ondsindede angreb (og i øvrigt får angriberne til at betale, fordi de så stadig spilder dusøren). Så ``helligheden''/``renheden'' af kæde-systemet bør altså ikke tage skade, så længe man bare i fællesskabet gør med til sådanne ``godsindede'' og ``ærlige'' soft forks. Ja; nej, det bør den ikke. 


%Uh, og man kan jo bare starte med en meget specifik formulering af semantik-"forken".. uh, og man kan endda bare vente på et angreb, og så formulere "semantik-forken" ud fra denne. Så skal man bare sørge for at gøre fællesskabet forberedt på en samentik-afstemning for at afværge angreb. Men ja, og desuden kan man bare sætte en udløbsdato på, hvad end man beslutter, for så vil kæden jo igen være "ren" (hvis man ser det på den måde) efter udløbsdatoen, for en semantisk "fork," vil jo i reglen kun være en fork midlertidigt fra den originale semantik (for alt andet end lige vil der så kun være penge i at mine til på en fork, hvis den opfylder de semantiske restriktioner, hvilket i sidste ende også vil gøre det til den længste fork). Dette er faktisk smart, for så har angribere også kun en begrænset tid til at udtænke angreb, der prøver at splitte fællesskabet (hvad end de så ville få ud af dette..). Jeg tror stadig ikke, det bliver svært at fomulere en semantik, hvor det ikke rigtigt vil være muligt og/eller værd at prøve at angribe den, men man ved jo aldrig; det er altid klogt at tage forholdsregler, for der kan jo \emph{altid} være ting, man har overset. 



Okay, jamen det var jo nemt, så. Jeg bør dog alligevel prøve at forklare min idé til, hvordan man med semantiske soft forks måske kan nøjes med bare én soft fork, og så\ldots\ Tja, men det er jo i princippet lidt det samme, for i begge tilfælde skal man jo alligevel i sidste ende promptes af fællesskabet, når der er et nyt angreb (hvad der dog i øvrigt bestemt ikke bør blive mere en ét af (og mere sandsynligt nul), da dette jo sandsynligvis vil mislykkes, og angriberne vil tabe en væsentlig sum tilsammen på det --- og endda uden at kursen sikkert vil tage nogen skade af det rigtigt). Ja, så den ``semantiske soft fork,'' jeg ville foreslå, svarer egentligt bare til, at man i kæde-fællesskabet bliver enige om for noget tid at forberede sig på et dusørkontrakt-angreb og, hvis det så skulle finde sted, om så at gå med til et modtræk i form af at godkende en ny soft fork, der ignorerer den angrebs-fork, der er ved at blive bygget --- eller som måske lige er blevet bygget; bare man når at reagere i nogenlunde god tid. Og hvis majoriteten af fællesskabet (og særligt også inkl.\ de handlende parter) går med til dette og handler på dette, så får man altså afværget angrebet --- og dermed også gjort det kosteligt for angriberne (på nær for minerne, som jo så bare kan indløse dusørerne uden at det koster fællesskabet eller kryptovalutakursen noget, og som derved også kan betragtes som uskyldige, hvis alle altså kan se, at angrebet alligevel ikke vil lykkes i sidste ende). 

Cool, jamen så har jeg måske ikke så meget mere at sige om det? Jeg gør jo herved opmærksomheden på en sårbarhed i PoW-kæder, som kan være ret alvorlige, hvis ikke man som fællesskab tager sig sine forbehold i tide. Men jeg har så netop også samtidigt en god opskrift på, hvordan man kan forhindre angreb, og endda på en måde, så hverken de potentielle ofre eller valutakursen vil komme til at lide skade af den. Og forsvarsløsningen indebærer ikke nogen hard fork af kæden; kun en soft fork bestående af en midlertidig omfortolkning af gyldigheden af forks, som ret hurtigt igen vil forenes med den oprindelige fortolkning af kæden, fordi ingen rigtigt til sidst så vil være interesserede i at mine nye blokke til den ugyldiggjorte fork. 

Jeg burde måske lige overveje, hvad der så gør sig gældende for PoS-kæder i forhold til sådanne angreb\ldots\ Ah, men det er stort set det samme billede. Her vil det så bare i højere grad være stake-holderne, der gerne skal gå med på soft forken --- hvad de jo sikkert også gerne vil, for de er jo om nogen ikke interesserede i, at et angreb skal lykkes på kæden. Fint. 

%(Jeg kan jo bare vende tilbage til denne sektion, hvis der er noget nyt, jeg kommer i tanke om, eller noget, jeg har glemt.)

%Jeg har gået lidt med nogle tanker om muligheden for et poisoned data-angreb, og jeg var faktisk ikke så glad for, hvad jeg nåede frem til, men nu er jeg så heldigvis nået frem til, at poisened data-angreb faktisk sagtens også kan afværges af fællesskabet. For da sådanne angreb vil i bund og grund vil handle om, hvordan trejdeparter interagerer med kæden, og hvordan trejdeparter lovgiver omkring kæden, så er det derfor også 100 \% oplagt at bruge trejdeparter til at sikre sig, at alle kan stole på de soft fork forsvar, hvor hovedparten af fællesskabet sletter blokke med poisoned data i (og hvor rednings-tredjeparterne, der har fået godkendelse til at opbavere dataen så uploader kontrakter, hvor de kan udfordres til et spil, hvor de skal kunne vise efterspurgte mellemstadier i en hashing-algoritme (f.eks.\ flere hashes taget på hinanden) over den originale blok). Og i øvrigt kan man, hvis man særligt vil undgå plaintext poisoned data i enden af kæden, bare bruge en semantisk soft fork, sådan at minerne bliver tvunget til at skille blokke med poisoned data i plaintext fra. Jeg synes dog ikke, at dette er værd at nævne i den renderede tekst, for på en måde er det ret trivielt, og andre vil sikkert have skrevet om de samme tanker. Så jeg lader bare disse små noter blive herude i kommentarerne. (15.10.21) 

Nå jo, jeg skulle jo lige nævne, hvordan en semantisk soft fork også kunne lede til, at mining-lønnen kan reduceres. Denne fork, hvis man kan kalde den det overhovedet, handler bare om at vedtage et system til at spore mønter, så det kan ses hvor de kommer fra (også selvom vi snakker procentdele af en mønt, hvilket jo bare kan ses som, at mønten er veksel til mindre mønter; så sporer man jo bare disse). Jeg ved det ikke, men jeg tror muligvis at begrebeet ``colored coins'' handler om det samme. Den semantiske fork, hvis man kan se den sådan, handler så bare om i fællesskabet (og altså særligt de handlende parter), at blive enige om en intern kurs, hvor man så vurderer mønter minet efter en hvis data for at være mindre værd end mønterne minet før denne dato, og den interne kurs kan så endda aftage --- eller vokse; måske for at modsvare forudbestemte fald i mining-lønnen, hvilket man så kan modsvare med tilsvarende hop i den interne kurs --- over tid, hvis man ønsker. Man erklærer altså i fællesskabet en ny fortolkning af værdien af mønterne. Hvis man så følger denne fortolkning, så satser man altså på i bund og grund, at fremtidens handlende vil gøre det samme (om ikke andet, så i det mindste indtil man selv sælger sine mønter, men sådan er det jo altid). Derfor skal der altså være en klar grund til, at man gør det. Man er altså afhængig af, at fremtidens handlende selv vil synes, at, ja, de nyere mønter er mindre værd. Og dette vil jo kun gøre sig gældende, hvis der er en eller anden årsag til, at de skulle være mindre værd, f.eks.\ fordi det generelt menes, at disse mønter ikke har bidraget til hele kryptovalutasystemet i samme grad. Her tænker jeg, at et realistisk eksempel kunne være, hvis man på det pågældende tidspunkt finder ud af, at man ikke har nær så meget behov for minernes arbejde, som man havde tidligere, og/eller kan se, at dette arbejde nu udgør et kæmpe ressourcespild. Herved kan der så være grobund for at en omfortolkning, som man kunne brande som en ``grøn'' version af valutaen kan vinde popularitet, både i samtiden, hvilket så vil hænge sammen med, at man også forudser, at fremtidens handlende også vil finde den "grønne" version mere attraktiv at samle på. For man kan jo sagtens forestille sig, at det ikke vil være værd at samle på en version af valutaen, som enten anses for at have været skyld i et kæmpe ressourcespild, eller som anses for at ville have været skyld i et kæmpe ressourcespild, havde den vundet over den mere ``grønne'' valutafortolkning. 

Og lige for kort at samle op på en tidligere tanketråd, så vil det også for mig give god mening, hvis man kan se at en anseelig del af kædens minere på et tidspunkt selv har deltaget i et angreb på egen kæde, hvilket har forceret stake-holderne til at betale alle minere flere (dusør-)penge, jamen så må der bestemt også blive grobund for en (retfærdig) omfortolkning af valutaen, hvor mineres løn reduceres. Nu ikke at jeg tror, at dette bliver en realitet alligevel, uanset om dette kan være en afskrækkende faktor eller ej (for nu mener jeg jo som nævnt, at sådanne angreb sagtens kan afværges på anden vis alligevel, så minerne ville derfor højst sandsynligt ikke få noget udbytte af dette). 

Hm, jeg burde nok også, som jeg har gjort her, nævne det her med ``grønne omfortolkninger'' af valutaen i den tekst, jeg vil udgive nu her i første omgang. 



%(16.10.21) Uh, nu tror jeg faktisk, at jeg lige er kommet på en forandring af min donationskæde-idé, som nok giver mere mening..! Jeg tænkte nogle løse tanker i går aftes, og fik jeg lige samling på det nu her (efter at have ryddet op efter morgenbordet --- min mor var lige på besøg). Måske ville en mere simpel version af idéen være, at man bare enes om en politisk overbevisning om, at stater i fremtiden bør betale løn bagud, ikke bare til bidragsydere selv, men også til de mennesker, der har doneret tidligere til.. Hm, jeg ved ikke, om man kan kalde det "mere simpelt," det her, man lad mig nu lige fortsætte tanken.. ..Hm, jeg ved godt, jeg ikke har forklaret de nævnte tanker færdigt endnu, men nu kom jeg lige på, at man måske kunne lønne donorer ud fra et generelt princip om at tidlig betaling til bidragsydere er "mere værd" end senere betaling.. Og nu ved jeg godt, at jeg ryger lidt tilbage til et tidligere stadie af idéen (nemlig det stadie, som mine 11/6-noter også handlede om (og inden da)), men lad mig nu lige se.. ..Hm, måske er det ikke en helt dum idé, det der med at se på, at donationens værdi kan regnes for af større værdi, hvis den gøres tidligt.. ..Så at man vedtager at forsinkede donationer skal tilføjes ekstra kompensation, men at denne kompensation så i stedet kan gå til private donorer, hvis private donorer allerede har givet den oprindelige bidragsyder penge.. Man kunne endda lægge et ekstra beløb oveni, hvis kompensationen så gives til en privat donor (en donations-mellemmand) som en slags løn for det arbejde, der må ligge i at holde øje med, hvilke bidrag der kan være værd at investere i, og som tak for risikoen, de har taget, for hvis de får betalt mere end, hvad man i sidste ende vil donere til bidragsyderen, så må det nok bare være på egen regning; der gives kun refundering og ekstra-løn/kompensation for donerede beløb, hvis de er under eller lig, hvad man beslutter at bidragsyderen fortjener. Hm, så er det så bare vigtigt, at donationerne nøje får registreret deres tidspunkt, så man kan ordne dem i tid, og så skal donorer selvfølgelig kunne trække deres donationer tilbage, hvis andre ender med at komme før dem. ..Umiddelbart ikke en helt dum idé..! (Selvom det altså dog stadig ikke rejser sig til niveau af noget, der skal udgives i første omgang; men stadig..!) Og pointen med idéen er så, at hvis man tror på, at samfundet vil indføre dette i en nær fremtid, og tror på, at de vil indføre det med tilbagevirkende kraft (hvad man jo også klart bør), så kan man altså tage forskud på systemet og begynde at donere penge til "bidragsydere" (f.eks. open source-programmører, frivillige arbejdere, filantroper, iværksættere, hvor det er svært at monetisere, osv. osv.) allerede med et vist håb om, at disse donationer så på et tidspunkt vil blive refunderet med en ekstra kompensation oveni. Ja, det er da en cool nok lille idé --- og ja, jeg kan jo stadig nævne en implementation af den, som kører over en blockchain --- som jeg så kan nævne i forlængelse af min civilforeningsidé (måske sammen med min idé til "kundedrevne virksomheder," som jeg jo også stadig tror, kan have en lille smule gennemslagskraft i sig (selvom min civilforeningsidé nu har overtaget lidt)..).   



\phantom{\\}\noindent
\textbf{Donationskæde-idé}\\
Jeg havde jo en idé (og har jo stadig) til en ``donationskæde,'' som jeg har kaldt den, men nu (d.\ (16.10.21) i dag) er jeg faktisk kommet frem til, at idéen nok fungerer bedre på en anden, og efter min mening mere simpel facon. Jeg tror nemlig, at jeg i stedet bare vil foreslå den mere som bare overordnet bevægelse igen. Idéen minder så mere om mine tidligere versioner af den, men er nu, mener jeg, mere simpel. Idéen siger nu bare: Hey, kunne det ikke være smart, hvis samfundet i fremtiden belønnede, hvad jeg har refereret til som ``bidragsydere'' (i.e.\ folk, der f.eks. laver open source-programmering, nødhjælpsarbejde, filantropi-arbejde osv.), bagud, så de kan få belønninger for gammel arbejde, de ellers ikke har tjent på? Her kunne man så med fordel have det sådan, at folk, der får lov at vente længe på deres endelige belønninger fra samfundet, bliver kompenseret for forsinkelsen. Endvidere kunne man så have tredjeparter med i billedet, som kan være hurtigere og donere tidligt til sådanne ``bidragsydere'' (som i: bidrag til samfundet, underforstået). Herved bør kompensationen således tilkomme disse tredjepartsdonorer i stedet, i hvert fald i det omfang de har dækket belønningssummen med deres donationer. Hvis der er flere donorer til det samme ``bidrag,'' så må det være op til donorerne at have registreret præcist, hvornår donationen blev givet, så man kan uddele kompensationen i rækkefølge, hvis det samlede donationsbeløb overskrider den endelige belønning. Da dette så kunne skabe en situation, hvor sådanne tredjeparts-donorer/-investorer kunne blive tiltrukket til at gøre arbejdet med at vurdere ``bidrags'' værdier samt at tage risikoen ved at donere og så håbe på refundering og kompensation, så kunne man måske endda med fordel beslutte sig for i samfundet, at der skal lægges en yderligere kompensation oveni, der gør op for dette arbejde og denne risikotagning. Bemærk i øvrigt, at tredjeparts-donorerne (eller hvad, vi skal kalde dem) jo vil tage en ekstra risiko, hvis samfundet ingen gang her vedtaget denne ordning endnu, men hvor man altså bare satser på, at de i en nær fremtid \emph{vil} gøre dette, og at de vil indføre ordningen med tilbagevirkende kraft, når de gør det. Pointen med idéen bliver så, at påpege, at man da sagtens kan forestille sig, at samfundet vil udvikle sig sådan i en nær fremtid (især måske efter at civilforeninger vinder frem), og hvis mange nok så kan blive enige om, at dette kan være en sandsynlighed, jamen så kunne dette give folk, som er villige til at donere penge filantropisk, ekstra motivation for at gøre sådant; fordi der så endda kan være en chance for refusion og kompensation fra samfundet i en nær fremtid. 

Og den næste del af idéen, som så er et valgfrit tillæg, er så, at man måske kunne lave et blockchain-system, hvor folk kan registrere både bidrag og omtalte ``tredjeparts-donationer,'' og så veksle dem til digitale tokens, man kan samle på og handle med. Hm, i det hele taget kunne det måske være en idé, det her med en blockchain, hvor ``bidrag'' og donations-bidrag kan veksles til tokens, man kan samle på og handle med. For der er jo allerede et marked for blockchains, der gør det samme med kunst osv., så hvorfor ikke også med bidrag til samfundet? 

Man ja, det er altså min donationsidé i dens nuværende stadie. Jeg vil ikke udgive den her i første omgang, og når jeg gør, så bliver det nok bare måske i forlængelse af min civilforeningsidé. \ldots Hm, eller man kunne måske godt udgive sidste del af idéen om en simpel ``bidrags-blockchain,'' som lægger sig op af eksisterende blockchains såsom ``kunst-blockchains'ne,'' hvis vi kan kalde dem det, på et tidligere tidspunkt\ldots? Hm ja, det kan jeg jo lige tænke over. (Og fordi jeg nu nemlig har en løsning til NFTs, hvor man ikke behøver at ``brænde kloden af'' for at opretholde dem (e.i.\ mine ``grønne'' blockchains, hvor man giver sig selv lov til som fællesskab at soft-forke væk fra alle angreb, der tydeligvis prøver at reverserer kæden med ondsindede/skadelige hensigter), så gøres alle sådanne typer blochchains også bare så meget desto bedre (hvis de altså så skifter over til en sådan ``grøn semantisk fork'').)

Ja, jeg tror faktisk, at jeg vil nævne denne simple version af mine ``donationskæde-idé'' i form af en kæde, hvor bidragsydere (og donorer) kan registrere bidrag (og donationer), og hvor de så kan få ``vekslet'' dem til NFTs (som dog ikke behøver intensiv mining). Dette giver så nemlig i øvrigt også god anledning til at nævne ``grønne soft forks'' igen, og endda i en sammenhæng, hvor de giver endnu mere mening (for hvem gider at stjæle tokens, der repræsentere \emph{gode} bidrag (eller kunst for den sags skyld); hele systemet bygger jo alligevel på en vis vilje til at donere penge til skabere/bidragsydere, så hvad skulle en token så være værd, hvis man har stjålet den; dette vil jo gå imod hele konceptet).  



\subsubsection[Civ.f.]{Min idé til de såkaldte (for nu) civilforeninger}
%"%Nu er jeg så blevet lidt i tvivl, om jeg i virkeligheden skulle snakke om dette først, men det næste punkt er ellers så, hvad jeg har kaldt "civil-foreninger" hidtil. Tanken er simpelthen at oprette foreninger med det formål at finde frem til handlingsløsninger, som hele eller dele af medlemskaren kan begynde at udføre, hvis de vil forbedre deres økonomiske situation eller deres livssituation i det hele taget. I modsætning til et parti eller en fagforening er fokuset altså ikke på at indvægle nogle gode ledere, der så skal bestemme for gruppen, men i stedet er fokusset direkte på viden og på løsningsforslag. Medlemmerne donerer så nogle penge løbende, enten for at hyre folk til at analysere medlemmernes situationer og finde frem til løsninger, eller for at belønne (bagud!) folk, der har indsendt gode idéer og/eller faktuelle indsigter til gruppen, som gruppen kan få gavn af. Ja, idealet vil faktisk være, hvis man fokuserer meget på at tiltrække sidstnævnte form for bidrag; at man simpelthen udlover dusører for gode idéer og god viden, og så ellers selvfølgelig sørger for at efterkomme alle disse løfter, så man fortsat kan tiltrække idéer og viden i fremtiden. Medlemmerne skal så selvfølgelig kunne gå sammen (i undermængde-grupper) om at blive enige om at udføre en handling. Det bør så selvfølgelig være sådan, at folk, der får gavn af et løsningsforslag eller en viden, bør donere til disse bidrag, imens folk, der af en eller anden grund ikke kan få gavn af forslaget/informationen, selvfølgelig ikke behøver at donere i samme grad (og sandsynligvis slet ikke, hvis de slet ikke får noget ud af det). Løsningerne som kunne foreslås til sådan en forening kunne være alt lige fra, hvis vi tænker særligt på økonomiske ting, at forhandle sig til grupperabatter for foreningen, at oprette en boligforening til stille og roligt, én bolig ad gangen, at opkøbe medlemmernes boliger til sig selv, således at den effektive husleje bliver mindre (hvorved de sparede penge så kan gå til at opkøbe flere boliger), at analysere markedet og finde frem til monopoler og andre brancher, hvor der er et for stort markup, og så oprette et konkurrerende firma (som medlemmerne i foreningen jo passende så ikke bare kan investere i, men også støtte som forbrugere).. (Og måske skulle man netop bare nævne denne ellers ret vægtige idé lidt som en sådan forbipasserende sætning som den..) Det kan også være at genbruge ting internt i gruppen i stedet for at smide ud, hvorved dem der giver videre så kan få en lille bonus, alt efter hvor lang tid tingen holder (hvilket kan registreres i gruppen ved at modtager igen må spørge om samme type ting --- hvilket nemlig også lige skal koste modtageren en lille smule, så man ikke kommer ud for, at folk spørger uden behov). Det kan også være at støtte firmaer (som forbrugere og/eller som investorer), der behandler miljøet og dets arbejdere godt, og altså særligt hvis dette inkludere foreningens medlemmer (og/eller at medlemmerne færdes i og benytter samme miljø). "Grupperabatter" kan i øvrigt dække mange områder.. Og når vi snakker at ivestere i markeder, hvor der kunne være mulighed for forbedringer, så behøver man nok heller ikke samme investering i reklame, hvis idéen om at skabe firmaet kommer fra foreningens medlemmer, hvis medlemmerne er medejere til dels af firmaet (og må regnes for at have fuld gennemsigtighed.. have fuld indsigt i.. have mulighed for at få indsigt i alle sager i firmaet), og hvis de endda alligevel har sat sig for at støtte firmaet som forbrugere. Jeg har ikke styr på, hvor meget af et typisk firmas bugdet går til reklame og PR, så jeg ved ikke hvor mange penge, der er at spare her, men det må da være nogen. Og jeg vil også godt pointere, at selvom der kun måske er en lille smule at spare for folk, så betyder de små procenter altså mere end bare lige. For lad os sige, at man efter tyve år bare kommer på, at medlemmerne sparer 1 \%, end hvis det ikke var for foreningen. Det lyder jo ikke af vildt meget; man vil let kunne spare det samme bare ved at ændre sine egne vaner en smule. Jo, men hvor går de penge hen? Hvis de bare ryger tilbage i økonomien, så er alt fint nok, men et problem i vores nuværende samfund er, at de procenter vor gængse firmaer tjener ekstra i overskud, de går ofte til rige mennesker, der ikke er i stand (og/eller ikke har lyst, men ofte er de bare ikke rigtigt i stand; de kan godt købe flere ejendomme, ja tak, men det flytter bare formuen; de har svært ved at forbruge pengene ordentligt, så de ryger tilbage ned på hænderne af folk med meget lavere indkomster og formuer (så vidt jeg lige ved --- det er i øvrigt slet ikke sikkert, jeg vil have dette med, men nu brainstormer jeg det lige). Så lad os sige at 1 \% om året af folks forbrug (hvilket for mange svarer ca. til deres løn) går til rigmands-pengekasser (billedligt talt).  Efter 30 år vil tæt på 30 \% af pengene i omløb i økonomien så nu ligge indespærret i rige folks kister. Dette kan så bruges til at opkøbe folks ejendomme, så de, yay!, kan få nogle flere penge at købe varer for igen, og hvilket så igen kan begynde at gå tilbage til rigmandspengekisterne igen. Og sådan kan bare 1 \% hurtigt gå hen og blive mange procent.. Hm, eller har jeg regnet noget galt her..? Hm ja, for medmindre deresl løn falder, så vil de have råd til det sam.. Tja, eller boligpriserne kan jo lige netop stige.. ..Hvilket så bliver en fast ekstra udgift, i hvert fald for alle udlejere, hvilket effektivt set kommer til at give dem mindre løn.. ..Og lån for at folk alligevel kan leve udskyder jo så bare krisen.. Hm.. ..Ja, det er jo netop dette, der er problemet med vores nuværende økonomi; vi tænker ikke over langtidskonsekvenserne vil at tillade dem, vi køber vores varer ved (og lejer vores ting ved og giver os faste services osv.), at blive ved og ved med at tage en profit. Selvfølgelig skal folk have profit for at have indført noget nyt og godt til markedet som holder, og som forbrugerne kan blive glade ved, men det må bare ikke få lov til at blive ved og ved og ved og ved og ved med at generere profit --- og slet ikke når de profiterende på ingen måde lover at forbruge deres æg fra guldæg-hønen igen, men bare bruger dem til at opkøbe flere og flere andre værdier. Hvis der så nu var perfekte markedskræfter, så ville vi så slet ikke have dette problem, for så snart patenter og rettigheder er udløbet, så vil markedskræfterne træde ind, andre firmaer vil melde sig på banen og overskudet vil i sidste ende presses ned til noget næsten forsvindende. Så spørgsmålet, hvis vi gerne vil komme den kedelige beskrevne udvikling til livs, er: hvordan får vi forbedret markedskræfterne? Perfekte markedskræfter kræver jo eftersigende, at folk handler perfekt rationelt ud fra egeninteresser. Jamen måske er dette lige netop vejen til en løsning: at få folk til i højere grad at træffe nøjagtigt de beslutninger som handlende individer, der hjælper dem mest muligt. Med andre ord ved at få dem til at begynde at gå op i den \emph{ene procent}!.. Hm ja, kunne man måske ikke godt skrive noget a la dette..? Jo, måske. Det virker faktisk meget fornuftigt. Nå, men jeg skal så også lige ind på eksempler, hvor vi ikke snakker økonomiske tiltag. Jeg kunne så prøve at finde og nævne nogle mere psykologiske områder, eller bare privatlivs-områder i det hele taget, hvor folk kunne have gavn af på en måde at "analysere sig selv" som gruppe noget mere.. ..Ja, jeg må kunne finde nogle små ting lige at nævne, bare som forslag. Nå ja, og jeg bør nævne noget omkring politik. Ja, kort kunne man jo nævne, at løsningerne også kunne være forslag til politiske handlinger, som medlemmerne kan gøre. Jeg kunne måske så selv nævne nogle forslag om, at man bør langt mere kritisk og pragmatisk... Åh, lad mig lige vente to sekunder med det emne faktisk.. Lad mig i stedet lige tænke på, om der er andre ting, jeg skal nævne her. Ret nice, hvis jeg kan klemme min "forbrugerforenings"-idé ind her som en slags sidenote til civilforeningerne.. Tja, men så bliver jeg måske så til gengæld ret politisk, hvis jeg tager de sidste ting her med i teksten (og altså ikke lige venter en omgang til at udgive disse idéer i en senere tekst).. Men ja, måske går det alt sammen.. Okay, flere emner udover omkring politisk aktivitet?.. ..Hm, jeg burde nok også huske at nævne, at de rige jo må være mindst ligeså interesserede i at få vent udviklingen mod noget mere stabilt, medmindre de er totalt tegnebogs-junkies (/formuevækst-junkies). Jeg bør vist også lige pointere, at disse civil-foreninger (eller hvad dælen vi skal kalde dem) kan være arbitrært omfattende *(som i arbitrært store; de vil kunne gå på tværs af landegrænser endda).. Nå jo, jeg tror forresten muligvis, at jeg vil vente med at nævne idéen omkring, at man jo muligvis ville blive i stand til at uddele skyld til folk, der ikke deltager i og/eller ikke har lovet noget på kæden. Det er sikkert ret unødvendigt.. Tja, nu har jeg godt nok lige fremhævet vigtigheden i marginalerne.. Men jeg tror nu alligevel ikke, det bliver nødvendigt i starten, hvis man tænker på systemets udvikling, og jeg tror muligvis samtidigt, det vil virke lidt usmageligt for mange.. Jeg vil selvfølgelig nævne det efterfølgende, men måske bare ikke i min første tekst her (som jeg laver brainstorm-udkast til nu --- eller rettere: lige nu laver jeg overordnet brainstorm over, hvad mit brainstorm udkast (eller 0. udkast) skal indeholde, he). ..Nå, men det er altså lige før, jeg tror det var det. Der skal nok være nogle små ting, men det gør nemlig heller ikke noget, at jeg ikke får alle pointer på kryds og tværs med; det er nok endda fornuftigt ikke at gøre dette. Jeg vil i hvert fald slutte for i dag/aften, og så kan det jo altid være, at jeg kommer på mere at tilføje til i morgen eller i morgen. Alligevel en ret god dag med disse idéer og indsigter. (28.09.21) 
%(29.09.21) Jeg kan også lige nævne forsikring og forsikringsselskaber, som noget hvor man måske kunne spare penge (i en stor civilforening). I øvrigt kan man sagtens som forbrugere presse firmaer til at at handle mere og mere i forbrugernes favør, hvis der bare er mere en ét firma inden for branchen, for så kan man bare true (og følge op på det, hvis det skulle komme til det) med forbrugerboykot, hvis de ikke laver nogle små (og man kan nemlig bare starte i de små, så man ved, de vil gå med til det) ændringer. Når man så har gjort dette kan man begynde som forbruger at favorisere dette firma mere til fordel for konkurrenterne, og endda ligefrem begynde at nærmest boykotte konkurrenterne, indtil de laver samme og gerne ekstra ændringer også. Og på denne måde kan det så gå i ring, så man hele tiden presser firmaerne langsomt ned på et niveau, hvor overskuddet minimeres, og hvor servicen forbedres, gennemsigtbarheden (mangler ordet for det) øges, så firmaerne til sidst bliver nærmest helt gennemsigtige, og hvor miljøhensyn og arbejderes lønninger passer bedst muligt med civilforeningens (eller rettere den gruppe af foreningen, der har valgt at tage aktiv handling) interesser. Nå ja, og jeg kan også nævne.. sygdomsklassificering.. eller måske det skulle nævnes mere under folksonomies..? Tja, jeg kan jo bare nævne det enten her eller der. Okay, og så mangler jeg jo også lige hængepartiet fra i går (som jeg nærmest glemte igen): Politisk aktivitet. Hm, dette lægger godt i forlængelse af at presse firmaer til forbedring.. Man bør også presse sine politikkere til gennemsigtighed på samme måde, i hvert fald hvis man ikke gør andet.. Men ja, i virkeligheden bør man vel bare starte mere fra grunden bare og så oprette et totalgennemsigtigt parti (eller partier, hvis civilforeningen er spredt over flere lande). Dette betyder at alle beslutninger tages i form af offentlige overvejelser og offentlige interne diskussioner. Man sørger altså for det første for at opbygge og vedligeholde en offentlig hjemmel *[/manifest] i partiet, som alle medlemmerne løbende kan stemme om og justere. Man sørger så desuden for, at alle arbejdere/ledere i partiet optager/streamer videoer og/eller skriver rapporter om samlet set alt, hvad de foretager sig i deres funktion. Og her skal det altså fremgå, hvordan dise handlinger stemmer over ens med hjemlerne. Hvis medlemmerne så ikke er enige, skal de kunne reprimandere og også simpelthen erstatte arbejderen lynhurtigt. Medlemmerne kan så overvåge og diskutere effektiviteten af partiets hjemmel og dens udførsel og kan prøve at finde forslag til nye handlingsløsninger, hvis man tror, der er ting, man kan gøre for at opnå mere (hvilket jo passer til selve idéen omkring en civilforening, nemlig at det er fokuseret på viden og idéer til nye handlinger, man kan gøre). Og bum, så har man et parti, der arbejder efter medlemmernes bedste (i stedet for at man bare en gang imellem, med store mellemrum, vælger en række jakkesæt og tandpastasmil (i høj grad) med de gode ord i munden, og så ellers bare lader dem gøre, hvad de \emph{egentligt} har lyst til / er motiveret til indtil næste gang, man skal stemme). Så ja, dette er, hvad man bør gøre, men ellers kan man selvfølgelig også boykotte politikkere på skift og presse dem hen til der, hvor man gerne vil have dem (men dette er jo selvsagt bare ikke nær så effektivt).. Okay, men det var vel så det for denne brainstorm. Flere idéer ryger nok ned under mine "huskenoter" nedenfor. ... Nå ja, jeg kunne dog også lige hurtigt nævne her, at jeg måske også bør komme med min kommentar omkring, hvordan firmaer jo netop har overskud til at gå op i de små procenter, så der er ingen grund til, at en stor samling borgere/almindelige mennesker/civile ikke ville få gavn af, hvis man gjorde noget tilsvarende (og (mere eller mindre) hyrede folk til at gennemgå de økonomiske situationer og finde handlingsløsninger)."

Hm, det er måske lige lovlig rodet skrevet, og der også også punkter, hvor jeg nok bør tilføje ting, men ellers var mine brainstorm-noter ude i kommentarerne vist egentligt meget gode\ldots\ Især hvis man springer lidt hurtigt hen over tankestregerne. Men lad mig nu alligevel sørge for at skrive det her igen også. 

Lad mig bare for denne idé også tage afsæt i mine vægtede brugergrupper, hvorfor ikke? Vægtede brugergrupper er jo bare en udvidelse af normale (mere bare ``medlem eller ikke medlem''-)brugergrupper, og selvfølgelig vil det ikke være dumt, hvis foreningerne kan foregå online. Jo, man kan også implementere idéen med mere normale foreninger, og det vil bestemt heller ikke være dumt, hvis man sørge for at mødes i foreningerne, men jeg tænker nu alligevel, at det nok vil være meget godt, hvis jeg også lægger kraftigt op til, at foreningerne kan foregå online i høj grad\ldots\ Ja\ldots\ Nå, men lad mig bare tage afsæt i brugergrupperne her, og så kan jeg altid vende tilbage til dette spørgsmål. Okay.

Jeg har snakket om brugergrupper, hvor visse holdninger og/eller interesser kan være stærkt repræsenteret, hvilket man så f.eks.\ kan bruge til at finde ressourcer, der passer godt på ens holdninger og interesser (eller til at se, hvad rør sig i andre grupper, man ikke selv tilhører i så høj grad). Jeg har også forklaret om nogle andre måder at bruge sådanne grupper på, f.eks.\ til videnskabelige diskussioner og debat. Jeg har i disse forbindelser nævnt af brugergrupperne eksempelvis kan repræsentere befolknings- og faggrupper og, ja, interessegrupper generelt. 
Den idé, jeg vil forklare om her, handler så om kort sagt at bruge sådanne online interessegrupper til at finde frem til forslag, ikke bare som kan forbedre brugbarheden og brugeroplevelsen af at browse ressourcer på internettet, men som kan forbedre ting i medlemmernes liv generelt. 

Sådanne forslag kan handle om psykologiske emner, f.eks.\ hvordan man forbedrer forhold og interaktioner i ens privatliv (hvilket så kræver en brugergruppe, der fokuserer på psykologiske aspekter og bringer folk sammen, som har samme af sådanne). Det kan være forslag til, hvilke nogle aktiviteter man kan lave (måske sammen med andre i gruppen), og hvilke mål, man kan sætte sig selv i en fritidsliv. Det kan også være økonomiske forhold (hvilket så ville fordre mere økonomisk orienterede brugergrupper) f.eks.\ omkring medlemmers arbejdssituation eller omkring deres privatøkonomiske situationer (hvilket altså meget vel kan inkludere medlemmer på totalt på tværs af faggrupper, men som måske har lignende boligsituation eller lignende situation på anden vis). Det kan også være rigtigt brede grupper, som f.eks.\ kunne ønske ting som at modarbejde klimaforandringer eller andre ting, hvor forandring alt andet end lige kræver mere fælles handling (og hvor man så altså kan tro på, at man \emph{kan} skabe en forandring, hvis altså bare man kan samles om at gøre en fælles indsats).

En vigtig pointe, jeg så har, synes jeg, om dette, er at det så vil være rigtigt fordelagtigt, tror jeg, hvis man i sådanne grupper går meget op i processer til at finde løsnings-/handlingsforslag. Så med andre ord mener jeg, at man bør gå meget op i, hvad skal man sige, den ``videnskabelige'' del af det\ldots? Altså man skal gå meget op i processen omkring at finde på, udveksle og diskutere idéer til løsnings-/handlingsforslag i gruppen. 

I modsætning til f.eks.\ fagforeninger skal der nemlig ikke være tale om grupper, hvor man udnævner en ledelse til at bestemme over, hvilke ting man skal gøre, heller ikke selv når det kommer til foreninger, der handler om at gøre en fælles indsats for noget. Ikke direkte i hvert fald; man skal i sådanne grupper ikke udnævne en ledelse, der kan styre gruppen. At blive enige om at foretage handlinger skal mere, som jeg forestiller mig det, ske i et andet lag, hvor gruppemedlemmer så frit kan melde sig til eller fra visse handlingsforslag, som dele af gruppen går ind for at udføre. Og i det grundlæggende lag af foreningerne skal det nemlig så bare handle om at fremme selve udviklingen og diskussionen af diverse forslag. Og denne del kræver selvfølgelig ikke nogen central ledelse; udveksling og diskussion af idéer kan jo med fordel foregå frit. 

Men der hører sig mere til idéen end bare at oprette diskussionsgrupper. Jeg mener ydermere, at det så ville være en rigtig god idé, hvis man i disse ``foreninger'' (eller nærmere bestemt ``civilforeninger,'' som jeg overvejer at kalde dem) også går meget op i at donere penge til de medlemmer og/eller konsulenter, der kommer med gode løsnings-/handlingsforslag --- hvor det selvfølgelig er donorerne selv, der vurderer, hvor gode forslag er.

Foreningerne kan både donere penge til at hyre konsulenter til at finde frem til løsnings-/handlingsforslag, men man kunne også, med fordel tror jeg, i høj grad donere penge til tidligere forslag, sådan at man har et meget åbent forum, hvor folk frit kan komme med forslag, men hvor foreningen så sørger for at belønne gode forslag gavmildt, også selvom de ikke har hyret vedkommende. Og jo mere gavmild en forening er til at belønne gode forslag, desto idéudviklingsarbejde kan man således tiltrække (fra medlemmer i gruppen og fra udefrakommende). Og da jeg mener, at der vil være rigtigt mange tilfælde, hvor enkeltpersoners idéer og idéudviklingsarbejde kan blive enormt meget værd for den samlede forening (generelt), så tror jeg som oftest det bestemt kan betale sig for foreningen at donere gavmildt til udvikling af nye idéer.

Når man så går sammen i en delmængde af foreningen om at udføre en handling, så kan dette jo godt ofte kræve en form for administration eller ledelse, eller i det mindste kræve en eller anden form for løbende arbejde. Herved bør samme mængde af medlemmer altså så også bare aftale, at donere/betale penge løbende, så man kan hyre folk til at varetage disse opgaver (selvfølgelig hvis man altså kan se at hyren er det værd, ift.\ hvad medlemsmængden får ud af foretagendet). Så donerede penge fra medlemmer kan altså både gå til idéudvikling (hvilket er en meget vigtig del af hele idéen) og så også til at vedligeholde foretagender igangsat af foreningen.

Jeg har været inde på, at jeg mener, at donationsbeløbene bør være frivillige og særligt at donorerne skal kunne vælge, hvilke idéer og foretagender de selv vil støtte mest --- hvilket jo f.eks.\ bl.a.\ kan være ud fra, hvad de mener kommer dem selv mest til gavn. Man skal så måske bare passe på, at det ikke bliver for fristende for medlemmerne at skrue ned for deres donationer, hvis de alligevel får nytten fra andre brugeres donationer. Her kan man så i visse tilfælde sørge for, at nyttet fra foretagender i større grad går til de medlemmer, der har doneret over en vis grænse (eller hvordan man nu vælger at opdele det). Og ellers er der også en anden ting, man kan gøre for hjælpe med at holde donationsniveauet, og det er at medlemmer kan lave afhængige donationer, hvor hele beløbet kun bliver doneret afhængigt af, hvor gode resten af den relevante mængde af medlemsskaren er til at donere. (Jeg ved ikke, hvor effektiv dette vil være i praksis, men det kan man jo se på.) Nå ja, og ellers kan man selvfølgelig også gøre ting såsom bare at fremhæve gode donorer og/eller have offentlige oversigter over donationerne osv.  

Dette var en rimeligt fyldestgørende oversigt over idéen. Jeg bør bare lige nævne, at selvom det grundlæggende lag af foreningen bare handler om idéudviklingen og -udvekslingen, og ikke om at udføre handlingerne, så bør man selvfølgelig stadig sørge for i foreningen selv at implementere gode systemer, så brugere nemt kan gå i gang med processen i at igangsætte en handling/løsning. Med andre ord så skal foreningen altså ikke forvente, at medlemmerne igangsætter handlinger \emph{via} protokoller og systemer, som foreningen selv har stået for at implementere, men derfor bør foreningen stadig gøre dette (i.e.\ stå for at implementere sådanne). Men ja, det var ellers idéen kort fortalt.

Jeg kunne så komme med nogle eksempler på, hvad man kunne bruge sådanne foreninger til (altså hvilke ting, man måske kunne udføre), men jeg har jo allerede nævnt ting i min ``Noter omkring muligheder for det fremtidige marked generelt''-sektion, og man kan jo også læse mine brainstorm-noter til denne sektion (som jeg dog ikke gider at indsætte i den renderede tekst), som også giver lidt eksempler. \ldots Ja, det må være fint sådan her. :)

%Hm, måske kunne man nævne min model-app- / model-afstemnings-idé, hvor folk kan se hvad vej vinden blæser i gruppen. Hm, lad mig lige tænke over, hvad den, nu lidt gamle, idé kan...

\ldots\ Hm, jeg overvejer lige, om ikke man også burde foreslå noget a la mine ``dynamiske model-afstemninger'' for civilforeningerne, således at medlemmerne altså får et godt online miljø, hvor de kan gå ind og stemme på forskellige løsningsforslag og se, hvad andre i foreningen mener. Jo, nu hvor jeg skriver det, virker det jo egentligt klart: Ja, det vil være en rigtig god idé, hvis folk kan gå ind online og give deres meninger og generelle stemninger omkring forslag tilkende, og at medlemmerne så herved også kan få god oversigt over, ``hvad vej vinden blæser'' så at sige i foreningen, og altså hvilke forslag der er stemning for og ej. Jeg tror bare, jeg er kommet til at tage denne mulighed for givet, men det er den jo ikke helt. Så den er bestemt værd at fremhæve. Jeg kan mærke, at jeg også er kommet til at tage for givet, at medlemmerne også kan gå ind og diskutere ting med de værktøjer, jeg har beskrevet ovenfor, men dette er jo også en ting, jeg bør nævne eksplicit: Ja, foreningen skal inkludere sådan en online platform, i.e.\ hvor folk kan diskutere løsnings-/handlingsforslag, og hvor de også kan give deres meninger/stemninger og deres handlingsparathed til kende. Fint. 


*Ah, lad mig lige også hurtigt nævne, at iværksættere og/eller politikere, der har en idé, som de tror, en civilforening kan drage gavn af, jo så også selv kan tage initiativet og give deres forslag til civilforeningen, som så kan tage stilling til, om de tror, vedkommende kan levere et produkt, der er bedre for medlemmerne end det, de ellers har adgang til.


%Forslag kan belønnes bagud.. ("udlove dusører"..) (tjek)

%Det behøver ikke kun være nye forslag; det kan også være administrativt og løbende (konsuent-)arbejde for at holde gang i forslag, som dele af gruppen er ved at udføre.. (tjek)

%..nok lidt mere om at gå sammen om at udføre en handling/løsning.. ...måske nævne noget om afhængige donationer.. (tjek)

%Hm, videre kunne jeg komme med nogle eksempler, og jeg kunne også forklare lidt om det her med idéen at fokusere mere på marginalerne som en forening, der varetager privatøkonomierne (og som ser på, hvordan man kan være politisk forbrugere på en gavnlig måde).. (nvm)

%Og ja, hvis jeg så nævner disse ting, så kan jeg måske også godt komme med min påstand, om at dette måske kan være en ``kur til kapitalisme''.. (nvm)

% Og hvis/når jeg går så langt, så kan jeg jo også forklare om, hvad man kan gøre for at få bedre repræsentation og gennemsigtighed --- og mere kontrol fra bunden --- i politiske partier.. (nvm)






\subsubsection[Andet]{Andre ting, jeg gerne lige vil fremhæve fra dette notesæt generelt --- ikke nødvendigvis bare omkring emner fra de ovenstående sektioner}
Nu vil jeg jo rigtigt gerne udgive min min folksonomy-idé og mit blockchain-angreb og -forsvar. Jeg håber på at sidstnævnte i høj grad kan rette folks øjne imod mig og på mine andre idéer. Forhåbentligt har min rating-folksonomy-idé så også en smule gennemslagskraft, så folk også vil blive interesseret i denne. Jeg overvejer også at udgive min idé om kurser i første omgang, måske hvor jeg dog venter med det, der handler om debat, så jeg ligesom venter med dette og med debatsektionen. Men det må jeg lige se på. Jeg overvejer faktisk også, om jeg lige kan få tid til også at inkludere min QED-teori\ldots\ Man kunne jo lige gøre opmærksom på den, og forklare den overordnet, og så bare lige disclaime, at jeg ikke har haft tid til at gå beviset helt igennem endnu\ldots\ Tja\ldots\ Nå, det må jeg alt sammen lige se på. Ellers vil jeg altså gerne ret hurtigt udgive QED-teorien og mine debatrelaterede idéer; hvis ikke i første omgang så gerne kort efter. Jeg vil også gerne ud med mine eksistens-idéer og -tanker inden for en nær tidsramme. Civilforeningsidéen virker ``politisk'' nok til, at det nok er fornuftigt lige at vente med den til der er lidt ro på ift.\ de første idéer (så det hele ikke bare bliver forstyrret af dette). Og når først jeg har givet mig selv grønt lys til at udgive idéer, der kan virke politiske, så kan jeg jo også måske inkludere nogle af mine idéer fra ``Noter omkring muligheder for det fremtidige marked generelt''-sektionen ovenfor.

Som øvrige ting kan jeg jo også sagtens udgive mine idéer omkring ``kundedrevne virksomheder'' og om (min seneste version af) min ``donationskæde,'' som nu bare er et argument om, at hvis man kan forvente en vis ting (det jeg har beskrevet her ovenfor i slutningen af blockchain-sektionen) af samfundet i den nære fremtid, jamen så kan dette måske motivere til i højere grad at donere til folk, der har ydet et positivt bidrag (som fremtidens samfund vil drage nytte af og altså værdsætte). Dette er mine to idéer, som næsten stod distancen, men ikke helt; de er sikkert gode nok, men det er ikke sikkert, at der bliver brug for dem --- ikke hvis min idé til ``civilforeninger'' bliver så god, som jeg håber på\ldots\ Men ja, og så har jeg forresten også lige min idé omkring en app, hvor man kan lege med forskellige forslag\ldots hvad var status nu på den idé?\ldots\ (Altså min idé omkring ``dynamiske model-afstemninger?''\ldots) \ldots Okay, nu har jeg tilføjet en bemærkning omkring civilforeningerne, at de skal have noget tilsvarende, men jeg bør stadig lige overveje, om der var noget specifikt teknisk ved idéen, jeg lige skal huske at få med, eller om det rimeligt meget giver sig selv. Det kan jeg lige vende tilbage til.

Det er heller ikke sikkert, at jeg udgiver alle mine tanker omkring folksonomy-idéen, som jeg har skrevet ovenfor, og man kan i øvrigt sagtens finde på flere ting at sige. Så planen er altså, at jeg bare udgiver mine tanker, som jeg ikke inkluderer i første omgang, løbende, samt hvad jeg ellers kommer i tanke om da. 

Nå ja, og jeg har også min ITP-idé, hvilket vel også kan betegnes som en idé, der ikke helt bestod distancen som en idé, jeg tror på har gennemslagskraft, men som (helt klart i dette tilfælde) alligevel har potentiale. Problemet er bare, at hele dens succes afhænger meget af, hvor brugbart det i grunden bliver med ``formel matematisk programmering,'' som jeg forestiller mig paradigmet. Og i øvrigt er det endnu en af de ting, hvor det først vil vise sig, når der er kommet nok tilslutning til det og dermed nok arbejde omkring det, hvilket jo kræver, at folk kan se potentialet i det, så det er altså endnu en ting, hvor bolden ligesom først skal rulle, før det kommer ordentligt i gang (\emph{hvis} idéen altså holder i første omgang), og hvordan får man den lige til det? Men alt i alt tror jeg dog stadig på, at idéen har rigtigt meget potentiale, og når det kommer til stykket, har jeg faktisk svært ved at forestille mig, at det ikke \emph{på et tidspunkt} bliver et fuldstændigt udbredt paradigme; spørgsmålet er bare, hvor lang tid der går, før paradigmet vil være brugbart nok (for en stor del af brugbarheden afhænger bl.a.\ af, hvor store open source-fællesskaber man har, samt hvor gode automatiserede algoritmer man har til bl.a.\ at indsætte løsningsskabeloner og søge semantisk på programløsninger osv.). Lad mig lige gentage forresten, at det jeg ser som noget særligt ved lige netop min ITP-idé, er det, jeg har beskrevet om ``meta-antagelser'' (som altså formulerer en række mappe-invarianter), samt det i høj grad at bruge certifikater til bevisførelse (så man derved altså formaliserer den uformelle del af bevisførelser, der gør mange beviser overkommelige, som ellers ikke var det, hvis man skulle holde sig til ren formel logik). Og ja, så har jeg jo også lige beskrevet nogle idéer til at komme godt i gang med det hele, i.e.\ idéer til at implementere en brugbar ITP af denne form. 

Desuden synes jeg også bare, at de ting jeg har nævnt i de nedenstående sektioner, hvilket altså også bl.a.\ omfatter energi- og klimaløsninger, lykke, etik og det fremtidige samfund samt et par andre tekniske idéer. Hm, og sidstnævnte kunne i øvrigt også omfatte min lille idé fra ITP-noterne omkring en puslespilsapp, der er underholdende at bruge (ligesom andre (casual) ryk-rundt-puslespilsapps), og som samtidigt lærer brugeren fremgangsmåden at kende i matematiske metoder, således at brugen meget nemmere kan få disse fremgangsmåder ind under rygraden, når brugeren (måske efterfølgende) lærer teorien bag dem. (For når vedkommende så efterfølgende skal lave de tilhørende regneøvelser, så vil fremgangsmåderne allerede være halvt/delvist inde på rygraden. Og oplevelsen ved at lave disse regneøvelser vil så også føles meget nemmere og bedre for vedkommende (der sandsynligvis vil få et vist dopaminspark af pludselig at kunne bruge de indlærte metoder i en ny sammenhæng i form af at løse faktiske opgaver, som ikke bare kommer fra appen).) Men ja, anyway, mine nedenstående sektioner tror jeg også generelt er gode at læse (hvad jeg nemlig ikke helt kan sige om alle de ovenstående sektioner; der er også meget rod og mange uddaterede idéer). 

Ah, og bør også lige nævne kort, at jeg jo også har en faseoperator (fra mit fysik-bachelor-projekt og efterfølgende artikel) og et kvanteprogrammeringssprog (fra mit datalogi-bachelor-projekt), %Ekstra bindestreger indsat pga. rendering, som åbenbart ellers var svært for LaTeX(-compileren).
hvilke jeg vist ikke har skrevet om her rigtigt, og som jeg for begge ting bestemt også håber på, kan vække interesse og kan bruges. 

Cool, det var det! :) Nu skal jeg bare lige udfylde nogle hængepartier ovenfor (bl.a.\ omkring, hvad jeg skrev, jeg ville vende tilbage til her, nemlig om de ``dynamiske model-afstemninger'' *(\ldots ja nej, jeg har vist ikke så meget at tilføje der alligevel)) *(uh, og nedenfor skal jeg måske også lige %opsummere lidt 
samle op på noget %(Jeg skrev en ny paragraf, som på en måde opsummerede det lidt til sidst i korte træk, og nu synes jeg egentligt ikke, jeg behøver gøre mere (for nu).)
omkring mine eksistens-tanker), og så vil jeg eller (endeligt!) afslutte dette notesæt, og så vil jeg altså gå i gang med at skrive en tekst (på engelsk) til mine første udgivelser. 



%\subsubsection[nivi1]{}






%(11.10.21) Nå, jeg har tænkt lidt i forgårs og i går og er kommet frem til nogle mulige pointer omkring PoW, jeg nok gerne vil fremføre. Disse tanker indeholder også nogle nye idéer.. Jeg skal nu lige tænke over, om der nu også helt er en rød tråd fra disse tanker og så til PoPP; jeg føler næsten at jeg har den røde tråd, men jeg skal lige tænke lidt mere. Noget andet jeg skal tænke over, som jeg lige kom i tanke om nu her i dag, er, hvordan man nu også beslutter i PoPP (eller hvad det skal hedde), hvilke transaktioner kom først (og altså hvordan man undgår race attacks).. ..Tja, det kan jo gøres, hvis.. Hm, eller måske er det faktisk mere at tænke over her, end jeg lige regnede med..(?).. ...Ah, jeg kan sgu da også bare sige fuck det..! Jeg har jo nu et angreb, som jeg selv har en løsning på (som jeg ikke har forklaret helt endnu); det må da være fint! Og jo, man kan jo sikkert sagtens sætte sig ned og udtænke et system omkring kæder, som ikke har særligt meget brug for mining på noget højt tryk (det er jo i bund og grund om bare at oprette en form for åben database, og meget mere er der jo ikke i det), men hvad skal jeg bruge den idé til helt præcist? Jeg tror jo ikke rigtigt selv på den alligevel. Og jo, man kunne \emph{måske} godt finde en rød tråd imellem mit angreb, hvorfor det så måske kunne være ne god idé at konvertere, men dette argument vil så stadig bare være relativt; jeg tror stadig ikke rigtigt selv på grundløse KV'er. Så ja, jeg tror sgu, jeg vil nøjes med bare at gøre opmærksom på mit angreb, uden at det skal lede videre til noget større end det. For det er jo også en fin ting at dele i sig selv. Og det har mulighed for at gøre det, jeg egentligt håber på med al dette tænkeri omkring de mere gængse (ikke-donations-) KV'er, nemlig at det kan rette nogle øjne på mine andre idéer også. Spørgsmålet er så nu, om jeg vil forklare om min donationskæde i den endelige tekst, eller om jeg bare lige skal nævne den her. Hm, jeg kan jo skrive om den her, og så se på, om det er energien værd, at udarbejde det som en mere grundig tekst også.. (Uh, og angående det, så er det egentligt ikke sikkert, at min "renskrevne tekst" behøver at være så grundig heller, for jeg kan jo godt bare erklære, at formålet bare er at have en tekst, jeg kan støtte mig til, når jeg skal dele idéerne, hvad det jo også sådan set er.. Ja..) Ok. ... Hm, det er lige før, jeg faktisk ikke vil inkludere min donationskæde-idé i første omgang i min "endelige tekst" (hvad den jo så netop ikke bliver), men bare vil vente lidt med den.. Ja, det kan godt være, at det bliver såadn (lad mig bare antage det for nu).
%(12.10.21) I går kom jeg faktisk frem til, at jeg ikke behøver at gå så meget op i min donations-(lykke-)KV i det hele taget.. Der er helt klart noget ved idéen.. Hm, ja.. Ja, den er helt klart noget værd, og jeg vil også lige sørge for, at opsummere tankerne her, men jeg vil altså ikke bruge krudt på at udgive det i første omgang. Pointen er nemlig også, at når først man har civilforeninger, så kan bagud-belønning også bare implementeres via disse, og så vil der ikke rigtigt blive brug for noget andet (medmindre man altså ligefrem vil gå sammen om at uddele skyld til folk, som jeg har skrevet om nedenfor). Så jeg vil hellere udgive idéen om civilforeningerne først (men heller ikke nødvendigvis i "første omgang") og så nævne bagudbelønning i forlængelse af dette.








%%%Problemer med PoW (hvis ikke først):



%%%Donationskæde:


%(08.10.21) Uh, man kunne måske også...




%%%Forslag til mere specifik udformning af en donationskæde:





%
%Idéen handler altså sagt ud om vidensdeling og om at redigere tekster --- og ja, også andre ressourcer i princippet, men lad mig bare holde mig til tekster for nu --- i et fælleskab. 
%Inden jeg forklarer om den tekniske del af idéen (som egentligt er meget simpel), vil jeg gerne motivere idéen først ved at foreslå en ny form for Wikipedia-side (jeg kunne også have valgt andre vidensdelingssider, men Wikipedia er et godt udgangspunkt), som er mere alsidig ift., hvilke typer tekster den inkluderer. 
%Lad os således forestille os en wiki a la Wikipedia, men hvor hver artikeltitel ikke bare er en tekst, der beskriver, hvad artiklen bør handle om, men også indeholder prædikater ad libitum omkring, hvad der forventes af tekstens indhold og udformning. Et muligt prædikat kunne så eksempelvis bare være: ``bør leve op til standarderne for en almindelig Wikipedia-side,'' og på den måde kan denne wiki-siden sagtens indeholde alle Wikipedias artikeltekster som en undermængde. Men man skal altså også kunne oprette artikeltitler med alle mulige andre former for prædikater tilknyttet sig. Man behøver ingen gang at være bange for misinformation og ømtåleligt indhold i nogen højere grad end andre platforme, for man kan bare gøre det til en standard-indstilling, at brugere kun vises artikler med vise begrænsende prædikater på sig, og altså gøre så at brugere skal ændre avancerede indstillinger for overhovedet at få sådant materiale vist. 
%Eksempler på hvilke ``tekstprædikater'' man kunne gøre brug af på sådan en side, ...%(kommer så her...)
%
%Hvem skal så vurdere, hvilke prædikater skal sættes på hvilke artikeltekster? Det skal brugerne selvfølgelig ved at rate disse prædikater, helt i tråd med, hvordan det fungerede for folksonomy-idéen. Her er tanken bare, at artikeltitlerne i sig selv skal være faste, men at artikel\emph{teksterne} så tilgengæld ikke er knyttet fast til nogen specifik titel og i stedet bare knyttes til titlerne gennem relationer. Og det er så de relationer, som brugerne skal vurdere. Dette kunne så gøres ved, at brugere bare for hver tekst og hver titel i princippet afgiver vurderinger om, hvor godt titlen (med tilhørende prædikater) passer på teksten, men man kunne også gøre noget lidt smartere. %som jeg faktisk først lige er kommet i tanke om nu her, imens jeg har skrevet dette.
%Man kan nemlig også gøre mere som for folksonomy-siden, hvor alle tekster %...Nå nej, vent.. Bryder dette ikke så med hele min idé omkring de sammensatte tekster osv..? ..Nej, det gør det jo ikke.. Nej, fint.
%kan tilføjes prædikater og relationer, eller ``tags'' om man vil; det er også fair at bruge det udtryk, og at tænke dem som sådanne. Der skal dog stadig være faste titler, hvilket bliver forklaret, når vi når ned til de nye tekniske detaljer om idéen. Men så kan man nemlig bare implementere en algoritme på siden, der givet en artikeltitel (inkl.\ prædikater) finder frem til den eller de bedst mulige forslag til artikler, der opfylder disse prædikater (og handler om samme ting), og enten bare viser den bedste, eller viser nogen forslag, som brugeren så kan vælge imellem. Og her må det så selvfølgelig meget gerne på sigt blive sådan, at brugeren kan justere indstillingerne på, hvordan den algoritme, der bestemmer disse relevans-scorer, fungerer, og særligt skal man f.eks.\ gerne kunne udvælge `brugergrupper,' så algoritmen altså således kan bruge vægtede pointaggregater i sine udregninger (i stedet for bare gennemsnittet på tværs af alle brugere).
%
%Jeg kan forresten lige bemærke, at artikeltitler almindeligvist vil denotere et specifikt emne, begreb eller en specifik ting. Dette kan jo i princippet så også oversættes til et ``handler om''-prædikat (som muligvis kunne have flere varianter alt typen af, hvad det handler om). Dette kunne endda gå hen og blive ret smart, når vi når til et punkt, hvor alle emner og alle ting/begreber kan opsættes i en samlet kategori-model/-graf(/-ontologi?), for hvis der så ikke er noget passende hit på lige den titel (inkl.\ --- eller så måske bestående af --- tekstprædikater), man har søgt på, hvis man absolut holder sig til det præcise emne, eller den præcise ting, så kan algoritmen måske forsøge at se omkring emnet/tingen og måske søge på beslægtede begreber/ting/emner og/eller søge på mere generelle emner og/eller overkategorier. 
%
%
%Okay, og den tekniske del af idéen handler så om, faktisk at gøre mulighed for at tekster kan opdeles i lag --- og så også lægge op til at brugerne skal påtage en konvention om at gøre dette i høj grad. Denne lagdeling handler i princippet bare om at dele tekster op i dispositioner og tekst-moduler, som kan indsættes på dispositionspunkternes pladser. Hvert tekstmodul kan så i princippet selv være en lagdelt tekst bestående af disposition og tekstmoduler selv, eller det kan også bare være en atomar tekst. Idéen ligger så i at strukturere nævnte dispositioner lige netop ved brug af vores tekstprædikater. Og det smarte ved at gøre dette er, at så kan wiki-siden lige netop kan søge på de tekstmoduler, der passer bedst på pågældende tekstprædikater --- og i henhold til brugerens egne indstillinger for søgealgoritmen --- og kan så sætte dem ind automatisk (eller evt.\ prompte brugeren om at vælge, hvis nu der er flere muligheder med næsten lige gode relevans-scorer). Bemærk så også, at artikeltitlerne på denne måde så også kommer til at fungere helt ligesom sådanne dispositioner --- bare monadiske dispositioner så at sige, som altså kun består af ét modul. Og denne ligestilling er så mere end bare en sammenligning, for der er så ingen grund til, at større tekster ikke skal kunne indeholde, hvad der i andre sammenhænge kan ses som en selvstændig artikel, som et indre tekstmodul, og altså som en del af dets disposition. 
%
%*Så da jeg skrev ovenfor, at titlerne (med prædikater) ligesom skulle være faste objekter, så er det altså derfor: Når man navigerer til en artikel via dens titel på siden, så navigerer man altså først og fremmest til en (monadisk/unær) disposition, og selve teksten man så får vist afhænger af, hvad der passer bedst til den titel. 
%
%Jeg kan nu lige komme med nogle hurtige kommentarer om, hvordan man kan implementere disse lagdelte tekster, som jeg også har kaldt ``sammensatte tekster'' før i dette dokument, og hvilket man altså også kunne kalde dem. \ldots Vi kunne også kalde dem ``modulære tekster'' simpelthen. Ja, det lyder meget godt. (Og ellers kan man også kalde dem ``modulært opbyggede tekster.'') Og vil jeg ellers til gengæld ikke gå mere i dybden med tekniske overvejelser end dette. Modulære tekster kan så enten implementeres via en særlig syntaks, man beslutter sig for (hvor man så altså skriver en `disposition' som en tekst med denne særlige syntaks, og så kan opbygningen samt alle tekstprædikaterne for de forskellige moduler så parses fra denne), eller man kan implementere det mere abstrakt som en del af et HOL-system, hvor dispositionerne så kan tage form som tupler af prædikater essentielt set. Okay, og det var altså bare lige det, jeg ville påpege om dette.
%
%
%Det næste, vi så lige bliver nødt til at se på, er, hvad det vil sige at vurdere en disposition, for hvordan kan man f.eks.\ sige, at en disposition er ``letlæselig'' eller ``uddybende for emnet,'' hvis den i sig selv ikke indeholder noget brødtekst overhovedet? %Her skal man s... %skal man foreslå forskrifter (lidt ligesom mine "automatiske point") allerede her, så..?? ... ..Okay, alternativet ville vel være, bare at give pointene ikke-konditionelt.. Hm.. ..Og hvordan ville det virke, hvis man skulle lave forskrifter..? Så ville man jo skulle rate sig frem til dem.. ..Hm, i høj grad vil der jo være tale om ét prædikat, hvor man bare skal give en vægt for, hvor meget børnenes rating af prædikatet bidrager til forælderens rating.. Hm, men det er da netop også et rigtigt godt udgangspunkt..! Og det næste skridt er så bare, at lade flere forskellige prædiakter være input til et aggregeret prædikat for forælderen. ..Og dette må vel næsten være rigeligt så..? Hm, eller skulle man også lige kunne stemme om en kurve --- måske en logistisk-agtig kurve..? Ja, det kunne jeg også foreslå.. Selvfølgelig vil det i mange tilfælde være totalt overkill at bruge energi på at stemme om sådanne forskrifter, men hvis det er virkeligt store projekter, hvor der kan ske meget, så kan det give god mening.. ..Men ja, det med at stemme om vægte for hvert barn (og muligvis om flere børneprædikater) for et forælderprædikat, det kunne da give god mening.. Ah, og man kunne måske bare gøre det til en standard, at man så via disse vurderinger justerer en logisktisk kurve (ved at rykke midtpunktet). ..Hm, og måske skal dette så ikke så meget bruges som "vægte," men måske skal man i stedet mere tage et minimum..? ..Ja.. 
%%Jeg vil i denne forbindelse foreslå, at man faktisk for hvert givent prædikat, som... %Vent, der kan jo faktisk være to ting, der er interessante at stemme på. For det \emph{er} nemlig også interessant at vurdere, om en disposition er god eller ej givet at tekstmodulerne bliver udarbejdet ordentligt (hvor man altså ikke antager, at tekstmodulerne \emph{kan} udarbejdes som prædikaterne foreskriver --- f.eks. duer det ikke, hvis man bare siger teskt der argumenterer for at udsagn p er sandt, hvis man ikke nødvendigvis kan antage, at p er sand --- men hvor man altså vurderer, om tekstmodulerne med deres prædikater er realistiske, og hvis de er, så vurderer, hvor god dispositionen er, givet at tekstmodulerne så bliver udarbejdet og indsat ordentligt).. Hm, og man kan ikke blande disse to..? ..Hm, måske hvis man kan lave en rating af, om de forespurgte tekstmoduler i dispositionen er realitsike.. ..For hvis de så alle vurderes til at være det, så kan man jo bare se på topscoren.. nej.. Nej, så vil det være bedre, at vurdere en sidste vægt, der siger hvis.. hm, men så skal man stadig også have en realistisk-eller-ej-vurdering.. Men ja, man kunne altså have en ``er tekstmodulerne realistiske''-vurdering, en ``hvis de er realistiske og udarbejdet til fulde, hvor god er dispositionen så''-score, samt en.. ja, samt en forskrift, som kan justeres via flere små vurderinger (for hvert tekstmodul i dispositionen og evt. for flere end bare et prædikat (som som regel bare for ét prædikat pr. modul)), som så, vægtet i sidste ende med nr. 2 nævnte score her (og man behøver nemlig så ikke at inkludere ``er tekstmodulerne realistiske''-vurderingen er, for det kommer så af sig selv i algoritmen), fortæller hvor god en tekst dispositionen resulterer i, når man så indsætter et sæt af tekstmoduler (hvilket så typisk netop vil være det sæt, der maksimerer denne score). Ja..? ..Hm, men man for så ikke så meget brug for førstnævnte score andet end til at vurdere nye dispositioner.. Hm, skulle man så ikke blande de to (første) sammen, eller hvad kan man gøre..? Nej, det er fint at have det hver for sig, og så kan førstnævnte score bare være en form for advarsel, som man normalt ikke skal tænke så meget på, men hvis man ser at den er ratet højt for en disposition, så må man i hvert fald lige se ad, om der er hold i dette, og hvis der er, og hvis denne advarselsscore så bliver ved med at være høj (selvom der kommer flere og flere afgivne stemmer), så må dispositionen så ``kasseres'' af fællesskabet (fordi folk så vil have søge-algoritmer/-indstillinger, der vil nedprioritere den stærkt; og platformen kan så spørge forfatteren, om de må fjerne den fra databasen, men sådan noget gider vi ikke bekymre os om nu her; det hører til (implementerings)detajlerne). Ja, lad mig foreslå det sådan.. 
%Her skal vi holde tungen lige i munden, for en vurdering af en disposition kan jo betyde forskellige ting. Hvis en disposition bliver vurderet til at være ``letlæselig'' eller et andet prædikat, kan det f.eks.\ betyde, at hvis fællesskabet ellers bruger en rimelig tid på at udarbejde alle tekstmodulerne, der passer til alle de specifikke sektionsprædikater i dispositionen, så vil man forvente at den resulterende tekst opfylder prædikatet (f.eks.\ ``letlæselig''). Graden af vurderingen/ratingen vil så altså sige, hvor ``letlæselig,'' eller hvad det kunne være, den resulterende tekst ved indsættelse dispositionen har mulighed for at blive. Men når vi taler om ``en vurdering af en disposition'' kunne vi også i stedet tale om en vurdering af den bedste tekst, som dispositionen kan resultere i på \emph(nuværende) tidspunkt (og altså med de til den tid eksisterende tekster). For ikke at forvirre disse to forskellige ting, så tror jeg, jeg vil kalde det førstnævnte for dispositionens ``prædikat-vurdering'' og det andet for ``prædikat-scoren.'' 
%
%Angående ``prædikat-scoren'' så ville den mest simple løsning her bare være et tage minimumsvurderingen fra de indsatte tekstmoduler, gange dette med dispositionens prædikat-vurdering (hvis denne løber fra 0 til 1 --- eller kunne man også finde på andre funktioner for at sammensætte modul-scoren og dispositions-vurderingen til en endelig dispositions-score) og så lade dette være den resulterende dispositions-score. Jeg tror faktisk, at man i størstedelen af tilfælde vil kunne komme rigtigt langt med denne simple løsning, og at dette derfor vil fungere fint langt hen ad vejen *(ah, det er måske ikke helt rigtigt alligevel\ldots). Man kunne selvfølgelig dog også finde på mere avancerede løsninger, f.eks.\ %hvis nu visse sektioner er mindre vigtige for prædikatet end andre ift.\ teksten som helhed, eller 
%hvis nu man måske gerne vil aggregere flere forskellige modul-prædikater til et samlet prædikat.  %Sidstnævnte 
%Dette 
%kunne f.eks.\ være, hvis man har et prædikat så som: ``teksten lever op til den og den standard,'' som vi var inde på før (hvor vi snakkede om standarden for Wikipedia-artikler). Dette vil jo typisk involvere flere forskellige (mere atomare) prædikater omkring teksten, og her kunne det altså måske være smart så, hvis man automatisk kunne aggregere disse prædikat-score til en overordnet ``følger den og den standard''-score. %Lad mig derfor lige bruge lidt tid på at foreslå nogle mere avancerede løsninger, hvor man også kan tage højde for disse ting. Angående sidstnævnte forhold, kunne man jo eventuelt så bare have det, så man kan 
%En løsning her kunne være, hvis man kunne 
%indstille en global forskrift for, hvordan de indgående prædikaters score for omregnes til en score for det pågældende aggregat-prædikat, hvis vi kan kalde det det. 
%Man bør dog altid gange (eller hvad man gør aritmetisk) dispositionens vurdering på til sidst på den samme måde som i alle andre tilfælde, uanset hvad man gør. 
%%Og angående førstnævnte forhold, så kunne man måske have flere små ratings for hver tekstmodul i en disposition, hvor brugere (for hvert prædikat) kan vurdere, hvor vigtige modulerne er, når prædikatet for helheden skal gives en score. Hvis vi nu f.eks.\ stadig bruger minimummet, som foreslået ovenfor, så kunne man så f.eks.\ justere et offset for hver sektion, så man altså på nogen sektioner kan give lidt margin (så det ikke trækker scoren ned, hvis modul-scoren er inde for denne margin). Og ja, man kunne også finde på endnu mere avancerede ting, men jeg tror allerede at dette er rigeligt (og måske mere end rigeligt --- især med det sidste forslag her om ratings for hvert modul), for det vil nok alligevel kun være for meget få dispositioner, hvor det overhovedet giver mening at bruge så meget energi på scoren. \ldots Ja, det er endda lige før, at jeg bare skal droppe at nævne det med at vægte sektionerne\ldots\ Problemet er lidt bare at ``dispositioner'' nok tit kommer til at indeholde overskrifter også\ldots\ Eller, vent! Ah, pjat! For det første, så kan ``sektioner'' godt inkludere deres overskrifter (og deres omgivende struktur i det hele taget), men ikke nok med det, man kan også sagtens bare udforme prædikaterne sådan, at de tager højde for, hvilken type tekst der vurderes. For en tekst ved \emph{godt} selv, om den er en overskrift eller ej! Det kan man i hvert fald sagtens sørge for. Ja, never mind. Nu udkommenterer jeg dette, og så sletter/udkommenterer jeg også mit forslag omkring det.
%%Hm, men man kunne måske også få gavn af, hvis tekster kan vurderes ift. en kontekst..!.. Hvis nu man f.eks. kunne wrappe dem i en monade (måske i form af en slags "disposition"), som fortalte konteksten..? ..Hm, at kunne vurdere en tekst antaget en kontekst... ..Ja, antagelser! Man bør næsten kunne sætte antagelser på sektionsprædikaterne i dispositioner, og så man det netop bare være op til førnævnte advarselsvurdering, nemlig ``er tekstmodulerne realistiske''-vurderingen! Nice! ...Ah ja, og så kan man måske netop også bruge kontekst-antagelser i dispositionsvurderingen, som så, når man bruger teknikken med bare at tage minimum af alle sektioner, så alligevel kun tager de sektioner, der har pågældende kontekst-antagelse, som så nemlig også bl.a. kan præcisere sektionens funktion. 
%
%\ldots Okay, jeg kan se, at jeg alligevel skal omstrukturere ovenstående paragrafer en del, for det er meget rodet forklaret med den nuværende struktur. Men det gør ikke så meget, for så meget er det heller ikke, der skal forklares. Så jeg tror bare jeg vil færdiggøre disse brainstorm-noter, så teksten bare hænger nogenlunde sammen (og så alle pointerne bare er forklaret), og så kan jeg omstrukturere det, når jeg skal skrive en renere tekst over det.
%
%Lad mig derfor også bare lige færdiggøre det tekniske ting nu her, inden jeg gør tilbage til noget af det mere overordnet forklarende og motiverende. 
%
%Der er nemlig også et andet forhold, der kan gøre, at man måske også gerne vil have mere avancerede prædikat-score-algoritmer, og det kan være, hvis dispositionen ikke bare består kun af ligeværdige tekstsektioner, men består af andre ting også. Dette kunne være såsom faktabokse, figurer, opgaver, kildereferencer (måske for hvert kapitel) osv. Der er endda ingen, der siger, at dispositionsmodulerne ikke også skal kunne være selve titlerne \ldots\ %Hm.. ..Åh, jeg skal vist lige tænke lidt.. Jeg føler pludesligt, at der er et område her, hvor jeg ikke helt kan bunde: Det er lidt meget at skulle prøve at regne ud, hvordan dispositionskonventionerne bliver, og hvilke behov der bliver.. Ah pjat, mon ikke jeg godt kan tænke mig til et nogenlunde overblik. ... 
%%Okay, jeg er kommet frem til nogle ting, men jeg er muligvis ikke helt færdig med tænkeriet. Den store ting er, at man godt bare kan bruge en forskrift a la den, jeg skrev om ovenfor, til at skelne mellem vigtigheden af forskellige typer tekstmoduler, for disse moduler vil så have prædikater om sig, alt efter hvad de er... hm, eller vil de nu også det, for der er jo stadig problemet med, om en tekst selv kan vide, hvilken funktion den opfylder i en større tekst.. ..Hm, men kan man så ikke sætte struktur-prædikater i dispositionen.. eller kunne man ikke bare bruge prædikater, hvor man siger, denne teskt \emph{kan} vartage den og den funktion..?.. (Ej, jeg er af en eller anden grund også langsom i dag..) ..Hm, okay, der vel lidt to muligheder.. Man kan bruge generelle forskrifter, der virker på tværs af dispositioner, og/eller man kan bruge forskrifter, som er en del af en given disposition hver især.. ..Og ja, så kunne man måske netop bruge ``\emph{kan} varetage funktion''-prædikater.. ...Uh, men vent, hvis man nu kan have dispositions\emph{skabeloner}, så behøver man måske ikke de "generelle forskrifter"..! Hm..(!) ..Ja, og med kontekst-antecedenter også, så kunne det da blive ret godt.. ..Og tilsvarende kontekst-prædikater (som egentligt er relationer) for dispositionsmodulerne, også når det kommer til dispositionsskabeloner.. Hm, jeg har det som om, jeg alligevel har færden af noget her..!.. ..Ja, simpelthen. Det giver jo lige \emph{netop} mening at bruge tid på at lave forskrifter til dispositioner, hvis de kan bruges i mange forskellige sammenhænge. Fedt! Og så tænker jeg altså, at man rater de bedste forskrifter for automatisk at danne en score for et givent prædikat for sådan en disposition-skabelon. Og man kan så bare nøjes med, at dette skal gøres enkeltvis for hver skabelon; man behøver ikke at indføre noget generelt.. Tja, man kunne dog foreslå at lave et system med skabelonsklasser, hvor der så kan være inheritance.. :) Ja, fint. ..Det er faktisk virkeligt nice, det her.. (Dispositionsskabeloner bliver \emph{så} brugbare..!) ..Uh, og alt det med kontekst (hvor jeg nemlig har overvejet, hvad man f.eks. skal gøre, når man vil sikre sig, at en tekst ikke antager, at noget er introduceret, som ikke er det, og hvordan man sikrer, at udskiftelsen af en tidlig sektion ikke forhindrer forståelsen af følgende sektioner), så kan man bare klare alt dette med relationer, som dispositionsskabeloner kræver, skal være opfyldte! ..Hm, men er der en måde, hvorpå brugerne/forfatterne så kan reducere disse krav til mere specifikke ting (så man f.eks. kan præcisere, hvad en specifik sektion kræver, er introduceret før den selv)?.. ..Ja, hvis man kan oprette konditionelle tags/relationer. Så hvis man f.eks. har en sektion, hvor det skal vurderes og godkendes, at den ikke antager kendskab med noget, der ikke er introduceret endnu i den samlede tekst x men bør være det, så kan man vurdere at ``den ikke gør dette, så længe at emnerne/tingene/begreberne, y, z, ..., er introduceret forinden. Hermed skal man så kunne reducere, eller rettere forlænge, spørgsmålet til to vurderinger, nmelig om konditional-sætningen er sand og så om y, z, ... er introduceret forinden pågældende sektion i den samlede tekst. Dette kræver vel så, at dispositionsskabelonen kan indeholde relationer til hver sektion og så med den ovenstående tekst.. Ja, så man skal gerne have sådan nogle muligheder for på denne måde at strø relationer ud over sektionerne for en dispositionsskabelon. ..Hm, bør en relation så kunne vide, om den f.eks. er en konjunktion eller en disjunktion imellem de indre funktioner.. eller skal man bare have dette som en del af mulighederne i disp-skabelonen, så der i princippet kan blive O(n²) relationer i en tekststruktur, men hvor man måske med fordel kan lade være med at gemme disse i hukommelsen/lageret, men hvor algoritmen bare nøjes med at folde dem ud i forbindelse med automatiske tjek. Med andre ord kan der godt være O(n^2) relationer i en tekst i princippet, men for læsere vil de altid være forkortet til O(n) relationer, og de vil så aldrig skrives ud eksplicit, fordi de så ikke er beregnet til at blive vurderet af brugere, men kun er beregnet til at blive tjekket automatisk. ..Coo-ool! (Hvor er det fedt, når tilsyneladende "dovne dage" ender med at blive så givende! ^^) (..Og ja, jeg ved godt, at jeg sikkert har tænkt noget tilsvarende før, men derfor er det stadig super vitigt så at huske sådan nogle ting her igen i tide. :)^^) (..Føler dog umiddelbart, at dette er min skarpeste version af dispositionsskabelons(/tekststruktur)-idéen hidtil.. :)^^) ..Jeg skal så lige sørge for, at introducere idéen om at kunne tjekke prædikater/relationer som antecedenter til andre prædikater/relationer (for så automatisk at give konsekventen en positiv score). Men dette er vel rimeligt simpelt, for konsekventen kan så bare få en score a la antecedentens og konditionalens scorer ganget sammen (hvis det løber fra 0 til 1).. Ja, det må næsten bare være sådan --- det bliver i hvert fald det, jeg foreslår.. ..Ej, og nu \emph{kan} man faktisk argumentere rimeligt godt for, at idéen har et ok potentiale for at kunne bruges til programmering..!! ^^ ..For ja, det svarer vel til at indføre muligheden for at have kode-/programskabeloner med tjeklister, som er opdelt i mange små spørgsmål, og som så automatisk kan tjekkes for, om tingene er opfyldt for at skabelonen fungerer.. Og disse spørgsmål skal så ikke være maskinforståelige, men skal bare rates hver især (så mange af dem, som der er behov for; der kan jo bl.a. (måske) være disjunktioner, så man kun behøver ét godt svar..) af programmørerne (som i øvrigt kan have forskellige vægte; man kan jo bruge "brugergruppe"-vægte)..:) ..Og noget andet, som jeg ikke har nævnt endnu, men som jeg har i sinde at nævne i teksten, er, at programmering så også kan blive mere intuitiv, fordi designerne så bare kan bevæge sig på de mere abstrakte niveauer i koden (men hvor de stadig tvinges til at forklare tingene ret tydeligt (mere tydeligt, end de sikkert ofte gør), og hvor de så faktisk også letter programmørernes arbejde i de lavere niveauer, fordi disse så ikke skal designe de mere overordnede sammenhænge)..(!).. Hm.. (..Og dette svarer så til "intentional programming (paradigm)"-idéen..) ..Hm, hvor meget giver det mening, at forklare om for denne side af det..? ..Hm, ikke vildt meget, men jeg kunne jo nævne, at man i en god open sourve-verden, så bør ordne programmeringsløsninger ligesom for min folksonomy-idé (eller som for mange andre idéer, bl.a. det semantiske web), hvor alle skabeloner er semantisk kategoriserede, så de er lette at søge på, og så kan brugere vurdere, hvor godt løsningen løser forskellige ting. Og så kunne jeg måske nævne bevægelse af PC'er i et computerspil som et eksempel. Ja, lad mig bare komme lidt ind på de visioner --- det er kun oplagt, når nu jeg netop alligevel vil tale om web of applications-visionen overordnet set. Ja. 
%%(03.10.21) Det er fint, det jeg skrev her i går, men det med automatisk at reducere prædikater osv., det bør nu bare nævnes som en ekstra-ting, og altså høre til noget avanceret, man måske kan gøre på et tidspunkt.
%
%Ah, nu har jeg en stor forbedring til idéen (i forhold til hvordan jeg var ved at præsentere den)! Glem, hvad jeg skrev om at tage minimumsscoren fra modulerne og føre den videre, glem, hvad jeg skrev om generelle forskrifter for et specifikt prædikat. 
%%prædikat-scoren indtil videre (på nær hvad konceptet går ud på og sådan\ldots)! 
%Og lad mig så forklare om \emph{dispositionsskabeloner}! 
%
%I mange tilfælde kan de være gavnligt, hvis man kan have skabeloner for dispositioner. Dette kunne f.eks.\ være, hvis man skulle skrive en artikel eller en lærebog i fællesskabet. Her kunne man så have en dispositionsskabelon, der f.eks.\ siger, at ``der skal være en kildeliste til sidst i kapitlerne,'' hvis vi tænker på lærebogen, eller ``der skal være mindst én opgaveboks i hvert kapitel,'' eller alle sådan nogle ting. Og skabelonen kunne selvfølgelig også tage sig af det mere basiske som at ``der skal være en margin det og det'' (antaget at det er skrevet i et markup-sprog såsom HTML eller \LaTeX) eller ``alle kapitler skal have en titel'' osv.\ osv. Her kan man komme rigtigt langt med en syntaktisk definition af skabelonerne. En dispositionsskabelon kunne således helt grundlæggende bestå af en syntaktisk sprogdefinition (hvilket f.eks.\ kunne være en regex, men det ville dog også være gavnligt at indføre mere abstrakte udgaver, så selv folk, der ikke kan HTML, \LaTeX\ og/eller kender regular expressions, kan være med og kan designe og justere skabeloner til værker (artikler, bøger etc.)). Man burde så også gøre det sådan, at folk, der arbejder på et værk, som følger en vis skabelon (\ldots og nu indser jeg lidt, at det måske ikke er så klogt at kalde det ``dispositionsskabeloner,'' for det bliver egentligt ret forvirrende; navnet passer faktisk ret dårligt) --- ja, som vi måske bare burde kalde tekstskabeloner i stedet --- så kan indsætte flere kapitler og/eller sektioner, paragrafer, figurer osv.\ på en ret WYSIWYG måde\ldots\ %Ja jo, men så skal restriktioner såsom at der skal være mindst én af noget\ldots\ Nå nej, det går faktisk fint. Men ja, pointen er så, at sprogdefinitionen så meget gerne må... %Hm lad mig lige tænke lidt over det her.. ..Hm, man skal jo netop dele det op i lag, så man kun definere den yderste struktur først.. ..Ah ja, så never mind. Man kommer aldrig til at ændre på andet end.. Hm.. ..Ah vent, syntaksdefinitionen skal da ske via prædikater! Selvfølgelig..! Ja, og så skal syntaksprædikater altså indeholde en syntaksdefinition over gyldige input-dispositioner.. Så et tekstmodul kan have ne række normale prædikater og et syntaks-prædikat, og hvis den har sidstnævnte, så skal input være en disposition selv, der overholder syntaksen, må det ikke være sådan?.. ..Hm, og et spørgsmål bliver så, om dispositioner selv skal vide, om de følger en bestemt syntaks (og så kan de måske gemmes som automaton-input i virkeligheden.. måske), eller om syntaksen bare skal tjekkes? Hm, jeg hælder faktisk til det sidste.. Ja. (Og så kan altid automatisere og hjælpe brugeren i at holde sig til syntakserne efterfølgende (ved at tilføje smarte funktioner til ens redigeringsprogram..).) 
%
%Okay, vent lige to sekunder\ldots\ Never mind, det med WYSIWYG for nu. Og det er så gået op for mig, at vi bør snakke om ``skabelon-prædikater'' i stedet. Disse prædikater indeholder en syntaktisk definitioner over, hvad input skal overholde, som så i dette tilfælde skal være et dispositionsobjekt (hvis jeg holder mig til at kalde disse ``dispositioner'' (og lad mig bare gøre det for nu)) og altså ikke kan være atomare tekster. Og disse definitioner skal så kunne tjekkes automatisk, så søgealgoritmen kun søger efter og indsætter underdispositioner, der overholder syntaksen. Andre skabelon-prædikater (eller vi kunne også kalde dem syntaks-prædikater\ldots\ men skabelon-prædikater er der nu nok flere, der forstår) kan så potentielt være en del af, hvad der kræves for den indsatte disposition, og på den måde kan en skabelon altså komme til at definere en hel graf for, hvad værket skal overholde, men hvor det hele stadig er lagdelt og modulært, så at alle undersektionerne kan tjekkes for sig. Og ja, jeg har jo allerede nævnt nogle eksempler på, hvor det kunne være smart med sådanne faste skabeloner for værker. Man kan så starte med et skabelonprædikat som et helt ydre prædikat (og altså som en del af ``titel-prædikaterne'' til værket). Alle dispositioner og underdispositioner af værket kan så udformes med de frihedsgrader, der er, men skal så altså overholde en vis struktur, der kan tjekkes automatisk. Bemærk, at skabelons-prædikater kan fastsætte alle mulige prædikater for de indre dispositioner, ikke kun andre skabelonprædikater. Så skabelonerne kan altså også diktere alle mulige forskellige prædikater, som de indre moduler skal overholde, hvilket så selvfølgelig ikke tjekkes automatisk (når der ikke er tale om skabelonprædikater), men hvor brugerne rater og dermed godkender prædikaterne. Jeg vil endda forslå, at skabelonerne også for mulighed for at definere relationer \emph{imellem} modulerne, så man også kan kræve noget om sammenhængen mellem de forskellige moduler, men dette er en mulighed, der måske godt kan vente lidt med at blive implementeret. 
%
%Og lad os så spørge os selv: Hvordan skal det afgøres, hvornår prædikater er opfyldt, når de ikke kan tjekkes automatisk, men bare skal vurderes af brugerne? Jo, vi har jo været lidt inde på, at der skal være en ``prædikat-score,'' men nu bliver det meget nemmere, når vi har vores skabelonprædikater. %..Hm, hvordan skal det så egentligt fungere? Skal brugeren bestemme, hvad der gælde for en skabelon, eller skal skabelonen altid bestemme det..? Hm, burde man egentligt ikke bare adskille det og have en separat type af prædikater til at bestemme score-bestemmelsen..? ...Hov vent lige.. Med skabeloner, behøver man så overhovedet automatiske scorer..??.. ...Uh, kunne man ikke bare nøjes med score-forskrift-ratings for hvert enkelte værk, men så bruge score-skabeloner (som så kan hænge sammen med skabelonerne og tage udgangspunkt i deres syntakselementer..)..? ..Ja, så det ville altså sige, at skabelonerne ikke bare har én score-forskrift, men kan tilføjes flere forskellige, og så kan folk opvurdere en score-forskrift for enkelte værker, og ligesom for kategori-relevans-vurderingerne osv., så kan den underliggende algoritme så også evt. sørge for at forslå de mest populære score-forskrifter for den pågældende skabelon, når et nyt værks disposition oprettes.. ..Og her, og ved lignende tilfælde generelt (altså f.eks. kategori-relevans-vurderingerne), kan man måske implementere dette som et offset i relevansscoren..? ..Ja.. ..Jamen er det så ikke bare det, man skal gøre?(!).. ..Jo, lad mig gå ud fra det for nu..
%Prædikat-scorer skal så nemlig knyttes til %..Hm, men bliver det egentligt ikke svært at blande skabeloner sammen..? ...Hm, måske hjælper det, hvis skabelon-prædikater ikke behøver at være udefra-og-ind, men hvor "indre skabeloner" måske godt kan føje noget til den ydre struktur..?.. ..Nå nej, hvorfor behøver man at blande flere skabeloner sammen.. hm, det skulle vel netop være, så man kan foreslå de rette score-forskrifter..?.. ..Hm, men kunne man ikke godt dele skabelonerne op i lag også..? ..Hm, kan man have sammensatte syntakser, eller bliver det for kompliceret..? ..Ah, man kan måske have skabeloner, der har frihedsgrader i sig, og hvor den næste skabelon kan tage de frihedsgrader og definere yderligere syntaks for dette..??.. ..Ja, så at man på den måde kan lave arvelighed mellem skabeloner.. Ja, nu hvor jeg siger "arvelighed," så kan man jo sige, at selve idéen om arvelighed, hvor børnene så kan arve score-forskrifterne fra forælderne også, jo må være en ret god idé, og så er det bare et teknisk spørgsmål, hvodan dælen man lige implementerer sådan en arvelighed mellem skabeloner. Og min idé her med, at barneskabelonen kan tage elementer/produktioner for forælderen og så smække flere restriktioner på, kunne så være en god mulighed.. ..Ja, og det kan jeg jo så foreslå som en avanceret ting.. 
%disse specifikt. Nu bliver det vist lidt teknisk, så lad mig prøve at uddybe, hvad et skabelonprædikat er mere konkret. Det er en regex eller anden sprog-(syntaks-)automaton, der definerer en mængde af dispositionsobjekter (som altså i sig selv bare er en række prædikater, hvilket i øvrigt også kan inkludere andre skabelonprædikater). For at en tekst matcher en skabelon, skal teksten altså kunne produceres af pågældende regex/automaton, og hvis der så er nestede skabeloner, skal pågældende tekstmoduler så også hver især matche disse. ``Prædikat-score-forskrifter'' kan så tilknyttes en given skabelon, hvilket med udgangspunk i syntaksen definere, hvordan scorer fra de syntaktiske elementer (som altså tager form som prædikater til tekstmoduler) skal regnes sammen til en overordnet score for et givent prædikat. En sådan prædikat score kunne bl.a.\ sige, at for alle opgavebokse i teksten, skal gælde det og det omkring deres ``sværhedsgrad''-ratings (f.eks.\ at gennemsnittet skal være sådan og sådan, at der altid skal være nogle nemme spørgsmål i starten, eller hvad man nu kunne finde på). Denne forskrift skal altså så gå ind og måle på alle indeholdte opgavebokse i værket og aggregere dette til en samlet score for et prædikat, der jo passende netop kunne være omtalte ``sværhedsgrad''-rating (man må nemlig gerne genbruge tags/prædikater til forskellige typer tekster/ressourcer). Ja, bum. Jeg bør selvfølgelig også komme med andre eksempler, men dette var et ret godt eksempel. Andre eksempler kunne også være NSFW eller andre sprogstandarder, hvilket måske kunne gælde over det meste --- men måske man f.eks.\ kunne slacke på formaliteten i visse bokse. Det kunne også være ``forståelighed, hvis man har læst pensum $x$,'' hvor $x$ altså så kunne være en reference til et pensum. Og her kunne man så også nævne et eksempel, hvis man på et tidspunkt åbner op for relationer i skabelonerne, hvor man siger ``skal indgå i pensum $x$, eller skal være introduceret ovenfor i værket.'' Det skal så siges, at dette jo bare er et eksempel; måske kan man finde på endnu bedre eksempler, hvor man kunne ønske sig relationer.
%
%Og hvem bestemmer, hvilke prædikat-score-forskrifter skal bruges? Det gøres ved at stemme på en forskrift for en given disposition (hvilken så typisk vil have en skabelon, medmindre den bare har grundskabelonen, der bare hedder ``en række af tekstmodul-prædikater'') i brugerfællesskabet. Noget smart er så, at ligesom vi har set det før ovenfor, f.eks.\ da vi snakkede om kommentar-kategorier, kan platformen så holde øje med, hvilke forskrifter (for givne prædikater (alle forskrifter definerer nemlig en score beregning for ét prædikat)) generelt er populære for en given skabelon. Herved kan platformen så sørge for at foreslå denne forskrift for dispositioner med samme skabelon (og måske altså ved at give den et lille point-offset i ratingen). 
%
%En avanceret udgave af de her skabelonprædikater, som jeg gerne også bare lige vil foreslå, men som dog måske kan implementeres på et senere tidspunkt, er, hvis man åbner op for en mulig lagdeling for sådanne skabeloner også. Tanken er, at man så skal kunne oprette skabeloner, som viderebygger andre skabeloner, og den store pointe med dette er så, at disse så også kan arve forældrenes prædikat-score-forskrifter. For eksempel kan sådan noget som NSFW-tjek og andre standard-tjek jo betragtes som en ret grundlæggende ting. Det samme kan kravet om, at der skal være kilder, og at kilderne skal være passende for kapitlet og for det antagne tidligere pensum / læserens tidligere viden (\ldots ah, her for man jo også brug for relationer, så relationer skal nu alligevel forslås som en del af første pakke også (--- jeg skal bare ikke foreslå det med at reducere prædikater automatisk)). Så derfor kunne det være rart, hvis skabeloner kan opbygges i lag, så man ikke skal opvurdere alle sådan nogle standard ting for hver ny skabelon. Selvfølgelig vil et fællesskab jo kun bruge et begrænset antal skabeloner i deres værker, men der kan alligevel hurtigt blive en del forskellige, og ikke mindst vil skabelonerne stadig udvikle sig med tiden, så jeg mener altså, at man vil kunne spare energi, hvis man giver mulighed for, at man kan dele dem op i lag og benytte muligheden for `arvelighed.' Man kan sikkert implementere arvelighed på flere forskellige måder, men jeg føler, at jeg lige bliver nødt til at komme med et bud her. \ldots %..Hm, bør man så have "generelle skabeloner," som skal gøres åbne over for specifikation.. vi kunne også kalde dem "skabelonsklasser".. og så altså have.. skabelon-instanser, hvor man har defineret hele syntaksformlen for et værk..? Ja..
%Ah, for det første vil jeg gerne forslå, at man indføre ``skabelonklasser,'' som altså er en slags generelle skabeloner, hvorfra man så kan oprette specifikke (rigide) skabeloner, man kan bruge på værker. Vi kan så passende kalde disse konstante skabeloner for skabeloninstanser. Fint, men spørgsmålet er så lige, hvordan man kan gøre dette teknisk set, hvor jeg altså synes, jeg bør komme med et bud her. \ldots Tja, men må jo bare netop designe automatonerne/regex'erne i skabelonklasserne, så de er åbne over for videre specifikation. Dette må kunne gøres ved at %omstrukturere produktionerne, så man udfaktoriserer særlige produktioner, som så gives et navn...
%sørge for at strukturere alle produktionerne ... %monadiske...
%
%
%%Ja, det holder, det her med skabelonerne. For skabeloner til at opbygge tekster vil generelt bare være rigtigt brugbart, og når man så har dem, ja, så vil det også være oplagt, at bruge dem til at definere score-forskrfter med, så man kan få mere levende og dynamiske værker, som ikke skal omvurderes hver eneste gang et modul ændres. Og herved bliver det så også oplagt, at platformen kan holde styr på, hvilke forskrifter er populære for hvilke skabeloner, og ja, i det hele taget kan man så passende genbruge forskrifter herved. Og det med, at man også måske kan implementere opdelte skabeloner og arvelighed imellem dem, der har så bare mulighed for at gøre det endnu mere nice.
%%(04.10.21) Hm, nu er jeg dog lidt i tvivl om, hvor vigtigt det bliver med de automatiske scorer.. Hm, det kan vel måske fungere meget godt som en midlertidig rating for et værk, som stadig er i udvikling.. (eller som bare i det hele taget lige har fået ændringer).. Men ja, der må vel gerne i hvert fald også være en rating for specifikke kontante værker, som så kan vokse sig betydende, når værket har stået stille i lidt tid..?.. ..Hm, kom lige til at tænke på et eksempel, hvor en tekst eksempler tilpasser sig, hvad læseren har kendskab til.. Meget interessant om ikke andet.. ..Ah, vent, alt det her med scorer, var tanken egentligt ikke bare, at man så kunne bruge det til at vurdere dispositioner?..! For selve søge algoritmen, når man allerede har valgt en disposition, den kører vel bare top-down..? Hm, men den skal jo så vurdere, hvor gode diverese underdispositioner er.. Eller skal den?.. Kan man ikke bare lave hele dispositioner ad gangen.. og/eller kan folk ikke bare rate dem ellers..? ..Hm, jo måske er det slet ikke så nødvendigt med score og alt det halløj.. Lad mig lige se, hvis folk kan vurdere en dispositions potentiele, som jeg var inde på før, så er det jo rigtigt godt, og.. Hm, man kan man mon ikke bare gøre det så at man vurdere potentialet og samtidigt opstiller nogle tærskler for, hvor meget de individuelle prædikater i dispositionen skal være opfyldt?.. Eller kan man mon bare sige, at de alle tæller ens, og så.. eller man kunne fordele en vægt.. Hm.. Eller lade dem tælle ens og så bare have en konvention om, at tekstmoduler skal pakkes ind i monader, der kender tekstens kontekst.. ...Ja!.. Kontekst-kendende moduler, og bum! Så behøver man ikke nogen vægte..! 
%%...Okay, nu gik jeg lige en god lang tur og tænkte lidt over tingene, og jeg er faktisk lidt begyndt at se det på en ny måde.. Jeg er bange for, at jeg kommer til at skulle pille nogle af mine idéer ned og genoverveje hele idéen omkring, hvad man kan med de her lagdelte tekster (og hvad man genrelt skal kunne i et fællesskab, der har sat sig for at udbygge forklarende tekster..).. ..For dispositioner handler jo bare om at tegne en linje imellem emner og så sige, at hvis man læser om disse emner i den rækkefølge, så bør man opnå forståelse for det og det (og/eller blive kyndig på det og det område og/eller kunne løse de og de problemer).. Og et godt samarbejde i sådan et fællesskab handler vel så bare om at fremhæve de dispositioner, man mener kan være gode, og så i øvrigt fremhæve, hvis der er huller i de dispositioner.. ..Og ja, hvor god en disposition er kan så afhænge af prædikater, såsom hvor dybdegående teksten kan være, men så handler det jo så også bare om, at brugere skal kunne rate flere prædikater om en disposition en bare "godhed".. ..Så min pointe er altså: Er det ikke virkeligt simpelt.. behøver man alt muligt med skabeloner (jo skabeloner er gode, men hører de ikke nærmest mere til sin egen idé så alligevel) og automatiske scorer osv..? Behøver man overhovedet at have sådan et system, som jeg har tænkt på et helt grundlæggende plan (i.e. som er helt grundlæggende for min seneste version af denne idé), nemlig hvor man kan gå ind på en titel/disposition (hvor tilhørende prædikater også vælges), og så indsætter platformen tekstmodulerne automatisk..?!.. Jo, det lyder da virkeligt interessant, men.. vi er jo allerede på internettet; behøver tingene ligesom at blive samlet til ét dokument/"værk;" kan man ikke bare have en overordnet disposition med links til forklarende tekster for de enkelte delemner, og så kan man bare følge disse links i den rækkefølge (hvis man ikke kender til området)?..?!.. ..Så kunne man med andre ord ikke nå vildt langt bare med en Wikipedia, hvor der så er særlige typer af artikler, der beskriver en anbefalet vej ("linje mellem emner") til at blive kyndig på et emne, og så kan samme artikel jo også bare lige beskrive i hvilken grad man så kan forvente at blive kyndig ved at følge den rute. ..Og ja, hvis siden så bare netop er en side, hvor der er plads til alle mulige bud på "artikler," og hvor brugere så bare kan rate de forskellige artikler med forskellige prædikater, er det så ikke bare det..? Hm, på en måde er forskellen fra min seneste idé og så denne, at nu bliver "dispositioner" så bare tekster i sig selv, der udformes nærmest som en læseanbefaling, og som så bare har links til specifikke emner.. Hm.. Men ja, man bør så stadig i det mindste have det lagopdelt, så at disse links kan udformes som prædikater, så man ikke nødvendigvis navigerer direkte videre til en specifik artikel, men kan navigere hen til en søgning, hvor man så får de bedste forslag prædenteret.. ..Ja, dette giver da meget mere mening.. Igen: vi er allerede på internettet, så vi behøver slet ikke at fokusere på "sammenhængende værker;" adskilte tekster, der bare er bundet sammen af links, er fine!.. Og ja, især hvis disse links så bare lige kan være mindre direkte og i stedet kan lægge op til en søgning (og hvor brugere så også selv kan justere søgeprædikaterne og ikke mindst sætte flere på). ..Ja.. ..Okay, kan man så ikke nærmest bare implementere det med mit folksonomy-system, hvis der bare er tekst-ressourcer, og hvis man så også bare lige kan indsætte links i tekster, som behandles ved at man laver en søgning på prædikaterne, der udgår linket..? Og så kunne fællesskabet med fordel holde en ontologi over emner, hvor man kan.. Jo, det ville ikke være dumt af flere grunde, men i øvrigt kan man måske også bare sørge for at bruge "annotationer" i form af difs (som jeg har nævnt ovenfor) til at ændre på teksterne, og bl.a. hvis der er fremkommet en konvention om at referere til et emne med et andet navn.. Men ja, og man dog altså også med fordel bruge emnegrafer, hvor folk kan oprette links for, hvor semantisk ens de forskellige emne-tags.. nå ja, det system har man jo allerede i min folksonomy-idé, så never mind! Man skal så bare bruge emne-tags'ne fra folksonomy-idéen, når man skriver links i artiklerne (inkl. i dispositionerne). ..Bum.. ..Ja.. Så er det bare lige, hvad man gør med mine skabelonstanker.. ..Nå ja, og hvad man gør med programkode-tekster, for der skal det jo gerne netop blive til sammenhængende "værker"/dokumenter.. Hm.. 
%%Ah, jeg har det nu. Ja, det er rigtigt nok, at for ting større end hele kapitler, artikler eller sektioner generelt, der bare er store og selvstændige nok til beskrives med nogle ret overordnede prædikater, jamen så giver det nok oftere mening at have tekster, hvor sådanne kaptiler, artikler osv. bare indgår som links, man kan følge. Men når vi når ned til opbygningen dispositionen af selve kaitler, artikler og/eller sektioner små nok til, at deres tekstmoduler i stor grad kommer til at bestå af meget specifikke tekster, der passer lige ind i den sammenhæng, og som ikke nødvendigvis skal genbruges en hel masse.. uh, og særligt også som man ikke forventer, at folk vil søge på som noget selvstændigt.. ja: Når vi når ned til små nok tekstmoduler til, at de ikke er brugbare som selvstændige tekster (i.e.\ som tekster, der er værd at referere til selvstændigt med links i andre sammenhænge), jamen så kan det lige netop være smart med sådan en mulighed for at lave en disposition for sig, og så have det sådan at de indivudelle tekstudsnit kan udfoldes automatisk. ..Nej, vent.. Tja, på den anden side kan man jo også bare kopiere tekster (med passende referencer til det originale arbejde) og skrive dem om.. Og man kan så i øvrigt bare bruge en konvention om at liste prædikaterne for, hvad sektionen/artiklen skal indeholde, i den rækkefølge som man forventer, at det kommer i.. Hm, ja, så selvom min pointe skulle til at være, at man godt alligevel kan have fold-ud links, som hjemmesiden forventes automatisk at prøve at finde substituter for (og at "linket" således omdannes automatisk til en tekst, der passer til linket), så er jeg nu altså i tvivl igen.. Nå, men imens jeg tænker på det, kan jeg lige sige, at "skabelonerne" bare bør være, at man sørger for at udforme alle tekster i fornuftig markup, så de kan opsættes i forskellige layouts og med diverse tilføjelser til --- muligvis ved at man bare om-kompilere markup'et til et andet markup-sprog, hvis man nu f.eks. har det i HTML men hellere vil have det renderet med LaTeX eller noget andet. ..Hm, men er fold-ud-automatisk-dispositioner så en ikke så vildt brugbar idé, eller hvad..?(!).. Hm (måske).. (..Også fordi alt det med tesktudsnit-ratings og små rettelser, det kan man jo vel bare bruge "annotationer" til..) ..Hm, men kunne man måske foreslå fold-ud-dispositioner, måske især netop til programmering så..? ..Nå ja, jeg ville også have nævnt nu her, at angående "kontekst-kendende tekster," kontekst-prædikater og hvad, jeg ellers har snakket om i den smamenhæng, så giver det nemlig ikke rigtigt mening, når først vi snakker små nok tekstudsnit, at rate dem andet end ift. hvor godt de passer ind i den pågældende omsluttende \emph{selvstændige} tekst.. ..Men dette vil jo så afhænge af de andre tekstudsnit i denne samlede tekst, så her vil det vel så alligevel ikke give mening, at vurdere dem indbyrdes (eller så at lade dem kunne blive skiftet ud automatisk).. Nej ja, når de netop er afhængige af hinanden, så giver det ikke mening at behandle dem som uafhængige tekstudsnit. ..Hm, men man kunne stadig godt foreslå fold-ud-links, for der vil jo være artikler, der er relativt små, men hvor sektionerne alligevel er selvstændige; sådan er det jo f.eks. for gængse Wikipedia-artikler. Så ja, kan bør bestemt foreslå brugen af fold-ud-links, men det er så ikke sikkert at de bliver til meget andet end dette, i.e. til automatisk udpakning af relativt små tekster/artikler, hvor sektionerne alligevel er ret selvstændige.. ..Ja.. ..Tja, og dog: Jo, jeg kan også nævne programkode, hvor man af syntaktiske hensyn måske kan have gavn af, at links'ene kan blive foldet ud automaitsk direkte i den overordnede tekst.. 
%%(05.10.21) Okay, så nu tænker jeg jo lige over, om jeg så helt skal gå væk fra at skrive om mine tanker omkring programmering.. Som vel nærmest også på en måde næsten kan forkortes til: En mere visuel måde at gennemgå kode i et samarbejde ved brug af annotationer, som kan rates op og ned, og som kan få highlight-farvestyrke alt efter, hvor vigtige de er vurderet til at være. ..Og at man med dette så også kunne indføre en konvention om at bruge positive annotationer i høj grad; at sige "dette udsnit har jeg gennemgået, og jeg mener, at det bør virke," og så kan man herved få et overblik over, hvor gennemgået de forskellige udsnit er, og hvor sikre forfatteren og andre folk, der har gennemgået koden, er på, at koden holder. Det, og så er der også version control-tankerne, som vel svarer meget til intentionel programmering, gør det ikke?.. ..Hvor det særlige vel så bare lige er, at man her bruger ratings til at afgøre, om kodemodulerne passer til "intentionerne," som så er beskrevet i prædikaterne (som så kan være fold-ud-prædikater i bund og grund). Ah, men det er da også tre gode pointer! Dem bør jeg da helt klart lige nævne! Og de relaterer sig alle tre (rating-afhængige highlights, positive annotationer, og intentionel programmering med ratings) til ratings, så idéen kan så ses som en forlængelse af folksonomy-idéen. ..Hm, og kan "wiki-side-idéen" nu også det, eller skal den have sin selvstændige introduktion.. Nej, det passer måske faktisk meget godt også at introducere den i forlængelse af folksonomy-platformen. For pointen er nu bare, at man kan bruge tags til at klassificere forskellige typer artikler og andre værker, hvorved man så kan opnå langt højere alsidighed uden at det bliver uoverskueligt (fordi de forskellige typer altså er ordnet efter tags(/prædikater)). Coo-ool..! 
%
%
%*En paragraf, hvor jeg forklare om at arbejde i et fællesskab om at bygge sådanne tekster på denne måde...* %Her skal jeg nævne de version control-agtige aspekter af det.
%
%
%
%%Og angående førstnævnte **Uddyb** forhold, så vil jeg lige bemærke, at selvom jeg kalder det en ``disposition'' her --- og endnu værre at jeg bare har kaldt tekstmodulerne for ``sektioner'' flere steder --- så kan det jo godt være lidt mere end en række sektioner. ...
%
%%(Dette er ikke så relevant for monadiske dispositioner, fordi ...)
%%Nævn til sidst det med umuligheds-advarslen..
%%(Bemærk at dispositioner godt også kan indeholde omkringliggende ting / struktur og ...)
%
%
%Det næste vi kan diskutere, er så, ...%(annotationer...) %..Hm, hvilket vel så enten kan implementeres bare med de modulære tekster her, men hvor man måske også kunne bruge mere indre annotationer.. ..Ja, så det kan jeg vel netop påpege; at.. Hm, måske skulle man så have annotationer op i sit eget abstrakte lag, hvor de så automatisk enten bliver til indre annotationer eller til tekstprædikater/-tags alt efter, om om der så er tale om en atomar tekst.. eller en sammensat tekst..? ..eller et indre tekstudsnit.. Hm ja, dispositionerne skal jo også have prædikater om sig.. Ah ja, det mangler jeg at nævne ovenfor. Skal der så egentligt være en forskel? ..Det behøver der vel ikke, for så kan søge-algoritmen altid bare undersøge som en seperat ting, om nu teksten også er fyldt helt ud, eller om der er huller. Ja, fint!.. ..Hm, ah men semantisk set \emph{vil} det jo være forskellige prædikater, man bruger om henholdsvis udfyldte teskter (inkl. atomare tekster) og så dispositioner, men så kunne man måske netop også her gavne af et ekstra lag, når det kommer til tekstprædikaterne..! Hm...
%%Okay, jeg kom frem til i går aftes, at man netop bare skal kunne oprette tekstudsnit som (førsteklasses) ressourcer (hvilket så selvfølgelig skal ske automatisk, når man annoterer). Disse bliver så selvstændige tekster, der bare får en reference til forælderteksten. Og det er så bare disse referencer algoritmerne kan bruge, når den skal søge på relevante indre annotationer. Angående forskellen på dispositions- og tekstprædikater, så skal jeg dog lige se ad.. 
%












%(24.09.21) Lad mig lige hurtigt forklare lidt om, hvad jeg ellers har tænkt mig at fremføre, og særligt skrive lidt mere om min LKV (eller donations-kæde) og min nye KV idé, ikke mindst, for den har jeg ingen gang skrevet om i kommentarerne endnu! ..Lad mig se. Jeg vil jo skrive færdigt om min folksonomy-idé.. Når jeg nævner "brugerdrevet ML," så handler det bare om, at jeg tror, det ville være gavnligt, hvis brugerne selv fik adgang til (måske annonymiseret) data omkring, hvilke korrelationer, der er (i brugerholdninger osv.), og så simpelthen kan sætte deres hjerne på at finde forslag til, hvilke \emph{koncepter} disse korrelationer kunne skyldes. Ikke nok med, at man måske bedre kan finde frem til de egentlige underlæggende (virkelige) korrelationer (for de virkelige korrelationer behøver jo slet ikke være ortogonale, og behøver ingen gang at være lineært uafhængige), men man vil også bedre som bruger så kunne gætte sig til, hvor man selv lægger (hvis man har passende begreber for, hvad korrelationerne kommer af). Og ja, jeg tror også bare man vil lære en masse som fællesskab af at finde frem til korrelationer, da det jo siger meget om os og om vores psykologi. (Ikke at jeg behøver at inkludere alt dette i teksten; det er bare lige for at gøre rede for mine tanker. Jeg tror faktisk, at jeg vil gøre teksten om brugerdrevet ML-teknik (i.e.\ teknikken om at finde korrelationsvektorer) ret kort..). Okay, når vi så engang når ned til min "wiki-side-idé," så handler det altså om at forklare, for det første hvordan det kunne være smart, for tekster man samarbejder mange om (såsom en wiki!), hvis man bruger min "sammensatte tekster"-teknik, hvor alle afsnit (og gerne også mange under-udsnit; man kan opdele det ligeså fint, som man synes (alt efter hvad der lige giver mening)) får et prædikat, der beskriver, hvad det skal indeholde. Jeg har redegjort for idéen om at gøre dette i den ovenstående tekst, så jeg vil ikke sige så meget mere her. I teksten vil jeg så pointere, at denne idé også kan give plads til tekster med forskellige verdenssynspunkter og med forskellige grundantagelser, hvormed man så bare skal sørge for at gøre det klart via tekst-prædikatet, at danne tekst er udarbejdet / skal udarbejdes ud fra pågældende antagelser/grundsyn. Man kan også lave debatter, hvor hver påstand kan tilknyttes argumenter og modargumeter, lidt ligesom Kialo eller hvad, annotations-W3C-gruppen tænker, hvilket bestemt er gavnligt, men dog kan blive ret rodet i sig selv. Men jo, det er bestemt vigtigt med sådanne argumentgrafer, men de bliver først rigtigt gavnlige, hvis man altid sørger for at udforme sammenhængende tekster fra hvert relevante antagelses-(aksiom-)sæt/grundsynpunkt og/eller mere specifikke antagelser, der bedst muligt forklarer og redegører for alle relevante fakta for diskussionen (og selvfølgelig også dem som "modstanderne" har pointeret!), og diskuterer, hvor sandsynlige disse omstændigheder er ud fra de pågældende grundsynspunkter/grundantagelser og/eller specifikke antagelser omkring det diskuterede emne. Disse tekster behøver ikke at være udformet nødvendigvis rent af folk, der \emph{har} disse synspunkter/antagelser, men de \emph{skal} være skrevet nærmest som om de var, og de skal altså ligesom kunne godkendes af folk med det synspunkt. En sådan teksts mål er så, at redegøre for alle relevante argumenter og modargumenter og altså at give en overordnet sandsynlighed for, at fakta omkring det diskuterede emne er som det er --- og altså med andre ord sandsynligheden for, at fakta ville blive som det er eller "værre" (mindre sandsynligt), men man genafspillede tidslinjen på ny og ændrede tilfældige ting (så at tidslinjen ændres pr. kaosteori), så at pågældende fakta ligesom skulle genereres på ny, så at sige. Når andre så skal vurdere, hvor god sådan en redegørsel er, så kan de i første omgang kigge på, om alle relevante argumenter/modargumenter og alle relevante fakta er medtaget (og hvis ikke må man vurdere rapporten stærkt kritisabel), og derefter om logikken holder vand (og at konklussionerne altså ikke er helt ude i skoven --- og desuden at man ikke har sprunget let henover skridt i analysen, og ikke har prøvet at feje noget modargument til side uden at forholde sig ordentligt til det). Hvis alt dette er ok (hvad det gerne skal være; ellers må man skrive rapporten om igen), så kan man så i sidste ende se på konklussionen omkring sandsynlighederne, og hvis disse er helt vildt små kan man så konkludere at antagelserne er forkerte (men at rapporten stadig er god nok; rapporten er vigtigt også selvom man afviser hypotesen). Ok. Nå ja, og jeg vil forresten også nævne en idé, jeg har fået for nyligt, om at "sammensatte tekster" kunne udformes som rapportskabeloner (til det jeg lige snakkede om eller til alt muligt andet --- alt fra tekniske rapporter til humanistiske afhandlinger, og hvad har vi), hvor der i tekstprædikaterne til afsnittene også indeholder krav, hvilke nogle spørgsmål de skal besvare, og hvad der skal til for at besvare dem (f.eks.: Der skal skrives under på, at der er gjort det og det, og at resultatet af dette opfyldte de og de ting). Okay, er der andet jeg vil nævne..? Tjo, jeg kunne lige nævne, at web 3.1 refererer til, at jeg vil pointere, hvordan det næste skridt efter det semantiske web må være et.. prædiktivt web.. Det må med andre ord være de "prædiktive modeller," jeg har snakket om i det ovenstående, hvor man kan få oversat en mængde grundantagelser og et spørgsmål automatisk til en sandsynlighed for, at dette spørgsmål er sandt (altså at svaret er ja). Og her er "koncepter," der giver korrelationer super vigtige --- og er både en forhindring og en hjælp; korrelationer kan være vildt farlige, når det kommer til at nå den rigtige konklusion, men de kan også virkeligt være brugbare, når man finder frem til dem. Okay, jeg tror nogenlunde det er, hvad jeg lige vil nævne har omkring folksonomy- og wiki-idé.. Ja, jeg kunne nævne, hvad jeg har tænkt for brugergrupperne, men det skriver jeg alligevel lige om lidt i ovenstående tekst. (Og det er ret simpelt, og jeg tror, jeg har forklaret om det før i kommentarerne; det handler bare om at have administrator-/moderator-tokens af flere forkellige grader, som så kan deles ud (inklusiv måske negative token-værdier, hvis man vil opveje en underadministrators valg..)). Okay, jeg vil så lige nævne, at min idé måske også kan bruges til programmeringssamarbejder (nå ja, dette hører så også egentligt til wiki-afsnittet), hvis man nu er mange nok, der arbejder sammen, og at det så måske kan betale sig at gå mere op i den analytiske gennemgang af kode, især hvis man begynder at benytte et system, hvor hvert kode-/tekst-udsnit kan gives point alt efter, hvor korrekt det virker til at være. Hvis man så er mange om at arbejde sammen så hele koden er gennemgået og hvert udsnit ratet flere gange, så kan man måske lettere få øje på, hvor de mulige fejl kan være. Måske ville det så endda give mening, hvis man fokuserede mere på at skrive formelle argumenter for, hvorfor kode virker, hvorved andre brugere i fællesskabet kan rate'e udsnittende i disse argumeter og rate, om argumentet holder eller ej. Og når man så udskifter moduler, så vil man så let kunne finde frem til, præcis hvad modulet skal overholde, for dette vil så bare være at gennemgå alle argumenter, der referere til modulet, og se at de stadig holder (hvis man allerede har bygget modulet, og ellers kan man se, hvad dette \emph{skal} overholde). Ja, og hvis man virkelig kan opnå en konvention om så grundig analyse af koden i fællesskabet, så vil man jo også i langt højere grad opdage, præcis hvilke afhængigheder der er i koden. For man opdager nemlig bare rigtigt, rigtigt meget, når man sætter sig for at skrive en argumentation for, hvorfor noget holder, helt ud. Det kan jeg i den grad skrive under på. Så ja, min tanke er altså, at hvis man er mange nok om projektet, så kan en sådan grundig analyse måske betale sig, 1) fordi man så kan få en oversigt over, hvad folk synes om korrektheden af alle kodeudsnittene, 2) fordi det så kan blive langt nemmere at ændre på ting i koden og vide, at de andre moduler stadig gør, som de skal, og 3) fordi en commitment til det arbejde, der ligger i at opnå (1) og (2) også vil kaste den bonus af sig, at man nok også vil opdage mange flere ting om sin kode, i.e. når man tvinger sig selv og hinanden til at argumentere mere udførligt om korrektheden af kode-modulerne. Så ja, dette er mine tanker, men jeg tror nu måske, at jeg skal prøve at gøre det ret kort i teksten, for jeg kan jo egentligt slet ikke vide, om disse tanker holder; jeg har ikke erfaring nok med store programmeringssamarbejder. Men, for også at gå lidt videre, det ligger også meget godt op til den næste ting, som så kunne være at nævne et brugerdrevet "web of applications" eller "of programs," eller hvad end vi nu skal kalde det.. Hm, ja, og her er vi jo egentligt også stadig på "wiki-idéen" (og jeg har det lidt underligt over at kalde den det, for det virker lidt lamt ift., hvor vigtig og hvor smart idéen egnetligt er, men jeg har ikke lige fundet på, hvad jeg ellers skal kalde den.. --- Idéen er jo større end "wiki-siden;" "wiki-siden" var mere bare en vinkel, jeg fandt på et fremføre idéen.. Nå, men det er også totalt lige meget..) Men ja, det korte af det lange ved denne idé er bare, at bruge "wiki-idéen" med de sammensatte tekster (og altså med "tekstprædikater" osv.) til programmer også. Tekstprædikaterne for programmer vil så bare handle om programsemantikken også (samt udformningen), i stedet for at handle om NL-semantikken. Man kunne så have et open source-fællesskab, der udvikler programmer på denne omfattende wiki(-agtige side). Og til dette kan jeg så nævne, at man jo ville kunne inkludere bl.a.\ hemmesider i dette, og at man \emph{sagtnes} ville kunne bygge open source-applikationer, der kan matche nuværende hjemmesider (og også visse desktop apps).. Hm, og hvad er pointen her.. Tja, man skal så også lige få adgang til åbne databaser.. så det kræver jo donationer.. og var det ikke også lidt pointen bl.a..? At man bl.a. også kunne bruge min donationsskæde til at få dette godt i gang.. Hm, problemet er dog, at jeg måske ikke har overvejet, hvor stor efterspørgslen er for at erstatte de gængse platforme..(?).. Tja, det gider jeg ikke lige, dykke tilbage ind i nu, for om ikke andet kan jeg jo også bare sige, at hvis behovet er der, så vil sådan et open source-fællesskab jo kunne blive rigtigt interessant. Lad mig lige skifte linje en gang..
%Okay, det næste er så, at jeg vil foreslå en debat-SoMe-platform. Denne kan bygges lidt ud af wiki-siden og/eller af folksonomy-idé-platformene samt de brugergrupper, der vil opstå her, eller den kan bygges fra grunden af.. Nå ja, eller den kunne også bygges ovenpå (og af) eksisterende SoMe-platforme. Denne platform skal handle om, at brugere kan tilmelde sig forskellige (selvstyrende) brugergrupper, som repræsenterer befolkningsgrupper, grundholdninger/grundantagelser(/grundsynspunkter) (inklusiv grupper med politiske orienteringer osv.), interessegrupper (inkl. faggrupper osv.), og hvad man ellers kan finde på af grupper med visse alignments. Disse grupper kan have bredt optag, eller have mere kontrolleret optag. De kunne sågar begrænse optaget til "eksperter" eller "interlektuelle" og/eller andre mere elitære ting (og ikke noget galt i dette!), så at medlemmerne altså skal bevise, at de har visse kundskaber/kvalifikationer for at måtte deltage. Og hvad gør man så i disse grupper? Jo man diskuterer for det første, og man udformer også de argument-rapporter, som jeg snakkede om her lidt tidligere i denne tekst. Nr. 2 punkt her gider jeg ikke at snakke mere om nu. Det.. ah, vent! Der er også et vigtigt punkt imellem disse, og det er nemlig, at de vædder med hianden! Så ja, de diskuterer for det første frem og tilbage på platformen om hvad som helst. Der er ikke nogen særlige krav til disse diskussioner, men man kan selvfølgelig med fordel prøve at strukturere dem lidt, så det ikke bare er lineære jeg-siger-du-siger-diskussioner, men at de opbygges i en grafstruktur med argumenter og modsvar osv. Og dette bør altså, som jeg ser det, bare foregå rimeligt frit, uden at folk kan trække deres gruppes/gruppers omdømme ned, hvis de lige for skrevet en påstand lidt hurtigt, uden at tænke sig ordentligt om (og også uden at spørge resten af gruppen til råds; det behøves ikke her). Det bør også være sådan, at brugere kan markere, at de ikke lægger noget bag argumentet, men i princippet leger djævlens advokat (også selvom de dog ikke \emph{behøver} at være uenige med det de selv skriver, når de mrkerer dette), når de udformer et argument som "det kunne have lydt (fra en person fra en passende overbevisning eller med et passende bias til at sige sådan)." Sådan aktivitet bør endda værdsættes af det samlede fællesskab; det er kun godt at få så mange brugbare argumenter på bordet som muligt, inden man går i gang med at analyserer, og der er ingen grund til at de \emph{skal} komme fra folk, der også selv mener at påstandene er sande. Det er kun sundt med en pragmatisk kultur, hvor folk er gode til at diskutere tingene fra forskellige vinkler, og endda diskutere "imod sig selv," så at sige. Nå, men nu kommer den store pointe så. Det skal stadig forventes, at brugergrupperne løbende sørger for at erklære sig enige og uenige med påstande, og erklære sig eneige eller uenige med argumeter (i.e. i at argumeterne holder). Andre brugergrupper, der så erklærer sig af modsat holdning til et argument, skal så have mulighed for at udforde den anden brugergruppe, og sige, "vil I virkeligt gerne stå inde for denne påstand / dette argument, og er I klar til at vædde "omdømme-point" (eller hvad jeg har kaldt "street credits" i mine papirnoter, men hvad man måske kunne kalde "reputation credits/points" på engelsk..) på denne påstand?" Idéen er så, at alle grupper til hver en tid har uendeligt/vilkårligt mange "omdømme-point" at vædde med, men at der på hovedsiden af platformen holdes øje med for hver måned, hvor mange point/credits de forskellige grupper har vundet/tabt samlet set. (På denne måde vil "værdien" af disse credits også bare være relativ til, hvad der normalt vindes og tabes på en måned.. Ah, eller man kunne godt kontrollere værdien, ved at sige fremhæve det, når en gruppe når over eller under en vis mængde credits tabt eller vundet på en måned; på den måde vil "værdien" ikke flukturere arbitrært meget, men der vil ligesom være sat en vis forventning til den til at starte med.) Men ja, og det er sådan set bare det. Så er der lige det spørgsmål om, hvem der skal bedømme, når en påstand kan afvises eller påvises. Her skal det nævnes, vi vi her ofte snakker om påstande, der giver en specifik forudsigelse om fremtiden (eller om, hvad man senere finder ud af om nutiden, men det kan vi jo også bare kalde "fremtiden," da dette jo så handler om fremtidig viden). Så meningen er altså at brugergrupper særligt skal udfordre hinanden på ting, som kan formuleres som en forudsigelse for fremtiden. De to væddende grupper skal så bare vælge andre grupper som dommere, som afgør væddemålet, når på et tidspunkt at nok fakta har set dagens lys til, at man mener, man kan afvise eller påvise forudsigelsen. Hvis der på en eller anden måde sker noget ulødigt med denne dommerproces, jamen så er det bare om til brugerskaren i helhel om at udskælde den pågældende dom, og notere hver især, at måned n i år x var der et ulødigt resultat for et væddemål imellem gruppe a og gruppe b, så deres gevindst/tab for denne måned, skal altså tages med en asterisk. Cool. Og det er sådan set det, der er idéen. Den lyder ret simpel, men jeg tror virkeligt, at der vil kunne blive en \emph{kæmpe} energi omkring sådan en SoMe-side som denne. ..Ah vent, det bør også i øvrigt fremgå, \emph{hvilke} nogen andre brugergrupper, man har tabt/vundet penge til/fra, der der jo nemt kan blive meget forskellig prestige ift. dette. Desuden skal jeg også lige nævne, at en brugergruppe internt skal fungere ligesom de brugergrupper jeg vil (har i skrivende stund ikke gjort det endnu i teksten ovenfor) skrive om i teksten ovenfor, hvor der altså for det første kan være forskellige vægte til folk stemmer (altså hvor meget deres stemme vejer i beslutninger/vurderinger) og også forskellige moderator-/administrationsniveauer af gruppen (ift. hvem der skal optages og udsmides af gruppen, og hvordan omtalte stemmevægte skal fordeles). 
%Okay, og nu kommer vi så til min donations-kæde-idé. Jeg kan ikke huske, hvad jeg har beskrevet hidtil (måske har jeg beskrevet den..), men idéen er altså på en måde en lidt mere nede på jorden udgave af min "lykke-kryptovaluta." I stedet for at se på en idé, hvor folk skal gå med til at bruge et valuta-system, hvor gode handlinger bliver belønnet, men hvor alle så skal stemme om, hvad der er gode handlinger, og hvor gode de er, hvad så med bare at se på en idé, hvor folk bare selv i sidste ende styrer, hvad de mener, er de gode handlinger, og altså hvad de vil belønne med deres del af pengepengebidraget (i form af skatter i valuta-systemet, eller hvad det kan være)? Dette er som nævnt en ting, jeg har tænkt over før, men nu føler jeg altså, at det godt kan holde (med den version af idéen, jeg har nu). "Kæden" i denne idé er egentligt ikke nødvendigvis struktureret som en blockchain. Idéen kræver bare en åben, decentral database, hvor folk kan oploade løfter til. Vi snakker altså digitalt underskrevne blokke, hvori der står et offecielt løfte fra underskrevne person, formuleret i et (standard, gerne enegelsk, men det kan også sagtens være på andre sprog --- især hvis der er tale om meget lokale løfter) naturligt sprog. Disse løfter kan altså så særligt være løfter om at donere penge i fremtiden, til gode gerninger og/eller gerninger, der kommer vedkommende selv til gode. Grunden til, at "kæden" ikke behøver at være en egentlig kæde, men i princippet bare kan være en mængde af blokke uden links til hinanden, er, at så længe der vil være interesse i fællesskabet om at gemme alle de løfter, som folk har lavet og underskrevet, jamen så vil der også være nogen, der gør dette. Det vil dog ikke være helt dumt at linke blokkene på en eller anden led --- måske bare i en træ-graf i stedet for en enkelt kæde --- så man bedre kan holde styr på, om man mangler nogen blokke (ved at følge alle linkene baglæns og se om alle blokke er der). Så på en måde er den eneste ting, denne "kæde" har til fælles med gængse blockchains, altså bare, at den jo selvfølgelig helst skal foregå digitalt (for nemheden og overskuelighedens skyld) og at databasen gerne skal være åben og decentral. Apropos så kan man selvfølgelig også godt udarbejde en protokol for at udforme blokke i form af papir-dokumenter i stedet for digitalt, som så til hver en tid kan fotokopieres og uploades digitalt til "kæden" (som jeg hellere burde kalde noget a la "løfte-databasen"..), men det er nu nok ikke det værd --- også fordi det er ret smart, at blokkene kan bruge krypteringssignaturer, hvilket jo fungerer bedst digitalt. Okay, og hvad skal folk så love i dette system, og hvad skal få dem til at overholde, hvad de lover? Tja, løfterne kan være alt lige fra meget specifikke løfter om at ville betale/donere for en specifik gerning.. Nå ja, og jeg mangler også at besvare: Hvordan identificerer folk sig, når de underskriver en blok, så de kan holdes til "regnskab" for deres løfter? Det kan jeg lige svare på først. Folk skal uploade særlige identifikations-blokke, hvor de uploader fotokopier af fingeraftryk, hænder, øjne, ansigt, underskrift (hvis ikke dette så kan misbruges..), blilleder fra deres nærområde gamle familiebilleder genstande de ejer, gruppebilleder, billeder med kollegaer, historier fra deres liv som deres nære og/eller deres venner kan bekræfte, dokumenter, pas, kørekort, andet id... Det kan være alt muligt i princippet. Det skal dog ikke forstås som om.. Tja, altså selvom man jo i princippet skal være fri til at uploade alt, hvad man vil, så er dette altså mere bare forslag til, hvad de forskellige fællesskaber omkring dette system (i.e. omkring denne "kæde") kan vælge at bruge. Det er altså stadig smart, at folk, der skal bruge kæden, beslutter sig rundtom i diverse brugergrupper for nogle konventioner for, hvad man forventes at uploade, hvad man kan vælge at uploade, og hvad er en dårlig idé at uploade (f.eks. hvis underskrifter kan misbruges eller noget). Bemærk også, at nogle af de ting, jeg nævnte, ville kunne uploades af andre end vedkommende selv (f.eks. familie-billeder). Men dette kan stadig bruges som bevis for, at vedkommende er den, vedkommende giver sig ud for at være, fordi det ikke er så sandsynligt, at nære venner og/eller familie vil svindle personen, og hvis de gør, så er det jo ligesom relativt let for vedkommende at finde frem til den/de skyldige. Okay, så langt, så godt. Jeg nævnte så, angående løfterne, at der kunne være tale om meget specifikke løfter, men der kan altså også være tale om mere brede, generelle løfter. Dette kunne være løfter mere med en ordlyd a la, "de handlinger, der fører til en indkomst for mig, eller som fører til, at jeg sparer et beløb, vil jeg belønne med x procentdel af, hvad de indbragte eller sparede mig." Det kunne også være løfter, der lovede at betale for services/muligheder/applikationer/gaver/råd/hjælp/lærdom (for alt sådant eller for specifikke typer af sådanne ting) --- med andre ord ting, der gavner vedkommende andet end på pengepungen direkte --- til en vis grad --- en vis (anslået) procentdel, måske --- af, hvad man ville have betalt for det, hvis ikke man havde fået det "gratis." Dette er så selvfølgelig lidt sværere, for her kan det jo let blive let for folk bare at sige, "jeg ville ikke have betalt særligt meget her," hvis ikke man udformer løfterne lidt mere omhyggeligt. Noget man så kan gøre, er at oprette brugergrupper (og disse kan godt forresten minde administrativt --- og stemmevægt-mæssigt --- om de "brugergrupper," jeg har snakket om i det ovenstående), hvor det gøres til en fælles beslutning, hvor meget ting er værd. Ved så at donere en lille smule til administratorerne også, kan man hermed opnå muligheden for at lave mere troværdige løfter, der betaler for "ting" (i den lidt brede forstand af ordet, hvor services alt det jeg nævnte også kan kaldes "ting"). For en gruppeadministration bliver så nødt til at gøre sig umage med at bedømme værdien af tingene retfærdigt (også ift. hvor meget brug de specifikke medlemmer i gruppen har af dem), for ellers vil gruppen miste tillid blandt "bidragsyderne," som jeg har kaldt det hidtil, altså de mennseker der producerer, kreerer eller yder tingene/services'ne. Og herefter vil der så ikke være nogen grund til for medlemmerne at bruge brugergruppen længere og heller ikke nogen grund til så at donere til administrationen. Især fordi folk, der så bliver i en ulødig brugergruppe og bliver ved med at bruge den, kommer til at virke utroværdige selv.. Hm, og hvordan holder folk øje med, hvor troværdige og utroværdige folk på kæden er? Tja, der er der jo ikke andet for end at oprette særlige brugergrupper (som selvfølgelig måske kan varetage andre funktioner også, såsom at vurdere "bidrag") til at uddele tillidspoint til brugergruppen og på den måde holde styr på, for det første hvor veldokumenteret en brugers identitet er, og for det andet hvor lødig dennes aktivitet har været, og dermed hvor tillidsværdig vedkommende kan anslås at være. Nå ja, i øvrigt skal man også helst dokumentere donationerne/betalingerne over selve kæden, så fællesskabet kan følge med i, om løfterne bliver overholdt. I øvrigt kunne dette i en lidt alternativ version af denne "kæde" giveanledning til, at man faktisk oprettede en valuta også på "kæden." For ved ligefrem at kræve, at betalingerne/donationerne skal betales i form af kyptovaluta, så ville man få et system, der har et eksternt formål og som mange mennesker kan afhænge af, og hvor dette system så afhænger at pengemængden er af en vis størrelse --- og umiddelbart tror jeg, at en for lav pengemængde ift. populariteten af og penge-aktiviteten på "kæden" vil føre til en større efterspørgsel, som vil hæve kryptovalutaens kurs til en mere passende værdi, men jeg vil ikke hænges op på det; jeg har ikke tænkt så grundigt over det. Jeg hælder nemlig ikke selv til gå efter denne version af kæden, bl.a. fordi jeg ikke rigtigt tror, det i sidste ende vil være gavligt for brugerne, at binde sig til at betale hinanden med en særlig kryptovaluta, når man også bare kunne lade være, og bare love at betale hinanden i et eller andet mål af penge. Så glem dette nævnte alternativ, og lad os gå tilbage til en "kæde" uden nogen særlig kryptovaluta tilknyttet sig. Folk kan altså love betalinger i alle mulige former for penge (og kan f.eks. også love værdier målt i naturalier, hvis man synes at valutakurserne er for risikable (eller en samling naturalier og/eller andre værdier, så man sikre sig mere bredt, hvis man vil undgå varierende kurser)). Så længe de bare sørger for at registrere deres betalinger, så resten af fællesskabet kan følge med i, at alt er, som det bør være, så er det fint. Nå, men vi har stadig mere at snakke om angående løfterne, for løfterne kan også (selvfølgelig; de kan være på alle mulige måder --- hvis altså bare de er lovlige, selvfølgelig) inkludere forhold omkring vedkommendes egen fremtidige økonomi. Man kunne således i princippet love at give en vis procentdel af, hvad man har.. Hov, jeg mangler også at nævne, at selvom flere brugere bruger den samme brugergruppe til vurderinger af, hvor meget "ting" er værd, så kan den endelige pris jo stadig godt variere fra medlem til medlem i brugergruppen, da det jo kan variere, hvor meget de har brugt og/eller haft brug af "tingene," det er klart. Nå, men så er det jo nævnt nu. Angående at udlove at betale en vis del --- eller muligvis sætte et vist maksimum på betalinger/donation --- af, hvad man ejer, så kan dette jo være meget fint langt hen ad vejen, men... Hm.. Jeg tænkte, at hvis man snakkede pengebeholdningen, så ville dette så skabe en usikkerhed, fordi man så bare kan bruge sine penge på at købe noget, lige inden man betaler det, man har lovet fællesskabet.. Men hvis man bare bruger sin samlede værdi, inkl. de egnedomme, man har, så fungerer det jo allerede ret fint. For så er det bare spørgsmålet om, at folk selvfølgelig kan købe en hel masse services eller produkter, der ikke holder sin værdi, men måske forbruges. Men dette er alligevel en lidt langt ude ting at gøre. Men ja, tanken var nu, at man så bare kunne love ikke at gøre dette (på "kæden"), og ja, det er da også stadig værd at nævne, at sådan nogle ting kan man også tage højde for, når man underskriver løfter, og løfter kan altså også inkludere andet end bare selve betalingerne, men også f.eks., at man ikke sløser penge væk bevidst i mellemtiden, så man slipper for at betale. Men ja, sådan er der jo så meget; det er ligesom med almindelige kontrakter. Den eneste forskel på dette system og så gængse juridiske systemer er bare, at det eneste, der er på spil (i hvert fald umiddelbart) er ens ry og omdømme, og også måske at der ikke skal så mange instanser ind over, når man underskriver et løfte, men at processen om at vurdere tilliden til et løfte bare kan ske lidt løbende i fællesskabet. Og det bringer mig så til det sidste punkt: Hvad får folk til at holde deres løfter? Det gør det, at folk ellers kan miste ry/omdømme og miste tillid i fællesskabet samt til den omliggende verden fremover (for løfterne er jo offentlige). Her er det vigtigt for systemet, at løfterne generelt anses for ligeså vægtige, som løfter der gives mund til mund (eller på skrift, og hvad har vi): Løfter er løfter. Og folk har nemlig heldigvis en motivation til ikke at blive kendt, som nogen der bryder løfter. Og især i øvrigt ikke, hvis løfterne handler om at donere til gode bidrag til fællesskabet, for det vil alle jo se ilde til, hvis man gør. Så det.. Og hvad gavn kan der så komme af sådan et decentralt og frit løfte-system (altså et decentralt og ret simpelt juridisk system uden andet på spil en folk omdømmer, hvis de ikke holder deres løfter)? Jeg tror, der kan komme meget godt ud af det, for jeg tror der er mange områder, hvor enkelte mennesker har mulighed for at gøre en forskel for rigtigt mange mennesker, og altså til en \emph{stor} samlet værdi, men hvor de stadig ikke gør det, fordi de ikke har nogen måde så at tjene penge på disse gerninger. Jeg forestiller mig så, at et sådant system med tiden --- så snart folk bare for formuleret generelle nok løfter om tilbagebetaling, at de er dækkende selv for uforudsete gerninger, og så snart der kommer nok brugere på kæden, der løbende viser troværdighed --- vil kunne gøre helt op med sådanne forhold, så at folk, npr de vil beslutte, om de vil udføre en handling eller ej, som de ved vil gavne en masse mennesker, ikke behøver at bruge tid på at tænke over, hvordan de så får penge for denne handling, men bare kan gå til den velvidende, at resten af samfundet nok skal erkende og belønne bidraget på en værdig måde efterfølgende. Det tror jeg på. ..Især fordi dette denne realitet i sidste ende vil komme til at gavne \emph{alle} (og vil løbende gavne flere og flere medlemmer). På samme måde som at \emph{alle} (plus-minus nogle få måske, hvad ved jeg?) har gavn af vores nuværende pengesystem, selv folk, der ikke rigtigt har nogen penge i systemet, så vil et sådant godt system, hvor samfundet (samlet set altså; der er stadig tale om et decentralt system. Men folk vil dog sikkert i høj grad melde sig til faste brugergrupper i sidste ende, så at der dermed alligevel ender med at bliver nogle centre i systemet (ikke at dette betyder noget for denne pointe..)) altid sørger for at belønne gode/gavnlige handlinger også være til gavn for \emph{alle} i sidste ende. For selv om visse mennesker så vil donere mere, end de får via "kæden," så vil de jo få så meget andet gavn af "kæden," også inklusiv bare den lettelse at være i et system, hvor man ikke skal tænke så meget på, hvordan man monetiserer handlinger. Og her til sidst kan jeg lige, da denne tekst jo handler om, hvad jeg har i sinde at skrive om her, nævne, at jeg så også tænker at pointere i teksten, at et sådant løfte-system særligt kunne være gavnligt for udbredelsen af nogle af de ting, jeg har snakket om, såsom min wiki-side, min folksonomy-platform (især hvis vi snakker den mere åbne (som i open source), tværgående og brugerstyrede version af denne platform) og det jeg indtil videre kalder "web of applications/programs." Ja, i det hele taget når vi snakker open source og/eller "copyleft"-bidrag på internettet, så kan det altså være ting, hvor folk i fremtiden med et sådant donationsløfte-system i langt højere grad kan forvente at blive belønnet for deres gode bidrag. 
%Og nu når vi så endelig til min nye kryptovaluta-idé (hvor der altså rent faktisk er en KV knyttet til "kæden"). Faktisk bruger denne idé mange af de samme principper fra foregående "kæde." Eksempelvis er der heller ikke tale om en egentligt kæde, men igen bare en samling af blokke, også her som indeholder løfter, og som man derfor kan opbevare på mange forskellige måder. Og fordi der også er tale om løfter imellem mennesker, så skal der ligeledes nok altid være nogen, der vil være interesserede i at opbavere blokkende (og især når først penge bliver godt involveret (hvad end vi snakker donationsbeløb, eller de KV-penge, jeg nu vil beskrive)). Ja, og nu hvor jeg skriver om det, så kan jeg jo godt se, at de to "kæder" ikke bare minder meget om hinanden, men de kan faktisk implementeres som en og samme kæde. Så jeg vil derfor altså foreslå to måder at bruge en sådan løfte-"kæde" (eller et løfte-system, burde jeg jo hellere kalde det for nu) på. Den ene har jeg lige beskrevet, og handler altså om, at give løfter om betalinger/donationer for gode/gavnlige handlinger til andre mennesker. Den anden brug handler om at få en kryptovaluta op at køre på "kæden." Denne idé svarer i bund og grund til at oprette en bankvirksomhed over "kæden," hvor alle "bankens" "kontrakter" altså underskrives i form af sådanne (ikke egnetligt juridisk bindende) "kæde"-løfter (som jo alligevel vil være moralsk bindende i en eller anden forstand for de involverede). Og der er altså så tale om en komplet decentraliseret bank. Og inden vi kører for meget ud af bank-mateforen, så skal det siges, at dette jo også svarer til alle mulige andre kryptovalutaer: Her er der også tale om et fællesskab, der i princippet (på ret decentral vis, når det kommer til blockchains) på en eller anden måde går sammen om at at skabe.. ja, i sådanne tilfælde hype om valutaen, hvilket så (indirekte, om ikke andet) medfører en vis \emph{tillid} til, at valutaen vil bevare sin kurs. Nå ja, og man kan jo nok ikke nødvendigvis kalde dette "hype," men jo mere der handles med valutaen, og jo flere gange kursen svinger op igen efter en (større eller mindre) nedtur, vil der også komme mere tillid til, at valutaens kurs vil være nogenlunde stabil. "Investorer" i valutaen vil så altid påtage sig en risiko, men kan også få et udbytte af at have påtaget sig denne i sidste ende. På gængse kæder sker dette ikke i form af renter eller afkast, tvært imod, for valuta-besidderne skal jo i princippet løbende "betale" til minerne (i hvert fald hvis man regner med at pengedannelsen må formindske pengenes værdi tilsvarende, hvis man trækker andre forhold fra --- hvilket selvfølgligt er lidt svært for de gængse kryptovalutaer at slå fast, for på en måde er mange regler lidt ude af spil her). I stedet satser "investorer" i gængse kryptovalutaer bare på, at hypen stiger og ikke falder. Men alt i alt er derfor allerede ikke helt galt, at se disse gængse blockchains som en slags banke.. Tja, det vil jeg faktisk ikke diskutere her. Men det kan man altså nok lidt. Min KV-idé gør dog analogien endnu mere passende. Denne fungerer ved, at man først starter en PoW-kæde, hvor alle der har interesse i projektet kan melde sig til og begynde at mine pengene. Denne PoW-start-kæde må så gerne løbe i en vis tid, så alle potentielle "investorer" i denne fase har mulighed for at melde sig på banen --- og gerne så de ikke bliver straffet særligt meget, hvis de ikke lige akkurat var med helt fra starten. Efter en dealine ser man så i fællesskabet, hvad den længste kæde er, og når til en konsensus om, hvad denne var. Nå ja, her er det sikkert også en god idé at udfase miningen (ved at lade mining-lønnen gå mod nul) og at sætte en dæmper på, hvad folk kan overføre, som også går imod nul op til deadlinen (hvis man altså overhovedet giver folk mulighed for at overføre penge i denne fase, men det kan man gøre *(Ja, det er sikkert en rigtig god idé, for så bliver næste fase sikkert nemmere)). Nu har man så en "penge"-mængde fordelt på en fair måde alt efter, hvem der har vist mest interesse (og altså ofret mest computerkraft) på projektet. (Man kan sikkert også finde på andre måder at gøre dette på, men det er i hvert fald én løsning.) Nå ja, lad mig lige allerede sige, at deltagerne også skal have givet løfter om at deltage i skridtene i den næste fase allerede fra start af, så man på den måde lidt bedre kan regne med, at tingene ikke bare stopper efter startfasen. Det næste skridt er så at begynde at låne disse penge ud til folk (i små mængder først), hvortil man sørger for at give dem løfter om, at man vil købe en tilsvarende mængde krypto-mønter tilbage igen og betale det samme beløb for dem. Dette skal alle møntejerne altså gerne begynde at gøre (hvad de har lovet, at de vil som del af at komme med i fase 1, eller ved at have købt mønterne bagefter). Jo mere sikre folk generelt så bliver på, at penge-ejerne vil købe pengene igen, jo længere tid kan de så holde på pengene uden at bekymre sig om, hvorvidt valutaen kollapser imens. Og fordi penge-ejerne ikke lover at de vil købe specifikt de samme penge tilbage, så kan pengene nu løbe rundt. Dette gør også, at systemet nu godt kan holde til, at enkelte penge-ejere ikke holder deres løfter, for, nå ja, så kan andre jo bare købe pengene i stedet; så længe der er folk, der har lånt penge ud og lovet at give et vist beløb for dem, så vil pengene have denne værdi, medmindre der lige pludselig sker et eller andet voldsomt og en masse mennesker vil løbe fra deres løfter. Samtidigt, og nu kommer det mere spændende, er det så faktisk også meningen, at penge-ejerne løbende skal lade penge-kursen stige lidt. Dette kan nok gøres på mange måder. Det kan bl.a. gøres ved at man bare låner pengene ud til en højere værdi, og så bare giver låner eksklusiv ret til, at sælge pengene tilbage efter en vis tid. Hvis mange penge-ejere så gør dette på mere eller mindre på én gang, så vil den nye, lidt højere værdi jo stille og roligt komme til at blive værdien af pengene. Og det gør så ikke noget, at der stadig findes gamle løfter om at ville købe penge til en mindre pris, for hvem har ikke lyst til at love det? Men kan de samme penge-ejere så virkeligt bare hæve prisen på deres egen egendom på denne måde? Nej, faktisk ikke. (Det mener jeg ikke nødvendigvis, de kan.) Men hvis bare pengene kan ses at skifte nok hænder, således at stake i systemet går på omgang, og at "penge-ejerne" altså ikke bare er de samme, men at det skifter løbende, så tror jeg godt, det kan lade sig gøre. Alle nye købere af valutaen skal så bare lige som en del af købet gå med til at afgive et løfte om, at de løbende vil hæve prisen på at udlåne mønterne til andre. Og graden af, hvor meget kursen stiger, skal så også gerne i en eller anden grad matche, hvor meget stake i valutaen faktisk skifter hænder; ikke bare hvor mange transaktioner, der er, for transaktioner kan i princippet gå fra og til den samme stake-holder, men i stedet hvor meget pengene \emph{faktisk} skifter hænder. Og når pengene og den tilhørende stake i kæden således skifter hænder, så bliver en større og større andel af penge-ejerne altså folk, der har betalt "rigtige" penge, for at holde på krypto-pengene. Og disse skal jo gerne have noget for deres risiko. Dette kan enten være i form af, at valutaen stiger, men det skal den nu helst ikke gøre for evigt. I sidste ende skal den nemlig helst ikke overstige, hvad en normal bank vil have af... Hm..(?).. ..Hm, jeg bliver nu nok nødt til at stoppe for nu.. Men lige så jeg husker det, så fik jeg ikke nævnt min idé nu her med måske at lade prisen stige ved \emph{egen} risiko ved at udlåne til en pris og love at betale tilbage med en lidt højere pris --- måske svarende til en vis procentdel (eksempelvis halvdelen) er, hvad den nuværende kursstigning ser ud til at være.. Og ja, jeg skal også huske, at jeg indtil nu ikke rigtigt har talt om det med (men det skulle jeg lige til..) at begynde at låne pengene ud med en vis rente (for at få belønnet sin risikotagning ved at holde på pengene)..  
%(28.09.21) Så tror jeg, jeg så småt er ved at have styr på det igen. Var på Fyn i weekenden og tog en tænke-dag i går. Jeg synes jo, det blev lidt uldent, da jeg skrev om min BC-idé.. Men ja, nu har jeg vist en meget god idé om, hvor jeg vil hen med det. Jeg kan lige rette og sige, angående det omkring at valutakursen bør falde, når der mines flere mønter: Måske ikke alligevel. Man kan jo også se det som, at der allerede er en fast mængde af mønter, men at en hvis andel af dem bare ikke er låst op endnu. En anden ting er, at jeg vist ikke fik snakket om, at retten til donationsløfter gerne skal kunne sælges og handles med, når det kommer til min donations-kæde-idé. På den måde kan både donorer og "bidragsydere" sælge deres risiko, til hinanden (og kan således også eliminere den) og/eller til tredjeparter. Denne risiko kan så også være interessant, fordi folk så kan "investere" i projektet og ligesom satse penge på, at projektet vil blive en succes, og at folk generelt i sidste ende vil overholde deres løfter. Jeg kan ikke huske om jeg nævnte, at folk jo sagtens kan uploude (evt. fotokopier af) kontrakter osv.; alt er muligt, så længe det selvfølgelig bare er inden for lovens rammer. Desuden kan folk måske også tilknytte udsagn til løfter, der siger, hvilken straf man går med til, at fællesskabet kan give en, hvis man ikke overholder løftet. Hvis man kan understøtte dette med en juridisk kontrakt, er det jo ikke dumt, men ellers kan det jo (mener jeg) sagtens være, at systemet vil blive sådan en succes, at det kommer til være rammedannende for det fremtidige samfund. Og så kan det nemlig blive en realitet, at folk kommer til at kunne retsforfølges (hvilket også bare kan indebære, at de sendes en bøde og/eller skal betale lidt mere i skat), hvis de ikke har overholdt løfter, og hvis de har indvillet og samtykket i, at de vil modtage den pågældende form for straf (hvilket kan specificeres arbitrært nøjagtigt). Og det er så særligt netop den økonomiske straf, der vil være interessant her (det er ikke rigtigt gangbart med andre former, i hvert fald ikke før ude i fremtiden.. Tjo tja, det er selvfølgelig ikke mig, der bestemmer det..). Jeg forestiller mig således at folk eksempelvis kunne samtykke til, at "hvis ikke jeg overholder mit løfte om at betale x antal penge, så må I senere hen (når og hvis systemet bliver main stream nok til, at jeg bliver nødt til at rette mig efter det --- f.eks. i en fremtid hvor de fleste lønninger betales via systemet) beskatte mig 1,5 gange så meget (eller hvad denne faktor nu kunne være)." I øvrigt kunne man \emph{måske} erstatte "mig" med "mig eller min(e) arvtager(e), \emph{hvis} jeg har efterladt dem midler i overskud til at gælden godt kan betales (og hvis det for i fremtiden bliver konventionelt at kunne gøre dette).".. Hm, nå ja, dette for mig til at tænke på en idé, jeg har tænkt lidt over før: om man mon kunne være så fræk at lade skyldte penge kunne "klistres på" formuer, sådan at det skyldte beløb følger med formuen (i en bestemt rækkefølge, ift.\ hvornår de forskellige dele af den sælges)..? ..Dette lyder jo egentligt bestemt ikke dumt..(!).. Det er bestemt en ting, jeg har tænkt på i flere sammenhænge, også udover i denne. Bl.a. minder det også meget om, hvordan man kan omfortolke f.eks. BitCoin, så man kan reducere mining-lønnen eller lade den forsvinde. Men ja, jeg har bestemt også tænkt over denne mulighed omkring selvsamme emne, nemlig omkring lykke-valuta. ..Nå ja, jeg vil også endda have skrevet om det et sted i dette dokument. Nå men, vil dette give mening?.. Ja. Det behøver jeg ingen gang at tænke over rigtigt for at kunne sige. Jo, selvfølgelig bliver dette en god idé. Man skal så bare lige sikre sig, at man folk ikke kan hvidvaske pengene/værdierne, men dette bør kunne gøres ret nemt ved at bare at sætte en protokol for, hvordan skylden kan klistres om, hvis den primære protokol af en eller anden grund ikke holder. Så man opretter altså bare en række protokoller, hvor den ene tager over efter den anden, hvis den tidligere protokol ikke giver noget klart svar --- og angående at vurdere dette, kan man bare skrive noget så simpelt som, at "bestemmelsen skal tages af en gruppe mennesker, der er repræsentativ nok til, at alle ikke-gennemsnitlige biases bør være elimineret i tilstrækkelig grad (til at de ikke har betydning), og som er samlet specifikt til formålet, og som har deres formål og proces erklæret offentligt, så kan få af vide, at nu er en proces i gang til specifikt at afgøre forholdet (og så kan man nemlig nå at stikke fingeren i jorden og bekræfte, at fællesskabet generelt ved annerkende processens resultat)." I øvrigt kan den sidste protokol i rækken også bare være, at samle en sådan gruppe domsmænd til formålet, hvor disse så bare afgør, hvordan skylden nu fordeles på den mest fair måde (ud fra en hjemmel selvfølgelig). Jeg gør det til noget stort nu, men i virkeligheden tror jeg hurtigt, at man kan udforme nogle protokoller, der vil være rimeligt vandtætte. ..Og bum, så har folk lige pludesligt et meget større incitament for at holde deres løfter (hvis de kan; hvis nu løfterne ikke selv tager højde for 'evne til at betale,' så bør skyld-klistre-protokollerne alligevel gøre det). For herved afhænger spørgsmålet om, hvorvidt de skal betale, hvad de lover, ikke bare af, om deres ry/omdømme kan klare det --- og om der nu skulle gå forfald i, hvor højtideligt man tager disse kæde-løfter (i vedkommendes omgivende kultur(/samfund/lokalsamfund)) --- men afhænger i stedet bare af, om kæden (i.e. systemet) vil holde i sidste ende (i fremtiden!) eller ej. ..Ah, dette er virkelig en vigtig tilføjelse til idéen; det var godt jeg kom til at tænke på det! :) ..Det næste, som jeg så videre er kommet i tanke om nu her, er så, at man også meget vel kunne forestille sig, at der i fremtiden med dette sytem bliver en vis skat som alle betaler (i større eller mindre grad), og hvor pengene så fordeles til alle, og, i henhold til hele idéen omkring systemet, hvor de fordeles mere til folk, der har gjort et godt bidrag, og mindre til folk, der har gjort et ringe bidrag. Og med dette i tankerne, så kan vi jo se, at det faktisk måske \emph{kan} lade sig gøre (men ikke at det nødvendigvis \emph{bør}), at klistre-skyld på visse formuer, også selvom indehaver ikke selv har lovet noget af den væk.. Det er måske en anelse kompliceret, men folk kunne nemlig altså love, at de i fremtiden vil stemme (da det nemlig sikkert bliver en demokratisk proces) for en lavere indkomst til folk, der har købt dele af en formue, der kan spores tilbage til en vis nutidig formue, som nutidige vedkommende, der prøver at straffe pågældende indehaver, forbinder et "dårligt bidrag" til samfundet; eksempelvis hvis man har tjent sine penge på forkert/grum måde, der nok burde være ulovlig, men af forskellige grunde ikke er det i nutiden.. Man kan med andre ord altså love at "straffe formuen" i fremtiden vil at påklistre den skyld og så altså love systemtisk at fratrække penge doneret til folk, der har købt dele af selvsamme formue, og/eller at stemme for pengefodelingsordninger, der har samme effekt. ..På den måde kan man effektivt set i fællesskabet, hvis man er mange nok, der mener det samme, få magt til ligesom at trylle visse formuer mindre.. Hm, det virker selvfølgelig ret aggresivt, og måske skal jeg ingen gang foreslå det..(?).. Men ja, muligheden er der altså, så vidt jeg kan se.. ..Tja, jeg kan jo tænke over, om jeg vil nævne det.. Men fedt lige at (gen)indse den mulighed..! 
%Okay, og angående min KV-idé, så har jeg altså også lige gjort nogle tanker i går, og hvor jeg nu synes, at jeg har nogenlunde styr på det. Jeg var måske faktisk ikke helt ved siden af med det, jeg skrev. Men en vigtig ny pointe er, at man bør dele det op i offentlige kontoer og anonyme kontoer, hvor offentlige kontoer er dem, der bidrager til valutaens værdi og stabilitet, og hvor de anonyme så er klienter til det offentlige fællesskab. De er med andre ord "forbrugerne" af kryptovalutaen (og dens fordele ift. gængse valutaer). Ah vent, lad mig også lige nævne, at min umiddelbare idé omkring dette jo simpelthen var: Man må kunne erstatte simpel hype med simple løfter, når det kommer til blockchains. Og ja, det må man jo, men problemet er så lidt, når man først dykker ned i det, at hvis den løfte-baserede variant skal holde, så kræver det også bare lige, at den hype-baserede idé holder, når man dykker ned i den. Vi kan altså med andre ord have, at gangbarheden af en hype-baseret kæde medfører gangbarheden af en løfte-baseret kæde, men så skal vi stadig først også have vist gangbarheden af den hype-baserede variant, før vi kan godkende den løfte-baserede.. Jeg tror nogenlunde, at det er dette, der skabte problemer, da det kom til at uddybe det ovenfor.. Men jeg tror dog stadig, jeg kan sige noget meningsfuldt omkring muligheden for at danne en løfte-beseret KV, og så gør det i virkeligheden heller ikke noget, hvis dette ikke slår igennem. Så længe min donations-løfte-kæde bare slår igennem, og så længe jeg bare lige kan komme med nogle gode bud på alternativer, nu hvor jeg tror jeg kan (og nok har tænkt mig at) afsløre en mulig sårbarhed ved de gængse kæder. Det er så nemlig ret vigtigt, at jeg også har nogle gode lappeløsninger i samme omgang, så det ikke skaber panik (det har jeg ikke lyst til). ..Og det jeg tænker nu, er altså, at der skal være et offentligt fælleskab, der på en måde hjælpes ad med at holde værdien af mønten oppe og holde den stabil, og som i øvrigt sørger for, at de selv eller andre ikke hamstre mønter, men at der altid er en passende mængder mønter i omløb, og at mængden af inaktive penge altid kun er begrænset ift., hvor mange penge efterspørges klienterne generelt og altså, hvor mange der bør være i omløb.. Hm, og så tænker jeg jo noget med, at det offentlige fællesskab løbende belønner sig selv og/eller bliver belønnet (af klienterne) ud fra, hvor aktivt de har været med til at holde valutakursen stabil, hvor den bør være (og hvor godt de generelt har handlet efter kædens forskrifter..).. Hm, dette bør nu alligevel kunne forsimples en anelse, så det tror jeg lige, jeg vil tænke lidt mere over (og gå en middagstur nu her). Nu hvor jeg også lige har (gen)opdaget muligheden i at klistre skyld til fomuer, så folk ikke kan løbe fra deres løfter (givet donations-/løfte-kæde-systemets overlevelse og udbredelse), så virker det jo også som om, man må kunne bruge dette til at lave noget endnu mere simpelt og sikkert.. ...
%Nej, nu tror jeg faktisk ikke længere, at der er en måde at lave en løfte-baseret KV på --- ikke som holder til at blive gennemgået og analyseret (alt kan jo i princippet holde, hvis man aldrig gennemgår det for varm luft). Eller rettere ikke en, der fungerer tilnærmeslsesvis ligesom gængse KV'er. For hvis man tænker på værdien ved at have en anonym, digital og decentralt fungerende mønt (så man bl.a. kan lave smart-kontrakter på en nem måde osv.), så er der bare ingen grund til at "mønt-udlejerne" skal samle på de samme mønter selv. Og hvis man tænker på værdien i at have en valuta, der er (mere eller mindre) uafhængig af andre landes økonomier (og som man måske særligt drømmer om kan afløse gængse valutaer, hvis de skulle ramle sammen), for at folk skal kunne veksle deres formuer til denne valuta, så kan man bare ikke nå nogen meningsfuld vej med den slags cirkel-agtige løfter, jeg ellers lidt havde tænkt på; det eneste man kan gøre kun med løfter i denne forbindelse, er at lave et vist pyramidespil, og det vil altså altid gå ud over dem i bunden af pyramiden i sidste ende. Og disse vil altid så bare vælge alternative løsninger (hvilket man altid kan finde!), og så ramler det jo sammen. Det giver altså ikke mening i bund og grund at skabe en boble sammen i en gruppe for så at sælge boble-værdien videre til andre, i hvert fald ikke hvis disse andre kan gennemskue systemet (og ellers så er det jo bare rent humbug). Bedre alternativer vil altid være, når det kommer til at have en anonym, digital valuta, bare som bank-forening/-gruppe at underskrive juridiske kontrakter (så bindende som det til-den-tid-værende system kan klare --- og hvis dette så på et tidspunkt vil være delvist kollapset, så må man bare ty til et mere etos-baseret system (altså hvor man bare langsomt opbygger tillid til hinanden som handlende parter) i stedet), som så binder bank-gruppen til at betale tilbage for de digitale tokens (og muligvis med lidt mindre, end hvad man solgte dem for, da man udstedte dem). Dette system vil altid være at foretrække for "klienterne"/valuta-"forbrugerne" frem for et system, hvor et underligt system, hvor banken også opbygger en slags pengeboble samtidigt, som de så kan sidde på.. Det vil i hvert fald altid være mindst ligeså godt for dem som noget andet system.. Og når det kommer til at placere formuer i stabile værdier, så giver det ikke mening andet end bare at investere dem i steder, hvor der er brug for dem (medmindre man simpelthen altså bare køber egendomme for dem). Især når først donations-løfte-systemet kommer på banen, så er der ikke andet for end at investere pengene, hvor der er brug for dem --- hvilket så kan være i virksomheder, der kommer forbrugerne mere til gode, end hvordan det har været førhen.. Lad mig i øvrigt lige nævne, at når jeg har skrevet "hype" her i denne tekst, så har jeg nærmere bestemt refereret til den autenticitetsværdi, der kan være omkring mønter fra gængse blockchains. Der er jo ikke noget at sige til, at det koncept faktisk godt \emph{kan} holde, for ja, sådanne ting kan jo nemlig godt have værdi for mennesker. Så ja.. Hm, og tilbage er der vel bare at nævne, at.. især når altså løfte-kæden kommer op og køre (eller altså bare for udsigt til at ville det), så vil det altid kunne betale sig at investere formuer i noget, som en bred befolkningsgruppe har efterspurgt (for det ville aldrig kunne betale sig for denne gruppe ikke at belønne de investorer, der har gjort som forespurgt). Så man vil altså altid kunne finde noget sikkert at investere i, når først dette system kommer op og køre, og når først civil-økonomi-foreninger kommer op at køre (hvilket jeg ikke har skrevet om endnu), og når disse --- og folk generelt --- begynder at analysere samfundet og økonomien mere og gjort det hele mere gennemsigtigt for folk (effektivt set; det hele behøver ikke at give mening for alle; man skal bare kunne forstå de udarbejdede prognoser, og forstå hvad handlingsforslag kommer til at betyde for den enkelte). ..Ah, og jeg skal også lige sige: Jeg vil stadig påpage den "mulige sårbarhed" i gængse kæder, for jeg synes egentligt at den simple løsning, om bare i fællesskabet at sige nej til alle kæder, der på et tidspunkt har været skjult for offentligheden, fungere helt fint. For værdien kommer jo alligevel af "autenticitet," og hvem gider dog også så at samle på fordækte udgaver af KV-mønterne frem for de rene alternativer, hvor hele processen kan ses at have været foregået offentligt? Så man behøver ingen gang, at lave om på noget, hvis man ikke vil det! Bare: Sig nej til ikke-offentlige kæder. Og denne beslutning skal så nemlig gøres inden at for mange andre systemer bygges ovenpå, for jo flere systemer bygger ovenpå, jo sværere bliver det at ændre semantikken. Så det er altså kun godt for fællesskabet, at sådan en (lille) "sårbarhed" kommer frem i lyset tidligt. 
%Nu er jeg så blevet lidt i tvivl, om jeg i virkeligheden skulle snakke om dette først, men det næste punkt er ellers så, hvad jeg har kaldt "civil-foreninger" hidtil. Tanken er simpelthen at oprette foreninger med det formål at finde frem til handlingsløsninger, som hele eller dele af medlemskaren kan begynde at udføre, hvis de vil forbedre deres økonomiske situation eller deres livssituation i det hele taget. I modsætning til et parti eller en fagforening er fokuset altså ikke på at indvægle nogle gode ledere, der så skal bestemme for gruppen, men i stedet er fokuset direkte på viden og på løsningsforslag. Medlemmerne donere så nogle penge løbende, enten for at hyre folk til at analysere medlemmernes situationer og finde frem til løsninger, eller for at belønne (bagud!) folk, der har indsent gode idéer og/eller faktuelle indsigter til gruppen, som gruppen kan få gavn af. Ja, idealet vil faktisk være, hvis man fokusere meget på at tiltrække sidstnævnte form for bidrag; at man simpelthen udlover dusører for gode idéer og god viden, og så ellers selvfølgelig sørger for at efterkomme alle disse løfter, så man fortsat kan tiltrække idéer og viden i fremtiden. Medlemmerne skal så selvfølgelig kunne gå sammen (i undermængde-grupper) om at blive enige om at udføre en handling. Det bør så selvfølgelig være sådan, at folk der får gavn af et løsningsforslag eller en viden bør donere til disse bidrag, imens folk af en eller anden grund ikke kan få gavn af forslaget/informationen, selvfølgelig ikke behøver at donere i samme grad (og sandsynligvis slet ikke, hvis de slet ikke får noget ud af det). Løsningerne som kunne foreslås til sådan en forening kunne være alt lige fra, hvis vi tænker særligt på økonomiske ting, at forhandle sig til grupperabatter for foreningen, at oprette en boligforening til stille og roligt, én bolig ad gangen, at opkøbe medlemmernes boliger til sig selv, således at den effektive husleje bliver mindre (hvorved de sparede penge så kan gå til at opkøbe flere boliger), at analysere markedet og finde frem til monopoler og andre bracher, hvor der er et for stort mark-up, og så oprette et konkurrerende firma (som medlemmerne i foreningen jo passende så ikke bare kan investere i, men også støtte som forbrugere).. (Og måske skulle man netop bare nævne denne ellers ret vægtige idé lidt som en sådan forbipasserende sætning som den..) Det kan også være at genbruge ting internt i gruppen i stedet for at smide ud, hvorved dem der giver videre så kan få en lille bonus, alt efter hvor lang tid tingen holder (hvilket kan registreres i gruppen ved at modtager igen må spørge om samme type ting --- hvilket nemlig også lige skal koste modtageren en lille smule, så man ikke kommer ud for, at folk spørger uden behov). Det kan også være at støtte firmaer (som forbrugere og/eller som investorer), der behandler miljøet og dets arbejdere godt, og altså særligt hvis dette inkludere foreningens medlemmer (og/eller at medlemmerne færdes i og benytter samme miljø). "Grupperabatter" kan i øvrigt dække mange områder.. Og når vi snakker at ivestere i markeder, hvor der kunne være mulighed for forbedringer, så behøver man nok heller ikke samme investering i reklame, hvis idéen om at skabe firmaet kommer fra foreningens medlemmer, hvis medlemmerne er medejere til dels af firmaet (og må regnes for at have fuld gennemsigtighed.. have fuld indsigt i.. have mulighed for at få indsigt i alle sager i firmaet), og hvis de endda alligevel har sat sig for at støtte firmaet som forbrugere. Jeg har ikke styr på, hvor meget af et typisk firmas bugdet går til reklame og PR, så jeg ved ikke hvor mange penge, der er at spare her, men det må da være nogen. Og jeg vil også godt pointere, at selvom der kun måske er en lille smule at spare for folk, så betyder de små procenter altså mere end bare lige. For lad os sige, at man efter tyve år bare kommer på, at medlemmerne sparer 1 \%, end hvis det ikke var for foreningen. Det lyder jo ikke af vildt meget; man vil let kunne spare det samme bare ved at ændre sine egne vaner en smule. Jo, men hvor går de penge hen? Hvis de bare ryger tilbage i økonomien, så er alt fint nok, men et problem i vores nuværende samfund er, at de procenter vor gængse firmaer tjener ekstra i overskud, de går ofte til rige mennesker, der ikke er i stand (og/eller ikke har lyst, men ofte er de bare ikke rigtigt i stand; de kan godt købe flere ejendomme, ja tak, men det flytter bare formuen; de har svært ved at forbruge pengene ordentligt, så de ryger tilbage ned på hænderne af folk med meget lavere indkomster og formuer (så vidt jeg lige ved --- det er i øvrigt slet ikke sikkert, jeg vil have dette med, men nu brainstormer jeg det lige). Så lad os sige at 1 \% om året af folks forbrug (hvilket for mange svarer ca. til deres løn) går til rigmands-pengekasser (billedligt talt).  Efter 30 år vil tæt på 30 \% af pengene i omløb i økonomien så nu ligge indespærret i rige folks kister. Dette kan så bruges til at opkøbe folks ejendomme, så de, yay!, kan få nogle flere penge at købe varer for igen, og hvilket så igen kan begynde at gå tilbage til rigmandspengekisterne igen. Og sådan kan bare 1 \% hurtigt gå hen og blive mange procent.. Hm, eller har jeg regnet noget galt her..? Hm ja, for medmindre deresl løn falder, så vil de have råd til det sam.. Tja, eller boligpriserne kan jo lige netop stige.. ..Hvilket så bliver en fast ekstra udgift, i hvert fald for alle udlejere, hvilket effektivt set kommer til at give dem mindre løn.. ..Og lån for at folk alligevel kan leve udskyder jo så bare krisen.. Hm.. ..Ja, det er jo netop dette, der er problemet med vores nuværende økonomi; vi tænker ikke over langtidskonsekvenserne vil at tillade dem, vi køber vores varer ved (og lejer vores ting ved og giver os faste services osv.), at blive ved og ved med at tage en profit. Selvfølgelig skal folk have profit for at have indført noget nyt og godt til markedet som holder, og som forbrugerne kan blive glade ved, men det må bare ikke få lov til at blive ved og ved og ved og ved og ved med at generere profit --- og slet ikke når de profiterende på ingen måde lover at forbruge deres æg fra guldæg-hånen igen, men bare bruger dem til at opkøbe flere og flere andre værdier. Hvis der så nu var perfekte markedskræfter, så ville vi så slet ikke have dette problem, for så snart patenter og rettigheder er udløbet, så vil markedskræfterne træde ind, andre firmaer vil melde sig på banen og overskudet vil i sidste ende presses ned til noget næsten forsvindende. Så spørgsmålet, hvis vi gerne vil komme den kedelige beskrevne udvikling til livs, er: hvordan får vi forbedret markedskræfterne? Perfekte markedskræfter kræver jo eftersigende, at folk handler perfekt rationelt ud fra egeninteresser. Jamen måske er dette lige netop vejen til en løsning: at få folk til i højere grad at træffe nøjagtigt de beslutninger som handlende individer, der hjælper dem mest muligt. Med andre ord ved at få dem til at begynde at gå op i den \emph{ene procent}!.. Hm ja, kunne man måske ikke godt skrive noget a la dette..? Jo, måske. Det virker faktisk meget fornuftigt. Nå, men jeg skal så også lige ind på eksempler, hvor vi ikke snakker økonomiske tiltag. Jeg kunne så prøve at finde og nævne nogle mere psykologiske områder, eller bare privatlivs-områder i det hele taget, hvor folk kunne have gavn af på en måde at "analysere sig selv" som gruppe noget mere.. ..Ja, jeg må kunne finde nogle små ting lige at nævne, bare som forslag. Nå ja, og jeg bør nævne noget omkring politik. Ja, kort kunne man jo nævne, at løsningerne også kunne være forslag til politiske handlinger, som medlemmerne kan gøre. Jeg kunne måske så selv nævne nogle forslag om, at man bør langt mere kritisk og pragmatisk... Åh, lad mig lige vente to sekunder med det emne faktisk.. Lad mig i stedet lige tænke på, om der er andre ting, jeg skal nævne her. Ret nice, hvis jeg kan klemme min "forbrugerforenings"-idé ind her som en slags sidenote til civilforeningerne.. Tja, men så bliver jeg måske så til gengæld ret politisk, hvis jeg tager de sidste ting her med i teksten (og altså ikke lige venter en omgang til at udgive disse idéer i en senere tekst).. Men ja, måske går det alt sammen.. Okay, flere emner udover omkring politisk aktivitet?.. ..Hm, jeg burde nok også huske at nævne, at de rige jo må være mindst ligeså interesserede i at få vent udviklingen mod noget mere stabilt, medmindre de er totalt tegnebogs-junkies (/formuevækst-junkies). Jeg bør vist også lige pointere, at disse civil-foreninger (eller hvad dælen vi skal kalde dem) kan være arbitrært omfattende *(som i arbitrært store; de vil kunne gå på tværs af landegrænser endda).. Nå jo, jeg tror forresten muligvis, at jeg vil vente med at nævne idéen omkring, at man jo muligvis ville blive i stand til at uddele skyld til folk, der ikke deltager i og/eller ikke har lovet noget på kæden. Det er sikkert ret unødvendigt.. Tja, nu har jeg godt nok lige fremhævet vigtigheden i marginalerne.. Men jeg tror nu alligevel ikke, det bliver nødvendigt i starten, hvis man tænker på systemets udvikling, og jeg tror muligvis samtidigt, det vil virke lidt usmageligt for mange.. Jeg vil selvfølgelig nævne det efterfølgende, men måske bare ikke i min første tekst her (som jeg laver brainstorm-udkast til nu --- eller rettere: lige nu laver jeg overordnet brainstorm over, hvad mit brainstorm udkast (eller 0. udkast) skal indeholde, he). ..Nå, men det er altså lige før, jeg tror det var det. Der skal nok være nogle små ting, men det gør nemlig heller ikke noget, at jeg ikke får alle pointer på kryds og tværs med; det er nok endda fornuftigt ikke at gøre dette. Jeg vil i hvert fald slutte for i dag/aften, og så kan det jo altid være, at jeg kommer på mere at tilføje til i morgen eller i morgen. Alligevel en ret god dag med disse idéer og indsigter. (28.09.21) 
%(29.09.21) Jeg kan også lige nævne forsikring og forsikringsselskaber, som noget hvor man måske kunne spare penge (i en stor civilforening). I øvrigt kan man sagtens som forbrugere presse firmaer til at at handle mere og mere i forbrugernes favør, hvis der bare er mere en ét firma inden for brachen, for så kan man bare true (og følge op på det, hvis det skulle komme til det) med forbrugerboykot, hvis de ikke laver nogle små (og man kan nemlig bare starte i de små, så man ved, de vil gå med til det) ændringer. Når man så har gjort dette kan man begynde som forbruger at favorisere dette firma mere til fordel for konkurrenterne, og endda ligefrem begynde at nærmest boykotte konkurrenterne, indtil de laver samme og gerne ekstra ændringer også. Og på denne måde kan det så gå i ring, så man hele tiden presser firmaerne langsomt ned på et niveau, hvor overskuddet minimeres, og hvor servicen forbedres, gennemsigtbarheden (mangler ordet for det) øges, så firmaerne til sidst bliver nærmest helt gennemsigtige, og hvor miljøhensyn og arbejderes lønninger passer bedst muligt med civilforeningens (eller rettere den gruppe af foreningen, der har valgt at tage aktiv handling) interesser. Nå ja, og jeg kan også nævne.. sygdomsklassificering.. eller måske det skulle nævnes mere under folksonomies..? Tja, jeg kan jo bare nævne det enten her eller der. Okay, og så mangler jeg jo også lige hængepartiet fra i går (som jeg nærmest glemte igen): Politisk aktivitet. Hm, dette lægger godt i forlængelse af at presse firmaer til forbedring.. Man bør også presse sine politikkere til gennemsigtighed på samme måde, i hvert fald hvis man ikke gør andet.. Men ja, i virkeligheden bør man vel bare starte mere fra grunden bare og så oprette et totalgennemsigtigt parti (eller partier, hvis civilforeningen er spredt over flere lande). Dette betyder at alle beslutninger tages i form af offentlige overvejelser og offentlige interne disussioner. Man sørger altså for det første for at opbygge og vedligeholde en offentlig hjemmel i partiet, som alle medlemmerne løbende kan stemme om og justere. Man sørger så desuden for, at alle arbejdere/ledere i partiet optager/streamer videoer og/eller skriver rapporter om samlet set alt, hvad de foretager sig i deres funktion. Og her skal det altså fremgå, hvordan dise handlinger stemmer over ens med hjemlerne. Hvis medlemmerne så ikke er enige, skal de kunne reprimandere og også simpelthen erstatte arbejderen lynhurtigt. Medlemmerne kan så overvåge og diskutere effektiviteten af partiets hjemmel og dens udførsel og kan prøve at finde forslag til nye handlingsløsninger, hvis man tror, der er ting, man kan gøre for at opnå mere (hvilket jo passer til selve idéen omkring en civilforening, nemlig at det er fokuseret på viden og idéer til nye handlinger, man kan gøre). Og bum, så har man et parti, der arbejder efter medlemmernes bedste (i stedet for at man bare en gang imellem, med store mellemrum, vælger en række jakkesæt og tandpastasmil (i høj grad) med de gode ord i munden, og så ellers bare lader dem gøre, hvad de \emph{egentligt} har lyst til / er motiveret til indtil næste gang, man skal stemme). Så ja, dette er, hvad man bør gøre, men ellers kan man selvfølgelig også boykotte politikkere på skift og presse dem hen til der, hvor man gerne vil have dem (men dette er jo selvsagt bare ikke nær så effektivt).. Okay, men det var vel så det for denne brainstorm. Flere idéer ryger nok ned under mine "huskenoter" nedenfor. ... Nå ja, jeg kunne dog også lige hurtigt nævne her, at jeg måske også bør komme med min kommentar omkring, hvordan firmaer jo netop har overskud til at gå op i de små procenter, så der er ingen grund til, at en stor samling borgere/almindelige mennesker/civile ikke ville få gavn af, hvis man gjorde noget tilsvarende (og (mere eller mindre) hyrede folk til at gennemgå de økonomiske situationer og finde handlingsløsninger). 


%\subsubsection[nivi]{}





%Huskenoter:
% - Jeg kom lige på: God idé at nævne tekster med arbejdsopgaver som et eksempel (at forskellige varianter af tekster inden for samme emne, men altså med forskellige prædikater). (tjek)
% - Uh, og fokuser på idéen i en wiki-agtig platform som den naturlige overgang fra at folk ikke er vant til at tænke i tekst-relationer/-prædikater og så til at det vil blive naturligt for folk, at slå facts op på platformen/ene med passende prædikater om sig, og at man altså dermed også får den atomart-/skalart-data-relaterede del af det semantiske web udbredt som en naturlig ting for selv almindelige mennesker (som har benyttet platformen og dens muligheder og dermed lært at tænke i --- og begyndt at forvente --- at ting er ordnet semantisk). (31.08.21)
% - Mere aktualitet (i wiki) --- at man kan tage et udsagn fra et socialt eller andet medie og diskutere det. (hm, jeg kunne jo eventuelt bare foreslå et "aktuelt"-tag)
% - Husk semantisk søgning på programmeringsbiblioteker. (behøves ikke for nu)
% - Husk globale nøgler i tripletter. (behøves ikke for nu)
% - Husk at fokuser på point-aggregater (og mindre komplicereet FOAF..) og åbne algoritmer.. (hm, det skal jeg nok nå til, når det bliver vigtigt/relevant)
% - Diskussionen omkring formen på indhold bliver ret eksplicit og bliver en sag om ratings.. (ja..)
% - "pladsmangel." (mon ikke jeg tænkte på SCID her?.. Jeg tror, jeg har fremhæves, det der skal fremhæves..)
% Fik lige idé til rating med både positiv og så negativ rating på én gang (som en måde at give hurtig kontruktiv kritik på..), og så, hvis man nu kræver ratings af brugeren, at man måske putter et delay på "ved ikke"/"N/A".. (hm, dette bør jeg måske sørge for at nævne.. ..altså at det måske i mange tilfælde kan være smart at adskille ratings i flere, så (måske bl.a.) kritik og positive reaktioner er for sig.. Tja, men på den anden side, lad mig ellers bare lade dette være her ude i kommentarerne for nu)
% - Arbejdssjak som simpel strategi til at bestemme donationsfordelinger.. (tja, dette handler bare om, at man kan forsikre sig selv (på forskellige måder), hvis min "donationskæde" bliver implementeret, og man kan forvente fremtidige donationer som bidragsyder, men dette er alt i alt rimeligt trivielt, så det gider jeg heller ikke at dykke mere ned i her)
% - Psykologi-grupper (og/eller at man laver visse abstrakte antagelser..) er et vigtigt punkt..!.. (jep)
% - ..At det jo bare er det semantiske web på en måde.. før eller efter forklaringerne (på "wiki-side-idé")...  (tjek)
% - Self.: At man kan sælge risiko, og så ellers også det arbejde (eller købe, om man vil), som skal til for at bedømme sin bidragsaktie. (self., ja)
% - (04.09.21) Stikord fra min gode, lange gåtur (fire timer) her i (middag-)eftermiddags: Fokus på point, HTML(-WoApps), (globale nøgler) og HOL. Point til at implementere filtrene for SW-serverne. (Tracking ift. overlay..) Semantiske (produkt-)vurderinger.(!!) Strategier, eller dvs. forbrugs- og civil-foreninger, der udvikler og søger strategier for at forbedre medlemmers situation og tilværelse (hvilket kan ske via medlemsdiskussioner, hyring af eksperter og/eller bagudbelønning til indefra- eller udefrakommende foreslag). Her snakker så eksempelvis også iværksættere, der tror de kan overgå konkurrenter og levere et bedre produkt --- og tilsvarende men for politikere.(!) Noget man altid kan lægge pres på firmaer osv. for at opnå, er for at få mere gennemsigtighed.(!) Og så kom jeg i tanke om min idé til at forbrugerforeninger kunne investere og købe aktier i samme markede, hvortil jeg så kom på, at man jo herved også kan spare reklame (udover penge til overskudsafkast). Og man kunne i øvrigt også med fordel lave distribueringsfirmaer, og her kunne man i øvrigt gøre brug af handelsalgoritmer, så personerne ikke selv skal handle --- og måske er det dejligt at blive overrasket lidt løbende... hm, tja.. --- og så fik jeg også lige en idé med en vaskemaskine / et køleskab som eksempel.. (!) .. Og dette kan blive "en kur til kapitalismen"... (..kan man måske sige..) (..bare dette ikke er for vovet; umiddelbart virker det meget godt..) (..Hm, dette slår forresten lidt mine "kundedrevne virksomheder," på den måde, at hvis sidstnævnte holder, så vil denne idé alligevel finde de muligheder (især hvis jeg lige nævner, hvad "kundedrevne virksomheder"-idéen går ud på)..) (dette var nogle rigtig dejlige tanker; et rigtigt godt sporskift, og især altså idéen til "civilforeninger" var god. Jeg kunne måske lige nævne der her med, at iværksættere og/eller politikkere selv kan være initiativtagere og tage fat i en civilforening med et tilbud, skal jeg det..? ..Ja, nu har jeg lige nævnt de i en sætning sidst i sektionen)
% - Se mine 06/09-brainstorm-noter ovenfor. (de er nedenfor nu. Ah, det var der, jeg fandt ud af at folksonomy-idéen var så god en vinkel på det hele! Og endda også fik idéen til mine simple (vægtede) "brugergrupper", nice.. Tak for den påmindelse, ja)
% - Hvis vi snakker open source, decentral platform til at implementere rating-folksonomies, og ikke bare gængse platforme, der arbejder lidt sammen så at sige, så kan min decentrale måde at lave point-aggregater så komme ind i billedet.. (tja, "point-aggregater" bliver altså ikke nødvendigvis så vigtige i starten, det tror jeg ikke længere.. Og mon ikke jeg har nævnt, det der skulle nævnes om det, det tror jeg..)
% - "Videnskabsorienteret" angående fobrugerforeningerne.. (tjek; hvor jeg jo nu bare har genereliseret dem til "civilforeninger")
% - Eksempler på tekstprædikat-områder: "Arbejdsopgaver, (spørgsmål og svar?..), nsfw, videoer og humoristisk indhold, socialt indhold. [...]" (jo, klart nok)
% - "Utilstrækkeligheden af de konventionelle principper om kildeanvisninger (kort).." (hm, jeg kan bare lige sige her, at vi nu har en tendens til at vurdere kilder meget ud fra, hvor de kommer fra (kommer de fra "respekterede" kilder), og dette gøres så uagtet, at de "repekterede" kilder tit tager fejl i virkeligheden, at "repekterede kilder" nogen gang kan referere til hinanden, hvorved der så bliver en slags cirkelslutning (og der kan forekomme ret grumme cirkelslutninger af denne art), samt også uagtet specifik viden om at en given artikel fra en "respekteret kilde" tager fejl --- her kan man tit overse sådanne fakta, fordi man bliver blændet af selve kilde-etosen (og derfor ikke kigger længere i dybden))
% - Husk video-annotationer.. (tjek)
% - (11.09.21) "Kollektiv inteliggens" som billede omkring, at firmaer har overskud til at fokusere på detaljer, men ikke privat-personer (men det kunne man så netop få ved at danne forbruger-/civil-foreninger). (jeg har vist ikke nævnt metaforen, men den giver lidt sig selv; hvis man er mange nok, har man også større inteligensmæssigt overskud til at finde på og administrere forslag/handlinger)
% - Overvej at nævne idéen om p-ontologier, eller (nok) rettere p-modeller, som hvad vi måske kunne kalde web 3.1.. (hvis ikke ligefrem web 4.0, men jeg hælder nu mere til 3.1 (eller 3.5..)..) (tjek)
% - Semantisk gruppering af især anmeldelser (f.eks. af produkter) er stadig vigtigt. (tjek)
% - "Kan lide"-annotationer til videoudsnit (og andre ting, f.eks. "korrekthed"). (tejk)
% - "Frasortere (allerede læst) semantik fra [mængden af] mod-/med-argument-kommentarer.." (tjek)
% - ..At lave stikprøver, hvor man måske endda ligefrem beder de tilfældigt udvalgte brugere om at forholde sig til et nyt uploadet tag (som en måde at undgå spam).. (ja, det kunne man jo overveje)
% - Nævn at man jo også kan bruge annotationer til at opdele kommentarer semantisk, så man f.eks. kan fjerne dele af en kommentar, hvis den hører til en gentagen semantisk kommentargruppe. (uh, ikke helt dumt, men det er lidt for avanceret til, at jeg gider at nævne det (altså at kunne filtrere dele af kommentarer fra, eksempelvis fordi de er gentagelser af noget, man har læst))
% - Måske: "Husk at diskutere, om folk vil benytte sig af at vurdere og inddele kommentarerne i høj nok grad til at systemet er det værd." (tjek, vil jeg mene)
% - Hører ikke til her, men husk at tjek, at pollen-idé er/bliver nævnt under energi-sektionen. *(tja, måske ikke..) (det var lige en idé, hvor mine tankebaner om at bruge bioorganismer og mine tanker om at give havoverflader højere albedo blandedes, men jeg tror nu ikke, det holder..)
% - *Hører heller ikke til her, men husk: Aske-idé..! *("Måske kunne det endda blive kommercielt...") (tjek)
% - "Ah, man bør også kunne have "under-tags" (som i "underemner" for tags)..!" (tjek)
% - Husk spoiler-eksempel angående fler-inputs-tags (og "se denne ting før denne"..). I øvrigt kunne et andet godt eksempel være noget med, at puzzle-segmenter kunne minde om dem fra et andet spil (altså ting, hvor det ikke er så sjovt, hvis man har prøvet noget lignende før). (hm, lad mig lige putte en nål i denne stikordsnote til når jeg skal omkrive teksten om rating-folksonomies..)
% - "Wes Anderson-agtigt vibe." (tjek)
% - Hører ikke til her, men nævn lige lidt om farve-oplevelser m.m., og det her med, at man ikke vil bemærke, hvis det skiftede (hvis altså vi snakker, når det er helt afkoblet fra de fysiske love (og kun hører til de sjælelige)). (tjek; jeg har lige indsat en paragraf i eksistenssektionen i kommentarerne)
% - Husk at nævn fleksibilitet-fordelen, hvis man virkeligt for indført "semantisk version control" (som man ikke skal kalde det), eller hvad man nu skal kalde det.. (emnet er ikke så vigtigt længere, men jeg har endda sikkert nævnt dette alligevel)
% - Nævn politiske løfte-grupper (og værdien, der faktisk kan være i at tilbageholde stemmer helt).. (tjek)
% - Uh, jeg kunne måske ikke bare nævne symptomer (ift. folksonomy-idéen), men også juridiske forhold og straf for visse handlinger, og jeg kunne endda også nævne normer (og værdier) i samfundet. (hm, hvad tænkte jeg mere præcist her..? Ah ja, det var selvfølgelig noget med at kunne søge effektivt på juridiske forhold i samfundet og sådant. Det gider jeg ikke lige at nævne (andet end her ude i kommentarerne))
% - Uh, jeg kan vel egentligt godt også bare skære en masse detaljer fra i den første tekst, når jeg jo alligevel regner med at komme med en nummer to, og når jeg også bare lægger de her noter ud, hvor de kan dekrypteres (sikkert efter lidt tid).:) (For selvom det er lidt rodet skrevet, så er det vel forståeligt, hvis man bare bruger tid nok.) (jep)
% - Omkring debat-grupper, så forklar at de også gerne netop må opbygge p-modeller, i hvert fald når teknologien bliver til det. (det er så i første omgang blevet til "kurser" og så kan "p-modeller" altså bare blive næste skridt, man kan sigte imod i fremtiden)
% - Man kunne sige: "...Selvfølgelig afhænger brugbarheden af sådan semantisk struktur (og ratings osv.) stærkt af, hvor stor projektet er, og hvor længe kodebasen skal udvikles på (og være "i live," så at sige..)..." (tjek)
% - "Og her, og ved lignende tilfælde generelt (altså f.eks. kategori-relevans-vurderingerne), kan man måske implementere dette som et offset i relevansscoren..? ..Ja.." (ikke ikke 100 \%, men dette virker som noget, jeg allerede har overvejet og arbejdet videre på. Så lad mig bare sige: tjek)
% - Uh, nævn både rapporter og debat-dokumenter som eksempler på, hvad skabeloner kan bruges til!:) (jeg tror, jeg gik lidt væk fra "skabeloner" igen (som noget sensationelt..))
% - (05.10.21) Uh, jeg tror, jeg kan spice donations-kæde-idéen lidt op..! (Ved at foreslå et lidt mere rigidt system for, hvordan bidrag-claims skal uplaodes til nogle PoS-agtige (men det kan selvfølgelig godt bruge PoEtos/PoReputation (jeg skal i øvrigt lige søge på det keyword (altså Po\emph{Reputation}) noget mere..) i stedet) blockchains.) Dette kommer så til at udgøre et system, hvor bidragsydere kan slå sine bidrag sammen og på den måde forsikre hinanden og samtidigt så også blive betalt direkte i KV-mønter i samme omgang. Og det bliver så også simpelt for tredjeparter at købe sig ind på denne risiko (imod betaling i KV-mønter). Jeg tror det er ret vigtigt for mig at komme med sådan et praktisk forslag, hvor der er optegnet en specifik løsning, som folk så bedre kan forholde sig til, og hvor man sågar også hurtigt for nogle faktiske KV-mønter i spil. (nu har jeg en anden version af idéen, så.. tja, og uanset hvad, så kan jeg godt huske denne tanke. Jeg tror, jeg har nævnt, hvad der var værd at nævne om min "donationskæde")
% - Hm, en kamformet blokkæde.. (--"--)
% - "Væddemål om, hvad folk vil mene \emph{efter} kursusteksterne.!" (hm, her var der måske noget interessant, som jeg måske skal forklare nærmere.. ... Jep! Tjek)
% - (Hører ikke til her, men:) "Prog. in life." Altså: Nævn ting omkring, at man gerne må stige i muligheder og frihed i løbet af livet --- og gerne også "velstand," hvis man kan det.. (tjek!)
% - (Hører ikke til her, men:) Højskole-liv med point og sådan.. (tjek)
% - Husk at sørge for, at mit kvanteprogrammeringssprog lige er nævnt et sted i disse noter. (nu har jeg nævnt det i opsummeringen/konklusionen, og jeg vil muligvis endda også nævne det i første udgivelse, så det må være: tjek)
% - ..Hygge-sport, lejrskoler.. løbende uddannelse og spil.. fællesprojekter (selvfølgelig).. Aktivitets- og hverdags-profiler samt interesse- og (måske til dels) persontype-sammensætninger.. (tjek)
% - Jeg bør lige læse min "Øvrige tanker om ... økonomiske og forretningsmæssige emner"-sektion igen.. ... Gjort, og har også læst "Noter omkring muligheder for det fremtidige marked generelt"-sektionen, hvilket egentligt var den jeg tænkte på; en hel udmærket sektion alt i alt..  (så, tjek)
% - ..exp(-iHt)-transformation.. *(og sammenlignet med exp(-iSv)..) (tjek..)
% - Jeg mangler vist nævne noget om at have en god kutur i debat-bugergrupperne (og diskussionsretningslinjerne), om at "djævlens advokat"-argumenter er meget velkomne; det skal være helt okay og endda værdsat, hvis man som person arbejder på at komme med argumenter, også selvom man ikke nødvendigvis tror på og/eller står inde for dem selv. (tjek)
% - (20.10.21) Ah, jeg kom på i nat, at jeg bare kan udgive nogle virkeligt korte sektioner om QED-teori og eksistens. Og jeg vil nok så også nævne mit kvanteprogrammeringssprog og måske endda min faseoperator (hvor jeg måske endda bare kan nøjes med at nævne den fulde, og at jeg ikke har kunne finde noget i litteraturen andet end, hvor det omhandlede "kvantestøj").
% - Ah, det var det. (Fik også endeligt skrevet om "brugerdrevet ML" her lidt tidligere.) Nu tror jeg sgu næsten, jeg er færdig..! 17.03.21-20.10.21, sikke en tur..! ..Lidt vild følelse.. Nå, men nu skal jeg jo allerede hurtigt videre (og med noget af det samme), men alligevel.. ..Nice..


%Husk andre former for aggregater (mere komplicerede).
%(Og ML og psyk.)

%Stikordsnote: Hm, "hjælpsom person"-point...
% ...Hm, giver det mening at indføre ratings af kommentarer...? ..Ja, det gør det vel.. (!) (<- som i 'husk!')   
% Noget andet er: Hvis man laver et separat netværk, som gængse platforme kan hente kommentarer og ratings fra, så ville det vel være smart at lægge op til, at de pågældende sider fastsætter et globalt nøgle-skema til at identificere objekter på hjemmesiden, som altid skal kunne benyttes (helst, så man ikke skal opdatere al dataen), hvorved netværket så kan query'e siden om dets objekter via disse nøgler. .. ..Hm, ja og omvendt kan hjemmesiden nemlig så query'e netværket om kommentarer og ratings.. Ja..(!..) 
%Uh, brugergrupper som postscripts til rating-udsagnene/-tags'ne..!!
%Og nævn eksempel med "safe-space." (!) 
%Og husk at nævne, at for offentlige vurderinger og kommentarer især kan man jo også med fordel bruge algoritmer (lidt ligesom på FB), som viser kommentarer fra ens eget netværk (og måske i flere lag, i.e. venners venners etc.) først. Dette kan i øvrigt også bruges som en måde for den enkelte at sikre sig, at dennes kommentarer bliver set og hørt; ved at trække på ens eget sociale netværk først (og igen minder dette også om FB..).  
%Og nævn hvordan siderne selv kan vælge, hvor mange ting, de vil inkludere, hvilket er en god ting ved denne vinkel, hvor man starter systemet med nogle mere centrale (web 2.0-)enheder, der kan kontrolere og begrænse det, så brugere ikke drukner i verbositet, men at man kan starte med det mest brugbare og så udvikle derfra..! :) 
%Hm, og måske man så kan udskyde ML-delen af det lidt, for det er måske lige en tand for kompliceret ift. resten, og jeg kommer allerede langt i den retning nu med brugergrupperne bare... Tja.. Hm.. Tja, eller måske kan det bare ligesom komme til sidst, og hvor man lige kan advare, om at det er lidt mere avanceret. 



%Ah, angående stikordsnoten, "Hm, giver det mening at indføre ratings af kommentarer...? ..Ja, det gør det vel..", bliver dette ikke *[ikke? Hm, men det har så skiftet siden (hvis ikke det bare var en fejl med det 'ikke'..); de er vigtige for min vinkel pt.] en vigtig idé-retning for den første del af idéen (fra denne nye vinkel)..!? Nemlig at få kommentarerne til et indholdsobjekt blandet ind i folksonomy-ratings'ne også (på en eller anden måde, som jeg lige skal finde ud af mere præcist..).! ..Uh, og angående "hjælpsom person-point," som jeg også lige tænkte på, så kan kommentar-vurderinger komme i spil her, hvilket så kan gå hen og aggregres til 'person-prædikater'..!.. Hm.. ..Ah ja, hvilket jo også er rigtig god motivation for at udvide systemet..!.. (..Så altså fremhæv at en vinkel for at få bruger-til-bruger-vurderinger i spil er via kommentarer...) Hm, og hvordan skal kommentarerne blandes ind i det hele?.. Hm, vel egentligt bare at man gør, så at disse kan vurderes, og at man så indfører.. ja, brugerdrevede (custom-lavede) filtre til kommentarerne, hvilket jo så også bliver et vigtigt og særligt punkt.. ..Jep, vigtige pointer, og vigtig vinkel..!

%(09.09.21) Jeg har nogle punkter fra i går aftes løse tænkeri. Jeg skal nævne undgåelse af spoilers, især også ift. teksthighlights. Så kom jeg også i tanke om, at systemet også skal kunne bruges til at filtrere selve feed'et på en side. (!) Kommentar-emne-tags skal kunne føjes til nemt til et objekt (så kommentarerne kan grupperes efter tags, som hurtigt kan blive indført lokalt til formålet). Her vil det så også være smart.. hvis objekter kan arve tags fra.. hm.. Hm, jeg tænkte vel: Hvis kommentar-emne-tags kan arves fra objekt-emne-tagsne.. Hvilket jo umiddelbart virker godt, hvis ikke det bliver for kompliceret..? Hov, men i første omgang tænkte jeg på, at kommentar-(emne/)semnatik-tags gerne må kunne have børn, så man kan gruppere dem først i en overmængde, men derfra stadig så være fri til at inddele dem yderligere op. (Det var altså, hvad jeg tænkte på med min første blob-graf-tegning.) Men ja, så tænkte jeg vist lige efterfølgende, at det måske kunne være smart, hvis sådanne semantik-tags kan være der fra start og altså afhænge lidt af objekt-emnet.. ..Ja, så når brugere vil tilføje kommentar-semantik-tags, så må foreslagene altså gerne kunne afhænge af emne-tags'ne for objektet. Og ligeledes kan man sige det samme om objekt-emner og kategori-emner.. Hvilket på en ideel side også kommer fra ratings --- det er nemlig et senere punkt fra i går, at jeg skal huske også at nævne problemet omkring fejlkategorisering af objekter på gængse sider (hvor 'tags' og 'kategorier' tit er to adskilte ting). Og dette vil så i praksis kunne implementeres, hvis forslagene også kan afhænge af de tidligere valgte emne-/semantik-tags. Så tag-forslag bør altså både kunne afhænge af de tidligere valgte søskende samt de valgte tags for forælderen (eller en over-forælder for den sags skyld). Forslagsalgoritmerne må gerne være brugerdrevne (ideelt set) ligesom alt andet her. .. Og de her kommentar-semantik-tags (hvor alle "tags" jo nu har en eksplicit rating på sig, så det altså ikke bare er et spørgsmål om 'er' vs. 'er ikke') bliver jo nu rigtigt vigtige for min pointe om bedre, mere semantiske produkt-vurderinger. Hm, og så bør man i øvrigt have ratings-der hører med semantik-grupper også, så folk altså både kan give kommentarer og annekdoter samt oprette relevante ratings inden for det semantiske område.. ..Hm, jeg har et punkt om "original content" som bare er et eksempel på, hvad man kan bruge kommentarfiltrene til, og så har jeg et punkt omkring, hvorvidt man skal bruge moderator-udregnede aggregat-point, hvilket jeg lige kan vende tilbage til senere. Kommentarer skal selvfølgelig også kunne tilføjes rating-tags, og måske (det kan jeg lige se på), skal man også kunne oprette nye ratings som en del af en kommentar, hvor brugerne så altså kan rate et udsagn, der hører til selve kommentaren (således at man altså skal læse kommentaren for at have kontekst til ratingen).. Ja, det er måske ikke dumt.. Husk video-annotationer.. ..Jeg kom også frem til i går, at det kunne være nice, hvis folk kunne.. hm, men det kan jo gøres via kommentarerne: Man kan have kommentar-semantik-grupper, der giver relevante links for indholdet. Og dette kan i øvrigt være et godt eksempel, hvor det giver god mening at have et overemne, der hedder 'relevante links,' og hvor man så kan have flere relevante underemner så som 'kilder,' 'original post,' 'forklarende tekster for kontekst-emnet' osv. ..Nice.. :) Og som jeg tænkte på nu til morges, så kommer man også let til at nå til et punkt, hvor man kan forvente, at brugere selv tagger deres kommentarer. :) ..Hm, skal man kunne tagge sine kommentarer med parametre, så man f.eks. kan give sit eget bruger-id med som parameter..? ..Ratings må forresten også gerne kunne opdeles i semantik-grupper.. ..Så brugere skal altså kunne se ratings om selve indholdsobjektet ude til siden, samt grupper af kommentarer-og(/eller)-links. ..Hm, kunne det mon ikke være en del af pointen her, at SW kan startes mindre formelt (og på allerede eksisterende platforme).. ..Ja..!.. (Ja, hvorfor definere en masse formelle ontologier med bestemte faste URL'er for dokumentationerne? Hvorfor ikke bare referere mere løst til ting, og så linke til dokumentationer, når det er relevant, men uden det store palaver, hvor altså brugere frit selv kan lave nye dokumentationer og dele dem med fælleskabet (og så bare sige, at folk er velkomne til at kopiere de udsagnsdokumentationer)..).. Hm... Tja ja, det er værd at diskutere.. Hm, men man kan nok godt sige noget ret skarpt om, hvor vigtigt det er at kunne udforme selv komplicerede udsagn frit, og hvor alle kan være med, i stedet for at være begrænset af tripletter og faste ontologi-definitioner (/ontologier; jeg skal lige se på, hvad den rigtige definition af ordet er). Hm ja, og det hænger jo egentligt naturligt sammen med, at der er forskellige mål: Det handler om, at der er forskellige tilgange alt efter, om man gerne vil gøre ting mere automatiske og få maskinerne mere med, så de kan "forstå" indholdet, eller om man bare gerne vil strukturere indholdet bedre for brugerne selv (uden at tænke så meget på automatisering --- andet end når det kommer til brugerdrevne algoritmer, hvilket også er forskelligt fra de algoritmer, gængs SW har i tankerne, som i stedet mere er centrale algoritmer, som kan indgå i SPARQL, og hvad har vi). Nice..  






%(06.09.21) Hm, måske skulle jeg starte med at diskutere, hvorfor man bør fokusere mere på "point"...? For når den ged så er barberet, så kan jeg ligesom nemmere forklare det andet.. Hm, men spiller de to ting egentligt ikke sammen? Altså at "fokusere mere på at kunne finde frem til de rette udgaver" og at "fokusere på point-ratings"..? Hm, det gør de vel..? Ja..! Hm, kunne man sige: "fjern fokuset fra at beskrive, hvad eksperter/ekspertgrupper mener, og flyt det mere over på, hvad generelle brugere mener, og så kan man altid lægge filtrene og aggregatalgoritmerne over bagefter (og udvikle dem løbende)"..? ...Hm, "rating-folksonomies" er da en totalt banebrydende idé i sig selv..!!.. 
%..Og ja, det passer da rigtigt godt sammen: Den røde tråd er da: fokuser på det semantiske web ift. brugervurderinger --- og også på så at arbejde sammen om at opbygge tekster (som kan varieres pga. semantisk version control..).. ..Hm, og kan simpel FOAF ikke bare være i form af grupper svarende til mine "moderator-lister," og hvordan disse fungerede..? (Jo.)
%Ej, det er da en total god rød tråd, jeg har fundet her! Også fordi det rating jo også er rigtig vigtigt for prædikaterne til wiki-side-idéen; det er helt centralt, at man så kan rate disse prædikater. Og angående "semantisk version control," eller hvad jeg skal kalde det, så kommer dette også rigtig godt naturligt i forlængelse af, at man jo gerne så vil slippe for at rate hver ny version af en tekst, når man i stedet kan dele det op i lag..! Og nu har jeg endda en komerciel idé i dette, hvis man bare nøjes med dette område af mine idéer, og f.eks. ikke går ind på min Web of Apps-idé.. Eller det kunne man måske godt gøre, og så sige, at dette vil være en mulig løsning, især hvis gængse platforme ikke rubber sig og indfører rating-folksonomies m.m., og/eller hvis de ikke åbner systemet nok op (så man kan genbruge sine FOAF-m.m.-netværk, og så man kan tracke indholdsobjekter på tværs af platforme (så man altid kan hente annotationer på tværs af platforme)). ..? Ja, det lyder faktisk slet ikke dumt.. (!) ..Hold da op, hvor fedt..!.. (06.09.21)
%I forhold til dispositionen kan jeg måske bare komme ind på det semantiske web, nemlig "how it relates to SW," mere i slutningen af redegørelsen..  ..Hm, og ellers kan jeg vel faktisk beholde lidt den samme opdeling, hvor "forbrugerplatformsidé" bare bliver lidt mere til "forbrugerforeningsidé".. ..Og så vil jeg måske opdele det lidt skarpere.. Nå ja, på nær at jeg måske faktisk bare optager idé 2 (WoApps) i idé 1 (som nu har udgangspunkt i rating-folsonomies, hvor jeg før tog udgangspunkt i "wiki-side-idé"). ..Og så skal jeg jo have mine diskussions-/debat-punkter med.. Hm, men det kunne jeg godt have i et punkt for sig efter første punkt (som er redegørelse for r-folksonomies osv.). ..Eller skal det være i et punkt efter..? ..Ja, det er sikkert meget fornuftigt. 
%..Hm, kan det lade sig gøre for gængse platforme også at deltage på en måde i WoApps..? ..Ja, de kan jo godt bruge de samme principper (med "kanaler med tiltag" og sådan)..! ..Ja, og jeg tror så, at jeg vil udelade snak om at erstatte gængse platforme til en sektion for sig. (06.09.21)







\newpage
\subsection{Min QED-teori}

**(29.11.21) I nedenstående noter har jeg skrevet at $\Pi_{A_{||}}$ skal gå som $k^2$, men det passer ikke. Min løsning var at $\Pi_{A_{||}}$ skal gå som $k^{-1}$ (hvilket betyder at $\Pi_{A_{||}} \sim k^{-2}$, og så har jeg altså bare være sløset og kommet til at referere til en ``$k^2$-afhængighed'' i mine papirnoter, og deraf kommer fejlen altså). Så vi har altså $\Pi_{A_{||}} \sim k^{-1}$, og ikke $\Pi_{A_{||}} \sim k^{2}$.\\


Indtil videre se bare følgende tekst kopieret fra en tidligere noteopsamling, som jeg skrev, så den burde være til at følge for en med forstand på fysikken (og som har mit bachelorprojekt for reference), og som, så vidt jeg ved, burde indeholde en holdbar argumentation (medmindre altså jeg lige har lavet nogle fejl):\\

{\slshape
QED-teori:

Nogle samlede tidligere noter (for QED-teori):
Idéen er grundlæggende den, at udlede en relativistisk Hamiltonian for QED ved at kvantisere $A^\mu$ ligesom jeg gjorde i mit bachelorprojekt. Men hvor jeg i mit bachelorprojekt ikke nåede til, hvordan jeg skulle behandle $A^0$/$A^V$-delen eller den ikke-transverse del af $\boldsymbol A$, så har jeg løsningen på dette nu: Man skal inkludere Dirac-electroner, som altså er punkt-partikler, der opfylder Diracligningen, men hvor $A^\mu$ nu er kvantiseret. Vi ser altså på et system, hvor $A^\mu$ kan antage alle værdier (som et diskretiseret felt) sammensat med et system af punkt-partikler, der alle har en position og et Dirac-spin (de er 4-spinorer). Kvantetilstanden er så en bølgefunktion over dette rum og den en midlertidig Hamiltonian er så den der kommer ved at kvantisere felt-delen af den klassiske Lagrangian, men hvor man via Trotter holder Dirac-delen som en kvanteoperator, der for hvert tidskridt tages på Dirac-elektronerne med fastholdt felter (og hvor felterne så bevæger sig ud fra deres Hamiltonian, som jeg udledte i mit bachelorprojekt, men hvor V og ikke-tranverse del af $\boldsymbol A$, kald den $A_{||}$, er inkluderede). På grund af gauge-symmetrien i Dirac-ligningen kan man så se, at V og $A_{||}$ kommer til at være ret frie i den resulterende Hamiltonian. Den mest essentielle del af hele min idé kommer så nu: Man tager så $\Pi_V$ til at være lig 0 for alle bølgevetorer, $k$, og for hver elektron og hver k lægger man et bidrag til $\Pi_{A_{||}}$, som svarer til at et bidrag af $\Pi_{A_{||}}$ gående som $k^2$, hvor $k$ er translateret så den pågældende elektron er i origo af koordinatsystemet. For alle $k$ er bølgefunktionerne af $\Pi_V$ og $\Pi_{A_{||}}$ så frie plan-bølger over felt-amplituden (altså den klassiske felt-amplitude) af den givne felt-bølgevektor. Disse "frie" bølger har så en "kinetisk energi" og denne er så 0 får $V$, idet $\Pi_V$ = 0, og for $\Pi_{A_{||}}$ er den uendelig. Men hvis man skærer den konstante (og trivielle!) uendelige energi fra hver enkel elektrons Coulumb-felt genereret ved $\Pi_{A_{||}} \sim k^2$, ligesom man gør det i klassisk fysik, og kun kigger på den ekstra energi, der kommer fra hvert elektron-elektron par (fordi deres potentialer overlapper), så ser man en Coulomb-interaktion! Hvis vi skærer det kontante enkelt-elektron bidrag til energien fra, ser vi altså, at den kinetiske energi fra $\Pi_{A_{||}}$ kommer til at udgøre vores Coulomb-interaktion. Og hvad der er endnu mere fantastisk er så, at man kan udlede, at denne løsningsrestriktion på $\Pi_{A_{||}}, \Pi_V$, som jo er de frie komposanter som følge af gauge-symmetrien, nemlig at $P \Pi_{A_{||, k}} \sim \sum_{electron\_id} (k'^2)$, $k':=$ Tranlation$(electron\_id; k)$, er bevaret i tid, når vi tidsudvikler systemet i henhold til vores felt-integrationer og Dirac-operationer! *(Og samtidigt er den også bevaret under en Lorentz-transformation! (Vigtigt))\\
Jeg glemte at nævne, at den pågældende Hamilton-operator er den man får ved at tilføje et Lorenz-gauge-agtigt term (et term proportionelt med det term, man sætter lig 0 i Lorenz-gaugen) til den gængse Lagrangian for EM, således at der bliver et kinetisk term for V og $A_{||}$, som så bliver til udtryk med $\Pi_V$ og $\Pi_{A_{||}}$, når man Legendre-transformerer (så vidt jeg da husker). Bagefter føjer man så som sagt et Dirac-led til den resulterende EM-Hamiltonian.\\
Man går så videre herfra ved at konjugere Dirac-ledet, og altså omfortolke de negative energi-løsninger ved at tænke sig et Dirac-hav, hvor alle de negative energier i udgangspunktet så er fyldt op, men hvor de så kan exciteres op, hvorved de danner et elektron-positron-par, hvor positronen så svarer til det efterladte hul i Dirac-havet, bare omfortolket som en partikel. Man får herved omskrevet Direc-ledet via $\gamma$-hæve-sænke operatorer, der nu virker på alle $k$-tilstande, både på elektroner og positroner, men også på alle vakuum-tilstande. Alle matrix-elementerne, som udgøres af disse $\gamma$-hæve-sænke termer, altså udledes ved at diskretisere og begtragte et tilsvarende system i form af et Dirac-hav med huller og elektroner ovenpå (gerne med flere elektroner end huller, det gør ikke noget).\\
Nu har vi så den endelige Hamiltonian, men hvordan i alverden kan vi vise, at dette monster, denne tyr i en porcelensbutik, er well-behaved? Hvis man ser på den resulterende Dirac-oparator, så genererer den umiddelbart til alle tider et underligt mange elektron-position-par med uendelig stor overgangsamplitude, og hvis vi ser bort fra vakuummet har den også uendelig stor samlet interaktion med det elektriske felt (dvs.: $k \to \infty$ giver et ubegrænset (uendeligt stort) bidrag til den tidsudviklede tilstand ($k \to 0$ gør i øvrigt tilsyneladende det samme hvis man laver 1.-ordens pertubation)).\\
Men det jeg så endeligt kom frem til her i starten af året (kan se, det var d. 12. januar forresten), var, at man faktisk godt kan bruge haler, som man lægger til tilstand, som så faktisk kan sørge for, at kanselere amplituderne tilstrækkeligt, sådan at den resulterende tilstand efter en operation med min $H_{QED}$ bliver en endelig tilstand. Dette kan så gøres med ligeså små haler, det skal være, og samlet set kan man så kanselere de samlet set uendeligt store overgangsamplituder, når k går mod uendeligt, tilstrækkeligt til at den resulterende tilstand er endelig med så lille en samlet hale-tilstand lagt til, som det skal være. Det man gør er, at man for hver k-tilstand, som udgangs-tilstanden, som man prøver at tilnærme, har overgange til, sørger for, at der er en hale af tilstande med en ekstra foton i, som kan absorbere denne foton og blive til pågældende tilstand og således kanselere så meget af denne, som det skal være. Hver tilstand i disse haler sendes så også selv ind i uendeligt store tilstande, når man opererer med $H_{QED}$, hvis ikke man også kurerer dette med nye haler (med to fotoner mere i sig end de forrige haler (og som så kan gå til tilstande i disse haler efter to absorptioner)). Jeg troede så i noget tid inde da, at en sådan behandling ville gøre $H$ (\_\{QED\}) ikke-symmetrisk (non-Hermitian ifølge fysik-lingo), fordi man så hele tiden ville få et bidrag fra disse hale-hale overlap (altså overlab af deres resulterende tilstande, når de operares på med H), men det mener jeg så nu, at man kan komme uden om. Pointen er her bare, at man bare skal sørge for, at opbygge sine basisvektorer til Hilbertrummet af en følge af vektorer, hvor hver ny basis-vektor sørger for at lade de forhenværende basis-vektorers haler være (hvis ikke helt, så bare i tilstrækkelig grad), og altså ikke prøver at kanselere overgange til tilstande i k-basen, som er en del af forhenværende basis-tilstandes haler i den nye basis. Dette kan lade sig gøre, fordi halerne kan gøres så små som muligt, og de kan derfor også gøre mindre og mindre for hver basis-vektor, således at størrelsen af alle haler sammenlagt altid vil have endelig (og så lille som det skal være) størrelse, når man ser tilbage på en række af forhenværende basis-tilstande. Man kan så hurtigt se, at hvis basis-vektorerne ikke deler nogen af disse haler (eller at overlappet forsvinder i det mindste), så vil bidraget til ($\bra{\psi_n}$ H $\ket{\psi_m}$) fra halerne af de to basis-tilstande (også) forsvinde. Det kan hermed vises at H er symmetrisk (Hermitian i fysik-lingo).\\
Når først man har vist at H er symetrisk, er det faktisk ret let at vise, at H er selv-adjungeret. Ifølge sætninger i Brian C. Hall, Quantum Theory for Mathematicians, skal man bare vise, at billedet af H +- iI er tæt ("dense") i Hilbertrummet. Eftersom vi kan danne en følge af baser, der går mod k-basen, men hvor H $\ket{\psi_n}$ er endelig for hver basis-tilstand, $\ket{\psi_n}$, i en af disse baser i omtalte følge, så kan vi altså bare vælge en basis tilstrækkeligt tæt på k-basen til at approksimere en vilkårlig ønsket tilstand... Hov, jeg skal indrømme, at jeg lige mangler denne del med, hvordan man viser helt præcist, at man kan nå alle ønskede tilstande ved operation af H på en tilstand i H's domæne... ... Ah, jo. Svaret er faktisk ret simpelt. Vi behøver ikke at vælge vores basis-følge først. Vi skal bare finde en eller anden tilstand i H's domæne, og det domæne inkluderer alle de tilstande vi kan finde på, der overgår til endelige tilstande, når vi virker på dem med H. Hvis vi starter med en tilstand i H's domæne approksimeret til tilstanden, $\ket{\psi}$, som vi ønsker at approksimere, så kan vi i første omgang lave en approksimation af denne ved brug a k-tilstande fra et underrum af Hilbert-rummet med et begrænset k og et begrænset partikel-antal. Kald denne nye vektor for $\ket{\psi'}$. Lad os operere på denne tilstand med H så vi får en ikke-vektor (en uendelig vektor), der bare fremkommer ved at summere over alle k-bidrag fra H. Vi har så en uendelig bølgefunktion, der ikke kan fortolkes som en vektor, fordi den som sagt har uendelig størrelse. Men vi kan for hver k-tilstand i $\ket{\psi'}$ finde et udsnit af vores nye ikke-vektor bestående af k-tilstande, som sendes tilbage til pågældende k-tilstand ved absorption af en foton. Vi kan så gange alle disse udsnit med en skalar, så man får præcis den omtalte k-tilstand tilbage ved (naiv igen) operation med H for hver k-tilstand i $\ket{\psi'}$. Man får så også mere end dette pga. alle de nye tilstande, som vores udsnits-vektor danner ved operation med H, og som vi ikke har sat nogen haler til at kurere endnu. Men vi er nu frie til at gøre dette. Vi kan sågar sætte vilkårligt små haler på vores udsnitsvektor, så alle tilstande, der ikke er en del af de originale k-vektorer fra $\ket{\psi'}$, bliver kanseleret så meget, det skal være, således at den hale-påsatte udsnitsvektor kommer til at approksimere $\ket{\psi'}$ så godt, som det skal være, når vi opererer på den med H. Fordi denne resulterende aproksimation til $\ket{\psi'}$ er endelig, vil den hale-påsatte udsnits-vektor være en del af H's domæne, og fordi $\ket{\psi'}$ approximerer $\ket{\psi}$ vil H taget på pågældende vektor altså også approksimere $\ket{\psi}$. Vi har hermed en fremgangsmåde for hver vektor, $\ket{\psi}$, til at finde en vektor i H's domæne, således at H taget på denne vektor approksimerer $\ket{\psi}$ vilkårligt godt. Dette viser.. nå, ja, jeg skulle lige have erstattet H med H +- iI, men vi behøver faktisk ikke at benytte (det er nemlig kun en fordel) +-iI i dette tilfælde. Så med H +-iI i stedet for H viser dette altså, at Range(H +- iI) er tæt i Hilbert-rummet. Der med ved vi ifølge Hall, at H er selv-adjungeret.\\
Godt, så. Nu er der så bare tilbage, at "gå tilbage langs udledningen" for at vise, at vores udledte, selv-adjungerede H er relativistisk. Til dette kan vi benytte, dette brugbare faktum, som jeg fandt frem til her i sommers: Vi kan bruge, at det, at H er selv-adjungeret, medfører, at de totale (ikke-kanselerende) amplitude-flux til en k-tilstand (hvis vi betragter H i k-basen) ved en lille (f.eks. infinitesimal) tidsudvikling af en tilstand må være endelig, hvilket vil sige at den samlede amplitude-flux, når man summerer over alle andre k-vektorer må være endelig. Hvis vi ser på det ikke-kanselerende amplitude-flux-bidrag fra en halen af k-tilstande med $k>\Lambda$, vil denne altså være forsvindende for $\Lambda$ gående mod uendelig. Dette medfører, at vi for en vilkårlig vektor altså kan approksimere en lille tidsudvikling vilkårligt godt med et ultraviolet cut-off. Den resulterende tilstand efter tidsudviklingen vil så være en ny vektor inden for samme cut-off. Dette lille trick gør resten simpelt. Hvis vi ser på en Trotter-Lie-ekspansion, så kan vi altså approksimere en endelig sekvens af små tidsudviklinger vilkårligt godt ved bare at vælge en stor nok værdi af $\Lambda$. Herfra er det også let at se, at man kan approksimere hver tidsudvikling vilkårligt godt ved at diskretisere k. Vi har altså nu en sekvens af samme operator, der både er diskret og tilmed endelig, og denne operator-sekvens approksimerer tidsudviklingen for vores (selv-adjugerede) Hamilton-operator, H. Nu kan vi gå direkte tilbage samme vej, som vi kom, og først omfortolke de indgående positroner og elektroner som et Dirac-hav, hvorved vi kan opnå en diskret og endelig Dirac-operator på et endeligt og diskret partikel-system (om end dette system indeholder rigtigt mange partikler; en for hver negativ-energi k-tilstand (to for hvert k) plus antallet af elektroner minus antallet af positroner). Da feltet også er diskret kan vi nu bruge, hvad vi allerede har vist. Vi kan tilføje to yderligere dimensioner (også diskrete og endelige!) til vores kvantesystem, nemlig V- og $A_{||}$-felterne, og antage at disse har bølgefunktionen givet ved $\Pi_V = 0$, $\Pi_{A_{||, k}} \sim \sum_{electron\_id} (k'^2)$, $k':=$ Tranlation$(electron\_id; k)$, uafhængigt af resten af frihedsgraderne. Vi har så vist, at denne omdannelse af systemet er ækvivalent med det kvantesystem vi havde før kun med de transverse dele af $A^\mu$-feltet, hvis vi bare sørger for at fjerne Columb-interaktionen fra H. Nå ja, jeg fik ikke nævnt, at den fulde H selvfølgelig inkluderer det Coulomb-led, som skal sættes på i stedet, når vi fjerner $V$ og $A_{||}$. Vi har også vist, at vi har lov til at separere Dirac-delen fra $A^\mu$-delen i Trotter-Lie-ekspansionen, og nu kan vi så Legendre-transformere feltet for sig, hvorved vi opnår et felt-integrale med Dirac-partikler (4-spinor-partikler) ovenpå, der for hvert tidsskridt i path-integralet skal operares på med Dirac-operatoren (altså en flere-partikel-version). Jeg har ikke nævnt dette, men dette samlede felt-integrale svarer til et normalt felt-integrale kendt fra litteraturen, men hvor det resulterende bidrag for hver felt-path i integrationen er en Dirac-partikel-operator og ikke bare en skalar-fase. Vi kan så vise, at alle sådanne felt-integraler kan Lorentz-transformeres (efter ret meget samme princip, som jeg beskrev i mit bachelorprojekt), hvor gauge-symmetrien så tages i brug for at vise at den Lorentz-transformede tilstand også opfylder vores restriktioner (altså vores $\Pi_V = 0$ og $A_{||}$ -restriktioner). Vi kan altså vise, at det er ligegyldigt om vi Lorentz-transformerer først, tidsudvikler i det nye system, og transformerer tilbage, eller om vi bare tidsudvikler i det originale system. Vi er hermed i mål.\\
Vi har nu vist at vores H, som består af en fri energi for fotonerne, en Couloumb-interaktion mellem elektroner og positroner, som faktisk ikke er helt triviel, fordi den lige skal transformeres korrekt fra vores Dirac-havs-fortolkning (i positions-basen) og så til vores elektron-positron-fortolkning (som i udgangspunktet er i k-basen), samt vores Dirac-led bestående af $\gamma$-hæve-sænke-operatorer, både er well-behaived og også relativistisk (og med det mener jeg, at den følger den specielle relativitetsteori).

Det var en ordenlig smøre. Til gengæld uddybede jeg det så meget, at en engageret fysiker med lidt tid til overs godt *(måske) ville kunne udlede de samme resulterer ved kun at gå ud fra ovenstående tekst.

Hov, vent lige. Jeg kan vel ikke vide, at alle vektorer, der ved naiv operation går til en endelig vektor ved operation med H er en del af domænet. Jeg skal altså nok lidt vælge en base først og så vise, at jeg med denne kan nå alle vektorer i Hilbert-rummet.. Jeg skal snart rejse over og holde jul på Fyn, så jeg tror bare, jeg lige lader det summe over de næste par dage (og f.eks. i toget på vej over), i stedet for at sidde her og bruge tid på at tænke over det.

(...) 
Jeg fandt på en løsning til at vise, at Range(H +- iI) er tæt og dækker hele Hilbert-rummet. (...) Hvis man starter med at definere H over alle vektorer i baser af en basis-følge, hvori basis-vektorene konvergerer mod vektorer, der sendes tættere og tættere på de rene k-tilstande af H *\{Dette kan man godt gøre uden at bryde H's symmetri.\}, så kan man derved vise, at denne H opfylder kravet. Man gør dette ved at lave halen mindre og mindre og sørge for at få den til at kvæle alle overgange væk fra den k-tilstand, vektor-følgen centrerer sig om, i hvert tilfælde. I øvrigt vil jeg lige nævne, at hvis ikke det duer at se på det fulde Dirac-hav, så var min tidligere tanke at se på spredningsmatricen og vise, at den er relativistisk, men jeg tror nu, at man godt kan komme igennem, som jeg beskrev det ovenfor.

(...)
Angående QED-idéen, så bør jeg også lige overveje at skrive om, hvordan man kan udvikle felt-integraler over idéen, som passer lidt med gængse Feymann-diagrammer, men (helt) sikkert ikke helt.

*(Yderligere note til det sidstnævnte:)
Man bliver nødt til at finde (eller at approksimere) egentilstande for elektronerne, hvis man danne tilsvarende Feynmann-diagrammer. Det er jo lidt hele pointen med dette projekt, at der bør ske så meget i Coulomb-interaktionen, at man ikke nødvendigvis kan forvente, at simpel pertubationsregning vil føre til det rette svar. Man kan sagtens lave felt-integralerne om til Feynmann-diagrammer, men man bliver så bare nødt til at starte og slutte med at integrere over en start- og slut-egentilstand, for ellers kan man ikke vide, at resultatet er sigende for noget (hvis man altså bare bruger k-tilstandene fra den frie teori).

*\{(12.02.20) Man behøver vist faktisk ikke at se på amplitude-fluxet. Man kan bare se på en approximeret start- og slut-tilstand, begge med cut-offs på... Nå nej, der kan jo stadig være overgange først ud og så ind over cut-off'et. Men så er det derfra nemt at argumentere for, at enhver tilstands udvikling kan approksimeres med et passende cut-off (for både transitioner ud og ind må aftage med cut-off længden). Hvis vi ser på et cut-off i første omgang af, hvilken mængde vektorer vi vil sikre os bliver approksimeret tilstrækkeligt, så kan vi så bagefter sætte et ydre cut-off på transitionerne under udviklingen, og dermed bliver alle vektorerne inden for det indre cut-off approksimeret tilstrækkeligt. Men et spørgsmål jeg vel så faktisk mangler at besvare er vel så, om man herfra kan komme tilbage til de rene k-vektorer, så man kan sikre sig at de fysiske symmetrier holder? Skal vi så faktisk bruge et tredje, indre indre cut-off? Hm, men så har man ikke samme argument for, at amplitude-fluxet forsvinder, for... Ah, jeg kan bare bruge prop. 10.14 i Hall, og så gå direkte til at betragte de rene k-vektorer?... Nej, nu tror jeg, jeg har det. Jeg tror den bedste løsning er bare at lave en Trotter-ekspansion i henhold til prop. 10.14 i Hall (ved at bruge definitionen af H) og så betragte den fulde sum (over hele k-rummet). Da denne må være endelig for hvert k, må man kunne lave et cut-off i summen, og hermed får vi vores Trotter-ekspansion over et endeligt rum af k-vektorer. Eller dvs. det er ikke som sådan en Trotter ekspansion, for man har ikke approksimeret eksponentialfunktionerne endnu til første orden, og det behøver man heller ikke at gøre i dette bevis. Man behøver vist heller ikke at separere eksponentialfunktionerne, og kan så bare bruge den fulde $U(t)=\exp(-iAt)$... Hm, men jeg vil jo gerne alligevel lave Trotter-ekspansionen så snart jeg vil diskretisere tiden, så den skal nok komme alligevel. Men ja, det virker som den mest simple løsning; bare at argumentere ud fra den nævnte sum og bruge, at den skal være endelig.

Angående det med Dirac-havet, så tror jeg, det er nemmere, som jeg også havde tænkt mig, sidst jeg slap emnet (før disse noter; så altså i sommers), hvis man bare samler hæve-sænke-operatorer sammen, som jeg nævnte i en tilføjelse ("til spredningsmatricen"). Disse (Dirac-)hæve-sænke-operatorer bør af konstruktion overholde SR. Det eneste man gør er jo at omfortolke transitionerne. Jeg tror så, at dette vil føre til en mere elegant løsning, end at opfylde det diskretiserede k-rum med et Dirac-hav. Jeg er heller ikke helt sikker på, at det ville holde lige så sikkert, men det burde til gengæld som sagt holde, hvis man argumenterer ud fra Dirac-hæve-sænke-operatorerne, som man, lige for at præcisere, altså betragter i et felt-path-integrale, hvor man har konventeret fotonerne til $A_{\perp}$. Skal man egentligt så også smide resten af $A^\mu$ på først? Tja... Tjo, man skal egentligt bare argumentere før, at alle transitioner i den diskretiserede teori (med konjungerede hæve-sænke-operatorer (og således omfortolkede negativ-energi-løsninger)) passer med en transition i et diskretiseret rum med ikke-konjungerede h.-s.-operatorer og med det fulde $A^\mu$-felt. Hm, jo så man kan vel argumentere således, hvis den ukonjungerede teori er relativistisk, så må den konjungerede teori også være det; begge teorier med fotoner og ikke felter. Så man ser altså bare på den fulde S-matrix. Jamen vi kan jo vise, at den ukonjungerede teori (med ukonjungerede HS-op'er) er relativistisk (overholder SR), og så følger konsekventen. Og for at vise antecedenten, så er det så, at man går videre og danner $A^\mu$ ud fra fotonerne og Coulomb-interaktionen. Yes, det virker mere fornuftigt.\}
}
\\

Bemærk, at $\Pi_V = 0$, $\Pi_{A_{||, k}} \sim \sum_{electron\_id} (k'^2)$ er den store idé i alt det her,\footnote{Hvilken jeg i øvrigt fik ret kort tid (var det to uger eller to måneder?\ldots det var dog to måneder, kan jeg se, nemlig i maj 2017, men altså stadig ret kort tid) efter mit bachelorforsvar. **Rettelse: Det skal være $k^{-1}$ i stedet (som jeg også nu har nævnt i starten af denne (under-)sektion).} hvis man regner fra, hvor jeg var nået med mit bachelor projekt (hvor jeg hele tiden sigtede imod at få elektronerne ind som frie partikler og på at tilføje deres dynamik i form af en Hamilton-operator og ikke i form af en Lagrange-funktion (for det har nemlig aldrig givet mening for mig, når der er tale om fermioner; jeg kan simpelthen ikke vikle min hjerne om den tanke, og dette var én af grundende til, at jeg begav mig ud på denne rejse)). Resten derefter er mest bare\ldots\ om end der er nogen vigtige tanker, og en hel (\emph{hel}!) masse arbejde,\footnote{Især fordi jeg virkeligt bare ikke kunne fatte, hvordan jeg skulle kurere de uendelige (afbillede) vektorer og de uendelige parproduktioner --- især ikke før jeg fandt Hall, og selv da tog det lang tid, før jeg endelig anede løsningen (bl.a.\ fordi jeg også lavede en forkert udregning undervejs, der konkluderede, at det ikke kunne lade sig gøre, og som fik mig til at lægge det på hylden endnu en gang (indtil jeg endelig tog det frem igen nogle måneder efter, der i januar 2019)).} i det (bl.a.\ var det da en meget god idé, om end ikke en ikke helt uoplagt idé, at tage seriøst fat på Dirac-havet i bevismetoden\ldots), så er resterende altså mest bare at få beviset ført igennem og få fundet de rette approksimationsskridt og kureret alle de uendeligheder, der tilsyneladende opstår. 
Nå ja, og jeg skulle måske også lige understrege vigtigheden i det med at denne restriktion er bevaret under en Lorentz-transformation (i systemet af Dirac-elektroner oven på et $A^\mu$-felt, og hvor man altså kan beskrive dynamikken med et felt-integrale af $A^\mu$-feltet, hvor hver ``path'' (altså hver `integrale-sti,' kunne vi kalde det på dansk) i ikke går fra fase til fase men går fra elektron-vektor til elektron-vektor (i et flerpartikelrum af 4-spinor-partikler), hvor overgangen er givet ved Dirac-ligningen over det nu klassiske felt i integrale-stien). I øvrigt kan jeg ikke huske, om potensen er rigtig her i $\Pi_{A_{||, k}} \sim (k'^2)$, så jeg stoler altså bare på mine noter. Og i det hele taget stoler jeg i øvrigt også bare på, at jeg ikke har lavet nogen store fejl i mine udregninger, så selvom jeg ikke har tænkt mig at gøre det nu, så bør jeg jo helt klart lige gå igennem det hele, inden jeg prøver at udbrede mig om idéen --- hvad jeg jo bør alligevel af flere grunde, men det var altså særligt for lige at understrege, at jeg sagtens kan have regnet forkert et sted; det har jeg jo været ude for 1.000 gange før. 





%Hm, jeg tror faktisk ikke alligevel, at jeg vil gå særligt meget op i at udgive og skabe interesse om min QED-teori her i starten; det kan jo sagtens stadig være, at der er et eller andet ved den, der ikke holder, og så kan det jo gå hen og blive et større arbejde, og samtidigt er det heller ikke sikkert, det bliver særligt nemt at overbevise folk om, at der er noget nyt i det (hvad der jo er.. 7, 9, 13, men det skulle komme meget bag på mig, hvis en lignende teori eksistere i forvejen..). Så Jeg tror egentligt bare jeg vil fokusere på min ITP-idé her i starten, og så kan jeg altid se, hvor meget tid jeg får til at arbejde på fysikken og skrive den (længe ventede) artikel. For idéen med at fokusere på fysikken også ville være den hype, det eventuelt kunne give mig, så jeg derved måske kunne få folks øre i højere grad, men på den anden side, så vil det nok i virkeligheden være bedre bare at fokusere på at forklare om ITP-idéen og således fokusere på den hype, der kan dannes om idéen, når den bare står på sine egne ben. Og den tid, jeg ville bruge på fysikartiklen og den efterfølgende udbredelse af den (altså kontakte mit fysiknetværk og prøve at få nogle her interesserede), kunne jeg jo også bare bruge på at selv at skabe en prototype til min ITP-idé, hvad der jo heller ikke ville være helt dumt. Jeg føler på en eller anden måde, jeg skylder fysikken, at gå op i min teori og følge projektet til døren, men på den anden side så er det da også en fornuftig nok ting at sige: nå ja, men jeg kan jo ikke vide, at den er noget nyt; det kan jo sagtens være, at den allerede er dækket af andre teorier; at den er et korolar til andre teorier nærmest, så hvorfor skulle lokummet brænde med at få teorien ud, især når nu jeg har en anden idé, som jeg virkeligt tror kan gøre en kæmpe forskel inden for relativt kort tid for samfundet (imodsætning til, lad os være ærlige, min QED-teori, selvom der nok skal blive en vis værdi i den, det er klart (hvis altså den virkeligt bringer noget nyt med sig teoretisk set, 7, 9, 13..)). ..Nå ja, og en tanke med at udgive det var også lidt, at når så først det er gjort, og jeg har lagt nogle tidsløste noter ud med resten, så ville jeg i princippet kunne få lidt ro i sindet over, at nogen nok skulle nå at opdage mine idéer uanset hvad, men dette argument / denne trang holder bare ikke rigtigt. For for det første vil en artikel over min ITP-idé virke meget bedre til at skabe interesse over mine andre idéer på kort sigt, og hvis vi ser på den længere bane, jamen så gør det ikke nogen forskel ifølge min umiddelbare analyse. For vi skal nok få sem-webbet og alt det tilhørende på et tidspunkt inden for en nær fremtid, så der bliver altså ingen pointe i at tidslåse mine noter med tanke om at de skal findes frem i en halvfjern fremtid. Nej alt i alt er det meget bedre at bruge krudtet på ITP-idéen, og så også gerne lave en god artikel som kan oploades til en god og bestandig database (og hvorfor så ikke også lige lave nogle tidslåste noter også, det er jo lynhurtigt), så ja, det er altså nok der, jeg bør bruge krudtet. ..Og lige en ekstra note: Det smarte ved tidslåste noter, hvorfor jeg også vil gøre det, er også, at jeg så bare kan fokusere på det grundlæggende først, uden at tænke på risikoen ved, at jeg af en eller anden grund så ikke når at komme ud med mine mere avancerede idéer (man ved jo aldrig, hvad der kan ske, og det er jo som sagt nemt og hurtigt nok lige at gøre). (05.05.21) 
%*Jo, nu er min plan (d. (06.06.21) i dag) godt nok at skrive en idé-skitse over QED-teorien også, og så faktisk bare dele den ud frit og forklare om teorien frit, inden jeg udgiver en artikel (til en preprint-side). Så kan jeg jo bare se om der er interesse først, og hvis der er, så kan jeg så begynde at skrive en artikel.




%*(01.09.21) Lad mig lige nævne kort her ude i kommentarerne, at jeg virkeligt ikke har meget tilovers for konventionel kvantefeltteori.. For mig virker det i hvert fald virkeligt fjollet. På den anden side skal det så siges, at næsten hver gang, jeg har været skeptisk omkring et eller andet, som eksperter har været enige om, så har jeg oftest taget fejl. Så min skepsis skal selvfølgelig tages med et gran salt, også fordi jeg heller ikke nåede at bevæge mig så langt ind i teorien (det var for svært at finde motivationen, når intet var tilnærmelsesvist velbegrundet, og at man bare skulle acceptere tonsvis af ugyldige matematiske manipulationer (Srednicki), og så bare bevæge sig videre). ..Og at bruge felter af antikommutatorer, som aldrig kan blive noget kontinuert, for så at behandle dem dem kontinuere felter og bruge differentieloperatorer på dem osv... I stedet for bare at indføre fermioner som partikler; hvorfor skal det være felter, de giver jo ingen menning..! Nå, men jeg har altså et ben i siden på konventionel kvantefeltteori og kan ikke lade være med at synes, at det virker som det rene humbug, men jeg er selvfølgelig også farvet af, at jeg gerne vil fremføre min alternative tilgang til det, især nu hvor det virker til at være lykkedes at finde en teori, der holder (..og som rent faktisk giver mening..).. Tja, eller det er så slet ikke fordi, at jeg har set mig endnu mere sur på det konventionelle, efter at jeg fandt mit resultat, slet ikke. Min utilfredshed med de gængse tilgange kom helt klart først. Men ja, alt dette er ikke noget, jeg skal råbe højt om; det kan sagtens være, at jeg tager fejl, og at de gængse tilgange har nytte. Og der er ingen grund for mig at til finde disse pessimistiske sider frem. Nå ja, og når nu vi taler om konventionelle tilgange, som jeg ikke rigtigt regner for noget, så synes jeg altså også det er fjollet at tage singulariteter (i sorte huller altså, ikke i Big Bang, det er fint nok) seriøst. Det må næsten være muligt at vise, at singulariteter ikke kan dannes i klassisk GR (og altså hvis de ikke var der i forvejen). Så kan man så fantasere om, at kvantemekkanikken kan komme til hjælp og faktisk gøre, at der kan dannes singulariteter (...) Hvorfor?! Hvorfor i al verden skal vi pludselig købe denne idé. Hvorfor vil folk \emph{gerne have} singulariteter; de skaber jo et væld af paradokser... det er jo nærmest definitionen på en singularitet, for fanden..! Hm, det \emph{er} faktisk definitionen på en singularitet.. En grænse hvorefter fysikkens gældende love ikke kan fortsætte kontinuert, er definitionen vel.. Så hvis hvad der sker efter denne grænse er ren fantasi, hvorfor skal vi så tage det seriøst.. Og især når det i første omgang er en fantasi, at der skulle komme nogle eksterne love fra kvantemekanikken (som vi endnu ikke kender til, btw; der er ingen ting i kvantemekanniken, der siger at singulariter har mere grund til at opstå) og rede.. Rede hvad? Rede retten til at tage disse grimme ting, som singulariteter er for fysiske love, seriøst?..?? Jeg forstår det ikke. ...Hm, måske kan det hjælpe, hvis flere folk tilskrev sig CUH (og måske især hvis man så også tilskriver sig at tro på, at Oplevelser er, hvad Eksisterer, på det mest grundlæggende plan), for det ville måske afmystisere det hele en del, og folk ville så måske få en mere pragmatisk tilgang til tingene... (Tja, hvem ved.?..) 
%(16.09.21) Nå ja, lad mig forresten endelig også nævne, at jeg ikke selv kan se (og jeg er dog selvfølgelig biased her), hvordan den transverse foton-elektron-interaktion skal kunne give et Coulomb-felt. Og tillige kan jeg slet, slet ikke se, hvordan en korrektion fra højere-ordens pertubationregning så skal kunne give en ændring til den magnetiske interaktion, men ikke til Coulomb-kraften?? Men ja, selvfølgelig er jeg baised, og jeg har slet ikke sat mig særligt godt ind i, hvad de konventionelle argumenter er (det hele er så kryptisk, når man snakker gængs QFT, og jeg orker det ærligt talt ikke). 

%(29.09.21) Uh, og lad mig lige nævne også noget omkring kosmologi. Der er ikke et emne, jeg har specielt meget forstand på inden for fysik, og jeg ved godt, at det virkeligt er amatøragtigt at sidde her og komme med alternative forklaringer, men here goes. xD Jeg vil bare nævne, at hvis nu GR skal være foreneligt med kvantemekanik, så har jeg personligt lidt svært ved at se, hvordan man skulle kunne have en asymptotisk opførsel såsom et sorthul, hvor bevægelse går asymptotisk i stå i uendelig lang tid. Hvis det skal kunne beskrives af en kvantemekanisk teori, så må man næsten forvente, at "sorte huller," i det omfang de eksisterer (og jeg tror jo ikke rigtigt på Schwarzchild- sorte huller), på et tidspunkt vil eksplodere igen --- måske ikke fuldstændigt på én gang; måske bare i bølger lidt ad gangen. Så langt, så godt. Med det på sinde, og i øvrigt også med tanker på sinde om, at universet jo, så vidt jeg tror, bør have en prior-sandsynlighed fra CUH/MUH (eller rettere fra min version af dem; mine tanker afviger sikkert lidt fra de tidligere), og derfor meget gerne, i hvert fald ifølge hvad jeg er kommet frem til på nuværende tidspunkt (men også ifølge andre versioner af teorien, ikke kun min nuværende), må indeholde... Hm, må det nu egentligt også det..? Nå, det er også lige meget, det kan jeg diskutere og overveje en anden gang, men om ikke andet, så vil tanken om, at universet ikke bare kun er 13,7 mia. år gammel, men i virkeligheden er langt, langt ældre (som i: svimlende) give ret god mening i mange eksistensteorier.. Men ja, hvis vi bare holder os til fysikken, så er der stadig bare det med, at sorte huller jo nok, hvis de skal være kvantemekaniske, må eksplodere igen på et tidspunkt. Og med dette på sinde, så ville det jo give meget god mening a priori (og altså før man kigger på den kosmologiske data, som jeg jo ikke har meget viden omkring), hvis det kendte univers var født ud af en sådan eksplosion. Ok. \emph{Så} langt, så godt. Og den næste ting er så, at jeg før har undret mig over, hvis det kendte univers' massetæthed bare forsætter og fortsætter efter dets nuværende rumfang.. Lad mig spørge sådan her, hvis man har en klynge af stjerner og/eller planeter og nu øger afstanden imellem dem, hvad sker så med den lokale GR-tidsfolængelse ift. set fra et punkt langt væk? Det er ret nemt at svare på, især hvis man kan bruge pertubationsregning. Jeg mener at jeg bekræftede på et tidspunkt, hvis ikke jeg husker galt (bare med en hurtig betragting, men alligevel), at der vil være en forskel i omtalte tidsforlængelse proportionelt med, hvor stor klyngen er. Så den umiddelbare tanke er altså så: hvis vi er i en virkeligt, virkeligt stor klynge (for slet ikke at tale om en uendelig klynge..) med en vis massetæthed, og hvis alting bevæger sig væk fra hinanden (måske fordi vi kommer fra en eksplosion), hvor meget vil tidsforlængelsen imens denne ekspanderende bevægelse (men af fysiske objekter og ikke af rummet selv) sker? (Og denne tanke bliver i øvrigt nemmere at forholde sig til, hvis man antager at vi \emph{er} en del af en lokal klynge med stor massetæthed kommende fra et sort hul fra et "tidligere univers," så at sige (selvom dette selvfølgelig så vil være flydende), for så \emph{kan} vi forestille os et punkt langt, langt væk, fra al masse, hvorved det så giver mening at sammenligne tidsforlængelsen med dette.) Og spørgsmålet bliver så hurtigt: vil tidsforlængelsen i denne hypotese kunne passe med, at vi observerer, at universer langt væk virker rødforskudte, netop fordi de vil have udsendt deres lys på et tidspunkt, hvor der var en anden lokal tidsforlængelse, end den der er nu? Tja, jeg er ikke sikker på, at jeg kender et modargument til dette (også i øvrigt fordi at den såkaldte "accelererende" udvidelse vist nok, hvis jeg hørte min lektor rigtigt engang, er lidt en misforståelse --- medmindre man altså forstår det rigtigt; jeg hørste det nemlig som om, at den procentvise udvidelse af rummet var konstant, men at dette så bare medfører en acceleration, når det kommer til objkternes bevægelse..). Man kunne måske argumentere ud fra en begivenhedshorisont, når det kommer til, hvilke fortidige objekter langt ude, som vi her kan "føle," men hvis vi kommer fra et punkt med virkelig.. Tja.. Jo, der kunne der måske være mulighed for et argument.. Men på den anden side kunne man, med min begrænsede viden (og for begrænsede tid til at prøve at regne på det), godt tænke, at en sådan udvidelse af de fysiske objekter, hvis dette fører til en udvidelse i selve rummet, muligvis også give en inerti til denne rummelige udvidelse (sådan plejer det i hvert fald ofte at være, når fysiske objekter og felter interagerer, og jeg ved i hvert fald fra eksistensen af tyngdebølger, at rum(-tid) kan opnå inerti i dets bevægelser). Men ja, det er dog ikke sikkert, at dette ville modvirke et event-horizon-modargument, hvis et sådant kan opstilles.. Men ja, anyway, det var altså lige mine tanker omkring dette. I øvrigt kan jeg nævne, at der jo eftersigende ser ud til at være en forskel i baggrundstrålingen, således at vores lokale univers ser ud til nærmest at have en orientering. Dette kunne jo måske forklares, hvis vi kommer fra et eksploderende sort hul (og måske mere end det kan forklares af en stor heterogenitet i det nyfødte univers).. 
%(30.09.21) Nå ja, en lille sidenote, som jeg også gerne ville have nævnt, er, at hypotesen om, at vi kommer fra et rigtigt stort eksploderende sort hul, ikke rigtigt duer, hvis universet udvider sig konstant på et globalt plan. Så kan det tidligere univers i hvert fald ikke have været ligesom vores, for den mulige masse af det sorte hul vil være stærkt begrænset af tidligere begivenhedshorisoneter som følge af denne udvidelse. Men hvis den udvidelse, vi ser, netop bare er lokal, så er der ikke det problem, og så kan sorte huller fra et døende univers vokse sig lige så store, som det skal være (og størrelsen vil så bare afhænge af den globale massetæthed og af tiden, det tager for et "sort hul" at eksplodere igen). 




%Noter:
% - Bryder man mon ikke L-invarians i gængs QFT (når man ophæver gauge-symmetrien på naiv vis)?
% - Man kan arbejde videre ved f.eks. at analysere, hvornår man kan lave dette gauge-eliminerings-trick og bevare (Lorentz-)symmetrien. Men på kortere sigt bør man også prøve at se på, om man ikke kan finde en måde at lave pertubationsregning på ved at bruge et cut-off. 

\subsection{Eksistensteori}

Indtil videre se bare den udkommenterede brainstorm-tekst i denne sektion, lige neden for denne tekst i kildefilen.

*Jeg kan også lige kopiere det ind her:\\
{\slshape
\%Brainstorm:\\
\%(11.05.21) Jeg tror faktisk, der er noget galt med CUH med endelige oplevelser kun, hvis man summerer i en standard rækkefølge med stigende sætningslængde undervejs, når gennemsnittet beregnes. For det virker til, at det mindst simpelt beskrevne univers så vil få en frekvens gående imod 100 \%, og hvis der så er et simplere univers end vores, hvor oplevelser som vores \emph{kan} lade sig gøre, men bare er vildt usandsynlige, så ville vi altså måtte konkludere, at al fortid bare har været rent tilfælde og at "illusionen" må opløses når som helst. Og selv hvis der bare findes et simplere univers med en 100 \% sandsynlighed, som \emph{ikke} kan inkludere vores oplevelser, så ville det stadig være underligt at være en del af et fåtal med virtuelt 0 \% sandsynlighed for at forekomme, når vi ikke kender multiversets "levealder." Men det gode er så, at jeg egentligt heller ikke ser nogen grund til, at oplevelserne bør være endelige, og hvis begge ting kan lade sig gøre, så må begge ting forekomme pr. eksistensens/multiversets symmetri. Der er så lige en hage mere, der nok er mere vigtig. Man kan sige det samme, om universer der indeholder flere sjæle end en, eller indeholder en sammensat, overordnet sjæl, om man vil (et 'sjælefelt,' som jeg lidt har kaldt det (metaforisk) et sted nedenfor..). Ja, og dette er jo pludseligt en langt større hage, for så kræver det faktisk, at det ikke giver mening med en sammensat sjæl i det store hele. Jeg må dog sige, at jeg alligevel er tiltrukket af denne mulighed, for jeg har netop også skiftet lidt holdning til det. Det viser, at jeg nok var lidt biased før og/eller at jeg er det nu, hvad jeg derfor bare vil antage, at jeg er, men når jeg tænker over det nu, selv når man tænker på hjernespaltning eller hjerne-sammensmeltning: hvis to hjerner kan gå sammen og lave en højere bevidsthed, enten over tid eller med det samme, så må der stadig være tale om en seperat ny oplevelse. Hm, nu hvor jeg skriver det, klinger det ikke ligeså overbevisende, men jeg føler altså lidt selv, at jeg godt kan finde argumenterne. ..Jeg tror nu altså mere på, at oplevelser er... Ja, jeg har jo allerede arbejdet med antagelsen om, at oplevelser \emph{er} roden af, hvad eksisterer, og hvis konceptet om en oplevelse dermed er noget helt fundamentalt, og at en 'oplevelse' altså betegner noget helt fundamentalt ved den samlede eksistens, så giver det også bare bedst mening, hvis hver oplevelse har ét subjekt. Jo, og nu klinger det pludseligt meget mere overbevisende, for hvordan kan det også være anderledes. Subjektet defineres vel ud fra oplevelsen, så hvis oplevelsen indeholder mange facetter, så må subjektet føle og opleve alle disse facetter. Ja, det virker altså fornuftigt nok. Så den samlede eksistens er altså eksistens af en mængde oplevelser, der hver især har et subjekt, som i hver vores tilfælde er os selv (så vi er hver især roden til hele universet under denne antagelse (så når man gør godt eller ondt mod en anden, så gør det stadig ligeså godt eller ondt på den anden, men teknisk set foregår denne glæde/smerte bare i et andet "univers," hvis man vil se det sådan)). Nå, og så kommer spørgsmålet til universet: hvilke af denne mængde af oplevelser vil eksistere, og hvornår --- og hvis tid ikke giver mening at spørge om, så i hvilken rækkefølge foregår de i (og foregår de eventuelt samtidigt (i tid eller i rækkefølgen) eller ej)? Og her må den samlede eksistens / det samlede multivers (af oplevelser) så svare, jamen ud af alle de mulige bud, som I kan finde på (og måske også nogle bud, der går over jeres hoveder), er der selvfølgelig en vis, ikke nødvendigvis proper, undermængde af disse bud, der giver mening ud fra den fundamentale logik om alt *[eller rettere i det "fundamentale sprog" så at sige] vil der også være en vis opdeling af alle disse muligheder *[ligesom der vil være i alle andre sprog]. Og da den samlede eksistens har perfekt symmetri relativt til den grundlæggende logik om alt, vil alle de forskellige muligheder, \emph{hvis} der er mere end én, hvad der dog meget vel kan være omvendt (at der er \emph{én} mulighed for, hvordan et del-multivers kunne udarte sig), forekomme som en slags mere eller mindre ligeværdige del-multiverser (alt efter om den grundlæggende logik har en rangering af disse mulige del-multiverser eller ej (og hvis ikke, vil de alle sammen bare forekomme lige meget)). Alle disse del-multiverser, hvis der er mere end ét, vil så give et svar på, hvilke oplevelser forekommer og "hvornår." Du, læser, er så en af disse oplevelser hørende til det samlede multivers, og hvis du skal regne dine egne prior-sandsynligheder ud for din fremtid, skal du altså så i princippet lave et muligvis vægtet gennemsnit af alle del-multiversernes gennemsnit af proir-sandsynligheder, hvilket hver især findes ud fra antagelsen om, at du som jeg-individ ikke kender tidspunktet siden starten af dit multivers, og dermed må summe over alle oplevelser (som er lig "universer" i denne begrebsverden) fra start og fremefter, og se hvordan prior-sandsynlighederne konvergerer (og hvis de ikke konvergerer, så se på hvordan de statistisk set svinger (så godt som dette nu kan lade sig gøre --- det er jo ikke fordi man nogensinde behøver at gøre alt dette i praksis; vi skal bare gerne kunne opnå, at prior-sandsynligheden for et univers a la det vi observerer (med dets tilsyneladende faste love osv.) ikke har sandsynlighed 0 i det samlede multivers)). 
\\\%Nå ja, og det skal nok lige nævnes, at når jeg skriver CUH, så tænker jeg egentligt ikke på Turing-maskiner; jeg tænker i stedet på endelige logiske sætninger fra hvad end dette fundamentale logiske "sprog" (hvad det nok ingen gang er; det er nok mere profound end det) konstituerer. 
\\\%Jeg kan også lige nævne her, at jeg har i sinde så at forklare lidt om, hvordan spørgsmålet om, hvordan en bevidst oplevelse kan koble på et fysisk system, nu bliver trivielt, for hvis oplevelser er det grundlæggende, der "eksisterer," så er fysikkens love altså beskrevet som en del af oplevelsens love. Så de to ting var simpelthen bare ikke seperate love til at starte med. Hvis man skulle finde alternative teorier til min, så tror jeg alligevel altid, at dette vil være løsningen på det problem på en eller anden måde, nemlig ved at antage at et univers' love ikke bare må inkludere de "fysiske love," men også må inkludere love for for, hvordan oplevelser kobler til de fysiske genstandes bevægelser. Med andre ord skal et univers' love altså både beskrive opførslen er det "fysiske" samt opførslen af sjælene (hvilket i min nuværende yndlingsteori i princippet bare er én sjæl, men hvor der så tilgengæld bare er en masse tilsvarende universer med lidt andre sjæle-love/-variable, således at deres sjæle godt kan koble sig fast på andre bevidstheder i stedet). Dette bringer mig i øvrigt til spørgsmålet om, hvad en "sjæl" (en "sjæl" er en fin nok metafor, men lidt uheldigt at det får en associere til spøgelser osv., for i virkeligheden vil "sjælen" så være ligeså pragmatisk (og lige så "logisk" beskrevet (hvor jeg lige misbruger ordet 'logisk' en anelse)) en størrelse som de "fysiske love" i universet) gør efter dets væsens / dens hjernes død? Tja, jeg tror i reglen ikke, at sjælen og dermed universet vil stoppe på baggrund af en hjernes død. Det kan sagtens være, at universer kan have et slutpunkt, men der er ingen speciel grund til, at dette skal afhænge af, at sjælen mister sin nuværende rod. Det kan man sagtens forestille sig, men man kan også sagtens forestille sig, at sjælen bare ligesom går videre til den næste hjerne, den kan slå rod i, hvorved "oplevelsen" fortsætter derfra. Der er nemlig (så vidt jeg kan se) ingen, der siger, at "oplevelser" ikke kan inkludere en slags bevidstløse overgange, hvorefter der vil være helt eller delvist *(helt i vores tilfælde) hukommelsestab imellem de to puntker. Og hvis man så befinder sig i et univers med denne form for reinkarnation, så er der forresten heller ingen der siger, at denne reinkarnation skal overholde temporaliteten ifølge universets "fysiske love;" man vil sagtens kunne reinkarneres i en helt anden tid, forud eller bagud. Men det gode er i øvrigt så, at alt dette behøver man ikke at spekulere over, efter min mening, for "jeg'et"/"subjektet" er kun en metafor i den store sammenhæng. Om det er din "sjæl," der reinkarneres som et nyt væsen, eller om dette ikke er tilfældet, men at det sker i et andet univers med en anden sjæl --- eller om der slet ikke sker reinkarnation overhovedet, men at der bare findes et nyt univers med et nyt væsen og en ny "sjæl," så er det underordnet. Det betyder ikke noget. Vi har ingen grund til at håbe bedre ting for vores "sjæl" frem for andres "sjæle" efter døden, eller når som helst, for "sjæle" vil, efter min mening, alligevel bare være gjort af det samme "stof" set i det samlede multivers. Man kunne således sagtens antage, at vi skifter sjæl for, lad os sige, hvert minut som eksempel, og dette vil stadig ikke være noget, vi behøvede at tænke på. For "sjælen" er bare nogle fysiske love for det individuelle univers, og er ligeså levende/død som resten er universets ("fysiske") love. "Sjælen" er altså bare nogle love, der kobler en oplevelse til et system af "fysiske" objekter. Vi, som subjekter, er ikke vores "sjæl," med den definition, jeg bruger her, men er selve "oplevelsen," som denne sjæl producerer *[hvilket måske netop også gør mit valg af "sjæl" som term for det lidt uheldigt; måske man skulle finde et bedre ord for det..]. Så som subjekter kan vi altså være ligeglad med, hvilke sjæle, der producerer vores oplevelse, og eksempelvis om der er flere sjæle om det (som i eksemplet, hvor det skifter hvert minut) eller ej. Og vi kan sågar også være fuldstændigt ligeglad med, om en oplevelse foregår på tværs af universer (hvis f.eks. mit subjektive univers ophører om et minut, men at der eksisterer et andet univers til gengæld hvor Mads Juul Damgaards oplevelse så til gengæld starter og forsætter efter samme tidspunkt), efter min mening. For det eneste, der betyder noget, er bare den samlede mængde oplevelser i det samlede univers; det er selve roden af hele eksistensen. Hvordan og hvornår disse forskellige oplevelser bliver "produceret" af det samlede univers er helt underordnet i min optik. *[Nej, det passer jo så ikke helt, for rækkefølgen (hvad end "rækkefølgen" præcist indebærer i den fundamentale logik for alt) betyder noget. Rækkefølgen afgør jo nemlig frekvensen, og frekvensen af oplevelser betyder alt. Hvor frekvent Mads Juul Damgaard er i multiverset kan f.eks. have betydning, hvis man bekymrer sig om sit eget (for mit vedkommende) subjekt og om reinkarnation osv., og generelt vil det også bare være betydende for, hvad kan siges at være det samlede forhold mellem godhed og ondskab / lykke og smerte i multiverset. Men hvis vi ser bort fra ændringer i frekvensen, ja, så kan man altså godt sige, at det er underordnet.]
\\\%Der er også et interessant spørgsmål om, hvordan en "sjæl" og dens hjerne producerer lige præcis de specifikke følelser, som vi har, og ikke nogle lidt andre; hvor vi dog ikke kan kende forskel, fordi vi alrdig har kendt andet. Et godt eksempel er spørgsmålet om, hvorvidt den følelses- og sanse-mæssige oplevelse af farven 'rød' er den samme for mig, som den er for dig, eller for en tredje. Det er et spørgsmål, der kan være interessant at fundere over, men ikke et spørgsmål, man nødvendigvis behøver at tænke over, når man teoretiserer om eksistens, for her kan man også bare hurtigt sige, at hvis det faktisk giver mening, at den samme hjerne i forskellige versioner af det samme univers (med de samme begivenheder) kan føre til forskellige subjektive, sansemæssige oplevelser (uden at hjernen dog vil reagere anderledes på denne forskel), jamen så må vores samlede eksistens pr. dens generelle symmetri jo bare indeholde alle de mulige varianter af disse sanse-oplevelser (eller vi kan tænke på de som varianter af sjæle-typer, hver med deres forskellige "sanseoplevelses-moduler") i et eller andet omfang (symmetrisk pr. den fundamentale logik).
\\\%Angående mit punkt om mit "besjælingsprincip," så handler det vel bare om, at der jo er ingen der siger, at "sjælen" skal koble meget nøjagtigt til de mikroskopiske processer; sjælen kan godt koble til mere abstrakte forhold omkring en hjerne. Vi mennesker har en evne til at kunne tillægge selv døde ting følelser (besjæling), som jo må siges at være en meget abtrakt proces. Selvom jeg personligt tror, at en livlig oplevelse kræver en kompliceret hjerne, så er der altså ingen der siger, at den faktiske "besjæling" af den fysiske hjerne, ikke også i princippet kunne fungere på en ret abstrakt måde.
\\\%Selvom vi nu bare har sagt i det ovenstående, at alle mulige del-multiverser bare må forekomme i en eller anden symmetrisk ligevægt, så kan det stadig være interessant at prøve at spekulere over, hvilke nogle del-multiverser, der giver mening. Dette vil pr. den måde, jeg bruger begrebet 'del-multivers' her, altså indebære at spørge sig selv om, hvilke nogle lovmæssigheder der kan give mening for mængden og for rækkefølgen/temporaliteten af oplevelser i et (del-)multivers. Et fint bud kunne så være, at der gælder, at alle oplevelser bare starter samtidigt, men med en naturlig orden efter, hvor kortfattet de kan beskrives i den fundamentale logik, og at der så er en global tid i multiverset lig den subjektive tid af disse samtidige oplevelser, og hvor en, ikke nødvendigvis proper, undermængde af disse oplevelser er uendelige (hvormed deres lovsætninger dermed ikke behøver at inkludere et sluttidspunkt eller en slut-begivenhed). *(Undermængden kan også godt være nul, for så længe uendelige oplevelser bare \emph{er} mulige i det sammensatte multivers, så vil prior-sandsynlighederne altså stadig konvergere til noget ikke-(fuldstændigt-)kaotisk..) Et andet fint bud kunne være, at oplevelserne kommer ved at den fundamentale logik simpelthen "regner sig selv ud," således at vi alle i bund og grund er besjælede udregninger af, hvad der er sandt og falsk omkring eksistensen som helhed. Og ordningen for dette bud kunne så være, at specifikke (altså korte) oplevelser er ordnet ud fra, hvor "nemme" de er at finde frem til, og hvor "hurtigt" man kan finde frem til de pågældende svar omkring dem, hvis vi metaforisk set ser den "fundamentale logik om alt" som en slags (perfekt) intelligens/computer, der udregner alt hvad der er sandt eller falsk, ikke på nogen som helst global tid, som denne "intelligens"/"computer" (som metaforer) lever på, men stadig med en vis rækkefølge, og hvor alt "tid" så kun er, hvad de herved producerede oplevelser opfatter.  Et tredje fint bud kunne være, at der \emph{er} ét eller flere perfekt eller til-sammen-perfekte væsen(er), eller rettere "rene intelligenser" %er en eller anden 
(perfekt pr. den fundamentale logik om alt), der så aktivt til sammen udtænker alle mulige oplevelser og herved også føler disse oplevelser. Rækkefølgen og mængden af oplevelserne herved kan så sagtens stadig være ligesom i det første eller det andet tilfælde, hvilket altså henholdsvis vil sige, at alle oplevelser ordnes efter deres beskrivelser i den grundlæggende logik (eller den version af det, som pågældende intelligenser forstår (da de kun er "perfekte" til sammen, og da "perfekt" kan indebære mange ting)), og hvor de eventuelt så bare udtænkes samtidigt, eller at alle oplevelser udtænkes i takt med, at intelligenserne "deducerer" dem. Jeg tror bare, jeg stopper her, men man kan sikkert finde på flere idéer til, hvad der kan ligge i kernen af et (del-)multivers, og man vil helst sikkert i øvrigt også kunne finde på andre fornuftige ordninger af alle oplevelserne, som sådanne del-multiverser kunne medfører (jeg har jo kun nævnt to ret forskellige, men der må næsten være flere imellem disse to, og flere generelt..). Bemærk i øvrigt, at en mulig version af den tredje mulighed nævnt her kunne være, at de til-sammen-perfekte væsener i dette del-multivers så bare simpelthen \emph{er} alle dette del-multivers' "sjæle" i bund og grund, som så enten passivt eller, hvad mere interessant er, helt eller delvist \emph{aktivt} så vælger, hvilken oplevelse, de vil gennemleve (og hvor disse intelligente og helt eller delvist aktive sjæle så dermed måske også kunne være udødelige og dermed bare blive ved med at vælge en ny oplevelse efter den anden). Jeg synes disse muligheder er meget interessante, og det vil bestemt også være interessant at finde på flere (samt at analysere dem alle yderligere). 
\\\%Nå ja, i øvrigt er der også den mulighed, at man har et (del-)multivers (eller selvfølgelig et samlet multivers; det kan vi jo ikke vide på stående fod) hvor alle oplevelser bare har en fast grundlæggende længde --- eller at de måske bare er ordnet efter subjektiv tidslig længde --- som noget fundamentalt, og at oplevelsernes ordning (udover den eventuelle ordning efter subjektiv længde i første omgang) så også bare er efter, hvor let de kan beskrives i det fundamentale "logiske sprog," som multiverset "beskrives" og/eller "beskriver sig selv" ud fra. Denne mulighed vil heller ikke, så vidt jeg kan se, løbe ind i paradokset, der medfører en 100 \% frekvens for det mest simple univers, hvilket der ellers kan være fare for i variantioner af CUH/MUH. (11.05.21)
\\\%Nå ja, og jeg bør også lige forklare det her med, at der ikke kan være en resolution/opløsning i, hvor tætte to forskellige oplevelser kan være, før at de tæller som den samme oplevelse og kun opleves én gang i det samlede multivers. Hvis dette var tilfældet, så ville vi jo bare enten have komplet kaos, eller også ville det kræve et meget endeligt multivers (altså endeligt i både tid, rum og antal universer). For hvis ikke multiverset så var så endeligt og begrænset (i en helt vild grad), ville alle mulige forskellige tænkelige oplevelser jo forekomme mindst én gang, og dermed vil de oplevelser, der følger fysiske og statistiske lovmæssigheder, altså ikke blive mere frekvente end de kaotiske. Hermed er det så altså ikke sagt, at der ikke kan være en endelig opløsning omkring, hvad en oplevelse er, det kan der sagtens være (jeg er endda tilbøjelig til at tro dette; det giver god mening for mig). Men det går bare ikke for teorien, at to identiske oplevelser fra to forskellige universer og/eller tider så kun tæller for én oplevelse i det samlede multivers / den samlede eksistens statistik. Åh, og jeg skal forresten også lige slå fast (hvilket på en måde er lidt relateret til det emne (om opløsning)), at jeg altså selv klart er tilbøjelig til at tro på, at den fundamentale logik for den samlede eksistens ikke indeholder uendelige sætninger; dette giver ikke rigtigt mening for mig. En logisk sætning vil altid indeholde en endelig (men ikke begrænset; sproget er ikke begrænset i sætningslængderne) mængde information. Så selv om jeg tror på (er jeg begyndt (jeg var nemlig skeptisk over for dette tidligere)), at "oplevelser" (så i min teori altså et univers' samlede (enlige) oplevelse) godt kan være uendelige, så tror jeg stadig på, at de altid skal være beskrevet med endeligt meget information (dog ikke begrænset). Jeg tror dog ikke, at uendelige sætninger med uendeligt meget information i vil ændre det statistiske billede af teorien.. tja, eller det ved jeg ikke.. Men uanset hvad, synes jeg altså bare ikke rigtigt, det giver mening at forestille sig. Så jeg er altså muligvis lidt en eksistentialist (metematisk) *[Hm, mon ikke jeg tænkte `konstruktivist' her\ldots?], hvis jeg altså forstår termet korrekt.. Med andre ord tror jeg ikke på, at universer kan følge teorier, der ikke er konstruerbare (hvad end det kommer til beskrivelsen af de fysiske ganstande og dets dynamikker, eller når det kommer til sjæle og sansninger). Og når det kommer til matematikken, er jeg i øvrigt af samme grund ikke interesseret i matematiske spørgsmål, der kun er interessante (/spændende/underlige at begribe), hvis man prøver at forstå dem i ikke-konstruerbare teorier. Ikke-konstruerbare teorier er nemlig for mig ligesom bare noget mennsket har opfundet; det samlede multivers består af universer, der alle følger konstruerbare lovmæssigheder (mener jeg). (Det var i øvrigt pga., at jeg opdagede, hvad konstruerbare teorier (eller 'konstruerbare universer,' men det bliver forvirrende her) var, at jeg ligesom var i stand til at tænke på eksistensspørgsmålet i et nyt lys (da jeg endelig kom til at tænke på dette emne efterfølgende).) (Uh, og lige for at præcisere igen, når jeg derfor taler om CUH, så tænker jeg altså selv --- for jeg er nemlig ikke 100 \% sikker på, hvad det konventionelle term indebære; om det indebærer det samme, som jeg antager, når jeg bruger det --- på MUH, men bare hvor det understreges, at der ikke bruges nogen strukturer til at beskrive universerne, der ikke tilhører en mængde af strukturer har et kontruerbart/formulerbart map til/fra \textbackslash omega, altså til/fra de naturlige tal. Så universerne, som jeg bruger termet, CUH, er altså med andre ord stadig beskrevet ved matematiske sætninger, men kun af eksakt formulerbare sætninger, og dermed altså uden nogen forekomster af ikke-konstruerbare strukturer.) (11.05.21)

\%(13.05.21) Hm, lige en interessant tanke lidt angående besjælingsprincip: Man kunne også tænke sig, at bevidstheden/"sjælen" faktisk kobler sig, ikke bare på den faktiske dynamik af et system, men også på selve lovene bag.. Tanken er så, at bevidsthed med denne antagelse ikke vil koble på tilfældige bevægelser, der bare pr. et lykketræf kommer til at opføre sig som en hjerne, men at bevægelserne skal komme fra en naturkraft, der får bevægelserne til at ske (fordi bevidstheden altså så kobler mere på denne naturkraft, og altså udgør en slags "besjæling" af denne kraft). Bare lige en interessant tanke.. I øvrigt kunne det med denne antagelse muligvis igen blive en mulighed, at vi \emph{er} en del af den mest simple teori, som bevidsthed kan fremkomme af (som altså som nævnt (er jeg lidt kommet frem til) vil have en 100 \% frekvens med visse multivers-antagelser).
}\\

(06.06.21) Jeg kom i tanke om (i går tror jeg), at der jo egentligt godt kan være en forskel på den resulterende konklusion på korrekt etik, hvis man mener at ``sjælen'' (dvs.\ det oplevende modul i den grundlæggende logiske struktur af det fulde univers (inklusiv bevidsthed også, og altså ikke bare det ``fysiske univers'')) med det samme hopper videre til en ny hjerne efter dennes død, og hvis man så samtidigt mener at ``sjælen'' lever evigt, eller at dens levetid ikke afhænger af dødstidspunkter men kun af subjektiv tid. For i så fald kan det jo på en måde betale sig, at ende ens liv tidligt, hvis det går hen og bliver kedeligt, og i det hele taget kan det betale sig at leve mere risikabelt (fordi man så i praksis kan leve som om, at man ``respawner'' (hvilket man jo nok gør uanset hvad, men spørgsmålet er, om det ikke kan betale sig, hvis vi bliver i spilmetaforen (som om man bevarede hukommelse imellem livene), altid at spille et spil helt ud og få så meget ud af det som muligt)). Hm, hvad tænker jeg om dette?\ldots\ Tja, det gør nok egentligt ikke så meget. Man finder nok ikke ud af, hvad svaret er, og derfor må man jo bare prøve at finde en god middelvej imellem de to hypoteser --- særligt også fordi, hvis begge ting \emph{kan} lade så gøre, så \emph{vil} begge ting forekomme pr.\ multiversets ultimative symmetri, og så vil vi ikke selv kunne sige, hvilken type (fulde) univers vi lever i. Så konklusionen bliver nok lidt bare, at man bør undgå lange kedelige liv (hvilket særligt kan blive relevant, når vi udvikler livsforlængende teknologier), og selvfølgelig bør man generelt sørge for at mindske kedsomhed, men man bør dog ikke ligefrem undertrykke mennesker, der ikke lever i konstant ekstase (eller deromkring *[Lige for at understrege det, så var dette et hyperbolsk billede, så når jeg forklarer sprogbilledet ved at skrive ``eller deromkring,'' så burde jeg egentligt skrive `eller meget mindre,' for det er sådan sætningen skal forstås (bare så fremtidige læsere, der ikke er bekendt med de nuværende sprogtendenser også er med).]). Jeg burde næsten tilføje en reference til dette i min etik-sektion, men\ldots Hm, jo det gør jeg faktisk lige for en god ordens skyld. 



(15.07.21) Hm, måske skal jeg lige overveje, om ikke det ville give god mening, hvis der kun er én (uendelig) bevidsthed\ldots (Jeg kan ikke lige huske, hvor meget denne mulighed indgik i min ovenstående brainstorm, men føler altså, det er værd at overveje noget mere\ldots)

(30.07.21) Uh, og har jeg nævnt, at besjæling jo kan bruge fysikkens love, og at en mulig løsning på 100 \% kaos-paradokset, hvis det skulle se ud til ellers at være uundgåeligt, altid kan være, at de mest simple former for sjæle-objekt-koblings-love i universerne med væsentlig margin er sjæle-koblings-love, der besjæler hjerner via de fysiske love. Sådanne sjæle-koblings-love kan jo tage udgangspunkt i naturens kræfter så at sige, når oplevelserne skal produceres, så man på en måde oversætter de fysiske kræfter, der får en hjerne til at køre rundt, til vilje og drivkræfter for den selv-bevidste oplevelse. Hm, og man kunne faktisk argumentere for, at dette ville være en naturlig løsning\ldots\ Det er lidt svært at forklare mere præcist, og det er i øvrigt også en lidt stor mundfuld, hvis man virkeligt skal filosofere/teoretisere over sjæle-objekt(i.e. hjerne)-koblings-lovene. Men man kunne altså sikkert godt gøre et argument for, at oplevelser beskrives mest simpelt (når et univers' samlede love skal opskrives), hvis man tager udgangspunkt, ikke i selve bevægelserne, men i kræfterne, der får hjerne-elementer til at bevæge sig rundt, når man skal koble en bevidst oplevelse til dette. Og herved får vi så med det samme skåret alle de kaotiske universer fra, der ellers måske kunne give anledning til 100 \% kaos-paradokset, for så skal de fysiske love nemlig være styrende for en hjernes (indre) bevægelser, før den kan holde en bevidsthed i sig. Spændende nok. \ldots\ Ah jo, det har jeg lige netop været inde på før. Det var lige præcis det, jeg skrev om i den sidste paragraf af mine ovenstående brainstorm-noter. 

(31.07.21) I går aftes kom jeg frem til noget rigtigt interessant. Konceptet om, hvad en `oplevelse' er, indeholder jo allerede konceptet om en subjektiv tidslig længde på oplevelsen, for hvad er en oplevelse uden en subjektiv tidslig fornemmelse? \ldots I hvert fald altså når det kommer til, hvad vi forstår ved en oplevelse; det kan jo i princippet godt være, at multiverset indeholder helt anderledes, eksotiske former for bevidste oplevelser. Men hvis der er, må det dele multiverset i flere mere eller mindre ligeværdige dele, hvor oplevelses-subjekterne hver især kan filosofere uafhængigt af hinanden om, hvad deres del-multivers består af. Med andre ord behøver vi (selvsagt) aldrig at bekymre os om spørgsmålet: ``Hvorfor er vi væsner af lige netop vores fundamentale oplevelsesform og ikke en eller anden anderledes form?'' Vi kan selvfølgelig overveje, om der mon kan eksistere andre, fundamentalt anderledes oplevelsesformer, men det kan aldrig komme til at være betydende for spørgsmål omkring vores egen univers-prior-sandsynlighed. Nå det var lidt et sidespring, og jeg er faktisk selv nok tilbøjelig til at mene, at en oplevelse altid vil have et koncept om `tid' i sig, men det er nu ikke noget, jeg vil tænke over her. Men uanset hvad, vil jeg altså påstå, at der i vores fundamentale koncept omkring, hvad en oplevelse af vores form/type indebærer, ligger en subjektiv tid som noget indbygget i konceptet. Og dermed vil spørgsmålet omkring, hvor lang tid en oplevelse varer altså ligge meget tidligere i den fundamentale logik om alt, end når vi når til, hvilke præcise oplevelser af disse former/typer, der så eksisterer. Der må med andre ord gælde, at eksistensen i sin helhel først skal ``besvare,'' hvad en oplevelse indebærer, og hvor mange forskellige former for oplevelser der kan være i den samlede eksistens, før at ``den'' kan begynde at ``besvare'' (håber det er ok, at jeg personificerer `den samlede eksistens') hvilke specifikke oplevelser, der så eksisterer af hver oplevelsesform. Og hvis dette rigtigt nok gælder, så må valget af oplevelseslængde dermed naturligvis ligge tidligere end spørgsmålet om lovene, der beskriver oplevelsens udformning, når vi kigger på frekvenserne/sandsynlighederne af forskellige oplevelser. Vi ved så ikke som subjekter selv, hvad den fundamentale længde af en oplevelse indebærer. Måske kan den være ganske kort og danne en nedre grænse for, hvad man kan opfatte som separate oplevelser i tid --- hvilket så i øvrigt indebærer at hele vores tilværelse teknisk set sker på tværs af universer, hvis vi beholder min tidligere nomenklatur og står fast ved, at et `univers' betegner en eksistens af en helt specifik `oplevelse.' Da jeg ser subjektet som uafhængigt af universer (og uafhængigt af `sjæle' ift.\ den nomenklatur, jeg er endt på, hvilken jeg dog også burde prøve at ændre lidt), så er dette ikke noget problem for mig, nemlig at hver lille atomare ``frame'' i vores subjektive oplevelse foregår for sig i princippet i hvert sit univers set med den fundamentale logisk øjne. Men andre muligheder er så også, at oplevelser har en længere længde, så at en `oplevelse' dermed kan indeholde adskillige separate sindstilstande, hvilket også giver god mening for mig i øvrigt *(selvom det dog ikke er det, jeg hælder mest til, hvilket jeg vil begrunde her nedenfor om lidt), og en særlig mulighed inden for denne mængde er så, at oplevelser er uendelige (hvorved ``sjælen'' så bare vil skifte hjerne efter vores død, muligvis efter en vis limbo-periode af varierende længde (alt andet end lige) bestående af mere eller mindre neutrale sindstilstande (i.e. måske med nogle mikroskopiske fluktuationer, men jeg tror vi snakker under insekt-intelligens-niveau, hvis jeg skal give min umiddelbare mening), inden sjælen igen finder frem til en hjerne at koble til --- og her behøver man så ikke rigtigt at bekymre sig om spørgsmålet, ``hvad hvis nu min ``sjæl'' ikke har den evne?'' for efter basalt set ``ingen tid'' set i forhold til multiversets uendelighed vil kun de sjæle, der har evne til at koble sig på en ny hjerne (evt.\ ved at gå tilbage i tiden; der er ingen grund til, at dette ikke skulle være en mulighed) overleve). Og hvis multiverset så indeholder omtalte oplevelsesformer af varierende længde, og at alle subjekter således ikke bare er en række atomare sindstilstande, der på en måde ikke egentligt hænger sammen, men hvor subjektet så alligevel vil have opfattelsen af, at de hænger sammen, og for mig at se er der ingen grund til, at multiverset ikke skulle indeholde længerevarende oplevelser (også fordi jeg lidt er landet på, at jeg har det bedst med tanken om en oplevelse som noget dynamisk, der går fra tilstand til tilstand, og ikke som selve tilstandende --- det giver således ikke super meget mening for mig, at oplevelser kan være frosset i tid (for som jeg ser det: hvis man fryser alle atomer i en hjerne fast i rummet, så vil der da ikke være nogen bevidst oplevelse hos denne hjerne, imens den er frosset?)), så\ldots (*Jeg fik ikke færdiggjort denne sætning, men se næste sætning efter den følgende.) Så jeg hælder altså til de længerevarende oplevelsesformer, der ligesom går imellem sindstilstande (da jeg altså selv hælder til at mene, at en `oplevelse' er noget, der i sin natur går fra tilstand til tilstand). Og hvis der altså eksisterer længerevarende oplevelsesformer, jamen så vil vi som subjekter med al sandsynlighed kunne regne med, at vi er en del af enten en uendelig eller en nærmest uendelig (set med vores øjne) oplevelse. Ikke at dette egentligt betyder så meget i det store hele, for hvis vi er en del af en meget kort oplevelse, jamen hvad er så problemet ved det? Så er det bare sådan verden grundlæggende fungerer, men de betyder ikke noget for os som subjekter, at vi er gjort af en række selvstændige oplevelser frem for et længerevarende hele (i hvert fald ikke efter min mening, men jeg vil nok ikke være den eneste). Og uanset hvad, så har vi nu heller ikke nogen fare (som jeg kan se det) for 100 \% kaos-paradokset (hvori al fortidig systematik i tilværelsen med al sandsynlighed bare vil have været en enorm tilfældighed og altså vil opløses hver øjeblik, det skal være (hvilket jo ikke kan være rigtigt, for så vil det jo altid være langt mere sandsynligt, at vi som mennesker bare har overset noget i vores logiske udredning)). For hermed har hvert ``univers'' jo ét selv-bevidst subjekt i sig, og oplevelses-længden er allerede valgt, så at sige, før vi begynder at skulle opstille, hvilke mulige oplevelser der kan være af disse former. Så når vi herefter begynder at opstille disse på en række for at prøve at besvare, hvad sandsynlighederne/frekvenserne vil være i den (og dermed prøver at sige noget om vores egen univers-prior(-sandsynlighed)), så vil det altså ikke være naturligt at inkludere længden på oplevelsen i dette spørgsmål. Og dermed kan der altså ikke komme nogen kaos-paradokser ved, at muligheden for konstruktioner af store tal vil fremhæve de absolut mest simple mulige univers-love og gøre deres frekvens gående mod 100 \%. I forvejen kan man måske nok alligevel argumentere for, at så længe der bare findes uendelige oplevelser, så kan der findes en løsning på 100 \% kaos-paradokset, og måske kan man endda finde argumenter uden dette. Men jeg er ikke sikker på, at man ligefrem kan sige noget endeligt, der maner det helt til jorden --- det vil i hvert fald blive en længere diskussion, hvor man sikkert vil blive ved med at finde argumenter og modargumenter (tror jeg umiddelbart)\ldots\ Men med dette dejlige(!) (og virkeligt fornuftige, det synes jeg virkelig!) argument, om at det faktisk giver mest mening, at faktorisere tidslængden ud for sig, når de mulige eksistenser skal opstilles, så giver det mere eller mindre sig selv. Og lige for at understrege det, der kommer efter, når man så videre skal finde frem til (eller rettere lave et overslag på) univers-prioren, så er jeg altså overbevist om, at det vil konvergere til det samme uanset hvordan, man vælger at ordne de fysiske --- og de sjælelige --- love for de mulige universer. Virkelig fed indsigt! :) Og med denne bliver svaret på, hvad der eksisterer nemlig bare, ``alt hvad der kan eksistere.'' Og uanset hvilket sprog man så opstiller lovene for de eksisterende universer i (hvis man går med til at de oplevelsesformer bare opstilles først, og især hvis man altså så også vil gå med til at konceptet om ``tid'' absolut kun hører med under konceptet om oplevelsesformen, og altså er defineret af denne), så mener jeg altså, at svaret om, hvad der eksisterer og med hvor stor frekvens, vil konvergere til det samme (og altså \emph{være} det samme). Og dermed kan vi altså tale om ``alt hvad der kan eksistere,'' uden at behøve at bekymre os om, hvad det `grundlæggende sprog for alt' er, da det netop bare kan være hvad som helst, og dermed også kan være noget så fint som `samlingen af alle mulige sprog.' Så vi behøver med andre ord (umiddelbart, og hvis jeg altså har ret) ikke at være bange for, at vi har en bias som subjekter, når vi tænker på ``alt hvad der kan eksistere,'' for svaret vil blive det samme uanset, hvilken bias man kommer fra --- i hvert fald når vi altså nøjes med at se på den del af multiverset, som vi kan forstå med vores menneskelige logik (og så bare lader resten være). Og i øvrigt synes jeg også bare, at dette billeder opklarer spørgsmålet om, ikke bare ``hvad,'' men også ``hvornår'' ting sker i den samlede eksistens. Svaret er her bare, tid er kun subjektiv; oplevelser eksisterer uafhængigt af hinanden og `tid' giver basalt set kun mening at snakke om imellem sindstilstande af den samme oplevelse. Og idet at oplevelser kan være uendelige (det hælder jeg i hvert fald til, er sandt), jamen så vil multiverset altså aldrig stoppe, selv ikke hvis man ser på fortolkningsbilledet, hvor alle oplevelser bliver afspillet parallelt og sat i gang samtidigt, og hvor visse oplevelser muligvis så ophører efter en vis tid, for selv i dette billede, vil der altid være oplevelser, der bliver ved med at køre. Og som nævnt vil det i øvrigt så som subjekt være mest fornuftigt at antage, at man er en del af en uendelig (eller måske nærmest-uendelig) oplevelse (for hvorfor skulle nutiden lige netop være så tæt på multiversets start (hvilket vil sige oplevelsens start, for tid er jo subjektiv)?). \ldots Ah, og jeg kan i øvrigt også lige nævne, at tanken om loopende oplevelser også giver god mening, så sjælen så at sige bare starter forfra efter en på forhånd givet tidslængde. Dette har jeg indtil videre bare talt med som en `uendelig oplevelse,' men der er alligevel en markant forskel på universer, som looper fordi de er uendelige, men at entropien i dem ikke er, og så universer, der looper fordi det bare hører sig med til den fundamentale oplevelsesform, at oplevelsen looper, efter dens tidslængde er forløbet. Nice nice nice. Så denne paragraf indeholder altså nu min yndlingstilgang til det, og indeholder altså med andre ord min nuværende tro. 


(18.08.21) På min ferie nu her (så omkring for to uger siden) kom jeg til at tænke på, at vores kvanteverden, som vi jo lever, jo rent faktisk kræver sjælelove! Der bliver nødt til at være nogle lovmæssigheder for, hvordan sjælen kobler til kvantebølger, måske medmindre man tror på, at der er en mekanisme, der løbende får bølgerne til at kollapse, og hvis man så altså er materialist, der tror på at bevidsthed er en naturlig konsekvens af (intra-)hjerne-bevægelser\ldots\ Hm ja, men helt klart stadig værd at påpege dette, især fordi jeg faktisk tror, at folk vil være mere tilbøjelige til at tro på sjælelov-drevet ``kollaps'' frem for spontant kollaps, hvor sjælen så halter efter og følger en klassisk sti i stedet for at følge de kvantemekaniske bølger direkte, da dette jo strider lidt mod antagelsen om, at sjæl/bevidsthed er en naturlig konsekvens af hjernebevægelser\ldots\ okay, jeg er lidt på dybt vand nu, men uanset hvad, er det korte af det lange også bare: Det er værd at nævne. For jeg mener nemlig selv, at `sjæle-love' faktisk er et ret simpelt og elegant svar på spørgsmål omkring de kvantemekaniske fortolkninger. Og eftersom mange mennesker tager many-world hypothesis seriøst, er der i det hele taget også bare brug for, at nogen lige påpeger problemerne med denne fortolkning (og det er der helt sikkert folk, der har gjort, men åbenbart ikke tydeligt nok). (Disse problemer er i øvrigt, at many-world-hypotesen medfører ren kaos, medmindre man opstiller love for, hvordan sjæle deler sig, hvilket medfører, at hypotesen kræver mange flere ekstra-love, der ikke har noget med den rene kvantemekanik at gøre. Så selvom hypotesen umiddelbart kan se ren ud, så ender den faktisk med at være den mest beskidte, hvis først man dykker ned i det. Denne beskidthed er faktisk ikke som sådan et problem i sig selv, og many-world-hypotesen kan også være sand efter min mening (dog kun med en lille sandsynlighed som jeg umiddelbart ser tingene), men uanset hvad så nytter det bare ikke noget at se many-world-hypotesen som en måde at undgå de urene ting som bølgekollaps på, for man ender altså bare med mange flere ``urene'' ting. Nå ja, og jeg kan også lige nævne, at nogle af disse urene ting (og hvorfor jeg siger at det vil ``medføre kaos'') kommer af, at man så skal til at beslutte, hvordan sjælen deler sig, når der ikke er 50/50 \%, men at der måske i stedet er en sandsynlighedsopdeling på noget a la $1/\sqrt{5}$ og $(\sqrt{5}-1)/\sqrt{5}$ osv. (Dette bør der dog helt klart være andre, der har påpeget\ldots)) 

Uh, det er forresten da også en vildt sej vinkel på min teori, at se den ift.\ Schrödingers kat\ldots! Her vil der jo normalt være et paradoks, medmindre man antager en form for many.world-hypotese (hvilket jo er fint; many-world er som sagt bare ikke mere matematisk ren end andre fortolkninger (overhovedet), men den er bestemt stadig en mulighed --- og efter min tro vil der altså være many-world-universer som en del af det samlede multivers (for hvis noget giver mening, så eksisterer det også et sted)). Men jeg kan så bidrage med min eksistens-fortolkning (eller tro rettere) og sige, at ``der er et univers, hvor katten er død\ldots'' Hm, man skulle måske lige ændre tanke-eksperimentet til, at katten får en (lille) spand måling helt ud over sig. Så ville jeg sige: ``Der er et univers, hvor katten er ren, og et univers, hvor den er målet blå, men i et univers, hvor en iagttager uden for kassen har en bevidst, der kan stille sig selv spørgsmålet, om katten er død eller levende, da vil katten bare være en kvantebølge uden bevidsthed, for der er kun én bevidst iagttager pr.\ univers. Jeg synes sådan set ikke, at dette er et vildt indsigtsgivende eksempel på min teori, men fordi den bringer en ny mulighed til et alment kendt paradoks, så kunne det måske være en god måde, at vække folks interesse på. :)

%(04.09.21) Jeg skal forresten lige genoverveje, hvorfor man ikke bare beskriver oplevelser ud fra et rum af, hvad kan lade sig gøre; hvorfor skal oplevelser beskrives ud fra hjerner i et eller andet fysisk univers?
%(06.09.21) Det skal de heller ikke. ... Nej, min 31/7-sektion ovenfor holder fint vand. ..Tja, bortset fra, at det med at "besvare spørgsmålet om tidslængden før oplevelseslovene," nok ingen gang er nødvendigt.. (!) Hm..  Nej, umiddelbart virker det endnu mere simpelt, end jeg har gjort det til, men jeg tror lige, jeg vil vente med at tænke videre over det og lige arbejde videre på noget andet først. (Jeg tænkte btw over dette emne i går, og kom frem til, at det hele var ret simpelt, men nu skal jeg altså lige opsummere de muligheder, jeg kom frem til, samt tænke over, om "besvare spørgsmålet om tidslængden før oplevelseslovene"-tanken stadig er nødvendig.)
%(07.09.21) Update: Ja, hvis man tænker i subjektiv tid, og hvis man holder sig til én oplevelse pr. "univers," så tror jeg altid man vil undgå 100 \%-kaos.. (Og der kan så muligvis blive en statistisk forskel, hvis man siger at oplevelser altid looper eller ej (og også alt efter, om de så kun gør det nogen gange eller slet ikke)..) ..Og jeg er blevet tilhænger af kun at regne i subjektiv tid, og at der kun er én oplevelse i et univers (en 'oplevelse' \emph{er} 'universet') --- og i øvrigt også, at der kan findes uendelige oplevelser (..hvilket måske også er krævet for ikke at få 100 \% kaos i visse tilfælde..?..), og videre er jeg også tilhænger af, at alle oplevelser er uendelige (men er ikke ligefrem stålsat på sidstnævnte (og det gør heller ikke nogen forskel med mine andre antagelser)). Så her er bare en rigtig dejlig, lækker teori, og så kan man jo bare sige, at hvis der også findes alternativer, der er ligeså gode --- eller bedre --- jamen så er det jo bare skønt; jeg er bare glad for at have en god teori, som føles som en virkeligt god kandidat. Og selv hvis den så ikke er rigtig, jamen så er jeg klart overbevist om, at noget der praktisk talt giver det samme resultat er sandt i stedet. :) 

%(13.09.21) Hold da op, sikke en dag i dag endte med at blive.! Jeg kom lige i tanke om, at der også er et anden rimelig godt alternativ til min yndlings-hypotese --- selvom jeg nu nok alligevel hælder mere til CUH-universer bestående af uendelige oplevelser (og hvor altså selve funktionen af universernes love hver især er at beskrive lovmæssighederne omkring en 'oplevelse'). Men dette alternativ er, at oplevelser er små og atomare alligevel (og sker som en slags frames, om man vil), og at... hm, én hypotese kunne faktisk være, at de bliver afspillet én ad gangen, men ellers tænkte jeg, at de alle bliver.. "afspillet" samtidigt, og så bare looper hver især. For nu hvor jeg tænker tid mere som noget udelukkende subjektivt (eller rettere at jeg er begyndt at hælde til at tænke tid i de baner, og altså at dette er blevet lidt mere naturligt for mig..), så giver dette billede også mere mening, for så kan man nemlig bedre forestille sig, at de "looper," i.e. at de "venter" på sig selv om at blive "afspillet" færdig, før de starter forfra igen. Så ja, dette hører bestemt til en vigtig hypotese/teori, der er værd at nævne.
%(15.09.21) Hov nej, så får man jo problemer med 100 \% kaos igen (hvis man antager alle mulige små, loopende oplevelses-frames)..
%(15.09.21) Hov!... Hm, er der faktisk ikke også en fare for kaos i min (nuværende) yndlingsteori? For mængden af information til at beskrive ens oplevelse vil vel gennemsnitligt set være praktisk talt uendelig, og bliver dette ikke på en eller anden måde et problem --- om ikke andet virker det da til at være noget rod, man bør prøve at løse (/ finde hoved og hale i)...(!) ... Det kommer selvfølgelig an på, hvordan univers-lovene ligesom er "opbygget" (eller rettere hvordan man kan se dem som værende opbygget), gør det ikke?.. For hvis man bare har et logisk sprog og ordner sætningerne leksikalt, så kan man jo hurtigt nå til punkter, hvor oplevelsen er komplet beskrevet, og hvad gør man så med resten..?.. Åh, giver dette overhovedet mening, altså? He, pludselig virker det meget mere rodet..! ... Hm, det giver vel mening sådan på en eller anden led, men det var bare virkeligt dejligt før, da opbygningen af dette sprog ikke så ud til at betyde noget (hvad jeg stadig håber, vil vise sig for 1-subjekt-pr.-univers-teorierne..)... 
%(16.09.21) Okay, pyha, hypotesen hvor universer beskriver ét subjekts oplevelse, og hvor der bare bogstaveligt talt er alle mulige universer af denne slags, og altså dermed alle mulige oplevelser, kan vist godt holde stadigvæk. Pointen er, at for lange sætninger, så skal der fra start lægges op til i sætningen, at der kan komme mere information senere. Og sådanne oplæg kan så meget vel være, at oplevelsen med jævne mellemrum springer imellem forskellige oplevelser, og at resten af sætningen dermed kan gå på at liste forskellige oplevelser --- hvilket så nærmest svarer til at have universer i ``konjunktion'' med hinanden (hvilket nok er det typiske argument for, at MUH/CUH kan opføre sig pænt og tillade en høj frekvens lav-informations-universer ligesom vores (tilsyneladende er)), men bare hvor disse univeser så ikke ``afspilles'' parallelt med hinanden, men ``afspilles'' mere som, når en enkelt processorkerne kører flere programmer på én gang (hvor ``processorkernen'' så altså er den centrale ``sjæl'' for universet). Og dermed vil lav-information --- som btw er forskellig fra minimal-information, hvilket jo også må være et kaotisk univers, som jeg ser det --- så få en vis anseelig frekvens selv i universer med praktisk taget uendeligt information (i gennemsnit). Hm, jeg kom lige til at tænke på også, at hvis der er en mekanisme, der føder universer løbende, så har man slet ikke det problem, men jeg er nu ikke særligt glad for den hypotese. Men ja, så selv hvis al information i univers-sætningerne skal indgå som en betydende del af beskrivelsen af den iboende oplevelse, og at man altså ikke bare må have junk-information, så vil lav-informations-universer stadig få en anseelig frekvens (hvis man summerer tingene op på en naturlig måde (i.e. uden besvidste konstruktioner for at opnå det modsatte)). Så denne hypotese holder altså stadigvæk.:) En anden hypotese, som jeg kom til at tænke på i går (lige inden jeg fik den netop beskrevne indsigt), er at universer godt kan inkludere flere ``sjæle,'' men at alle universer hver især er begrænsede i udgangspunktet med sin egen endelige ``regnekraft'' (som altså dog godt kan variere fra univers til univers). På en eller anden måde har jeg det ret godt med denne tanke om en begrænset ``regnekraft'' ..hvilket nok forresten egentligt er meget naturligt.. så hvis ikke det var fordi, at jeg ret godt kan lide tanken om én sjæl pr. univers (fordi jeg synes at det på en måde begrunder eksistens; at eksistens \emph{er} oplevelse, og at det samlede multivers derfor bare består af ``alle mulige oplevelser'' (bogstaveligt talt)..). (..Ja, min begrundelse er nærmest, at man kan vende lighedstegnet om, så man både kan sige.. 'al mulig eksistens' = 'alle mulige oplever'.. og.. hm nej, det er vel mere at fjerne 'mulig'/'mulige' fra sætningen.. ja.. Tja, det er egentligt også lige meget, for pointen er bare, at det føles rigtigt for mig, men jo, det kommer sig jo rigtigt nok af, at jeg godt kan lide at antage 'al eksistens' = 'al mulig eksistens'.. og ja, når man så ender på 'al eksistens' = 'alle mulige oplevelser' (med antagelsen om at 'al mulig eksistens' = 'alle mulige oplever'), så føles det bare rigtigt rart, for så har man ikke nogen brug for en "global tid" i universet eller nogen "proces"/"mekanisme" til at føde universer og så videre, for alt dette følger med i konceptet omkring, hvad en 'oplevelse' er på et grundlæggende plan i ("logikken" bag) den samlede eksistens..) 

(16.09.21) Jeg har nogle yderligere kommentarer til omkring eksistens, men jeg vil lige tænke noget mere over dem, før jeg skriver dem ind her. Lige nu står disse tanker bare kort beskrevet ude i kommentarerne i kildedokumentet (i.e.\ i form af \LaTeX-kommentarer). 


(19.10.21) Mine kommentar-noter der fra bl.a.\ d.\ 16/9, som jeg skrev, at jeg ville skrive ind her i den renderede tekst, handler bare om at overveje: Hvad sker der, hvis multiverset består af universer beskrevet med endeligt meget information, og hvis vi kan se det som og regne på dette (hvis vi f.eks.\ skulle bestemme prior-sandsynligheder for vores eget univers) som om, de er tekststrenge af et formelt (matematisk) sprog (og hvor vi altså kan pille alle ikke-wffs (well-formed formulas) fra); hvis alle sætninger, der ikke beskriver et meningsfuldt univers også bare pilles fra, hvad står man så tilbage med? Og nærmere bestemt var min frygt så, at hvis vi så begynder at summe dem sammen og se på gennemsnit, vil prior-sandsynligheder konvergere til noget, der er fuldstændigt kaotisk, fordi hovedparten af universer så må siges at bestå af \emph{meget} lange (uendelige stort set) beskrivelser? For hvis man nu forestillede sig et sprog, hvor man hele tiden tilføjer ting ``Uh, og forresten: Hvis det og det sker, så gælder det og det og så sker det og det'', så må der også komme praktisk taget uendeligt mange af sådan nogle indskydelser til universet. Det vil altså komme praktisk taget uendeligt mange bonusregler til det gennemsnitlige univers, hvilket så meget vel kunne føre til uendeligt kaos. Men dette scenarie antager, at man kan komme med lappende regler efter at man har formuleret grundreglerne, og det kan man jo typisk ikke, når ting formuleres formelt. I vores menneskesprog kan vi godt sige ting som: ``Der er i øvrigt én undtagelse til denne regel,'' eller noget i den stil, når vi forklare noget, hvorved vi så basalt set tillader os at sige noget usandt, men hvor vi så bare retter det bagefter. Vi kan f.eks.\ tillade os at sige: ``partikler bevæger sig ifølge Newtons love,'' og så bagefter sige: ``hvis en troldmand siger `wingardium leviosa' og vifter sådan og sådan med en tryllestav, så kan partikler stoppe med at følge Newtons love og begynde at svæve i stedet.'' Men ja, når ting skal formuleres formelt, så bliver man altid nødt til at forudsige, hvor der kommer tilføjelser, så man kan åbne sætningerne op for sådanne senere modifikationer. Jeg synes personligt, det er mere fornuftigt at antage at et billede, hvor universets sætninger kan ses som at være opbygget af formelle sætningsstrenge frem for at antage et billede, hvor universets sætninger ikke er formelle, men bare skal kunne fortolkes af\ldots\ Ja, af hvem, kunne man så netop spørge --- så ville det netop for mig at se nok kræve, at man multiverset først ligesom definerer for sig selv så at sige (altså metaforisk talt selvfølgelig), hvordan de ikke-formelle ting skal formuleres, og så er vi tilbage til noget, der tilsvarer det rent formelle. Og jeg er så kommet frem til (ikke at jeg har tænkt meget over det, men det virker nu fornuftigt nok), at hvis den gennemsnitlige univers-sætning rigtigt nok praktisk taget er uendeligt lang, og vi ser på sandsynlighederne, jamen så er der stor sandsynlighed for, at informationen bare går primært til f.eks.\ at beskrive en konstant (eller flere konstanter) med høj præcision. Eller alternativt kan sætningen være åben over for, at oplevelsen hopper imellem forskellige universer, hvorved en vilkårlig lang række delsætninger til at beskrive disse universer så kan forekomme. Man kunne tænke mere over det, men alt i alt bliver konklusionen nu nok, at der bestemt vil være en vis anseelig sandsynlighed for, at universet vil opføre sig som, hvad jeg lidt kalder et ``lav-informations-univers,'' hvilket altså er universer som vores, der umiddelbart opfører sig, som om dets love er beskrevet med ret lidt information, i hvert fald hvis man ikke tæller konstant-præcision med, men mere ser på den overordnede struktur af lovene, og altså hvor mange sætningskonjunktioner (`og,' `men,' `hvis \ldots, så \ldots' osv.) lovene f.eks.\ har. Så ja, der er altså umiddelbart ingen grund til at antage, at en hypotese om, at det samlede multivers (hvilket vil sige ``al eksistens'') består af ``alt hvad der kan eksistere,'' samt en hypotese i øvrigt om, at ``alt hvad der kan eksistere'' er lig ``alle mulige oplevelser'' --- nå ja, plus en hypotese om, at ``alle mulige oplevelser'' kan beskrives i et formelt sprog (muligvis et stort ``underlæggende logisk sprog for al eksistens,'' som vi måske ikke kan forstå som dødelige væsner, men muligvis \emph{kan} det bare beskrives af formelle sprog såsom SOL, som vi godt kan forstå) --- vil føre til 100 \% kaos. Tvært imod mener jeg altså, at man vil kunne lave et rimeligt ok argument for, at hypotesen (altså de nævnte hypoteser her tilsammen (som altså den hypotese, jeg klart holder mest af)) vil føre til, at der bliver en anseelig frekvens af ``lavinformations-universer,'' som jeg kalder dem, såsom vores eget. Hm, jeg burde nok hellere kalde dem ``tilsyneladende lavinformations-universer,'' for at det ikke forvirrer, for informationen kan jo meget vel altså i gennemsnittet blive uendeligt for denne hypotese. 



%"..exp(-iHt)-transformation.. *(og sammenlignet med exp(-iSv)..)" (tjek)

%(19.10.21) Hm, jeg overvejede at nævne noget omkring det interessante i, at hvis der ikke er noget kollaps, så vil universets (og her kan man godt begrænse sig til f.eks. det kendte univers (nuværende)) bølgefunktion altid bare være én rotation væk fra at bekrive eksempelvis nutiden. I vores univers er der mange symetrier, og der er således mange kvantetranformationer, vi kan lave, hvor vi ser bølgefunktionen på en anden måde, men hvor resultaterne er de samme. En lidt funky transfomation er så Lorentz-transformatioen, hvor man jo netop ændre tiden langt ude.. Hm.. Tja, måske er der ingen, der vil finde dette interessant, men det er om ikke andet et tankeeksperiment, der siger noget om, at bevidsthed ligesom må have nogle \emph{regler} for, hvordan den kobler på det fysiske, for ellers.. Tja, eller viser dette tankeeksperiment bare noget andet interessant, nemlig at alt vil ske til hver en tid. Hm.. Lige for at præcisere lidt, så går tankeeksperimentet ud på at "skifte billede" og transformere bølgefunktionen (og operatorerne) over i et billede, hvor alt simpelthen bare er roteret tilbage eller frem i tiden. Og så kan man så spørge, hvis alle disse billeder er lige "rigtige," hvornår sker tingene så faktisk; sker de så ikke bare til alle tider? Selvfølgelig kræver dette en antagelse om, at "alle billeder er lige rigtige," men hvis ikke man vil gøre den, jamen så indrømmer man jo lidt, at der er nogle bestemte regler for, hvordan bevidsthed kopler til det fysiske, som ikke bare er totalt symetrike, men som har noget arbitrært i sig. Hm, og min tanke var så, at man herved måske kunne fange nogen ved dette tankeeksperiment og gøre dem splittede på, hvad de så mener, hvis de er tilbøjelige til både at være meget (ontologisk) materialister samtidigt med at de godt kan lide, når symetrier i universet betragtes som noget ophøjet, og så samtidigt med at de måske dog ikke er så glad for konklussionen om, at "alt sker hele tiden".. Tja.. Nu har jeg da lige nævnt det (og så kan jeg også se, at der nok skal arbejdes lidt for at gøre tankeeksperimentet interessant..). 

%(19.10.21) Følgende for også bare lige lov at stå herude i kommentarerne. Jeg har overvejet før det her med, om vi kan have forskellige oplevelser af de samme ting, f.eks. farveoplevelser (et kendt spørgsmål), og når nu vi så snakker om, at "oplevelser kobler til hjerner," og når nu det jo er klart, at de så muligvis kan gøre dette på forskellige måder, så kunne man jo oplagt spørge om: Kan forskellige koblingslove så føre til forskellige oplevelser af samme ting, f.eks. farveoplevelser? Men her er et rimeligt kort svar faktisk, at så længe de "fysiske love" ikke afhænger af sjæle-/oplevelses-koblingslovene i universet, så vil der ikke kunne være en forskel i oplevelsen, som hjernen kan registrere bevidst. For hvis de registrere en forskel bevidst, så vil dette jo netop give (eller om ikke andet kunne medføre) en ændring i de fysiske bevægelser og positioner. Så medmindre de fysiske love afhænger af oplevelser (en form for konstruktivisme (altså den humanistiske og ikke den matematiske variant)), og vores tanker og fornemmelser dermed direkte kan ændre på den fysiske verden, så er svaret simpelthen: Hvis der er forkellige i, hvordan vi oplever ting, så kan vi aldrig sel kunne registrere det rigtigt. Ja, man ville sågar kunne have et alternativt univers til vores, hvor oplevelseslovene skiftede med jævne mellemrum uden at de fysiske love ændrer sig. Og selv i sådant et alternativt univers (som jo så må kunne lade sig gøre) vil vi altså ikke kunne registrere disse skift i oplevelseskaraktererne rationelt; de vil aldrig føre til ændrede tankemøsntre i vores hjerne, og vi vil aldrig kunne stoppe op og tænke over dem --- selv hvis skiftene rent faktisk får tingene lige pludselig til at føles anderledes. 





%Brainstorm:
%(11.05.21) Jeg tror faktisk, der er noget galt med CUH med endelige oplevelser kun, hvis man summerer i en standard rækkefølge med stigende sætningslængde undervejs, når gennemsnittet beregnes. For det virker til, at det mindst simpelt beskrevne univers så vil få en frekvens gående imod 100 \%, og hvis der så er et simplere univers end vores, hvor oplevelser som vores \emph{kan} lade sig gøre, men bare er vildt usandsynlige, så ville vi altså måtte konkludere, at al fortid bare har været rent tilfælde og at "illusionen" må opløses når som helst. Og selv hvis der bare findes et simplere univers med en 100 \% sandsynlighed, som \emph{ikke} kan inkludere vores oplevelser, så ville det stadig være underligt at være en del af et fåtal med virtuelt 0 \% sandsynlighed for at forekomme, når vi ikke kender multiversets "levealder." Men det gode er så, at jeg egentligt heller ikke ser nogen grund til, at oplevelserne bør være endelige, og hvis begge ting kan lade sig gøre, så må begge ting forekomme pr. eksistensens/multiversets symmetri. Der er så lige en hage mere, der nok er mere vigtig. Man kan sige det samme, om universer der indeholder flere sjæle end en, eller indeholder en sammensat, overordnet sjæl, om man vil (et 'sjælefelt,' som jeg lidt har kaldt det (metaforisk) et sted nedenfor..). Ja, og dette er jo pludseligt en langt større hage, for så kræver det faktisk, at det ikke giver mening med en sammensat sjæl i det store hele. Jeg må dog sige, at jeg alligevel er tiltrukket af denne mulighed, for jeg har netop også skiftet lidt holdning til det. Det viser, at jeg nok var lidt biased før og/eller at jeg er det nu, hvad jeg derfor bare vil antage, at jeg er, men når jeg tænker over det nu, selv når man tænker på hjernespaltning eller hjerne-sammensmeltning: hvis to hjerner kan gå sammen og lave en højere bevidsthed, enten over tid eller med det samme, så må der stadig være tale om en seperat ny oplevelse. Hm, nu hvor jeg skriver det, klinger det ikke ligeså overbevisende, men jeg føler altså lidt selv, at jeg godt kan finde argumenterne. ..Jeg tror nu altså mere på, at oplevelser er... Ja, jeg har jo allerede arbejdet med antagelsen om, at oplevelser \emph{er} roden af, hvad eksisterer, og hvis konceptet om en oplevelse dermed er noget helt fundamentalt, og at en 'oplevelse' altså betegner noget helt fundamentalt ved den smalede eksistens, så giver det også bare bedst mening, hvis hver oplevelse har ét subjekt. Jo, og nu klinger det pludseligt meget mere overbevisende, for hvordan kan det også være anderledes. Subjektet defineres vel ud fra oplevelsen, så hvis oplevelsen indeholder mange facetter, så må subjektet føle og opleve alle disse facetter. Ja, det virker altså fornuftigt nok. Så den samlede eksistens er altså eksistens af en mængde oplevelser, der hver især har et subjekt, som i hver vores tilfælde er os selv (så vi er hver især roden til hele universet under denne antagelse (så når man gør godt eller ondt mod en anden, så gør det stadig ligeså godt eller ondt på den anden, men teknisk set foregår denne glæde/smerte bare i et andet "univers," hvis man vil se det sådan)). Nå, og så kommer spørgsmålet til universet: hvilke af denne mængde af oplevelser vil eksisterer, og hvornår --- og hvis tid ikke giver mening at spørge om, så i hvilken rækkefølge foregår de i (og foregår de eventuelt samtidigt (i tid eller i rækkefølgen) eller ej)? Og her må den samlede eksistens / det samlede multivers (af oplevelser) så svare, jamen ud af alle de mulige bud, som I kan finde på (og måske også nogle bud, der går over jeres hoveder), er der selvfølgelig en hvis, ikke nødvendigvis proper, undermængde af disse bud, der giver mening ud fra den fundamentale logik om alt, og ud fra samme logik vil der også være en vis opdeling af alle disse muligheder. Og da jeg, den samlede eksistens har perfekt symmetri relativt til den grundlæggende logik om alt, vil alle de forskellige muligheder, \emph{hvis} der er mere end én, hvad der dog meget vel kan være omvendt (at der er \emph{én} mulighed for, hvordan et del-multivers kunne udarte sig), forekomme som en slags mere eller mindre ligeværdige del-multiverser (alt efter om den grundlæggende logik har en rangering af disse mulige del-multiverser eller ej (og hvis ikke, vil de alle sammen bare forekomme lige meget)). Alle disse del-multiverser, hvis der er mere end ét, vil så give et svar på, hvilke oplevelser forekommer og "hvornår." Du læser/lytter er så en af disse oplevelser hørende til det samlede multivers, og hvis du skal regne dine egne prior-sandsynligheder ud for din fremtid, skal du altså så i princippet lave et muligvis vægtet gennemsnit af alle del-multiversernes gennemsnit af proir-sandsynligheder, hvilket hver især findes ud fra antagelsen om, at du som jeg-individ ikke kender tidspunktet siden starten af dit multivers, og dermed må summe over alle oplevelser (som er lig "universer" i denne begrebsverden) fra start og fremefter, og så hvordan prior-sandsynlighederne konvergerer (og hvis de ikke konvergerer, så se på hvordan de statistisk set svinger (så godt som dette nu kan lade sig gøre --- det er jo ikke fordi man nogensinde behøver at gøre alt dette i praksis; vi skal bare gerne kunne opnå, at prior-sandsynligheden for et univers a la det vi opserverer (med dets tilsynelidende faste love osv.) ikke har sandsynlighed 0 i det smalede multivers)). 
%Nå ja, og det skal nok lige nævnes, at når jeg skriver CUH, så tænker jeg egentligt ikke på Turing-maskiner; jeg tænker i stedet på endelige logiske sætninger fra hvad end dette fundamentale logiske "sprog" (hvad det nok ingengang er; det er nok mere profound end det) konstituerer. 
%Jeg kan også lige nævne her, at jeg har i sinde så at forklare lidt om, hvordan spørgsmålet om, hvordan en bevidst oplevelse kan koble på et fysisk system, nu bliver trivielt, for hvis oplevelser er det grundlæggende, der "eksisterer," så er fysikkens love altså beskrevet som en del af oplevelsens love. Så de to ting var simpelthen bare ikke seperate love til at starte med. Hvis man skulle finde alternative teorier til min, så tror jeg alligevel altid, at dette vil være løsningen på det problem på en eller anden måde, nemlig ved at antage at et univers' love ikke bare må inkludere de "fysiske love," men også må inkludere love for for, hvordan oplevelser kobler til de fysiske genstandes bevægelser. Med andre ord skal et univers' love altså både beskrive opførslen er det "fysiske" samt opførslen af sjælene (hvilket i min nuværende yndlingsteori i princippet bare er én sjæl, men hvor der så tilgengæld bare er en masse tilsvarende universer med lidt andre sjæle-love/-variable, således at deres sjæle godt kan koble sig fast på andre bevistheder i stedet). Dette bringer mig i øvrigt til spørgsmålet om, hvad en "sjæl" (en "sjæl" er en fin nok metafor, men lidt uheldigt at det får en associere til spøgelser osv., for i virkeligheden vil "sjælen" så være ligeså pragmatisk (og lige så "logisk" beskrevet (hvor jeg lige misbruger ordet 'logisk' en anelse)) en størrelse som de "fysiske love" i universet) gør efter dets væsens / dens hjernes død? Tja, jeg tror ikke reglen ikke, at sjælen og dermed universet vil stoppe på baggrund af en hjernes død. Det kan sagtens være, at universer kan have et slutpunkt, men der er ingen speciel grund til, at dette skal afhænge af, at sjælen mister sin nuværende rod. Det kan man sagtens forestille sig, men man kan også sagtens forestille sig, at sjælen bare ligesom går videre til den næste hjerne, den kan slå rod i, hvorved "oplevelsen" fortsætter derfra. Der er nemlig (så vidt jeg kan se) ingen, der siger, at "oplevelser" ikke kan inkludere en slags bevidstløse overgange, hvorefter der vil være helt eller delvist *(helt i vores tilfælde) hukommelsestab imellem de to puntker. Og hvis man så befinder sig i et univers med denne form for reinkarnation, så er der forresten heller ingen der siger, at denne reinkarnation skal overholde temporaliteten ifølge universets "fysiske love;" man vil sagtens kunne reinkarneres i en helt anden tid, forud eller bagud. Men det gode er i øvrigt så, at alt dette behøver man ikke at spekulere over, efter min mening, for "jeg'et"/"subjektet" er kun en metafor i den store sammenhæng. Om det er din "sjæl," der reinkarneres som et nyt væsen, eller om dette ikke er tilfældet, men at det sker i et andet univers med en anden sjæl --- eller om der slet ikke sker reinkarnation overhovedet, men at der bare findes et nyt univers med et nyt væsen og en ny "sjæl," så er det underordnet. Det betyder ikke noget. Vi har ingen grund til at håbe bedre ting for vores "sjæl" frem for andres "sjæle" efter døden, eller når som helst, for "sjæle" vil, efter min mening, alligevel bare være gjort af det samme "stof" set i det samlede multivers. Hvis man kunne således sagtens antage, at vi skifter sjæl for, lad os sige, hvert minut som eksempel, og dette vil stadig ikke være noget, vi behøvede at tænke på. For "sjælen" er bare nogle fysiske love for det individuelle univers, og er ligeså levende/død som resten er universets ("fysiske") love. "Sjælen" er altså bare nogle love, der kobler en oplevelse til et system af "fysiske" objekter. Vi, som subjekter, er ikke vores "sjæl," med den definition, jeg bruger her, men er selve "oplevelsen," som denne sjæl producere. Så som subjekter kan vi altså være ligeglad med, hvilke sjæle, der producerer vores oplevelse, og eksempelvis om der er flere sjæle om det (som i eksemplet, hvor det skifter hvert minut) eller ej. Og vi kan sågar også være fuldstændigt ligeglad med, om en oplevelse foregår på tværs af universer (hvis f.eks. mit subjektive univers ophører om et minut, men at der eksisterer et andet univers til gengæld hvor Mads Juul Damgaards oplevelse så til gengæld starter og forsætter efter samme tidspunkt), efter min mening. For det eneste, der betyder noget, er bare den samlede mængde oplevelser i det samlede univers; det er selve roden af hele eksistensen. Hvordan og hvornår disse forskellige oplevelser bliver "produceret" af det samlede univers er helt underordnet i min optik. 
%Der er også et interessant spørgsmål om, hvordan en "sjæl" og dens hjerne producerer lige præcis de specifikke følelser, som vi har, og ikke nogle lidt andre; hvor vi dog ikke kan kende forskel, fordi vi alrdig har kendt andet. Et godt eksempel er spørgsmålet om, hvorvidt den følelses- og sanse-mæssige oplevelse af farven 'rød' er den samme for mig, som den er for dig, eller for en tredje. Et spørgsmål der kan være interessant at fundere over, men ikke et spørgsmål, men nødvendigvis behøver at tænke over, når man teoretisere om eksistens, for her kan man også bare hurtigt sige, at hvis det faktisk giver mening, at den samme hjerne i forskellige versioner af det samme univers (med de samme begivenheder) kan føre til forskellige subjektive, sansemæssige oplevelser (uden at hjernen dog vil reagere anderledes på denne forskel), jamen så må vores samlede eksistens pr. dens generelle symmetri jo bare indeholde alle de mulige varianter af disse sanse-oplevelser (eller vi kan tænke på de som varianter af sjæle-typer, hver med deres forskellige "sanseoplevelses-moduler") i et eller andet omfang (symmetrisk pr. den fundamentale logik).
%Angående mit punkt om mit "besjælingsprincip," så handler det vel bare om, at der jo er ingen der siger, at "sjælen" skal koble meget nøjagtigt til de mikroskopiske processer; sjælen kan godt koble til mere abtrakte forhold omkring en hjerne. Vi mennesker har en evne til at kunne tillægge selv døde ting følelser (besjæling), som jo må siges at være en meget abtrakt proces. Selvom jeg personligt tror, at en livlig oplevelse kræver en kompliceret hjerne, så er der altså ingen der siger, at den faktiske "besjæling" af den fysiske hjerne, ikke også i princippet kunne fungere på en ret abstrakt måde.
%Selvom vi nu bare har sagt i det ovenstående, at alle mulige del-multiverser bare må forekomme i en eller anden symmestrisk ligevægt, så kan det stadig være interessant at prøve at spekulere over, hvilke nogle del-multiverser, der giver mening. Dette vil pr. den måde, jeg bruger begrebet 'del-multivers' her, altså indebære at spørge sig selv om, hvilke nogle lovmæssigheder der kan give mening for mængden og for rækkefølgen/temporaliteten af oplevelser i et (del-)multivers. Et fint bud kunne så være, at der gælder, at alle oplevelser bare starter samtidigt, men med en naturlig orden efter, hvor kortfattet de kan beskrives i den fundamentale logik, og at der så er en global tid i multiverset lig den subjektive tid af disse samtidige oplevelser, og hvor en, ikke nødvendigvis proper, undermængde af disse oplevelser er uendelige (hvormed deres lovsætninger dermed ikke behøver at inkludere et sluttidspunkt eller en slut-begivenhed). *(Undermængden kan også godt være nul, for så længe uendelige oplevelser bare \emph{er} mulige i det sammensatte multivers, så vil prior-sandsynlighederne altså stadig konvergere til noget ikke-(fuldstændigt-)kaotisk..) Et andet fint bud kunne være, at oplevelserne kommer ved at den fundamentale logik simpelthen "regner sig selv ud," således at vi alle i bund og grund er besjælede udregninger af, hvad der er sandt og falsk omkring eksistensen som helhed. Og ordningen for dette bud kunne så være, at specifikke (altså korte) oplevelser er ordnet ud fra, hvor "nemme" de er at finde frem til, og hvor "hurtigt" man kan finde frem til de pågældende svar omkring dem, hvis vi metaforisk set ser den "fundamentale logik om alt" som en slags (perfekt) intelligens/computer, der udregner alt hvad der er sandt eller falsk, ikke på nogen som helst global tid, som denne "intelligens"/"computer" (som metaforer) lever på, men stadig med en vis rækkefølge, og hvor alt "tid" så kun er, hvad de herved producerede oplevelser opfatter.  Et trejde fint bud kunne være, at der \emph{er} ét eller flere perfekt eller til-sammen-perfekte væsen(er), eller rettere "rene intelligenser" er en eller anden (perfekt pr. den fundamentale logik om alt), der så aktivt til sammen udtænker alle mulige oplevelser og herved også føler disse oplevelser. Rækkefølgen og mængden af oplevelserne herved kan så sagtens stadig være ligesom i det første eller det andet tilfælde, hvilket altså henholdsvis vil sige, at alle oplevelser ordnes efter deres beskrivelser i den grundlæggende logik (eller den version af det, som pågældende intelligenser forstår (da de kun er "perfekte" til sammen, og da "perfekt" kan indebære mange ting)), og hvor de eventuelt så bare udtænkes samtidigt, eller at alle oplevelser udtænkes i takt med, at intelligenserne "deducere" dem. Jeg tror bare, jeg stopper her, men man kan sikkert finde på flere idéer til, hvad der kan ligge i kernen af et (del-)multivers, og man vil helts sikkert i øvrigt også kunne finde på andre fornuftige ordninger af alle oplevelserne, som sådanne del-multiverser kunne medfører (jeg har jo kun nævnt to ret forskellige, men der må næsten være flere imellem disse to, og flere generelt..). Bemærk i øvrigt, at en mulig version af den trejde mulighed nævnt her kunne være, at de til-sammen-perfekte væsener i dette del-multivers så bare simpelhen \emph{er} alle dette del-multivers' "sjæle" i bund og grund, som så enten passivt eller, hvad mere interessant er, helt eller delvist \emph{aktivt} så vælger, hvilken oplevelse, de vil gennemleve (og hvor disse intelligente og helt eller delvist aktive sjæle så dermed måske også kunne være udødelige og dermed bare blive ved med at vælge en ny oplevelse efter den anden). Jeg synes disse muligheder er meget interessante, og det vil bestemt også være interessant at finde på flere (samt at analysere dem alle yderligere). 
%Nå ja, også i øvrigt er der også den mulighed, at man har et (del-)multivers (eller selvfølgelig et samlet multivers; det kan vi jo ikke vide på stående fod) hvor alle oplevelser bare har en fast grundlæggende længde --- eller at de måske bare er ordnet efter subjektiv tidslig længde --- som noget fundamentalt, og at oplevelsernes ordning (udover den eventuelle ordning efter subjektiv længde i første omgang) så også bare er efter, hvor let de kan beskrives i det fundamentale "logiske sprog," som multiverset "beskrives" og/eller "beskriver sig selv" ud fra. Denne mulighed vil heller ikke, så vidt jeg kan se, løbe ind i paradokset, der medfører en 100 \% frekvens for det mest simple univers, som der eller kan være fare for i variantioner af CUH/MUH. (11.05.21)
%Nå ja, og jeg bør også lige forklare det her med, at der ikke kan være en resulotion/opløsning i, hvor tætte to forskellige oplevelser kan være, før at de tæller som den samme oplevelse og kun opleves én gang i det samlede multivers. Hvis dette var tilfældet, så ville vi jo bare enten have komplet kaos, eller også ville det kræve et meget endeligt multivers (altså endeligt i både tid, rum og antal universer). For hvis ikke multiverset så var så endeligt og begrænset (i en helt vild grad), så ville alle mulige forskellige tænkelige oplevelser jo forekomme mindst én gang, og dermed vil de oplevelser, der følger fysiske og statistiske lovmæssigheder, altså ikke blive mere frekvent end de kaotiske. Hermed er det så altså ikke sagt, at der ikke kan være en endelig opløsning omkring, hvad en oplevelse er, det kan der sagtens være (jeg er endda tilbøjelig til at tro dette; det giver god mening for mig). Men det går bare ikke for teorien, at to identiske oplevelser fra to forskellige universer og/eller tider så kun tæller for én oplevelse i det samlede multivers / den samlede eksistens statestik. Åh, og jeg skal forresten også lige slå fast (hvilket på en måde er lidt relateret til det emne (om opløsning)), at jeg altså selv klart er tilbøjelig til at tro på, at den fundamentelle logik for den samlede eksistens ikke indeholder uendelige sætninger; dette giver ikke rigtigt mening for mig. En logisk sætning vil altid indeholde en endelig (men ikke begrænset; sproget er ikke begrænset i sætningslængderne) mængde information. Så selv om jeg tror på (er jeg begyndt (jeg var nemlig skeptisk over for dette tidligere)), at "oplevelser" (så i min teori altså et univers' samlede (enlige) oplevelse) godt kan være uendelige, så tror jeg stadig på, at de altid skal være beskrevet med endeligt meget information (dog ikke begrænset). Jeg tror dog ikke, at uendelige sætninger med uendeligt meget information i vil ændre det statistiske billede af teorien.. tja, eller det ved jeg ikke.. Men uanset hvad, synes jeg altså bare ikke rigtigt, det giver mening at forestille sig. Så jeg er altså muligvis lidt en eksistentialist (metematisk), hvis jeg altså forstår termet korrekt.. Med andre ord tror jeg ikke på, at universer tror jeg ikke på, at universer kan følge teorier, der ikke er konstruerbare (hvad end det kommer til beskrivelsen af de fysiske ganstande og dets dynamikker, eller når det kommer til sjæle og sansninger). Og når det kommer til matematikken, er jeg i øvrigt af samme grund ikke interesseret i matematiske spørgsmål, der kun er interessante (/spændende/underlige at begribe), hvis man prøver at forstå dem i ikke-konstruerbare teorier. Ikke-konstruerbare teorier er nemlig for mig ligesom bare noget mennsket har opfundet; det samlede multivers består af universer, der alle følger kontruerbare lovmæssigheder (mener jeg). (Det var i øvrigt pga., jeg opdagede, hvad konstruerbare teorier (eller 'konstruerbare universer,' men det bliver forvirrende her) var, at jeg ligesom var i stand til at tænke på eksistensspørgsmålet i et nyt lys (da jeg endelig kom til at tænke på dette emne efterfølgende).) (Uh, og lige for at præcisere igen, når jeg derfor taler om CUH, så tænker jeg altså selv --- for jeg er nemlig ikke 100 \% sikker på, hvad det konventionelle term indebære; om det indebærer det samme, som jeg antager, når jeg bruger det --- på MUH, men bare hvor det understreges, at der ikke bruges nogen strukturer til at beskrive universerne, der ikke tilhører en mængde af strukturer har et kontruerbart/formulerbart map til/fra \omega, altså til/fra de naturlige tal. Så universerne, som jeg bruger termet, CUH, er altså med andre ord stadig beskrevet ved matematiske sætninger, men kun af eksakt formulerbare sætninger, og dermed altså uden nogen forekomster af ikke-konstruerbare strukturer.) (11.05.21)

%(13.05.21) Hm, lige en interessant tanke lidt angående besjælingsprincip: Man kunne også tænke sig, at bevidstheden/"sjælen" faktisk kobler sig, ikke bare på den faktiske dynamik af et system, men også på selve lovene bag.. Tanken er så, at bevidsthed med denne antagelse ikke vil koble på tilfældige bevægelser, der bare pr. et lykketræf kommer til at opføre sig som en hjerne, men at bevægelserne skal komme fra en naturkraft, der får bevægelserne til at ske (fordi bevidstheden altså så kobler mere på denne naturkraft, og altså udgør en slags "besjæling" af denne kraft). Bare lige en interessant tanke.. I øvrigt kunne det med denne antagelse muligvis igen blive en mulighed, at vi \emph{er} en del af den mest simple teori, som bevidsthed kan fremkomme af (som altså som nævnt (er jeg lidt kommet frem til) vil have en 100 \% frekvens med visse multivers-antagelser).






\newpage
\subsection{Energi-, agrikultur- og klimaløsninger}

\subsubsection{Tang og alger}
Jeg har haft tænkt lidt over forskellige måder, man måske kunne dyrke tang/alger på store vandarealer, som en mulig energi-/klima-løsning. Den overordnede tanke er, at selvom plantedyrkning ikke nødvendigvis er den mest effektive løsning til at opfange solens energi på, så kunne en billig og skalerbar løsning, hvor man dyrker tang/alger i store arealer på havoverfladen, måske konkurrere med andre løsninger alligevel. Jordens er jo som bekendt spækket med store blå ørkner, så hvis man kunne udvikle noget nemt skalerbart, så kunne kvantitet måske gøre op for manglende kvalitet. Og det seje ved planter (inkl.\ alger) er jo også, at de kan formere sig selv og ``bygge'' sig selv på en måde. I øvrigt kunne tang/alge-arealerne muligvis også medvirke til, at jorden fik en højere albedo. Nå, så det er altså motivationen. Specifikke idéer jeg har haft, har så bl.a.\ været at gro tang via nogle næringsholdige kapsler, hvorpå tangen kan gro, og hvor tangen kan opsøge næringen fra kapslerne uden at det spilder ud i havvandet. Måske hvis næringsbeholderen havde et net, hvor tang kan gro som en fast hinde omkring beholderen/kapslen. Så ja, det var lige en hurtig idé der, som måske (men kun måske) kunne være værd at regne videre på. *[Man kunne også overveje at have rør ned til dybere vandlag og pumpe næringsholdigt vand op til overfladen til sin algeproduktion. Dette er muligvis en ret vigtig idé i sig selv, også selvom jeg glemte at nævne den her i første omgang.] *[Jeg havde nær også glemt at nævne min mere oprindelige idé, nemlig at fremelske alger, der kan optage næring fra olie, eller en anden tynd film, der kan spredes på havoverfladen, sådan at man nemt kan gøde store arealer, hvor alger så kan klæbe til havoverfladen i et lag og udgøre store arealer af fotosyntetiserende materiale. Hvis algerne også klæber til hinanden kunne man måske løbende høste algerne ved at suge dem op (i en fastankret båd eller en anden form for station), måske nærmest ved at skimme dem. I øvrigt kunne jeg lige nævne, at hvis algerne skal klæbe til en oliefilm, så skal de måske selv udskille olie på overfladen, så måske vil et muligt ekstra skidt kunne være, at man måske bare høster denne olie og enten sier eller skiller algerne fra (og måske pumper dem ud igen i sidstnævnte tilfælde). Jeg synes dette er en rimeligt spændende idé, så det er lidt sjovt, at jeg åbenbart glemte den helt i første omgang, da jeg skrev denne undersektion.] Den næste idé er lidt mere sci-fi, og det er at skabe en symbiose mellem lange tangplanter og tang, der kan ligge i overfladen (som en slags krone; så behøver de lange tangplanter ikke at fokusere på energiproduktionen), hvor de lange planter (stammen eller stilken\ldots\ eller rødderne, kunne man også se dem som) så måske kunne nå ned til mere næringsholdige vande og suge næringen op til overfladetanget. En yderligere sci-fi-agtig idé oven i denne, kunne være at fremelske og/eller gensplejse tang, så det afgiver olieholdige frø, som kan høstes, i stedet for at man skal høste selve planten. Nå ja, og en anden, endnu mere sci-fi (men altså nok ikke ligeså fornuftig/realistisk), løsning til det samme kunne være at fremelske en symbiose (eller at parasitisk forhold; samarbejdet behøver ikke at ``gavne'' selve planten (men bør selvfølgelig dog være skånsomt nok for planten)) med et dyr, der kan suge saft fra planten og danne det olie/fedt af dette, hvor man altså så høster disse dyr. Okay, så det var vist det, jeg ville nævne om mine samlede alge-tanker\ldots\ Nå nej, der er også lige en endnu ældre idé, jeg lige kan nævne, og det var at prøve at fremelske tang/alger, der kan opsuge næring fra en olieopløsning, således at man kan gøde sine specialdesignede alge-marker ved at (gyde) sprede olie på vandene. Hvis algerne ligesom kan klæbe lidt til den olie, så olien ikke bare diffunderer ud i resten af havet, kunne dette måske betale sig over at bygge og producere specialdesignede kapsler til at holde på næringen. Nå ja, og én idé mere: Man burde også lige se på (hvis ikke man har gjort det (det har jeg ikke)), om det eventuelt ville kunne betale sig bare at bygge nogle (evt.\ bølgeenergi-drevne) maskiner til at hive næring op fra havdybet i stedet. 




\subsubsection{Flere bio-orienterede energi- og agrikultur-idéer}
Jeg har også lige en række andre små idéer, som alle sammen på en eller anden måde indebærer at fremelske organismer til at varetage funktioner i et samlet (resurseudvindings-)system. Ja, og når jeg lige tænker over det, så er selve tilgangen til denne fremelskelsesproces nok en stor del af selve idéerne. Mange af mine idéer her handler nemlig i en eller anden grad om at skabe en symbiose, både mellem forskellige arter dyr, men også særligt mellem menneskelig teknologi. Hvis man således kan få en fremelsket art dyrs reproduktive adfærd til at afhænge af menneskelig teknologi, tror jeg, det kan åbne for adskillige muligheder for at bruge dyr mere i agrikultur og resurseudvinding (inklusiv solenergiudvinding).

Hm, når nu jeg lige har tænkt lidt over det, så er der nok ikke så mange af disse idéer, der er værd at nævne, men jeg har bl.a.\ en idé om at udnytte græshopper eller lignende, enten ved at dyrke dem til spise eller brændsel, eller ved at bruge dem som arbejdskraft til at samle plantemateriale ind fra vilde områder. Denne idé er i øvrigt også ret illustrerende for, hvad jeg mener, når jeg nu nævnte at idéerne ofte har handlet om at kontrollere dyrs reproduktion (i et symbioseforhold mellem dyr og menneskeskabte teknologier). Tanken er her at dyrke store flokke af græshopper i kontrollerede forhold, således at man muligvis (det ved jeg ikke, men måske kunne det være værd at undersøge, hvis idéen ellers holder) kan dyrke meget større flokke, end hvad der naturligt vil forekomme, og så slippe græshopperne ud som unge i områder med meget vild bevoksning. Når græshopperne så har vokset sig store bør de så returnere (og dette er så en adfærd, man gerne skal fremelske) til centeret, hvor de kom fra for at reproducere. Ved så at fremelske dem sådan, at de faktisk kun kan reproducere med teknologiske hjælpemidler sikrer man sig, at de ikke ellers spreder sig i naturen (og gør at man alt andet end lige ret nemt kan stoppe hele processen igen). Centeret vil så på ny opdyrke en ny generation af græshopper og slippe dem ud igen, når de er klar. En del af en sådan type teknologi er så, at centeret (eller hvad vi skal kalde det) så også herved let får mulighed for at videreselektere dyrene, både så de bevarer deres fremelskede egenskaber, og måske også så det forfiner egenskaber yderligere eller udvikler nye. I dette tilfælde kunne man f.eks.\ sørge for primært at videreføre generne fra de største (og måske mest fedtholdige (eller proteinholdige), hvis det er det, man er ude efter) græshopper. En anden idé kunne så som nævnt også være, at benytte græshopperne til at samle plantematerialet ind i stedet for bare at spise det. I så fald kunne man så sikkert med fordel blande disse idéer. Jeg tænker så, at græshopperne bare kunne have som adfærd, at samle plantematerialet i bunker uden for tæt bevoksning, så maskiner (eller andre typer dyr, hvis dette giver mening) efterfølgende kan komme til det. \ldots Hm, man kunne måske også prøve at få dem til at samle det på græssletter, hvor man så kan have græssende dyr gående\ldots\ Og så kunne man eventuelt bruge dufthormoner til at fastsætte afsætningsområderne, hvis dette så bliver gavnligt, og hvis man altså kan formå at fremelske dette\ldots\ Ja, det kunne også være nogle muligheder. Jeg ved ikke helt, hvordan man kunne videreselektere på denne adfærd\ldots\ Nå jo, måske kunne man selektere via gruppe selektion ved at kunne få græshopperne til at udskille forskellige dufthormoner på kommando, og så måske sørge for at lukke dem lidt ud i grupper med forskellige dufte (men måske er dette så ikke nødvendigt\ldots) og hvor man så efterfølgende måler på, hvilken duftgruppe var mest repræsenteret i plantematerialebunkerne. Man kunne måske også bare lukke dem ud i hold til forskellige tider og så holde øje med den tidslige forskel i bunkeproduktionerne. Jep, så det var altså denne idé. \ldots\ Nej, ikke helt; jeg kom lige i tanke om en tilføjelse til den. Græshopperne, eller hvad det måtte være (det kunne også være en anden specialselekteret dyreart), kunne også få til opgave at så planterne også (og måske hjælpe med at dyrke dem). Således kunne produktet altså ikke bare blive enten dyre-biomasse eller plantemateriale, men kunne sågar være faktiske afgrøder. Dyrene kunne således f.eks.\ så hvede rundt omkring på ellers vilde områder for så efterfølgende høste dem igen, når de bliver modne. Denne tillægsidé kunne så måske også have den fordel, at det kunne føre til lidt mere spændende områder (delvist) til dyrkning af afgrøder, fordi man måske så ikke nødvendigvis behøver så plane landskaber, som maskiner kan køre på. 

Uanset om vi snakker mere vilde områder, eller bare sædvanlige landbrugsområder, så kunne brugen af specielt fremelskede dyr i landbrug selvfølgelig også bare generelt være en ting, der muligvis kunne være værd at forske noget mere i. Der er jo nærmest ikke de funktioner, man ikke på en eller anden måde kunne forestille sig, blev varetaget af specialfremelskede dyr, og hvor dyrebestanden så bare skal passes ved at formere dem kunstigt som nævnt og lukke dem ud på de rigtige tidspunkter (ud måske guide dem på anden vis, ligesom at jeg nævnte duftstoffer ovenfor), men hvor dyrene og deres landbrug ellers stort set er selvkørende. Det gode ved sådan nogle bio-teknologi-løsninger er så også, at når man først har udviklet systemet, så er det nemt at skalere op, for det kræver så mere eller mindre bare, at man avler nogle flere dyr.


Jeg har også tænkt en del i, om man kunne tænke sig til noget tilsvarende men i havet (da jeg jo som nævnt har tænkt meget i, om ikke man kunne finde en måde at opdyrke og/eller på anden vis udnytte disse store blå sletter). (Jeg tror i øvrigt også det var laks, der inspirerede mig først.) På den anden side så er fisk jo allerede ``specialdesignede'' til at formere sig så meget som muligt, og hvis de bliver spist, så er det jo bare af større fisk. Så som jeg lidt er kommet frem til nu, så skulle grunden til at prøve at tænke i specialfremelskede fisk være, hvis man gerne vil reducere skibstrafikken og fiskeriarbejdet ved at lade fiskene migrere selv tilbage til centre, hvor de let kan fiskes op, og hvor de så også kan formeres i den forbindelse. Her kunne der så godt nok også være meget at spare, for fiskeriindustrien bruger sikkert en masse resurser, som den er nu, og i øvrigt bliver en sådan type fiskeri nok også mere skånsom for miljøet, fordi man så ikke fisker en masse andre typer fisk (såsom hajer), hvaler og skildpadder (osv.) op også. Hm, jeg kom lige til at tænke på i øvrigt, at måske kunne denne teknologi blive drevet af konkurrence imellem lande, fordi man jo muligvis så kan forøge fiske bestanden i sine egne farvende, hvis man kan dyrke sine specialfisk i hele havet, men få dem til så at migrere til sit eget farvand (oveni hvad der ellers måtte være af fisk naturligt), når de bliver modne nok. Ja, måske kunne dette blive en positiv udviklingsfaktor (og hvis ikke må man jo bare opsætte internationale love imod det, men mon dog, for andre lande kan jo bare følge stand\ldots\ tja, det er nok ikke til lige at sige, men det gør heller ikke noget). 

Derudover kan jeg så kun komme på tre ting, man måske kunne gøre for at udnytte havene mere produktivt, og af disse involverer kun én af dem i høj grad specialfremelskede organismer. De to andre er for det første den idé, jeg har nævnt (helt kort), om at man måske kunne prøve at hive næring op fra dybet, og den anden idé er så bare simpelthen at prøve at sprede næringsstoffer (måske med Haber-Bosch) i havet, hvis dette kan føre til øget fiskevækst i sidste ende (eller plantevækst, hvis man så også får en måde at høste dem på). Angående den første af de to så har jeg allerede nævnt at dette måske enten kunne gøres med rør (og måske med bølge- eller vinddrevne turbiner til at danne cirkulation) eller med --- muligvis specialfremelsket --- tang, der kan række ned som en slags rødder til det næringsholdige vand. Så ja, de to ting kunne måske være værd at forske i, men ellers har jeg også følgende idé, som dog er ret højt oppe på sci-fi-skalaen (men lad mig bare nævne den alligevel). Måske kunne man fremelske en form for mangrovetræer, der danner et tæt rodnetværk, som enten er stort nok til at kunne fungere som en tømmerflåde, eller tæt nok til at kunne fungere som en båd. I sidste tilfælde ville det så kræve en vandskyende hinde, eventuelt ved at udvikle et symbioseforhold med algeorganismer på rodoverfladen, der kan udgøre et vandskyende lag. Dette ville måske ikke være helt dumt i betragtning af, at en sådan symbiose nok alligevel vill være påkrævet, hvis man skal kunne udsætte mangrovetræerne for rent saltvand. Måden træerne kunne formere sig på kunne så enten være ved at et træ gror sit afkom i tørhed i ``båden'' og først udskiller det, når det selv kan svømme, at træerne bare formerer sig helt ukønnet ved bare at rodnettet bare vokser og vokser og flere tømmerflåder/både derved opstår, eller ved at man simpelthen dyrker dem i laboratorier (eller ``centre,'' hvis jeg skulle være konsistent med mit ordvalg (ha!)) og sætter dem ud ved siden af de andre, når de er modne nok. Nå ja, rodnettet kunne jo i øvrigt eventuelt starte med at fungere som en tømmerflåde, men så få mere og mere bådform, når træet vokser. Hvis så dette netværk af levende både/tømmerflåder kommer bliver stort nok, så der breder sig ud på det dybere vand (hvis man starter ved en kyst eller en ø), så kunne man måske, hvis det skal være endnu mere sci-fi, fremelske en symbiose med en en form for vandplanter, der kan vokse nedad mod bunden og sætte sig fast på sten og klipper og fungere som ankre til mangrove-øen/flåden. Og hvis vi er ude på endnu dybere vand kunne man måske endda forstille sig, at disse ankre også kunne fungere (i stedet eller samtidigt med) som næringsrødder, hvis de kan nå at række ned i næringsholdigt vand. Det er også lige vigtigt at nævne, inden jeg kommer for godt i gang, at tanken så altså er, at mangroveflåden skal gerne kunne vokse eksponentielt som funktion af arealet, og at hvert træ derfor skal kunne skubbe hinanden længere og længere væk fra hinanden, når de vokser og/eller når nye ``træbåde'' dannes imellem de gamle og vokser. \ldots\ Ja, og i den forbindelse vil det så være bedst, hvis man ikke skal gro træerne i laboratorier, men at de kan vokse af sig selv. Hvordan skal træerne drikke vand. Tja, hvis ikke det er helt nok med at filtrere saltvand, så kunne man måske forske i mere sci-fi-hurlumhej ved at se på, om man ikke også kunne krydse nogle kaktusplanter eller noget i den stil indover, som kan fange og kondensere fordampet vand fra havoverfladen (hm, og måske træerne så skulle gøre plads til, at der kan skinne sol på havoverfladen)\ldots\ Ah, eller måske kunne have salttålelige vandabsorberende plantemoduler, som kan fungere som væger, og så have et modul omkring disse, der kan opfange det fordampede vand igen\ldots\ He, nu bliver det ret spacet\ldots\ Jo, måske hvis man fremelsker nogle rør, som kan have væger inden i sig, og som kan fange vinden, så den blæser igennem rørene\ldots\ Så kunne der nemlig være drikke-moduler på indersiden af disse rør, der ligesom drikker imellem vindsugene\ldots\ Tja, det er da noget, man kan tænke over. Men ja, nu vil jeg ikke komme med flere symbiose-forslag. Tanken med disse træflåder er så, at mangrovetræerne enten har frugter, som kan høstes, eller at bådene/tømmerflåderne faktisk bliver store og rummelige nok til også at kunne understøtte andre vækster\ldots\ nå nej, det går jo ud på et, når nu vi snakker om sådan en symbiose-organisme. Så ja, tanken er bare, at træerne på en eller anden måde afgiver frugter (og/eller anden biomasse). Hvis bådene bliver store nok kunne man i øvrigt også måske bygge oven på dem og derved få flere beboelige arealer. Jeg ved godt at denne idé nok ikke kommer til at give mening inden for en overskuelig fremtid, men om ikke andet, så kan den måske sætte nogle tanker i gang hos andre. (05.06.21)

Jeg har også tænkt i, om ikke man kunne bruge (specialfremelskede) dyr/fisk til at transportere næring op fra havdybet. Her jeg jeg så tænkt på hvaler og på dybhavsdyr, der suger vand og næring og svømmer op til overfladen, når de dør, men jeg har endnu ikke tænkt så meget over det. Måske kunne man finde på noget omkring dybhavsdyr, der stiger op til overfladen, men det kan jeg ikke lige selv se mig ud af (altså om der er en lille mulighed her et sted eller ej). 


(06.06.21) Jeg kom lige på en idé her i morgenbadet omkring at fremelske undervandsedderkopper til at spinde spindelvævsræb/-snore ned til havbunden som ankre for mangroveflåden. Jeg følte lige den var værd at nævne, ikke mindst fordi det også leder videre til tanken om, at man jo også generelt i princippet kunne fremelske edderkopper til væve stoffer og/eller ræb til produktion eller til byggeprocesser (antaget at hele denne fremelskelsesproces er til at gøre). Og dette leder så også i øvrigt videre til, at man også kunne overveje, om der er andre dyr, der kunne være anvendelige i produktion og/eller byggeri. %*(Vandfiltrering kunne i øvrigt også være et område, hvor man måske kunne bruge bioorganismer...) Tja, det er næsten for trivielt til at nævne. 



\subsubsection{Andet}
(24.09.21) Lige en helt anden lille idé: Der er jo noget med, at akse fra en vulkan eller fra en stor nok brand/eksplosion kan komme op i et ydre lag i atmosfæren, hvor det tager lang tid, før det falder ned igen. Asken kan herved afkøle planeten, hvilket åbenbart kan forsage en ``nuklear vinter,'' hvis mængden er stor nok (og måske hvis det kommer fra en nuklear eksplosion ift.\ navnet, men det er lige meget). Nå, den oplagte idé ville så være at overveje, om det kan lade sig gøre og betale sig kunstigt at spy materiale op i de øverste atmosfærelag (eller hvor det nu kan blive længst; det kunne også være at jetstrømmene eller noget kunne være en idé, hvad ved jeg?) for at køle planeten? Måske kan man endda overveje, om ikke man kunne finde mere egnet materiale end akse, måske noget med en lavere massefylde og/eller noget der dannes i former, der ikke daler hurtigt til jorden (ligesom fjer, og hvad har vi). Og lige for at skabe lidt mere spænding om idéen: Måske man endda kunne investere i at udvikle et særligt velegnet materiale (og/eller måden at danne aerosolerne i velegnede former) til formålet og så patentere og sælge sin opfindelse til verden(s regeringer) som en klimaløsning. Anyway, jeg synes idéen var værd at nævne. 




%\subsubsection{Lagring af energi}
%Her her jeg også lige en lille idé. Måske kunne man lagre energi i isbjerge struktureret lidt ligesom vulkaner, men bare med vand i ``krateret'' i stedet for lava. ... Hm, og så var tanken, at man så kun ville skulle bruge en vis faktor mere energi på at pumpe vand op over ``isvulkanen'' for at bygge den, end hvad energikapaciteten bliver. Men jeg mangler lidt at finde ud af en måde, hvorpå man kan sikre sig, at isen ikke bare fryser i toppen... Tjo, man kunne jo filtrere saltet fra bygge-vandet og beholde det i lagringsvandet... Hm, problemet er så, at temperaturen jo umiddelbart falder i højden, men måske man bare kunne nøjes med at bygge mange små ``vulkaner''...? Hm, måske, men jeg tror alligevel lige jeg udkommentere paragrafen/sektionen, så det bare kommer til at stå i kommentarerne. (For som jeg ser det nu, bliver idéen stadig ret temperaturfølsom, og det går ikke rigtigt: Man skulle gerne kunne opsætte den på et (ant-)arktisk område, hvor det så bare kan blive stående året ud. ...Hm, medmindre at disse ``vulkaner'' kan dannes ret smalle og... Tja, nej, uanset hvad kræver idéen mere tænkning.)

%**(27.12.21) Nå ja, jeg kom lige i tanke om, at jeg også skal nævne, at min idé om isbjerge egentligt også kunne måske kunne være en måde at sænke havniveauet. Og måske kan man også bare satse på den idé, og så behøver man så ikke nødvednigvis at forme det som bjerge. Man kan bare pumpe vand ud på (midten af) nord- eller sydpolen, og så er tanken, at det måske kan fryse her. Og hvis varmen fra frysningen kan undslippe jorden ud i rummet, så man ikke bare varmer polerne op i processen (hvilket det må kunne, mener jeg), og hvis idéen i det hele taget kan give mening fra et fysisk (f.eks. kunne spørge: Er der koldt nok på polerne til at saltvand altid fryser?) og et økonomisk perspektiv, så kunne dette altså være en idé til at modvirke en af følgerne af klimaforandringerne.


%Nå ja, og en anden paragraf, der også bare kan blive her ude i kommentarerne, er, hvor jeg lige vil nævne mine tanker omkring at prøve at forhøje havområders albedo for at modvirke global opvarmning. Jeg har således overvejet lidt, om det ville kunne betale sig at sprede et tyndt lag af hvidt materiele (væske eller små flydende kugler). Jeg har også tænkt på, at man måske kunne sprede en ultratynd film i stedet, som pga. thin film reflection sikkert også vil forhøje albedoen. Selvfølgelig vil sådan en film skulle fornyes løbende, når den opløses i vandet, så jeg er ikke sikker på, om det ville kunne betale sig, og det er jeg heller ikke angående at sprede hvide pigmenter på vandoverfladen. Men det var stadig værd lige at nævne tankerne her i kommentarerne, føler jeg. (For man kunne jo arbejde videre på dem.)
%*Man kunne i øvrigt også overveje, om noget tilsvarende kunne give mening, bare med hvor man forsøger at forhøje vandfordampningen i stedet for albedoen (hvis man gerne vil have flere skyer og/eller mere regn, af klima- og/eller miljømæssige årsager).
%..! Jeg skulle lige til at nævne, nu her i forbindelse med, at jeg også lige har tilføjet min oprindelige idé om olieglade alger, der kan gødes via spredning af en oliefilm (og det er d. (13.06.21) btw), at man så måske kunne overveje, at fremelske alger til at udskille sådan en sådan thin film-membran (omkring sig, hvis de selv ligger i overfladen, eller som kan flyde op, hvis de selv lever nede i vandet) til at forhøje albedoen. Og nu ledte dette så videre til: Ja, eller måske kunne man for den sags skyld fremelske andre arter til samme. Aha, hvad med fisk? Ville det give mening, at fremelske fisk til at udskille thin film-materie?.. Hm, måske vil det være lidt voldsomt.. Men fiskesperm kan jo allerede godt fylde store arealer... Kunne man mon fremelske fisk til... he.. ..jo, til at producere mere sæd som hanner, og så til mere spontant at udløse det med jævne mellemrum (og altså også uden for ynglesæsonen). Hm, og måden man så kunne kontrollere bestanden på --- og samtidigt opretholde et passende selektionspres --- ville så måske være, hvis man så fremelsker hunnerne, så de migrerer til et center, når de bliver drægtige, og afleverer deres æg der, hvorefter man så kan sprede dem tilfældige steder i det omkringliggende hav. Tja, en sjov tanke, og muligvis en, der er værd at arbejde videre på --- man kunne f.eks. også tænke tankerne over på andre mulige dyre- eller plante- (eller for den sags skyld bakterie-/arke-)arter også og se på mulighederne der.



%**(06.12.21) Lad mig lige besøge disse tanker igen (bare herude i kommentarerne). Jeg har som nævnt en idé om, at man kunne sprede en thin-film på havoverfladen eller små (hvide) partikler af en art for altså at øge jordens albedo (eller dens reflektion med andre ord). Jeg har ikke troet vildt meget på holdbarheden af denne idé siden jeg tænkte over den sidst, men nu kom jeg lige i tanke om, at hvis nu man på en eller aden måde kunne udvikle en holdbar film eller et holdbart materiale til partiklerne (eller selvfølgelig også et billigt materiale, hvilket er en anden snak; det var jo også det jeg tænkte med min akse idé her ovenfor (hvor jeg vist så bare lige snakkede om de øvre luftlag i stedet for om havoverflader..)), sådan at de tager lang tid for at opløses/nedbrydes i (/på) vandet, jamen så kunne idéen måske igen få ben at gå på..  








%Hm, noget med gobler, måske især i havdybet?... ...A la græshoppe-idéen eller til at transportere næring op.. Ja!... Måske! ... (tjek; nu har jeg bare lige nævnt tanken)

%Jeg har f.eks.\ leget med tanken om at fremelske et system, hvor græsningsdyr og rovdyr nærmest arbejder sammen... Hm, måske ikke alligevel. Tanken var, at rovdyrene så kunne bære græsningsdyrenes æg og/eller sæd tilbage... Men nej, jeg tror ikke der er behov for at tænke så meget i transpotationen/migrationen af biomassen, hvis man jo alligevel godt kan få de nederste dyr i kæden til at migrere, hvad man nok kan..

%Brain: 
%Okay, der er idéen om migrerende krill... Hvor man så muligvis kunne bruge større dyr til at transportere dem... Hm, men er dette virkeligt noget værd ift. bare at fiske fisk op i både, som man altid har gjort? Umiddelbart ikke. Det vil sige, ikke medmindre man kan forhøje formeringen af dyrene/planterne nedert i kæden, og så er vi vel oppe ved ovenstående idé-type... ..Hm, måske man ville kunne gøre noget på landjorden med græshopper (i.e. at dyrke græshopper), men... Hm, så skulle det være fordi man giver dem den fordel, at deres afkom kan vokse op (dyrkes) i sikre rammer, så man kan danne forstørrede flokke ift., hvad der kan lade sig gøre naturligt... Ja, vel lidt ligesom laksebrug, hvor man slipper dem ud og lader dem komme tilbage igen (tror jeg vistnok man gør nogen steder), men bare med andre dyr også... Ja, det må altså være værd at nævne, f.eks. med græshopper.
%Angående dyre-landbrug, kan jeg bare lige nævne tanken helt kort (er nærmest beskrevet bare i udtrykket).
%Angående myrestrøm.. Den er jo lidt mere over i det sci-fi-agtige... Og den hører lidt sammen med idéen om træbatterier.. Hm, men lad mig næsten bare skrotte begge idéer... For myrestrøm handler om at få myre til at adinistrere et system, der laver strømproduktion af plantenedbrydning, men dette vil jo så ikke rigtigt kunne betale sig (ikke i en nær fremtid i det mindste) i forhold til f.eks. græshoppe-idéen. Og træer der på en eller anden måde (muligvis ved en symbiose) kan generere strøm er vist også lidt for sci-fi (og langt ude i fremtiden) til at nævne (også taget i betragtning hvor oplagt tanken egentligt er). 








%Noter:
% - "Krill + større fisk til transport og oliedannelse" (og måske også bevogtning..).
% - "Myrerstrøm," hvor det bare er myrerne, der kultiverer batterielementerne..
% - Tang eller rør-maskiner (tjek på rør) (eller (barde)hvaler, ha (ja, nej: det duer selvfølgelig ikke (men lidt sjov tanke); næringsholdigt vand må hentes op med enten pumper eller plante-osmose)) til at trække næring op.. (..så tjek)
% - "Alge-idé, el-åle, træ-batterier, lava-pumper, migrerende krill, trekant-glas-array-solfangere." (tjek på det meste; selvom jeg kun rigtigt har nævnt de to, men el-åle-træ-batterirer er ikke længere interessante for mig, lava-pumper er en ret triviel ide (kan man finde en måde at pumpe lava op, eventuelt med væske, for så at udnytte varmeenergien?). Trekant-glas-array-solfangere-idéen kan jeg bare lige nævne hurtigt her i kommentarerne: Jeg regnede mig engang lidt (jeg var dog ret ung) frem til, at tynde og spidse salt-trekanter side om side kun vil lade lys fra en vinkeler tæt på spidsernes retning igennem (fra spidsens side af), og hvis det så kan lade sig gøre, med denne struktur eller med en anden, at bygge sådan et filter --- og hvis man kan finde en måde at bygge det ret billigt på --- så ville man kunne bygge solfangere, der nærmest kommer til at have en parabol-effekt uden at have større areal, ikke når det kommer til effektiviteten, men når det kommer til den temperatur(-forskel), man kan opnå (mellem varmerørene og omgivelserne).)
% - "Fiske-hval-idé men med insekter-fugle." (græshopper eller lignende kan transportere sig selv effektivt, så der er ingen grund til at inkludere fugle her) Hm, men hvis det kom til at transportere plantemateriale... ..Så må græshopperne bare bunke det sammen nær en vej, og så kan man finde en måde at transportere det videre derfra. Ikke også?.. Jo, eller hvis man er langt væk fra en vej, så bare et passende sted, hvor man så kan transportere det videre, enten via en ny fremelsket art dyr (f.eks. landgående) eller bare via maskiner (nok det sidste). 
% - "Dyre-landbrug."
%(tjek, tjek, tjek)

\subsection{Andre tekniske idéer}


\subsubsection{Idé til et virtual reality (VR) helkropsinterface}
Nu hvor jeg har tænkt en smule over det igen, så er idéen ret simpel, nærmest triviel. Idéen er at have en helkropsdragt med stempler (eller puder) til at lægge tryk på huden, og at dragten så også kan stivne i hvert led, så man kan få følelsen af at presse eller slå på noget uden at armen eller benet går igennem det i VR-verden. Må denne måde vil man kunne gå på underlag, klatre på vægge m.m.\ og interagere med objekter på en realistisk måde. Det må også gerne være muligt for dragten at påføre lemmerne kræfter, så man kan få følelsen af at ens lemmer skubbes (og altså ikke bare af en statisk kraft). Man kunne jo f.eks.\ bruge hydraulik til at manipulere dragtens stilling og opnå begge disse ting. Dragten skal så også være opspændt på en måde, så den kan rotere, sådan at brugerens balancesans også passer med kroppens stilling i VR-omgivelserne. En lille, ikke helt triviel detalje er så, at idet dragten jo nok vil tilføje til brugerens volumen, så er det nok smart at sørge for, at bruger-avataren også er tilsvarende udvidet, så man kan røre sine egne håndflader og hænder i det hele taget i VR-verden --- også så man er i stand til at interagere med små objekter. Og hvis brugeren styrer en fysisk robot i den virkelige verden i stedet for en VR-avatar, så må man vel bare sørge for, at robotten er tykkere i lemmerne end brugeren med dragten på, og så tænker jeg at dette også snildt vil kunne lade sig gøre. Så jeg tror altså også denne teknologi (som der i øvrigt sikkert nok skal være mange andre, der har tænkt på) kan bane vejen for brugerstyrede, finmotoriske robotter, som så kan fjernstyres fra lang afstand. 

Så altså alt i alt en meget simpel idé, men jeg har alligevel valgt at nævne den her, for det kan alligevel sætte gang i nogle tanker, fordi man så altså kan tænke sig, at full-immersive VR kan blive en realitet ret hurtigt. Dette kan så både medføre, at folk kan arbejde fysisk rundt omkring på nærmest hele jorden hjemme fra deres egne stuer *(eller besøge forlystelser via robot-avatars for den sags skyld), men ikke nok med det, man kunne også sagtens forestille sig, at folk også vil bruge en stor del af deres fritid i VR-verdner, om ikke andet så i det mindste når grafikken også bliver god nok. Og i modsætning til hvis teknologien var mere som i The Matrix, så kan brugeren stadig holde sig i form i denne VR-verden. Dette gør også, at det nok ikke er ligeså essentielt for menneskeheden at udtænke det perfekte økonomiske system, som gavner alle bedst muligt, og som kan holde for evigt, for i sidste ende kommer vi nok til at bruge meget af vores tid i omgivelser, vi helt selv kan vælge, og hvor man således bare lige skal blive enige med sin lokale gruppe medmennesker om, hvad lovene skal være for omgivelserne, men derudover kan få det, ret meget lige som man vil have det. 


%Idéen er at have en solid og rummelig ``dragt,'' eller måske nærmere en puppe, udspændt i frit rum, så den også kan rotere (medmindre man dropper dette og i stedet lader brugerne vende sig til at ignorere deres balancesans) og hvor dragtens inderside så har små stempler, der sammen kan danne forskellige overflader, samt indstille forskellige hårdheder (ved at kunne justere stemplernes endemoduler, som udgår den overflade, brugeren berører, og ved at justere selve stemplernes affjedringsstyrke). Det skal så selvfølgelig være hardcodet i dragtens, at der ikke skal kunne formes farlige overflader herved. Selve brugeren skal så have kroppen udspændt på en blød og behagelig måde inde i dragten/puppen på en måde, så brugeren kan ende med at ignorere den faste berøring fra pågældende seler. Som nævnt kan det være, at personen også skal kunne roteres (pga.\ balancesansen), men her er det altså nok nemmest (tænker jeg) bare at rotere hele dragten i så fald. Dragten/puppens stempler skal så forme overflader, som passer til den VR-verden, som brugeren skal agere i. 
%
%Hm, jeg kan ikke huske, hvor meget disse tanker har været med i idéen hidtil, men jeg kom lige til at tænke på en mulig opgradering af løsning, nemlig hvor stemplerne ... %Hm, tror faktsik lige jeg vil vende tilbage til det her emne. Man kunne nemlig godt lige overveje nogle flere muligheder her..
%%Stempler på stempler, muligvis objekter og gribe kløer på stemplerne til at give brugeren objekterne, og videre muligvis objekter, der selv kan transformere sig selv til en vis grad.
%%Hm, men min originale idé var vel egentligt bedre om bare at have en helkrops-VR-dragt, der kan stivne i alle led, og som også har en lille moter i hvert led, så brugerens lemmer også kan påvirkes af ikke-statiske kræfter. Ja.


\subsubsection{Sci-fi-agtig idé til rumelevatorer}
Jeg kan ikke lade være med at nævne denne idé hurtigt. Den er ret ny. Idéen er at bruge store ringe med superflydende væske i omkring jorden (men hvor en ende af ringen svæver længere ude i tyngdepotentialet), hvor væsken flyder hurtigt rundt i ringen (som altså har rør med væske indvendigt) og skaber kraft på ringen (dem modstående kraft til centripetalkraften på væsken), så ringen holdes rund og svævende (det er i hvert fald tanken). Jeg ved ikke, hvor elliptisk man kan gøre ringen, og dermed hvor meget længere hver enkelt ring kan nå i tyngdepotentialet, men hvis man bare har et stort nok netværk af ringe, må det næsten kunne lade sig gøre, at nå så langt ud, som man vil. Hvis væske-fluxet er konstant langs hele ringen på trods af tykkelsen\ldots\ eller nok endnu bedre hvis man kunne lave stationer med strømhvirvler, som falder i hastighed, så kunne man herved lave kølestationer (for isoleringen kan jo aldrig være helt perfekt, og det må vel umiddelbart antages at solen og/eller jorden kan skinne (lys og/eller varmestråling) på ringene), endda selv hvis man finder frem til, at det ikke kan lade sig gøre at køle på væsken, når den er i fart. Denne idé kunne også bruges i et virkeligt sci-fi scenarie, hvor man lavede hele habitater ud af sådanne ringe.

%**(22.11.21) Hm, denne idé holder nu nok kun til sci-fi-fortællinger, så det er lige før, jeg bare bør udkommentere den.. Oh well, fuck det; jeg lader det bare stå sådan her.. 

%Noter:
% - VR.  (tjek)
% - Rumelevatorer (netværk af ringe må da faktsik næsten virke..! (15.05.21)) (tjek)




%\subsubsection{mat-app (i kommentarerne)} ((23.06.21) btw)
%Jeg kunne måske også nævne min idé til en matematik-puslespils-app, men jeg tror bare jeg lader det blive herude i kommentarerne, for jeg har vist også nævnt det i nogle af mine tidligere ITP-noter nedenfor. Tanken er, at ligningsløsning jo eksempelvis er en diciplin, hvor man gerne skal få en intuition om, hvilke nogle handlinger med må tage (bl.a. hvordan man må bevæge termerne rundt). Så ved at have en puslespil, hvor man løser ligninger bare ved at forsøge at rykke rundt på termer og se hvad der lykkes, må man jo så efterhånden (ret hurtigt) lære, hvilke nogle handlinger kan lade sig gøre eller ej. Naturligvis kunne hvert princip introduceres, før det tages i brug, og muligvis med en forklaring om, hvorfor handlingen er gyldig, men dette kan man også bare få at vide senere. Men det er jo så stadig op til brugeren løbende at huske mulighederne til at gennenfører de de videre "puslespil" (i form af ligninger). Appen behøver slet ikke være begrænset til ligninger. Alle opgave typer, hvor elever kan have det med at sidde fast og ikke vide, hvad de skal gøre, og hvor de gerne også skal lære at se de mønstre, der fortæller dem, hvilken handling er oplagt, alle sådanne opgaver kan man inkludere. Dette kunne f.eks. også være når det kommer til at vælge formler (samt passanede kendte vinkler og kanter) til at udlede en ny vinkel kant (ja, og man kunne også i princippet fortsætte videre til mere komplicerede geometriopgaver). Så geometri-opgaver ville altså også være oplagte. Der er selvfølgelig ingen grund til at begrænse appen til de opgaver, der ligger allermest læringspotentiele i ift.\ denne læringsidé med at lære ting ved at få det i hånden (på app-skærmen) og prøve sig lidt frem (og løbende lære mønstre at kende, så man hurtigere kan finde den korrekte handling). Man kunne også sagtens tilføje aritmetik-"puslespil" (som så måske bare føles lidt mere som puslespil og mere som quizzes og/eller normale regneopgaver) for så at gøre appen mere komplet (og så selvfølgelig overveje, hvordan man evt. kan gøre tingene på en måde, så brugeren bliver trænet i at finde brugbare mønstre --- måske ved at give et spin på nogle af de mere simple regneopgaver, ved at fokusere på forskellige brugbare teknikker omkring dem og/eller brugbare tænkeværktøjer til at løse sådanne opgaver). Men appen kunne dog sagtens bare starte kun med ligningsløsning, hvilket nok ikke ville være så svært at programmere. 

%**(22.11.21) He, denne idé er faktisk mere værd.^^ Men ja ja, det må bare blive ude her i kommentarerne. Jeg vil lige nævne en ny idé omkring dette, jeg fik for nogle dage siden: Jeg tror også, det kunne være en smart måde, at lære addition og subtraktion på, hvis man ser det visuelt på skærmen i form af klodser, der er stablet på hinandne --- og med en lineal/højdemåler ved siden af dem, så man kan se højderne. Brugeren kan så få vist to stabler (og muligvis også negative stabler, enten ved at stablen går ned under gulvet i stedet, eller ved at de er en anden farve --- måske alt efter den ønskede indstilling), og hvor brugeren så skal trykke på den ønskede højde for en tredje stabel. Når brugeren er tilfreds kan denne så klikke 'svar' --- eller muligvis kunne man også bruge en indstilling, så brugeren også lige skal vælge det rigtige tal i bunden ud af mulige svarmuligheder (hvis man gerne lige vil sikre sig, at de abstrakte tal også kommer ind i billede, så brugeren bliver tvunget til at danne en fortolkning af disse tal), og hvis det så er forkert, jamen så må brugeren bare forsøge igen indtil højden passer. (Og hele princippet er så, at lysten til hurtigt at klare "puslespillet" så giver motivation for at lære at få en forståelse for, hvordan man hurtigt rammer de rigtige højder --- og måske kunne man endda have en tidscore, så man vinder flere point/stjerner/smiley'er, jo mere derudaf det går.) 


%**(14.02.22) Jeg kom i tanke om noget forleden, som jeg lige vil nævne her en gang. Jeg tænkte på, at vi jo næsten i fremtiden (den nære) må kunne få byggeri, der er bygget med en masse små moduler, som så er nemme at skille ad og bygge om til noget nyt (meget ligesom lego). Jeg tænker bl.a. på brug af en slags interlocking mursten og plader, som er ligeså holdbare og rigide, men som nemt alligevel kan skilles ad, hvis man bare gør det én ad gangen. Hvis man så vil have sin bygning bygget om, så kan det jo godt være, at man lige må rive tapet ned osv., men når bare skelettet er ombyggeligt, så tror jeg, det kunne være nice. Forestil dig f.eks. så, at man gerne vil sætte sit hus sammen med en ven (eller at man gerne vil flytte fra hinanden igen). Tanken er så, at så kan man bare ligesom lægge sine (modulære) bygningsressourcer sammen (eller opdele dem igen). Og hvis de altså er meget alsidige, og hvis alle interfacerne (i interlocking mekanismerne) passer godt sammen, så kan disse byggematerialer jo følge en hele livet i princippet (eller hele virksomhedens liv, hvis vi snakker dette). Og ja, tanken er nemlig lige netop, at interfacerne i høj grad skal passe sammen. Så det er ligesom den grundlæggende idé, men så kom jeg også i tanke om noget andet interessant, nemlig at der så kunne være rigtigt smart, hvis man også samtidigt kunne få det sådan, at alle bygninger, som så i øvrigt kan bygges med et 3d-program, også samtidigt vil have en procedure for, hvordan tingene skal samles, som det tilhørende program altså selv skal kunne udregne, når man designer sin bygning. Og hvis materialerne ligesom så samtigigt indeholder en slags skinner eller lignende, som robotter af mindre størrelse kan køre på, så er min (lidt sci-fi-agtige) tanke altså, at programmet dermed kan udregne en fuld protokol for, hvordan sådanne robotter skal køre rundt og samle bygningen brik for brik (og tilsvarende hvordan de så skal skille den ad igen (helt eller delvist)). Jeg forestiller mig altså så en virksomhed, som sælger disse specielle (alsidige) ``byggesten'' og samtidigt også udlejer specielle robotter, som automatisk kan samle og skille alle bygninger ad (hvis de er designet på en ordentligt måde --- hvilket nævnte 3d-program, som virksomheden så også kan sælge, altså bør kunne tjekke). Synes disse tanker lige var værd at nævne.^^ Hm, jeg har det som om, der var en ting mere.. ..Nej, det var vist det.:)



\subsection{Øvrige tanker om bl.a.\ lykke, etik og evolutionsteori, og også om det fremtidige samfund generelt}

Jeg vil nok ikke skrive så meget til denne sektion. Jeg vil bl.a.\ bare lige nævne, at jeg har nogle tanker omkring lykke; at det er noget, der i høj grad skabes ved at vi opnår at indgå i gode sociale relationer, hvor man gør mange ting for hinanden og hele tiden sørger for at prøve at give hinanden glæde og lykkelige stunder. Lykke er nemlig i høj grad noget, man giver til hinanden, og ikke helt så meget noget, man kan finde på egen hånd. Heldigvis er alt dette noget, vi også kan blive meget klogere på i fremtiden, og en god fremtid vil være en der i høj grad undersøger, hvad der skaber lykkelige samfund, og prøver at skabe forskellige sådanne. Det skal vi nok finde ud af.

Jeg har i øvrigt før brugt termerne (i gamle noter for lang tid siden) `normer' og `værdier' som noget der henholdsvis skal opfyldes for ikke at bringe os ringeagtelse og for at bringe os velanseelse og lykke. Selvom man måske ikke skal bruge lige disse begrebssemantikker, så kunne det måske stadig være værd at fokusere på disse to drivkræfter, og så prøve på at sørge for, at det er de værdiskabende drivkræfter, der mest er tilstede i det pågældende samfund, man prøver at indrette. 

Og apropos normer så tror jeg vi vil have stor glæde af i fremtiden at kunne formalisere vores normsystemer mere og mere, så folk aldrig vil være i tvivl i dagligdagen om, hvad man bør gøre for ikke at træde andre over tæerne, hvad der i givet fald ville være konsekvenserne ved det, hvis man gjorde det, og også hvornår andres handlinger bør irettesættes eller tolereres. Jeg tror at dette kan blive en realitet ved, at man altid har hurtig adgang til at opslagsværk og/eller en AI, der lynhurtigt kan svare på sådanne spørgsmål. Således kan man bare vedtage alle sociale normer osv.\ i en demokratisk ontologi, og derefter bliver det så ret nemt for folk at sameksistere på en god måde, hvis de ønsker det --- og hvis man så er uenig med de gennemsnitlige holdninger, så må man jo bare flytte til et alternativt samfund (hvilket vil være let nok, tror jeg, får jeg forudser nemlig, at fremtidens globale samfund kommer til at bestå af en masse relativt små lokalsamfund, så folk let kan finde nogen de passer sammen med, og så man også lidt kan eksperimentere med alle mulige forskellige systemer på det globale plan). Alt dette gælder i øvrigt i lige høj grad, hvad end vi taler om lokalsamfund i den virtuelle eller den fysiske verden. 


Og når nu verden således vil blive opdelt i så mange lokalsamfund, så tror jeg i øvrigt, at en rigtig vigtig kilde til glæde, gavn og lykke bliver, at disse samfund arrangere events og festivaller m.m.\ for hinanden, hvor medlemmerne i lokalsamfundet så kan være med til at arrangere og forberede og eksekvere disse events i højere eller mindre grad alt efter lyst og behov. Det gør ikke noget, tror jeg, at der så bliver en tværgående økonomi ud af dette, hvor lokalsamfund tjener penge på hinanden eller lignende, så at mere aktive samfund, der tiltrækker og servicerer mange turister, også bliver prioriteret højt på gæstelisterne til events fra andre samfund. Men det kan man jo alt sammen finde ud af. Jeg syntes dog det var værd at nævne det her billede med en fremtid, hvor de fleste menneskers primære arbejdsfunktioner (især i en fremtid, hvor så meget sikkert vil være automatiseret) nok bliver at arrangere (m.m.) events og festivaler osv.\ for hinanden. Kunstneriske og sportslige arbejder vil selvfølgelig også blive en del af et sådant post-scarcity-samfund, hvad end det er forbundet med et specifikt event eller ej.

Og hvis man nu er bange for, at vi som mennesker indeholder psykologiske parametre, der altid vil stå i vejen for at danne et lykkeligt og stabilt globalsamfund, jamen så vil jeg bare pointere, at vi også kommer til at få kortlagt menneskets psykologi mere og mere. Vi kommer altså nok simpelthen til at kunne regne på disse ting ret præcist, og det vil derfor ikke blive nær så svært at komme disse tendenser i forkøbet, som man ellers kunne tænke sig.


Angående retssamfundet så kommer teknologien omkring løgndetektion sikkert til (måske endda inden for en overskuelig fremtid, hvis vi er heldige og gode) at udvikle sig så meget, at det ikke bliver nogen sag at finde frem til, hvad folk er skyldige og ikke-skyldige i. Og hvis vi så bare (``bare'') lige for udviklet vores politiske system nok til, at der heller ikke rigtigt er fare for, at dette vil udnyttes for folk i magthavende positioner til at bevare denne magt, så vil dette jo være en skøn teknologi at have. 

*Jeg kan i øvrigt også lige tilføje, hvis nu man bekymre sig om den politiske og/eller psykologiske side af et fremtidigt samfund, at de fleste mennesker altså nok vil leve en stor del af deres tilværelse i VR-verdner, og så er der altså særligt ingen grund til at blive i et lokalsamfund, hvor man ikke føler sig til rette, eller i en (VR-)verden, hvor der ikke er ressourcer nok, for så vil løsningen jo bare være at anmode om at skifte til en anden server. 

*Jeg føler også lige, at jeg bør understrege nogle ting fra det ovenstående, for det har alligevel fået meget få sætninger i dette notesæt ift., hvor vigtigt det er. Hm, lad mig egentligt bare gentage det for det første: ``At [lykke] er noget, der i høj grad skabes ved at vi opnår at indgå i gode sociale relationer, hvor man gør mange ting for hinanden og hele tiden sørger for at prøve at give hinanden glæde og lykkelige stunder. Lykke er nemlig i høj grad noget, man giver til hinanden, og ikke helt så meget noget, man kan finde på egen hånd. Heldigvis er alt dette noget, vi også kan blive meget klogere på i fremtiden, og en god fremtid vil være en der i høj grad undersøger, hvad der skaber lykkelige samfund, og prøver at skabe forskellige sådanne.'' Den første del af dette udsnit er en påstand, jeg går og tror på, og som jo kan undersøges nærmere i fremtiden (og kan holdes op mod nutidige teorier for den sags skyld; det skulle næsten undre mig, hvis der ikke er en teori, der siger det samme --- og måske er der også en teori, der inkludere mine tanker omkring ``normer'' og ``værdier,'' hvad ved jeg?). Den sidste del er den, jeg særligt gerne vil understrege. Jeg håber endda, at vi ret hurtigt, og altså inden for en forhåbentligt rigtig nær fremtid, vil begynde i langt højere grad at analysere, hvad der skaber lykke, og særligt hvad der skaber lykkelige samfund (for jeg tror nemlig virkeligt at lokalsamfundene bliver nøglen). Jeg håber dermed også, at vi snart får en kultur, hvor der er meget mere fokus, både i samfundet generelt og også, når det kommer til folks individuelle mål i livet, på at konstruere diverse mere eller mindre nytænkende lokalsamfund, hvor der kan eksperimenteres med forskellige norm- og værdisæt --- i.e.\ lokalsamfund med klare mål for, hvad der skal være normerne, som altså på en måde ikke er lykkeskabende i sig selv, men kan ses som de ting, man sætter sig for at overholde, så der kan blive grobund for de lykkeskabende ting *(som eksempel kan det halv-paradoksalt nok skabe mere overordnet lykke, hvis man afstår fra en lykke-kilde i en vis tid, fordi det så kan opbygge glæden, når man så endelig får eller får gjort den ting; dette ser jeg som eksempel på en gavnlig norm), og hvad der så skal være værdierne og kilderne til lykke i det pågældende samfund (hvilket jo i øvrigt så gerne skal en mangfoldig mængde, så man har mange forskellige lykkekilder, hvilket giver plads til mere mangfoldighed i lokalsamfundet (hvilket er essentielt for vi kan jo ikke alle være ens), men også gavner hver enkle, fordi en mere alsidig lykke-pallet vil holde længere). Jeg tror at dette fokus og denne udvikling virkeligt vil være vigtigt, selv her i den rimeligt nære fremtid, for jeg tror for det første, at vi kan komme ret langt på relativt kort tid, hvad lykke angår, når vi først går ordentligt til emnet --- og især når vi også har gode diskussions- og analyseværktøjer i fællesskab --- og for det andet vil det også kun blive gavnligt for den teknologiske udvikling, hvis folk så generelt kommer til at finde sig bedre tilpas i deres personlige liv. 


%*(17.10.21) Jeg overvejer at tilføje noget mere om lykke, men jeg kan jo lige så brainstorme lidt over det her først, og så se, hvad jeg vil tilføje. Ved ikke om jeg vil starte på dette i dag, eller om jeg venter til (nok) i morgen...
%Tja, for det første kunne man måske sige noget mere om det her med, at lykke er noget, man giver til hinanden. Ja, det jeg særligt lige kan nævne, er altså, at forskning og idéudvikling omkring at skabe lykkeligere tilværelser for folk derfor i høj grad bør fokusere på, at strukturere et fællesskab, hvor man analyserer / har analyseret sig frem til, hvilke lykkekilder der er. En rigtig vigtig overordnet lykkekilde i sådan et fællesskab/samfund (og måske altså et specielt udtænkt lokalsamfund (dog uden at det er helt lukket, og hvor det heller ikke er sådant et vildt dogmatisk et; i starten især vil det være rigigt smart, hvis sådanne lokalsamfund spænder godt sammen med "den virkelige verden," altså den virkelige globaløkonomi, for der går jo lang tid, før vi når til et frit post scarcity-samfund, hvor man dermed ikke behøver at forvente bidrag fra de små lokalsamfund til det globale fællesskab)) vil så formentligt være, hvor meget man føler sig værdsat i de ting man gør, og de ting man bidrager med (..og ting man siger osv. osv.) af andre fra fællesskabet. Jeg tror på, at dette er en virkelig vigtig omstændighed, når det kommer til lykke, og en stor grund til, at lykke er så flygtig for os at opnå, nemlig at lykke i høj høj grad kommer fra, hvor meget man føler sig værdsat i den, man er, og det, man gør. Og denne værdsættelse skal jo, 1), komme fra andre og, 2), føles ægte for os. Derfor er det en rigtig svær ting at opnå, og i øvrigt også en rigtig svær ting at "designe" sig frem til. Men jeg tror dog bestemt på, at man kan gøre en del som lokalsamfund/fællesskab for at opnå dette, nemlig ved simpelthen at gøre folk i gruppen meget mere klar over, hvor vigtigt det er ofte (aktivt) at vise værdsættelse til andre i fællesskabet. For når man først bliver vant til dette i høj grad, jamen så skal det nok også komme til at føles meget naturligt, og der skal nok være nok tilfælde at tage af i løbet af hverdagene, hvor man har ægte værdsættelse for andre at vise. Især hvis altså oven i købet samfundet er godt designet, som at folk hver især \emph{har} ting at bidrage med, roller at udfylde, ting at fremvise, ting at være for hinanden (og derudover også har nok personlige mål og hobbyer at forfølge, så at individerne ikke kun hviler i deres indbyrdes roller over for hinaden, men også i deres individuelle, personlige glæde-/lykke-kilder, hvilket altså så vil give overskud til også at vægt på og flid i de indbyrdes roller og på de handlinger man giver i og til fællesskabet). Og jeg mener bestemt ikke noget, at man som individer hver især er klar over alle disse forhold. Jeg tror således ikke man skal være særligt bange for, at det så kommer til at føles kunstigt, f.eks. når man viser værdsættelse, bare fordi denne værdsættelse ligesom er "planlagt" på en måde. Jeg tror altså ikke, man behøver at være bange for, at det hele kommer til at virke "planlagt." For medmindre noget er helt galt (hvilket jo kan ske, og så må man prøve at finde et andet fællesskab at leve i), så vil sådanne følelser komme inde fra, hvis ikke lige i starten (hvis man altså har en fase først, hvor man skal vende sig til tingene), så sikkert meget hurtigt. Personligt kan jeg huske, at min efterskolelære, Erlend, i starten fik snakket til os om, hvor vigtigt det jo er at give komplimenter, når man har dem, og jeg kan huske, at det helt klart virkede positivt (jeg føler da, at jeg selv blev lidt bedre, og det tror jeg også andre gjorde) på fællesskabet. Så ja, jeg er glad for at jeg lige fik knyttet nogle flere ord omkring disse ting. Jo, mange mennesker kan have gavn af positiv psykologi rettet imod individet selv, men der hvor det virkeligt kommer til at batte, det er, når vi bliver i stand til at overveje positiv psykologi mere i fællesskaber og, ja, gøre folk mere opmærksomme på, hvordan man indbyrdes giver lykke til hinanden. Og når vi for økonomisk og ressourcemæssigt frihed nok i vores samfund til ligefrem at kontruere små lokalsamfund, hvor sådan nogle ting kan udleves i høj grad, så tror jeg virkeligt, det vil begynde at batte.
%Jeg har vist så forklaret lidt om, hvordan jeg kunne forestille mig sådanne lokalsamfund, men lad mig bare lige brainstorme over det her kort også. For det første mener jeg jo, at vigtige projekter, som man altid vil kunne give sig til, selv i et overflodssamfund, hvor der ellers ikke er brug for så meget "arbejde," vil være at afholde festivaller og andre events, hvor folk fra andre fællesskaber (men også fra fællesskabet selv selvfølgelig) kan komme og deltage og nyde godt af de ting og begivenheder, som er klargjort og planlagt. Sport og spil vil selvfølgelig også generelt være en ting, hvor man kan finde glæde individuelt, og hvor fællesskaber kan interface meget med hinanden på en god måde. Kunst, musik, drama vil jo også altid være en ting (og dette kan jo også udstilles bl.a. til sådanne events), og det vil indretningsdesign også. Uddannelse og læring vil også blive en stor glædegiver på dette tidspunkt; man kan i det hele taget bruge mange elementer fra efterskoler højskoler i sådanne fremtidssamfund, bare hvor det nu er den voksne tilværelse vi snakker om. Og alt dette kan krydres med, at man laver systemer, hvor man løbende kan vinde priser og point, og på den måde have det sjovt med at udfordre sig selv og hinanden --- og andre fællesskaber --- med forskellige ting. Og ja, hvis det er nogle foretagner, hvor man også kan mærke, at man udvikler sig som person, så er det også kun en fordel, for dette kan der også ligge rigtigt meget glæde og tilfredshed i. 
%Ah, jag har det godt med, at jeg lige fik nævnt disse ting. Er der mere, jeg skal skrive? Nå jo, måske at man selvfølgelig skal kunne skifte imellem fællesskaber i løbet af ens liv. I den forbindelse mener jeg også, at fællesskaber med fordel kan have forskellige profiler, både ift., hvilke aktiviteter man forventer at bruge krudt på, og.. Ja, og det kan også være hvilke normer og værdier man har i gruppen, og ellers kan det i øvrigt også være ift., hvilke persontyper fællesskabet gerne skal bestå af. Fællesskaber kan altså godt designes, så at personerne passer godt til hinanden. Man skal så selvfølgelig bare sørge for på et mere globalt plan, at alle persontyper kan finde et godt sted at høre hjemme, og at ingen bliver ladt helt ude i kulden. 
%Angående det med at skifte fællesskab i løbet af sit liv, så minder dette mig om, at man måske godt kan have restriktioner på, hvornår f.eks. børn må flytte hjemmefra osv., og det mindede mig så i det hele taget om et andet punkt, nemlig hvornår det kan være okay at sætte restriktioner i et fællesskab. Det jeg lidt vil ind på her er mine "normer" (altså med den ret præcise definition, jeg bruger på begrebet). Med andre ord: Vil det være fornuftigt at have begrænsninger på folk impulser og på folks glæder, fordi dette så i sidste ende kan føre til større lykke samlet set, også for selve individet? Kort sagt: Ja; ja, det kan nemlig godt føre til større lykke samlet set, og derfor ja, det kan det være. Jeg tror faktisk også på nogle punkter der er nogle omstændigheder ved vores nuværende (generelle) levemåde, hvor vi godt kan være mere opmærksomme om at begrænse os. Her tænker jeg særligt på børn. Jeg tror der er flere punkter, hvor vi kommer til at tillade nogle lidt for lette dopaminkilder for vores børn, som ikke er så gavnlige i længden, fordi det vender børnene til at prøve at opsøge lignende former for dopaminkilder i fremtiden. Jeg tænker bl.a. på ikke at overstimulere børn med glæden ved at få materielle ting, f.eks.\ seje/søde ting i lysende farver, som de kan blære sig med overfor kammeraterne, fordi det så vender børnene til, når de bliver ældre at opsøge glæder på denne meget basale materialistiske form. Det er dog ikke noget jeg skal gøre mig klog på, jeg ville bare lige indskyde tanken her. Jeg har bestemt også nogle frygte omkring stort forbrug af tablets for børn (som bl.a. omhandler at kunne fantasere sig til ting selv i stedet for at legen bare er forprogrammeret, men der er nu også mange andre ting ved det), men ak, der er jo sikkert tusindvis af andre mennesker, der har tænkt de samme tanker, så det behøver jeg vist heller ikke sige mere om her. Jeg kan dog lige sige, at love og regler, som begrænser beboerne (og måske i nogen tilfælde endda friheds"berøver" beboerne (bl.a. børn, der måske ikke har lov til bare at flytte væk fra deres forældre)) i nogen grad i visse begrænsede tilfælde kan være ok, hvis bare folk får muligheden for at flytte væk, når de bliver myndige nok, og hvis de løbende får valget igen og igen. En særlig ting, jeg har haft i tankerne før, er hvis nu det kunne give mening (ift. beboernes lykke) at afskærme lokalsamfundet meget fra resten af den globale verden. Dette kunne f.eks. være, hvis man gerne vil diesigne et samfund, hvor teknologien ikke er den samme som i det globale samfund, men hvor beboerne ikke skal gå og blive misundelige på det de ser andre udenfor samfundet have. Det kunne også være, hvis man er i en VR-verden, hvor beboerne ikke skal mindes om i løbet af tilværelsen, at tingene ikke er så "virkelige" som i den "virkelige verden." Det kunne også være, hvis man har et samfund, hvor der ikke bare er overflod af ressourcer, og hvor folk derfor er oprigtigt afhængige af hinanden og af, at folk ikke bare rejser deres vej lige pludselig.. Hm, man kunne sikkert finde et væld af andre eksempler.. Men det korte af det lange er altså bare: Nej, dette er ikke uetisk, hvis bare man følger nogle klare etiske retningslinjer (som vi til den tid må udarbejde i globalsamfundet), hvor man sætter protokoller for, hvordan folk skal få mulighed for at slippe ud og rejse væk fra sådanne lokalsamfund (hvad end dogmerne er kraftige eller ej). Og for samfund, hvor en del af afskærmningen handler om at afskærme viden, så må man jo bare implementere en rød-pille-blå-pille-agtig protokol, hvor individerne altså løbende kan præsenteres for et valg om at blive i deres verden eller at træde et skidt ud og få udvidet deres viden om omverden en anelse, hvorved protokollen kan igen kan spørge vedkommende, om denne vil gå længere ud eller gå tilbage. 
%Cool! Det blev jo lidt sci-fi her til sidst, men altså stadig nogle virkeligt gode tanker lige at få skrevet ned på denne eftermiddag. ^^ (18.10.21) 

*(18.10.21) Jeg har lige nogle ekstra noter ude i kommentarerne her (i kildekoden) omkring fremtidens lykkelige (forhåbentligt) samfund, som man kan læse, hvis man vil. :)
%(Nemlig kommentarerne omkring denne paragraf; over og under.)

%(20.10.21) Ah, og noget andet, jeg lige bør nævne også, er at samfund sikkert med fordel kan designes sådan, at folk får flere og flere muligheder og friheder, og flere og flere glædeskilder i løbet af livet. Og også gerne på en måde, hvor individet så endda får en følelse af at \emph{opnå} disse nye muligheder. Der kan f.eks. være visse prøvelser, man skal bestå (men uden nødvendigvis at man bare kan speedrunne dem, hvis man er rigtig god; det må nok gerne ske med visse minimumsintervaller imellem sig), før man kan gå op til et nyt trin i livet. Det behøver ikke at være sværre prøvelser overhovedet --- måske kan sværheden endda tilpasses individet --- men man skal bare have følelsen af et \emph{opnå} det næste trin. Jeg tror nemlig at rigtigt meget lykke også kommer fra følelsen af at opnå ting i livet --- og faktisk ikke mindst også fra et løbende arbejde, der så vil være ved at vedligeholde det, man har opnået; så længe at man bare føler sig sikker på, at man er i stand til dette vedligeholdelsesarbejde. Så jeg mener altså, at der kan ligge rigtigt, rigtigt meget lykke i for det første at opnå nye ting i livet (om så disse "skridt" er mere eller mindre designet, eller om de er en del af en ikke-designet tilværelse (som vores nuværende)), hvilket så er en mere kortvarig, men stadig stor lykkefølelse, og også især at have følelsen af at efterfølgende at udføre et godt arbejde, hvor man vedligeholder det opnåede. I vores nuværende samfund kan sidstnævnte især være "arbejdet" i at vedligeholde sociale relationer (så når jeg siger "arbejde," så mener jeg altså bare aktive handlinger for at pleje ens (og muligvis andres) livssituation). Og hvad man går og "arbejder" for i de mere designede fremtidssamfund kan også \emph{sagtens} bl.a.\ være ting som at opretholde sociale relationer (bare fordi tilværelsen i en eller anden forstand er lidt "designet," så behøver tilværelsen ikke af den grund være opfyldt meget af "kunstige" forhindringer/udfordringer; den eneste forskel er bare, at man analyserer tilværelsen, og prøver at sørge for, at der er disse lykkekilder til stede i en eller anden form). Og noget andet, som så også kan være grund til, at tilværelsen kan opbygges i skridt (på mere eller mindre "kunstig" vis (og i øvrigt mener jeg heller ikke her, man skal være alt for bange for det "kunstige" i det, for følelserne skal nok blive ægte, også selvom individet ved, at smafundet omkring dette til en vis grad er designet til at være, som det er)), er simpelthen også, at folk, mener jeg, ikke vil få den samme lykkefølelse fra ting, hvis de tidligere har været vant disse ting eller bedre ting. Hvis man f.eks. starter sit liv i luksus og slutter sit liv i mere spartansk, så vil man ikke få nær så meget lykke som for den omvendte rækkefølge. Og hvis man f.eks. starter sit liv spartansk og slutter med middel luksus, så kan dette også give ligeså meget lykke, som hvis man starter med middel luksus og slutter høj luksus. Vores lykkefølelse ved ting er altså i høj grad afhængig af, hvad vi har været vant til før, og hvordan den relative progression af, hvad vi opnår i løbet af livet, forløber. Og særligt tror jeg altså bestemt på, at man vil opleve et lykkeligere liv, hvis man starter mere begrænset (og endda muligvis meget begrænset) og så løbende opnår bedre og bedre fysiske og sociale forhold omkring sig, end hvis man bare opnår disse ting lynhurtigt og så bare bruger resten af ens liv i dette samme stadie. Så det at lykkefølelse kan afhænge af, hvad vi har været vant til, giver altså endnu en grund til, at det kan være fordelagtigt at tænke "trin"/"skidt" ind i tilværelsen, når man analyserer og designer lokalsamfund/fællesskaber. Ah, det var nice også lige at få dette med! ^^


%**(07.01.22) Jeg kan lige tilføje nogle få ting. For det første vil jeg gerne tilføje, at jeg jo i den nære fremtid håber meget, at "civil-foreningerne" også kan spille en stor rolle i at finde frem til måder at forbedre folks tilværelse på. Og særligt håber jeg, at man i en meget nær fremtid vil begynde at fokusere meget mere på at indrette sig i lokale fællesskaber. Det bliver rigtigt godt, når først folk kan finde ud af at organisere sig i bofællesskaber, så man kan blive rigtig nær de personer, der passer sammen med en, og som man så kan forestille sig at dyrke relationer med igennem lang tid (og typisk faktisk et helt liv). Sådanne lokale fællesskaber, som så vil være en slags bofællesskaber oveni, kan så indgå i større fællesskaber, hvor man så kan have et sammenhold på tværs af de involverede bofællesskaber om at arrangere alle mulige ting sammen, f.eks. sport og spil, klubber og arrangementer.. Alle mulige ting. Og jo mere niche noget bliver, så kan man bare ligesom gå længere op i fællesskabsniveau (for der kan være mange niveauer: et bofællesskab, et (bo)fællesskabsfælleskab, et fællesskabsfællesskabfællesskab osv.).. Det samme gælder også, hvis man vil konkurrere på et højere og højere plan med noget: Man kan starte med at konkurrere i sit ejet fællesskab og så ellers gå højere og højere op. Men ja, og jeg håber altså, at folk ret hurtigt vil indse, at sådanne ordninger kan blive en kæmoe stor kilde til, at folk bedre kan finde rod og finde lykke i den nære fremtid. Og i øvrigt så håber jeg jo også bare, måske også inden man når til at samle sig i bofællesskaber (osv.), at civil-foreningerne kan være en stor kilde til at finde frem til fællesskaber man passer godt sammen med (også inden at man ligefrem flytter tæt sammen), og at de også kan være en god kilde til at finde/opdage, dele og undersøge, hvad kunne være gode råd ift. at forbedre ens livssituation og ens tilværelse generelt (hvad ofte, mener jeg, ideelt set vil kræve mere en bare handler fra individet selv, men vil kræve en fælles handling fra individet og resten af den gruppe, som individet ønsker at finde lykke med --- for lykke er langt nemmere at finde i fællesskab end selv; det er mere noget vi giver til hinanden end noget vi opnår selv). Og ja, det korte af det lange er altså, at jeg håber (og tror!), dette kan blive en realitet i en nær fremtid også (og forhåbentligt allerede så snart "civil-foreningerne," eller hvad der svarer til, kommer op og køre).
%Nå, det næste jeg så også gerne lige vil skrive kort om, og som jeg ikke rigtigt ved, hvor ellers jeg skal skrive, er lidt om tanker omkring opdragelse. Jeg vil bare lige dele, at jeg synes jeg lagde mærke til nogle ting omkring min opdragelse, som jeg synes var rigtig gode, og som jeg gerne selv vil prøve at efterligne til den tid. Og det er, at jeg synes, mine forældre var gode til generelt ligesom at tage os, mig og min bror, med på hele opdragelsesopgaven.. Jeg synes jeg blev gjort meget bevidst om, at det var en proces, og at man ligesom skulle lære at håndtere forskellige ting som barn for at få en god opdragelse. Og jeg fik meget hurtigt meget stor tillid til, at mine forældre havde det bedste i sinde for os, og at når vi skulle rette os efter noget, så var det for vores eget bedste. Desuden så hjalp det også helt sikkert, at vores forældre gav os og viste så meget kærlighed for os. (De havde i øvrigt også ventet længe på os (og arbejdede begge to med børn i øvrigt; var lærere begge to), så vi var i den grad højt ønskede.) Og dette er nemlig også noget, jeg selv vil prøve at gøre dem efter. Sørge for virkeligt at give mine børn meget kærlig, og så samtidigt også give dem forventninger. Og jeg tror så, efter egen erfaring (for sådan følte jeg det), at når børn virkeligt kan mærke, at forældrene elsker dem og vil dem det bedste (og sætter sine børns behov rigtigt højt), så vil de også blive meget lydhøre over for irettesættelser. Jeg kan huske, at noget af det sidste jeg ville, det var at skuffe mine forældre, som jeg elskede så højt. Og hvis man så fik en sur mine fra ens far, som vi nogen gange gjorde, så skyndte man bare at rette sig efter dem. Hm.. Og til det vil jeg bare lige sige, at jeg synes godt nok meget man støder på ude i offentligheden, forældre, der bare ikke har særligt meget overskud til deres børn. Ja, én ting er, at forældrene man møder ofte \emph{langt} hellere vil side med hovedet inde i en fucking skærm, end at de vil engage med deres unger. Og selv simpel kommunikation er jo så vigtig for børnenes udvikling.. Nå, nu bliver jeg lidt en gammel mand.. Men ja, noget andet er, at man også bare tit støder på mangel af overskud. Og tit bliver dette formuleret direkte: "Jeg orker det ikke" eller jeg "gider det ikke," i stedet for: "Nej, ikke sådan og sådan, fordi sådan og sådan." Og selvfølgelig er der bare mange forældre, der vitterligt ikke orker deres børn ordentligt, fordi de har et hårdt liv, men derfor kunne man måske stadig godt gøre dem opmærksmomme på, at børnene faktisk har langt bedre af, hvis de bare for en anelse mere velvijle og interesse sendt i deres retning. .."Nej, du må ikke lægge varerne op i posen, for det tager for lang tid for dig," i stedet for lige at bruge få sekunder ekstra på at barnet kan få lov at prøve noget, der faktisk er interessant og lærerigt for det (selvom det kan virke så simpelt og ikke-lærerigt for en voksen..). Nå, måske kom jeg lidt af sporet.. Hm nej, det var fint nok. Jeg tror altså på, at man ville kunne få meget ud af, at lære folk, og altså nybagte forældre især, mere om, hvor vigtigt det er at engage på en god måde med et barn og at vise interesse hver eneste dag for barnet og dets læring og udvikling. Nå ja, og noget andet, som man tit hører, er: "Fordi jeg siger det." Dette kan ofte være ret uskyldigt; den har jeg da også selv fået som barn, hvis ikke af forældre så af pædagoger. Men jeg tror nu alligevel, at hvis man tager sig selv i at sige "fordi jeg siger det," og hvis man har hjernen til det (og det burde ikke kræve særligt meget), så brug da lige en lille stund på at filosofere dig frem til, jamen \emph{hvorfor} er det, at barnet ikke skal/må gøre sådan og sådan? Er det for barnets eget skyld? Og i så fald, hvorfor? Og når det så er gjort, så kan man prøve at tænke over, om der er en måde at kommunikere dette til barnet på en simpel måde næste gang. Og hvis de så har spørgsmål til det, man forklarer, så udnyt da bare dette til så netop, at tage dem med på deres egen opdragelse (som jo var mit første punkt her), og prøve at forklare dem lidt nærmere om, hvad det er for nogle grunde, der er til at barnet genrelt ikke skal gøre sådan og sådan. Dette kan måske også gøre det nemmere (hvad ved jeg?) næste gang, for så kan man måske lære barnet, at når jeg siger sådan og sådan, så er det altså der og der for, og så behøver barnet måske ingen gang rigtigt at spørge næste gang. Så ja, når jeg forhåbentligt engang bliver forælder, så vil jeg da prøve på altid at se barnets spørgsmål, både som noget ærligt (og jeg tror nemlig på, at børn tit er ret ærlige, når de spørger; at det er fordi, der er nogle rammer, de simpelthen er i tvivl om), og som ikke andet end en god mulighed for at forklare ting til barnet. Nå, men det kan jeg jo sagtens side og sige; tiden vil vise, om jeg vil kunne finde ud af dette altid. ..Hm, var der ellers noget, jeg ville sige..? Selvfølgelig hører det også med, hvis ikke dette var klart, at jeg virkeligt vil prøve at sørge for at vise stor kærlighed og interesse i mine børn.. Ellers noget..? Hm jo, jeg vil også lige fremhæve ærlighed som et godt nøgleord, jeg vil gå efter. Og dette hænger så bl.a. sammen med ikke at sige "bare fordi jeg siger det" osv. Jeg vil prøve, har jeg i sinde, at være så åben og ærlig som muligt for mine børn.. Ja, og dette hænger jo netop også sammen med det at tage dem "med på deres egen opgragelse." Jeg vil således (prøve at) være åben omkring, hvad deres "opdragelse" er for en størrelse, og ligesom få dem med på hele idéen, i stedet for bare at gøre det til noget, som kun den/de voksne skal tænke på. ..Og selv hvis jeg gør nogle manipulerende ting, så at sige, så vil jeg også bare være åben omkring disse teknikker efterfølgende.. Min far var også god til ikke at sætte for mange mure op omkring os, og bl.a. så forsøgte han ikke at sige "Pas på, det er farligt!" hele tiden. I stedet sagde han gerne f.eks., "nu falder I selv i vandet, ikk?" (når vi gik på stenene i havnekanten f.eks.). Og dette koncept var han også meget åben omkring, og hvorfor ikke? Hvorfor ikke være åben omkring de "tricks," man ligesom bruger som forælder? Selv hvis det direkte involverer en løgn, så kan man jo altid bare røbe sandheden og forklare hvorfor nogle dage (eller måske måneder eller år alt efter, hvad der er tale om) efter. Ok, det var vist det, jeg lige vil sige. Bare lige nogle tanker om opdragelse, og om specifikt hvad jeg umiddelbart vil prøve at sigte efter, når jeg forhåbentligt engang bliver forælder. Og det skal selvfølgelig bare tages med et gran salt det hele: Jeg skal nok nå at blive meget klogere, hvis og når jeg selv får børn. 

%**(18.02.22) Jeg fik for lidt tid siden den idé, at jeg lige ville prøve at tilføje en lille reklamerende tekst for mine "lokalsamfund"/bofællesskaber. Så det gør jeg lige her: Forestil dig en fremtid, hvor mennesker generelt har muligheden for at vælge et bofællesskab, hvor folk har tilsvarende interesser, der passer på ens egne. Forestil dig dermed, at du er omgivet af mennesker, der har lyst til at dyrke de samme ting. Det kunne være diverse sportsgrene eller sportslige aktiviteter, det kunne være teater, kunst, musik, dans eller andre ting, enten med mål for øje at optræde med det eller på et mere stille og roligt plan. Det kunne også være littere interesser, videospil, popkultur, musikinteresser, film og andre medier. Det kunne også være interesser i at lave events, holde fester, fejre traditioner, og være værter for gæster fra andre fællesskaber med overlappende interesser, enten for at holde fest eller fejre noget sammen eller måske for at dyrke en specifik interesse sammen. Det kunne også være arbejdsrelaterede interesser om at bygge eller skabe ting sammen, og det kunne også være om at bygge skabe noget kreativt sammen mere som en hobby/beskæftigelse *(også inkl. sådan noget som at lave mad og/eller deserter, brygge drikke, designe tøj eller andre modeartikler eller lave/bygge/designe hvad end ellers, det kunne være). Selvfølgelig vil alle i et bofællesskab aldrig lige have samme interesser (hvilket også er sundt, for hver person vil jo gerne have sin hel egen identitet), men pointen er bare, at alle personer i fællesskabet føler, at de i høj grad omgiver sig med folk, de kan dele og dyrke deres interesser sammen med. Og forestil dig så i øvrigt også, at disse mennesker, du omgiver dig med, også har lignende livsfilosofier omkring, hvordan man opfører sig, og hvordan man er overfor andre i forskellige sammenhænge. Her vil der selvfølgelig også altid være variation i virkeligheden, men pointen er igen bare at finde sammen med et fællesskab, som deler de samme holdninger og tilgange til ting i høj grad som en selv. Sikke en utopi, hva? Og det kan blive endnu bedre, hvis vi også forestiller os, at folk har, hvad de skal bruge af ressourcer og mad osv. i høj grad, og at teknologien også er så udviklet, at folk med ro i sindet bare kan koncentrere sig om de beskæftigelser og interesser, de har i dette bofællesskab, Man kunne således forestille sig, at økonomien på det tidspunkt i virkeligheden mere bare kommer til at handle om, netop at holde mange gode events osv.\ for nabo-/venne-fællesskaber, sådan at man dermed (udover at få glæden ved sine egne events) også vil blive inviteret til mange events osv. selv. Og for at sætte prikken over i'et, så forestil dig videre, at denne fremtid også har en stor adgang til et højt udviklet Web 2.1/3.0, som jeg forestiller mig det, hvor det særligt er muligt i høj grad at søge og få råd omkring alt muligt, man selv kan bruge i dagligdagen, enten som enkeltperson eller som fællesskab. Vi kunne således snakke om at søge inspiration til nye aktiviteter eller interesser/beskæftigelser, eller det kunne også være ting såsom at søge råd til konflikløsning eller på, hvordan man får et bedre forhold/sammenhold i gruppen. Der skal sikkert nemlig også være rigeligt med interesserede fællesskaber, der har lyst til at beskæftige sig med netop dette, og som altså endda ville have lyst til at gå ind og coache andre fællesskaber direkte (og altså høre om og prøve at finde løsninger til deres problemer specifikt). Og ja, som nævnt kunne dette også være på et individuelt plan, hvor man søger råd til nye interesser/beskæftigelser eller til at blive bedre i sine sociale relationer og til at forbedre disse. Nu ligner det virkeligt noget! Det er da en fremtid, der er værd at sigte imod! Det synes jeg, og.. (hvis vi lige går tilbage til 2.-persons-reklamesnakken en gang: xD) Det gør du måske også?;) He, og min afsluttende kommentar bliver så, at jeg faktisk tror at denne fremtid muligvis kunne være meget nært forestående endda, hvis altså vi bare gør os umage nok. Hvis vi arbejder nok for det (også til dels politisk og/eller som forbrugere osv.), så tror jeg på, at vi endda godt kan nå noget tæt på denne drøm (i hvert fald i den vestlige verden..) inden for vores egen levetid (i.e. min og din, hvis du læser dette i, hvad der i skrivende stund er (omkring) nutiden). 




Om etik vil jeg bare lige nævne, at hele spørgsmålet om, hvad der fundamentalt set er etisk, og hvad der ikke er, ikke er nær så svært at svare på, som det lidt bliver gjort til. Jeg tror en af grundene til, at spørgsmålet har virket så svært for mange, nok er, at folk både blander konceptet om personlige etik-/moral-regelsæt og konceptet om, hvad man bør stræbe efter som samfund, sammen. Hvis man spørger til personlige regelsæt først, så gør man det efter min mening helt forkert. Det er nemlig dumt at se spørgsmålet på et individuelt plan, før man har besluttet sig for, hvad der er rigtigt og forkert, når det kommer til, hvad vi gerne vil opnå i verden som fælleskab. Først må man svare på sidstnævnte, og så kan man derfra begynde på at overveje, hvad der så bør betragtes som rigtigt og forkert som individer, hvis man skal agere på en god måde i et samfund, der stræber efter en vis ting. For hvad der er rigtigt og forkert som person må jo i sidste ende afhænge af, hvad man ønsker at opnå i verden som fællesskab. Samtidigt bør man i øvrigt også skelne mellem, hvad der er de rigtige ting at gøre, givet at aktøren er i stand til, har lyst til og kan få sig selv til at gøre, hvad der er helt korrekte etisk, og så hvilke nogle regelsæt en given aktør \emph{faktisk} bør forsøge at holde sig til. De to ting er nemlig ikke ens i praksis. Man kan nemlig sagtens komme ud for --- ja, faktisk som reglen snarere end undtagelsen --- at en aktør ikke vil være i stand til at følge de eksakt etisk korrekte pga.\ egeninteresse, og hvad mere er, at forvente at aktøren prøver at bryde med sin egeninteresse kan selv have negative konsekvenser, hvad det overordnede mål for samfundet angår. Hvis man således forventer at folk skal følge principper, der går meget imod deres natur, så kan dette jo føre til ringere lykke for samme, hvilket jo naturligvis kan være imod den samlede målsætning. Og dette må man altså endeligt ikke se bort fra, når man diskuterer emnet! Ja, tvært imod: et samfund der ikke formår at opstille et system, hvor deltagerne kan handle på etisk vis, uden at det hele tiden (mere eller mindre) kommer til at stride imod deres natur, vil simpelthen bare ikke være grundigt nok udtænkt så. 

Man bør altså først derfor prøve at finde frem til, hvad man bør prøve at stræbe efter som samfund/civilisation, og derefter kan man så prøve at implementere en løsning i fælleskab, som sigter mod disse mål. Denne implementation kan så godt indebære, at folk og instanser for tildelt etiske regelsæt (/moralkodekser), som de forventes at følge, men disse regelsæt/moralkodekser skal så tage højde for, at selve individerne/instanserne også har egeninteresser, det kan medføre negative konsekvenser (ift.\ fællesmålsætningen) at brydes for meget med. Næste punkt er så rent faktisk at svare på dette spørgsmål: Hvad bør man stræbe efter som samfund? Jo, det er der også et simpelt svar på. Efter min mening bør man simpelthen forestille sig, at man bliver reinkarneret én gang som alle eksistenser (antaget at alle eksistenser i universet har en lige høj frekvens af sjæle, se eksistensteori-sektionen ovenfor) i hele universet, og at det altså dermed i alle henseender er i ens egeninteresse (hvad end man rent faktisk tror dette er lige gyldigt; man skal forestille sig det uanset hvad) at stemme for et system, der bedst muligt hjælper disse eksistenser bedst muligt samlet set. Ved at dykke ned i eksistensteori (se mit afsnit af samme navn ovenfor, selvom det godt nok indtil videre står en smule kryptisk) kommer man endda ret let frem til, at noget helt tilsvarende faktisk også \emph{må} være gældende i princippet for vores univers, nemlig at vi i alle praktiske henseender må genfødes i nærmest uendeligt mange variationer og dermed også alle variationer imellem de eksistenser vi selv observerer omkring os (i.e.\ vores medmennesker i vores tilfælde). Men uanset hvis man ikke tror på dette, så må dette stadig fungere som det mest fornuftige grundlag for at opstille fællesmålsætninger i et samfund. Man skulle nemlig nok være ret uærlig, hvis man som ledende gruppe i globalsamfundet besluttede sig for (efter en filosofisk analyse), at nogle menneskers (eller eksistensers) lykke ikke er helt ligeså vigtige som andres (og hvor de mest vigtige eksistenser så ``tilfældigvis'' er stærkt repræsenteret i denne ledende gruppe). Så ja, jeg mener altså, at vi i sidste ende må komme frem til dette `reinkarnationsprincip,' som jeg kalder det, som vores etiske grundlag for samfundet i fremtiden, hvorfra man så kan prøve at implementere samfundsmæssige systemer for at opfylde dette princip. 

*(06.06.21) En lille tilføjelse angående reinkarnationsprincippet, som muligvis kan gøre spørgsmålet en smule mere diskutabelt, kan nu findes sidst i eksistens-afsnittet ovenfor. I bund og grund handler det bare om, at det endnu ikke er givet, om man så skal se det som at man lever alle liv én gang eller uendeligt mange gange (lige efter hinanden), og hvis man beslutter sig for at antage en statistisk blanding af disse to (hvad man sikkert må), så kan det altså så blive spørgsmålet, hvor meget man skal vægte hver af de to muligheder. Dette kan f.eks.\ få konsekvenser for, hvor meget man konkluderer at folk skal have lov til at leve lange (og med tiden mere og mere kedelige) liv, endda selv i princippet også selvom de ikke er i vejen for andre. Så det kan altså få betydning, men jeg tror, man (med meget stor sandsynlighed) nok skal finde frem til et fornuftigt svar på dette forhold (og sandsynligvis også uden nogen ekstreme eller halv-ekstreme konsekvenser på det). 



Og angående evolutionsteori vil jeg bare lige nævne, at vi altså godt lige kan irettesætte de tilhængere kin selection, der ikke tåler, at man anser tingene fra et gruppeperspektiv. Vi skal ikke ligefrem tilbage til naiv evolutionsteori, hvor lemminger eller whatever ofrer sig selv for flokken, men den efterfølgende kin selection-bølge (for ikke at tale om dem der påstår, at det er bedst at se tingene fra et genperspektiv) har vist været en overreaktion, som jeg ser det (fra min dog ret lægmandsagtige position --- så tag denne paragraf med et gran salt). 
%Jeg skal forresten lige huske at læse op på det der eksempel, som har hørt flere steder, og som jeg ved min bror har lært om *(på biologi-bachelorstudiet i Århus for et år eller to siden), hvor konklusionen er at arbejderbier (eller var det myrer?) selektere deres sterile søskende af et vist køn frem for andre: Hvis jeg virkeligt har hørt rigtigt (og det har jeg jo nok ikke så, for ellers ville være lidt for dumt (men bør nok alligevel tjekke bare for en sikkerheds skyld)), så holder dette jo på ingen måde, for inclusive fitness-parameteren forøges jo med lige præcis nul, når man hjælper sterile søskende (da nytte i evolutionsøjne altid udregnes ud fra den statistiske forøgelse af afkom).


Nå ja, jeg havde nær glemt også at nævne noget om evolutionspsykologi. Der er et emne, jeg har haft en meget stor kærlighed til. Jeg tror ikke længere help på, at det kan forklare nær så meget, som jeg gjorde engang, men jeg tror stadig, det er et emne, hvor man kan komme langt længere, end hvor vi er nu. Jeg tror således, at man ville have gavn af for det første at gå helt væk fra tanken om et stenaldermenneske, der pludselig befinder sig i nye omgivelser, det slet ikke er vant til. Vores adfærd bygger nemlig ikke på basale instinkter og reflekser, sådan som mange evolutionspsykologer eller har malet billedet, men vi er tvært imod væsner med meget abstrakte og komplicerede bevæggrunde, og hvis adfærd er rigtig fleksibel, så vi kan tilpasse os effektivt til vores omgivelser vi fødes ind i som fænotype, så vi bedst muligt --- eller rettere så godt som vi kan; vi er bestemt heller ikke \emph{perfekte} til opgaven (hvilket i øvrigt både er på godt og på ondt) --- kan opnå at tilgodese de bevæggrunde. For det andet kunne man også godt tillade inkludere gruppeselektion-tanker noget mere i overvejelserne, når vi teoretisere om vores psykologi --- ikke at man skal glemme, at det individuelle selektionspres selvfølgelig har været højt i vores udvikling, men der har også været et betydeligt pres på os i form af kampen mellem grupper, hvilket bestemt har været betydende (for folkefordrivelser og folkemord (i hvert fald delvise, men det indgår vist allerede i begrebet) har bestemt forekommet --- i hvert fald så vidt jeg ved som lægmand, hvilket jo dog skal teges med et gran salt). 

Nå ja, og noget andet er netop, at gruppeselektion faktisk på en måde kan blive mere betydende, når den kan forstærkes af andre kræfter internt i en gruppe. Her tænker jeg særligt på det seksuelle selektionspres internt i en gruppe, og på den interne sociale struktur generelt. Ganske simpelt kan faren for ydrer fjender eller faren fra omgivelserne generelt gøre, at en gruppe vil opsætte sociale (hierarki-)strukturer, der fremmer altruistiske handlinger frem for dovne og/eller selviske handlinger (inklusiv ikke-handlinger). Når først sådan et socialt system er fastlagt, så vil folk ikke have individuel gavn af at bryde for meget med normerne, for det vil resultere i ringeagtelse for gruppen, hvilket vil få en til at falde i det sociale, og det seksuelle, hierarki. Så sociale grupper kan på denne måde opnå systemer, hvor de kan få de individuelle drifter til at arbejde \emph{for} gruppens fællesinteresser om overlevelse og ikke imod den. Lad mig understrege, at pointen her altså er, at når først et system er dannet, så vil det ikke kræve særligt meget energi for gruppen at opretholde (vi kan jo selv tænke os til forskellige fortidige samfundsstrukturer, hvor dette gælder, og hvor systemet altså har været ret stabilt overfor de individuelle drifter). Vi kan se det som en slags Nash-ligevægt, hvor man så som samfund opnår, ved at nudge den rette ligevægt igennem, at de individuelle drifter kommer til at arbejde til fællesskabets gode. En særlig joker er så også i denne forbindelse, at der jo er en lille ting kaldet det seksuelle selektionspres også. Dette vil i sin natur være / have været lidt sværere at nudge og styre i den rigtige retning som gruppe end den overordnede sociale struktur i gruppen. Men man kan påvirke den. Man kan således skabe en kultur i sit samfund, hvor man ser ned på kvinder eller mænd med visse seksuelle præferencer / mage-præferencer. Så disse præferencer kan altså også lægges under for normer, hvilket så kan forstærke gruppen sociale kohæsion yderligere. En anden ting, der dog også er værd at nævne, hvilket især er interessant for ikke-sociale dyr --- og også for sociale dyr, der bare ikke har de samme kognitive egenskaber som os, hvilket altså så vil sige alle dyr andre end mennesker --- er at selv uden muligheden for at udvikle normer omkring seksuelle præferencer (hvilket i dyrenes verden jo er lig deres ``mage-præferencer''), vil altruistisk adfærd i gruppen kunne fremmes af interne selektionskræfter, nemlig af det seksuelle selektionspres alene. I denne forbindelse er det så mere bare trial and error, når det kommer til en gruppes overlevelse, fordi de så bare må være heldige, at de seksuelle trends kommer til at passe med fællesinteresserne og ikke, i værste fald, omvendt. For menneskegrupper kan held omkring seksuelle trends i øvrigt også spille ind, men her har vi altså også haft yderligere redskaber til at nudge disse trends i den rigtige retning så at sige. Men igen gælder der altså, at vi dyregrupperne kan danne en form for Nash-ligevægt, som kan formå at sætte de individuelle selektionskræfter lidt i skak, når det kommer til de modsatrettede kræfter, der kan være i forbindelse med det individuelle selektionspres og presset fra gruppeselektionen. En ind-til-benet slægtskabsselektions-tilhænger (en ``al gruppeselektionstænkning er naivt, fordi jeg siger det!''-person (og nu overdriver jeg selvfølgelig lidt med vilje her som et retorisk virkemiddel; jeg er ikke 100 på, at disse personer faktisk findes (jeg er kun lægmand, og jeg var endda ret ung, da jeg virkeligt interesserede mig for dette emne), men hvis de gør, så må de da nemt kunne overbevises om, at gruppeselektion faktisk er et vigtig konceptuelt værktøj til at anskue og forstå tingene i mange sammenhænge)) ville sige, at gruppeselektionens kræfter aldrig rigtigt vil kunne overvinde den individuelle selektions kræfter, og at tilsyneladende altruisme derfor kun kan opstå som en slags bivirkninger til f.eks.\ sociale forhold eller pga.\ seksuel selektion. Jo, det er på sin vis også sikkert rigtigt nok, men de glemmer så bare lige, at de grupper, hvor (Nash-)ligevægten svinger en anden vej så vil uddø, eller i det mindste undertrykkes, i genpølen, og dermed \emph{er} der så (pr.\ definition) tale om vaskeægte gruppeselektion. Og bemærk i øvrigt at selve en arts evne til at nudge disse sociale og seksuelle forhold i den rigtige retning, hvilket som nævnt om ikke andet kan gøre sig gældende for sociale dyr, kan i sig selv også undergå (gruppe-)selektion, hvilket kan forstærke tendenserne til at udvikle altruisme i diverse dyre- eller menneskesamfund yderligere. Jeg er glad for, at jeg lige kom i tanke om at føje dette til notesættet her, for det er som sagt et emne, jeg har haft en meget stor kærlighed til (næsten på linje med min kærlighed for fysik på det tidspunkt), så det er rart lige at få det med også. *Nå ja, og det at arter overvejende har udviklet kønnet formering frem for ukønnet er også et klart eksempel på netop dette. Det kan \emph{kun} forklares med gruppeselektion (medmindre altså man ad bagveje forklarer det via individuel naturlig selektion, og dog så bare argumenterer ud fra gruppeoverlevelse (men ligesom ud fra individets synspunkt), hvortil man kan sige: hvis man har lov til det, jamen så er der ingen diskussion i første omgang; så er gruppeselektion en okay anskuelse af tingene). Og det er altså også et eksempel, netop hvor der skabes et system, som de individuelle selektionskræfter ikke så let kan bryde igen, og hvor gruppeselektionskræfterne så formår at gøre disse systemer udbredte. Hm, her er det dog hverken et system af seksuelle præferencer eller sociale forhold men simpelthen et system af rent fysiologiske forhold. Her kan man så i øvrigt teoretisere omkring, lige for at få en ekstra tangent med, om ikke så selve forholdet, at kønnet formering giver anledning til seksuelle præferencer, som jo videre kan være med til, hvis jeg har ret, at stabilisere systemer, således at gruppeselektion i endnu højere grad kan få tag i artens udvikling (som så kan være med til at sikre dens overlevelse), om alt dette ikke dermed kunne være en ekstra faktor, der har været med til fremme kønnet formering. Nemlig at systemet, der er kønnet formering, ikke bare fremmer gruppens (i dette tilfælde artens, for grupper med forskellige reproduktive egenskaber vil være (pr.\ definition stort set) forskellige arter) overlevelsesegenskaber pga.\ større genetisk fleksibilitet i gruppen (for bl.a.\ at kunne overleve sygdomspresset samt ændringer i omgivelserne), men også pga.\ efterfølgende forhøjede muligheder for gruppeselektionen at få endnu mere indflydelse over gruppen. Det er disse sjove tanker, hvor forskellige selektionspres kan indvirke på hinanden i forskellige sammenhænge på kryds og på tværs, og endda også i flere lag så at sige, der bl.a.\ i høj grad var med til at få mig til at elske emnet så meget i sin tid. :) (Og selvom effekten selvfølgelig bliver mindre og mindre for hvert lag, og derfor måske ofte må konkluderes at være ret lille, hvis overhovedet tilstedeværende, så er det stadig, synes jeg, super interessant at overveje sådanne nogle mulige sammenspil.) 

*(16.09.21) Uh, man burde faktisk næsten lave en simulering a la due-høg-simuleringerne, men hvor man tilføjer seksuel selektion i billedet\ldots! Jeg er næsten sikker på, at en sådan simulering ville vise, at når man tilføjer seksuel selektion på en måde, så at dyrene i civilisationen har en præference-parameter, der kan mutere og variere, og som afgør om dyret er mest tiltrukket af andre due- eller høge-typer, så vil due-typerne vinde frem i højere grad, end hvis man ikke havde den parameter med, og især også hvis man inddeler simuleringen i områder med begrænset (men kun en smule) migration på tværs af områderne\ldots\ Hm, men kan man så ikke argumentere for, at det ikke er gruppeselektion, der er på spil\ldots? Tja, det kan man vel, hvis man også ser samme resultat helt uden opsplitning af området\ldots\ Hm, men vil seksuel selektion virkeligt drive de seksuelle præferencer væk fra hinanden, hvis de to relevante ``grupper'' er i tæt nærkontakt med hinanden? Hm, godt spørgsmål. Jeg tror bare jeg vil lade spørgsmålet stå her for nu og fortsætte med mine andre ting, men altså ikke udkommentere denne paragraf foreløbigt.

%*(16.09.21) Hm, det er egentligt ret trivielt at argumentere for, at gruppe selektion eksisterer og har været relevant for at form liv på jorden, selv hvis man ser bort fra mine pointer om at tage seksuel selektion og/eller intra-gruppe-systemer, hvor medlemmerne kan opnå mere eller kan miste ``social status'' inden for gruppen (hvilket kan betyde noget ikke bare for parringsmulighederne men også for mulighederne for at opnå føde, ly, og beskyttelse osv.), ``i brug,'' så at sige, for at opnå grobetingelser for altruisme indenfor gruppen. For man må jo kunne blive enige om, at der i tidens løb har været mange dyregrupper, der har splittet op og udviklet sig forskelligt --- og muligvis til forskellige arter, men ikke nødvendigvis så langt. Og videre vil der så have været tilfælde, hvor grupperne er blevet bragt sammen igen pga.\ færre ressourcer eller forandringer i miljøet eller så videre, hvor grupperne så ikke bare har samlet sig til én gruppe igen (af forskellige årsager, bl.a.\ noget så simpelt som at de seksuelle præferencer kan have udviklet sig i forskellige retninger), men har endt med at kæmpe om pladsen essentielt set. Og ved mange af disse lejligheder vil vindergruppen altså blive den, der havde en fordel, fysisk eller adfærdsmæssigt set, der gjorde at de kunne udkonkurrere den anden gruppe. Ligeledes vil der være eksempler, hvor hver af de to grupper har mødt udfordringer i hver deres område, imens de har været splittet, og hvor det så har hændt, at den ene gruppe uddøde, imens den anden gruppe overlevede og til slut var i stand til at overtage første gruppes territorium. Også i sådanne tilfælde vil den overlevende gruppe ikke have været helt tilfældigt udvalgt, men vil have været en gruppe, der har været mest modstandsdygtig overfor forandringerne. Dette kan ofte forklares med naturlig selektionsbrillerne på, men ikke rigtigt i tilfælde, hvor den ene gruppes bedre overlevelsesdygtighed skyldtes en tilfældighed set med NS-brillerne på. %..Hm, måske, men bliver det ikke stadig også bare et spørgsmål om semantik i sidste ende?..:\ ..Pas. Jeg vil også lige lade denne gamle traver ligge for nu..
%(Tidligere forsættelse fra "[...] overfor forandringerne":) Godt nok kan sådan modstandsdygtighed overfor omgivelserne også komme via naturlig selektion, men der er alligevel en forskel på de to selektionskræfter. Kort og godt er den ene selektionskraft ligeglad med det samlede antal af individer, der har de relevante gener, men indvirker kun på at fremme frekvensen, hvor det andet selektionspres, fordi der er noget at vinde som gen ved at overleve miljø-udfordringer... Hm, nej dette holder vist ikke..
%Hm nej, man ender jo stadig ud i en strid om semantik, så disse argumenter går vist ikke.. ..Jeg synes bare, at ting som kønnet formering og... Hm..






%Nå ja, og så kan jeg også lige nævne hurtigt, at man jo også må kunne udvikle kunstige lunger, kunstig mavesæk (eller tilsvarende) og kunstige nyrer/lever osv.\ så man kan få sig en fuld backup, hvis nogen af ens organer skrider ud. Hm, men hvad er det rigtigt værd før man så... Tjo, hvis ellers ens krop er intakt, så kan dette jo bruges, men hvis ikke, så bliver denne teknologi først rigtig anvendelig, når vi også enten kan koble et interface til rygraden (f.eks.\ allerede i starten af nakken, så man kunne blive en cyborg kun med hovedet intakt) eller kan koble et interface til hjernen på anden vis, så folk kan komme til at bevæge sig igen (enten som delvis robot (cyborg), som fjerntyret robot-avatar eller som VR-avatar). Men ja, så der er alligevel nok lidt vej til denne teknologi, og da det også er rimeligt trivielt udkommenterer jeg bare denne paragraf igen.


%**(28.12.21) Her om aftenen d. 23/12 (så for nogle dage siden nu) kom jeg til at tænke på: Hvorfor fryser vi ikke bare vores hjerner ned og gemmer dem til eftertiden. På et eller andet tidpunkt i fremtiden \emph{må} vi da (hvis civilisationen altså når så langt og ikke går til grunde inden da) få teknologi, der kan scanne (måske vil at scanne og fjerne et lag af celler af gangen, hvem ved?) en frosset hjerne, læse cellernes kemiske tilstande og så gendanne den, måske digitalt. Så ville man hermed få mulighed for at opleve fremtiden (og nyde at gå på opdagelse i den og prøve forskellige ting), og fremtiden vil så få mulighed for at høre historier fra fortiden. Og der vil helt klart blive efterspørgsel på at genoplive folk fra gamle dage, så man kan møde og snakke med dem. Det vil jo svare lidt til at kunne få besøg af turister fra et fremmedartet land, men bare hvor det er fra fremtiden. (Og i øvrigt skal denne efterspørgsel jo ganges med, hvor lang tid det fremtidige samfund varer, så man skal da helt sikkert nå at blive vækket til live igen mindst én gang.) Så hvorfor fryser vi ikke bare vores hjerner ned, eller dem, der gerne vil opleve den fjerne fremtid. Selvfølgelig kan det være, at man dør som gamel og i en tilstand, hvor man ikke får så meget ud af det, men på den anden side kan det jo sagtens være, at teknologien i fremtiden vil kunne reversere alderdommen i hjernen i så fald. Og så kunne man altså prøve lige at få lidt ungdommelig vigør tilbage samt opleve en eksotisk fremtid (og få lov at fortælle sine historier!). Der må da være en kæmpe mulighed her, som vi har overset. Ja, og dette vil da i øvrigt kunne blive noget af et forretningsforetagende, hvis jeg ikke er den eneste, der er så begejstret for den tanke. Jeg ved at de nutide kvaler er, at man ikke har teknologi til at fryse hjerner(/hoveder) ned, uden at cellerne sprækker, og de bliver lidt til mos. Men hvis man som sagt bare skal scanne de kemiske tilstande i cellerne, så gør det da ikke noget, at cellevæggende er revnede! For mig virker dette som en rigtig spændende idé, og en som jeg er lidt spændt på at fremføre.





%Noter:
% - "fremtid med turisme osv. (og prestige...)." (tjek)
% - Mere formelle sociale normer --- hvor man også bedre kan introducere frihed inden for normerne mere formelt. (tjek)
% - Løgndetektion. (tjek)
% - Psykologien (videnskaben) udvikler sig også mere og mere. (tjek)
% - "Vi kan aldrig forvente, at folk går imod deres natur;" men formentligt behøver vi det ikke (det er så derfor, vi må benytte gode systemer i stedet). (tjek)
% - "Reincarnationsprincip hvad angår lykke/velfærd." (tjek)
% - Evolutionsteori. (tjek)




















%\subsection{Opsummering over de ting, jeg vil arbejde videre med nu (06.06.21)}
%
%Nu vil jeg så lige skrive nogle idé-skitser til nogle af de idéer, jeg tror har mulighed for at skabe interesse omkring sig. Jeg tror jeg vil skrive dem på engelsk. Tanken er, at jeg så også har noget skriftligt af give med, når jeg skal forklare om idéerne, så jeg ikke er tvunget til at forklare det hele verbalt (og så lytterne ikke skal huske det hele på én gang). 






%\subsubsection{En mere semantisk side a la Stack Overflow (skitse-noter)}
%
%Jeg vil i første omgang skrive en idé-skitse over min idé omkring en mere semantisk Stack Overflow-*(/Git-)side. Jeg vil forklare lidt om fremtidsmulighederne for matematisk/semantisk programmering, som jeg bl.a.\ har nævnt i dette notesæt (selvfølgelig ikke det hele, men bare lige nogle udvalgte gode pointer). *(Hm, måske vil jeg egentligt mest bare fokusere på mulighederne omkring semantisk dokumentation, og særligt ift.\ mulighederne dette kan bringe med sig for bedre at udlicitere arbejdsopgaver og arbejde sammen om projekter online, samt hvordan jeg tror en masse gængse platforme kan blive bedre, hvis de involverer brugerne noget mere --- og gør brugerfladerne mere åbne og justerbare. \ldots\ Ja, og så tror jeg faktisk jeg vil formulere teksten, så jeg bare inkluderer tanker om, hvordan en Stack Overflow- og/eller Git-side så kunne ændres (eller opbygges fra ny), så den/de kommer til at kunne bære et sådant fællesskab, der altså kommunikerer programmeringsløsninger med semantisk dokumentation (og gerne via en ontologi, som jeg foreslår det). Og så kan jeg forresten altid bare pointere som en sidebemærkning, at det måske kunne være værd at se på, om man skulle prøve at bygge en ny side/platform beregnet på denne type programmeringsvidensdeling.) 
%%\ldots\ Hm, måske skulle jeg så dele det op, så jeg først laver et afsnit, hvor jeg slår et slag for et paradigme med større brug af semantisk dokumentation? Ja, det tror jeg, jeg vil gøre. Så omstrukturerer jeg også denne sektion (og udkommenterer dette).) %Nej vent, måske giver det ikke mening at splitte det op..
%
%Og til spørgsmålet omkring, hvordan vi så når til det punkt, vil jeg påpege, at man ved at skifte ITP-paradigmet en anelse, måske så faktisk kunne komme ret hurtigt og stærkt i gang med hele denne semantiske teknologi. Jeg vil således foreslå, at starte med et formelt logisk sprog med formelle og simple inferensregler så faktisk starter i den nærmest helt uformelle ende, og lader alle konklusioner udføres på baggrund af pointsystemer udviklet og varetaget af et onlinefællesskab, hvor brugere simpelthen giver pointvurderinger til korrektheden (samt muligvis også andre kvaliteter såsom bl.a.\ entydigheden) af bevisskridt/argumenter i de uploadede beviser. Dermed kan man sige, at processen i bund og grund kommer til i starten at svare meget til, hvordan matematik udvikles i dag, nemlig ved at et fælleskab arbejder sammen om at tjekke bevisskridt manuelt --- og hvor man så også implicit har mere tiltro til nogen kilders udsagn om korrektheden frem for andre --- bare hvor al denne kommunikation så foregår online, og hvor der som en særlig del af platformen indgår, at brugerne selv er frie til at benytte deres egne pointsystemer, når vedkommende skal vurdere tillid til andre brugere og ultimativt til korrektheden af bevisskridt. Selvom man derved kan starte med vilkårligt lav grad af formalitet i sine beviser i fællesskabet, skal sproget hvori man udformer sine argumenter dog stadig bygges på et formelt logisk sprog som fundament, ikke mindst fordi fællesskabet så derved løbende for mulighed for at præcisere og formalisere individuelle bevisskridt/argumenter mere og mere, så de måske i sidste ende kommer til at bestå af formelle logiske inferensregler. Da jeg vil foreslå et refleksivt logisk fundament, hvilket jeg vil forklare nærmere om nedenfor i \textbf{Mine idéer omkring ITP (skitse-noter)}-sektionen, behøver disse inferensregler dog ikke kun bestå af de helt simple af slagsen, før man når til et punkt, hvor bevisskridtet også kan verificeres automatisk; de kan sagtens være komplicerede regler. Herved får vi altså så en Stack/Math Overflow-agtig side, dog med et åbent pointsystem, som brugerne selv kan definere og bruge alle mulige versioner af, og hvor det så er muligt på sigt at formalisere diverse argumenter og bevisskridt mere og mere, så det beror mindre og mindre på vurderinger fra brugere, der har gennemgået dem, men så de i højere grad kan vises automatisk ud fra nogle mere grundlæggende antagelser (og i sidste ende matematiske aksiomer, når man når så langt ned). Jeg håber så, at det i den forbindelse vil være til generelt at formalisere diverse argumentationsformer og bevismetoder, så man inden for en overskuelig fremtid kan begynde at genbruge sådanne metoder forskellige steder, på en måde så man sparer, at folk skal gennemlæse og verificere korrektheden a de brugte metoder på ny hver gang. Jeg håber altså, at man ret hurtigt vil begynde at kunne formalisere metoder, så folk kan tilægge en vurdering af korrektheden til en given metode én gang for alle og dermed ikke skal vurdere brugen af den i hvert enkelt bevis, den indgår i. Men selv hvis dette ikke bliver tilfældet, og altså hvis brugerfællesskabet stadig skal vurdere hvert bevisskridt/argument for sig, så tror jeg altså stadig, man kan komme rigtigt langt med en sådan internetplatform (plus fællesskab). For hvis en tekst er interessant for mange mennesker, så vil det også være det værd som læsere, at sørge for at de enkle delargumenter i teksten bliver vurderet. 
%%Jeg håber i øvrigt også på, at brugerne i deres egen ende kan få mulighed for at bruge specielle metoder til at autoudfylde kode-templates, når implementere en (matematisk dokumenteret) funktion eller klasse, hvor disse templates så allerede indeholder alle de lemmaer, der skal vises for i sidste ende at vise korrektheden af den udfyldte template. Men også her kan man sige, at dette dog ikke nødvendigvis behøver at blive tilfældet, før at idéen er god; det er den stadigvæk.
%
%
%Et af de store problemer med sider så som Stack Overflow udover den simple form for op-ned-rating er nemlig, at de kun indeholder enkeltpersoners tekster (evt.\ med redigeringer). Her mener jeg altså, at man kan komme langt længere, hvis de endelige tekster, som brugerne i sidste ende kommer til at læse, når de søger efter viden, i langt højere grad er tekster, som brugerne har samarbejdet på at udforme. 
%Og ved samtidigt at gøre pointsystemet helt åbent, så brugerne selv kan udvikle og bestemme over søgealgoritmerne og filtrene for, hvad de får vist, så skal forfatterne altså ikke kæmpe nogen indbyrdes kamp om at få deres version frem; hvis deres bidrag er gavnligt for en vis mængde brugere, så vil bidraget også efterhånden stige en point i de pointsystemer, som pågældende brugermængde benytter samlet set. Lad mig også lige præcisere her, at det så er meningen at tekster skal brydes meget op i semantiske elementer, så hver sætning formelt set får defineret, hvad dets funktion er i teksten; er det et argument for en vis anden sætning i teksten, eller måske en konklusion af andre sætninger. Hver sætning skal så med tiden altså tilføjes graf-kanter til andre sætninger --- eller til eksterne udsagn, som enten er refereret til eller er implicit forstået af læserne (eller rettere de fleste læsere, hvis de har det forventede niveau, og hvis ikke så vil det nemlig være gavnligt med disse eksterne referencer, så alle læsere i sidste ende har mulighed for at forstå teksten) --- som viser disse semantiske sammenhænge. En bruger må dog meget gerne have mulighed for at uploade en ret bar tekst, hvor alt dette semantiske metadata altså endnu ikke er tilføjet, og så kan det altså være op til fællesskabet generelt, selvfølgelig hvis teksten altså er interessant nok, at analysere og kvalificere hele tekstens semantiske opbygning --- og selvfølgelig komme med forslag til rettelser og tilføjelser (også i form af eksterne referencer) osv.\ osv. Og alle disse tilføjelser og rettelser skal altså så kunne rates af andre brugere med forskellige point, og i sidste ende skal en bruger altså kunne få serveret den tekst, inklusiv metadata, der giver så mange point som muligt (selvfølgelig alt efter, hvor grundigt man vil have serveren til at arbejde på forespørgslen (og dermed hvor langt tid, man kan vente på svaret)) i brugerens foretrukne pointsystem (som altså udregnes ved at tage et vægtet gennemsnit af de samlede ratings, hvorved brugeren altså kan vægte visse brugergruppers svar højere end andres (og her er det selvfølgelig også brugeren selv, der får lov at vælge, hvordan brugergrupperne skal klassificeres i sidste ende), og hvor der så efterfølgende tages et vægtet gennemsnit af diverse typer point, som man har inkluderet (så brugeren f.eks.\ får mulighed for at vægte `hurtigt at læse' og/eller `let at forstå som begynder på området' højere end f.eks.\ `detaljeret' eller `omfattende for emnet,' eller hvad det ellers kunne være)).
%
%
%Sidst i sektion \ref{efter_milawa} ovenfor kan man så læse om, ca.\ hvordan jeg nok ville indrette serverne til platformen. Jeg bør for resten lige se på, hvordan man kunne implementere en måde for serveren at servere del-ontologier ud fra brugerens point-system-præferencer med relationelle databaser. Man bør også have en opbygge en underlæggende begrebsontologi til siden, men her kan man faktisk snyde en anelse ved bare at kunne referere til andre online-leksika, så man på den måde allerede effektivt set får en ret dækkende ontologi over begreber (ved altså at brugerne bare kan referere til leksikonerne, når de skal bruge et term i en tekst, f.eks.\ i titlen eller som stikord, og som andre brugere dermed bare kan søge på via leksika-URL'en\ldots\ tja, på den anden side kunne man også bare skrabe pågældende leksika og opbygge en begrebsontologi på selve siden, for så er man ikke afhængige af, at URL'erne ikke skifter).
%
%
%Udover matematik og andre tekniske områder er det så altså særligt programmering, platformen er tiltænkt at handle om (hvorfor jeg referere meget til Stack Overflow som et udgangspunkt). Jeg tror så altså på at man her kan opnå en slags Stack Overflow, hvor de uploadede programmeringsløsninger også har opgivet data omkring sig for, både hvor grundigt de er testede, og hvor grundigt de er læst og analyseret af diverse brugere (med forskellige kvaliteter ift.\ deres aktivitet og stilling af og på platformen).
%
%Da fokusset således i høj grad er på programverifikation (både ift.\ maskin- og menneskeverifikation, og både via kodeanalyse og/eller testing), så ville det jo ikke være dumt, hvis man også kunne benytte fællesskabet til også at udvikle og forbedre server-algoritmerne til platformen. Dette kræver selvfølgelig, at man gør hele systemt ret open source, men det er også min hensigt i forvejen. Platformen kommer nemlig i utrolig høj grad til at være båret af brugerne, så den eneste vej frem vil være, at gøre platformen så venlig og gavnlig, og ikke mindst åben, for brugerfællesskabet som overhovedet muligt. Vil der så være nogen penge at tjene her? Ja. Hvis man brander sig på, og får skabt et godt image af, at man standhaftigt er så brugerfællesskabs-venlige som muligt, og at brugerne derved virkeligt kan stole på en, så vil de også bakke op om foretagendet og vil gerne donere/betale for at sørge for, at man (som firma/organisation) bliver ved med at administrere servicen. Men der er faktisk også endnu flere penge at tjene på denne open source-sti. Det kommer jeg til om et øjeblik. Men for lige at komme tilbage til serveralgoritmerne, så kan man dog altså bare starte med faste serveralgoritmer, og så langsomt åbne mere og mere op for, at brugerfællesskabet kan få indflydelse, når dette modnes nok, og når man selv som firma/organisation er klar. En ting er så, at de forslåede algoritmer skal verificeres (ud fra et point-system som firmaet/organisationen selvfølgelig bestemmer), men en anden ting er at de også skal stemmes igennem af brugerfællesskabet. Her kan der så selvfølgelig være delte meninger og behov, så i den forbindelse bør man lige sætte et demokratisk system op, eventuelt på en måde så de mest betalende/donerende brugere for tilsvarende mest stemmeret. Her har jeg i øvrigt en idé til, hvordan man kunne strukturere en sådan online afstemning, som jeg vil skitsere i \textbf{[...]online afstemning (skitse-noter)}-sektionen nedenfor.
%
%
%Det ville være godt med en hjemmeside, hvor brugerne kan browse alle platformens ontologier, men derudover tænker jeg også, at brugerne skal have et program (helst bare helt open source) til at tilgå serverene på. Ligesom at serveralgoritmerne på sigt kommer til at blive brugerstyret, så bør brugerne selvfølgelig også kunne bygge deres egne browseralgoritmer, så de i deres ende kan præprocessere deres søgningerne for på den måde at kunne opnå større effektivitet i server-browser-interaktionen. Samtidigt vil det også være smart for programmører og matematikere (eller folk der arbejder inden for andre tekniske områder) med en IDE (til hvad end de nu laver), som kan kommunikere med platformen og hente brugbare sætninger, metoder, programmeringsløsninger osv.\ ned, når de er relevante til arbejdet. Jeg ville så foreslå også at basere sådan et program i matematisk logik, og selvfølgelig gerne i den samme logik, som platformen er bygget på. Jeg vil således foreslå samme logiske fundament, som jeg vil skitsere nedenfor i \textbf{Mine idéer omkring ITP (skitse-noter)}-sektionen.
%
%
%Nu vil jeg så vende tilbage til det jeg teasede om, at der er mange flere penge at tjene, end bare ved at administrere en platform med servere, hvor brugerne får meget indflydelse på algoritmerne. Nå jo forresten, inden da kunne jeg måske også lige nævne, at hvis man nu er lidt bekymret for, om brugerne bare vil skifte til en anden serveradministrator, hvis man gør det hele open source, så kunne man måske sørge for bare at tage brugerne endnu mere med på en lytter, også når det kommer til, hvad abonnement-priserne skal være, og hvad arbejderlønningerne skal være, og således faktisk give dem lidt medbestemmelse her. Dette kræver så selvfølgelig en afstemningsproces, som skal kunne foregå digitalt, online --- det kunne eksempelvis være den jeg foreslår her i i \textbf{[...]online afstemning (skitse-noter)}-sektionen nedenfor. Og når dette så er nævnt, så kan jeg her gå videre til at foreslå, at man søsætter en hel IT-virksomhed på baggrund af platformen, som så er beregnet til %at fungere som slags forening af programmører, ... %Hm, hvordan bliver det nu med at sætte betalingsmure op..? ...Hm, skal man overhovedet det, hvis man har en kundedrevet virksomhed?.. Det skal man vel delvist, eller hvad?... Men ja, en del af idéen ved at gøre det lønningerne kundebestemte er jo netop bl.a., at behovet for betalingsmure og lukkethed generelt bliver mindre. Uh, man bør da forresten næsten have mine nye modelprincipper, som jeg fandt på her d. 11/06 (i går), se "*Yderligere tanker omkring kryptovaluta"-sektionen..! Det må jeg lige nævne i ovenstående noter et sted... ..Det er godt nok nogle principper, som særligt er smarte for... nej; jeg skulle til at sige NL-kæder, men man kan også implementere det samme helt fint bare med juridiske kontrakter. ... Okay, nu er det nævnt ovenfor. Angående betalingsmure så skal man vel nok kunne opsætte disse. Så virksomheden/foreningen skal altså kunne sætte sådanne op, hvad jeg førhen har set som en klar mulighed.. Og hvorfor er lige, at dette er en så klar mulighed..? ... Hm, nu fik jeg lige nogle tanker omkring, at bidragere måske kunne overvåge sig selv i arbejdsprocessen, hvis man skal erstatte andre folks bidrag, som de har trukket tilbage igen.. hm, nej for man kan bare terpe... ..Selvfølgelig kunne man give sig selv det dogme, at folk skal hyres i uvidenhed om opgaven, men dette virker så ikke for populære bidrag, hvor der er stor sandsynlighed for, at folk kender til løsningen... Nej, det er jo nok bedre, det her med at folk ligesom bare kan sælge deres bidrag til foreningen.. Ja, og så kunne det netop være mine 11/06-idé, der gør, at man kan forsikre bidragere om bagudbelønning, hvis man føler behov for at kunne dette.. Hm, ja, er det ikke nærmest bare det?.. Ah! Jo, det er det da! Ha, jeg havde næsten helt glemt, hvordan et godt system til bagudbetaling ikke bare skulle føre mere retfærdighed med sig, men også potentielt en hurtig udvikling af et semi-open source-virksomhed --- og særligt ift.\ IT-løsninger..! He, jo, så jeg skal helt klart foreslå idéen i forbindelse med denne programmør-forening-virksomhed. (12.06.21) Og hvad er alternativet så ellers; er det bare, at man som forening/virksomhed køber alle de bidrag, man har råd til (evt.\ ved salg af ens egne aktier)? Hm.. Tja, tjo, men idéen holder da, så jeg behøver vel ikke overveje alternativet, eller gør jeg? Jo, det er meget fint at have med, og der også en fornuftig nok løsning på dette, og det er bl.a., at man benytter brugernes vurderinger til at hjælpe den bureaukratiske proces omkring at bestemme belønninger, også i starten. Man behøver nemlig ikke at vente på, at kontrakterne siger, at man skal gøre dette; hvis det giver mening, kan man bare gøre det forinden, og så er man (som ejere/administratorer af virksomheden) i dette tidsrum bare fri til at justere denne medbestemmelsesmagt og trække den frem og tilbage, som man ønsker, og man er også endda fri til at bestemme sine egne pointpræferencer, ift. bl.a. hvilke brugere, man lytter mest til osv. Vi kan tage YouTube og Twitch som hurtige eksempler, hvor lønninger til skabere bestemmes ret automatisk ud fra diverse "point" (altså views, likes, og hvad har vi) skaberen optjener for en video/stream. I øvrigt kunne man så måske med fordel også offentliggøre... Ja, man kunne med fordel offentliggøre hele regnestykket, og det bør man helt klart, da gennemsigtighed er yderst vigtig her (for bl.a. hele brandet), men ved særligt at offentliggøre, hvor meget... der gives ud pr. tidsinterval i lønninger... nej, never mind; hvis bare man offentliggør regnestykket, hvad man alligevel bør, og måske eventuelt også udgiver en simplificeret model og/eller andre overordnde parametre og estimatorer, så brugerfælleskabet nemt kan få en god idé omkring, hvor store lønninger gives for hvad, jamen så er man jo godt i gang. Om man så (måske kortvarigt) binder sig juridisk til at følge dette regnestykke, så man er sikker på at lokke bidrag til, eller om man bare forsøger at opbygge mere og mere tillid løbende, det må man jo så bare finde ud af. 
%at opkøbe rettigheder omkring brugeres uploadede programmeringsløsninger for så at tjene penge på dem igen ved at sælge og/eller udlicitere brugen af diverse applikationer, som disse løsninger tilsammen opbygger. 
%%Og lad mig allerede nævne nu, for jeg teasede, at der nok kunne være mange penge at tjene på dette: Er det så meningen, at man herved skal forsøge at malke brugerne (som i dette tilfælde så vil arbejde som en slags freelance-programmører for virksomheden) for deres bidrag, og aggressivt forsøge at opnå en så høj mark-up på handlen. Nej, faktisk ikke. Jeg omvendt tror på, at det faktisk bliver virkeligt vigtigt som virksomhedens forretningsledelse, hvis denne idé skal du, at være nærmest så meget på programmørernes side, som muligt, så virksomheden nærmest kommer til mere at være en forretningsforening for programører i stedet frem for en mere konventionel tech-virksomhed. %Hm, og idéen ville næsten også faktisk give mere mening, hvis den var non-profit til at begynde med, så hvordan fletter jeg lige disse to ting sammen, så man også holder muligheden åben for at appellere til investorer (og gå den mere gængse iværksættervej)?.. Tja, det er jo faktisk egentligt et væsentligt spørgsmål det her, for jeg vil jo egentligt gerne bare dele det med så mange som muligt, men så mister idéen muligvis appellen til investorer, fordi det så nok ikke er til at skjule, at en mere non-profit konkurrent hurtigt vil udkonkurrere den mere kommercielle tilgang... Så hvad er overhovedet idéen ved at rette teksten mod investeringsinteresserede..? Hm, jeg har lyst til bare at sige fuck it, men jeg skal nok træde varsomt her, for det ville også være dumt at miste en mulighed. Ja, og da hele min mission nu bliver, at fange folks interesse i et godt omfang, så er det jo farligt at skære muligheder for, netop at fange folks interesse af... Hm, tjo, men interessen omkring idéen står og falder med, om der er mulighed for som iværksætter at tjene penge, jamen så vil man også gøre sig fortjent til penge, hvis man begiver sig ud som iværksætter, for så vil dette jo være risikofyldt. Og så må folk jo også anerkende dette arbejde og denne risikotagning. Hm, men skal jeg så ligefrem argumentere for dette?.. Og sige: vent og se, og hvis non-profit-virksomheder ikke opstår på baggrund af disse idéer, jamen så betyder det, at der er grobund for en kommerciel virksomhed..? Hm, er det ikke sådan, at det netop ikke duer med en non-profit virksomhed, fordi virksomheden gerne skal tiltrække investorer fra starten, så man kan begynde at betale lønninger af tidligt? Jo, er det ikke det, der er (eller var) svaret..? Jo, det må det være. *(Det er ikke helt dumt med en god idé med markedspotentiele, men med mine andre indéer, og især måske mine idéer om blockchain, så behøver jeg vist ikke være vildt bekymret for, om mine idéer vil sprede sig (og hvor er det dejligt at kunne sige det.. 7, 9, 13..). Men ja, så jeg bør altså nok, som nævnt, fokusere teksten mere i retning af: "hey, hvad med et nyt paradigme inden for programmering (m.m.)?!")
%%Ja, fordi virksomheden kommer til at blive så brugerdrevet og brugerstyret, som jeg forestiller mig, den gør, så ville en tilsvarende non-profit-virksomhed nok hurtigt kunne udkonkurrere den kommercielle udgave, hvis ikke det var fordi, at virksomheden også vil være meget afhængig i starten af investorer, så den kan betale for diverse brugerbidrag, der skal opkøbes rettigheder til. Og fordi man altså så kommer til at gå denne balancegang alligevel mellem at være en virksomhed henholdsvis for brugerne/programmørerne og for investorerne, så vil der trods alt, mener jeg, også blive en rimelig god skilling at tjene både som iværksætter og som iværksættende investor. Dette er dog nok det sidste, jeg vil nævne angående disse forhold omkring ... Hm, alt dette må kunne formuleres kortere og bedre.. (Jeg udkommenterer altså lige dette.)
%For at kunne give freelance-programmørerne løn skal man så selvfølgelig gerne kunne tiltrække investorer først. For iværksættelsen af virksomheden udgør jo en risiko, og uanset hvordan man vender og drejer det, så vil disse investorer, hvem de så end er, forvente en vis fortjeneste, hvis planerne lykkes. Dermed kan man nok ikke (i hvert fald hvad jeg ved af) satse på, at starte en helt non-profit-virksomhed. Da virksomheden dog kommer til at blive meget afhængig af programmørfællesskabet, kan det dog måske være en rigtig god idé at sørge for at signalere klart til programmørerne, om at virksomheden %ikke vil blive grådig for dermed at forværre forholdene for programørerne løbende til fordel for investorerne. Ja, det ville endda være rigtigt smart, hvis man på en eller anden måde kunne forsikre programmørerne om, ...
%også er fastsat på at opretholde gode lønninger til programmører, og ikke i sidste ende vil lade programmørfælleskabet i stikken til fordel for investorernes interesser. For jeg tror nemlig det bliver vigtigt at signalere om sådan en bæredygtig fremtidig udvikling for virksomheden, så programmørerne kan forvisses om, at de bakker op om noget godt. Dermed tror jeg altså, at virksomheden kommer til at skulle gå lidt en balancegang imellem programmørernes og investorernes interesser, måske endda mere end så mange andre virksomheder er tvunget til. Så for at vende tilbage til udsagnet om, at der er mange penge at tjene som invester, så tror jeg dog muligvis man skal skrue forventningerne en anelse ned, hvis man nu havde en fremtidig udvikling i sinde, hvor man virkeligt malker programmørerne (lidt ligesom at f.eks.\ Uber har været beskyldt for at malke deres freelance-arbejdere ret meget). For som jeg ser det, ville programmørerne så bare begynde at bakke op om en mere fair konkurrent i stedet, men på den anden side skal det så også siges, at jeg ikke helt begriber, hvorfor folk ikke bare forleder visse nuværende tech-virksomheder med samme argument, så jeg kan bestemt ikke siges at forstå disse forhold helt. Men et argument kunne muligvis være, at programmører, pga.\ at branchen allerede indeholder så meget iværksætteri i dets nuværende stadie, måske er mere vandt til at tænke i iværksættermuligheder. Men hvis altså ikke man er overbevist om dette, så kunne jeg i princippet bare stoppe salgstalen her.
%
%Jeg har dog også nogle andre idéer, jeg gerne vil nævne, som jeg tror, har potentiale til virkeligt at tiltrække stor opbakning omkring virksomheden, både fra programmørerne, men også fra brugerfællesskabet generelt. Den første idé er at bruge mine principper fra den idé, jeg her kalder en `kundedrevet virksomhed.' Jeg vil forklare om denne idé i \textbf{[...] kundedrevet virksomhed}-sektionen nedenfor, da den også kan bruges for sig og i andre sammenhænge. Jeg vil forslå, at man læser denne sektion, men basalt set går idéen altså ud på, at ...
%
%En anden idé, man også kunne bruge, er ...
%
%
%
%%Skal jeg huske at få noget med omkring mere semantiske Git-prjokter og særligt så med mapperetningslinjer og begrænsede actions, men tage tage som med programmør, eller bliver det for tangentialt? 





%\subsubsection{Mine idéer omkring ITP (skitse-noter)}
%Jeg vil også lige lave en idé-skitse over...

%Min egne tanker om et ITP-system med meta-antagelser (hvorfor det både er naturligt i denne forbindelse og også ret elegant, vil jeg mene..).
%Brugerbestemmelse over serveralgoritmer. 
%Ville jeg nu også inkludere brugerbestemmelse i det hele taget, og hvad nu med min demokrati-model? Jo, det ville jeg; jeg ville bare ikke nævne noget om brugerovertagelse. Så det skal nævnes mere som en mulig måde at få brugerne mere engageret på..



%\subsubsection{Mine idé til en kryptovaluta, hvis semantik kan afhænge af udsagn formuleret i naturlige sprog (skitse-noter)} %hvor den grundlæggende semantik bag valutaen, og dermed for pengefordelingen, kan afhænge af udsagn formuleret i et naturligt sprog (ligesom f.eks.\ gængs jura), hvorved fortolkningen af disse i sagens natur vil ske helt decentralt, og hvorved man således kommer til at kunne løse forhold, der ellers ville kræve en central dommerenhed, på en decentral måde (skitse-noter)}
%For lige at præcisere denne overskrift som det første handler idéen altså om en kryptovaluta, hvor den grundlæggende semantik bag valutaen, og dermed for pengefordelingen, kan afhænge af udsagn formuleret i et naturligt sprog (ligesom f.eks.\ er tilfældet for gængs jura), hvorved fortolkningen af disse i sagens natur vil ske helt decentralt, og hvorved man således kommer til at kunne løse forhold, der ellers ville kræve en central dommerenhed, på en decentral måde.


%Brain: (15.06.21) Nu kan jeg jo se, at jeg nok lidt har glemt ift.\ de seneste blockchain-idéer, at jeg også altid bør sammenlinge idéen med konventionelle systemer, og se hvad det tætteste alternativ ligesom er, i stedet for bare at sammenligne med andre blockchain-teknologier. Så det må jeg hellere også lige sørge for at gøre grundigt for denne idé. Hvad ville det tætteste, mere konventionelle alternativ være? ... ...Tja, der er jo noget i denne idé, fordi det jo er en decentral valuta (hvilket jo ser ud til at være i ret høj efterspørgsel).. Jeg har lige tænkt lidt flere tanker, og har planer om at tænke endne mere over emnet. Jeg har nemlig tænkt på, hvordan kæden så sættes i gang osv.. Et lille stikord til (tangent af) de tanker er: kurver.. Hm... ..Okay, hvis jeg lige skal braine noget mere, så vil en NL-KV måske også bare være smart, fordi det kunne være en hurtig måde at komme i gang med min dynamiske demokrati-model.. Så man lynhurtigt kan komme i gang med en åben og dynamisk investeringsfond... Tja, men en fond med et vagere og, ikke mindst, uprøvet system til at danne de nødvendige kontrakter i... Hm, jeg synes altså, at der er noget potentielt virkeligt brugbart (virker det til) i NL-valutaer.. Og ja, det kan jo føre en lykke-/gavn-valuta med sig, men dette kunne man vel også opnå på andre måder? Ja, for det første ved jo at bruge idéen på de kundedrevne virksomheder også, så de med tiden (efter endnu længere tid) bare bliver (demokratisk) folkedrevet.. Hm... Ja, det smarte ved NL-fortolkninger i en valuta er jo velsagtens, hvis man skal sætte det op på en spids, at det netop kan være decentralt (men stadig ret effektivt på papiret ift.\ konventionelle blockchains).. ...Ja, og det er jo også et interessant mål i sig selv --- det er i hvert fald et, som mange finder interessant. Men jeg kan dog mærke, at jeg lige skal overveje noget mere, hvordan man får gang i sådan en kæde, for hvis man egentligt ikke rigtigt behøver minere, hvem skal så mine de første mønter? Og mon ikke også det bliver et problem, at der så ikke nødvendigvis bliver et fast program til at mine... Hm, og som ikke opdateres på en fastlagt måde..? Ja, kommer der med andre ord ikke til at ligge mange problemer i også, at protokollen har elementer, som mennesker skal fortolke? ..Især når en dette endda kan inkludere, hvordan minerne skal arbejde i det hele taget.. Hm, på den anden side kunne alt dette bare netop være en grund til at starte kæden mere konventionelt --- måske bare med en kortereudløbstid på, hvor længe der kan mines nye mønter --- hvilket måske i sig selv faktisk er meget godt..(?) Hm, lyder umiddelbart ikke dumt.. .. Uh, hvad med noget mere lavpraktisk: Hvad med refleksive blockchainknuder, hvor der så stemmes en ny kildekode igennem efter hver fase, eller dvs.\ i hvert fald efter den første? Så egentligt ret meget a la det, jeg tænkte engang (om en demokratisk, "selv-opdaterbar" kæde); hvorfor ikke..? Det løser så ikke umiddelbart energi-problemet i sig selv, men man kunne jo netop så eventuelt kombinere det med NL-fortolkninger også. Hm.. Ah, men her er bliver der så et problem med, om folk kan have tillid til, at kæden ikke udvikler sig i en gal retning fra den enkeltes synspunkt (og alle skal så forholde sig til denne risiko)... ..Hm, men hvis nu man bare kan udstede mønter/tokens ret frit på kæden, og at de refleksive knudeprogrammer så bare handler om, hvordan man registrere og regner på transaktioner (og hvilket data hver enkel knude skal gemme og ting i den stil)..? Og at NL-fortolkningerne så bare kommer til at afgøre, ikke hvilke tokens, der kan handles med og hvordan, men nærmere hvad de er værd.. Hm, tja, tjo... Føles som om, der er nogle ting galt her (men måske ikke), men nu vil jeg lige nævne en tanke, jeg lige fik: Man må da forresten kunne lave en kæde, hvor knuder holder øje med det samlede netværk, og måler på, om der er konsensus omkring den samme historik, og hvor knuderne så måske kan stikke nogle yderligere handshake-følere ud rundt omkring, hvorefter de tilsammen kan opnå grønt lys for så at slette/ignorere al cementeret historik efter det punkt..! Ja, det må det jo.. Hm, men alt det her kan man jo med PoS-kæder. Hele NL-idéen skulle være for, at opnå PoS i praksis, men hvor de "cantrale enheder" kun kan være dette, hvis de følger den reelle semantik, for ellers bliver folk nødt til at finde nogen andre (når altså at den underliggende semantik i kæden ikke ikke sættes af nogen parter, men afhænger af reel sandhed).. Hm.. Ah! Men det går jo ud på et. For hvis man vil løse energi-problemet, så skal kæden kunne finde nogle tredjeparter (..ikke?..), og så er det jo lige meget, om disse vælges med et lavniveau-protokol eller med en abtrakt (NL-)protokol.. medmindre man altså netop kan opsætte  NL-regler, som de skal følge, og hvis ikke de (reelt) gør dette, vil pengefordelingen straffe dem og deres klienter..? He, det bliver lidt kompliceret nu.. ..Hov vent, back up.. vil en kæde, hvor hver knude holder øje med hinanden ikke netop kunne bruges som en decentral kæde med minimalt strømforbrug..?(!) Jeg tænker så lidt, at hver knude så vælger et aktivitetsniveau, som knuder binder sig til at overholde, og hvor der så kan mines en smule penge her ved et højere aktivitetsniveau, men hvor mining-lønnen kun passer med, hvad der er behov for af trafik (hvor denne så enten kan sættes demokratisk på en eller anden passende måde, men en anden, måske endnu bedre løsning kunne bare være, at brugere kan fulde "gas" på deres transaktioner, som minerne så kan få (og hvor aktive minere så vil få en større andel af den samlede gas i en periode end dem, der er mindre aktive)).. ..Og så er det faktisk ret vigtigt det her med, at knuderne faktisk kan være ligeglade med allerede cementeret historik, for det er vigtigt, hvis ikke man skal rende ind i de problemer med 51 \%-angreb, jeg har snakket om, at konsensusen ikke skabes så meget på baggrund af historikken, men mere bare som en løbende konsensus iblandt de knuder, der arbejder ud fra kædens protokol. Hm, det virker næsten for godt til at være sandt (hvilket sådan nogle ting så har det med også at være), så jeg skal helt klart lige tænke noget mere over det, men umiddelbert kan jeg ikke lige se noget i vejen for det. Spændende. (15.06.21) Og i øvrigt lige en sidste bemærkning inden tænkepause: Da NL-kæderne jo kan medføre sine egne problemer (og bestemt er et uprøvet territorie, som er lidt svært at sige noget om), og da fordelen ved dem i forhold til så mange andre løsninger alligevel kommer til at ligge i, at de udgør et ikke-centralt system, jamen så ville det jo være meget bedre, hvis man kunne finde en mere konkret lavniveauløsning (og som ikke er særligt langt fra gængse løsninger (og slet ikke så "ud af boksen")), hvis man rent faktisk kan opnå det samme med sådan en.
%(19.06.21) Okay, jeg er tilbage. Det blev en kort pause, før jeg jo så godt kunne se, at det ikke duer med en slags PoS, bare hvor det er minere og ikke stakeholders, der styrer; det går jo ikke. Jeg kom så dog ret hurtigt til at tænke videre over nogle andre muligheder, men disse holder vist heller ikke helt i sidste ende. Det var noget omkring ligesom at spille et spil, hvor minerne løbende skal konkurrere om, hvem der kan huske flest gyldige transaktionsblokke, eller rettere hvor de spiller et spil, hvor det kan betale sig at kunne huske flest mulige gyldige blokke. Hm, nu føler jeg sjovt nok, at jeg skal tænke lidt over det igen... For ellers er de to ting, jeg er kommet frem til nemlig, at PoS ikke holder, og at NL-kæder lidt er den eneste løsning, men jeg kom dog, som den anden ting, også på en mulig (som i måske) måde, hvor man kan reducere strømforbruget i PoW-kæder. Jeg fik også nogle forskellige andre idéer, som nok ville kunne bruges til at forbedre PoW, hvis ikke PoW havde noget fundamentalt i sig, som gør at ingen af idéerne virker.. Som var hvad..?.. For min omtalte idé til at reducere strømforbruget kunne da, så vidt jeg ved, godt være en mulighed.. Lad mig lige prøve at huske, hvad den fundamentale ting var.. Tja, det var jo noget i retning af, at hvis man formindsker de påkrævede udregninger med for meget ift., hvor stor værdi der er transaktionerne, så 51 \%-angreb jo blive mulige. Men min løsning gør nu bare, at minere kun skal mine blokke en paralel konkurrence (om at bygge en gyldig blok først) en procentdel af tiden. Så man kan reducere energiforbruget med den modsatte procent, og dette er så samtidigt med, at angribere stadig ville skulle opnå 51 \% af computerkraften, for at angribe kæden. ... Nå, nu kan jeg mærke, at jeg alligevel kommer til at tænke lidt mere over nogle alternativer til PoW, for hvis man nu havde en kæde, hvor miningprocessen gjorde, at data kan uploades til kæden, så bliver det muligvis rigtigt svært at angribe en mere PoS-agtig kæde.. Hm... Uh, hvis vi lige går tilbage til PoW, så kunne man måske forstærke min idé, hvis man kunne kræve, at miners arbejdede sammen på hver deres tal i en udregning.. som så ville få konstant, ikke-tilfældig tid.. Hm... ..Okay, jeg har gang i lidt mange tanker på én gang. Jeg har også lige tænkt noget med nogle spil, hvor brugere kan bevise, at de er mennsker over for hinanden.. Men nu vil jeg også lige nævne noget vigtigt, nemlig at et comeback imod argumentet over NL vs. PoS er, at man altid bare kan adoptere en ret simpel NL-antagelse til enhver PoS-kæde, som er i gang, nemlig at kæden kun er noget værd, hvis man aldrig nogensinde har valgt angriberes side, hvis der har været et klart angreb, men at man altid har været tro imod de lødige validators, og har fulgt disse, også uanset om de har været i undertal under et angreb. Og så er denne ret vigtige "fordel" ved NL-kæder pludselig ikke rigtigt en fordel alligevel. *(Men hvad der er "lødigt" kan jo mudres med tiden.. (f.eks. kan karteller jo godt få indført et ikke-neutralt internet-netværk..)) Man kunne dog så måske argumentere for, at NL-fortolkninger så måske er påkrævede (eller bare rigtigt nyttige) i teorien, og hvis de alligevel er dette, jamen hvorfor ikke se på yderligere muligheder, for at bruge NL-fortolkninger? ..Nå tilbage til nogle af de uafsluttede tanker (men lad mig understrege, at dette var en ret vigtig indskydelse for mig, for dette er vist ikke rigtigt gået op for mig før, og denne fordel ved NL-kæder var ret vigtig for mig, når det kommer til min genåbnede interesse i KV-løsninger her på det seneste). Nå.. identitetsbevisende spil.. Hm... ...Okay, jeg er også kommet på nogle tanker nu omkring en kæde, hvor det som en del af dens brugbarhed skal være muligt for brugere, at kunne uploade krypterede dokumenter/kontrakter, som de kan have mulighed for at låse op senere. ...Ja, og som kan brydes på en tid, som kun er en lille faktor længere, end tiden for at mine nye blokke, sådan at... Ja, eller dette bliver måske faktisk lige meget, for folk... Nej, det er stadig en ret god idé. For så skal ens fællesskab bare sørge for (og undskyld at jeg lige springer forklaringen af idéen lidt over; jeg er nemlig ved at brainstorme dens opbygning), at låse disse blokke delvist op, for at se, om nogen af dem skulle indeholde beviser for, at en vis bruger er ved at betale en anden bruger (typisk den der har uploadet beviset). Herefter kan alle nu afbryde alle andre transaktioner med den førstnævnte bruger, hvis det nævne (ca.)-beløb har en størrelsesorden, der gør videre handel med brugeren (paralelt mens pågældende betaling måske er ved at ske) usikker. Pointen er så også, at brugere kan vælge at uploade gamle underskrifter fra validatorerne *(altså fra en anden fork, hvis de har underskrevet mere end én fork).. hvis det er PoS.. Hvad det jo er.. ..Hm, er det ikke bare det, der er løsningen, de to ting, jeg lige skrev?..(!).. ...Jo, det er da lige før, jeg tror det!.. Det, og så den idé, jeg har haft i hovedet med, at opnå en måde, hvorpå validators'ne ikke kan forhindre, at kæden på et tidspunkt selv tjekker diverse claims om, at de har underskrevet andre kæder, uden at de samtidigt gør kæden mere og mere ineffektiv, indtil forbrydelsen accepteres på kæden (eller at kæden altså køres i sænk pga. ineffektivitet). Jeg har endnu ikke fundet den præcise protokol, der får denne tiltagende ineffektivitet til at ske, men jeg har nogle tanker til, hvordan det måske kunne lade sig gøre, som jeg er optimistisk omkring.. For jeg tænker nemlig, at brugere så bare kan blive ved med at uploade hint, og hvor miningprocessen så kræves, at tilfældigt udvalgte hints på blokken bliver taget i brug, og med vægt i sandsynligheden på gamle hints, der endnu ikke har været fulgt, hvorved mining processen altså bliver mere og mere besværlig, jo mere man som korrupt kæde skal prøve at ignorere hints, der afslører dobbeltsignaturen, som man prøver ikke at få offentligt frem på kæden (fra de korrupte aktøres synspunkt), hvis en blok bliver minet ved brug af hints'ne. Så med andre ord skal en del af blokminingen altså kræve, at man løser puslespil fra tidligere blokke, hvorved dobbeltsignaturer så potentielt kan afsløres, og hvor det altså skal blive svære og svære, at overse gamle puslespil, der burde være løste på pågældende tidspunkt i kædens udvikling (hvis ikke sammensværgelser aktivt prøvede at forhindre dette i at ske). Ja, det må man da helt klart kunne lave! Og man må da også kunne lave en kæde, så der er behov for, at folk kan uploade krypterede dokumenter (som bl.a. kan indeholde hints, men som altså også bare kan indeholde nødvendigt data til normale transaktioner), hvor at pågældende skal kunne være sikre på, at de ikke bliver låst op før der er gået noget tid..? Hm.. Tja, kunne én mulighed ikke være, fordi miningen netop vil kræve sådanne puslespil..? Tjo, men måske er det ikke det smarteste at tænke i.. Hm, hvad hvis nu betalinger bare skal kræve, at kontrakterne har været slået op med puslespil, der ikke bliver løst inden kontraktens fuldførsel..? Hov lige en indskydt idé: PoW er sårbart over for rumvæsner, hvilket måske er værd at pointere.. eller magtfulde instanser, der kontrolere den nyeste teknologi.. Ja, det er måske værd at pointere. Hm, men så er hele vores digitale system godt nok sårbart overfor rumvæsner, så never mind.. Nå tilbage til de tidligere (vigtige!) tanker, omkring hvorvidt man kan finde en måde, hvorpå kædens funktionalitet afhænger af, at brugere kan uploade puslespil med stor frihed. ..Jamen kunne det ikke netop bare være noget med, at de handlende skal satse på kædens hastighed, når de handler, hvilket, fordi det er PoS, svarer til at satse på, at validatorerne ikke angriber kæden.. hm, og man kunne måske også få disse til at satse med, sådan at udfordrene til kædehastigheden står til at vinde både over de handlende og over validatorerne, hvis kæden går for langsomt.. Hm, men dette er vel ikke rigtigt til, hvis validatorerne styrer kæden.. Hm... ..Ah, kunne man måske gøre noget med, at transaktionen bare kræver, at en hvis protokol over flere blokke finder sted inden at puslespillet løses, men at betalingen også først træder helt i kraft, hvis puslespillet løses --- måske endda inden for et vist antal maks-blokke?(!) Hm, så skal det bare være beviseligt, at puslespillet er nemmere at løse (og sikkert gerne med en god margin), end det er at mine dette maksantal af blokke.. Åh, har den så nu; kunne dette ikke simpelthen være løsningen?..(!?) ..Hm, i denne løsning, kunne de handlende måske selv arbejde sammen med korrupte validatorer for at sikre, at transaktionsdataen ikke indeholder beviser om dobbeltkontrakter og/eller afslørende hints til tidligere beviser, men så skal de handlende jo stole på, at ikke ét medlem af sammensværelsen vil sælge informationen videre til... Tjo, på nær at kæden stadig i princippet så bare kan ignorere uploads, der løser puslespillet for tidligt --- hvis altså sammensværgelsen alligevel kontrollerer og gennemgår hver blok... Ah, men brugere kan da bare indlægge hints i deres kontonumre og krypteringsnøgler! (For man kan jo bare farme, indtil man får de ønskede bits (som så dog ikke kan være vildt mange ad gangen.. ..Tja, men det kan man dog med kontonumrene, hvis disse er frie..).) Ja, og (forsat fra parentesen i parantesen her lige før denne sætning) det ville jo netop gøre kæden mere og mere ineffektiv, hvis validatorsammensværgelsen skal analysere alle kontonumrene først for skjulte hint; dette må jo blive mere og mere umuligt med tiden, at filtrere alle hintne fra, må det ikke?! Uh, ja og især fordi man kommer til at kunne gemme en stor chunk information bare ved et bloknummer og en (lille) krypteringsnøgle, hvis man selv kom med indholdet i pågældende blok. Hm okay, nærmer jeg mig nu?
%Nå nu tænkte jeg bare over dette emne hele resten af dagen. Jeg er faktisk kommet frem til en mulig løsning, der virker meget interessant (som bruger samme princip, men som er noget lidt andet end bare normal PoS..)..! Den handler om at dele kontoer op i grupper, hvor man skal migrere for at handle med andre kontoer (for de skal gøres inden for en gruppe), og hvor gruppen validere blokke til en lokal kæde sammen, hvor der muligvis bruges one-tap-koder, som så også kan bruges til at lægge omtalte puslespil op. Og når man modtager penge i én gruppe, kan man så sørge for at have konti i andre grupper, hvorved en one-tap-kode kan gives med et puslespil, der dokumenterer handlen (efter noget regnetid) --- eller afslører en dobbeltsignaturer. Og grupper der tages i at ignorere puslespillene, selvom de burde være løst, for værdien af deres mønter sat til 0. Ved ikke, om jeg har glemt noget centralt, men ellers har idéen altså med de nævnte ting at gøre. Men jeg må tænke videre over den i morgen. (19.06.21)
%(23.06.21) Jeg havde virkelig en kreativ bølge her de seneste par dage, men selvom der kom en del ret seje idéer ud af det, så tror jeg faktisk ikke, det er værd at arbejde videre på nu her. Jeg føler faktisk, at jeg næsten har en holdbar idé omkring, at have en kæde a la den jeg har skrevet om lige ovenfor, hvor der kan sladres om andre kæder, og hvor man bruger folks spillelyst som en yderligere resurse, ved at man køber folks løsninger til spil, som bruger kædereferencer til deres RNG-seed. På den måde kan man afgøre, hvilken kæde var ude i offentligheden, og hvilken kæde var skjult, hvis den offentlige kæde får nys om en anden kæde, men hvor kontoerne, der har givet dobbeltsignatur, ikke længere kan sanktioneres, og hvor man jo også gerne formelt vil kunne beslutte, hvilken en er den "rigtige." Jeg tror egentligt på, at dette ville kunne fungere i princippet, men så kræver det, at man har spil, som mennesker både er gode til ift. bots, og helst som de generelt gider spille. Men for at dette kan lade sig gøre ordentligt bør folk helst have mulighed for at tilføje nye spil til kæden. Men hvordan tjekker man så, at spillene er menneske og ikke bot-venlige; et spil kunne nemlig endda være konstrueret, så man skal kende en bestemt nøgle for at kunne skrive en effektiv bot, og ellers kunne spillet godt ligne et normalt menneskefavoriserende spil. Hvis ikke det var for dette faktum, og hvis altså ikke programmører på denne måde kunne skjule mulige (bot-)genveje i deres kode, så ville idéen måske faktisk du, for så kunne to kæder, der afslører hinanden, bare duelere på, hvem der kan få den bedste samlede score, når man inkluderer alle spil på de to kæder (og hvis den skjulte kæde har spillet mere af et spil en den offentlige, gør det ikke noget, for den offentlige kæde ville altid nemmere kunne reproducere den skjulte kædes arbejde end omvendt --- og så ville det nemlig også være ligegyldigt, hvis nogen af spillene faktisk var botfavoriserende, for disse kan man så også bare reproducere). Men problemet kommer altså, når en kæde kan uploade spil, som de alene har nøglen til at kunne skrive gode bots til (og man kan sagtens konstruere spil, hvor disse bots er i stand til at skjule deres nøgle i deres løsninger). Jeg har bl.a. tænkt, at hvis bare man så kunne tilføje en sætning om, at en kæde til hver en tid må udfordre den anden kædes spil, ved at (evt. efter at have om-kompileret programmet) antage, at et vist tal i programmet skjuler en nøgle, kan føre til et bot-venligt spil, hvorved det så bliver den udfordrede kædes ansvar at vise, at tallet ikke indeholder nogen skjult nøgle, og altså modvise omtalte antagelser.. Men ja, det bliver jo hurtigt rigtigt kompliceret, og da jeg ikke selv rigtigt tror så meget på hele idéen, så synes jeg ikke, det er det værd. Selv mine egne NL-kæder er jeg jo ingen gang overbevist om, vil være en holdbar teknologi. For selvom man måske godt kunne konstruere en mønt, der kunne være så gavnlig for folk, at de vil værne om NL-fortolkningerne, jamen så er der dog stadig ingen rigtig grund til, at det så lige skal være krypto.. Ja, for når man inkludere alt det, så bevæger idéen sig allerede meget langt væk fra det gængse (ift. andre KV'er), og hvis man derimod bliver i det lavpraktiske og bare inkludere nogle NL-udsagn om at være "offentligt tilgængelig," jamen så er idéen... værd at nævne, men ikke super langt væk fra en (holdbar af slagsen) PoS-kæde, for her vil det jo alligevel nok, alt andet end lige, være sådan at kæde-fællesskabet vil tolke kæden, hvis det kommer til et angreb, hvor fællesskabet skal vælge imellem to forks. Jeg tænkt nogle mulige sårbarheder i PoW og i gængse PoS-kæder, men jeg er slet ikke sikker på, at de ikke bare et eller andet sted er kendte. Jeg kunne lige kigge lidt nærmere på PoS-sårbarheder, hvis jeg får tid, men jeg tror faktisk ikke det er det værd. Så med andre ord tror jeg faktisk, at jeg vil nedtone betydningen af KV i dette notesæt. Måske kan jeg dog ikke udkommentere det, da det vist var KV-tanker, der ledte til min 11/06-idé.. Ja.. Ja, så jeg tilføjer altså bare lige nogle disclaimers ved overskrifterne..

%(01.07.21) Jeg har lige et par idéer, jeg gerne vil nævne, også selvom jeg ikke er så interesseret i blockchain længere. Det første er lige, at jeg faktisk nu tror, at man godt kan lave et system, hvor botsne kan opfanges (nemlig ved en protokol a la den, jeg også lidt foreslog, hvor man så viser at en vis mængde af programmer kan løses hurtigt af en vis (én-til-én) mængde af bots, og hvor det så bliver den forsvarende kædes opgave at vise, at programmet ikke ligger i denne mængde). Problemet er så bare, at jeg ikke er sikker på, at man kan lave interessante computerspil, der ikke ret nemt kan laves bots til at løse hurtigt, hvis man virkeligt vil det; det fleste computerspil kan der sagtens skrives bots til. Så jeg gider stadig ikke rigtigt den idé. Den lidt større idé, som jeg (også) fik i forgårs var så, at man kunne have en PoW-kæde a la BitCoin, hvor mining-lønnen bare afhang af.. hvor mange transaktioner blev gjort i det pågældende tidsinterval på kæden.. Hm, holder denne idé nu også?.. Nej, det gør den vel ikke rigtigt.. Ja, fuck det..



%\subsubsection{Idé til oprette en kryptovaluta som en slags investeringsfond, som har til formål at koble flere og flere reelle (juridisk gyldige) værdipapirer på kryptomønterne (således at møntejerne er garanteret at kunne veksle deres mønter til en andel af det samlede værdioverskud) (skitse-noter)}
%Overskriften siger en del ift., hvad idéen går ud på. Tanken er så, at fordelen ved at oprette en investeringsfond som en kryptovaluta er, at ... %Hm, tja, det behøver vel ikke at være en kryptovaluta.. Idéen er vel lidt bare, at man har nogle aktier i... Hm, er idéen bare en virksomhed? Har jeg bare genopfundet idéen om en virksomhed? For mønterne kan vel erstattes med aktier, værdipapirene er jo så bare virksomhedens værdier/kapital, og det at den er demokratisk om end vægtet: jamen sådan foregår det jo også i mange konventionelle firmaer, nemlig hvor flere aktier medfører flere stemmer, og hvor det nemlig kan være investorne, der bestemmer i sidste ende over virksomheden. Så ja, har jeg faktisk ikke bare genopfundet `et firma' her, eller havde jeg også nogle mere specielle tanker? Måske ikke, men jeg trænger af en eller anden grund til at gå en tur (selvom klokken kun er 10), og så kan jeg jo lige tænke over det imens.. (15.06.21) He, nej, jeg behøver vist ikke at gå den tur først. Jeg læste lige paragrafen, hvor jeg skrev om idéen først, og kan godt allerede se, at idéen nok ikke afviger særligt fra idéen om et normalt firma, når det kommer til stykket. He. x) ..Ja for det "bureaukratimæssigt effektive" ved idéen var egentligt bare i fom af min dynamiske, demokratiske model-styring, var det ikke? Aeh, næsten.. Jeg tænkte måske også lidt i, at det juridiske arbejde ville blive lidt nemmere, fordi det så kunne komme lidt efterfølgende, men svarer jo på den anden side også bare til en generel investeringsfond. Så pointen er altså lidt, at der ikke rigtigt er nogen grund til, at gøre det krypto. Og som jeg også lige nævnte i den brainstorm, jeg lige er begyndt på i undersektionen over denne, så kan det godt være, at idéen måske kunne slå eksisterende kæder i princippet, men dette er altså ikke nok; det går ikke med en idé, der bare er en gængs idé gjort krypto. ... Tja, der kunne muligvis være et argument i at bruge en strategi, hvor ikke starter med juridiske bindinger, men hvor man altså løbende opbygger en tillid, som der så bliver større og større motivation for de involverede i at holde, men det er vel nærmest også det..




%\subsubsection{Idé til en demokratisk styret (krypto-)valuta som en løsning til at varetage et samfunds politik omkring lønninger til forskellige arbejdsopgaver og andre samfundsbidrag på demokratisk vis (skitse-noter)}
%...




%\subsubsection{Idé til en form for online afstemning (skitse-noter)}
%Denne idé (som jeg også har beskrevet under sektion \ref{kundedrevet} ovenfor) handler om en demokratisk afstemningsproces, hvor folk løbende kan ændre deres stemmer når som helst online, hvorved de hermed så kan begynde trække de politiske retningslinjer i en ny retning. Disse politiske retningslinjer behøver så ikke nødvendigvis at handle om regeringers politik, men kan også sagtens handle om den politiske retning for en virksomhed eller forening, og måske endda kun for et helt bestemt område for denne virksomhed/forening. Et eksempel kunne være, hvad jeg foreslår omkring lønningsmodellen for en `kundedrevet virksomhed,' hvilket jeg bl.a.\ skriver om her i \textbf{Mine idé til en mere kundedrevet virksomhed}-undersektionen. 
%
%%Da det hele faktisk bliver noget teknisk og detaljeret er det ret vigtigt, at forklare målene og fordelene først. Så det må jeg lige huske på til når jeg skal skrive en bedre og mere ren version.
%
%Idéen er så i første gang at formulere en metamodel --- altså en model over, hvilke nogle modeller man kan danne i systemet, hvor en sådan metamodel altså eksempelvis kunne være i form af en definition af nogle formelle byggesten til disse, lad os kalde dem `indre modeller' her --- over hvilke nogle (indre) politiske modeller, som stemmedeltagerne samlet set kan indstemme i processen. Her er det så meningen, at virksomheden eller det politiske parti, eller hvad det kan være, skal binde sig juridisk (eller via andre former for løfter, man er interesseret i at holde) til at følge denne model. Hvis der i denne forbindelse kan være undtagelser til, hvor præcist man kan følge en indstemt model, så kan man så vurdere, om man vil prøve at inkludere disse undtagelser som en del af modellernes byggesten (i.e.\ af metamodellen), eller om man vil formulere undtagelserne for sig, samt en protokol for, hvad man gør i stedet, hvis modellen stemmes ind på et område, hvor man ikke kan følge den præcist. Det kan nemlig være, som jeg ser det, at det sidstnævnte kunne blive lettere i praksis.
%
%Man kan så se metamodellen som en definition af et samlet \emph{parameterrum} (og hvis læseren ikke kender dette til begreb, menes der altså et slags koordinatsystem af parametre --- man kan således forestille sig et to- eller tredimensionalt koordinatsystem i hovedet, men der kan dog godt være tale om flere end tre parametre), for det hører sig nemlig med, at metamodellen gerne skal indeholde muligheden for variable (reelle) parametre i dens indre modeller. Metamodellen må dog også gerne kunne indeholde diskrete (i.e.\ ikke-kontinuerte) overgange imellem forskellige indre modeller. Samlet set kan vi altså tænke på det samlede parameterrum, som metamodellen definerer, som en mængde af kontinuerte parameterrum, hver især med et vist helt antal dimensioner og med en euklidisk geometri (hvilket bare betyder, at de kan ses som helt normale koordinatsystemer med vinkelrette akser).
%
%Mit forslag går så på, at folk stemmer ved at påføre den nuværende politiske model, som kan ses som et punkt i et af disse parameterrum, en hvis \emph{kraft} (som begrebet kendes fra fysikken (søg på `klassisk mekanik' eller på `Newtons love')), hvorved de altså så kan trække ``punktet'' i en vis retning (og jeg burde næsten også sætte anførselstegn omkring `retning' her, hvis ikke metaforen omkring at ``trække noget i en politisk retning'' var alment kendt og forståelig). Og for så at folk ikke nødvendigvis skal holde øje hele tiden, og se om nu de skal ændre kraftretningen, hvis punktet ændrer placering, så foreslår jeg, at folk derfor bare får lov at stemme med et helt \emph{kraftfelt} (som også kendes fra fysikken, se f.eks.\ `elektromagnetisme' og særligt måske `det elektriske felt'). Herved kan stemmererne således i princippet bare stemme én gang i hele sit liv --- eller i metamodellens liv, hvis nu denne kommer til at blive opdateret undervejs --- med et kraftfelt, der fylder hele metamodellen, og så kan ens stemmekraft derfra bare være givet ud fra denne. Men hvis en stemmer ændre holdning og/eller får ny information, skal denne altså til hver en tid, som nævnt, have mulighed for at logge på og ændre sit `stemmekraftfelt,' som vi kan kalde det. 
%
%Nu mangler jeg dog, som man måske vil have bemærket, at svare på, hvad man gør, når det kommer til diskrete overgange mellem de indre modeller. Med andre ord hvad gør vi for at sikre, at brugere også kan bevæge punktet, der angiver den nuværende politiske model, imellem to adskilte koordinatsystemer? 
%%Mit forslag er her, at man sørger for, at alle særskilte, %variable\footnote{Og med `variable' modeller mener jeg altså her nærmere bestemt modeller, som er parametriseret udelukkende med reelle parametre.}
%%kontinuerte indre modeller, eller alle koordinatsystemerne, om man vil, selv bliver klistret op i et  %(euklidisk\footnote{Medmindre jeg tager fejl, og at euklidiske rum ikke kan betegne rum med et uendeligt antal dimensioner, hvilket jeg nemlig vil foreslå, hvis nu metamodellen udgør en uendelig mængde af kontinuerte indre modeller.}) %Nå nej, jeg vil jo egentligt foreslå en hypersfære, så never mind..
%%geometrisk rum, således at hver stemmers stemmekraft også kan gives kraftretninger, der forsøger at trække modelpunktet på tværs af de særskilte indre modeller. Dette kan implementeres ... Hov, der er noget jeg har overset her. Enten skal metamodellen sørge for at de forskellige modeller er arrangeret på en måde, så de faktisk selv kan beskrives via et begrænset antal parametre, og således at "koordinatsystemerne" selv kan sættes op i et overordnet koordinatsystem med et konstant, endeligt antal dimensioner, eller også skal folk jo bare stemme om de særskilte, kontinuerte indre modeller, hvor den mest populære så vinder; for en hyperfsære med antal dimensioner lig med antal særskilte modeller vil jo bare give det samme i praksis. Hm... Oh well, metamodellen vil jo alt andet end lige kunne ordnes i et endeligt antal dimensioner, for den vil jo i sidste ende være defineret ud fra en endelig mængde information. Og det vil nok ikke være vildt svært at beslutte sig for en fornuftig ordning. Så ja, jeg vil faktisk forslå at metamodellen parametriseres fra start, så det hele bliver ét rum effektivt set, og hvor alle diskrete overgange så modelleres som kontinuerte, men bare hvor der så sker en overgang, når den kontinuerte parameter overskrider en tærskel (halvvejs i rummet) imellem to modeller med diskrete forskellige. Så ja, lidt kort sagt skal metamodellen defineres så alle diskrete forskellige gøre kontinuerte, men hvor der så bare "rundes op og ned" for alle indre modeller med en ikke-heltallig værdi, for noget der kun giver mening som en diskret, heltallig størrelse. Hvis vi ser metamodellen som en samling model-byggeklodser, hvor nogle klodser så selv kan indeholde parametre, så skal både alle disse klods-parametre altså transformeres om til reelle parametre, men også alle spørgsmål om, hvorvidt man placerer en given klods eller ej, skal altså også modelleres via reelle parametre (således at de samlede "figurer," man kan bygge, altså også opstilles i et kontinuert koordinatsystem).
%Mit forslag er her, at man sørger for at transformere hele den samlede metamodel om til en kontinuert model ved for det første at omdanne alle diskrete parametre, der kan indgå i de variable indre modeller, til reelle parametre, hvor man så bare runder værdien af til nærmeste heltal, når det kommer til semantikken. Dette er dog ikke helt nok, for man må sikkert forestille sig, at den oprindelige metamodel (i dens mere naturlige form) nok også bestod af diskrete valg ift., hvilke nogle ``klodser'' man vælger til at bygge de indre modeller op. Et realistisk billede vil således nok være, at metamodellen udgør en samling ``byggeklodser,'' som hver især kan indeholde diskrete og kontinuerte parametre i sig. Hvis vi ser bort fra disse ``byggeklods-parametre,'' er der altså stadig et helt rum af muligheder, ift.\ hvordan man sætter disse klodser. Det bør dog også her være muligt, at parametrisere dette mulighedsrum, så det bliver et faktisk parameterrum --- og endda med endeligt mange parametre realistisk set. Disse parametre skal selvfølgelig så selv vælges som reelle, således at valget mellem at sætte en klods eller ej modelleres som noget kontinuert, men hvor man igen så bare runder udfaldet af til den nærmeste meningsfulde diskrete værdi. Herved kan metamodellen gøres fuldt kontinuert. %Bemærk, at hvis vi antager, at ``byggeklodsernes'' antal af parametre er begrænset (altså aldrig overstiger et vist antal), så vil metamodellen således kunne transformeres om til en kontinuert rum med et endeligt antal dimensioner, nemlig af $n+m$ dimensioner, hvor $n$ er den øvre grænse ... Nå nej, for antallet af klodser er jo alt andet end lige ikke begrænset..
%Bemærk dog, at hvis antallet af ``byggeklodser'' ikke er begrænset, og/eller hvis antallet af klods-parametre ikke er det, så vil rummet have uendeligt mange dimensioner. Dette gør dog ikke umiddelbart noget. For om man spiller et spil, hvor hver deltager kan hive et punktobjekt i 10.000 forskellige retninger, inklusiv blandede retninger, eller i et uendeligt antal retninger, betyder ikke noget i praksis; efter et vist punkt vil det ikke ændre billedet længere, hvis man tilføjer flere nuancer til de, i dette tilfælde politiske, retninger, man kan trække i. Lad mig dog lige for illustrationens skyld antage, at antallet både af klodser og af klodsparametre er begrænsede, og lad os kalde disse grænser for $m$ og $k$. Hvis antallet af parametre, man skal bruge for at definere det samlede rum af mulige ``byggekonstruktioner,'' så er lig $n$, så vil samlede antal dimensioner af det rum, som stemmererne skal definere deres kraftfelt i, være lig $n+m k$. Bemærk så i denne forbindelse, at der vil være områder af det samlede rum, hvor en parameters værdi ikke vil have betydning for semantikken bag modellen. I dette tilfælde vil den eneste grund til at lægge sin kraft delvist på denne parameter så bare være, hvis man er i grænselaget tæt på en anden model (hvor der er en diskret overgang i modelsemantikken), hvori denne parameter igen bliver betydende, og man derfor på forhånd derfor vil sørger for at justere denne, så den har en fordelagtig værdi, hvis denne grænse skulle overskrides. Og bemærk videre, at selv hvis vi ikke begrænser $m$ og $k$, så vil der i dette billede (altså det med byggeklodser og klodsparametre, hvilket jeg mener, sandsynligvis bliver et godt billede på forholdene) til hvert et tidspunkt kun være et endeligt antal klodser i spil, og dermed vil der altid kun være et endeligt antal parametre, der er betydende for modelsemantikken til et givent tidspunkt. Og hermed kommer folk altså ikke til i praksis at skulle overveje uendeligt mange parametre, selv på trods af at rummet altså i teorien er af et uendeligt antal dimensioner. 
%
%Nu kunne man så spørge sig selv, hvorfor bruge så meget energi på at forklare, hvordan metamodellen kan opbygges? Kan man ikke bare sørge for at finde frem til en eller anden model, der af semantisk dækkende for området, og som kan parametriseres med reelle parametre, og så må den være god nok? Tjo, det er ikke sikkert, efter min mening. For én ting er, om den samlede model i sidste ende bliver semantisk dækkende for området, men en anden ting er, hvordan metrikken (altså den geometriske udformning, der definerer afstandene imellem punkter) af modellen kommer til at blive, for dette bliver jo betydende, når man skal til at hive og trække %... Hov.. Hm, kom jeg ikke netop frem til, at metrikken ikke vil betyde noget, når man gør tingene på denne måde??... ..Hm, det går vel i virkeligheden ud på, at majoriteten, der nogenlunde er tilfredse med, cirka hvor modelpunktet havner, hvis man ser bort fra interne uenigheder, ikke bare er interesseret i selve området, men også i, hvordan landskabet ser ud omkring punktet, for hvis farlige konsekvenser truer lige rundt om hjørnet, så kan det blive farligt at fokusere for meget på de interne stridigheder. Og det er nok derfor, at jeg i bund og grund vil/ville have mange semantiske gentagelser i metamodellen, så man også i sidste ende får frihed til at vælge landskabet omkring sig som overordnet majoritet. Men hertil kan man så på den anden side måske sige, at hvis bare metamodellen designes på en måde, så der altid vil være... Hm, så der altid vil være semantiske gengangere, men med vilkårligt transformerede lokale metrikker, eller så landskabet bare generelt designes glat nok?.. Det sidstnævnte kan man måske ikke forvente...(?) Hm, man kunne måske også bare indføre en variabel metrik, som folk kan bruge noget af deres stemmekraft på at ændre lokalt.. Og hvis dette potentielt giver for meget informations-garbage, så kan man bare løbende approksimere metrikken, så den bliver næsten-identisk, men får cuttet en masse information fra sig. (Eller også kan de lokale ændringer bare propagere ud i resten af parameterrummet på en måde som bevarer/begrænser informationsmængden/entropien..) Hm.. Tjo, tja, der er nok flere muligheder her, men på den anden side er det jo stadig vigtigt nok at nævne de her ting, for jeg kan sagtens tænke mig situationer, hvor man gerne vil have en metamodel i form af et eller andet specielt, muligvis konventionelt, sprog, såsom et konventionelt logisk sprog eller et naturligt sprog ligefrem (f.eks. i forbindelse med mine NL-kæder). Så kan jeg ikke bare nævne dette faktum (nemlig omkring hvorfor man måske kunne have lyst til at vælge konventionelle sprog), og så i den forbindelse nævne, at man så måske lige skal sikre sig, at model er dækkende, ikke bare ift. semantikken, men også ift. metrikken omkring ethvert givet semantisk punkt, som folk har tendens til at kredse over? Og i denne forbindelse kan jeg jo så sige, at, jamen, hvis sproget er stærkt nok, så vil man også kunne definere både metrik og semantik, som man vil have dem i sine indre modeller, og dermed kan man få det som man ønsker sig det (samlet set) i sidste ende. Ja, det må være sådan..
%
%%Lad mig nu lige nævne et par småting angående den samlede metamodel. For det første kunne man måske tænke, om ikke det kunne blive et problem, hvis koord
%%Lad mig i øvrigt lige nævne, at der alt andet end lige godt kan være
%
%
%Nu har jeg så skrevet `kraftfelt' i det ovenstående, og det er også en meget god måde at se det som, fordi man ligesom kan forestille sig, at man ``trækker'' i en hvis retning (med en hvis kraft). I virkeligheden bør der dog i stedet hellere være tale om et hastighedsfelt, således at summen af alle alle folks ``kræfter'' et det givne punkt for modellen, ikke kommer til at resulterer i en acceleration, men en hastighed. Jeg tror dog, jeg vil blive ved med at kalde de et `stemmekraftfelt,' for det føles mere intuitivt (fordi tanken om at bidrage med en hastighed ikke, efter min mening, er særlig intuitiv ift.\ at bidrage med en kraft). Dette forstærkes også af, at vi er så vant til friktion i vores dagligdag, og er dermed vant til, at ting, der trækkes/skubbes, for det første går i stå af sig selv, hvis man stopper med at trække i en vis retning, og for det andet kun opnår en begrænset maksimal hastighed, hvis man vedvarende trækker i samme retning --- og billede er altså næsten ækvivalent med, hvad man opnår med et hastighedsfelt. 
%Det kan faktisk også være, at det ville give mening, at folk påførte en kraft (og dermed altså en acceleration) på modelpunktet i stedet for et hastighedsboost, for det kunne gøre visse ting mere belejlige så som, hvis man genre skal tilbagelægge, hvad der viser sig at være store afstande i parameterrummet ift.\ den generelle hastighed. Man kunne så også se på, om stemmererne skal have lov til at omdanne deres kraft til en friktionskraft, der modvirker bevægelse. Men her vil jeg dog i stedet foreslå, at man bare har en ekstern afstemning (som i øvrigt også kan fungere på samme måde, hvor folk så trækker med kræfter hver især, men så bare hvor dette parameterrum altså er endimensionelt), hvor folk så kan stemme om og derved justere en skaleringsfaktor for hastighedsstørrelserne relativt til parameterrummets afstandsmetrik. På denne måde opnår man nemlig også herved, at folk ret let kommer til at kunne stabilisere modellen, hvis der er stemning for dette. %Ja, det var en bedre løsning. (14.06.21) 
%
%%[...] Så jeg vil altså også fortsætte med at kalde det for `kraftfelter' i det følgende. %Hm, men ville det egentligt være så dumt rent faktisk at bruge kraftfleter, for så kunne man måske også implementere det, jeg lige har tænkt på angående, at deltagerne nok helst gerne også effektivt set skal kunne sætte bremsende effekter op for modellen, ved simpelthen at inkludere sådanne bremsende effekter i kraftfeltet, således at brugere kan vælge at konvertere deres kraft til en friktionskraft, som så kommer til at fungere som en statisk friktionskraft, hvis ikke den samlede trækkende/skubbende kraft overstiger den samlede friktionelle (og hvis punktobjektet også allerede står stille). Og ydermere ville det så heller ikke være så farligt, at der kan blive store afstande imellem brugbare modeller, for hvis folk styrer en acceleration og ikke en kraft, så bliver alle sådanne afstande, som man vil overkomme i flok, jo i praksis meget mindre. Så ja, er det ikke faktisk kræfter, man skal bruge, frem for hastighedsboosts (..ah 'boosts' var et godt udtryk lige at huske i øvrigt)? Det virker da næsten sådan.. (14.06.21) ..Hm, man hvis man så kan opnå en høj hastighed, som tager lang tid at bremse, så skulle man måske sætte en automatisk opskaleringsfaktor på friktionelle kræfter, hvis de over stiger 50 \%, så folk i flok endnu hurtigere kan bremse modellen igen, hvis de fleste vil dette.. Men ja, det korte af det lange er lidt, at der er mange ting, man kunne gøre her, så måske jeg bare lige kan nævne de overordnede muligheder.. Uh, en mulighed kunne jo også være i øvrigt, at folk også løbende stemte om hastighedsstørrelsen ift.\ parameterummets metrik som en ekstern afstemning.. Hm..
%
%%Det bør være klart for enhver, at stemmerernes mulige kraftfelter skal være begrænset hver især ift., størrelsen på kræfterne. Kraftstørrelserne behøver dog ikke at være begrænset nedadtil, for der er jo intet i vejen for, at brugerne kan melde sig ubeslutsomme og/eller ligeglade i visse områder. Dette kunne f.eks.\ være i området omkring deres politiske mål, hvor de så kan have mulighed for at lave en lille kraftfri lomme. %... Hov, det skal jo ikke være en kraft som sådan; det skal jo nærmere være et hastighedsfelt...
%Dette var altså de primære hovedtræk i idéen. Til disse kan jeg så lige præcisere, at der selvfølgelig skal være en øvre grænse på kraftstørrelsen i folk stemmekraftfelt. Denne grænse vil jo så i et typisk demokratisk system være det samme for enhver deltager, men i visse andre sammenhænge så som virksomhedspolitikker eller andre mere lokale og/eller private sammenhænge kkan det godt give mening at denne øvre grænse kan variere fra person til person, og også for den sags skyld variere over tid afhængigt af en eller anden udvikling. Jeg bør også lige nævne, at folk så selvfølgelig skal have mulighed for at definere deres stemmekraftfelt ud fra en matematisk funktion (som dog selvfølgelig gerne må være stykvist defineret), så at de ikke er tvunget til at justere kraftvektorerne i hvert punkt af parameterrummet på egen hånd. Bemærk i denne forbindelse, at der heller ingenting er til hinder for, at folk bare vælger deres stemmekraftfelt ud fra visse populære muligheder. Dermed behøver folk altså ikke selv lægge en masse arbejde i det, hvis de ikke har lyst; de får muligheden for lægge meget arbejde og tanke i deres stemmekraftfelt, men kan altså også bare vælge én, de har fået anbefalet. Derfor har de så også mulighed for at bruge meget mindre energi på afstemningssystemet, end hvad et konventionelt afstemningssystem kræver, for her skal man jo jævnligt møde op og afgive sin stemme, også selvom ens stemme måske forbliver det samme det meste af ens liv. 
%
%%Så dette er altså hovedtrækkene i min idé. Hertil kan jeg så nævne, at
%
%Lad mig også lige adressere det faktum, at der godt kan være flere punkter .. med samme semantik ...
%
%%Husk (ny idé, (13.06.21)) mulighed for RW-sandhedsparametre.
%%Brugere kan i øvrigt måske også definere partielt, hvor der så kan komme alarmer. ..Nå ja, og det kan også være, at de nogle steder vil trække med mindre kraft (så bare skriv at kraftstørrelsen i kraftfelterne skal være begrænsede (til 1)).
%
%%Nogle ord om, at der godt kan være flere punkter med samme semantik --- men måske med en anden metrik omkring sig.
%
%%Måske en kommentar omkring, hvad man kunne gøre, hvis man implementere det, så man har "forbudte områder" i sin metamodel.
%
%%Nogle ord om, bl.a. at man stadig bare kan følge et parti, hvis man vil, men hvorfor det alligevel kan være smart at holde fast i magten til at skifte mening.
%


%\subsubsection{Idé til et mere dynamisk justerbart og mere nedefra- og medlemsstyret politisk parti % --- der potentielt kan ændre hele den politiske proces og gøre regeringer mere repræsentative direkte for folkets ønsker
%(skitse-noter)}
%Og hvis jeg for det første skulle give en mere detaljeret overskrift her, ville det være: Idé til et mere dynamisk justerbart og mere nedefra- og medlemsstyret politisk parti, der potentielt kan ændre hele den politiske proces og gøre regeringer mere repræsentative direkte for folkets ønsker.
%
%...





%\subsubsection{Mine idé til en mere kundedrevet virksomhed (skitse-noter)}
%...
%
%
%\subsubsection{Mine idé om at binde sig til at give lønninger ud fra en fair model, som også løbende kan justeres, så ... Hm, jeg må lige finde på en passende titel... (skitse-noter)}
%...




\subsection{Opsummering} %Jeg kommer nok til at lave en anden afsluttende tekst. Ja.
Min opsummering/konklusion til dette notesæt kan faktisk tages direkte fra ovenstående ``Opsummering af idéer \ldots''-sektion, og nærmere bestemt fra sidste undersektion af det. Så det afsnit vil jeg bare indsætte her, og så bliver dette den samlede konklusion:

{\slshape
Nu vil jeg jo rigtigt gerne udgive min min folksonomy-idé og mit blockchain-angreb og -forsvar. Jeg håber på at sidstnævnte i høj grad kan rette folks øjne imod mig og på mine andre idéer. Forhåbentligt har min rating-folksonomy-idé så også en smule gennemslagskraft, så folk også vil blive interesseret i denne. Jeg overvejer også at udgive min idé om kurser i første omgang, måske hvor jeg dog venter med det, der handler om debat, så jeg ligesom venter med dette og med debatsektionen. Men det må jeg lige se på. Jeg overvejer faktisk også, om jeg lige kan få tid til også at inkludere min QED-teori\ldots\ Man kunne jo lige gøre opmærksom på den, og forklare den overordnet, og så bare lige disclaime, at jeg ikke har haft tid til at gå beviset helt igennem endnu\ldots\ Tja\ldots\ Nå, det må jeg alt sammen lige se på. Ellers vil jeg altså gerne ret hurtigt udgive QED-teorien og mine debatrelaterede idéer; hvis ikke i første omgang så gerne kort efter. Jeg vil også gerne ud med mine eksistens-idéer og -tanker inden for en nær tidsramme. Civilforeningsidéen virker ``politisk'' nok til, at det nok er fornuftigt lige at vente med den til der er lidt ro på ift.\ de første idéer (så det hele ikke bare bliver forstyrret af dette). Og når først jeg har givet mig selv grønt lys til at udgive idéer, der kan virke politiske, så kan jeg jo også måske inkludere nogle af mine idéer fra ``Noter omkring muligheder for det fremtidige marked generelt''-sektionen ovenfor.

Som øvrige ting kan jeg jo også sagtens udgive mine idéer omkring ``kundedrevne virksomheder'' og om (min seneste version af) min ``donationskæde,'' som nu bare er et argument om, at hvis man kan forvente en vis ting (det jeg har beskrevet her ovenfor i slutningen af blockchain-sektionen) af samfundet i den nære fremtid, jamen så kan dette måske motivere til i højere grad at donere til folk, der har ydet et positivt bidrag (som fremtidens samfund vil drage nytte af og altså værdsætte). Dette er mine to idéer, som næsten stod distancen, men ikke helt; de er sikkert gode nok, men det er ikke sikkert, at der bliver brug for dem --- ikke hvis min idé til ``civilforeninger'' bliver så god, som jeg håber på\ldots\ Men ja, og så har jeg forresten også lige min idé omkring en app, hvor man kan lege med forskellige forslag\ldots hvad var status nu på den idé?\ldots\ (Altså min idé omkring ``dynamiske model-afstemninger?''\ldots) \ldots Okay, nu har jeg tilføjet en bemærkning omkring civilforeningerne, at de skal have noget tilsvarende, men jeg bør stadig lige overveje, om der var noget specifikt teknisk ved idéen, jeg lige skal huske at få med, eller om det rimeligt meget giver sig selv. Det kan jeg lige vende tilbage til.

Det er heller ikke sikkert, at jeg udgiver alle mine tanker omkring folksonomy-idéen, som jeg har skrevet ovenfor, og man kan i øvrigt sagtens finde på flere ting at sige. Så planen er altså, at jeg bare udgiver mine tanker, som jeg ikke inkluderer i første omgang, løbende, samt hvad jeg ellers kommer i tanke om da. 

Nå ja, og jeg har også min ITP-idé, hvilket vel også kan betegnes som en idé, der ikke helt bestod distancen som en idé, jeg tror på har gennemslagskraft, men som (helt klart i dette tilfælde) alligevel har potentiale. Problemet er bare, at hele dens succes afhænger meget af, hvor brugbart det i grunden bliver med ``formel matematisk programmering,'' som jeg forestiller mig paradigmet. Og i øvrigt er det endnu en af de ting, hvor det først vil vise sig, når der er kommet nok tilslutning til det og dermed nok arbejde omkring det, hvilket jo kræver, at folk kan se potentialet i det, så det er altså endnu en ting, hvor bolden ligesom først skal rulle, før det kommer ordentligt i gang (\emph{hvis} idéen altså holder i første omgang), og hvordan får man den lige til det? Men alt i alt tror jeg dog stadig på, at idéen har rigtigt meget potentiale, og når det kommer til stykket, har jeg faktisk svært ved at forestille mig, at det ikke \emph{på et tidspunkt} bliver et fuldstændigt udbredt paradigme; spørgsmålet er bare, hvor lang tid der går, før paradigmet vil være brugbart nok (for en stor del af brugbarheden afhænger bl.a.\ af, hvor store open source-fællesskaber man har, samt hvor gode automatiserede algoritmer man har til bl.a.\ at indsætte løsningsskabeloner og søge semantisk på programløsninger osv.). Lad mig lige gentage forresten, at det jeg ser som noget særligt ved lige netop min ITP-idé, er det, jeg har beskrevet om ``meta-antagelser'' (som altså formulerer en række mappe-invarianter), samt det i høj grad at bruge certifikater til bevisførelse (så man derved altså formaliserer den uformelle del af bevisførelser, der gør mange beviser overkommelige, som ellers ikke var det, hvis man skulle holde sig til ren formel logik). Og ja, så har jeg jo også lige beskrevet nogle idéer til at komme godt i gang med det hele, i.e.\ idéer til at implementere en brugbar ITP af denne form. 

Desuden synes jeg også bare, at de ting jeg har nævnt i de nedenstående sektioner, hvilket altså også bl.a.\ omfatter energi- og klimaløsninger, lykke, etik og det fremtidige samfund samt et par andre tekniske idéer. Hm, og sidstnævnte kunne i øvrigt også omfatte min lille idé fra ITP-noterne omkring en puslespilsapp, der er underholdende at bruge (ligesom andre (casual) ryk-rundt-puslespilsapps), og som samtidigt lærer brugeren fremgangsmåden at kende i matematiske metoder, således at brugen meget nemmere kan få disse fremgangsmåder ind under rygraden, når brugeren (måske efterfølgende) lærer teorien bag dem. (For når vedkommende så efterfølgende skal lave de tilhørende regneøvelser, så vil fremgangsmåderne allerede være halvt/delvist inde på rygraden. Og oplevelsen ved at lave disse regneøvelser vil så også føles meget nemmere og bedre for vedkommende (der sandsynligvis vil få et vist dopaminspark af pludselig at kunne bruge de indlærte metoder i en ny sammenhæng i form af at løse faktiske opgaver, som ikke bare kommer fra appen).) Men ja, anyway, mine nedenstående sektioner tror jeg også generelt er gode at læse (hvad jeg nemlig ikke helt kan sige om alle de ovenstående sektioner; der er også meget rod og mange uddaterede idéer). 

Ah, og bør også lige nævne kort, at jeg jo også har en faseoperator (fra mit fysik-bachelor-projekt og efterfølgende artikel) og et kvanteprogrammeringssprog (fra mit datalogi-bachelor-projekt), %Ekstra bindestreger indsat pga. rendering, som åbenbart ellers var svært for LaTeX(-compileren).
hvilke jeg vist ikke har skrevet om her rigtigt, og som jeg for begge ting bestemt også håber på, kan vække interesse og kan bruges. 

Cool, det var det! :) Nu skal jeg bare lige udfylde nogle hængepartier ovenfor (bl.a.\ omkring, hvad jeg skrev, jeg ville vende tilbage til her, nemlig om de ``dynamiske model-afstemninger'') *(uh, og nedenfor skal jeg måske også lige opsummere lidt omkring mine eksistens-tanker), og så vil jeg eller (endeligt!) afslutte dette notesæt, og så vil jeg altså gå i gang med at skrive en tekst (på engelsk) til mine første udgivelser. 
}

Når jeg her har skrevet ``nedenstående'' og ``ovenfor'' og sådant, så skal det selvfølgelig altså bare lige forstås som relativt til den originale sektion. Eventuelle fremtidige rettelser (inden jeg afslutter skrivningen på dette notesæt helt) vil bare tilføjes til den originale sektion og denne kopi.

%Opsumering også ift. ITP-idé og sådan.. 



%De vigtigste takeaways fra alt dette, er for det første mine tanker omkring det samentiske web, og særligt det her med, at jeg nu altså tror, det semantiske web kan startes som et netværk, der fokusere meget på (html-)tekster, og hvor et centralt omdrejningspunkt bliver variable tekster, som i stedet for at have atomare tekster i sine sektioner, har prædikater omkring, hvad teksterne skal opfylde, 
%%
%og hvor dokumentet dermed kan ændre sig løbende, alt efter om der kommer nye bedre sektioner, måske med rettede fejl, udredninger af noget tvetydigt eller opdateringer ift.\ nye fakta og hændelser. Til denne idé hører sig så, at pointsystemerne skal være frie og drevet af brugerne, så hver forskellig gruppe af brugere kan få vurderet tekster efter deres egne protokoller osv. Og jeg er også overbevidst om, at mere nuancerede pointsystemer, end hvad man ser på nettet pt., vil blive utroligt stort og føre mange fordele med sig.
%Derudover tror jeg bare, man kan komme rigtigt langt, hvis vi fremmer en programmerings- og argumentations-kultur, hvor folk giver point, hver gang de læser en tekst, som beskriver, hvis de har forstået teksten, om de tror den er korrekt ...
%%(og undersektioner osv. --- selv bare de individuelle semantiske dele af )
%
%%"Semantisk modulære og dynamisk opdaterbare dokumeter/tekster (og hvad dette indebærer) med et frie pointsystemer, som brugerne selv kan oprette, vedligeholde og deltage i (båret af certifikater, så det kan være arbitrært decentralt)."..








\subsection{Lille appendiks}
Jeg påbegyndte denne lille sektion omkring en opsummering af min idé omkring at starte det semantiske web via en programmerings-vidensdelingsside, og indtil videre har jeg ikke gennemgået den for at finde ud af, om jeg skal beholde det, eller om jeg skal udkommentere det. Så indtil videre står den her:
\\\\
{\slshape
	\textbf{En mere semantisk side a la Stack Overflow (skitse-noter)}
	Jeg vil i første omgang skrive en idé-skitse over min idé omkring en mere semantisk Stack Overflow-*(/Git-)side. Jeg vil forklare lidt om fremtidsmulighederne for matematisk/semantisk programmering, som jeg bl.a.\ har nævnt i dette notesæt (selvfølgelig ikke det hele, men bare lige nogle udvalgte gode pointer). *(Hm, måske vil jeg egentligt mest bare fokusere på mulighederne omkring semantisk dokumentation, og særligt ift.\ mulighederne dette kan bringe med sig for bedre at udlicitere arbejdsopgaver og arbejde sammen om projekter online, samt hvordan jeg tror en masse gængse platforme kan blive bedre, hvis de involverer brugerne noget mere --- og gør brugerfladerne mere åbne og justerbare. \ldots\ Ja, og så tror jeg faktisk jeg vil formulere teksten, så jeg bare inkluderer tanker om, hvordan en Stack Overflow- og/eller Git-side så kunne ændres (eller opbygges fra ny), så den/de kommer til at kunne bære et sådant fællesskab, der altså kommunikerer programmeringsløsninger med semantisk dokumentation (og gerne via en ontologi, som jeg foreslår det). Og så kan jeg forresten altid bare pointere som en sidebemærkning, at det måske kunne være værd at se på, om man skulle prøve at bygge en ny side/platform beregnet på denne type programmeringsvidensdeling.) 
	%\ldots\ Hm, måske skulle jeg så dele det op, så jeg først laver et afsnit, hvor jeg slår et slag for et paradigme med større brug af semantisk dokumentation? Ja, det tror jeg, jeg vil gøre. Så omstrukturerer jeg også denne sektion (og udkommenterer dette).) %Nej vent, måske giver det ikke mening at splitte det op..
	
	Og til spørgsmålet omkring, hvordan vi så når til det punkt, vil jeg påpege, at man ved at skifte ITP-paradigmet en anelse, måske så faktisk kunne komme ret hurtigt og stærkt i gang med hele denne semantiske teknologi. Jeg vil således foreslå, at starte med et formelt logisk sprog med formelle og simple inferensregler så faktisk starter i den nærmest helt uformelle ende, og lader alle konklusioner udføres på baggrund af pointsystemer udviklet og varetaget af et onlinefællesskab, hvor brugere simpelthen giver pointvurderinger til korrektheden (samt muligvis også andre kvaliteter såsom bl.a.\ entydigheden) af bevisskridt/argumenter i de uploadede beviser. Dermed kan man sige, at processen i bund og grund kommer til i starten at svare meget til, hvordan matematik udvikles i dag, nemlig ved at et fælleskab arbejder sammen om at tjekke bevisskridt manuelt --- og hvor man så også implicit har mere tiltro til nogen kilders udsagn om korrektheden frem for andre --- bare hvor al denne kommunikation så foregår online, og hvor der som en særlig del af platformen indgår, at brugerne selv er frie til at benytte deres egne pointsystemer, når vedkommende skal vurdere tillid til andre brugere og ultimativt til korrektheden af bevisskridt. Selvom man derved kan starte med vilkårligt lav grad af formalitet i sine beviser i fællesskabet, skal sproget hvori man udformer sine argumenter dog stadig bygges på et formelt logisk sprog som fundament, ikke mindst fordi fællesskabet så derved løbende for mulighed for at præcisere og formalisere individuelle bevisskridt/argumenter mere og mere, så de måske i sidste ende kommer til at bestå af formelle logiske inferensregler. Da jeg vil foreslå et refleksivt logisk fundament, hvilket jeg vil forklare nærmere om nedenfor i \textbf{Mine idéer omkring ITP (skitse-noter)}-sektionen, behøver disse inferensregler dog ikke kun bestå af de helt simple af slagsen, før man når til et punkt, hvor bevisskridtet også kan verificeres automatisk; de kan sagtens være komplicerede regler. Herved får vi altså så en Stack/Math Overflow-agtig side, dog med et åbent pointsystem, som brugerne selv kan definere og bruge alle mulige versioner af, og hvor det så er muligt på sigt at formalisere diverse argumenter og bevisskridt mere og mere, så det beror mindre og mindre på vurderinger fra brugere, der har gennemgået dem, men så de i højere grad kan vises automatisk ud fra nogle mere grundlæggende antagelser (og i sidste ende matematiske aksiomer, når man når så langt ned). Jeg håber så, at det i den forbindelse vil være til generelt at formalisere diverse argumentationsformer og bevismetoder, så man inden for en overskuelig fremtid kan begynde at genbruge sådanne metoder forskellige steder, på en måde så man sparer, at folk skal gennemlæse og verificere korrektheden a de brugte metoder på ny hver gang. Jeg håber altså, at man ret hurtigt vil begynde at kunne formalisere metoder, så folk kan tilægge en vurdering af korrektheden til en given metode én gang for alle og dermed ikke skal vurdere brugen af den i hvert enkelt bevis, den indgår i. Men selv hvis dette ikke bliver tilfældet, og altså hvis brugerfællesskabet stadig skal vurdere hvert bevisskridt/argument for sig, så tror jeg altså stadig, man kan komme rigtigt langt med en sådan internetplatform (plus fællesskab). For hvis en tekst er interessant for mange mennesker, så vil det også være det værd som læsere, at sørge for at de enkle delargumenter i teksten bliver vurderet. 
	%Jeg håber i øvrigt også på, at brugerne i deres egen ende kan få mulighed for at bruge specielle metoder til at autoudfylde kode-templates, når implementere en (matematisk dokumenteret) funktion eller klasse, hvor disse templates så allerede indeholder alle de lemmaer, der skal vises for i sidste ende at vise korrektheden af den udfyldte template. Men også her kan man sige, at dette dog ikke nødvendigvis behøver at blive tilfældet, før at idéen er god; det er den stadigvæk.
	
	
	Et af de store problemer med sider så som Stack Overflow udover den simple form for op-ned-rating er nemlig, at de kun indeholder enkeltpersoners tekster (evt.\ med redigeringer). Her mener jeg altså, at man kan komme langt længere, hvis de endelige tekster, som brugerne i sidste ende kommer til at læse, når de søger efter viden, i langt højere grad er tekster, som brugerne har samarbejdet på at udforme. 
	Og ved samtidigt at gøre pointsystemet helt åbent, så brugerne selv kan udvikle og bestemme over søgealgoritmerne og filtrene for, hvad de får vist, så skal forfatterne altså ikke kæmpe nogen indbyrdes kamp om at få deres version frem; hvis deres bidrag er gavnligt for en vis mængde brugere, så vil bidraget også efterhånden stige en point i de pointsystemer, som pågældende brugermængde benytter samlet set. Lad mig også lige præcisere her, at det så er meningen at tekster skal brydes meget op i semantiske elementer, så hver sætning formelt set får defineret, hvad dets funktion er i teksten; er det et argument for en vis anden sætning i teksten, eller måske en konklusion af andre sætninger. Hver sætning skal så med tiden altså tilføjes graf-kanter til andre sætninger --- eller til eksterne udsagn, som enten er refereret til eller er implicit forstået af læserne (eller rettere de fleste læsere, hvis de har det forventede niveau, og hvis ikke så vil det nemlig være gavnligt med disse eksterne referencer, så alle læsere i sidste ende har mulighed for at forstå teksten) --- som viser disse semantiske sammenhænge. En bruger må dog meget gerne have mulighed for at uploade en ret bar tekst, hvor alt dette semantiske metadata altså endnu ikke er tilføjet, og så kan det altså være op til fællesskabet generelt, selvfølgelig hvis teksten altså er interessant nok, at analysere og kvalificere hele tekstens semantiske opbygning --- og selvfølgelig komme med forslag til rettelser og tilføjelser (også i form af eksterne referencer) osv.\ osv. Og alle disse tilføjelser og rettelser skal altså så kunne rates af andre brugere med forskellige point, og i sidste ende skal en bruger altså kunne få serveret den tekst, inklusiv metadata, der giver så mange point som muligt (selvfølgelig alt efter, hvor grundigt man vil have serveren til at arbejde på forespørgslen (og dermed hvor langt tid, man kan vente på svaret)) i brugerens foretrukne pointsystem (som altså udregnes ved at tage et vægtet gennemsnit af de samlede ratings, hvorved brugeren altså kan vægte visse brugergruppers svar højere end andres (og her er det selvfølgelig også brugeren selv, der får lov at vælge, hvordan brugergrupperne skal klassificeres i sidste ende), og hvor der så efterfølgende tages et vægtet gennemsnit af diverse typer point, som man har inkluderet (så brugeren f.eks.\ får mulighed for at vægte `hurtigt at læse' og/eller `let at forstå som begynder på området' højere end f.eks.\ `detaljeret' eller `omfattende for emnet,' eller hvad det ellers kunne være)).
	
	
	Sidst i sektion \ref{efter_milawa} ovenfor kan man så læse om, ca.\ hvordan jeg nok ville indrette serverne til platformen. Jeg bør for resten lige se på, hvordan man kunne implementere en måde for serveren at servere del-ontologier ud fra brugerens point-system-præferencer med relationelle databaser. Man bør også have en opbygge en underlæggende begrebsontologi til siden, men her kan man faktisk snyde en anelse ved bare at kunne referere til andre online-leksika, så man på den måde allerede effektivt set får en ret dækkende ontologi over begreber (ved altså at brugerne bare kan referere til leksikonerne, når de skal bruge et term i en tekst, f.eks.\ i titlen eller som stikord, og som andre brugere dermed bare kan søge på via leksika-URL'en\ldots\ tja, på den anden side kunne man også bare skrabe pågældende leksika og opbygge en begrebsontologi på selve siden, for så er man ikke afhængige af, at URL'erne ikke skifter).
	
	
	Udover matematik og andre tekniske områder er det så altså særligt programmering, platformen er tiltænkt at handle om (hvorfor jeg referere meget til Stack Overflow som et udgangspunkt). Jeg tror så altså på at man her kan opnå en slags Stack Overflow, hvor de uploadede programmeringsløsninger også har opgivet data omkring sig for, både hvor grundigt de er testede, og hvor grundigt de er læst og analyseret af diverse brugere (med forskellige kvaliteter ift.\ deres aktivitet og stilling af og på platformen).
	
	Da fokusset således i høj grad er på programverifikation (både ift.\ maskin- og menneskeverifikation, og både via kodeanalyse og/eller testing), så ville det jo ikke være dumt, hvis man også kunne benytte fællesskabet til også at udvikle og forbedre server-algoritmerne til platformen. Dette kræver selvfølgelig, at man gør hele systemt ret open source, men det er også min hensigt i forvejen. Platformen kommer nemlig i utrolig høj grad til at være båret af brugerne, så den eneste vej frem vil være, at gøre platformen så venlig og gavnlig, og ikke mindst åben, for brugerfællesskabet som overhovedet muligt. Vil der så være nogen penge at tjene her? Ja. Hvis man brander sig på, og får skabt et godt image af, at man standhaftigt er så brugerfællesskabs-venlige som muligt, og at brugerne derved virkeligt kan stole på en, så vil de også bakke op om foretagendet og vil gerne donere/betale for at sørge for, at man (som firma/organisation) bliver ved med at administrere servicen. Men der er faktisk også endnu flere penge at tjene på denne open source-sti. Det kommer jeg til om et øjeblik. Men for lige at komme tilbage til serveralgoritmerne, så kan man dog altså bare starte med faste serveralgoritmer, og så langsomt åbne mere og mere op for, at brugerfællesskabet kan få indflydelse, når dette modnes nok, og når man selv som firma/organisation er klar. En ting er så, at de forslåede algoritmer skal verificeres (ud fra et point-system som firmaet/organisationen selvfølgelig bestemmer), men en anden ting er at de også skal stemmes igennem af brugerfællesskabet. Her kan der så selvfølgelig være delte meninger og behov, så i den forbindelse bør man lige sætte et demokratisk system op, eventuelt på en måde så de mest betalende/donerende brugere for tilsvarende mest stemmeret. Her har jeg i øvrigt en idé til, hvordan man kunne strukturere en sådan online afstemning, som jeg vil skitsere i \textbf{[...]online afstemning (skitse-noter)}-sektionen nedenfor.
	
	
	Det ville være godt med en hjemmeside, hvor brugerne kan browse alle platformens ontologier, men derudover tænker jeg også, at brugerne skal have et program (helst bare helt open source) til at tilgå serverene på. Ligesom at serveralgoritmerne på sigt kommer til at blive brugerstyret, så bør brugerne selvfølgelig også kunne bygge deres egne browseralgoritmer, så de i deres ende kan præprocessere deres søgningerne for på den måde at kunne opnå større effektivitet i server-browser-interaktionen. Samtidigt vil det også være smart for programmører og matematikere (eller folk der arbejder inden for andre tekniske områder) med en IDE (til hvad end de nu laver), som kan kommunikere med platformen og hente brugbare sætninger, metoder, programmeringsløsninger osv.\ ned, når de er relevante til arbejdet. Jeg ville så foreslå også at basere sådan et program i matematisk logik, og selvfølgelig gerne i den samme logik, som platformen er bygget på. Jeg vil således foreslå samme logiske fundament, som jeg vil skitsere nedenfor i \textbf{Mine idéer omkring ITP (skitse-noter)}-sektionen.
	
	
	Nu vil jeg så vende tilbage til det jeg teasede om, at der er mange flere penge at tjene, end bare ved at administrere en platform med servere, hvor brugerne får meget indflydelse på algoritmerne. Nå jo forresten, inden da kunne jeg måske også lige nævne, at hvis man nu er lidt bekymret for, om brugerne bare vil skifte til en anden serveradministrator, hvis man gør det hele open source, så kunne man måske sørge for bare at tage brugerne endnu mere med på en lytter, også når det kommer til, hvad abonnement-priserne skal være, og hvad arbejderlønningerne skal være, og således faktisk give dem lidt medbestemmelse her. Dette kræver så selvfølgelig en afstemningsproces, som skal kunne foregå digitalt, online --- det kunne eksempelvis være den jeg foreslår her i i \textbf{[...]online afstemning (skitse-noter)}-sektionen nedenfor. Og når dette så er nævnt, så kan jeg her gå videre til at foreslå, at man søsætter en hel IT-virksomhed på baggrund af platformen, som så er beregnet til %at fungere som slags forening af programmører, ... %Hm, hvordan bliver det nu med at sætte betalingsmure op..? ...Hm, skal man overhovedet det, hvis man har en kundedrevet virksomhed?.. Det skal man vel delvist, eller hvad?... Men ja, en del af idéen ved at gøre det lønningerne kundebestemte er jo netop bl.a., at behovet for betalingsmure og lukkethed generelt bliver mindre. Uh, man bør da forresten næsten have mine nye modelprincipper, som jeg fandt på her d. 11/06 (i går), se "*Yderligere tanker omkring kryptovaluta"-sektionen..! Det må jeg lige nævne i ovenstående noter et sted... ..Det er godt nok nogle principper, som særligt er smarte for... nej; jeg skulle til at sige NL-kæder, men man kan også implementere det samme helt fint bare med juridiske kontrakter. ... Okay, nu er det nævnt ovenfor. Angående betalingsmure så skal man vel nok kunne opsætte disse. Så virksomheden/foreningen skal altså kunne sætte sådanne op, hvad jeg førhen har set som en klar mulighed.. Og hvorfor er lige, at dette er en så klar mulighed..? ... Hm, nu fik jeg lige nogle tanker omkring, at bidragere måske kunne overvåge sig selv i arbejdsprocessen, hvis man skal erstatte andre folks bidrag, som de har trukket tilbage igen.. hm, nej for man kan bare terpe... ..Selvfølgelig kunne man give sig selv det dogme, at folk skal hyres i uvidenhed om opgaven, men dette virker så ikke for populære bidrag, hvor der er stor sandsynlighed for, at folk kender til løsningen... Nej, det er jo nok bedre, det her med at folk ligesom bare kan sælge deres bidrag til foreningen.. Ja, og så kunne det netop være mine 11/06-idé, der gør, at man kan forsikre bidragere om bagudbelønning, hvis man føler behov for at kunne dette.. Hm, ja, er det ikke nærmest bare det?.. Ah! Jo, det er det da! Ha, jeg havde næsten helt glemt, hvordan et godt system til bagudbetaling ikke bare skulle føre mere retfærdighed med sig, men også potentielt en hurtig udvikling af et semi-open source-virksomhed --- og særligt ift.\ IT-løsninger..! He, jo, så jeg skal helt klart foreslå idéen i forbindelse med denne programmør-forening-virksomhed. (12.06.21) Og hvad er alternativet så ellers; er det bare, at man som forening/virksomhed køber alle de bidrag, man har råd til (evt.\ ved salg af ens egne aktier)? Hm.. Tja, tjo, men idéen holder da, så jeg behøver vel ikke overveje alternativet, eller gør jeg? Jo, det er meget fint at have med, og der også en fornuftig nok løsning på dette, og det er bl.a., at man benytter brugernes vurderinger til at hjælpe den bureaukratiske proces omkring at bestemme belønninger, også i starten. Man behøver nemlig ikke at vente på, at kontrakterne siger, at man skal gøre dette; hvis det giver mening, kan man bare gøre det forinden, og så er man (som ejere/administratorer af virksomheden) i dette tidsrum bare fri til at justere denne medbestemmelsesmagt og trække den frem og tilbage, som man ønsker, og man er også endda fri til at bestemme sine egne pointpræferencer, ift. bl.a. hvilke brugere, man lytter mest til osv. Vi kan tage YouTube og Twitch som hurtige eksempler, hvor lønninger til skabere bestemmes ret automatisk ud fra diverse "point" (altså views, likes, og hvad har vi) skaberen optjener for en video/stream. I øvrigt kunne man så måske med fordel også offentliggøre... Ja, man kunne med fordel offentliggøre hele regnestykket, og det bør man helt klart, da gennemsigtighed er yderst vigtig her (for bl.a. hele brandet), men ved særligt at offentliggøre, hvor meget... der gives ud pr. tidsinterval i lønninger... nej, never mind; hvis bare man offentliggør regnestykket, hvad man alligevel bør, og måske eventuelt også udgiver en simplificeret model og/eller andre overordnde parametre og estimatorer, så brugerfælleskabet nemt kan få en god idé omkring, hvor store lønninger gives for hvad, jamen så er man jo godt i gang. Om man så (måske kortvarigt) binder sig juridisk til at følge dette regnestykke, så man er sikker på at lokke bidrag til, eller om man bare forsøger at opbygge mere og mere tillid løbende, det må man jo så bare finde ud af. 
	at opkøbe rettigheder omkring brugeres uploadede programmeringsløsninger for så at tjene penge på dem igen ved at sælge og/eller udlicitere brugen af diverse applikationer, som disse løsninger tilsammen opbygger. 
	%Og lad mig allerede nævne nu, for jeg teasede, at der nok kunne være mange penge at tjene på dette: Er det så meningen, at man herved skal forsøge at malke brugerne (som i dette tilfælde så vil arbejde som en slags freelance-programmører for virksomheden) for deres bidrag, og aggressivt forsøge at opnå en så høj mark-up på handlen. Nej, faktisk ikke. Jeg omvendt tror på, at det faktisk bliver virkeligt vigtigt som virksomhedens forretningsledelse, hvis denne idé skal du, at være nærmest så meget på programmørernes side, som muligt, så virksomheden nærmest kommer til mere at være en forretningsforening for programører i stedet frem for en mere konventionel tech-virksomhed. %Hm, og idéen ville næsten også faktisk give mere mening, hvis den var non-profit til at begynde med, så hvordan fletter jeg lige disse to ting sammen, så man også holder muligheden åben for at appellere til investorer (og gå den mere gængse iværksættervej)?.. Tja, det er jo faktisk egentligt et væsentligt spørgsmål det her, for jeg vil jo egentligt gerne bare dele det med så mange som muligt, men så mister idéen muligvis appellen til investorer, fordi det så nok ikke er til at skjule, at en mere non-profit konkurrent hurtigt vil udkonkurrere den mere kommercielle tilgang... Så hvad er overhovedet idéen ved at rette teksten mod investeringsinteresserede..? Hm, jeg har lyst til bare at sige fuck it, men jeg skal nok træde varsomt her, for det ville også være dumt at miste en mulighed. Ja, og da hele min mission nu bliver, at fange folks interesse i et godt omfang, så er det jo farligt at skære muligheder for, netop at fange folks interesse af... Hm, tjo, men interessen omkring idéen står og falder med, om der er mulighed for som iværksætter at tjene penge, jamen så vil man også gøre sig fortjent til penge, hvis man begiver sig ud som iværksætter, for så vil dette jo være risikofyldt. Og så må folk jo også anerkende dette arbejde og denne risikotagning. Hm, men skal jeg så ligefrem argumentere for dette?.. Og sige: vent og se, og hvis non-profit-virksomheder ikke opstår på baggrund af disse idéer, jamen så betyder det, at der er grobund for en kommerciel virksomhed..? Hm, er det ikke sådan, at det netop ikke duer med en non-profit virksomhed, fordi virksomheden gerne skal tiltrække investorer fra starten, så man kan begynde at betale lønninger af tidligt? Jo, er det ikke det, der er (eller var) svaret..? Jo, det må det være. *(Det er ikke helt dumt med en god idé med markedspotentiele, men med mine andre indéer, og især måske mine idéer om blockchain, så behøver jeg vist ikke være vildt bekymret for, om mine idéer vil sprede sig (og hvor er det dejligt at kunne sige det.. 7, 9, 13..). Men ja, så jeg bør altså nok, som nævnt, fokusere teksten mere i retning af: "hey, hvad med et nyt paradigme inden for programmering (m.m.)?!")
	%Ja, fordi virksomheden kommer til at blive så brugerdrevet og brugerstyret, som jeg forestiller mig, den gør, så ville en tilsvarende non-profit-virksomhed nok hurtigt kunne udkonkurrere den kommercielle udgave, hvis ikke det var fordi, at virksomheden også vil være meget afhængig i starten af investorer, så den kan betale for diverse brugerbidrag, der skal opkøbes rettigheder til. Og fordi man altså så kommer til at gå denne balancegang alligevel mellem at være en virksomhed henholdsvis for brugerne/programmørerne og for investorerne, så vil der trods alt, mener jeg, også blive en rimelig god skilling at tjene både som iværksætter og som iværksættende investor. Dette er dog nok det sidste, jeg vil nævne angående disse forhold omkring ... Hm, alt dette må kunne formuleres kortere og bedre.. (Jeg udkommenterer altså lige dette.)
	For at kunne give freelance-programmørerne løn skal man så selvfølgelig gerne kunne tiltrække investorer først. For iværksættelsen af virksomheden udgør jo en risiko, og uanset hvordan man vender og drejer det, så vil disse investorer, hvem de så end er, forvente en vis fortjeneste, hvis planerne lykkes. Dermed kan man nok ikke (i hvert fald hvad jeg ved af) satse på, at starte en helt non-profit-virksomhed. Da virksomheden dog kommer til at blive meget afhængig af programmørfællesskabet, kan det dog måske være en rigtig god idé at sørge for at signalere klart til programmørerne, om at virksomheden %ikke vil blive grådig for dermed at forværre forholdene for programørerne løbende til fordel for investorerne. Ja, det ville endda være rigtigt smart, hvis man på en eller anden måde kunne forsikre programmørerne om, ...
	også er fastsat på at opretholde gode lønninger til programmører, og ikke i sidste ende vil lade programmørfælleskabet i stikken til fordel for investorernes interesser. For jeg tror nemlig det bliver vigtigt at signalere om sådan en bæredygtig fremtidig udvikling for virksomheden, så programmørerne kan forvisses om, at de bakker op om noget godt. Dermed tror jeg altså, at virksomheden kommer til at skulle gå lidt en balancegang imellem programmørernes og investorernes interesser, måske endda mere end så mange andre virksomheder er tvunget til. Så for at vende tilbage til udsagnet om, at der er mange penge at tjene som invester, så tror jeg dog muligvis man skal skrue forventningerne en anelse ned, hvis man nu havde en fremtidig udvikling i sinde, hvor man virkeligt malker programmørerne (lidt ligesom at f.eks.\ Uber har været beskyldt for at malke deres freelance-arbejdere ret meget). For som jeg ser det, ville programmørerne så bare begynde at bakke op om en mere fair konkurrent i stedet, men på den anden side skal det så også siges, at jeg ikke helt begriber, hvorfor folk ikke bare forleder visse nuværende tech-virksomheder med samme argument, så jeg kan bestemt ikke siges at forstå disse forhold helt. Men et argument kunne muligvis være, at programmører, pga.\ at branchen allerede indeholder så meget iværksætteri i dets nuværende stadie, måske er mere vandt til at tænke i iværksættermuligheder. Men hvis altså ikke man er overbevist om dette, så kunne jeg i princippet bare stoppe salgstalen her.
	
	Jeg har dog også nogle andre idéer, jeg gerne vil nævne, som jeg tror, har potentiale til virkeligt at tiltrække stor opbakning omkring virksomheden, både fra programmørerne, men også fra brugerfællesskabet generelt. Den første idé er at bruge mine principper fra den idé, jeg her kalder en `kundedrevet virksomhed.' Jeg vil forklare om denne idé i \textbf{[...] kundedrevet virksomhed}-sektionen nedenfor, da den også kan bruges for sig og i andre sammenhænge. Jeg vil forslå, at man læser denne sektion, men basalt set går idéen altså ud på, at ...
	
	En anden idé, man også kunne bruge, er ...
	
	
	
	%Skal jeg huske at få noget med omkring mere semantiske Git-prjokter og særligt så med mapperetningslinjer og begrænsede actions, men tage tage som med programmør, eller bliver det for tangentialt? 
}











\phantom{\\\\}
\phantom{\\}
(05.01.21) Den følgende tekst er fra da jeg troede (hvortil jeg jeg lige har skiftet mening nu), at min ``wiki-side-idé'' skulle have sin egen sektion og introduceres som en ret selvstændig idé. Men nu synes jeg, at den hører bedre hjemme som en forlængelse af min ``folksonomy-idé.'' I modsætning til hvad jeg har gjort før, tror jeg dog ikke, at jeg bare vil beholde teksten i sektionen, og så lave endnu en ``jeg starter forfra''-sektion, for idéerne skal nok flettes lidt ind ift.\ de punkter, jeg ellers har sidst i sektionen. I kommentarerne (i kildekoden) til denne tekst kan man (som så ofte) også se de tanker og idéer jeg gjorde undervejs, imens jeg skrev det. 
\\\\
{\slshape 
\textbf%[Wiki-side m.m.]
{Videre til tanker omkring wiki-side m.m. (så dette er altså en undersektion til ``forfra igen igen''-sektionen ovenfor)}\\
Når jeg kigger tilbage, ser det ud til, at det er starten af ``\textbf{Ny tilgang\ldots}''-sektionen samt starten er første ``\textbf{Ny ny tilgang\ldots}''-undersektion (hvis man ikke tæller mine udkommenterede noter med\ldots), som opsummere tankerne indtil videre\ldots\ Ah, der bør næsten også være andre steder\ldots\ Nå, men lad mig bare opsummere dem igen her, så alligevel.

Hvis vi tillader os selv at bygge videre fra forrige idé, så kan man sige, at denne idé handler om en teknik til at samarbejde om at redigere tekster --- eller andre ressourcer --- i et stort fællesskab. Idéen gør så i høj grad også brug af, at brugerene skal rate tester/tekstudsnit (/ressourcer/ressource-udsnit) med semantiske prædikater og relationer (svarende til at rate ``tags'' folksonomy-idéen). Og idéen kan så ligeledes også med stor fordel gøre brug af min `brugergruppe'-idé (såvel som andre algoritmer til at fordele stemme- og/eller tillidsvægte til brugere i netværket). Udgangspunktet for denne idé er bare et andet; den handler om vidensdeling frem for folksonomies og brugeranmeldelser/-vurderinger. Jeg kunne derfor også have startet med denne idé, og så have introduceret mine idéer omkring at system, hvor brugere i høj grad giver vurderinger af prædikater/relationer (om end vi tænker på dem som ``tags'' eller ej), og om stemmevægt-fordeling og ``brugergrupper'' i denne sammenhæng. Og hvis jeg så forklarede om folksonomy-idéen bagefter, så kunne jeg altså have skåret hjørner her i stedet for, som nu, ved denne idé. Men nu har jeg altså valgt denne rækkefølge, og så vil jeg altså ikke gentage de samme pointer her, som jeg allerede har forklaret om i ovenstående sektion.

Idéen handler altså sagt ud om vidensdeling og om at redigere tekster --- og ja, også andre ressourcer i princippet, men lad mig bare holde mig til tekster for nu --- i et fælleskab. 
Inden jeg forklarer om den tekniske del af idéen (som egentligt er meget simpel), vil jeg gerne motivere idéen først ved at foreslå en ny form for Wikipedia-side (jeg kunne også have valgt andre vidensdelingssider, men Wikipedia er et godt udgangspunkt), som er mere alsidig ift., hvilke typer tekster den inkluderer. 
Lad os således forestille os en wiki a la Wikipedia, men hvor hver artikeltitel ikke bare er en tekst, der beskriver, hvad artiklen bør handle om, men også indeholder prædikater ad libitum omkring, hvad der forventes af tekstens indhold og udformning. Et muligt prædikat kunne så eksempelvis bare være: ``bør leve op til standarderne for en almindelig Wikipedia-side,'' og på den måde kan denne wiki-siden sagtens indeholde alle Wikipedias artikeltekster som en undermængde. Men man skal altså også kunne oprette artikeltitler med alle mulige andre former for prædikater tilknyttet sig. Man behøver ingen gang at være bange for misinformation og ømtåleligt indhold i nogen højere grad end andre platforme, for man kan bare gøre det til en standard-indstilling, at brugere kun vises artikler med vise begrænsende prædikater på sig, og altså gøre så at brugere skal ændre avancerede indstillinger for overhovedet at få sådant materiale vist. 
Eksempler på hvilke ``tekstprædikater'' man kunne gøre brug af på sådan en side, ...%(kommer så her...)

Hvem skal så vurdere, hvilke prædikater skal sættes på hvilke artikeltekster? Det skal brugerne selvfølgelig ved at rate disse prædikater, helt i tråd med, hvordan det fungerede for folksonomy-idéen. Her er tanken bare, at artikeltitlerne i sig selv skal være faste, men at artikel\emph{teksterne} så tilgengæld ikke er knyttet fast til nogen specifik titel og i stedet bare knyttes til titlerne gennem relationer. Og det er så de relationer, som brugerne skal vurdere. Dette kunne så gøres ved, at brugere bare for hver tekst og hver titel i princippet afgiver vurderinger om, hvor godt titlen (med tilhørende prædikater) passer på teksten, men man kunne også gøre noget lidt smartere. %som jeg faktisk først lige er kommet i tanke om nu her, imens jeg har skrevet dette.
Man kan nemlig også gøre mere som for folksonomy-siden, hvor alle tekster %...Nå nej, vent.. Bryder dette ikke så med hele min idé omkring de sammensatte tekster osv..? ..Nej, det gør det jo ikke.. Nej, fint.
kan tilføjes prædikater og relationer, eller ``tags'' om man vil; det er også fair at bruge det udtryk, og at tænke dem som sådanne. Der skal dog stadig være faste titler, hvilket bliver forklaret, når vi når ned til de nye tekniske detaljer om idéen. Men så kan man nemlig bare implementere en algoritme på siden, der givet en artikeltitel (inkl.\ prædikater) finder frem til den eller de bedst mulige forslag til artikler, der opfylder disse prædikater (og handler om samme ting), og enten bare viser den bedste, eller viser nogen forslag, som brugeren så kan vælge imellem. Og her må det så selvfølgelig meget gerne på sigt blive sådan, at brugeren kan justere indstillingerne på, hvordan den algoritme, der bestemmer disse relevans-scorer, fungerer, og særligt skal man f.eks.\ gerne kunne udvælge `brugergrupper,' så algoritmen altså således kan bruge vægtede pointaggregater i sine udregninger (i stedet for bare gennemsnittet på tværs af alle brugere).

Jeg kan forresten lige bemærke, at artikeltitler almindeligvist vil denotere et specifikt emne, begreb eller en specifik ting. Dette kan jo i princippet så også oversættes til et ``handler om''-prædikat (som muligvis kunne have flere varianter alt typen af, hvad det handler om). Dette kunne endda gå hen og blive ret smart, når vi når til et punkt, hvor alle emner og alle ting/begreber kan opsættes i en samlet kategori-model/-graf(/-ontologi?), for hvis der så ikke er noget passende hit på lige den titel (inkl.\ --- eller så måske bestående af --- tekstprædikater), man har søgt på, hvis man absolut holder sig til det præcise emne, eller den præcise ting, så kan algoritmen måske forsøge at se omkring emnet/tingen og måske søge på beslægtede begreber/ting/emner og/eller søge på mere generelle emner og/eller overkategorier. 


Okay, og den tekniske del af idéen handler så om, faktisk at gøre mulighed for at tekster kan opdeles i lag --- og så også lægge op til at brugerne skal påtage en konvention om at gøre dette i høj grad. Denne lagdeling handler i princippet bare om at dele tekster op i dispositioner og tekst-moduler, som kan indsættes på dispositionspunkternes pladser. Hvert tekstmodul kan så i princippet selv være en lagdelt tekst bestående af disposition og tekstmoduler selv, eller det kan også bare være en atomar tekst. Idéen ligger så i at strukturere nævnte dispositioner lige netop ved brug af vores tekstprædikater. Og det smarte ved at gøre dette er, at så kan wiki-siden lige netop kan søge på de tekstmoduler, der passer bedst på pågældende tekstprædikater --- og i henhold til brugerens egne indstillinger for søgealgoritmen --- og kan så sætte dem ind automatisk (eller evt.\ prompte brugeren om at vælge, hvis nu der er flere muligheder med næsten lige gode relevans-scorer). Bemærk så også, at artikeltitlerne på denne måde så også kommer til at fungere helt ligesom sådanne dispositioner --- bare monadiske dispositioner så at sige, som altså kun består af ét modul. Og denne ligestilling er så mere end bare en sammenligning, for der er så ingen grund til, at større tekster ikke skal kunne indeholde, hvad der i andre sammenhænge kan ses som en selvstændig artikel, som et indre tekstmodul, og altså som en del af dets disposition. 

*Så da jeg skrev ovenfor, at titlerne (med prædikater) ligesom skulle være faste objekter, så er det altså derfor: Når man navigerer til en artikel via dens titel på siden, så navigerer man altså først og fremmest til en (monadisk/unær) disposition, og selve teksten man så får vist afhænger af, hvad der passer bedst til den titel. 

Jeg kan nu lige komme med nogle hurtige kommentarer om, hvordan man kan implementere disse lagdelte tekster, som jeg også har kaldt ``sammensatte tekster'' før i dette dokument, og hvilket man altså også kunne kalde dem. \ldots Vi kunne også kalde dem ``modulære tekster'' simpelthen. Ja, det lyder meget godt. (Og ellers kan man også kalde dem ``modulært opbyggede tekster.'') Og vil jeg ellers til gengæld ikke gå mere i dybden med tekniske overvejelser end dette. Modulære tekster kan så enten implementeres via en særlig syntaks, man beslutter sig for (hvor man så altså skriver en `disposition' som en tekst med denne særlige syntaks, og så kan opbygningen samt alle tekstprædikaterne for de forskellige moduler så parses fra denne), eller man kan implementere det mere abstrakt som en del af et HOL-system, hvor dispositionerne så kan tage form som tupler af prædikater essentielt set. Okay, og det var altså bare lige det, jeg ville påpege om dette.


Det næste, vi så lige bliver nødt til at se på, er, hvad det vil sige at vurdere en disposition, for hvordan kan man f.eks.\ sige, at en disposition er ``letlæselig'' eller ``uddybende for emnet,'' hvis den i sig selv ikke indeholder noget brødtekst overhovedet? %Her skal man s... %skal man foreslå forskrifter (lidt ligesom mine "automatiske point") allerede her, så..?? ... ..Okay, alternativet ville vel være, bare at give pointene ikke-konditionelt.. Hm.. ..Og hvordan ville det virke, hvis man skulle lave forskrifter..? Så ville man jo skulle rate sig frem til dem.. ..Hm, i høj grad vil der jo være tale om ét prædikat, hvor man bare skal give en vægt for, hvor meget børnenes rating af prædikatet bidrager til forælderens rating.. Hm, men det er da netop også et rigtigt godt udgangspunkt..! Og det næste skridt er så bare, at lade flere forskellige prædiakter være input til et aggregeret prædikat for forælderen. ..Og dette må vel næsten være rigeligt så..? Hm, eller skulle man også lige kunne stemme om en kurve --- måske en logistisk-agtig kurve..? Ja, det kunne jeg også foreslå.. Selvfølgelig vil det i mange tilfælde være totalt overkill at bruge energi på at stemme om sådanne forskrifter, men hvis det er virkeligt store projekter, hvor der kan ske meget, så kan det give god mening.. ..Men ja, det med at stemme om vægte for hvert barn (og muligvis om flere børneprædikater) for et forælderprædikat, det kunne da give god mening.. Ah, og man kunne måske bare gøre det til en standard, at man så via disse vurderinger justerer en logisktisk kurve (ved at rykke midtpunktet). ..Hm, og måske skal dette så ikke så meget bruges som "vægte," men måske skal man i stedet mere tage et minimum..? ..Ja.. 
%Jeg vil i denne forbindelse foreslå, at man faktisk for hvert givent prædikat, som... %Vent, der kan jo faktisk være to ting, der er interessante at stemme på. For det \emph{er} nemlig også interessant at vurdere, om en disposition er god eller ej givet at tekstmodulerne bliver udarbejdet ordentligt (hvor man altså ikke antager, at tekstmodulerne \emph{kan} udarbejdes som prædikaterne foreskriver --- f.eks. duer det ikke, hvis man bare siger teskt der argumenterer for at udsagn p er sandt, hvis man ikke nødvendigvis kan antage, at p er sand --- men hvor man altså vurderer, om tekstmodulerne med deres prædikater er realistiske, og hvis de er, så vurderer, hvor god dispositionen er, givet at tekstmodulerne så bliver udarbejdet og indsat ordentligt).. Hm, og man kan ikke blande disse to..? ..Hm, måske hvis man kan lave en rating af, om de forespurgte tekstmoduler i dispositionen er realitsike.. ..For hvis de så alle vurderes til at være det, så kan man jo bare se på topscoren.. nej.. Nej, så vil det være bedre, at vurdere en sidste vægt, der siger hvis.. hm, men så skal man stadig også have en realistisk-eller-ej-vurdering.. Men ja, man kunne altså have en ``er tekstmodulerne realistiske''-vurdering, en ``hvis de er realistiske og udarbejdet til fulde, hvor god er dispositionen så''-score, samt en.. ja, samt en forskrift, som kan justeres via flere små vurderinger (for hvert tekstmodul i dispositionen og evt. for flere end bare et prædikat (som som regel bare for ét prædikat pr. modul)), som så, vægtet i sidste ende med nr. 2 nævnte score her (og man behøver nemlig så ikke at inkludere ``er tekstmodulerne realistiske''-vurderingen er, for det kommer så af sig selv i algoritmen), fortæller hvor god en tekst dispositionen resulterer i, når man så indsætter et sæt af tekstmoduler (hvilket så typisk netop vil være det sæt, der maksimerer denne score). Ja..? ..Hm, men man for så ikke så meget brug for førstnævnte score andet end til at vurdere nye dispositioner.. Hm, skulle man så ikke blande de to (første) sammen, eller hvad kan man gøre..? Nej, det er fint at have det hver for sig, og så kan førstnævnte score bare være en form for advarsel, som man normalt ikke skal tænke så meget på, men hvis man ser at den er ratet højt for en disposition, så må man i hvert fald lige se ad, om der er hold i dette, og hvis der er, og hvis denne advarselsscore så bliver ved med at være høj (selvom der kommer flere og flere afgivne stemmer), så må dispositionen så ``kasseres'' af fællesskabet (fordi folk så vil have søge-algoritmer/-indstillinger, der vil nedprioritere den stærkt; og platformen kan så spørge forfatteren, om de må fjerne den fra databasen, men sådan noget gider vi ikke bekymre os om nu her; det hører til (implementerings)detajlerne). Ja, lad mig foreslå det sådan.. 
Her skal vi holde tungen lige i munden, for en vurdering af en disposition kan jo betyde forskellige ting. Hvis en disposition bliver vurderet til at være ``letlæselig'' eller et andet prædikat, kan det f.eks.\ betyde, at hvis fællesskabet ellers bruger en rimelig tid på at udarbejde alle tekstmodulerne, der passer til alle de specifikke sektionsprædikater i dispositionen, så vil man forvente at den resulterende tekst opfylder prædikatet (f.eks.\ ``letlæselig''). Graden af vurderingen/ratingen vil så altså sige, hvor ``letlæselig,'' eller hvad det kunne være, den resulterende tekst ved indsættelse dispositionen har mulighed for at blive. Men når vi taler om ``en vurdering af en disposition'' kunne vi også i stedet tale om en vurdering af den bedste tekst, som dispositionen kan resultere i på \emph(nuværende) tidspunkt (og altså med de til den tid eksisterende tekster). For ikke at forvirre disse to forskellige ting, så tror jeg, jeg vil kalde det førstnævnte for dispositionens ``prædikat-vurdering'' og det andet for ``prædikat-scoren.'' 

Angående ``prædikat-scoren'' så ville den mest simple løsning her bare være et tage minimumsvurderingen fra de indsatte tekstmoduler, gange dette med dispositionens prædikat-vurdering (hvis denne løber fra 0 til 1 --- eller kunne man også finde på andre funktioner for at sammensætte modul-scoren og dispositions-vurderingen til en endelig dispositions-score) og så lade dette være den resulterende dispositions-score. Jeg tror faktisk, at man i størstedelen af tilfælde vil kunne komme rigtigt langt med denne simple løsning, og at dette derfor vil fungere fint langt hen ad vejen *(ah, det er måske ikke helt rigtigt alligevel\ldots). Man kunne selvfølgelig dog også finde på mere avancerede løsninger, f.eks.\ %hvis nu visse sektioner er mindre vigtige for prædikatet end andre ift.\ teksten som helhed, eller 
hvis nu man måske gerne vil aggregere flere forskellige modul-prædikater til et samlet prædikat.  %Sidstnævnte 
Dette 
kunne f.eks.\ være, hvis man har et prædikat så som: ``teksten lever op til den og den standard,'' som vi var inde på før (hvor vi snakkede om standarden for Wikipedia-artikler). Dette vil jo typisk involvere flere forskellige (mere atomare) prædikater omkring teksten, og her kunne det altså måske være smart så, hvis man automatisk kunne aggregere disse prædikat-score til en overordnet ``følger den og den standard''-score. %Lad mig derfor lige bruge lidt tid på at foreslå nogle mere avancerede løsninger, hvor man også kan tage højde for disse ting. Angående sidstnævnte forhold, kunne man jo eventuelt så bare have det, så man kan 
En løsning her kunne være, hvis man kunne 
indstille en global forskrift for, hvordan de indgående prædikaters score for omregnes til en score for det pågældende aggregat-prædikat, hvis vi kan kalde det det. 
Man bør dog altid gange (eller hvad man gør aritmetisk) dispositionens vurdering på til sidst på den samme måde som i alle andre tilfælde, uanset hvad man gør. 
%Og angående førstnævnte forhold, så kunne man måske have flere små ratings for hver tekstmodul i en disposition, hvor brugere (for hvert prædikat) kan vurdere, hvor vigtige modulerne er, når prædikatet for helheden skal gives en score. Hvis vi nu f.eks.\ stadig bruger minimummet, som foreslået ovenfor, så kunne man så f.eks.\ justere et offset for hver sektion, så man altså på nogen sektioner kan give lidt margin (så det ikke trækker scoren ned, hvis modul-scoren er inde for denne margin). Og ja, man kunne også finde på endnu mere avancerede ting, men jeg tror allerede at dette er rigeligt (og måske mere end rigeligt --- især med det sidste forslag her om ratings for hvert modul), for det vil nok alligevel kun være for meget få dispositioner, hvor det overhovedet giver mening at bruge så meget energi på scoren. \ldots Ja, det er endda lige før, at jeg bare skal droppe at nævne det med at vægte sektionerne\ldots\ Problemet er lidt bare at ``dispositioner'' nok tit kommer til at indeholde overskrifter også\ldots\ Eller, vent! Ah, pjat! For det første, så kan ``sektioner'' godt inkludere deres overskrifter (og deres omgivende struktur i det hele taget), men ikke nok med det, man kan også sagtens bare udforme prædikaterne sådan, at de tager højde for, hvilken type tekst der vurderes. For en tekst ved \emph{godt} selv, om den er en overskrift eller ej! Det kan man i hvert fald sagtens sørge for. Ja, never mind. Nu udkommenterer jeg dette, og så sletter/udkommenterer jeg også mit forslag omkring det.
%Hm, men man kunne måske også få gavn af, hvis tekster kan vurderes ift. en kontekst..!.. Hvis nu man f.eks. kunne wrappe dem i en monade (måske i form af en slags "disposition"), som fortalte konteksten..? ..Hm, at kunne vurdere en tekst antaget en kontekst... ..Ja, antagelser! Man bør næsten kunne sætte antagelser på sektionsprædikaterne i dispositioner, og så man det netop bare være op til førnævnte advarselsvurdering, nemlig ``er tekstmodulerne realistiske''-vurderingen! Nice! ...Ah ja, og så kan man måske netop også bruge kontekst-antagelser i dispositionsvurderingen, som så, når man bruger teknikken med bare at tage minimum af alle sektioner, så alligevel kun tager de sektioner, der har pågældende kontekst-antagelse, som så nemlig også bl.a. kan præcisere sektionens funktion. 

\ldots Okay, jeg kan se, at jeg alligevel skal omstrukturere ovenstående paragrafer en del, for det er meget rodet forklaret med den nuværende struktur. Men det gør ikke så meget, for så meget er det heller ikke, der skal forklares. Så jeg tror bare jeg vil færdiggøre disse brainstorm-noter, så teksten bare hænger nogenlunde sammen (og så alle pointerne bare er forklaret), og så kan jeg omstrukturere det, når jeg skal skrive en renere tekst over det.

Lad mig derfor også bare lige færdiggøre det tekniske ting nu her, inden jeg gør tilbage til noget af det mere overordnet forklarende og motiverende. 

Der er nemlig også et andet forhold, der kan gøre, at man måske også gerne vil have mere avancerede prædikat-score-algoritmer, og det kan være, hvis dispositionen ikke bare består kun af ligeværdige tekstsektioner, men består af andre ting også. Dette kunne være såsom faktabokse, figurer, opgaver, kildereferencer (måske for hvert kapitel) osv. Der er endda ingen, der siger, at dispositionsmodulerne ikke også skal kunne være selve titlerne \ldots\ %Hm.. ..Åh, jeg skal vist lige tænke lidt.. Jeg føler pludesligt, at der er et område her, hvor jeg ikke helt kan bunde: Det er lidt meget at skulle prøve at regne ud, hvordan dispositionskonventionerne bliver, og hvilke behov der bliver.. Ah pjat, mon ikke jeg godt kan tænke mig til et nogenlunde overblik. ... 
%Okay, jeg er kommet frem til nogle ting, men jeg er muligvis ikke helt færdig med tænkeriet. Den store ting er, at man godt bare kan bruge en forskrift a la den, jeg skrev om ovenfor, til at skelne mellem vigtigheden af forskellige typer tekstmoduler, for disse moduler vil så have prædikater om sig, alt efter hvad de er... hm, eller vil de nu også det, for der er jo stadig problemet med, om en tekst selv kan vide, hvilken funktion den opfylder i en større tekst.. ..Hm, men kan man så ikke sætte struktur-prædikater i dispositionen.. eller kunne man ikke bare bruge prædikater, hvor man siger, denne teskt \emph{kan} vartage den og den funktion..?.. (Ej, jeg er af en eller anden grund også langsom i dag..) ..Hm, okay, der vel lidt to muligheder.. Man kan bruge generelle forskrifter, der virker på tværs af dispositioner, og/eller man kan bruge forskrifter, som er en del af en given disposition hver især.. ..Og ja, så kunne man måske netop bruge ``\emph{kan} varetage funktion''-prædikater.. ...Uh, men vent, hvis man nu kan have dispositions\emph{skabeloner}, så behøver man måske ikke de "generelle forskrifter"..! Hm..(!) ..Ja, og med kontekst-antecedenter også, så kunne det da blive ret godt.. ..Og tilsvarende kontekst-prædikater (som egentligt er relationer) for dispositionsmodulerne, også når det kommer til dispositionsskabeloner.. Hm, jeg har det som om, jeg alligevel har færden af noget her..!.. ..Ja, simpelthen. Det giver jo lige \emph{netop} mening at bruge tid på at lave forskrifter til dispositioner, hvis de kan bruges i mange forskellige sammenhænge. Fedt! Og så tænker jeg altså, at man rater de bedste forskrifter for automatisk at danne en score for et givent prædikat for sådan en disposition-skabelon. Og man kan så bare nøjes med, at dette skal gøres enkeltvis for hver skabelon; man behøver ikke at indføre noget generelt.. Tja, man kunne dog foreslå at lave et system med skabelonsklasser, hvor der så kan være inheritance.. :) Ja, fint. ..Det er faktisk virkeligt nice, det her.. (Dispositionsskabeloner bliver \emph{så} brugbare..!) ..Uh, og alt det med kontekst (hvor jeg nemlig har overvejet, hvad man f.eks. skal gøre, når man vil sikre sig, at en tekst ikke antager, at noget er introduceret, som ikke er det, og hvordan man sikrer, at udskiftelsen af en tidlig sektion ikke forhindrer forståelsen af følgende sektioner), så kan man bare klare alt dette med relationer, som dispositionsskabeloner kræver, skal være opfyldte! ..Hm, men er der en måde, hvorpå brugerne/forfatterne så kan reducere disse krav til mere specifikke ting (så man f.eks. kan præcisere, hvad en specifik sektion kræver, er introduceret før den selv)?.. ..Ja, hvis man kan oprette konditionelle tags/relationer. Så hvis man f.eks. har en sektion, hvor det skal vurderes og godkendes, at den ikke antager kendskab med noget, der ikke er introduceret endnu i den samlede tekst x men bør være det, så kan man vurdere at ``den ikke gør dette, så længe at emnerne/tingene/begreberne, y, z, ..., er introduceret forinden. Hermed skal man så kunne reducere, eller rettere forlænge, spørgsmålet til to vurderinger, nmelig om konditional-sætningen er sand og så om y, z, ... er introduceret forinden pågældende sektion i den samlede tekst. Dette kræver vel så, at dispositionsskabelonen kan indeholde relationer til hver sektion og så med den ovenstående tekst.. Ja, så man skal gerne have sådan nogle muligheder for på denne måde at strø relationer ud over sektionerne for en dispositionsskabelon. ..Hm, bør en relation så kunne vide, om den f.eks. er en konjunktion eller en disjunktion imellem de indre funktioner.. eller skal man bare have dette som en del af mulighederne i disp-skabelonen, så der i princippet kan blive O(n²) relationer i en tekststruktur, men hvor man måske med fordel kan lade være med at gemme disse i hukommelsen/lageret, men hvor algoritmen bare nøjes med at folde dem ud i forbindelse med automatiske tjek. Med andre ord kan der godt være O(n^2) relationer i en tekst i princippet, men for læsere vil de altid være forkortet til O(n) relationer, og de vil så aldrig skrives ud eksplicit, fordi de så ikke er beregnet til at blive vurderet af brugere, men kun er beregnet til at blive tjekket automatisk. ..Coo-ool! (Hvor er det fedt, når tilsyneladende "dovne dage" ender med at blive så givende! ^^) (..Og ja, jeg ved godt, at jeg sikkert har tænkt noget tilsvarende før, men derfor er det stadig super vitigt så at huske sådan nogle ting her igen i tide. :)^^) (..Føler dog umiddelbart, at dette er min skarpeste version af dispositionsskabelons(/tekststruktur)-idéen hidtil.. :)^^) ..Jeg skal så lige sørge for, at introducere idéen om at kunne tjekke prædikater/relationer som antecedenter til andre prædikater/relationer (for så automatisk at give konsekventen en positiv score). Men dette er vel rimeligt simpelt, for konsekventen kan så bare få en score a la antecedentens og konditionalens scorer ganget sammen (hvis det løber fra 0 til 1).. Ja, det må næsten bare være sådan --- det bliver i hvert fald det, jeg foreslår.. ..Ej, og nu \emph{kan} man faktisk argumentere rimeligt godt for, at idéen har et ok potentiale for at kunne bruges til programmering..!! ^^ ..For ja, det svarer vel til at indføre muligheden for at have kode-/programskabeloner med tjeklister, som er opdelt i mange små spørgsmål, og som så automatisk kan tjekkes for, om tingene er opfyldt for at skabelonen fungerer.. Og disse spørgsmål skal så ikke være maskinforståelige, men skal bare rates hver især (så mange af dem, som der er behov for; der kan jo bl.a. (måske) være disjunktioner, så man kun behøver ét godt svar..) af programmørerne (som i øvrigt kan have forskellige vægte; man kan jo bruge "brugergruppe"-vægte)..:) ..Og noget andet, som jeg ikke har nævnt endnu, men som jeg har i sinde at nævne i teksten, er, at programmering så også kan blive mere intuitiv, fordi designerne så bare kan bevæge sig på de mere abstrakte niveauer i koden (men hvor de stadig tvinges til at forklare tingene ret tydeligt (mere tydeligt, end de sikkert ofte gør), og hvor de så faktisk også letter programmørernes arbejde i de lavere niveauer, fordi disse så ikke skal designe de mere overordnede sammenhænge)..(!).. Hm.. (..Og dette svarer så til "intentional programming (paradigm)"-idéen..) ..Hm, hvor meget giver det mening, at forklare om for denne side af det..? ..Hm, ikke vildt meget, men jeg kunne jo nævne, at man i en god open sourve-verden, så bør ordne programmeringsløsninger ligesom for min folksonomy-idé (eller som for mange andre idéer, bl.a. det semantiske web), hvor alle skabeloner er semantisk kategoriserede, så de er lette at søge på, og så kan brugere vurdere, hvor godt løsningen løser forskellige ting. Og så kunne jeg måske nævne bevægelse af PC'er i et computerspil som et eksempel. Ja, lad mig bare komme lidt ind på de visioner --- det er kun oplagt, når nu jeg netop alligevel vil tale om web of applications-visionen overordnet set. Ja. 
%(03.10.21) Det er fint, det jeg skrev her i går, men det med automatisk at reducere prædikater osv., det bør nu bare nævnes som en ekstra-ting, og altså høre til noget avanceret, man måske kan gøre på et tidspunkt.

Ah, nu har jeg en stor forbedring til idéen (i forhold til hvordan jeg var ved at præsentere den)! Glem, hvad jeg skrev om at tage minimumsscoren fra modulerne og føre den videre, glem, hvad jeg skrev om generelle forskrifter for et specifikt prædikat. 
%prædikat-scoren indtil videre (på nær hvad konceptet går ud på og sådan\ldots)! 
Og lad mig så forklare om \emph{dispositionsskabeloner}! 

I mange tilfælde kan de være gavnligt, hvis man kan have skabeloner for dispositioner. Dette kunne f.eks.\ være, hvis man skulle skrive en artikel eller en lærebog i fællesskabet. Her kunne man så have en dispositionsskabelon, der f.eks.\ siger, at ``der skal være en kildeliste til sidst i kapitlerne,'' hvis vi tænker på lærebogen, eller ``der skal være mindst én opgaveboks i hvert kapitel,'' eller alle sådan nogle ting. Og skabelonen kunne selvfølgelig også tage sig af det mere basiske som at ``der skal være en margin det og det'' (antaget at det er skrevet i et markup-sprog såsom HTML eller \LaTeX) eller ``alle kapitler skal have en titel'' osv.\ osv. Her kan man komme rigtigt langt med en syntaktisk definition af skabelonerne. En dispositionsskabelon kunne således helt grundlæggende bestå af en syntaktisk sprogdefinition (hvilket f.eks.\ kunne være en regex, men det ville dog også være gavnligt at indføre mere abstrakte udgaver, så selv folk, der ikke kan HTML, \LaTeX\ og/eller kender regular expressions, kan være med og kan designe og justere skabeloner til værker (artikler, bøger etc.)). Man burde så også gøre det sådan, at folk, der arbejder på et værk, som følger en vis skabelon (\ldots og nu indser jeg lidt, at det måske ikke er så klogt at kalde det ``dispositionsskabeloner,'' for det bliver egentligt ret forvirrende; navnet passer faktisk ret dårligt) --- ja, som vi måske bare burde kalde tekstskabeloner i stedet --- så kan indsætte flere kapitler og/eller sektioner, paragrafer, figurer osv.\ på en ret WYSIWYG måde\ldots\ %Ja jo, men så skal restriktioner såsom at der skal være mindst én af noget\ldots\ Nå nej, det går faktisk fint. Men ja, pointen er så, at sprogdefinitionen så meget gerne må... %Hm lad mig lige tænke lidt over det her.. ..Hm, man skal jo netop dele det op i lag, så man kun definere den yderste struktur først.. ..Ah ja, så never mind. Man kommer aldrig til at ændre på andet end.. Hm.. ..Ah vent, syntaksdefinitionen skal da ske via prædikater! Selvfølgelig..! Ja, og så skal syntaksprædikater altså indeholde en syntaksdefinition over gyldige input-dispositioner.. Så et tekstmodul kan have ne række normale prædikater og et syntaks-prædikat, og hvis den har sidstnævnte, så skal input være en disposition selv, der overholder syntaksen, må det ikke være sådan?.. ..Hm, og et spørgsmål bliver så, om dispositioner selv skal vide, om de følger en bestemt syntaks (og så kan de måske gemmes som automaton-input i virkeligheden.. måske), eller om syntaksen bare skal tjekkes? Hm, jeg hælder faktisk til det sidste.. Ja. (Og så kan altid automatisere og hjælpe brugeren i at holde sig til syntakserne efterfølgende (ved at tilføje smarte funktioner til ens redigeringsprogram..).) 

Okay, vent lige to sekunder\ldots\ Never mind, det med WYSIWYG for nu. Og det er så gået op for mig, at vi bør snakke om ``skabelon-prædikater'' i stedet. Disse prædikater indeholder en syntaktisk definitioner over, hvad input skal overholde, som så i dette tilfælde skal være et dispositionsobjekt (hvis jeg holder mig til at kalde disse ``dispositioner'' (og lad mig bare gøre det for nu)) og altså ikke kan være atomare tekster. Og disse definitioner skal så kunne tjekkes automatisk, så søgealgoritmen kun søger efter og indsætter underdispositioner, der overholder syntaksen. Andre skabelon-prædikater (eller vi kunne også kalde dem syntaks-prædikater\ldots\ men skabelon-prædikater er der nu nok flere, der forstår) kan så potentielt være en del af, hvad der kræves for den indsatte disposition, og på den måde kan en skabelon altså komme til at definere en hel graf for, hvad værket skal overholde, men hvor det hele stadig er lagdelt og modulært, så at alle undersektionerne kan tjekkes for sig. Og ja, jeg har jo allerede nævnt nogle eksempler på, hvor det kunne være smart med sådanne faste skabeloner for værker. Man kan så starte med et skabelonprædikat som et helt ydre prædikat (og altså som en del af ``titel-prædikaterne'' til værket). Alle dispositioner og underdispositioner af værket kan så udformes med de frihedsgrader, der er, men skal så altså overholde en vis struktur, der kan tjekkes automatisk. Bemærk, at skabelons-prædikater kan fastsætte alle mulige prædikater for de indre dispositioner, ikke kun andre skabelonprædikater. Så skabelonerne kan altså også diktere alle mulige forskellige prædikater, som de indre moduler skal overholde, hvilket så selvfølgelig ikke tjekkes automatisk (når der ikke er tale om skabelonprædikater), men hvor brugerne rater og dermed godkender prædikaterne. Jeg vil endda forslå, at skabelonerne også for mulighed for at definere relationer \emph{imellem} modulerne, så man også kan kræve noget om sammenhængen mellem de forskellige moduler, men dette er en mulighed, der måske godt kan vente lidt med at blive implementeret. 

Og lad os så spørge os selv: Hvordan skal det afgøres, hvornår prædikater er opfyldt, når de ikke kan tjekkes automatisk, men bare skal vurderes af brugerne? Jo, vi har jo været lidt inde på, at der skal være en ``prædikat-score,'' men nu bliver det meget nemmere, når vi har vores skabelonprædikater. %..Hm, hvordan skal det så egentligt fungere? Skal brugeren bestemme, hvad der gælde for en skabelon, eller skal skabelonen altid bestemme det..? Hm, burde man egentligt ikke bare adskille det og have en separat type af prædikater til at bestemme score-bestemmelsen..? ...Hov vent lige.. Med skabeloner, behøver man så overhovedet automatiske scorer..??.. ...Uh, kunne man ikke bare nøjes med score-forskrift-ratings for hvert enkelte værk, men så bruge score-skabeloner (som så kan hænge sammen med skabelonerne og tage udgangspunkt i deres syntakselementer..)..? ..Ja, så det ville altså sige, at skabelonerne ikke bare har én score-forskrift, men kan tilføjes flere forskellige, og så kan folk opvurdere en score-forskrift for enkelte værker, og ligesom for kategori-relevans-vurderingerne osv., så kan den underliggende algoritme så også evt. sørge for at forslå de mest populære score-forskrifter for den pågældende skabelon, når et nyt værks disposition oprettes.. ..Og her, og ved lignende tilfælde generelt (altså f.eks. kategori-relevans-vurderingerne), kan man måske implementere dette som et offset i relevansscoren..? ..Ja.. ..Jamen er det så ikke bare det, man skal gøre?(!).. ..Jo, lad mig gå ud fra det for nu..
Prædikat-scorer skal så nemlig knyttes til %..Hm, men bliver det egentligt ikke svært at blande skabeloner sammen..? ...Hm, måske hjælper det, hvis skabelon-prædikater ikke behøver at være udefra-og-ind, men hvor "indre skabeloner" måske godt kan føje noget til den ydre struktur..?.. ..Nå nej, hvorfor behøver man at blande flere skabeloner sammen.. hm, det skulle vel netop være, så man kan foreslå de rette score-forskrifter..?.. ..Hm, men kunne man ikke godt dele skabelonerne op i lag også..? ..Hm, kan man have sammensatte syntakser, eller bliver det for kompliceret..? ..Ah, man kan måske have skabeloner, der har frihedsgrader i sig, og hvor den næste skabelon kan tage de frihedsgrader og definere yderligere syntaks for dette..??.. ..Ja, så at man på den måde kan lave arvelighed mellem skabeloner.. Ja, nu hvor jeg siger "arvelighed," så kan man jo sige, at selve idéen om arvelighed, hvor børnene så kan arve score-forskrifterne fra forælderne også, jo må være en ret god idé, og så er det bare et teknisk spørgsmål, hvodan dælen man lige implementerer sådan en arvelighed mellem skabeloner. Og min idé her med, at barneskabelonen kan tage elementer/produktioner for forælderen og så smække flere restriktioner på, kunne så være en god mulighed.. ..Ja, og det kan jeg jo så foreslå som en avanceret ting.. 
disse specifikt. Nu bliver det vist lidt teknisk, så lad mig prøve at uddybe, hvad et skabelonprædikat er mere konkret. Det er en regex eller anden sprog-(syntaks-)automaton, der definerer en mængde af dispositionsobjekter (som altså i sig selv bare er en række prædikater, hvilket i øvrigt også kan inkludere andre skabelonprædikater). For at en tekst matcher en skabelon, skal teksten altså kunne produceres af pågældende regex/automaton, og hvis der så er nestede skabeloner, skal pågældende tekstmoduler så også hver især matche disse. ``Prædikat-score-forskrifter'' kan så tilknyttes en given skabelon, hvilket med udgangspunk i syntaksen definere, hvordan scorer fra de syntaktiske elementer (som altså tager form som prædikater til tekstmoduler) skal regnes sammen til en overordnet score for et givent prædikat. En sådan prædikat score kunne bl.a.\ sige, at for alle opgavebokse i teksten, skal gælde det og det omkring deres ``sværhedsgrad''-ratings (f.eks.\ at gennemsnittet skal være sådan og sådan, at der altid skal være nogle nemme spørgsmål i starten, eller hvad man nu kunne finde på). Denne forskrift skal altså så gå ind og måle på alle indeholdte opgavebokse i værket og aggregere dette til en samlet score for et prædikat, der jo passende netop kunne være omtalte ``sværhedsgrad''-rating (man må nemlig gerne genbruge tags/prædikater til forskellige typer tekster/ressourcer). Ja, bum. Jeg bør selvfølgelig også komme med andre eksempler, men dette var et ret godt eksempel. Andre eksempler kunne også være NSFW eller andre sprogstandarder, hvilket måske kunne gælde over det meste --- men måske man f.eks.\ kunne slacke på formaliteten i visse bokse. Det kunne også være ``forståelighed, hvis man har læst pensum $x$,'' hvor $x$ altså så kunne være en reference til et pensum. Og her kunne man så også nævne et eksempel, hvis man på et tidspunkt åbner op for relationer i skabelonerne, hvor man siger ``skal indgå i pensum $x$, eller skal være introduceret ovenfor i værket.'' Det skal så siges, at dette jo bare er et eksempel; måske kan man finde på endnu bedre eksempler, hvor man kunne ønske sig relationer.

Og hvem bestemmer, hvilke prædikat-score-forskrifter skal bruges? Det gøres ved at stemme på en forskrift for en given disposition (hvilken så typisk vil have en skabelon, medmindre den bare har grundskabelonen, der bare hedder ``en række af tekstmodul-prædikater'') i brugerfællesskabet. Noget smart er så, at ligesom vi har set det før ovenfor, f.eks.\ da vi snakkede om kommentar-kategorier, kan platformen så holde øje med, hvilke forskrifter (for givne prædikater (alle forskrifter definerer nemlig en score beregning for ét prædikat)) generelt er populære for en given skabelon. Herved kan platformen så sørge for at foreslå denne forskrift for dispositioner med samme skabelon (og måske altså ved at give den et lille point-offset i ratingen). 

En avanceret udgave af de her skabelonprædikater, som jeg gerne også bare lige vil foreslå, men som dog måske kan implementeres på et senere tidspunkt, er, hvis man åbner op for en mulig lagdeling for sådanne skabeloner også. Tanken er, at man så skal kunne oprette skabeloner, som viderebygger andre skabeloner, og den store pointe med dette er så, at disse så også kan arve forældrenes prædikat-score-forskrifter. For eksempel kan sådan noget som NSFW-tjek og andre standard-tjek jo betragtes som en ret grundlæggende ting. Det samme kan kravet om, at der skal være kilder, og at kilderne skal være passende for kapitlet og for det antagne tidligere pensum / læserens tidligere viden (\ldots ah, her for man jo også brug for relationer, så relationer skal nu alligevel forslås som en del af første pakke også (--- jeg skal bare ikke foreslå det med at reducere prædikater automatisk)). Så derfor kunne det være rart, hvis skabeloner kan opbygges i lag, så man ikke skal opvurdere alle sådan nogle standard ting for hver ny skabelon. Selvfølgelig vil et fællesskab jo kun bruge et begrænset antal skabeloner i deres værker, men der kan alligevel hurtigt blive en del forskellige, og ikke mindst vil skabelonerne stadig udvikle sig med tiden, så jeg mener altså, at man vil kunne spare energi, hvis man giver mulighed for, at man kan dele dem op i lag og benytte muligheden for `arvelighed.' Man kan sikkert implementere arvelighed på flere forskellige måder, men jeg føler, at jeg lige bliver nødt til at komme med et bud her. \ldots %..Hm, bør man så have "generelle skabeloner," som skal gøres åbne over for specifikation.. vi kunne også kalde dem "skabelonsklasser".. og så altså have.. skabelon-instanser, hvor man har defineret hele syntaksformlen for et værk..? Ja..
Ah, for det første vil jeg gerne forslå, at man indføre ``skabelonklasser,'' som altså er en slags generelle skabeloner, hvorfra man så kan oprette specifikke (rigide) skabeloner, man kan bruge på værker. Vi kan så passende kalde disse konstante skabeloner for skabeloninstanser. Fint, men spørgsmålet er så lige, hvordan man kan gøre dette teknisk set, hvor jeg altså synes, jeg bør komme med et bud her. \ldots Tja, men må jo bare netop designe automatonerne/regex'erne i skabelonklasserne, så de er åbne over for videre specifikation. Dette må kunne gøres ved at %omstrukturere produktionerne, så man udfaktoriserer særlige produktioner, som så gives et navn...
sørge for at strukturere alle produktionerne ... %monadiske...


%Ja, det holder, det her med skabelonerne. For skabeloner til at opbygge tekster vil generelt bare være rigtigt brugbart, og når man så har dem, ja, så vil det også være oplagt, at bruge dem til at definere score-forskrfter med, så man kan få mere levende og dynamiske værker, som ikke skal omvurderes hver eneste gang et modul ændres. Og herved bliver det så også oplagt, at platformen kan holde styr på, hvilke forskrifter er populære for hvilke skabeloner, og ja, i det hele taget kan man så passende genbruge forskrifter herved. Og det med, at man også måske kan implementere opdelte skabeloner og arvelighed imellem dem, der har så bare mulighed for at gøre det endnu mere nice.
%(04.10.21) Hm, nu er jeg dog lidt i tvivl om, hvor vigtigt det bliver med de automatiske scorer.. Hm, det kan vel måske fungere meget godt som en midlertidig rating for et værk, som stadig er i udvikling.. (eller som bare i det hele taget lige har fået ændringer).. Men ja, der må vel gerne i hvert fald også være en rating for specifikke kontante værker, som så kan vokse sig betydende, når værket har stået stille i lidt tid..?.. ..Hm, kom lige til at tænke på et eksempel, hvor en tekst eksempler tilpasser sig, hvad læseren har kendskab til.. Meget interessant om ikke andet.. ..Ah, vent, alt det her med scorer, var tanken egentligt ikke bare, at man så kunne bruge det til at vurdere dispositioner?..! For selve søge algoritmen, når man allerede har valgt en disposition, den kører vel bare top-down..? Hm, men den skal jo så vurdere, hvor gode diverese underdispositioner er.. Eller skal den?.. Kan man ikke bare lave hele dispositioner ad gangen.. og/eller kan folk ikke bare rate dem ellers..? ..Hm, jo måske er det slet ikke så nødvendigt med score og alt det halløj.. Lad mig lige se, hvis folk kan vurdere en dispositions potentiele, som jeg var inde på før, så er det jo rigtigt godt, og.. Hm, man kan man mon ikke bare gøre det så at man vurdere potentialet og samtidigt opstiller nogle tærskler for, hvor meget de individuelle prædikater i dispositionen skal være opfyldt?.. Eller kan man mon bare sige, at de alle tæller ens, og så.. eller man kunne fordele en vægt.. Hm.. Eller lade dem tælle ens og så bare have en konvention om, at tekstmoduler skal pakkes ind i monader, der kender tekstens kontekst.. ...Ja!.. Kontekst-kendende moduler, og bum! Så behøver man ikke nogen vægte..! 
%...Okay, nu gik jeg lige en god lang tur og tænkte lidt over tingene, og jeg er faktisk lidt begyndt at se det på en ny måde.. Jeg er bange for, at jeg kommer til at skulle pille nogle af mine idéer ned og genoverveje hele idéen omkring, hvad man kan med de her lagdelte tekster (og hvad man genrelt skal kunne i et fællesskab, der har sat sig for at udbygge forklarende tekster..).. ..For dispositioner handler jo bare om at tegne en linje imellem emner og så sige, at hvis man læser om disse emner i den rækkefølge, så bør man opnå forståelse for det og det (og/eller blive kyndig på det og det område og/eller kunne løse de og de problemer).. Og et godt samarbejde i sådan et fællesskab handler vel så bare om at fremhæve de dispositioner, man mener kan være gode, og så i øvrigt fremhæve, hvis der er huller i de dispositioner.. ..Og ja, hvor god en disposition er kan så afhænge af prædikater, såsom hvor dybdegående teksten kan være, men så handler det jo så også bare om, at brugere skal kunne rate flere prædikater om en disposition en bare "godhed".. ..Så min pointe er altså: Er det ikke virkeligt simpelt.. behøver man alt muligt med skabeloner (jo skabeloner er gode, men hører de ikke nærmest mere til sin egen idé så alligevel) og automatiske scorer osv..? Behøver man overhovedet at have sådan et system, som jeg har tænkt på et helt grundlæggende plan (i.e. som er helt grundlæggende for min seneste version af denne idé), nemlig hvor man kan gå ind på en titel/disposition (hvor tilhørende prædikater også vælges), og så indsætter platformen tekstmodulerne automatisk..?!.. Jo, det lyder da virkeligt interessant, men.. vi er jo allerede på internettet; behøver tingene ligesom at blive samlet til ét dokument/"værk;" kan man ikke bare have en overordnet disposition med links til forklarende tekster for de enkelte delemner, og så kan man bare følge disse links i den rækkefølge (hvis man ikke kender til området)?..?!.. ..Så kunne man med andre ord ikke nå vildt langt bare med en Wikipedia, hvor der så er særlige typer af artikler, der beskriver en anbefalet vej ("linje mellem emner") til at blive kyndig på et emne, og så kan samme artikel jo også bare lige beskrive i hvilken grad man så kan forvente at blive kyndig ved at følge den rute. ..Og ja, hvis siden så bare netop er en side, hvor der er plads til alle mulige bud på "artikler," og hvor brugere så bare kan rate de forskellige artikler med forskellige prædikater, er det så ikke bare det..? Hm, på en måde er forskellen fra min seneste idé og så denne, at nu bliver "dispositioner" så bare tekster i sig selv, der udformes nærmest som en læseanbefaling, og som så bare har links til specifikke emner.. Hm.. Men ja, man bør så stadig i det mindste have det lagopdelt, så at disse links kan udformes som prædikater, så man ikke nødvendigvis navigerer direkte videre til en specifik artikel, men kan navigere hen til en søgning, hvor man så får de bedste forslag prædenteret.. ..Ja, dette giver da meget mere mening.. Igen: vi er allerede på internettet, så vi behøver slet ikke at fokusere på "sammenhængende værker;" adskilte tekster, der bare er bundet sammen af links, er fine!.. Og ja, især hvis disse links så bare lige kan være mindre direkte og i stedet kan lægge op til en søgning (og hvor brugere så også selv kan justere søgeprædikaterne og ikke mindst sætte flere på). ..Ja.. ..Okay, kan man så ikke nærmest bare implementere det med mit folksonomy-system, hvis der bare er tekst-ressourcer, og hvis man så også bare lige kan indsætte links i tekster, som behandles ved at man laver en søgning på prædikaterne, der udgår linket..? Og så kunne fællesskabet med fordel holde en ontologi over emner, hvor man kan.. Jo, det ville ikke være dumt af flere grunde, men i øvrigt kan man måske også bare sørge for at bruge "annotationer" i form af difs (som jeg har nævnt ovenfor) til at ændre på teksterne, og bl.a. hvis der er fremkommet en konvention om at referere til et emne med et andet navn.. Men ja, og man dog altså også med fordel bruge emnegrafer, hvor folk kan oprette links for, hvor semantisk ens de forskellige emne-tags.. nå ja, det system har man jo allerede i min folksonomy-idé, så never mind! Man skal så bare bruge emne-tags'ne fra folksonomy-idéen, når man skriver links i artiklerne (inkl. i dispositionerne). ..Bum.. ..Ja.. Så er det bare lige, hvad man gør med mine skabelonstanker.. ..Nå ja, og hvad man gør med programkode-tekster, for der skal det jo gerne netop blive til sammenhængende "værker"/dokumenter.. Hm.. 
%Ah, jeg har det nu. Ja, det er rigtigt nok, at for ting større end hele kapitler, artikler eller sektioner generelt, der bare er store og selvstændige nok til beskrives med nogle ret overordnede prædikater, jamen så giver det nok oftere mening at have tekster, hvor sådanne kaptiler, artikler osv. bare indgår som links, man kan følge. Men når vi når ned til opbygningen dispositionen af selve kaitler, artikler og/eller sektioner små nok til, at deres tekstmoduler i stor grad kommer til at bestå af meget specifikke tekster, der passer lige ind i den sammenhæng, og som ikke nødvendigvis skal genbruges en hel masse.. uh, og særligt også som man ikke forventer, at folk vil søge på som noget selvstændigt.. ja: Når vi når ned til små nok tekstmoduler til, at de ikke er brugbare som selvstændige tekster (i.e.\ som tekster, der er værd at referere til selvstændigt med links i andre sammenhænge), jamen så kan det lige netop være smart med sådan en mulighed for at lave en disposition for sig, og så have det sådan at de indivudelle tekstudsnit kan udfoldes automatisk. ..Nej, vent.. Tja, på den anden side kan man jo også bare kopiere tekster (med passende referencer til det originale arbejde) og skrive dem om.. Og man kan så i øvrigt bare bruge en konvention om at liste prædikaterne for, hvad sektionen/artiklen skal indeholde, i den rækkefølge som man forventer, at det kommer i.. Hm, ja, så selvom min pointe skulle til at være, at man godt alligevel kan have fold-ud links, som hjemmesiden forventes automatisk at prøve at finde substituter for (og at "linket" således omdannes automatisk til en tekst, der passer til linket), så er jeg nu altså i tvivl igen.. Nå, men imens jeg tænker på det, kan jeg lige sige, at "skabelonerne" bare bør være, at man sørger for at udforme alle tekster i fornuftig markup, så de kan opsættes i forskellige layouts og med diverse tilføjelser til --- muligvis ved at man bare om-kompilere markup'et til et andet markup-sprog, hvis man nu f.eks. har det i HTML men hellere vil have det renderet med LaTeX eller noget andet. ..Hm, men er fold-ud-automatisk-dispositioner så en ikke så vildt brugbar idé, eller hvad..?(!).. Hm (måske).. (..Også fordi alt det med tesktudsnit-ratings og små rettelser, det kan man jo vel bare bruge "annotationer" til..) ..Hm, men kunne man måske foreslå fold-ud-dispositioner, måske især netop til programmering så..? ..Nå ja, jeg ville også have nævnt nu her, at angående "kontekst-kendende tekster," kontekst-prædikater og hvad, jeg ellers har snakket om i den smamenhæng, så giver det nemlig ikke rigtigt mening, når først vi snakker små nok tekstudsnit, at rate dem andet end ift. hvor godt de passer ind i den pågældende omsluttende \emph{selvstændige} tekst.. ..Men dette vil jo så afhænge af de andre tekstudsnit i denne samlede tekst, så her vil det vel så alligevel ikke give mening, at vurdere dem indbyrdes (eller så at lade dem kunne blive skiftet ud automatisk).. Nej ja, når de netop er afhængige af hinanden, så giver det ikke mening at behandle dem som uafhængige tekstudsnit. ..Hm, men man kunne stadig godt foreslå fold-ud-links, for der vil jo være artikler, der er relativt små, men hvor sektionerne alligevel er selvstændige; sådan er det jo f.eks. for gængse Wikipedia-artikler. Så ja, kan bør bestemt foreslå brugen af fold-ud-links, men det er så ikke sikkert at de bliver til meget andet end dette, i.e. til automatisk udpakning af relativt små tekster/artikler, hvor sektionerne alligevel er ret selvstændige.. ..Ja.. ..Tja, og dog: Jo, jeg kan også nævne programkode, hvor man af syntaktiske hensyn måske kan have gavn af, at links'ene kan blive foldet ud automaitsk direkte i den overordnede tekst.. 
%(05.10.21) Okay, så nu tænker jeg jo lige over, om jeg så helt skal gå væk fra at skrive om mine tanker omkring programmering.. Som vel nærmest også på en måde næsten kan forkortes til: En mere visuel måde at gennemgå kode i et samarbejde ved brug af annotationer, som kan rates op og ned, og som kan få highlight-farvestyrke alt efter, hvor vigtige de er vurderet til at være. ..Og at man med dette så også kunne indføre en konvention om at bruge positive annotationer i høj grad; at sige "dette udsnit har jeg gennemgået, og jeg mener, at det bør virke," og så kan man herved få et overblik over, hvor gennemgået de forskellige udsnit er, og hvor sikre forfatteren og andre folk, der har gennemgået koden, er på, at koden holder. Det, og så er der også version control-tankerne, som vel svarer meget til intentionel programmering, gør det ikke?.. ..Hvor det særlige vel så bare lige er, at man her bruger ratings til at afgøre, om kodemodulerne passer til "intentionerne," som så er beskrevet i prædikaterne (som så kan være fold-ud-prædikater i bund og grund). Ah, men det er da også tre gode pointer! Dem bør jeg da helt klart lige nævne! Og de relaterer sig alle tre (rating-afhængige highlights, positive annotationer, og intentionel programmering med ratings) til ratings, så idéen kan så ses som en forlængelse af folksonomy-idéen. ..Hm, og kan "wiki-side-idéen" nu også det, eller skal den have sin selvstændige introduktion.. Nej, det passer måske faktisk meget godt også at introducere den i forlængelse af folksonomy-platformen. For pointen er nu bare, at man kan bruge tags til at klassificere forskellige typer artikler og andre værker, hvorved man så kan opnå langt højere alsidighed uden at det bliver uoverskueligt (fordi de forskellige typer altså er ordnet efter tags(/prædikater)). Coo-ool..! 


*En paragraf, hvor jeg forklare om at arbejde i et fællesskab om at bygge sådanne tekster på denne måde...* %Her skal jeg nævne de version control-agtige aspekter af det.



%Og angående førstnævnte **Uddyb** forhold, så vil jeg lige bemærke, at selvom jeg kalder det en ``disposition'' her --- og endnu værre at jeg bare har kaldt tekstmodulerne for ``sektioner'' flere steder --- så kan det jo godt være lidt mere end en række sektioner. ...

%(Dette er ikke så relevant for monadiske dispositioner, fordi ...)
%Nævn til sidst det med umuligheds-advarslen..
%(Bemærk at dispositioner godt også kan indeholde omkringliggende ting / struktur og ...)


Det næste vi kan diskutere, er så, ...%(annotationer...) %..Hm, hvilket vel så enten kan implementeres bare med de modulære tekster her, men hvor man måske også kunne bruge mere indre annotationer.. ..Ja, så det kan jeg vel netop påpege; at.. Hm, måske skulle man så have annotationer op i sit eget abstrakte lag, hvor de så automatisk enten bliver til indre annotationer eller til tekstprædikater/-tags alt efter, om om der så er tale om en atomar tekst.. eller en sammensat tekst..? ..eller et indre tekstudsnit.. Hm ja, dispositionerne skal jo også have prædikater om sig.. Ah ja, det mangler jeg at nævne ovenfor. Skal der så egentligt være en forskel? ..Det behøver der vel ikke, for så kan søge-algoritmen altid bare undersøge som en seperat ting, om nu teksten også er fyldt helt ud, eller om der er huller. Ja, fint!.. ..Hm, ah men semantisk set \emph{vil} det jo være forskellige prædikater, man bruger om henholdsvis udfyldte teskter (inkl. atomare tekster) og så dispositioner, men så kunne man måske netop også her gavne af et ekstra lag, når det kommer til tekstprædikaterne..! Hm...
%Okay, jeg kom frem til i går aftes, at man netop bare skal kunne oprette tekstudsnit som (førsteklasses) ressourcer (hvilket så selvfølgelig skal ske automatisk, når man annoterer). Disse bliver så selvstændige tekster, der bare får en reference til forælderteksten. Og det er så bare disse referencer algoritmerne kan bruge, når den skal søge på relevante indre annotationer. Angående forskellen på dispositions- og tekstprædikater, så skal jeg dog lige se ad.. 




}




















\newpage
\section[Opfølgende]{Opfølgende noter omkring rating-folksonomy-system (22.10.21) \label{opfoelg-folksonomy}} %(--02/11..?)

Her er nogle opfølgende noter omkring min idé til rating-folksonomies (som jeg indtil videre kalder dem) fra ovenstående 17/3--20/10(-21)-notesæt (``mit store 2021-notesæt'' kan jeg måske også bare kalde det).

\subsection[Brainstorm-noter]{Brainstorm-noter om at simplificere idéen (22.10.21--02.11.21) \label{opfoelg-folksonomy-brain}}

Man kan jo nok komme rigtigt langt med et system, hvor man kan give prædikater til og dermed kategorisere alt fra ressourcer til kommentarer til tags og til kategorier selv\ldots\ Hm, eller lad os vende tilbage til sidste mulighed senere. Og hvis systemet så har specielle HOL-termer til at vurdere \emph{relevans} af alle disse, så er man vel altså allerede ret langt? For det eneste, man nærmest i bund og grund kan ønske sig derfra, er vel bare, at platformen også bliver god til at foreslå relevante ting lidt forud for, at de bliver vurderet.? Det kan f.eks.\ være rart, hvis en tidlig relevans-score kan stige, hvis tagget bliver vurderet meget. Hm, men man kunne jo også bare automatisk prompte brugeren, hvis brugeren vurdere et tag, der ikke er vurderet som relevant\ldots\ Tjo, måske\ldots\ Og ellers har vi også sådan noget som at se, at visse tags eller kommentar-kategorier er relevant givet, hvad vides om ressourcen (og altså særligt denne tag-vurderinger). Og vi har også sådan noget som at relevansen af et tag kan medføre relevans af andre tags. Alle disse ting kunne så dog godt bare blive platform-centralens opgave, hvorved den sikkert med fordel bare kan bruge machine learning. Og så behøver brugere ikke tænke så forfærdeligt meget mere over dette. Denne løsning kræver så lidt bare, at der er en central, medmindre selvfølgelig at man kan offentliggøre anonymiseret korrelationsdata\ldots\ Hm\ldots\ Tja, men uanset hvad, så er det måske meget gavnligt, at det bare foregår i et underlæggende lag, så at selve HOL-systemet, som brugerne interfacer med, holdes ret simpelt\ldots? Hm ja, det virker som en rigtigt gos idé, medmindre altså man kan finde en rigtigt simpel måde at implementere sådanne relevans-gæt på ellers\ldots\ Hm, og/eller medmindre man gerne vel bruge brugeres hjælp til at justere disse relevans-gæt, jo\ldots\ \ldots Hm, kunne det ikke bare ellers implementeres ved, at alle HOL-udsagn med en variabel parameter kan gives en ``medfører relevans''-vurdering til et andet HOL-udsagn (\ldots hm, eller måske er systemet så netop ikke HOL; måske er ordenen så begrænset\ldots), som også kun har én variabel parameter. Denne parameter skal så typisk ses som en ressource, og ``medfører (sandsynlig) relevans''-udsagnet kan så f.eks.\ herved sige, at udsagnet om at ressourcen tilhører kategori $x$, medfører at kommentarkategori $y$ sandsynligvis også så vil findes relevant. Og vi kunne så også starte med f.eks.\ en kommentar-kategori $x$ og overføre relevans til en kommentar-kategori $y$, eller vi kunne starte/slutte på en tag-kategori osv.\ osv. Ville dette ikke være en meget god måde, at åbne systemet op for, at brugere kan hjælpe med at justere relevans-gættene, og så kan man jo ellers stadig godt bare for det meste holde det nede i et underliggende lag?\ldots\ Hm, det virker altså som en ret god løsning. Og der er ikke noget begrænsende ved denne løsning, andet end at man lige gør det grundlæggende interface en tand mere kompliceret ved at indføre ``medfører sandsynligvis relevans''-udsagnene; man kan altid bygge videre på det bagefter, og evt.\ finde en måde at udviske denne lagdeling på. Okay, lad mig så bare sige dette for nu. Og så skal jeg ellers nævne, at man i diverse feeds skal kunne subtrahere kategorier fra feedene, og dette bliver så en måde hvorpå at implementere ``advarsler'' samt frasortering af kommentarer/tags/ressourcer med disse. Så den ting kan ligesom bare klares sådan. Og ellers skal jeg vel så nærmest bare overveje implementeringen af brugergrupperne til feed-sortering, ikke\ldots? \ldots Hm, og kan implementeringen af brugergrupper ikke bare handle om, at man vægter brugergrupper med kategorier, altså så at en søgeindstilling handler om at tage brugergrupper og så give dem vægte hver især på forskellige kategorier. (Og systemet har nemlig allerede automatisk implementeret mulighed for ``overkategorier'' og ``underkategorier,'' for her kan man bare sørger for at sådanne over-og-under-kategorier er forbundne med ``medfører relevans''-udsagn, og det er det eneste, man behøver (mener jeg). For hvis bare en kategori bliver synlig i feedet, så skal den nok også blive vurderet af mange nok. Og dermed behøver man ikke at implementere noget ISA-(hierarki-)system eller noget i den retning. (!!)) Nice\ldots! Og ja, de vægte man så fordeler til kategorier for hver brugergruppe, de ganges så bare på brugergruppens indre (stemme-)vægte og voila! \ldots Umiddelbart rigtigt cool. Lad mig lige summe lidt over det, og så vil jeg vende tilbage (og opsummere og/eller kommentere på det). \ldots

Jeg kom med det samme i tanke om, at man måske også kan bruge højere-ordens-prædikater som en slags kategori-kategorier netop som en hjælp til søgeindstillingerne (og altså brugergruppe-kategori-vægtfordelingen) ved, at man så bare kan fordele disse vægte via sådanne kategori-kategorier. Dette virker ikke helt dumt umiddelbart. Jeg kom også i tanke om, at hvordan forfatteren/skaberen til en kommentar/ressource vurdere ressourcen også bør kunne give ekstra-relevans-point i det tidlige stadie, men dette er altså også noget, som platformen jo bare kan implementere i et underliggende lag. Okay, så lad mig prøve at opsummere:

De mest vigtige term-typer og relationer for systemet, som jeg tænker det nu, bliver altså for det første ressourcer/kommentarer som den grundlæggende term type (hvor vi altså kunne have valgt at skille kommentarer og ressourcer ad, men jeg synes det er mest fornuftigt, hvis kommentarer bliver first-class citizens på linje med ressourcerne (og altså derved \emph{bliver} ressourcer)). Der skal så selvfølgelig være en relation imellem ressourcer/kommentarer, hvilket jeg bare kan kalde `termer' her, som siger ``term $x$ er en kommentar til term $y$.'' Denne relation skal automatisk sættes til 1, når en bruger skriver kommentaren\ldots\ Hm, men så bør man da faktisk næsten dele det op i stedet, for ellers bliver det jo bare helt specielt med sådan en relation, som man ikke skal vurdere (men som bare betragtes som sand, når den først er sat)\ldots\ Hm, men kan vi ikke bare indsætte `relevant' i udsagnet; kommer det så ikke til at give god mening igen\ldots? Jo\ldots\ Hm, og så kan vurderingen fra en kommentarforfatter om relevansen af pågældende kommentar bare pr.\ default sættes til 1 (/ 100 \%). Og så kan forfatteren altid nedjustere denne vurdering efterfølgende, hvis denne vil det. \ldots Ok, lad os sige det (for nu om ikke andet). Herudover skal brugere kunne indføre prædikater ad libitum, hvilket så implementerer vores (ressource-tags) og vores kommentarkategorier. Og de implementerer som nævnt også vores advarselslamper. Der skal så være to relationer, der henholdsvis siger: ``for termet, $x$, er prædikat-udsagnet, $p$, relevant at vurdere,'' og ``for termet, $x$, er kommentar-kategorien, $h_x$, relevant (som et ``kommentar-blad'' (se store 2021-notesæt ovenfor for en forklaring af begrebet) i kommentarfeltet). En ``kommentar-katogori'' kan så enten bare have typen som et normalt prædikat (som tager et term som input), men kan også have typen som en relation imellem to termer, hvor det så er underforstået at relevansen afhænger af, at første input er lig $x$ fra relevans-udsagnet. Det jeg tænker her, er altså, at kommentarfeltet skal kunne inddeles i kategorier, hvor brugeren så kan se og skifte mellem forskellige blade (hvorved forskellige sorteringer og filtreringer, i.e.\ ``kommentar-kategorier(/kategoriseringer),'' vises). Disse kommentar-kategorier må meget gerne kunne have\ldots\ hm, eller skal man mon sige, at de \emph{skal} have\ldots\ et input i form af den relevante ressource\ldots\ Hm\ldots\ Hm, nå nej, måske ikke, for vi har jo allerede et ``relevant for ressource-termet''-prædikat. Hm, så måske \emph{kan} vi godt sige, at kommentar-kategorier skal kunne have begge disse former --- hvornår for vi mon brug for den mere komplicerede form?\ldots\ Ah, det kan vi jo gøre, når kommentaren f.eks. ``svarer på et spørgsmål i ressourcen'' eller ``udtager punkter fra ressourcen'' eller ``påpeger fejl/mangler i ressourcen'' osv.\ osv. Ja, så kommentar-kategorier må altså gerne lige præcis kunne være sådanne relative kategorier. Hm, men nu bliver jeg i tvivl; hvad kan så være eksempler, hvor kommentar-kategorien \emph{ikke} er relativ til ressourcen på denne måde? Hm, for ``term $x$ er en relevant kommentar til term $y$''-udsagnet er jo netop også selv på denne form, så dette vel (måske) ligesom bare være grund-kommentar-bladet, så?\ldots\ \ldots Hm, men hvis denne grund-kategori er gyldig, så kan man jo altid opdele kommentarerne her yderligere med (normale) prædikater, kan man ikke? Hm, jo\ldots\ Jo. Okay, så jeg havde ret; $h_x$ skal både kunne være et simpelt prædikat bare, eller et prædikat i form af en relation, hvor første parameter indsættes i form af ressource-termet, $x$, og hvor anden parameter så holdes variabel. Cool\ldots! 
Okay, og hvis vi så går videre, så kunne man jo så overveje at inkludere et udsagn, der siger: ``kommentarkategorien, $h$, er ofte relevant for ressourcer af kategorien, $p$,'' hvilket så potentielt kan bruges af det (underliggende) lag, hvis formål er at gætte relevans af de forskellige tin i systemet (inden at folk har vurderet dem i høj grad). Men ja, nu bevæger vi os så altså lidt ud i periferien, men stadigvæk. Hvis vi så går væk fra kommentarerne og over til ressource-tags'ne, så har vi jo allerede vores ``for termet, $x$, er prædikat-udsagnet, $p$, relevant at vurdere''-udsagn. Vi kan også lige hurtigt nævne, at man kan have et udsagn, der siger: ``for termer, der opfylder $p$, vil det ofte være relevant\ldots\ at vurdere udsagnet $q$,'' nå nej, dette udsagn vil jo så faktisk allerede komme med mit ``kommentarkategorien, $h$, er ofte relevant for ressourcer af kategorien, $p$''-udsagn. Fint. Spørgsmålet er så, hvad man skal gøre ift.\ muligheden for at have tag-kategorier?\ldots\ 

(23.10.21) Ah, det virker altså til at give rigtigt god mening alt det her\ldots! Og de ting jeg mangler at opsummere er faktisk, vist nok, heller ikke særligt komplicerede\ldots! Men for at fortsætte, hvor jeg slap: Skal der være tag-kategorier? Ja, jeg kan godt lægge op til, at tags'ne faktisk for de samme muligheder for at blive inddelt i undermenuer (svarende til ``blade,'' men her kan det også bare være, at man vælger at vise dem en efter en i rækkefølge (og så bare har mulighed for at ekspandere undermenuen)) og for at sorteres efter relevans. Her kræver det så, at man indføre muligheden for højere-ordensprædikater, der siger: ``der gælder $P$ om $p$.'' *(Altså ``$P(p)$'' med andre ord.) Herved kan man så inddele tags'ne, repræsenteret ved deres udsagn, $p$, i kategorier beskrevet ved $P$. Hm ja, og så får vi forresten allerede muligheden for at lave overkategorier, hvilket så videre også kan bruges til søgeindstillingerne(/sorteringsindstillinger), når vægte skal uddeles til brugergrupper. Hvad har vi mere behov for? Hm, måske en relation, der så kan sige: ``Overkategorien, $P$, er relevant for term $x$,'' og måske en der kan sige: ``Overkategorien, $P$, er ofte relevant for termer, om hvilke der gælder $p$?'' Ja, det lyder fornuftigt. Og skal der være en pendant til min ``$h_x$''-relation her? Hm, altså: Skal der kunne være en slags $P_x$'er med?\ldots\ Ja\ldots\ Nej\ldots  Hm, nej; det bliver ikke relevant. Tags er nemlig beregnet til at søge på og til at sammenligne ressourcer med; hvis et ``tag'' har noget med den specifikke ressource at gøre, så er det en kommentar eller en kommentar-kategori, ikke et tag. Cool. Åh, og jeg bør forresten bemærke, at mit system her kræver, at folk kan indføre, ikke bare både prædikater ($p$'er), prædikat-prædikater ($P$'er), men også relationer imellem termer (hvilket jeg ikke vil kalde $h$'er her, for det var mere for at adskille det fra resten, da det jo var et symbolsk prædikat konstrueret ud fra en relation)\ldots\ Ja, lad mig i stedet bare holde mig til $p, q, r$ for både prædikater og relationer, og så kan jeg jo bare give 1 eller 2 som superscript, hvis det bliver nødvendigt at kunne kende forskel eller bare at præcisere generelt. Og jeg kan så lade $p_x$ betegne et symbolsk prædikat, som kommer fra en relation, hvor man bare har indsat et term som første parameter. Men tilbage til opsummeringen: Det var nærmest det! Så bør det også nævnes, at mine ``advarselslamper'' nu bare implementeres som tags, kommentar-kategorier eller tag-kategorier, alt efter om vi snakker ressourcer, kommentarer eller tags henholdsvis. I sorteringsindstillingerne skal brugerne så have mulighed for at trække relevans fra i diverse feeds, hvis de er vurderet til at høre til specifikke tags, kommentar-kategorier eller tag-kategorier. Så jeg skal altså ikke udvide systemet mere i dette lag for at gøre plads til denne mulighed. Hov forresten: Kommer jeg virkeligt ikke til at bruge brugerlavede relationer til andet end kommentar-kategorier? Næ, umiddelbart ikke\ldots\ Man kunne altid åbne op for det løbende, i form af at man gav brugerne mulighed for at lave mere formelle kommentarer (altså i form af relationer med input), sådan at dette kan føre til, at maskinalgoritmer bedre kan følge med\ldots\ Hm ja, dette er faktisk helt klart værd at nævne: Nej, det kommer ikke til at være en særlig del af den første version af platformen, men på sigt vil det blive brugbart, hvis brugere også kan lave formelle kommentarer, baseret på relationer med (reference-)input og ikke på bare tekst, hvorved kommentarerne så kan vises for begge ressourcer for det første og på sigt altså også blive brugt i semantiske søgealgoritmer. Cool nok. Hov, skulle jeg forresten ikke også have noget med semantisk beslægtethed (mellem tags); hvordan skulle det nu implementeres? Hm, bare ved tag-relevans overført fra prædikat $p$ til prædikat $q$? Ja. Jo, så der var heller ikke mere, der skulle gøres i den forbindelse. Uh, lad mig lige nævne en tanke, jag fik her tidligere: Tag-dokumentationer kan også nedvurderes til fordel for bedre udgaver, hvis de nu ikke tilstrækkeligt krediterer de tidligere udgaver, de er baseret på (hvad end disse tidligere udgaver selv var gode til at kreditere eller ej; originalt arbejde bør altid krediteres). Cool, så når vi allerede videre til brugergrupperne! Lad mig skifte paragraf.

Hm, jeg har egentligt ikke så meget ekstra at tilføje om brugergrupper. Hm, jo måske at man måske skal kunne have flere moderator-lag i den, så man kan have over- og undermoderatorer. Nå ja, og at der på sigt gerne må åbnes op for, at brugergrupper kan definere deres vægtfordelinger helt eller delvist med formelle funktioner, som platformen så selv kan bruge og opdatere vægtene automatisk. Moderatorer skal som nævnt kunne give ud af deres egne moderator-point, og ellers skal de kunne uddele og fratrække point fra lagene under, hvor der nederst er selve stemme-vægtene. Det vil nok ikke være dumt, hvis alt på platformen i øvrigt har et timestamp tilknyttet sig, hvilket kan gøre at man altid kan gå tilbage i tid, hvis man ikke kan lide en brugergruppes udvikling. Og ja, brugergrupper kan altså frit bygge oven på tidligere brugergrupper (og igen skal man altid huske at kreditere arbejde, hvor man kan, og hvor det ikke gøres automatisk), bl.a.\ ved så at starte fra et tidligere punkt og så forke brugergruppen derfra. Men også gerne ved at kunne have formelle definitioner, hvor stemmevægtene i sidste ende bestemmes helt eller delvist ud fra tidligere oprettede brugergrupper (hvilket altså også er noget, man kan åbne op for med tiden, og ikke behøver at have med helt fra starten).

Og så kommer vi vel til, hvad jeg har lagt op til omkring ``søgeindstillinger''/ ``sorteringsindstillinger,'' gør vi ikke? Dette handler som nævnt om at fordele en slag ``over-vægte'' til grupperen alt efter kategori\ldots\ Hm, men man kan jo forresten ikke bare bruge overkategorier til denne fordeling, hvis vi ser på mit nuværende system og også lige holder os til kun at se på tags'ne forresten, for hvem skal så vurdere hvilke tags hører til hvilke (tag-)overkategorier?\ldots\ Hm, godt spørgsmål. Jeg har nogle mulige løsninger, men lad mig lige gå en tur og tænke lidt over det\ldots\ \ldots\ Okay, den simple løsning vil jo faktisk bare være at sige, at inddeling i overkategorier er en simpel nok opgave til, at det kan administreres af én gruppe\ldots\ Men vil dette også holde på længere sigt, så?\ldots\ Hm, ah\ldots\ skal brugergrupper ikke også selv kunne implementere en (over-)vægtfordeling imellem andre grupper baseret på kategorier?\ldots\ Hm jo, men det løser ikke dette problem. Hm, er det så slemt bare at åbne op for, at man kan have prædikat(-prædikat\ldots)-prædikater af arbitrær orden, så man herved kan lave arbitrært mange inddelinger i overkategorier?\ldots\ Hm, også fordi hvis vi alligevel allerede er oppe i TOL *(tja, eller måske ikke; sammenligningen med formelle logiske sprog er i det hele taget lidt vag, fordi vi jo ikke har nogen kvantorer med eller noget) i princippet, så er der vel ingen grund til at frygte, at folk vil synes at HOL er mindre pænt\ldots\ Men det kan jo føre noget teknisk besvær med sig, at der så bliver uendeligt mange ordner, når vi ellers kunne klare os med et begrænset antal\ldots\ \ldots Uh, man kunne også bare have relationer imellem prædikat-prædikater, der implementere en slags mængde-relationer, og særligt relationen, $\subset$ (/$\subseteq$), hvilken man måske endda bare kan nøjes med. Nice! Så kan vi stoppe her ved, hvad vi nok kan kalde fjerde orden (når vi medtager konstante relationer; og så stopper vi ved tredje ordens, når det kommer til relationer (eller rettere prædikater), som brugerne selv kan indføre). Hm, men løser det nu helt vores problem alligevel\ldots? Ah jo, fordi så kan (over-)vægtfordelingen til brugergrupperne, hvad jeg måske bør begynde at kalde gruppevægte, jo bare (hvis der bliver behov for det) fordeles ved, at man starter med nogle brede kategorier og fordeler brugergruppe vægte for dem\ldots\ hm, og så kan disse brugergrupper så igen stå for en yderligere gruppevægt-fordeling\ldots? \ldots Ja, kunne man måske gøre det sådan, at disse brugergrupper herved for lov netop til at fordele vægte til lige netop tag-kategorier, som er vurderet til at være undermængder af de kategorier, de så er sat på?\ldots hm, men hvem vurderer så undermængde-relationerne? Hm, kan samme gruppe ikke bare gøre dette\ldots? Ah, nu bliver det en smule kompliceret; der må være noget nemmere, man kan gøre\ldots? \ldots Hm, skulle man overveje noget a la en af mine løsningsidéer fra lidt tidligere i dag?\ldots\ 
%Mulighed for løbende forgreninger af brugergrupper som en løsning..? (Og hvor brugergruppen, der bestemmer overkategorierne også bestemmer, hvilke underbrugergrupper, der står for at bestemme for de kategorier..) ..Ja, og så beholder man mængderelationen (imellem P'er), og så er det altså også overgruppen, der vurdere disse. Og denne sætter så også en tærskel, således at kategorier, som er vurderet \emph{nok} til at være en underkategori af den pågældende overkategori, for dem vil udpegede brugergruppe så også bestemme.. (Hov, jeg kom lige i tanke om, at man godt kan sagtens kan \emph{udtrykke} f.eks. min P-mængderelation her i SOL. Værd lige at huske på.. ..Hm, men dog ikke sikkert, at det kan oversættes så nemt til tripletter..?..) Hm, og dette system bør så kunne gøres rekursivt, og så burde det fungere.. Lyder umiddelbart meget fornuftigt, og så bliver vi altså på en begrænset (fjerne) orden af prædikater/relationer. Ok. Lad mig se, om det giver mening i morgen. (23.10.21) 
%(24.10.21) Okay, jeg tror virkeligt, jeg har fat i noget godt her. Og i går aftes fik jeg så en idé, der (tror jeg, håber jeg) virkeligt gør det godt. Variable brugergrupper! Dette vil sige brugergrupper, som i sin definition af sin vægtfordeling også inkludere variable parametre, som brugeren så selv kan stille på, og hvilket så ændrer vægtfordelingen. Jeg mener så nu endda, at dette kan komme til simpelthen at implementere alle sorteringerne/filtrene på platformen! Jeg fik i øvrigt også nogle idéer til to ontologier henholdsvis i aftes (som er en virkeligt vigtig idé!) og her tidligere i morges (som også kan være meget gavnlig for rigtig mange). Disse idéer afhænger så godt nok en lille smule af, hvordan jeg ville implementere sådanne ontologier (antaget at dette system bliver udbredt først), hvilket jo muligvis kan gøres oven på mit folksonomy-system (jeg bør i øvrigt finde et andet navn til det snart). For lige nu er jeg nemlig lidt i tvivl, hvor meget og hvordan jeg skal inkludere brugerlavede relationer på platformen.. Hm.. (Og jeg vil gerne lige finde ud af dette, inden jeg går videre, så jeg bedre lige kan skrive idéerne ind nedenfor..) ..Ah! Man kunne sgu da lave endnu en menu (måske til venstre for ressourcen) med relationer..!! Hm, umiddelbart super god idé, men lad mig lige slå koldt vand i blodet og tænke lidt over, hvad man så kan med denne menu, og hvor kompliceret det bliver at implementere den.. ..Ah, men menuen kan jo bare som menupunkter indeholde lige netop relationerne, og så vil indholdet i disse menupunkter altså bare være de.. Ja, og man kunne måske endda vælge, hvor den pågældende ressource selv indsættes i releationen: På første, anden eller en vilkårlig plads (så man behandler relationen som refleksiv..).. Hm, og så bliver spørgsmålet, om man ikke så også skal kunne erklære refleksivitet som en grundlæggende ting i relationer (det tror jeg faktisk, kunne være en god idé). Ja, jo. Så undermunerne indeholder altså referencer til andre ressourcer, for hvilke relationen til pågældende ressource er vurderet højt. Og denne menu kan bare være ligesom tag-menuen, således at undermunuerne kan ekspanderes via klik. Og nogle af de ting, man kan have i menuen er så "originale ressource" (hvis pågældende ressource i høj grad har kopieret fra en original ressource (eller flere)), "kildereference," "forrige/næste afsnit," "link til oversigt over kategorien (/ en kategori), pågældende ressource befinder sig inden for" osv. Dette er altså eksempler for almindelige, ret selvstændige ressourcer, og for ressourcer, der så kan være forbundne på forskellige måder i forskellige strukturer/hierarkier, der kan man så tilføje passende relationer til at implementere disse links. Vi snakker altså alle mulige former for links, som man også ville kunne finde på et fremtidigt (mere gængst) semantisk web. Og dette leder mig så til en anden idé, som jeg fik kort efter denne, og det var, at jeg måske kunne kalde min idé her (i stedet for rating-folksonomy-idé) for en idé til et 'manuelt semantisk web'.. Hm, eneste minus er, at det skaber associationer til "(hårdt) arbejde," men ellers synes jeg, det fungerer rigtigt godt, for så er det klart for alle, at det er et forstadie til det semantiske web --- hvilket det jo i høj grad må siges at være nu, nu hvor jeg har min relationsmenu! ..!! :) ..Okay, navnet holder faktisk måske, men lad mig bare lige summe noget mere over det. (Men jeg kan nu godt begynde at bruge det fremover her i mine noter.) Hold da op, det bliver bare rigtigt godt, det her.. Og lad mig så lige se kort på her, hvad man så skal gøre for denne menu, for at få vist de relevante ting til brugerne.. Det kræver jo selvfølgelig et "denne relation, $q^2$, er ofte relevant for termer, hvorom gælder $p^1$"-udsagn.. ..Hov, behøver man forresten egentligt overhovedet prædikat-prædikater, hvis man har subset-relationen? ..Det gør man da vel netop ikke.. Fedt nok, umiddelbart; jeg må lige overveje det mere i dybden, men umiddelbart virker det til, at jeg så ikke behøver mine (variable (i.e. "brugerlavede")) prædikat-prædikater, hvilket er ret nice, for det vil klart være pænest uden. ..Nå, og hvad kræver det udover "denne relation, $q^2$, er ofte relevant for termer, hvorom gælder $p^1$"-udsagnet? Man skal selvfølgelig også sikre sig, at brugere nu også kan oprette, bedømme og bedømme relevans for selve relationerne (to sidstnævnte ting altså ift. en specifik ressource). Hm, ja; om ikke andet er der jo bare ligesom tag-menuen, bare hvor vi ikke behøver tag-kategorier (..nå ja, så måske behøver vi altså stadig prædikat-prædikaterne? (hvilket så også er fint nok; bare det er der for et formål)) for at danne menupunkter, men hvor det bare svarer til at hver "tag" (som her altså er en relation) nu \emph{er} menupunkterne. Cool, cool, cool! 
%..Ah, nu indser jeg lige, at man måske kan bruge undermængde-relationen i stedet for prædikat-prædikater til at lave menupunkter. Men så bliver det sikkert mere besværligt, og så skal man også overveje, om man så ikke skal gøre noget tilsvarende for kommentar-kategorierne. ..Nej, jeg tror den bedste løsning simpelthen er den mest simple løsning --- som jo så både er simplere for brugerne (at forstå ikke mindst) og for programmørerne af platformen. 

(24.10.21) Okay, jeg har faktisk fået en masse gode idéer siden sidst; i går aftes og her i formiddags. Jeg tror for det førte, at jeg har fundet en rigtig god løsning til, hvordan brugergrupperne og sorteringsindstillingerne skal fungere. %Angående sorteringsindstillingerne her jeg forresten også et par ekstra tanker fra i går aftes, som jeg ikke nåede at nævne i brainstorm-noterne lige her ovenfor (altså dem jeg lige har skrevet her lidt tidligere). 
Men jeg har nu også ændringer til det mere fundamentale i systemet, som jeg lige har fundet frem til her i formiddags (se brainstorm-noterne ude i kommentarerne (imellem denne og forrige paragraf)), så dem vil jeg snakke om først. Jeg er kommet frem til, at man også skal have en ekstra menu --- gerne ude til venstre (hvor tag-menuen kunne være til højre for ressource-vinduet) --- med relationer! Pointen med denne relationsmenu skal så være, at tilføje links til hurtig navigation til relevante ting (i.e.\ andre ressourcer (men ikke af samme type nødvendigvis; det kan f.eks. også være en HTML-ressource med en oversigt over kategorien, som pågældende ressource tilhører)). Af denne grund er det ikke nok, at disse links bare kan findes nede i kommentarerne (under et passende blad), som jeg lidt havde tænkt det her indtil nu, for man skal kunne se dem klikke på dem med det samme. Og det hører sig heller ikke hjemme i tag-menuen (hvad ellers svarer til, hvordan jeg har haft tænkt det for endnu længere tid siden), for pointen med tags er derimod udelukkende til at søge på (og sortere/filtre) og sammenligne forskellige ressourcer. Relationer kan godt også få lov at optræde nede i kommentarerne (måske ved at de automatisk oversættes til tekst), men der skal altså også være en speciel menu til gode navigations-relationer. Hm, nu har jeg i øvrigt tænkt på at kalde det et ``manuelt semantisk web,'' men måske man også kunne overveje at bruge begrebet ``manuel navigation'' i høj grad\ldots\ Nå, det var et stort sidespring. Relationer skal altså også være en ret central del af systemet, og brugere skal altså kunne vurdere dem ligesom tag-prædikaterne osv. Der skal også være et ``relation $p^2$ er relevant for termerne, $x$ og $y$''-udsagn, og der må også meget gerne være et ``relation $p^2$ er ofte relevant for termer, hvorom gælder $q$''-udsagn. Nu skal man så bare holde tungen lidt lige i munden, for relationer er jo ikke nødvendigvis refleksive (eller er det \emph{reflektive}? \ldots Ah, ups! Det hedder symetrisk, det jeg taler om)\ldots\ ikke nødvendigvis er \emph{symmetriske}; der kan godt være forskel på parameter 1 og parameter 2. Hertil vil jeg faktisk foreslå, at man simpelthen deler alle relationer op i to typer, alt efter om de er symmetriske eller ej, således at forfatteren til relationen simpelthen for lov at erklære: Er relationen symmetrisk eller ej. Platformen skal så ikke kunne kende forskel på $p(x,y)$ og $p(y,x)$ for symmetriske relationer også skal altså ikke acceptere uploads af hverken $p(x,y)$ eller $p(y,x)$ hvis en af dem allerede er uploadet. Og den skal selvfølgelig vise relationen for begge ressourcer. Hvis $p$ er symmetrisk, skal ``relation $p^2$ er relevant for termerne, $x$ og $y$''-udsagnet ydermere også behandles som symmetrisk. Og for symmetriske $p$ er ``relation $p^2$ er ofte relevant for termer, hvorom gælder $q$''-udsagnet så ret trivielt. For relationer, der ikke er symmetriske, skal ``relation $p^2$ er ofte relevant for termer, hvorom gælder $q$''-udsagnet splittes op i to, simpelthen, og det samme skal alle andre relationer, hvor der kan være en forskel (hvis der bliver flere). Hm, man kunne dog eventuelt have det sådan, at når folk vil bedømme et ``relation $p^2$ er ofte relevant for termer, hvorom gælder $q$''-udsagn for et asymmetrisk $p$, så for de en radioboks eller en tilsvarende indstillingsmulighed, så de kan vælge, hvilken af de to parametre, man vurderer relevans for, og hvor man så måske kan få muligheden for at sige ``begge,'' hvorved man så bare giver den samme relevans-vurdering til begge sider af relationen (hvilket altså så kan bruges til relationer, hvor at have et link den ene vej er ligeså relevant som at have et link den anden vej).

Undermenuer i relationsmenuen behøver ikke nødvendigvis at bruge relations-kategorier, for det kan også bare være, at man kan nøjes med undermenuer i form af selve relationerne, og altså hvori termerne så kan vises (i sorteret rækkefølge alt efter relevans-score), der er vurderet til at kunne være nummer to input sammen med den pågældende ressource i den pågældende relation. Man kunne dog måske også give mulighed for yderligere inddeling, så man også indfører relations-kategorier, ligesom man har for prædikaterne (i.e. tags'ne).

Okay, jeg tror ikke, jeg behøver at sige meget mere om relationerne ellers, for de vil ikke blive særligt meget anderledes end prædikat-tags'ne. Man skal bare lige tage højde for, som nævnt, for de asymmetriske relationer, samt at hver relation her ikke bare udgør et enkelt og selvstændigt tag, men i stedet udgør en liste af referencer til andre ressourcer, som hver kan vurderes forskelligt (i forhold til deres relation til pågældende ressource (og for den pågældende relation)). Jeg kan bare lige slutte af med at sige, at relationerne f.eks.\ kan bruges til at indsætte kildereferencer, linkt til næste/forrige afsnit/episode (hvis ressourcen er en del af en serie af noget), link til en oversigt over den kategori (eller de kategorier), som ressourcen er vurderet til at tilhøre eller (ikke mindst) link til den/de originale ressource/er, hvis nu store dele af ressourcen kommer fra noget/nogle andre ressource(r). Og herudover kan der altså også være tale om links, der fungerer som kanterne i en ontologi, i.e.\ hvor ressourcen indgår i en eller anden grafstruktur, med specielle tilhørende prædikater og relationer, der implementerer kanterne og knudetyperne i grafen. Med andre ord kan man implementere alle de ontologier, som man kunne tænke sig, ligesom for det gængse semantiske web. (Og nu tænker jeg altså lidt min idé som et forslag til et ``semantisk web med manuel navigation'' (hvor man altså starter manuelt og så løbende evt.\ kan automatisere ved at indføre semantiske algoritmer på (eller som kan interface med) platformen).)

 
Okay, videre til mine nye idéer om brugergrupper. For det første kom jeg på, at brugergrupper kunne have mulighed for at opdele sig selv til hver en tid, hvorved de så erklærer flere undergrupper samt også hvilke kategorier, som hver undergruppe får lov at administrere. Gruppen kan så også have en rest-brugergruppe til alt, der falder uden for de valgte kategorier, og den \emph{skal} udpege en brugergruppe til at vurdere, hvilke ting falder inden for hvilke kategorier, hvilket altså dermed bestemmer, hvor undergrupperne har stemmeindflydelse. Den sidstnævnte brugergruppe kan dog bare sættes til den trivielle brugergruppe (hvor alle brugeres har samme stemmevægt), hvis man vil lægge op til, at dette skal styres af overliggende brugergrupper (der bruger pågældende brugergruppe som modul i sig). (Jeg bør egentligt i øvrigt måske begynde at kalde det noget a la ``stemmevægt-fordelinger'' i stedet for (vægtede) ``brugergrupper''\ldots\ Hm, det kan jeg lige overveje\ldots) For hvis andre brugergrupper vil bruge pågældende brugergruppe som modul\ldots\ Ah vent, man kan også bare sige, at brugergrupper altid får mulighed for at overskrive sådanne brugergruppes egne ``interne overgrupper,'' eller hvad vi skal kalde dem, som har til formål at vurdere og administrere kategorierne (og vælge, hvad hører til hvad). \ldots Hm, tja, der er vist flere ting, man kunne gøre her, så jeg må jo lige summe lidt mere over detaljerne, men idéen er nok meget god. 

Man kan så godt bruge $\subset$-relationen til at vurdere overkategorier i fællesskabet, men her skal jeg vist også lige tænke mig lidt mere om, for det bringer så nemlig tvivl om, hvorvidt man overhovedet har brug for prædikat-/relations-prædikater. Jeg kom i mine brainstorm-noter fra tidligere her i dag frem til, at man godt kunne, men at det måske så bliver mere besværligt at finde rundt i alt i alt for brugerne. Nu mens jeg skriver dette, så kan jeg mæske, at jeg måske nok skifter, og faktisk synes at det bliver mindst ligeså simpelt at bruge $\subset$-relationen til at lave tag- og andre kategorier --- for forskellen er vel bare, at kategori-prædikatet nu skal formuleres som et prædikat om ressourcer og ikke et prædikat om prædikater, hvilket da kun gør det mere intuitivt? Ah (dejligt), jeg tror sgu, jeg har ret. Så det betyder, at vi kan slippe af med prædikat-prædikaterne, hvilket er dejligt, for det er heller ikke så flot en ting. Ja, godt nok \emph{har} vi stadig prædikater/relationer af samme orden, for vi har jo bl.a.\ $\subset$-relationen, som jo tager prædikater, men dette gør ikke nær så meget, for det er bare en konstant relation. \ldots Ja, og for hvert par af prædikater/relationer kan den i øvrigt oversættes til FOL (ved brug af kvantorer), hvilket jo er ret rart på en måde at tænke på\ldots! :) Okay, det er altså ret nice, det her\ldots! Så er der altid en ret simpel oversættelse, hvis man f.eks.\ vil oversætte systemet til et triplet-system eller andet i den dur. Nice. :) \ldots Tja, eller kan tripletter nu også implementere kvantifiseringer (quantifications)? Hm, måske faktisk ikke\ldots\ Ah, eller jo måske, for tripletter kan jo godt være af højere orden, kan de ikke!? \ldots Hm, det ser ikke ud til, at der er noget typesystem omkring subjekt, prædikat, objekt, så jo, man må endda ligefrem kunne indføre min $\subset$-relation direkte som en del af en triplet-baseret ontologi. Nice nok. (Rart at systemts data let kan oversættes og overføres til andre datasystemer.) 

Nå, det næste jeg så skal ind på, er en idé, jeg fik i går aftes, omkring ``variable brugergrupper.'' Idéen kan (vist) implementeres ret simpelt. De undergrupper, en brugergruppe opretter i sig selv, behøver nemlig ikke at fordeles på disjunkte domæner, men kan også have (helt eller delvist) overlap i deres domæner. Og videre kan man så i overlappet (hvilket altså godt kan være hele domænet) indføre en parameter, hvor den overordnede fordeling af stemmevægt imellem to grupper så kan afhænge af denne (via en formel). Brugergruppen bliver således til en `variabel brugergruppe,' hvilket vil sige, at brugerne nu kan benytte brugergruppen i dens forskellige udgaver ved at stille på parameteren. En variabel brugergruppe kan også have flere parametre, bl.a.\ f.eks.\ hvis der nu er flere indre undergrupper i den, der deles om de samme overlap. Og via disse tror jeg så faktisk, at man ligefrem kan implementere alle sorteringsalgoritmerne (og filtrene) på platformen! Bemærk i øvrigt, at brugergrupper jo frit kan bruge andre brugergrupper som moduler i sig, og hvis en brugergruppe bruger en eller flere variable brugergrupper i sig, så arver denne bare alle disses variable/parametre tilsammen. Og der hvor en samlet brugergruppe har opdelinger i sig, så forskellige underbrugergrupper tager sig af forskellige kategorier, så er det altså bare op til ydre brugergrupper at administrere, hvordan disse kategoriseringer bestemmes (hvilket altså gøres ved at definere stemmevægt-fordeling til vurdering af $\subset$-relationen samt en tæskel for, hvor højt denne relation skal være vurderet, før den betragtes som sand). 

Brugere vægler så i første omgang en overordnet brugergruppe, som de vil benytte. De kan selvfølgelig også vælge flere og så skifte imellem dem. Når en brugergruppe er valgt, så kan brugeren justere alle parametrene. \ldots Hm, det er endda lige før, at man også skal kunne oprette en brugergruppe-type til simpelthen bare at give standard-indstillinger til variable brugergrupper. Ja, hvorfor ikke\ldots? Bemærk, at mine ``brugergrupper'' svarer til `decorator'-design pattern'et i OOP. Ok, men selvom parametrene altså kan sættes til standardværdier, kan brugeren altså altid gå ind og ændre på det (eller oprette sin egne nye standard (som kan uploades til platformen)). I øvrigt skal det så også kunne justeres, hvilke nogle parametre, der gerne må vises under ressourcerne, så man hurtigt kan ændre på dem, og hvilke man vil have kun skal vises i denne selvsamme (avancerede) indstillingsmenu, som vi altså snakker om nu. De parametre, man så stiller på under ressourcen og altså lige over kommentarfeltet, de ændrer så på alle sorteringer af alle kommentar-bladene/-kategorierne på én gang. Jeg tænker så endda også, at hvert blad godt må have lokale indstillinger, sådan at man kan ændre sorteringen for et specifikt kommentar-blad, man kigger på. Og hvordan gør man dette, hvis man skal gøre det\ldots? Jeg tænker umiddelbart, at når en brugergruppe assign'er en undergruppe til en specifik kommentar-kategori, og denne undergruppe er en variabel gruppe, jamen så er lige præcis de parametre, brugerne vil få at se i toppen af kommentarbladet, og som de så kan stille på. Ja, det lyder altså meget fornuftigt.

Ok! Det er lige før, jeg tror, at jeg ikke har flere todos nu; jeg tror jeg kom igennem alle mine idéer fra i går aftes og i formiddags, og det virker nu som om, jeg har nogenlunde styr på mit system. Men lad mig bare så holde fri nu og så summe over det hele til i morgen. :)

\ldots

(26.10.21) I forgårs aften kom jeg frem til, at kommentarerne godt kan gøres på en anden, mere simpel måde (og som nok er at foretrække). I stedet for bare at sige ``denne kommentar-ressource, $x$, er relevant for term $y$,'' så bør man nok i stedet gøre det, så at kommentarer har en forfatter i sig som ressource, og at denne forfatter så har en speciel rettighed til at uploade relationer med denne kommentar-ressource. Andre kan også gøre det, men så skal det bare fremgå tydeligt, når man ser på kommentaren, at forfatteren ikke selv har givet den i den kontekst. Forfatteren bør dog kunne få notifikationer om, hvis andre kopierer dennes kommentar til andre steder (f.eks.\ hvis de mener, kommentaren hører til under et andet faneblad (ah, der var ordet; fane(blad)!)), og bør så have mulighed for at godkende sammenhængen, så kommentaren her kan miste det specielle udseende, der signalerer at forfatteren ikke (endnu) har godkendt sammenhængen for kommentaren. Herved får vi et system, hvor der ikke behøves nær så meget arbejde, for at få kommentarerne nogenlunde ordnet; vi behøver at folk skal vurdere hver kommentar ind i de rigtige sektioner (og/eller vi behøver ikke et system (i et andet lag) til at gætte hvor kommentaren har relevans). Ikke at et sådant system ikke også sagtens ville kunne fungere og implementeres nogenlunde nemt, men det er bare et væsentligt mere kompliceret system (og dermed også sværere at forstå). Alternativet, som jeg foreslår her, er mere simpelt, og måske endda også bedre alt i alt (bl.a.\ fordi kommentarforfattere så har kontrol over dem). Og kommentar-kategorier bliver også hermed mere simple, for de vil så bare være relationerne, som kommentar-forfatterne giver til (eller godkender for) deres kommentarer. 

Okay, men så kom jeg også i forgårs aften til at tænke på nogle ting, som gjorde mig lidt i tvivl og fik mig til at overveje det hele omkring rating-tag-systemet. Så i går tog jeg en hel tænkedag, hvor jeg fik lidt styr på tankerne igen og kom bedre frem til, hvad de vigtige punkter ved idéen er. Jeg er jo selv egentligt langt mest passioneret omkring diskussions-(/debat-)delen, så jeg ser lidt rating-folksonomy-idéen som et skridt imod dette. Og det er jo så særligt fordi, systemet kan introducere de vægtede brugergrupper, som man skal bruge\ldots\ tja, eller man \emph{skal} ikke bruge dem, men det lægger stadig meget godt op til\ldots\ idéerne spiller godt sammen, er hvad jeg prøver at sige. (Men ja, det var faktisk tanken om: ``hvad skal overhovedet gå så meget op i folsonomy-idéen for?'' der fik mig tænke mig om nu her.) Men nu er jeg nemlig også kommet frem til, at jeg faktisk har nogle gode, simple punkter til rating-folksonomy-idéen, som ret nemt kan forklares, og at jeg faktisk \emph{ikke} behøver at komme så meget ind på open source-mulighederne i det, samt hvordan det spiller sammen med (og kan føre til) det semantiske web. Så jeg kan altså muligvis forklare om meget af idéen ret kort, og som sagt spiller det jo godt sammen med mine debat-idéer (i.e.\ begge ting kan muligvis stå på egne ben, både min r.f.-idé og min debatside-idé, men hvis kun den ene slår igennem, så kan den måske føre den anden med sig, fordi ligesom støtter hinanden ved begge to afhænge meget af `brugergrupper'). 
Jeg har forresten nogle flere bemærkninger relateret til navigation og til web of apps, men dem må jeg lige vende tilbage til senere. 
Men ja, nu tror jeg nemlig, at jeg kan holde mig rimeligt meget til tre vigtige punkter for idéen: Hvorfor rating-tag er smarte, hvorfor (vægtede) brugergrupper er smarte, og hvorfor ordnede kommentarer er smarte. Simpelthen! Angående andet punkt har jeg nu også en vigtig pointe, som jeg kom frem til i går, nemlig at brugergrupper virkeligt er god idé, fordi de giver pludselig meget mere værdi for brugeren til at vurdere ting. Med disse får brugeren nemlig pludselig \emph{selv} noget ud af, at vurdere ting (og komme med kommentarer, som andre vurderer) på webbet. For dette kan nemlig så alt sammen bruges til, at brugeren lærer mere om sig selv ved at lære om, hvilke brugergrupper man align'er med. Jeg tror så, at dette faktum kan øge vurderingsaktiviteten gevaldigt. Nå ja, og så skal jeg nemlig heller ikke glemme min idé om at offentliggøre korrelations-data (på anonymiseret vis) til brugerfællesskabet, så dette kan bruges til at få øje på korrelationer (der jo så kan være potentielle brugergrupper). Denne idé hører lidt sammen med, ``hvorfor brugergrupper er smarte,'' men det er også en selvstændig idé, så jeg kan jo se den lidt som et ``punkt 2 ½.'' Jeg skal selvfølgelig også nævne (omkring ``punkt 1''), at oversigter, hvor man kan se ressourcers vurderinger relativt til hinanden, er vigtig (og vil øge vurderingsaktivitet), men dette er en lidt mere oplagt idé. 

\ldots Okay, nu fik jeg lige tænkt over nogle flere ting, og særligt har jeg faktisk nu fået godt styr på mine ``brugergrupper.'' Jeg mener endda nu, at man ret nemt kan bruge dem til feed-sorterings-algoritmer (nemlig bare ved at bruge funktioner fra SQL). Inden jeg beskriver dette, kan jeg lige sarte med at pointere, at jeg nu vil begynde at kalde det `vægtfordelinger' i stedet, og så kun bruge begrebet (vægtede) `brugergrupper' specifikt om de moduler/systemer, som kan indgå i som led vægtfordelinger, hvor man har de omtalte ``moderator-vægte'' samt stemmevægte på det nederste plan, og hvor der altså er tale om et system, hvor medlemmerne har nogle tokens (nærmest) af forskellige slags, de kan dele ud i mellem sig. Lad mig også skynde mig at nævne, at en af de store ændringer, jeg lige kom på nu her (se mine brainstorm-noter ude i kommentarerne fra i dag (d.\ 26/10)), er, at de ``opdelte brugergrupper,'' hvor man har flere undergrupper, man tilknytter bestemte kategorier, nu ikke skal være på den måde. I stedet skal man bare have muligheden for at lave kategori-begrænsninger som en del af vægtfordelingsfunktionerne. Hm, dette kan jeg lige vende tilbage til. Jeg tror som nævnt nu på, at man sagtens bare kan få lov at skrive SQL-baserede funktioner som så kan implementere vægtfordelingsfunktioner. Hm, lad mig forresten lade `vægtfordelingsfunktioner' betegne, hvad der svarer til et felt / en proporty i en decorator-design pattern-agtig ``vægtfordeling-klasse'', som altså tager en eller flere tidligere vægtfordeling(er) og en funktion (SQL-baseret) og giver en ny (dekoreret) vægtfordeling. En speciel form for vægtfordeling er så en `vægtet brugergruppe,' som jeg lige har nævnt, hvad indebærer. Og for så at vende tilbage til kategori-begrænsninger som lovet, så kan dette altså ses som (og implementeres som) specielle v.f.-funktioner. Med andre ord kan vi altså have `vægtede brugergrupper,' som resulterer i vægtfordelinger, og herfra skal man så kunne konstruere mere avancerede vægtfordelinger ved at lægge funktioner over tidligere vægtfordelinger og altså blande dem sammen på forskellige måder, f.eks.\ ved at tage en linearkombination, ved at addere vægte eller ved at gange vægte sammen (eller ved en blanding af disse ting). En ny vægtfordeling kan uploades til platformen, så den kan bruges af andre, eller brugeren kan også bare beholde den selv og bruge den i sine egne sorteringsalgoritmer m.m. I øvrigt skal v.f.-funktioner også kunne bruge tiden for ressourcernes/termernes upload (altså deres timestamp) i sig --- dog helst på en ret simpel måde, så afhængigheden ikke er blandet alt for meget sammen med andet aritmetik, men kan faktoreres ud af regnestykket for så vidt muligt. Jeg behøver ikke at gå i detaljer med, hvordan jeg vil implementere de kategori-begrænsede v.f.-funktioner (nemlig ved brug af $\subset$-relationen, som jeg vistnok heller ikke vil komme ind på) i min første udgivelse her. Jeg kan bare nævne det og sige, at det må kunne lade sig gøre, og at jeg har en plan for, hvordan det kan lade sig gøre (som ikke er vildt kompliceret, men stadig lige en tak mere kompliceret, end hvad er værd at bruge tid på). 

En anden ting, jeg også skal nævne, som jeg kom på i går, er en idé til, hvordan brugere kan oprette anonyme kommentarer og vurderinger (og lad os bare kalde det `uploads' under et begreb), men hvor de stadig til en vis grad kan få lov at bruge deres stemmevægt i en brugergruppe til at fremføre uploadet. Hm, måske skal jeg faktisk lige tænke en anelse mere over det, for nu bliver brugergrupperne jo lidt mere flydende\ldots\ Ah, nå nej, det gør de ikke. Der bliver nu en rimelig begrænset mængde af relevante brugergrupper, og det bliver de brugergrupper, der fordeler tillid til folk\ldots\ Hm, men så skal jeg jo forresten til at ændre hele min terminologi, for så bliver ``brugergrupper,'' der har korrelerede meninger, nu ikke længere `brugergrupper,' men `vægtfordelinger' i stedet\ldots! Ah okay, så bliver terminologien altså lidt rodet for mig. Hm, lad mig lige tænke lidt over det (og så vende tilbage til min idé til anonyme uploads)\ldots\ \ldots Hm, hvad med at kalde de nederste systemer for FOAF(-agtige)-systemer og så kalde vægtfordelingerne i de næste lag, hvor man bl.a.\ kan bruge korrelationsvektorer for brugergrupper\ldots\ Hm\ldots\ Tjo, ja, og så kan man måske bare kalde det både brugergrupper og vægtfordelinger alt efter, hvad der lige giver mest mening\ldots? Eller man kunne måske kalde dem vægtfordelinger, men så påpege, at de former, hvad man kan kalde ``brugergrupper,'' og hvad man nemlig også kan se på som brugergrupper. Og denne måde at se på det holder så bare ikke, når først timestamps/tidsparameteren kommer ind i billedet og/eller når tingene bliver alt for komplicerede. Men så længe man kan se på vægtfordelingen som værende repræsenterende for en vis gruppe af mennesker med nogle bestemte karaktertræk (ift.\ deres meninger, deres kommentar-uploads og deres FOAF-forbindelser), så kan vi jo også fint kalde dem for ``brugergrupper.'' Hm ja, mon ikke jeg bare skal sige det sådan? Så (vægtede) ``brugergrupper'' er altså en måde at anskue visse (passende) vægtfordelinger på. Ja, ok for nu. Og tilbage til min idé til anonyme uploads: Brugere skal så have lov til at vælge FOAF-brugergrupper, og gerne nogle omfattende nogen af slagsen, hvor de så har en vis (middel eller god) stemmevægt, de gerne vil benytte til at fremføre deres anonyme upload. Brugergruppen\ldots\ hov, vent, min idé kræver så lidt, at sådanne brugergrupper har en central\ldots\ Hm, nå ja, men hvis vi snakker FOAF-netværkene specifikt, som jo har moderator-vægte osv., så er der jo også en form for central\ldots\ Hm\ldots\ %Men hvad så med sammensatte FOAF-brugergrupper..?.. Hm, jeg tror lige, jeg må tænke en anelse over dette.. 
%(27.10.21) Det er faktisk ret simpilt. Der skal jo være en central eller flere centraler uanset hvad, det arbejder jeg allerede ud fra, så løsningen er jo bare, at ja, brugere kan vælge en brugergruppe, og så er det ikke brugergruppen men platformen, der godkender anonyme uploads og registrerer dem. Det nytter forresten nok ikke noget, at serverne sletter ophavmanden igen, for man vil jo nok gerne undgå mulighed for gentagelser fra sammme ophavsmand (både af tilsvarende kommentarer men også især af vurderinger). Hm, og hvis man er bange for at blive hacket, så kunne man måske gardere sig ved bare at gemme et hash, som man kun kan reproducere med brugerens password. Når brugeren så uploader en anonym vurdering (eller kommentar, I guess..(?)..) og i samme forbindelse giver sit password, så kan serveren udregne hashet og tjekke, at brugeren ikke har givet samme vurdering før. Efterfølgende kan serveren så slette passwordet fra hukommelsen igen, hvad den alligevel vil. Hvis den nye vurdering godkendes, så gemmes det udregnede hash altså sammen med det (og hvor angribere altså så ikke skal kunne få noget ud af at læse hashet, og hvor det derfor gerne skal krydres med 'salt,' før man udregner det). 

(27.10.21) Okay, det er faktisk ret simpelt. Der skal jo være en central eller flere centraler uanset hvad, det arbejder jeg allerede ud fra, så løsningen er jo bare, at ja, brugere kan vælge en brugergruppe, og så er det ikke brugergruppen men platformen, der godkender anonyme uploads og registrerer dem. 

Okay, men tilbage til min forklaring: Brugere vælger altså en stor nok brugergruppe, hvor de har en ok stemmevægt og beder (en af) platformens service-udbyder(e), i.e.\ det (eller de) firma(er), der administrerer platformens database(r), og som bevarer folk login-oplysninger osv., om at registrere et anonymt upload. Hvis forespørgslen godkendes, så runder platformen brugerens vægt op eller ned (og/eller lægger en tilfældig lille værdi til (først)), så andre ikke kan se, præcis hvilken bruger det kommer fra, men kun kan se en nogenlunde værdi for vedkommendes stemmevægt. Uploadet bliver nu registreret som kommende fra den pågældende brugergruppe (eller `vægtfordeling' rettere), men ellers gemmes der ikke andre oplysninger, udover selvfølgelig den omtalte, ``maskerede'' stemmevægt. Eller dvs., man man kan godt gemme ekstra data, hvis brugergruppen er stor nok\ldots\ Nå nej, det jeg skulle til foreslå svarer bare til, at man vælger en mindre brugergruppe (som så måske kan repræsentere visse menings-korrelationsvektorer (altså hvad man lidt teknisk kunne benævne som `en undermængde af det vektorrum, som de offentligt tilgængelige korrelationsvektorer udspænder')), så never mind. Det vil dog så være en god idé, hvis man i brugerfællesskabet beslutter en konvention for, hvordan man udvælger sine brugergrupper, så folk valg ikke bliver præget af individuelle beslutninger, som så kan blive et kendetegn for individet og som i princippet kan spores tilbage til personen. Systemet fungerer bedst, hvis folk bruger samme konventioner, hvor variationen af, hvilke brugergrupper/vægtfordelinger folk vælger, sker ved at justere et begrænset antal parametre i en formel, som alle (eller i hvert fald en stor mængde af brugerskaren) så kan bruge.  

Jeg har i min kommentar-brainstorm (i kildekoden til denne tekst) nævnt, at serverne bare bør slette data om, hvem der har overvejet hvad, men det nytter forresten nok ikke i længden, for man vil jo nok gerne undgå mulighed for gentagelser fra samme ophavsmand (både af tilsvarende kommentarer men også især af vurderinger). Hm, og hvis man er bange for at blive hacket, så kunne man måske gardere sig ved bare at gemme et hash, som man kun kan reproducere med brugerens password. Når brugeren så uploader en anonym vurdering (eller kommentar\ldots?) og i samme forbindelse giver sit password, så kan serveren udregne hashet og tjekke, at brugeren ikke har givet samme vurdering før. Efterfølgende kan serveren så slette passwordet fra hukommelsen igen, hvad den alligevel vil. Hvis den nye vurdering godkendes, så gemmes det udregnede hash altså sammen med det (og hvor angribere altså så ikke skal kunne få noget ud af at læse hashet, og hvor det derfor gerne skal krydres med `salt,' før man udregner det). \ldots Hm, angående anonyme kommentarer, så kan man måske godt vedtage, at brugere ikke skal kunne uploade mere en én anonym kommentarer for hver kommentar-kategori / kommentar-relation for en given ressource. 


Fedt, det var de primære ting, jeg skulle følge op på her. Nu skal jeg bare lige nævne en ting omkring tag-dokumentationer og så lige kort nævne nogle tanker omkring ``WoApps'' / at platformen kan ``bootstrappe sig selv.''

Den første ting handler bare om, at jeg i mine ovenstående (2021-)noter fik sagt, at tag-dokumentationerne selv skal definere, hvordan semantikken bag rating-skalaen skal være, i.e.\ hvordan forskellige værdier skal tolkes. Jo, dette kan godt blive sådan i princippet, men det er en rigtig god idé, hvis tags alligevel for så vidt muligt holder sig til fælles konventioner for, hvordan disse skalaer skal være. Det kan f.eks.\ være noget med at sige, at for ja/nej-udsagn skal hvert punkt på skalaen betegne en grad af `enighed' (som man måske også kan uddybe nærmere), og/eller hvor sandsynlig man finder udsagnet (og/eller hvor sandsynligt det er, at man skifter mening) --- og man i øvrigt kan godt bruge en form for logaritmisk skala, hvis dette er mere brugbart og/eller intuitivt. Hm, måske vil et design, hvor brugeren altid kan se eksempler på, hvad skalaens punkter repræsenterer, når det afgiver en vurdering, om ikke andet være en ok start, men det vil være rigtigt smart, hvis man kan finde en eller anden fælles konvention for, hvordan skalaerne skal være indrettet, så folk kan få meget ud af at se en vurdering for sig selv, uden at skulle klikke sig ind og se, hvordan skalaen er defineret. Det må altså meget gerne være sådan, at brugere allerede kan få en god idé om, hvad en ting er vurderet til, bare ved at se selve vurderingsresultatet og altså uden at have kendskab til skala-dokumentationen. Selvom alt dette er en smule kompliceret, så tror alligevel folk generelt vil have en god intuitiv forståelse for, hvornår en skala er godt eller dårligt konstrueret/defineret, så hvis man bare sørger for tidligt også at skrive nogle generelle koncepter omkring, hvilke konventioner man nogenlunde bør holde sig til, som de engagerede brugere (som er dem der laver dokumentationerne) kan læse (og blive enige om) på et tidligt stadie, så tror jeg på, at det hele nok skal give sig selv meget godt derfra. Og så kan man bare forfine konventionerne løbende og arbejde på at ændre skalaerne for de mindst intuitive tags, der alligevel er blevet taget i brug af brugerne.

Angående det med, at platformen kan bootstrappe sig selv (hvis den udføres open source (hvad jeg dog ikke vil lægge op til i min første udgivelse her)), så har jeg bare lige lyst til at nævne nogle ting, bl.a.\ at ``ressourcer'' jo også kan betegne HTML-dokumenter, hvilket jo bl.a.\ kan være oversigter over andre ressourcer med tilhørende links i. Måske har jeg nævnt dette ift.\ navigations-relation-menuen, for så er tanken jo, at en brugbar navigations-reference kunne være til en ``oversigt over den kategori (eller den serie), som ressourcen, $x$, tilhører.'' Og disse kan så implementeres som HTML-ressourcer. I øvrigt kunne platformen så åbne op for, at HTML-ressourcer kan bruge visse (godkendte) funktioner til at kommunikere med serverne, og bl.a.\ anmode om sorterede lister af ressourcer (sorteret ud fra det system, jeg lige har beskrevet). Så HTML-ressourcerne kan således også komme til at kunne bruges som alternative brugerflader til siden. Og så er man kørende, for så kan engagerede brugere (hvis de er der) jo selv designe alternative brugerflader ad libitum til platformen, som så kan komme til at kunne alt det samme, som standard brugerfladen kan. Og brugere kan så bruge selve platformen til at vurdere og søge efter gode brugerflader til platformen selv. Og hvis man så bare kan stole på, at platformen vil engagere sig også og være gode til at gennemgå forslag til nye server-klient-kommunikations-funktioner, som brugerfladerne kan bruge, så er der ingen begrænsning for, hvad man kan opnå. (Og det er så denne vision, altså om en open source platform, der ``bootstrapper'' sig selv, ved at brugerne selv kan uploade nye brugerflader til den, og hvor brugere så i sidste ende kan komme til at designe alle mulige, ikke bare brugerflader, men også andre applikationer på siden, jeg nogen gange har kaldt ``web of apps'' i mine noter her.) En naturlig kritik af alt dette vil så være at spørge: Hvordan adskiller dette sig fra et hvilken som helst anden open source-applikation, og særligt: Hvordan adskiller det sig fra f.eks.\ GitHub? Her kan brugere jo også uploade nye brugerflader til platformen selv, så hvad er forskellen? Alt dette er rigtigt, og det er også bl.a.\ derfor, at jeg har nedprioriteret udgivelsen af denne idé, men jeg tror nu altså alligevel muligvis, der er noget stort at komme efter. Den store pointe, der gør, at jeg tror, min vision tilføjer noget til open source-bevægelsen, er at på sådan en platform, som den jeg har beskrevet, der kan open source-programmørerne få \emph{meget} mere opmærksomhed for deres arbejde. Hvis man således starter med en platform fuld af brugere, og hvor open source-programmører således har mulighed for at uploade forbedringer til brugerfladen \emph{på} selve platformen, så kan andre brugere jo følge dem og give dem opmærksomhed \emph{på} platformen (ligesom at alle mulige andre skabere kan få opmærksomhed). Brugere kan så følge gode programmører/programmør-grupper på siden, hvorfra de så kan få gode forslag til udvidelser af brugerfladen (i form af HTML-dokumenter, som brugerne kan starte fra samt browserudvidelser, de kan installere for at tilføje ting til disse HTML-startsider). Så man skal altså forestille sig, at open source-programmører her, hvad der svarer til ``kanaler'' på YouTube, hvor brugere kan følge dem og like'e deres ting (og altså give dem god opmærksomhed). I stedet for videoer eller andet \emph{indhold}, så er det her altså udvidelser til brugerfladen, som skaberne giver til brugerne. Så brugere følger altså ikke kanalerne for at få mere \emph{indhold} til hjemmesiden/platformen, men for at få forslag til udvidelser af rammen til platformen, så at sige (altså til brugerfladen). Og da dette bare vil være HTML/CSS- og JavaScript-arbejde, så vil en stor platform hurtigt kunne tiltrække en masse programmør-skabere, der har lyst og evner til at skabe udvidelser til platformen. Og hvis platformens enhed, der administrerer serverne så også er gode til holde øje med, om brugerfællesskabet har brug for flere server-klient-funktioner, så de kan udvikle flere, ikke bare brugerflade-designs, men også applikationer på platformen, så kan sådan en ``bootstrapped'' platform altså virkeligt nå langt, tror jeg. Men selv hvis platformen ikke udvikler sig videre derfra, og at den altså ikke ligefrem kommer til at udgøre et ``web of applications'' *(eller en ``omni-side,'' som er et begreb, jeg også har brugt i mine noter), så tror jeg dog altså allerede, at situationen, som jeg lige beskrev, hvor skabere på platformen også kan inkludere brugerflade-designere, virkelig kan blive godt. 


Det her er næsten for trivielt til at nævne, men lad mig nu også lige kort påpege, at det meget vel kan være værd for platformen, hvad end den er open source eller ej, at tage gængse sider og så gøre det nemt at navigere fra disse sider og til platformen, således at brugere kan browse ressourcer på andre (populære) sider på internettet, og så få en nem måde at navigere derfra og så til platformens referencer til de pågældende ressourcer, hvorved man så kan se diverse tag-vurderinger fra platformen. Jeg tænker altså, at det muligvis vil være smart, hvis brugere hurtigt kan tilgå vurderinger fra platformen til ting, også selvom de browser dem på andre platforme. Hm, tja, men det kan også være, at det vil være for meget arbejde, det ved jeg ikke. Men det kunne da være smart --- og det kunne muligvis bidrage til væksten af platformen (hvis det bliver nødvendigt), hvis folk lynhurtigt kan tilgå platformens-tag-vurderinger, når de browser ting på andre (populære) platforme. Og en sådan funktionalitet kunne så muligvis implementeres med en browserudvidelse, der for de pågældende sider kan læse, hvilke ressourcer man betragter, og indsætte et link på siden --- måske i form af et ikon. Ikonet kunne så i første omgang meddele brugeren, om der kan hentes data om pågældende ressource --- muligvis udløst af, at brugeren holder over ikonet (for at undgå for meget overskydende arbejde hos klienten og r.-tag-platformens servere). Og hvis ressourcen kan matches med en ressource på r.-tag-platformen, så kan ikonet eventuelt tilføje et link til platformens ressource med dens tag-vurderinger, og/eller det kan simpelthen hente de mest populære vurderinger over, så brugeren kan se disse direkte på den anden platform. Jeg følte, jeg bare lige blev nødt til at nævne denne tanke, ikke mindst fordi dette jo altså kunne være en mulig måde at forøge platformens vækst på, hvis nu der går lidt trægt (nemlig ved at tilføje brugbare ting til mere gængse sider). For produkter, man kan købe, vil det i øvrigt også være rigtig relevant, hvis der er links til andre platforme, hvor man kan købe produktet, hvis man ikke kan det på selve (r.-tag-)platformen, det er klart. Så alt i alt her (i denne pargraf) altså bare lige nogle små bemærkninger, jeg følte, jeg blev nødt til lige at nævne.


Men ellers var det sgu alt nok det, jeg ville nævne (og overveje / arbejde på) af opfølgende tanker og idéer! Nu vil jeg så lige sikre mig, at jeg har en god idé om, hvad skal med og ikke med omkring min r.f.-platform og beslægtede idéer i min første udgivelse, og så vil jeg gå i gang med at udarbejde den! 






(28.10.21) Hov, jeg har glemt en ting. Nu har jeg jo rimeligt godt styr på, hvordan brugerne kan opnå vurderingsresultater, og hvor de så kan justere brugergruppe-/vægtfordelings-indstil-%(skal force linjeskiftet her.)
linger for at se på, hvad forskellige grupper siger. Man hvad med mine tanker om at kunne lægge funktioner hen over rating-akserne?\ldots\ Nu snakker vi altså om feed-sortering. Ja, jeg mangler lige at komme ind på, hvordan man aggregere flere tags resultater til en endelig værdi, som sorteringen så kan følge. Tja, men nu giver det nærmest sig selv, for svaret bliver jo bare: Når det kommer til feed-sorteringer eller relevans-scorer for søgninger (som kan\ldots\ hov, måske skal jeg forresten også lige berøre søgeord-søgninger\ldots)\ldots\ så kan brugerne bare vælge en funktion for, hvordan tag-vurderings-resultaterne endeligt sættes sammen til én værdi. \ldots Hov! Det jeg skrev omkring brug af timestamps i algoritmerne, det er jo netop relevant for denne endelige `relevans-score-funktion,' kan vi kalde den her, og ikke så meget for brugergrupperne/vægtfordelingerne; ikke medmindre man fortrækker at bruge fortidige udgaver af, hvordan brugergrupperne så ud i ens brugergruppe-/vægtfordelings-algoritmer. Så ja, ret vigtigt, at jeg lige fik husket på, at der bliver disse to (ydre) lag i det; nemlig algoritmer til at finde frem til rating-resultaterne (og til at kunne variere imellem forskellige ``brugergrupper''), og så funktioner til at samle ting til en relevans-score. Ja, og der er jo forresten også alt det med, at man jo gerne vil have mulighed for at søge ikke bare på ekstremerne, men netop også på intervaller midt på akserne. 

Nå, hvordan dælen kan man så implementere muligheden for brugbare relevans-score-funk-%force newline
tioner, uden at det bliver alt for besværligt for brugere og for servere? Hm\ldots\ Ah, men man jo bare sørge for, at rel.-score.funktionen i første omgang bygger på funktioner, der kan udføres i SQL, og så kan man eventuelt udvide dette system ved at brugerne også kan smække en efter-sorteringsfunktion på, som så så kører på klientens side. Cool. Så brugere kan selv skrive disse funktioner, og kan så vælge en samling af disse, de vil bruge, og som de så kan skifte imellem til forskellige søgninger. Funktionerne skal så i første omgang bestå af server-side-lag, hvor funktionerne skal være kompatible med SQL, eller hvad serverne nu er gearet til, og ovenpå dette kan man så eventuelt have endnu et lag bare med klient-side-funktioner, som omsorterer og/eller frafiltrerer ting fra den liste, som server-side-funktionerne gav i første omgang. Hm, og den eneste tidsrelaterede funktion, man behøver i brugergruppe-/vægtfordelings-funktionerne, er så nok bare muligheden for at vælge et start- og/eller et slutpunkt for det tidsinterval, man vil se på. Hm ja, og dette vil jo klart kunne gøres i SQL (osv.), så det jo fint.

Hm, lad mig lige prøve at skrive vægtfordelingsfunktionerne op lidt mere formelt. Lad os benævne en vægtfordelinger, der kommer fra et modereret (FOAF-agtigt) system, som $f_0, g_0, \ldots$. En ny funktion, $f'$, kan så defineres ud fra tidligere funktioner, $f, g, \ldots$, på følgende måder. Man kan i første omgang altid sammensætte tidligere funktioner ud fra aritmetik: $f' := A(f, g, \ldots)$. Man skal også kunne\ldots\ Ah, nu ved jeg, hvad jeg var ude efter. Hvis man skal begrænse tidsintervallet, så må det skulle ske som det første skridt; før vi får vores $f_0, g_0, \ldots$ essentielt set. Og dette kan jo så blive lidt tungt for serveren, hvis den skal holde styr på vægte for alle tider\ldots\ Hov ja, der er også et andet problem. \ldots Hm, eller noget andet, jeg i hvert fald også skal tænke over\ldots\ \ldots Okay, lad mig bare antage, at platformen giver mulighed for, at brugerskaren kan gemme gamle versioner af brugergrupper, og/eller den gemmer snapshots løbende automatisk. Ok. Men nu er det så gået op for mig, at jeg lige skal overveje også, hvordan korrelationsvektorerne skal udregnes og offentliggøres. Hm, det handler vel om, at brugerne skal kunne anmode om (og måske ved at samle nok underskrifter til at blive prioriteret frem for andre ansøgninger), at platformen tager en specifik brugergruppe (i.e.\ en vægtfordeling) pr.\ anmodning, samt en mængde tags og en mængde ressourcer, og så udregner korrelationsvektorer og udgiver så mange af de mest signifikante korrelationsvektorer, som de kan, uden at det bryder anonymiteten hos\ldots\ Hov, men anonymitet er jo nu en ret triviel ting, er det ikke?\ldots\ Nå jo, never mind. Så platformen kan bare udgive så mange relevante korrelationer, som den gider at udregne\ldots\ hm, vent, hvad så hvis mange brugere vælger at uploade alt anonymt? Vil det være en god ting, at anonyme uploads også kan forbindes\ldots? Hm, eller hvad?\ldots\ Nej. Korrelationer må man opnå ud fra ikke-sensitive områder primært, og når det kommer til sensitive ting, så er det helt fint, hvis brugere bare lige kan vælge at tilføje lidt info til deres anonyme uploads, om hvor de befinder sig henne i rummet af forskellige typer brugere. Jep, så anonymitet bliver nu altså ret trivielt for platformen at overholde (hvilket er godt). Og denne kan altså bare, til den omtalte anmodning, udregne og offentliggøre så mange korrelationsvektorer, som den vil bruge ressourcer på at udregne. Når den har gjort det, offentliggør den så denne udregning (som jo nu allerede er offentlig i forvejen, nu hvor anonymiteten er triviel; den skal bare udregnes) i form af et korrelationsobjekt, kan vi kalde det, som siger, at ``brugergruppe $x$ til tiden, $t$, havde disse korrelationer på domænet $D$,'' hvilket så altså består af en mængde ressourcer (muligvis defineret som tilhørende en ressource-kategori, hvis siden implementere disse (hvad den bør gøre på sigt)) og tags. \ldots Hm, jeg brainstormer jo bare nu: Så kunne brugere måske\ldots\ Ah, vent, brugere kan jo også selv lave udregningerne, så inden da kan brugerne selv opnå objektet, og kan i øvrigt så vælge at (om)parametrisere vektorerne, som de vil, inkl.\ hvor de ender med et afhængigt sæt af vektorer (som så altså ikke udgør en ortogonal basis)\ldots\ Hm, men om dette nu også giver mening, kan jeg lige finde ud af, når nu jeg finder ud af mere præcist, hvordan algoritmerne kan benytte disse korrelationsvektorer. 
%Hm, sidenote: "User-driven machine learning" er da forresten ikke noget dårligt navn for det..! (Jeg har bare kaldt det det som et midlertidigt navn uden at tænke så meget over det, men navnet giver da faktisk mening, gør det ikke? :) (det synes jeg).)
\ldots Hm nej, jeg er altså ikke sikker på, at anonymitet bliver trivielt\ldots\ Hm nej, der er nok mange, der helst ville have det sådan, at andre ikke kan gå ind og se, hvad man har vurderet, og måske heller ikke hvad man har uploadet af kommentarer\ldots\ Hm, men kan man så ikke bare gøre det anonymt, og så kan man jo bare selv altid regne ud, hvilken ``brugertype,'' man tilhører\ldots? Men hvis mange så gør dette\ldots\ Ah jo, måske ville det være godt, hvis anonyme uploads ikke behøver at tilknyttes en brugergruppe, som jeg har beskrevet, men bare kan uploades til databasen, hvor denne så ikke oplyser noget om\ldots\ Okay, lad mig brainstorme videre ude i kommentarerne og så vende tilbage med resultaterne\ldots 
%Spørgsmålet er lidt, hvor vigtigt det er, det her med at brugere skal kunne tilknytte etos-vægt til anonyme kommentarer og vurderinger, kontra hvor vigtigt det er at benytte vurderinger... Ah, men skal jeg ikke bare dele det skarpt over der?..! Skal jeg ikke bare sige, at for uploadede termer (selvstædige ressourcer eller kommentarer *(og det bliver nok især relevant for kommentarer specifikt)), så kører det efter første princip, og når det kommer til vurderinger, hvilket jo netop er, hvad man kan lave korrelationsvektorer ud fra (det bliver langt sværere at få forfatterskab til kommentarer med ind i udregningen, og det er nok altså slet ikke det værd!), så offentliggør platformen altså ikke direkte, hvem der har vurderet hvad, medmindre brugerne vælger at offentliggøre dette for deres profil (men hvor de efterfølgende altid kan vælge at af-offentliggøre det igen og slette det fra deres profil). Og så er det, at...
\ \ldots Nå never mind; jeg hiver lige teksten ind her i det renderede, for den blev ikke så lang, før jeg fandt på noget værdifuldt: 
Spørgsmålet er lidt, hvor vigtigt det er, det her med at brugere skal kunne tilknytte etos-vægt til anonyme kommentarer og vurderinger, kontra hvor vigtigt det er at benytte vurderinger\ldots\ Ah, men skal jeg ikke bare dele det skarpt over der?\ldots! Skal jeg ikke bare sige, at for uploadede termer (selvstædige ressourcer eller kommentarer *(og det bliver nok især relevant for kommentarer specifikt)), så kører det efter første princip, og når det kommer til vurderinger, hvilket jo netop er, hvad man kan lave korrelationsvektorer ud fra (det bliver langt sværere at få forfatterskab til kommentarer med ind i udregningen, og det er nok altså slet ikke det værd!), så offentliggør platformen altså ikke direkte, hvem der har vurderet hvad, medmindre brugerne vælger at offentliggøre dette for deres profil (men hvor de efterfølgende altid kan vælge at af-offentliggøre det igen og slette det fra deres profil). Og så er det, at anonymitet igen bliver noget, som platformen skal sørger for ikke at bryde, når korrelationsvektorerne udregnes og (en delmængde af dem) offentliggøres. 

Okay, og så kommer vi tilbage til spørgsmålet om, hvordan det offentliggjorte korrelationsdata skal benyttes af brugerne? De kunne jo som sagt for det første få mulighed for at omparametrisere det, men ja, hvad skal de så igen bruge dette til helt præcist? \ldots Ah ja, pointen bliver vel, at man så skal kunne bruge det til at generere nye $f_0$'er med, som så bare kun skal indgå mellemled i udregningerne, og skal altså blive på server-siden. Men ja, det man opretter kan så bruge baserne\ldots\ Ah, og så behøver man selvfølgelig heller ikke at anmode om omparametriseringer; det skal man bare kunne gøre frit som en del af funktionskonstruktionen. Så brugeren kan altså vælge deres egen basis, som godt kan være uafhængig, og så give hver vektor i basen vægte. Disse vægte kan så selvfølgelig transformeres over i den offentliggjorte (ortogonale) basis for korrelations-objektet, så resultatet altså bliver en vægt-vektor, der kan prikkes med hver af brugernes data-vektor, som korrelations-objektet blev udregnet på bagrund af. Dette resulterer så i en ny vægtfordeling, som så altså kan være et mellemled i en endelig udregning. Og hvis man i øvrigt holder visse af disse vægte variable, så vil det jo resultere i en variabel vægtfordeling. (Hm, ikke at dette nok kommer til at kunne bruges til noget, for fordelingen bliver jo ikke offentliggjort\ldots) Ah, og disse vægtfordelinger skal så kunne indgå som $f_0$'er i algoritmerne, bl.a.\ især ved at gange dem med en anden $f_0$ fra et af de FOAF-agtige netværker, hvorved man så får en ny ``brugergruppe,'' som så er vægtet, så de brugere, som omtalte vægt-vektor peger mest på, får en tilsvarende større vægt i det endelige resultat. Nice. 

Brugere kan jo så bede platformen om at offentliggøre projektionen af deres egen vektor på et korrelations-objekt (som så måske netop kun er taget over et domæne af ikke-sensitive emner) på deres profil, men ja, jeg tror nu ikke jeg vil lægge op til, at nye brugergrupper kan dannes ved at skæve til disse værdier, for en bruger kan jo altid bare konstruere sine vurderinger lige netop på det specifikke domæne, så det passer med den (projekterede) vektor, de ønsker at få offentliggjort. Ja, så i virkeligheden behøver jeg måske ingen gang at nævne denne mulighed med, at folk skal kunne få offentliggjort deres vektorprojektion. Især fordi domæner jo med fordel kan være stikprøve-domæner, hvor man udvælger en mindre undermængde af ressourcer, tilfældigt eller målrettet, og så vil det netop være for nemt for brugerne at konstruere den vektorprojektion, de ønsker. Så ja, det er nok langt bedre, at man i brugergrupper, der prøver at inkludere brugere med en vis menings-korrelation i højere grad bare ser på kommentar-/ressource-uploads. Hm, men hvad skal man så egentligt bruge disse menings-korrelerede bruger(-del-)netværk til? Til debat-platformen? Hm ja, jo. Ok, men dette er så alt sammen bare noget, som disse FOAF-agtige brugergrupper/delnetværker (\ldots `brugernetværker?') må finde ud af selv (i.e.\ hvordan de vil optage brugere til netværket). 
%Cool resultater, jeg er kommet frem til her i dag. Nu (d. (28.10.21)) vil jeg summe over disse ting til i morgen. 

(29.10.21) Okay, jeg er kommet frem til nogle flere ting her i formiddags/middags. For det første vil jeg gerne pointere, at anonymitet nu stadig næsten er trivielt (i hvert fald ret let), for hvis platformen bare sørger for aldrig at offentliggøre korrelationsvektorer, der baserer sig på for få personer, og aldrig offentliggør de mellemregnede vægtfordelinger, hvor folk er vægtet og fra deres korrelationer, jamen så er der ellers helt fri leg, når det kommer til at stille på parametre i søgeindstillinger. For ingen kilder til vurderinger bliver nu offentliggjort, medmindre brugere selv gør det på deres profil, så folk kan ikke holde øje med, hvordan nye vurderinger fra en specifik bruger opdaterer relevans-score-resultaterne. Så hvis platformen altså bare lige sørger for dette, så kan brugere altså frit komme med vilkårlige algoritmer, hvor brugerskarens vægtfordeling kan blive justeres ud fra deres projektioner på de offentliggjorte korrelations-objekter (som rettere bør kaldes korrelations-vektorsæt). 

En anden ting, som er rigtig vigtig at sige, også især fordi den retter noget, jeg har skrevet tidligere, er, at måden man typisk vil bruge korrelations-vektorsættene på, \emph{ikke} er, som jeg har beskrevet det, hvor man basalt set vægter alle stemmevægte med $\bra{a} P \ket{v}$, hvor $\ket{a}$ er de brugervalgte variable, $\ket{v}$ er datavektoren til den pågældende bruger, der skal vægtes, og $P$ er en projektionsoperator, der projekterer $\ket{v}$ ind på korrelations-vektorsættets basis (og hvor vi så her bare har antaget at $\ket{a}$ allerede er projekteret ind på / givet i denne basis). Dette bliver \emph{ikke} nødvendigvis særligt relevant at gøre. Det man nok nærmere vil være interesseret i, er at udregne $(\bra{a} - \bra{v} P^\dagger) S^{-2} (\ket{a} - P \ket{v})$, hvor $S$ så er en diagonalmatrix med brugervalgte $\sigma$-værdier, og så indsætte resultatet, kald det $x^2$, i eksempelvis $f(x^2)=\exp(-x^2 / \sqrt2)$. På denne måde opnår man en vægtfordeling, hvor hver bruger har vægt ud fra en normalfordeling af, hvor tæt de er på vektor $\ket{a}$ i domænet for korrelations-objektet/-vektorsættet. 

Og for at understrege pointen fra min første paragraf her fra i dag, så kan brugere altså frit justere $a$ og $S$ som de vil, og kan også frit bestemme andre formler at bruge, hvis de vil det. I øvrigt søgte jeg lige på det, og eksponentielfunktioner er kompatible med moderne SQL-servere (fra 2008 og frem eftersigende). 

Jeg skal også lige nævne, at brugere jo passende kan vælge domæner til deres forespørgsler efter korrelations-objekter, hvor de specifikt forventer, at der er korrelationer. De kan med andre ord vælge specifikke emner/kategorier, såsom `politik,' `mad,' `film' osv., hvor korrelationerne er særligt interessante, og hvor man også forventer dem. Det kan jo sagtens være, at der f.eks.\ vil kunne findes korrelationer imellem visse politiske holdninger og visse præferencer til mad, men for det første, vil disse korrelationer sikkert være ret små og sjældne, og for det andet vil man nok heller ikke kunne bruge disse korrelationer til noget specielt, hvis man finder dem (andet måske end pussigheden i dem), som man ikke ville kunne opnå ved at kende korrelationerne i de to domæner hver for sig. 

Så ja, domæner bliver faktisk en rimelig vigtig del i det hele. Disse kan jo så laves pr.\ forespørgsel fra brugere, ligesom for (og måske altså netop i forbindelse med oprettelsen af) korrelationsobjekterne. Man kan så også udvide dette system, så domæner kan dannes og opdateres automatisk ud fra specifikke kategori-tags, og videre kan man muligvis bruge min $\subset$-relation til at opdatere kategorierne/domænerne løbende, men det hører alts sammen til avancerede detaljer til systemet (og som muligvis kan tilføjes og forbedres løbende). Men ja, det er altså stadigt vigtigt, at brugerskaren altså kan oprette domæner/kategorier på siden. Og disse kan altså bruges som domæner til korrelations-objekterne, og de kan i øvrigt også indgå i vægtfordelingsfunktioner, hvis man gerne vil sammensætte en vægtfordeling, hvor forskellige typer brugere har mere eller mindre stemmevægt, alt efter hvad emne/kategori/domæne den relevante ressource og det relevante tag tilhører. Så jeg mener altså stadigvæk, at der på sigt bør indføres mulighed for, at lave sådanne vægtfordelingsfunktioner, hvor vægtene ikke bare afhænger af, hvilken bruger man kigger på, men også af hvilken kategori af ressource og/eller tag, man kigger på. Men det hører selvfølgelig også til noget mere avanceret, som man bare kan udvikle og indføre på sigt. Hm, men ja, jeg kan nu stadig godt lige nævne muligheden.

Lad mig i øvrigt lige pointere, at man også godt kunne udvide systemet, så korrelations-objekter tilknyttes en slags dynamiske domæner, som kan opdateres løbende, når mere data form af flere ressourcer og flere vurderingssvar kommer til, og man kunne så endda åbne op for, at platformen så kan opdatere disse korrelations-objekter automatisk løbende. Så skal man så bare lige sørge for ikke at gøre det for ofte, og man skal huske stadig at sørge for, at ingen store ændringer må skyldes for få brugeres aktivitet, så man alt i alt altså stadig bare lige sørger for, at brugerenes anonymitet opretholdes.

Fedt, og så når vi til en ny idé, jeg fik i morges, nemlig at der på platformen skal kunne være en slags ``ønske-vurderinger,'' eller hvad man skal kalde dem, hvor folk skal kunne oprette, lad os bare kalde dem ``ressource-ønsker,'' som er et sæt af vurderinger (inkl.\ tag-vurderinger til at bestemme kategorien for den ting, man ønsker sig, medmindre man vil gøre dette på en anden måde\ldots), som brugeren kunne ønske sig flere ressourcer af. Hm, jeg har egentligt ikke tænkt så meget over idéen, så jeg brainstormer altså bare lidt nu. Hm, det kunne også være rart, hvis en bruger kunne vælge en ressource, og så bare komme med en specifik ønske-vurdering\ldots\ Hm, kan man gå ud fra, at alle ressourcer bare er grundigt kategoriseret, så man altid bare kan tage en (eller flere) passende kategori-tag(s) fra den/de ressource(r), man kigger på, og så uploade ønske-vurderinger til denne. Hm ja, det må jeg næsten bare tillade mig at gå ud fra; at platformens ressourcer vil blive opdelt godt i kategorier --- brede såvel som specifikke --- og at man så derfor bare kan oprette sin ønske-vurderinger ved at knytte dem til passende kategorier. Ok. Så ja, dette er altså noget, jeg vil foreslå (og endda inkludere i første udvidelse), og så må jeg altså også endelig huske at beskrive, hvordan der også skal være kategori-tags ad libitum, og at jeg forventer, at brugerskaren vil have stor gavn af disse, og at der derfor også vil være mange folk, der har lyst til at vurdere kategorier til ressourcer. %*(Og hvis brugere så vælger en eller flere tags til at definere en kategori, så kan de jo bare selv også angive en tærskel for hvert tag, som således kan pege nærmere på, hvordan kategorien defineres (og altså hvar dens grænser ligesom går..)..)

Nå! Var det så endelig det? Har jeg styr på det nu? \ldots\ 
%Hm, jeg skulle måske også lige nævne kort omkring, hvordan man kan opslitte tags i flere for at hjælpe korrelations-objekterne, hvis nu et interessant tag har flere en to peaks i fordelingen af vurderingssvar.. 

%Hm, hvordan lapper man huller, for bruger der ikke overlapper med stikprøve-korr.-objekter..? ...Nå nej, det får man ikke behov for: Korrelations-objekterne kan godt defineres over donæmer, som kan opdateres løbende, og hvor platformen så også kan opdatere korrelations-objekterne, men sidstnævnte opdateringer skal så netop kontrolleres og ikke gøres for tit, så man bevarer folks anonymitet. 


%Domæner.. (tjek..?)
%Ønske-vurderinger.. (tjek..?)
%At opdatere lidt løbende, men stadig kontrolleret. (tjek)


(31.10.21) Det havde jeg ikke helt. Jeg er nu kommet frem til, at min idé om modererede (vægtede) brugergrupper nu faktisk bare bliver en sidenote til idéen om ``brugerdrevet ML''\ldots! For i bund og grund bliver værdien ved at have ``brugergrupper'' primært bare, at man så kan skille bots og gengangere fra. Og dette kan man som platform gøre på tusindvis af måder. Det er ikke andet end et gængst problem, der allerede har gængse løsninger, og som ikke er særligt stort til at begynde med. Så det eneste, jeg faktisk behøver at gøre, er bare, at nævne, at platformen evt.\ kan administrere et vist system af relationer selv, som beskriver data om brugere, og som kan bruges til at verificere, at det ikke er bots, og at de ikke er gengangere, og ellers bør platformen altså også åbne op for nogle ting, der gør at brugere selv kan hjælpe til. For det første bør platformen åbne op for brugerlavede prædikater/relationer, som brugere kan bruge til at erklære ting om andre brugere, hvilket altså f.eks.\ kan bruges til at implementere et FOAF- og/eller et Web of Trust-system. Platformen kan dog godt overvåge og godkende/afvise nye prædikater/relationer, for at sørge for, at prædikater ikke er direkte fornærmelser osv. Videre bør platformen så også åbne op for, at brugere kan udforme deres egne algoritmer (som udregner vægtfordelinger), så de slev kan implementere FOAF-/WoTrust-systemer. Ikke at brugere nødvendigvis vil engagere sig så meget i dette, for emnet er nemlig lidt tørt, men det er stadig klogt at åbne op for muligheden. Og så kommer vi til, at platformen også bør åbne op for et mere ``blødt'' system, hvor vægtfordelingerne administreres af moderatorer, og hvor en sådan ``brugergruppe'' i stedet for en hård algoritme bare har et manifest, der beskriver, ud fra hvilken protokol vægtene i brugergruppen fordeles, og så er det op til moderatorerne om selv at overholde disse regler. På denne måde kan man opnå alverdens brugergrupper ret nemt og effektivt, og uden at skulle starte med at skrive en hel masse kode!. (Og på denne måde er det nemlig ikke så ``tørt'' med brugergrupper.) Platformen kunne i øvrigt som noget avanceret også åbne op for en blanding, hvor ``brugergrupper'' både kan styres af brugerlavede algoritmer og også af moderatorer. Så det er altså bare det, jeg skal sige! 

%(31.10.21) Okay, så en opsummering (/brainstorm) her ude i kommentarerne over, hvad min opsummeringstekst skal indeholde:
%Jeg skal altså forklare om tags med ratings. Hvorfor er dette smart.. Forklare om tag-dokumentationer.. Nævne helt kort at der er måder, hvorpå man kan vise relevante tags givet tidligere tags, og at jeg har en løsning (men behøver ikke at forklare den).. Jeg skal nævne det, at få en oversigt, så man kan se ting relativt til hinanden.. Nævn at ting kan kategoriseres med tags, og at jeg tror sådan kategorisering vil... nå nej, det giver jo lidt sig selv..
%Så kan jeg gå videre til at forklare, hvorfor ordnede kommentarer er smart, og hvorfor kommentarer også skal have mere en én rating. Måske kan jeg også nævne her, at der er måder, hvorpå man kan få relevanta kommentar-kategorier til at blive vist.. nej, måske ikke, for relevante kommentar-kategorier vil jo blive vist, når de får opmærksomhed.. ..Hm, skal jeg egentligt også nævne navigations-relationer/-links..? Hm, måske.. Jeg kunne også nævne $\subset$-relationen kort (altså som en måde at lave kategorier på), men måske ikke.. ..Hm, men ellers kan det være, at det bliver mere relevant at nævne (helt kort), nedenfor ved b.d. ML (og dets "domæner").. ...Lad mig nævne det her, og så kan det være, at jeg laver et afsnit nedenunder omkring at starte sem-web via en sådan platform med en masse brugere (og altså starte med en brugervenlig udgave; "det handler om at opnå data fra folk, og så kan man altid oversætte til et andet formelt sprog bagefter" (hvis det skal passe til andre systemr)).
%Jeg bør også nævne ønske-vurderinger her..
%
%Jeg kan forklare om brugerdrevet ML; hvordan det kan virke, og bl.a.\ også hvorfor det vil øge vurderingsaktiviteten gevaldigt på platformen (fordi brugere hermed selv faktisk får noget ud af at vurdere ting).. (Husk at folk også selv skal kunne kontruere og offentliggøre deres egne vektorer..) Og ja, det er så her, at jeg også lige kan nævne det forskelle muligheder, som jeg ser det, for "brugergrupper," bl.a.\ især min idé om, at brugergrupper med "bløde algoritmer" kunne være smarte.
%
%Jeg kan så nævne konceptet om, at åbne brugerfladen op for brugerlavede udvidelser, og om så at gøre det til en del af platformen, at der kan være, hvad der svarer til YT-kanaler og/eller som SoMe-profiler, bare hvor skaberen ikke skaber indhold men skaber brugerfladeudvidelser (og hvor folk kan følge, like, og donere). 
%Og så kan jeg nævne brugerdrevede feed-sorterings-algoritmer som en særlig ting, hvor brugerne kan få kæmpe stor gavn, og hvor der (7, 9, 13) automatisk må komme en kæmpe aktivitet --- ikke kun hos særligt engagerede og teknisk formående brugere. Hvis platformen allerede har en god brugerbase, vil denne idé, mener jeg, kunne få siden til at eksplodere; folk vil migrere til platformen fra andre ("feed-platforme") i store mængder. 
%..Uh, og måske jeg lige kunne tilføje nogle ord også om, at platformen jo kan give et sprog til at snakke med dets servere, så brugere hermed har frit spil til at programmere applikationer til platformen. Det er så ikke sikkert, at dette vil føre alverdens ting med sig, men måske, så hvorfor ikke tage chancen og prøve?. 
%
%*(skal nok rykkes ned) Hm, måske skulle jeg indsætte en lille note her om at nå det semantiske web via min platform idé. Måske passer det bedre før bruger-programmerings-noterne, men.. ..Hm ja, på en eller anden måde synes jeg, det passer bedre her. ..Nej, måske kan jeg faktisk gøre dette til sidst, efter debat, og så kan jeg følge op med ontologi-eksempler.
%
%..Hm, kunne jeg også lige sige nogle få ord, om at have mere alsidige wiki'er (basalt set) med prædikater på artikeltitlerne, og om så også at strukturere dem bare formelt og lagdelt?..
%Jeg skal nævne min idé omkring kurser.. Hm, nå ja, denne idé er jo lidt bare en videreudvikling af "wiki-side-idéen," så lad mig lige på et tidspunkt kigge på, hvad jeg helt præcist skal nævne (men dette kan godt vente til efter, jeg har skrevet opsummering over ovenstående punkter)
%..Uh, og jeg kunne måske lige skyde nogle hurtige ord om ratings til kollaborativ programmering ind her..(?) ..Ja, for jeg skal vil alligevel forklare om det, hvis jeg vil forklare kort om kollaborativ redigering..(?)
%
%Jeg skal nævne min debat-platform-idé. ...Nævn at det kan føre til rapporter/kurser og også til "p-modeller" på sigt.
%
%Hvis jeg vil nævne nogle ontologier, kan jeg jo snakke om vejen til det semantiske web her..(?)
%
%Nævn måske nogle ontologier..
%
%Jeg vil ikke forklare om mine civilforeninger her i denne omgang (selvom det er en vildt god idé, tror jeg).


(01.11.21) Jeg har brainstormet en ny disposition ude i kommentarerne og er nu klar til at gå i gang med at arbejde på et udkast (på engelsk), hvilket jeg gør i sektion \ref{web_ideas}.


%(02.11.21) Jeg kan lige nævne her ude i kommentarerne, at rating-folksonomy- og b.d. ML-platformen med fordel også godt på sigt kan medtage visse af folks personlige oplysninger til at indgå i korrelations-objekterne, igen så længe man bare sørger for aldrig at snævre domænet mere ind, end at alle brugere kan være helt anonyme, og at ingen individuel brugers vurderinger kan udregnes på baggrund af de offentliggjorte resultater, og heller ikke bare hvis den kan udregnes med en fejlmargin, der bliver for lille for individet.
%Jeg kan også lige nævne en hurtig idé omkring at frasortere bots og gengangere (og sock puppets), og der er, at man man måske på et tidspunkt kunne gøre det sådan, at folk kunne komme til at bevise, at de har tilgået et produkt, ved at de viser en kvittering for et køb, udlejning, eller adgangsgivende abonnoment til produktet. (Hm, og for streamings-platforme kunne man måske overveje, at platformen kan give en kvittering, hvis man har streamet nok af filmen/serien/episoden, således at brugeren altså ligesom indsamler kvitteringer ved at bruge platformen.) Bare en lille idé; det er ikke sikkert, at den bliver relevant at prøve at udføre. 


(02.11.21) Hov, jeg har lige fundet ud af noget, som faktisk er ret vigtigt at tilføje: Jeg har vist brugt ordet `platform' forkert. Jeg tror det kommer sig af, at min idé jo kommer fra en tidligere idé, hvor det hele ligeså godt kunne (og på et tidspunkt var idéen vist endda, at det burde) køre i en desktop-applikation i stedet for i en browser. (\ldots Ja, og når jeg tænkte på at køre det i en browser, så tænkte jeg endda at forbinde til localhost og ikke til WWW.) Så ja, jeg burde nu altså nok have haft skrevet `hjemmeside'/`website' langt hen af vejen her i mine noter (i hvert fald her fra i sommers og frem).




%Husk:
% - Jeg kan måske også nævne, at man kan udvide systemet så brugere kan bedømme relevans af kommentar-kategorier og navigations-referencer. (ja, husk)
% - Nævn "kvitteringer" her i disse noter (måske bare her ude i kommentarerne..). (tjek)



%Navigations-relationer kan i udgivelsen så bare nævnes kort (og relationer bliver jo alligevel introduceret med kommentarerne). (ok)
%Stadarder for rating(-dokumentationerne). (tjek)
%Html-oversigter og videre WoApps, hvor serverne måske endda kan udgive et programmeringssprog, man kan bruge. ..Og "omni-side".. (tjek)
%Mere opmærksomhed til o.s.-programmører (svarende til min idé om programør-/tiltag-"kanaler").. (tjek)
%Navigation fra andre sider til platformen og mulig extension, hvor man kan hente data fra platformen (og måske også endda bare lige dennes html/css indover).. *Åh, gider jeg nu også, at skrive om dette?.. For bliver det overhovedet særligt relevant..(?).. ...Ah, fuck it, nu har jeg skrevet en hurtig paragraf om det. (så tjek)
%
%Uh, og jeg også overveje, hvordan brugergrupperne kan blive i en mere simpel form, for disse kan (og bør) også simplificeres (om ikke andet så bare til min første udgivelse).. (gjort)
%Hm, og skal jeg overveje feed-sortering (og brugergrupper) mere..?.. (gjort)
%*Uh, jeg skal også skrive nogle ting om nye idéer omkring anonyme vurderinger. (mangler i den renderede tekst) *(tjek)
%*Hm, har jeg egnetligt også en god pointe omkring vigtigheden i at give korrelationsdata med til data, eller hvad..?.. ...Tja, nej.. Pointen er bare, at man kan rigtigt meget, hvis man bare lige kan få serveret ret atomare korrelationsvektorer omkring ratings også (og ikke bare får ét resultat i form af gennemsnit/antal). Og med gode atomare korrelationsvektorer kan man nemlig opnå en masse.. (ok)


%(26.10.21) Brainstorm over brugergrupper: Lad mig bare starte med at nævne, at jeg nu tænker at dele det op i 'brugergrupper,' som betegner et system med moderator-vægte og stemme-vægte, og så 'vægtfordelinger,' som er hvad man får ud af brugergrupperne. Men vægtfordelinger kan så også indebære mere, bl.a.\ kan de inkludere flere brugergruppers (og andre) vægtfordelinger i sig, og de kan også bruge forskrifter for at fordele vægten, muligvis med variable parametre i, som brugerne så selv kan stille på i sidste ende. Et problem med alt dette er bare, at hvis man gør det for åbent, så... Nå nej, jeg har jo nogle nye idéer, hvor brugerne kan sikre sig, at deres anonyme vurderinger (og kommentarer) forbliver anonyme; nemlig ved at uploade dem til en brugergruppe, som så lægger en maske over brugerensstemmevægt (ved enten at lægge en tilfældig værdi til (med fortegn) eller at runde op eller ned til en værdi). På denne måde kan brugeren altså stadig bruge sin stemmeindflydelse hos brugergrupper til at få sine anonyme vurderinger og kommentarer set, men uden at vurderingerne/kommentarerne kan spores tilbage til brugeren på nogen måde (udover hvis platformens servere kompromiteres, men disse kan med fordel også smide data omkring ophavsmændene til anonyme vurderinger/kommentarer væk (og altså slette det)). Ja, og på den måde bliver der total frihed ift.\ at lave vægtfordelingsfunktioner.. Hm, nå ja, og brugere kan også vælge at tilknytte yderligere korralationsparametre til vurderingerne/kommentarerne (og altså tilkendegive lidt om, hvilken type man er --- hvorved man selvfølgelig bare skal sikre sig, at denne undermænge er stor nok i brugergruppen til at uploads'ne forbliver anonyme herved). Og ja, så er der nemlig fri leg mht. vægtfordelingsfunktionerne. 
%Ok. Men ja, nu skal jeg så overveje (igen), hvad mulighederne er (og hvad man kan opnå, uden at det kræver alt for meget).. Hm.. ..Og ting, jeg har i luften, er sådan noget, som at man automatisk kan bruge korrelationsdataen til at justere vægtfordelingen, og at jo kan opdele en brugergruppe, så vægten afhænger af kategorien, og.. er der andet..? Jo, og hvis først man bruger funktioner og/eller variable parametre, så skal jeg også til at overveje, hvordan klient-server-sammenspillet så bliver for at kunne give disse muligheder.. ..Hm, men ellers kan man jo i starten bare nøjes med konstante vægtfordelingsfunktioner, som brugere altså søger om at få godkendt, og hvor serverne så simpelthen bare kan gemme værdierne i databasen.. Hm, og i første udgivelse kan jeg i øvrigt bare nævne, at man også kan gøre noget for at opdele en brugergruppe på forskellige kategorier, men hvor jeg så ikke behøver at komme ind på hvordan (og jeg kan bare nævne, at det nemlig kræver lidt flere ting indført til systemet (navnlig min $\subset$-relation)). ..Okay, så for konstante brugergrupper, der går det fint, og så er spørgsmålet bare, hvad man kan gøre med variable brugergrupper (..med custom-funktioner som hver enkle bruger kan bruge privat, hvis dette behøver være en mulighed..), og med feed-sorteringer..? ..Ah, men for variable brugergrupper, der kan man da bare bruge simple SQL-funktioner til at danne de aggregerede stemmevægte! Så det kan man da sagtens indføre! Cool, og så er vi jo allerede super langt med de få ting.. ..Angående kategori-opdelte brugergrupper, så kan platformen jo bare gemme én værdi for hver ressource (eller rettere hver ressources tag). Så glem altså det med overlappende undergrupper, som jeg ellers snakkede om. Og når det så kommer til.. ah, vent. Hm, kan man ikke bare i stedet for opdelte brugergrupper så bruge kategori-\emph{begrænsede} brugergrupper. Og så kan man jo bare bruge disse i forbindelse med sammensatte brugergrupper, og så får man sine "opdelte brugergrupper"..! Jo, cool! ..Hov, man kan jo også faktisk netop godt lave et helt decorator-system med SQL, så man kan faktisk ret nemt implementere et system, hvor brugerne selv kan konstruere deres (decorator design pattern-agtige) sammensatte brugergrupper *(nej, 'vægtfordelingsfunktioner') og så simpelthen bruge disse til søgninger og til feed-generering.! (For man kan jo også gange resultater sammen i SQL til at få nye resultater.) Og hvis det begynder at gå langsomt, så kan brugere jo bare anmode om at "kompilere" vægtfordelingsfunktioner, hvorved serverne så kan gemme resultaterne i databasen. ..Uh, og så kan man jo også nemt inkludere tiden (timestamps; tiden for uploadet) i funktionerne, og bum! Så har man alle de ting man behøver til at kontruere feed-sorteringer med! Fedt, fedt, fedt. Så kræver det lidt mere arbejde af serverne, at "kompilere" funktioner, der bruger timstamps --- især hvis det er på en indviklet måde, hvor tidsafhængigheden ikke kan faktoreres ud til en simpel funktion --- fordi serverne så ville skulle opdatere værdierne hele tiden, hvis de skulle gemme dem, men ja, så må man jo bare prøve at holde tidsafhængigheden til simple, konventionelle funktioner, så serverne kun skal opdatere disse løbende (og ikke også alle de "kompilerede funktioner," de indgår i). Cool! Og så er der kun lige det med, at kommentar-fanebladenes sorteringer måske kan komme til automatisk at bruge forskellige sorteringsindstillinger og alt det, men det kan jeg jo bare lige nævne kort og sige, at her er der noget, man kan overveje at udvikle på sigt. Nice! :) (Har det lidt sådan: to fluer med ét smæk! :))
%Hm, angående det med kategori-grupper for kommentarerne, så kommer det jo til at fungere på samme måde, som for ressource-kategorierne, og så er der bare lige ellers den mulige tilføjelse/udvidelse, hvor man altså kan stille variable i funktionen under selve fanebladet, imens man er på det, men det hører jo klart til noget af det avancerede, jeg ikke behøver at komme ind på. 





%Husk advarsler og kategori-subtraktion. (tjek)
%Husk mulighed for prædikat-prædikater til at lave specielle overkategorier til det formål at hjælpe med brugergruppe-vægtfordeling-søgeindstillinger.. (hvis man altså ikke bare vil lave disse ud fra normale kategorier (som nemlig også kan implementere "overkategorier" på simpel vis)).. (tjek)

%Nævn i udgivelse, at siden så lige skal --- udover at gøre at formelle relationskommentarer kan ses hos begge ressourcer --- implementere et interface på sigt, så de semantiske algoritmer kan interface med databasens relationer og prædikater og bruge dette til at finde frem til ressourcer med (altså via semantiske queries). (behøves nok ikke i udgivelse alligevel)


**(13.11.21) Jeg er nu også kommet frem til en del ændringer af både idéerne omhandlende kommentarsektionenen (hvor jeg nu i høj grad vil bruge voteringer til at up-down-vote kommentarerne og til at sortere dem i faner), og også af rating-tag-idéerne (hvor jeg nu bl.a.\ har tilføjet nogle nye former for tags, f.eks.\ person-type-tags og en ny form for ønske-tags, og har fundet på en måde, hvorpå brugerne kan forklare deres samlede vurderingsscore af en ressource). Men jeg vil nu ikke beskrive dette i denne sektion. Se i stedet mine noter (bl.a.\ også brainstorm-noter (ude i kommentarerne)) til `Draft to first publication'-sektionen. 



**(06.12.21) Der er nu også kommet en del nye ændringer til. Selvom jeg også gerne vil skrive om nogle mere simple udgaver af idéen, så har jeg også nu et specifikt system, som simpelt nok i sin grundlæggende opbygning og som alle mine andre idéer kan implementeres med. Samtidigt har jeg fået nogle designmæssige idéer, som jeg virkeligt er glad for, bl.a.\ hvordan man skal kunne rate under faner\ldots\ nå nej, jeg har vist heller ingen gang snakket så meget om `faner' på dette punkt endnu\ldots\ Men ja, det må man også læse sig til nedenfor. Men for at forsætte forrige sætning: Man skal kunne rate ressourcer/elementer under faner ved at nærmest at flytte dem op og ned. Men dette og meget mere kan man altså læse sig til nedenfor. Jeg har dog ikke skrevet så meget af det ind som renderet tekst pt., men jeg kan vende tilbage hertil, når jeg kan referere til en rederet sektion (eller flere). I skrivende stund står meget af det faktisk i udkommenterede brainstorm-noter lidt forskellige steder, men mange af mine seneste fremskridt står primært under `Web ideas2'$\to$`Organized comment sections' *(og det kan egentligt også godt være, at det står en del forklarende i 1'eren også (i den tilsvarende undersektion)), så indtil videre kan man jo starte der, hvis man vil læse om dem. Men ja, ellers vender jeg også tilbage hertil med en opdatering, som kan komme til at stå efter denne paragraf.  









\subsection[Helt ny tilgang]{Helt ny tilgang (som ingen gang tager udgangspunkt i tag-ratings længere) (15.12.21)}
Som jeg jeg skrev i den sidste (indskudte) paragraf i sektionen lige her ovenfor. Så er jeg faktisk kommet til en del ændringer siden min `Summary/brainstorm-(2021-)noter ovenfor, og nu har jeg endda lige fundet på nogle helt nye ændringer til systemet igen. Sammenhængen mellem disse systemer og mit ``folksonomy-rating-system'' er stadig, at målet er et system, der kan fremføres som en ny hjemmeside, som brugere hurtigt kan få stor gavn af, og hvor dette system så med tiden vil kunne komme til at indeholde alt det, der skal til for det semantiske web m.m. 

Indtil i forgårs aften gik min løsning ellers på at lave en side, hvor alt var opdelt i faner, og hvor brugere ratede relationer/prædikater omkring elementer(/ressourcer) ved (i høj grad) at flytte dem op og ned i listerne under disse faner. *(Så denne version tager altså mere udgangspunkt i idéen om ``organiserede kommentarer'' (m.m.) end i idéen om tag ratings.) Så fanerne skulle altså allesammen være en slags prædikater i virkeligheden, og jeg tænkte endda, at enhver underfane i bund og grund bare kunne tilføje et prædikat til en samlet konjunktion, der så gav de prædikat som elementerne/ressourcerne sorteredes efter. Men nu er jeg altså ikke længere tilfreds med den idé, for jeg kan nemlig nu se nogle mangler i den. Der var særligt nogle problemer med at kunne navigere hen til de rigtige emner og de rigtige sorterings-/visnings-prædikater samtidigt. .\,.\,Ja, det gik ikke helt som det var, men nu føler jeg altså, at jeg allerede har nærmet mig en ny løsning.

Nu bygger systemet ikke længere på, at man har et træ af faner og underfaner; ikke som en fundamental del af selve opbyggelsen. Nu har man i stedet en mere konstant række faner og/eller (vertikale) split-screen-vinduer (alt efter om de er ekspanderet eller kollapset (horisontalt)), hvor man ser forskellige liste-visninger (af forskellige ting) i hver fane/vindue. (Og ja, man kan i princippet godt nøjes med aldrig at se mere end én liste ad gangen, og man kan sagtens nøjes med maks to i øvrigt til det meste brug (tror jeg).) Hver af disse faner (lad mig bare kalde dem det for nu) har så et prædikat, eller rettere en relation (så det vil jeg også begynde at sige nu i stedet), som definerer, hvad der vises i listen. Disse relationer vil så nemlig lige præcis typisk ikke bare være prædikater, men vil også tage andet input end selve liste-element-input-parameteren, som altid holdes variabel. Og dette input vil så blive sat ud fra, hvilke elementer brugeren har selekteret. Der er to faner, der er særligt vigtige for systemet, foruden en tredje som det hele på en måde i sidste ende handler om (nemlig som skal resultere i den endelige oversigt over elementer/ressourcer, som brugeren er ude efter), og det er for det første en fane, der viser `relaterede emner', og en fane, der viser `relaterede visnings-relationer' (eller `sorterings-relationer' kunne man også sige, men nu kalder jeg dem nok mere `visnings-' fremover). Det er så i høj grad meningen, at brugeren skal finde frem til de emner (/kategorier) og de visnings-relationer, som brugeren ønsker til den endelige fane, via disse to faner. Og dette sker så ved (sandsynligvis ofte lidt på skift) at vælge termer (hm ja, `termer' er nok bedre at bruge generelt end `elementer'.\,.) fra den ene og den anden fane, hvorved disse termer så tilføjes til et ``arbejdsbord'' (eller `workspace') over termer. Jeg vil nok gerne have, at termer bliver typet i det grundlæggende, så der bliver nogle fundamentale typer såsom særligt `emner' og (visning-)`relationer.' Når et nyt emne-term er valgt til arbejdsbordet (hvorved det automatisk sættes som de ``selekterede'' term) eller bliver selekteret fra arbejdsbordet, så skal begge de to faner skifte, så henholdsvis relevante emne-termer og relevante relationer vises i de to faner. Når en relation tilføjes til eller selekteres fra arbejdsbordet, skal relations-fanen ændres, så den viser relevante relationer for den relation (men emne-termet kan så fortsat være selekteret samtidigt også). Hm, lad mig lige tænke lidt over disse selektioner, så, to sekunder.\,. .\,.\,Hm, man kunne jo gøre det ud fra den rækkefølge, termerne er selekteret i, og så kunne man så bare samtidigt have det så at termer kan indsættes mere manuelt i fane-relationerne, hvorved indsættelsen så bliver mere konstant og altså skal fjernes eller udskiftes manuelt igen for at forsvinde derfra. Jep. Så vi snakker altså her om en term-søgning/-browsing, hvor brugeren har to lister, hvorfra denne kan vælge nye termer (til arbejdsbordet), og hvor listerne så efterfølgende altså kan varieres på flere måder, nemlig ved at bruge de valgte termer. Og de to lister handler altså så om at vise ``relaterede termer'' af den ene og den anden slags til de valgte termer. Nu kommer vi så til den tredje fane. Her skal brugen gerne kunne tilføje samlede visnings-relationer via ret simple handlinger, nemlig ved bare at tilføje emner og del-visnings-relationer (for den endelige visnings-relation kan nemlig bygges bl.a.\ fra andre visnings-relationer, som allerede eksisterer i databasen (og som brugeren altså kan finde frem til i sin browsing)). Der skal gerne være to standard måder at føje emner til denne tredje fane, hvilken vi jo kan kalde `ressource-fanen' (ja, det lyder fornuftigt nok, nemlig at kalde de endelige termer, som brugeren er ude efter i alt dette, for `ressource-termer'). I den ene tilføjer man et emne, og så skal der kun vises ressourcer, som brugerbasen har stemt ind under pågældende emne. I den anden standardmåde vælger brugeren en vægt, og så gives ressourcerne point ud fra denne vægt, og ud fra, hvor højt de er vurderet til at passe på pågældende emne. Og for del-visnings-relationerne så er standardmåden bare at blive promptet eventuelt, hvis der skal flere indput til for at lave et prædikat, og så ellers spørge brugeren om en vægt også her, hvorved ressource-termerne så efterfølgende også får point alt efter, hvor højt de er rated for det givne prædikat (hvilket sikkert typisk bare vil være et prædikat uden andet input) ganget med den pågældende vægt. Okay, det var lidt rodet sagt, men jeg håber, at pointen går nogenlunde igennem. Brugerne skal altså med andre ord nemt kunne tilføje disse tre ting til det samlede ressource-(visnings-)prædikat, nemlig prædikater til at filtrere indholdet, så der henholdsvis kun vises ressourcer fra et givent emne, så ressourcer får et antal point ud fra, hvor godt de passer på et emne, og så ressourcer får et antal point efter, hvor godt de passer på et prædikat. (Og antallet af pointene i de to sidste tilfælde kan altså så afhænge af en brugervalgt vægt til lejligheden.) Desuden skal der også være en mere avanceret mulighed til føje ressource-prædikater til ressource-fanen, og dette er via et arbejdsbord, hvor brugeren selv kan sammensætte prædikater (og altså på en lidt mere lav-niveau måde). I starten kunne man her endda simpelthen bare have en syntaks, således at brugerne simpelthen skriver sine egne prædikater (ud fra de valgte byggeklodser fra ``arbejdsbordet''). 

Når brugeren så endeligt får sin ressource-liste, så kan brugeren gøre to overordnede ting. Brugeren kan enten finde en ønsket ressource og vælge at få dennes indhold vist, og ellers bare bevæge sig videre med sit liv, eller brugeren kan også måske sige: ``hm, jeg er måske lidt uenig med denne sortering, så jeg vil lige selv tilføje mine vurderinger af ressourcen.'' Så er det altså her at brugeren netop kan rate ressourcerne. Dette skal så gerne (se mine tidligere noter til de tidligere versioner af idéen under ``organized comments''-undersektionerne i ``web ideas'' 1 og 2-sektionerne nedenfor --- selvom jeg måske udkommenterer disse på et tidspunkt) kunne ske ved at brugeren simpelthen kan flytte op og ned på ressourcerne, hvorved de tildeles point efter deres nye placering. Noget vigtigt i denne forbindelse er så, at brugeren skal kunne vælge givne del-prædikater fra det samlede visnings-prædikat og rate ressourcerne under disse, for som regel vil det nemlig ikke være særligt gavnligt, hvis man rater for det sammensatte prædikat. Ofte vil det være langt bedre at vurdere ressourcerne enkeltvis for hver af de mere atomare prædikater (som man dermed også vil kunne forvente, at andre brugere også rater). Og særligt kan brugerne også herved rate sådanne prædikater, som vi allerede har talt om, nemlig omkring tilhørsforhold til diverse emner, og desuden omkring hvordan og hvor godt emner og relationer relaterer sig til andre emner og relationer. 

Angående det sidstnævnte så er det særligt relationer såsom, ``dette emne-term repræsenterer et over-emne til dette andet emne-term,'' osv.\ osv.\ (altså hvor `over-' kan udskiftes med `under-,' og hvor `emne' kan udskiftet med `relation')%.
, vi skal bruge her.


%Bemærk at typerne mest bare er der for at... (tjek)
%Prædikat-grupper.. (og vertikalt opdelt main-visning for sådanne grupper).. (tjek)

(17.12.21) Okay, jeg vil gerne færdiggøre denne sektion, men nu har jeg brugt i går og i dag på at overveje strukturen af min udgivelse (omkring web-idéer), og hvor jeg også særligt har overvejet, hvad man kan sige helt kort om tingene. Og jeg vil gerne fortsætte med denne mere abstrakte del, og så kan jeg altid vende tilbage til mine implementeringsforslag her. 


(23.12.21) Lad mig lige prøve at færdiggøre denne sektion. Til den ovenstående tekst mangler jeg nok særligt lige at tilføje en idé om, hvordan man også skal kunne have faner, som selv er inddelt i flere faner. Lad os f.eks.\ tage vores ``relaterede emner''-fane, som altså er beregnet til hele tiden at vise emne-forslag til det selekterede term, hvad end dette er et andet `emne', en (visnings-)`relation' eller en `ressource,' eller hvad vi nu har af typer. Men her kan der jo være forskellige kategorier af forslag. Emne-forslag kan jo både have karakter af ``overemner'' til emnet (hvis vi nu siger, at vi har selekteret et emne-term i dette eksempel), ``underemner'' eller ``søskende-emner'' (som hører ind under et delt overemne), eller det kunne også i princippet være i form af emner, der på en vis anden måde har forbindelse til selekterede emne. Så det ville derfor ikke være dumt, hvis dette vindue / denne fane selv kunne opdeles i faner, så brugeren f.eks.\ kan vælge en underfane af ``overemner,'' hvis denne nu specifikt leder efter et vist overemne. Jeg vil derfor foreslå, at brugeren ikke bare kan definere en ``visnings-fane'' ud fra en enkelt relation (eller måske én relation for hver type, hvilket også ville være naturligt), men også ud fra en hel gruppe af relationer, som altså hver især kan resultere i en underfane. 
Så langt så godt, men jeg har så også en yderligere idé i denne forbindelse, og det er, at der skal kunne defineres en slags ``hovedfane'' (main tab) for sådan en gruppe, således at brugeren får vist en horisontalt opdelt liste som standardvisning for en (grupperet) fane, hvor del-listerne viser et lille antal elementer fra en given underfane hver især. Og hvis brugeren så, som i vores eksempel, vil lede specifikt efter f.eks.\ ``overemner,'' og hvis overemne-del-listen ikke indeholder, hvad brugeren søger efter, så kan denne altså klikke på ``overemne''-underfanen i fanegruppen og se den fulde liste der. 

Måske kan man endda have flere niveauer i fanegrupperne, således at underfanerne også i sig selv potentielt set kan bestå af fanegrupper (og så videre). (Så hver fane kan i så fald enten bestå af en enkelt liste med en enkelt visnings-relation, eller den kan være en fanegruppe, som jeg lige har beskrevet, og hvis det samme så også gælder for hver underfane i en gruppe, så har vi jo hermed en ``rekursiv type'' så at sige, og kan altså få arbitrært mange niveauer i fanerne.)

Bemærk i øvrigt, at alle de her typer egentligt bare er for, at der kommer lidt mere orden i det hele for brugernes skyld. Men der behøver altså ikke umiddelbart være nogen forskellige restriktioner.\,. Ah, på den anden side vil man nok gerne indføre dette i praksis: Man vil nok gerne f.eks.\ vedtage for databasen, at emne-termer bare skal være korte tekster, og at relations-termer måske også kun må være semi-korte tekster, der måske også følger et bestemt format. Ja, men min umiddelbare grund for at indføre typer til dette system, er egentligt mere bare, så at visnings-relationerne nemmere kan ordnes, så de hver især kun beregnes på en enkelt type, og at brugeren derfor så bare skal vælge en relation for hver type til en given fane, hvis man gerne vil have, at fanen skal opdateres for alle typer, når man selektere en ny type. (Og alternativt kan man i øvrigt også bare have faner, der ikke opdateres, hvis man selekterer et term af en type, som ikke indgår i fanens domæne (og hvor der altså ikke er en relation tilknyttet fanen, som er defineret for denne type).) 


Okay, så det ver vist mine hængepartier, når det kommer til den tekst, jeg var i gang med. Så hvad skal jeg mere sige til dette emne? Skal jeg lige forklare en lidt mere overordnet version af idéen?\,.\,. 
.\,.\,Tja, giver det ikke sig selv, hvis man kan forstå, hvilket problem jeg ligesom prøver at løse.\,.\,? .\,.\,Tjo, de er sikkert fine nok, som de er, de her noter (selvom de godt nok kunne skrives mere forståeligt, men det går nok.\,.). Jeg kunne dog lige bruge lidt tid på at overveje, hvor vigtigt det egentligt er med sådan et system, hvor visnings-prædikater gøres til sådan en central ting i systemet, men det kan jeg jo altid bare tænke lidt over i den kommende tid. Jeg tror umiddelbart denne sektion er fin, som den er nu.

\ldots\ Nå, nu kom jeg så lige i tanke om nogle andre ting, jeg mangler at skrive om her. Jeg mangler at skrive om, hvordan man på sådan en sige også kan vælge forskellige HTML-udvidelser til at vise ressourcerne med for det første, og faktisk også til at vise listerne med i selve de her faner, jeg har snakket om. .\,.\,Ah, men idéen her var faktisk ret simpel.\,.\,! Den er bare, at man indfører to nye faner til formålene, som dog i princippet kan være kollapset langt det meste af tiden (og for mange brugere måske nærmest hele tiden). For selvom disse to faner i princippet fungerer meget ligesom de andre, så gælder der dog for dem, at den øverste mulighed bare bliver valgt automatisk.\,. Hm.\,. Er dette nu også den bedste idé.\,.\,? I øvrigt så kunne liste(/fane)-HTML-termerne også vælges, ikke bare for ressource-fanen, men for de andre faner også.\,. .\,.\,Hm, jeg kan måske lige tænke lidt mere over det. Men det korte af det lange er ellers bare, at brugerne gerne selv skal kunne vælge HTML- (og/eller JavaScript-) udvidelser, som bestemmer, hvordan ressource-siderne ser ud, når man klikker sig ind på en ressource, eller som bestemmer, hvordan listerne skal renderes i selve fanerne. Og her må det så i øvrigt også gerne kunne blive sådan, at man i praksis kan bruge flere forskellige indstillinger på én gang, men som så hver især træder i kraft alt efter, hvilken kategori (og/eller hvilket ``emne'') af ressourcer, man søger efter / klikker sig ind på.\,. .\,.\,Ja ok, lad det bare blive ved denne overordnede idé indtil videre. 




\newpage

\section{Andre opfølgende noter (24.10.21)} %(--23/12..?)

Andre opfølgende noter til mit store notesæt ovenfor (her fra 2021) i form af idéer og rettelser.


\subsection{Flere idéer til nyttige ontologier *(og mere)}

Jeg har nok brugt ordet `ontologi' lidt forkert ovenfor, fordi jeg har tænkt og brugt det som betegnende en graf af linked data, og altså \emph{med} dets indeholdte data, og ikke bare som sættet af de mulige kanter og prædikater i grafen. Men sidstnævnte er jo også ret essentielt, så min overskrift på denne sektion fungerer stadig med den rigtige brug af ordet.


\subsubsection{Opgave-ontologi}

(24.10.21) I går aftes fik jeg lige en rigtig god idé til at man burde lave en ontologi over opgaver, som folk mener, at vi kan og bør gøre som civilisation --- eller som lokalt samfund; man kan også godt inkludere mindre opgaver, som relaterer sig og/eller bør løses mere lokalt. Men ja, jeg forestiller mig altså, at folk simpelthen uploader beskrivelser af udfordringer og opgaver (bl.a.\ open source programmeringsopgaver), som vi kan løse som et fællesskab. Folk kan så vurdere relevansen af disse, samt hvor realistiske (og gavnlige kontra kostelige) de er, og kan uploade løsningsforslag, beregninger og oversigter, argumenter og modargumenter osv. Jeg tror dette kan blive en kæmpe succes, for jeg tror, der er et hav af opgaver, som vi kan løse relativt let i flok, og som vil komme os meget til gavn ift., hvad det kræver, men som vi bare ikke er klar over eller tænker nok på i fællesskab til, at der bliver gjort noget ved dem. 

Dette er en god, simpel idé, som ikke virker for ``politisk,'' og som jeg sagtens kan udgive i første omgang(e). 

Og nu hvor min folksonomy-idé, som jeg netop nu også simpelthen kalder ``idé til et manualt semantisk web'' (se mine brainstorm-noter ude i kommentarerne til sektion \ref{opfoelg-folksonomy-brain} (fra d.\ 24/10)), også kan implementere alverdens ontologier på en god måde, så kan man altså implementere denne ontologi, samt den/de næste, jeg har tænkt mig at nævne, via min ``rating-folksonomy-platform,'' som jeg har kaldt den hidtil. (Man indfører nemlig bare passende relationer og prædikater (og evt.\ specielle `brugergrupper' til formålet), og så kan man ellers gå i gang med at fylde objekter ind i ontologien.)




\subsubsection{Opførsels-ontologi}

(24.10.21) Her i morges fik jeg også lige den idé, at man kunne lave en ontologi over opførsel. Her kunne folk uploade eksempler (enten med kilder, hvor opførslen er observeret, eller som en tekst, der beskriver et virkeligt eller et tænkt eksempel) på forskellige situationer, hvor folk har opført sig (eller hvor man tænker sig, at folk har opførts sig) på en bestemt måde. Nu kan resten af fællesskabet så vurdere eksemplerne på efter forskellige parametre, og ved så at se på, hvad forskellige brugergrupper vurderer opførelserne til, kan man altså så få overblik over, hvad folk i forskellige grupper mener er rigtigt og forkert og/eller tilforladeligt og ikke tilforladeligt. Folk vil så kunne browse forskellige eksempler på siden og herved få en rigtig god idé om, hvad folk fra deres samfund mener (og også alt efter hvilken gruppe de tilhører) om forskellige opførelser; hvad er okay og hvad er set ned på. En af grundene til, at denne idé måske er mere værd, end den umiddelbart kan lyde, er, at medierne i nutiden, inklusiv særligt de sociale medier, er med til i høj grad at fremme de mere ekstreme synspunkter, således at folk med mere koldt vand i blodet ikke optræder så meget i folk feeds, som dem der raser, og/eller dem der har ret ekstreme holdninger (som folk så reagerer i høj grad på), ift.\ hvor udbredte de mere rolige og afslappede holdninger kan være. Så selvom jeg altså ser dette som en ret lille idé, så tror jeg alligevel, den kan have væsentligt potentiale i sig. 





\subsubsection{I forlængelse af `opgave-ontologi:' Idéudviklings-platform} %Altså en mere kreativ diskussions-(/debat-)platform..
(08.11.21) Det er gået op for mig, at jeg også bør følge op på både debat-platforms-idéen og opgave-ontologi-idéen med en lidt mere kreativ version af en debat-/diskussions-platform. Mine tanker for, hvordan debat-platformen har fungeret, har nemlig langt meget op til konkurrerende diskussion imellem forskellige (\emph{mod}-)parter. Dette tror jeg også, vil fører rigtigt meget konstruktivt med sig, men man bør også lave en alternativ udgave af det, hvor de diskuterende parter mere har samme mål (selvom de godt kan have forskellige og konkurrerende meninger undervejs), hvilket eksempelvis kunne være at udvikle en eller anden løsning til et problem. Jeg vil også fremføre denne idé i den tekst, jeg arbejder på nu, men jeg føler nu alligevel at den er så vigtig, at jeg også lige bør skrive den her for en god ordens skyld.

Idéen kan så selvfølgelig oplagt sættes sammen med min ovenstående idé til en ``opgave-ontologi,'' så man altså får en linked database over diverse opgaver, som menneskeheden kunne arbejde på (både lokale og globale problemer), som brugernetværket kan komme i tanke om, og at man så samtidig også har nogle gode muligheder og procedurer til at diskutere disse problemer og finde løsninger til dem. 

Og hvad skal sådan en procedure så bestå af? Jo, i modsætning til for debat-platformen så behøver vi jo ikke helt på sammen måder at flere parter skiftes ad til at komme med bidrag, og vi behøver ikke dommere på helt samme måde. Hm tja, måske kan dommere faktisk være en meget god ting\ldots\ Så er det bare mere individuelle bedømmelser af folk og deres bidrag, som dommergrupperne kan stå for at give\ldots\ Ja, det er måske ikke en helt dum mulighed. Men man behøver i hvert fald ikke specielt meget at gøre det sådan, at forskellige parter har tur på skift. Man kan godt stadig have runder med bidrag på samme måde, men kan altså bare løsne restriktionerne lidt, så alle hele tiden kan bidrage. Ah, og diskussionen kan også være mere åben for alle, så folk frit kan hoppe på løbende, hvis de finder på en god idé. Men ellers kan det sådan set køre lidt på samme måde. Bedømmelser gives så bare ikke så meget til en hel gruppe på en gang, men mere til enkeltindivider (enten fordi de har opført sig på en god eller dårlig måde, eller fordi de har kommet med et vigtigt bidrag, måske i form af en god idé (eller bare et godt argument)). Man bør også stadig, mener jeg, bruge min konvention om, at der hele tiden arbejdes på en oversigt over status af del-diskussioner, så bidragende ikke fortaber sig i dybderne af en graf, men at de hele tiden ``hives op til overfladen,'' så folk hurtigt kan få et godt overblik, hvad der foregår nede i ``argumentations-dybderne.'' 

Og lad mig lige fremhæve, at udviklingsgrupper i virksomheder m.m.\ jo benytter sig af alle mulige smarte protokoller, såsom SCRUM og `Agile development,' eller hvad disse ting nu går og hedder. Man har altså udviklet en masse gode teknikker til effektive samarbejder. Der halter den videnskabelige verden nok lidt efter. Dette system kører jo meget med artikler, hvilket jo gør at enkeltpersoner eller relativt små grupper skal gøre en masse arbejde i deres ende, for at udvikle helstøbte artikler, som andre vil finde værd at læse. Hvis man nu i stedet kunne gøre processen mere åben og mere bred, så videnskabsfolk bare kan uploade deres resultater og argumenter til en linked database enkeltvis, så at alle resultater og argumenter kan findes, der hvor de er relevante, så kan man have hele diskussionen af alle de forskellige argumenter og resultater i et stort fællesskab af interesserede parter, og så når man får diskuteret sig frem til, hvilke resultater og argumenter holder, så kan man så i samme brede fællesskab effektivt (fordi der så vil være så mange desto flere hænder om det) udarbejde en rapport, der opsummerer og forklarer alle konklusionerne.  

Og ja, denne idé, hvor man altså tweaker min debat-platform-idé en anelse og blander den sammen med min opgave-ontologi-idé (og med dette mener jeg selvfølgelig, at udviklingsdiskussionerne skal tage udgangspunkt i den opgave i (opgave-)ontologi-grafen, som de prøver at finde løsninger til), kan altså muligvis være vejen frem til en sådan realitet. 





\subsection[M.o. Eksistens]{Mere om eksistens og bevidsthed}

(21.11.21) Jeg tænkte lidt over eksistens og særligt over bevidsthed i går aftes. Jeg kom i første omgang lidt frem til, at jeg nok har undervurderet spørgsmålet om, på hvilke måder ``sjæle'' kan koble til hjernebevægelser.. ..Ja, jeg har jo lidt bare tænkt: Nå ja, der kan gives følelser til de fysiske tanke-bevægelser, og så er det forklaret; så er vores oplevelse bestående af en ``sjæl'' koblet til en hjerne, der så oversætter de fysiske følelser til bevidste følelser.. Men så kan man lige netop spørge, hvorfor er vi så så én med vores hjerne.. Hvorfor er alt hvad vi føler lige netop, hvad hjernen.. føler og omvendt.. Hm, nu hvor jeg lidt føler, at jeg har løst ``problemet,'' så er det lidt svært at finde tilbage til, hvad problemet egentligt var\ldots\ ..Tja jo, det var lidt det spørgsmål, jeg kæmpede med i går for en stund.. Og løsningen er så bare at indse, at en ordenlig ``rig'' oplevelse må være en oplevelse, der inkluderer tanker i høj grad; ikke bare sanser. En edderkop kan jo også f.eks.\ have skarpe sanser, men dens oplevelse vil stadig ikke være nær så rig/dyb som vores. Og hvis vi ligefrem har oplevelser bestående af sanser men uden selvbevidsthed, så må det være en ekstremt fattig/flad. Ok, men hvad er forholdet imellem selvbevidste og ikke-selvbevidste oplevelser i multiverset, er det næste spørgsmål så.?. Og der kom jeg så frem til i går aftes, at det faktisk, mener jeg nu, er ret naturligt at tænke mere i ``computer-udregninger'' frem for ``hjernebevægelser,'' som jeg ellers har gjort meget hid til. Med andre ord hvis vi skal forklare\ldots\ eller forstå\ldots\ hvad en oplevelse er\ldots\ Hm, det er lidt svært at forklare mine tanker, så i stedet for at spilde mine morgentimer på det, så tror jeg lige jeg vil summe lidt mere over det og så vende tilbage. Men jeg kan godt lige skrive nu her, at jeg altså nu (indtil videre\ldots) er gået over til at se selve computer-udregningerne som værende hvad ``sjælene'' kobler til og ikke så meget selve hjernes/computerens fysiske bevægelser. Man kunne således transformere hjernen til en anden form for computer, og det ville (nok\ldots) ikke gøre nogen forskel, så længe ``algoritmen'' (og dermed alle del-udregningerne) ikke ændrer form. Så ``sjæle'' kobler ifølge denne hypotese mere naturligt til\ldots\ Hm ja, mon ikke faktisk hypotesen kunne inkludere et abstrakt, platonisk begreb om, hvad en udregning/computer/algoritme er, og at multiverset så i sidste ende giver liv til disse (alverdens) computer-udregninger\ldots(?\ldots) ..Altså så det platoniske begreb i multiverset om, hvad oplevelser indebærer, kunne altså (ret naturligt, tænker jeg nu) ligesom indeholde i sig\ldots\ eller bygge på, kunne man måske også sige\ldots\ et begreb omkring ``algoritmer''/``udregninger''/``processeringer''\ldots\ Og når spørgsmålet så ligesom skal besvares omkring, ``hvad findes / hvad er?'' så må multiverset pr.\ dets krav om symmetri altså være nødt til at svare: ``alle udregninger/processer\ldots\ Ok, jeg vender tilbage til dette, og prøver at forklare det bedre (og vil også tænke mere over det).  

(22.11.21) Okay, så jeg gjorde altså nogle tanker i går. Jeg tror som nævnt ikke nødvendigvis længere, at sjæle kobler så meget til fysiske bevægelser; det er nok mere naturligt for dem at koble til konceptet om en maskine.. Selvfølgelig skal man så definere, hvad en maskine kan være, men dette er langt nemmere, end at gå den anden vej og på en måde ``forudsige,'' hvordan hjerne-maskiner kan operere og så (tilfældigvis) have en sjæletype til stede i universet, der kan koble til disse bevægelser. Jeg tror altså nu, at der også derfor skal meget mere information til at beskrive sidstnævnte. Så dermed får vi faktisk lidt automatisk det ``besjælingsprincip,'' der jo muligvis altid kan redde os fra 100 \%-kaos-resultatet. Pointen med det begreb er jo netop, at det kunne være mere naturligt for multiverset at puste liv i hjerner ved at ``fortolke'' (om man vil) deres funktioner som bevidstheder. Og det tror jeg altså på nu; at dette er den mest naturlige ting at gøre, hvis man vil beskrive, hvad en bevidst oplevelse kan være, nemlig ved at oplevelsen ud fra en abstrakt maskine, og så derefter definere en sådan maskine ud fra nogle sætninger, som så kan blive de fysiske love. I denne forbindelse er jeg så begyndt ikke at hælde helt så meget nødvendigvis til, hvad der før nu var min yndlingshypotese, fordi jeg ikke længere er så sikker på, at der kun er én bevidsthed pr.\ univers ad gangen. For hvis man kun har én bevidsthed, så skal man jo dele ``maskinen'' (som det hele så drejer sig om) op i to dele, nemlig en bevidsthedsdel og en ekstern del til at give input til førstnævnte. Men hvorfor ikke bare have, at hele universet er ``maskinen'' og at bevidsthed opstår i de lommer, hvor det giver mening (altså i lommer, hvor algoritmen bliver meget registrerende over for omgivelserne og endda måske bliver selvbevidst)?\ldots\ Hm, det er nu lige før, at jeg stadig faktisk hælder mest til den gamle hypotese om, at maskinen på forhånd skal have defineret (eller i virkeligheden nok snarer være defineret ud fra), hvordan bevidsthed opstår i den. Så ja, jeg hælder altså nok stadig mest til en hypotese, der adskiller den bevidste del fra omgivelserne (hvilket jeg også lidt var min konklusion til sidst i går). Men det andet giver stadig klart mere mening for mig nu, så jeg vil ikke længere forslå min yndlingshypotese, som noget der kunne være eksklusivt sandt i multiverset. \ldots Hm, det kunne det godt; jeg er bestemt stadig åben over for tanken om, at der kunne være et fundamentalt princip om, at universer altid beskriver netop én bevidst (eller knap så ``bevidst'') oplevelse. Men nu er jeg altså endnu mere åben over for, at begge ting kan lade sig gøre, og hvis begge ting kan lade sig gøre, så vil begge ting være sandt i et vidst omfang et eller andet sted i det samlede multivers (pr.\ dets symmetri). I så fald vil der altså også være universer, hvor hele universet kan betragtes som en hjerne, og hvor selv-bevidste oplevelser opstår i lommer, på en måde lidt som en slags hvirvelstrømme i en samlet strøm. 

Så nu har jeg altså to grunde til ikke at bekymre mig længere over 100 \% kaos-situationen; for det første fordi det nok er mere naturligt at ordne universernes love efter beslutnings-automatoner frem for efter leksikal orden, hvis man endelig skal ordne dem, og fordi det er langt sværere at definere en maskine ud fra tilfældige bevægelser, frem for hvis man kan bruge fysiske lovmæssigheder til at beskrive, hvordan maskinen virker. Så selvom et univers, hvor alt bare er kaos, sikkert er nemmere at beskrive, så bliver det dog sværere at beskrive, hvordan disse kaotiske bevægelser bliver til bevidste oplevelser, hvis man ikke har nogen fysiske lovmæssigheder at besjæle som en konkret maskine. 

Så dette er altså alt sammen fint, og jeg er selv tilfreds for nu med disse tanker. Jeg kan vidst godt læne mig lidt tilbage nu og ikke tænke så meget mere over dette emne (og måske først tage det op igen, når jeg bliver gammel. ;) (Det var, hvad sagde til mig selv førhen (før sommeren '19), da jeg slet ikke kunne få bevidsthedsspørgsmålet til at give mening på nogen måde og altså nærmest gav helt op.)). Men jeg er dog kommet frem til, at jeg ikke vil udgive disse tanker med det første. Jeg føler nemlig ikke rigtigt, at jeg har noget konkret nok at sige om det --- det bliver i hvert fald lidt svært at udtrykke det, så der bliver en klar konkret pointe med det. Og nu hvor jeg er begyndt at overveje, ikke at tage det med (jeg arbejder jo pt.\ på et notesæt, som jeg vil ``udgive'' i en vis forstand), så synes jeg faktisk rigtigt godt om den tanke. Jeg sørger jo for, at disse noter også bliver gemt for eftertiden selv i værste tilfælde, så hvorfor bekymre mig om, at udgive noget nu --- især når nu jeg føler, at det bliver mere diffust?\ldots\ Jo, det er stadig værd at påpege idéen om, at eksistens kan centrere sig om selve \emph{oplevelser} frem for \emph{fysisk eksistens} (forfra der så kan \emph{opstå} oplevelser), og det er værd at snakke om de to muligheder med, at universer så enten kan deles op i en bevidsthedscentral og ekternt input, eller at universet bare kan være én samlet maskine til at genere oplevelser (som så løbende genereres i lommer nærmest), men tja, der er sikkert andre, der har tænkt lidt de samme tanker før mig alligevel, og uanset hvad, så er det også lidt diffust (og svært at forklare om; det er jo meget baseret på, hvad jeg \emph{føler} giver mening --- ikke at der ikke er rationelle argumenter i det, men de argumenter hviler ligesom stadig på et skellet af følelser\ldots). Det kunne også sikkert være interessant for nogen, at høre om og forstå mine ``100 \%-kaos''-tanker, men dette er også en mindre vigtig ting. Og sidst men ikke mindst er der min pointe om, at multiverset er uendeligt, og at alle liv/oplevelser vil eksistere et eller andet sted. Denne indsigt gør for det første dødsangst rimelig meningsløs; jo, frygt for døden er forståelig og giver bestemt mening, men angst for selve tanken om at være død giver ikke rigtigt mening når tilværelsen er så uendelig. Vi vil leve vores liv igen på alle mulige måder (og altså med alle mulige variantion, man kan tænke sig) --- selv i vores eget univers, hvis dette bare er tilstrækkeligt stort, hvad det sandsynligvis er (universets forventningsværdi vil nemlig være uendelig, når det kommer til størrelsen af det (for pr.\ symmetrien i det samlede multivers må alle mulige størrelser eksistere)). *(Og selv hvis universet virkeligt vil blive ved med at udvide sig, og selv hvis udvidelsen virkeligt accelerere, så gælder dette stadigvæk: Hvis størrelsen er nærmest uendelig, så vil den effektive levetid af universet også blive nærmest uendelig, for der vil så altid findes lommer, hvor en højere massedensitet (enten som var der til at starte med, eller som opstår tilfældigt) vil forsage, hvad der effektivt svarer til en ``yngre'' lomme af universet. Og alt dette er i øvrigt kun nødvendigt at betragte, hvis vi antager, at der er en global tid, hvor alting i hele universet sker samtidigt. Jeg hælder selv mere til, at universet har et origo, hvor både rum og tid starter, og hvorfra universet så dannes i kanten af en lyskegle, der bevæger sig ud fra dette origo, og hvor universet dermed bliver yngre og yngre, jo mere man bevæger sig væk fra origo. Ikke at denne antagelse er nødvendig, men den er dog god til at forklare, hvorfor vi ser et univers fuldt af lys og energimæssigt liv, når vi kigger ud, og ikke bare ser, at vi ligger i en lomme\ldots\ Hm, nej for hvem siger, at små lommer så vil være meget mere sandsynligt end store lommer; variationerne kan jo netop godt have med start-energidensiteten at gøre\ldots\ Nå, alt dette er også lige meget. Der er et utal af muligheder for at nå frem til, at selv hvis vi kun kigger på sandsynlighederne i vores eget univers, så kan sandsynligheden for, at universet ser ``ungt'' ud, når vi kigger ud på det, selvom dets levetid er uendeligt, sagtens være stor. Gode forklaringer her inkluderer, at universet altså har et origo i rum og tid som beskrevet, eller at udvidelsen enten vil vende på et tidspunkt (så universet altså hele tiden dør og genfødes), eller man kan tage fat i min hypotese om, at det kendte univers bare kommer fra en faktisk eksplosion (og stof og ikke bare af rum). Okay, dette blev en ret lang indskudt bemærkning, men det er nu meget dejligt at tænke på, at vores eget univers med god sandsynlighed også kan være uendeligt (nærmest), ikke bare i rum men også i tid.) Så selv i vores eget univers, vil vi sandsynligvis leve uendeligt (nærmest) mange gange og med uendeligt mange variationer. Det interessante er så også, at, selvom dette virker lidt langt ude i første omgang, at disse variationer også vil inkludere alle mulige oplevelser \emph{imellem} to forskellige liv. Selvom det virker lidt langt ude først, så \emph{vil} der være en kontinuert overgang mellem dit liv (og altså alles) og mit (og altså alles). Så grænserne imellem os viskes altså også ud praktisk set. Så hvis vi nu f.eks. fulgte en række af liv startende med vores eget, hvor der bare sker en lille variation hver gang, jamen så vil vi på ét eller andet tidspunkt (selv vis vi udtager næste liv i rækken tilfældigt) ``tunnelere'' (for at bruge et begreb fra kvantemekanikken) over til selv liv, der er markant anderledes end vores, og disse ``tunneleringer'' vil så blive ved med at ske, så vi på et tidspunkt, hvis vi følger rækken langt nok, vil have levet alle de liv vi kan forestille os, inklusiv liv, der er nærmest fuldstændigt tilsvarende til de liv, som de personer, vi kender (i vores nuværende liv), lever. Så med denne anskuelse, så kan vi altså sige at ``what goes around comes around'' i en meget bogstavelig forstand. 

Er denne samlede tilværelse i multiverset så mere lykkelig, end den er smertelig? Ja, det tror jeg helt klart på. Jeg har flere gode argumenter for dette, og jeg vil også sikkert kunne finde på endnu flere, men de fleste vil nu nok nå til samme konklusion umiddelbart. Så jeg vil ikke bruge tid på at argumentere her, for så er det nok klogere at begynde diskussionen, når jeg rent faktisk møder nogen, der kan tro det modsatte (og som så har argumenter for dette).

Så ja, jeg tror altså bare, jeg vil lade emnet blive ved dette for nu, og altså heller ikke prøve at udgive noget om det her med det første. :) 

\ldots\ Ah, og uanset hvad er det forresten stadig dejligt, at man i det mindste ikke \emph{behøver} en antagelse om, at bevidsthed kan \emph{opstå} ud fra andre eksisterende entiteter (og altså ud fra ``fysisk'' materiale). For bevidsthed kan altså godt bare være roden til selve eksistensbegrebet, og altså bare være selve hvad det \emph{vil sige} at eksistere. 

\ldots Nå ja, og en sidste ting, jeg lige vil påpege: Jeg har tit tænkt meget i at besvare, ``hvad kan eksistere?'' som det helt fundamentale spørgsmål. Og pointen er så, at når man så følger det fundamentale spørgsmål efter med et spørgsmål om, ``hvad eksisterer så faktisk,'' så må dette pr.\ kravet om symmetri altid skulle besvares med ``det hele.'' Men det er dog ikke givet, at ``hvad kan eksistere?'' virkeligt er det fun\ldots\ nå nej, så skal jeg faktisk formulere det lidt anderledes\ldots\ Hm\ldots\ Hm, jeg ville egentligt sige noget om, at ``hvad kan være \emph{sandt} for en realitet?'' også kunne være det fundamentale spørgsmål, men er det ikke det samme\ldots? Hm, ``hvilket sæt af sandheder kan udgøre en realitet?'' kontra ``hvad kan eksistere?'' hm, det er da vel det samme\ldots? Ja\ldots\ Jeg tror dette kommer sig af, at jeg jo i min ene (yndlings eller næsten-yndlings) hypotese antager, at ``eksistens $=$ oplevelser'' allerede er givet som en fundamental sandhed, men ja, hvis ikke man kan antage dette, og at man i stedet antager, at det både kan være sandt og falskt for multiverset, så må man jo ``gå tilbage til'' en mere abstrakt hypotese, hvor ``eksistens $=$ oplevelser'' bare indgår som én af mulighederne\ldots\ Ja, ok. %(Og jeg beholder bare denne paragraf i det ikke-udkommenterede.. ..For selvom jeg på en måde ikke fik sagt noget som sådan med paragrafen, så fik jeg det alligevel: Jeg \emph{fik} udtrykt, hvad jeg egentligt gerne ville påpege, nemlig at ``eksistens $=$ oplevelser'' godt bare kan være ét område af det overordnede multivers, og jeg fik også lige nævnt formuleringen om, ``hvilket sæt af sandheder kan udgøre en realitet,'' hvilket jeg nemlig også gerne ville.)

(24.11.21) Lige en hurtig bemærkning: Man kan forresten se det lidt som en mellemting imellem de to hypoteser, om at et univers udspringer af en (og kun én) oplevelse og om modsat at oplevelser bare opstår, når hjerner avancerede nok forekommer, hvis man tager en hypotese, hvor en overordnet ``oplevelse'' ligesom står for at udregne og sanse hele universet, og hvor denne overordnede oplevelse så indimellem sanser vores hjerner på sin ``rute'' igennem udregningerne. Man kunne således forestille sig, at alle systemer store som små har en lille grad af besjæling (hvor overoplevelsen får en følelse forbundet med, at de fysiske legemer/entiteter gerne ``vil'' bevæge sig et vist sted hen (som resultat af de fysiske kræfter --- men hvor der nu altså kommer en følelse og en slags ``vilje'' ind i billedet)), og hvor disse følelser for så avancerede systemer, som vores hjerner er, så ligefrem kan resultere i (selv-)bevidste oplevelser. I denne hypotese kan kontinuiteten af vores egen oplevelse så på en måde lidt være en illusion: Hver gang overbevidstheden kommer forbi vores hjerne i udregningerne og ``besjæler'' denne, så dannes et enkelt øjeblik i vores egen bevidste oplevelse, og at overbevidstheden, som vi er en del af, så måske når at opleve et utal af andre bevidste oplevelser, inden den igen returnere til det næste oplevelses-øjeblik i vores liv, det mærker vi så ikke selv. Vores hjerne vil hele tiden selv føle, at øjeblikkene sker direkte efter hinanden. Så ja, dette var bare lige for at nævne en måde, hvor man godt kan antage ``eksistens $=$ oplevelser,'' men hvor en oplevelse så alligevel godt kan udgøre en samling af bevidste oplevelser på tværs af universet, der sker samtidigt (selvom der altså stadig i princippet kun er én oplevelse på spil som roden til universet).

%..Jeg kan faktisk ret godt lide denne hypotese. Jeg ved ikke, om den ligefrem slår den hypotese, hvor der bare er én bevidsthed / én oplevelse pr. univers (og hvor der altså ikke rigtigt er sideløbende oplevelser), men den er i hvert fald mindst tæt på at være oppe på siden af den (og måske mere endda..) i mine øjne..  


%(24.11.21) Lige for at forklare det med et univers-origo lidt nærmere, kan jeg lige her ude i kommentarerne forklare, at det svarer til, at hvis man står i universets origo, så vil det univers man kigger ud på faktisk svare til den egentlige tid, hvor ting sker. Så den egentlige samtidighed vil altså være, hvad man se fra det punkt, så når man f.eks. kigger ud til grænsen, hvor nye stjerner fødes, fra det punkt, jamen så de stjerner faktisk fødes til den selvsamme tid, som man ser det ske derfra. Dette kan alt sammen lade sig gøre, fordi relativitetsteorien giver os en øvre grænse for, hvor hurtigt information (og kræfter/påvirkelser) kan rejse; så kan en sådan ujævn samtidighed (eller udgaver med lyskegler, der er fladere, end hvad grænsen tillader som den maksimale spidshed) godt sagtens lade sig gøre.




(27.12.21) Okay, jeg tror faktisk lige, jeg vil prøve at opsummere mine eksistens-tanker, måske endda lidt bedre, og måske har jeg også nogle ting at tilføje.

Hvis vi starter udefra, så tror jeg altså på, at der må være en overordnet perfekt symmetri i det samlede multivers (i.e.\ den samlede eksistens). Hvad ``perfekt symmetri'' så vil sige, det kommer jo an på, hvordan logikken bag eksistens helt grundlæggende er skruet sammen. Der må jo være logisk fundament bag det hele, der kan rumme alle sandheder, og alle udsagn i det hele taget, der kan siges om, hvad der eksisterer, eller hvad der kan eksistere. Og jeg tror altså på, at alt hvad der kan eksistere i henhold til denne fundamentale logik om alt, det må eksistere. Og jeg tror at denne ``lov,'' så at sige, så er selve kilden til, at ting eksisterer (frem for ikke at gøre det). I øvrigt kunne man spørge, hvorfor skal \emph{alt} eksistere; hvorfor kunne det ikke lige ligeså godt være \emph{intet}? Fordi, tror jeg, `intet' her ikke er på lige fod med `alt:' `Intet' er bare en undermængde af `alt' efter min mening. Jeg tror nemlig, at hvert \emph{univers} repræsenterer én mulighed af noget, der kan eksisterer, og `intet'-muligheden vil så bare føre til ét univers ud af alle mulige, hvor intet altså eksisterer. Vi kan også sige at `eksistere,' når begrebet beskrives og bruges af den fundamentale logik, altså ikke beskriver noget om den fundamentale logik selv. Vi bør, mener jeg, altså se `den fundamentale logik' i sit eget (højeste) lag over alt andet. Og når den fundamentale logik taler om at `eksistere,' så handler det altså om, hvilke universer kan forekomme. Så hvis man i den fundamentale logik formulerer, ``der eksisterer intet,'' så vil dette altså handle om den samlede eksistens, men bare om ét muligt univers. Det var vist lidt rodet sagt, men meningen bør nok kunne forstås efter et par genlæsninger. 

Det næste spørgsmål bliver så, hvad formulerer de sætninger i den fundamentale logik, der får universer til at eksistere? \ldots Hm, eller rettere kunne man måske sige: Hvad vil det så overhovedet sige at ``eksistere'' i første omgang (ifølge den fundamentale logik). Her ser jeg følgende overordnede muligheder. Der kunne gælde, at eksistens helt fundamentalt set snakker om fysiske objekter og om rum. Her vil ethvert univers så i bund og grund være en samling matematiske love og start-/randbetingelser, som samlet set udgør, hvad eksisterer i det univers. Jeg tror ikke selv på denne mulighed, for jeg er overbevist om, at bevidsthed kræver sin egen behandling, som ikke kan ``beskrives matematisk,'' med hvilket jeg altså nærmere bestemt (og denne opklaring er vigtig) mener: kan beskrives med SOL (i.e.\ anden-ordenslogik). Jeg tror altså ikke at bevidsthed kan fremkomme af i universer, der er beskrevet med noget så simpelt som SOL (i hvilket alt, hvad vi forbinder med `matematik' normalt, kan beskrives). *(Nå jo, jeg bør dog lige nævne, hvis vi skal være fair mod denne hypotese, at spørgsmålet om, hvordan bevidste oplevelser kan opstå ud fra fysiske objekter(s bevægelse), bare kunne være et fundamentalt princip helt indgroet i den fundamentale logik om alt. Så ville denne hypotese alligevel kunne give mening. Men jeg vil dog sige, at idéen om, at der skulle være et helt fundamentalt princip om dette objekt-til-bevidstheds-fænomen (som altså går igen ud igennem hele den samlede eksistens), langt fra virker sandsynlig i mine øjne.) 
Men der er også andre muligheder. Vi kunne også, og denne hypotese tiltaler mig ret meget, have at begrebet `eksistere' simpelthen i bund og grund retter sig mod, hvad der kan opleves. Med andre ord er hypotesen altså, at oplevelser (bevidste, om man vil) er selve det, eksistens handler om; de er selve, hvad \emph{kan} eksistere. I denne hypotese vil ethvert univers altså beskrive en oplevelse, og alle de såkaldte \emph{fysiske} love for et univers vil så i virkeligheden være indeholdt i formuleringen af, hvad universets \emph{oplevelse} indebærer. Universer af denne karakter vil så i første omgang skulle beskrive i deres love, hvordan oplevelsen forbinder til de ``fysiske'' objekter, og vil derefter (eller forinden for den sags skyld) skulle beskrive, hvordan de fysiske objekter så bevæger sig interagerer, og hvordan og hvorhenne de alle starter. 

%Der er også endnu en mulighed, som også kan give mening for mig, og det er, at begrebet om `eksistens' også simpelthen indebærer spørgsmål oppe på det plan, hvor vi f.eks. kan spørge, om universer kan starte med at beskrive objekter, som automatisk kan føre til bevidsthed, eller om de...

Der er også endnu en mulighed, som også kan give mening for mig, og det er, at der i den fundamentale logik simpelthen er flere muligheder for, om universer starter med at beskrive en oplevelse eller de fysiske objekter osv. Her kan der så i øvrigt også f.eks.\ være flere muligheder for, hvad en ``oplevelse'' indebærer. Taler vi om én enkelt oplevelse i hvert univers, eller kan universer godt indeholde flere oplevelser, der opleves ``samtidigt,'' og kan ``én oplevelse'' godt være mere end en enkelt jeg-oplevelse, så at sige, eller skal der være én følelse af et `jeg' pr.\ oplevelse. Jeg tror på, at det godt kan være, at der gælder et entydigt svar på hver af disse spørgsmål i den fundamentale logik, men ellers kan det altså også bare være, at flere forskellige ting kan være muligt (og så skal de så alle eksistere pr.\ den ``perfekte symmetri'' i al eksistens). 

Som jeg ser det, kan `eksistens' altså være fuldstændigt forbundet med konceptet om en `oplevelse' på et helt fundamentalt niveau, eller også er den fundamentale logik om alt bare i stand til at formulere, hvad konceptet `oplevelse' indebærer, hvorved dette fænomen så vil indgå i en undermængde af alle eksisterende universer. 


Det var ligesom det helt grundlæggende. Så kunne vi snakke om `tid,' vi kunne snakke om `sandsynlighed'\ldots\ eller vi kunne snakke om udformningen af den fundamentale logik: Er det et sprog, som vi kender dem, med grammatik og det hele, eller hvad er det lige? Ja, lad mig starte der. Der er egentligt ikke så meget at sige, for vi kan nemlig nok aldrig komme til at svare på dette. Men lad mig lige uddybe spørgsmålet. Jeg har allerede udelukket SOL som det fundamentale sprog, for jeg tror ikke på, at dette kan beskrive konceptet om oplevelser, men derfor kunne man nu stadig godt forestille sig et sprog på samme måde som SOL, som så bare er mere ekspressivt (som i: man kan udtrykke mere i det). Og så kunne man dermed videre forestille sig, at vi kunne opstille alle universer i en rækkefølge af alle well-formed sætninger i det sprog, hvor der ikke findes selvmodsigelser. Denne rækkefølge, kunne så i øvrigt give os vores `sandsynligheder' i universerne (ved at antage, at vores univers dermed er ``et tilfældigt nummer'' i denne (dog uendelige) rækkefølge), men det kan jeg vende tilbage til. Der er dog også en masse andre muligheder, man kan tænke sig til. Den fundamentale logik kunne også bestå af en mængde af udsagn eller spørgsmål med en vis ordning i, sådan at man måske starter med ét spørgsmål, alt efter om svaret er ja eller nej, går man videre til et nyt spørgsmål, der så aldrig er i modstrid med de tidligere svar. Vi kan med andre ord forestille os, hvad der svarer til et flow chart over spørgsmål/udsagn (og hvor begge svar på hvert spørgsmål altid vil være logisk konsistent med de tidligere svar). Og man kan sikkert også forestille sig meget mere. Dette er bare de to idéer, jeg har overvejet. Tillige kan det fundamentale sprog selvfølgelig også være en forening af flere forskellige af sådanne sprog. Således kunne nogle universer i eksistensen være ``udvalgt'' fra en rækkefølge af sætninger fra et grammatisk sprog, og nogle kunne være ``udvalgt'' i et flow chart-sprog osv. Vi kan ikke a priori udelukke nogen af disse muligheder for den fundamentale logik. 


Og så kan vi komme til spørgsmålet om `sandsynlighed.' Nu har jeg nærmest introduceret det lidt allerede. Det går på, hvordan man så i princippet skal regne prior-sandsynligheden for sit univers, hvilket i bund og grund svarer til at overveje, hvordan universer ligesom ``udtrækkes'' fra den samlede mængde universer, når de ``bringes ind i eksistensen.'' Dette skal dog kun ses som en metafor, for selvfølgelig ``udtrækkes'' de ikke af nogen mekanisme, for hvis vi har en mekanisme, så er vi allerede \emph{inde} i et univers pr.\ mine definitioner hidtil. Og jeg tror i øvrigt heller ikke på, at der findes nogen global tid i det samlede multivers, så universerne ``bringes'' ikke ``ind i eksistensen,'' de er der allerede, har altid været der, og er der altid lige meget og på samme måde; tid findes ikke i multiverset. Men denne sidste påstand er selvfølgelig bare min egen overbevisning; man kunne godt argumentere for det modsatte også. Jeg kan vende lidt tilbage til `tid' igen senere\ldots\ Nå nej, jeg bliver faktisk nok nødt til at forklare om det nu, inden vi går videre med `sandsynlighed.' Jeg tror altså, at begrebet om tid er et indre koncept i den fundamentale logik, og at tid dermed kun giver mening inden for de mulige eksistenser. For universer, hvor `oplevelsen' er selve, hvad eksisterer i universet, hvad end dette gælder for alle universer eller bare en delmængde, så er spørgsmålet ret ligetil. Hvis en `oplevelse' eksisterer som noget helt centralt for et univers, jamen så er tiden i dette univers bare den tid, som `oplevelsen' oplever. \ldots


(28.12.21) Ah, nu har jeg faktisk tænkt nogen nye tanker, som virker til at være ret vigtige.\,.\,! Jeg begynder hermed måske faktisk at hælde meget mod en opfattelse, der svarer til at alt sker hele tiden, og hvorved det så også giver god mening, hvis oplevelser så bare er diskrete.\,. Hm, tja.\,. Okay, lad mig lige se på det hele. 

Jeg tror, det måske kan være rigtigt givende at dykke ned i spørgsmålet om global tid. Og som jeg så kom lidt frem til her til morgen, så må man næsten kunne sige, at enten er den fundamentale logik statisk / uden tid, og så må enhver model, der beskriver multiverset, være nødt til at være uændret under tidsforskydninger, eller også må.\,. Ah, den går måske ikke helt så let. Den anden mulighed, jeg lige ville skrive, og som jeg skriver nu, er, at den fundamentale logik kan ``udregne sig selv'' mere eller mere. Eller man kunne også tænke sig en overordnet (``perfekt'') form for ``intelligens'' (nærmest) --- som man næsten kunne tænke på som en slags fundamental gud, men altså en meget abstrakt gud --- som så udregner alt, hvad kan forekomme i den fundamentale logik, og hvor disse udregninger så medfører, at guden også giver liv til de oplevelser, der udregnes. Hermed vil der så på en måde kunne være en global tid, ikke fordi den ``perfekte'' gud (hvis vi bliver i dette billede af denne (``den anden'') mulighed) bruger endelig tid på at udregne ting, men fordi der så alligevel bliver en rækkefølge i alle eksistenser. Og hvis vi tænker os at det er en ``intelligens,'' der så udregner det hele, så kunne vi tænke os, at tid ligesom opstår som en subjektiv følelse, når intelligensen/guden går igennem alle udregningerne. Ok. Men der er nok også en tredje mulighed, og det er jo lidt det, jeg har hældt til før, nemlig at multiverset har en start (og at logikken ikke bare er statisk) ligesom for vores ``anden mulighed'' her, men hvor alle oplevelser så bare starter i det punkt, så at sige, og leves som en kontinuer ting. Denne tredje mulighed svarer så enten til, at ``guden'' ikke behøver at udregne ting i rækkefølge, men.\,. Hm, tja. Det er måske forkert, men det får mig til at tænke på, at man også godt kan bringe billede om en (abstrakt) ``gud'' ind i første mulighed, hvis bare guden så regner alle udregninger på én gang, hele tiden. Ok, men lad mig så lige tænke lidt.\,. .\,.\,Okay, måske kan den tredje mulighed lige akkurat godt give mening. Lad mig lige starte med at understrege, at denne snak om en ``gud'' kun skal ses som et billede. Jeg antager stadig, at den fundamentale logik er startpunktet for det hele, så hvis man forestiller sig en ``intelligens,'' der udregner (og føler/oplever) det hele, så er denne altså bare en del af --- eller rettere en personificering af --- den fundamentale logik i virkeligheden. Okay, men for at vende tilbage til den ``tredje mulighed'' her så mener jeg nu nok kun, at den samlede eksistens i bund og grund kan være ikke-statisk, hvis den ligesom ``udregnes,'' på den måde at logiske slutninger ligesom kan følge bagefter i forlængelse af andre logiske slutninger. Men fordi en hel oplevelse, som jeg-personen oplever som noget langt og kontinuert, i princippet kunne foregå i et enkelt tidsligt punkt i det samlede multivers.\,. Ja, så kunne man godt have ret meget det samme billede som i den første mulighed, men bare hvor de diskrete oplevelser nu kan have forskellige subjektive længder og kan strække sig over længere subjektive perioder. Dette vil så næsten være det samme billede som for første mulighed, bortset fra at oplevelses-øjeblikke, som er en del af lange oplevelser, så alt andet end lige nok vil tælle mindre end for kortere oplevelser, når det kommer til prior-sandsynligheds-udregninger.\,. Pointen er, at så skal man jo nok alt andet end lige udregne sin prior ved at regne med, at ens univers(-oplevelse) er ``udtrykket'' mellem oplevelser af forskellige længder. Dette er i stedet for, at hver lille oplevelses-frame bliver ``udtrukket'' mellem alle mulige frames. .\,.\,Hm, men måske kommer dette faktisk ikke til at betyde så meget for prior-udregningen, når den gennemsnitlige informationsmængde til at beskrive en oplevelse (alt andet end lige.\,.) vil være uendelig.\,. (Indskudt: Den anden mulighed kunne i øvrigt hurtigt blive lig hver af de to andre, hvis man bare antager, at en oplevelse vil blive ved med at opleves, når først den er ``udregnet'' af den fundamentale logik, således at eksistens ikke kun sker på randen af, hvor logikken er nået i sine udregninger (af sig selv), men også hele tiden sker i hele volumenet af udregninger.\,.) .\,.\,Det ville så i øvrigt være sejt, hvis man en dag kunne argumentere for, at de to prior-udregninger konvergerer mod det samme, men det ville så nok være i et langt senere kapitel.\,. .\,.\,Ah, i forhold til den indskudte bemærkning, så kunne man i øvrigt også forestille sig, at der er en naturlig rækkefølge i, hvilke sætninger følger efter andre sætninger, ligesom at visse sætninger som regel vil udledes før andre sætninger, hvis man skal udlede samtlige sætninger løbende i en formel logik, men at den fundamentale logik dog godt ``kan finde på'' at gen-udregne sætninger, også selvom den også kunne have brugt i et ``bevis,'' at sætningen allerede var udledt. Hm, dette er da faktisk en ret interessant tanke. Pointen med hele billede af den ``anden mulighed'' her er jo, at den fundamentale logik ligesom ``udleder'' alt, hvad eksisterer (og som måske så netop bliver ``bragt til live'' ved at det bliver ``udregnet''), lidt som hvis man skulle udlede alle sætninger i en formel logik fra ende af og så til uendelig. Bemærk at der så ikke nødvendigvis behøver at være en fuldstændig ordning af alle sætninger, men at den fundamentale logik godt alt andet end lige kan udregne flere sætninger på én gang. (Og hermed kan man så se det mere som en disk eller (hyper-)kugle, der udvider sig i sætningsrummet, og det er så derfor, jeg snakkede om ``rand'' og ``volumen'' før.) Og min pointe nu er så, at hvis den fundamentale logik ikke.\,. hm, ikke altid bare tager den korteste rute til en ny sætning, men i stedet ligesom når at tage alle mulige bevis-ruter i sidste ende, når logikken (eller sætnings-kuglen udbygges), så kan den nemlig måske også ``vælge'' at udføre nogle ``unødvendigt lange'' beviser indimellem, hvor allerede viste sætninger ikke genbruges, men selv udledes på ny. Og så vil alle de oplevelser, der vækkes til live af disse ``udregninger'' så forekomme igen (i den globale tid). Jeg skrev ``vælge,'' som om logikken kun gør én ting ad gangen, og som om den så ``vælger tilfældigt,'' men hvis der skal være perfekt symmetri i den samlede eksistens (og dermed også i alle fundamentale mekanismer), så skal den fundamentale logik altså altid tage alle mulige valg, den kan --- men dog ikke nødvendigvis på samme tid. Så et mere rammende billede vil nok være en graf med knuder i form af sætninger, som udvider sig løbende i sætningsrummet, men hvor der også samtidigt kommer flere og flere kanter (hvor vi så snakker en type graf, hvor knuder kan have flere kanter i mellem sig og af forskellige arter) imellem hvert par af knuder, som tiden (den globale tid, der netop er i denne hypotese) går. Og her repræsenterer kanterne altså så forskellige ``beviser.'' Hm, og måske skal kanter så ikke bare kun kunne være mellem to knuder, men flere knuder skal også kunne forbindes via kanterne (hvilket så ville resultere i noget, der er svært at tegne), hvis man altså skal tage højde for, at beviser meget vel godt kan tage udgangspunkt i flere sætninger. Men det kan man jo altid tænke over, hvis man vil. Ok. 

Jeg kan tænke videre en anden gang over, hvor meget ``tredje mulighed'' vil blive lig ``første mulighed'' ovenfor, men jeg kan sige, at jeg nu altså selv hælder mere til den første mulighed, end jeg gjorde før. Og det gode er, at alle disse muligheder vil medføre en konklusion om, at der ingen grund er til at frygte døden. Hm, om man så frygter livet, det kan dog stadig være.\,. Ja. I tredje mulighed kan man godt argumentere for, at man kan have grund til at frygte livet, for her kan subjektiv fortolkning nemlig være, at man så ``skal leve'' den samme oplevelse igen og igen, og hvis man så har haft et hårdt liv indtil nu, så kan man være bange for, at man bare looper tilbage til ens egen fødsel, og ikke når at leve andre (sandsynligvis mere behagelige) liv inden da.\,. Hm, men lidt sjovt, at den tredje mulighed kun giver mening, hvis man kan argumentere for, at den svarer til første mulighed, og i første mulighed har vi ikke det problem, for så er jeg'et (i.e.\ jeg-oplevelsen) allerede delt ud på mange ``universer'' (som jeg har brugt/defineret ordet, `univers,' i denne sammenhæng). .\,.\,Hm, ja, man kunne jo så spørge, hvorfor i alverden looper man så lige tilbage til samme oplevelse? .\,.\,I princippet looper man jo slet ikke; det hele sker bare hele tiden.\,.  .\,.\,Ja, så hvis man har haft det dårligt, så kan man nok godt finde noget at bekymre sig over (angående spørgsmålet om liv efter død), hvis man virkeligt vil; det kan man nok ikke komme udenom.\,. Men selv hvis vi vælger at bruge den fortolkning (og dermed regne med, at vi looper på et tidspunkt, ligesom hvis universet stoppede og genfødtes), så tror jeg nu alligevel personligt, at det må høre til det usandsynlige med oplevelses-love, der rent faktisk starter og slutter med en enkelt (fysisk) hjernes skabelse og død.\,. Hvis man ud fra en grundlæggende logik har skabt et begreb om en mekanisme, hvor en oplevelse kan følge en (enkelt) fysisk hjerne kontinuert, så giver det for mig mest mening, hvis der så bare bliver en limbo-periode, hvor oplevelses-mekanismen --- det vi kunne kalde `sjælen' --- afsøger det lokale (eller måske globale) fysiske univers efter et nyt holde punkt, der minder om det gamle. Dette må så næsten, alt andet end lige, bare være.\,. Ja, det var da egentligt interessant: Det næste holdepunkt i rækken af, hvad en sjæl vælger at koble til, og hvad den altså så oplever, må da alt andet end lige bare blive det næste punkt, der passer bedst på det gamle. Hm.\,. Okay, lad mig lige afbryde mig selv og tænke lidt videre over disse tanker.\,. .\,.\,Tja, man kunne nu godt sagtens forestille sig sjæle, der så bare holder, hvis de ikke kan finde et acceptabelt nyt begyndelsespunkt i det lokale rum og inden for en begrænset tid. .\,.\,Oh, well. Jeg kunne så overveje, om den gennemsnitlige uendelige mængde information pr.\ oplevelse kan gøre noget.\,. Hm, men ellers kunne man måske godt argumentere for, at der egentligt ingen grund er til, at sjæle ikke bare finder en vilkårlig anden hjerne. I princippet kan de jo også sagtens hoppe mellem hjerner hele tiden, uden at vi jo vil opdage det. (Indskudt: I øvrigt kan der godt være flere sjæle pr.\ oplevelse, ligesom at jeg skrev, at der godt kan være flere jeg'er på en oplevelse (pr.\ min definition her).) .\,.\,Ja, det vil for mig være mere naturligt, hvis sjæle tog udgangspunkt i rum og tid, når de ligesom går fra punkt til punkt (på en kontinuert oplevelsesakse, for ellers var vi jo i ``den første mulighed''-antagelsen), i stedet for at tage udgangspunkt i følelser/hjerne-konfigurationer. Så på den måde, vil det efter min mening være langt mere sandsynligt, at sjælen bare vælger den hjerne, der er tættest på, når man dør (eller måske besvimer for den sags skyld), og så fortsætter der. (Og der er nemlig ingen grund til, at en sjæl vil presse andre sjæle væk fra den hjerne, hvis der er flere sjæle i samme univers til samme tid. De er jo ikke fysiske objekter, der kan frastøde hinanden, så der er ingen a priori grund til, at de vil dette.) Ok. Nu gider jeg ikke bruge mere tid på dette for nu. Man skal have have haft et dårligt liv og oven i købet være ret pessimistisk, hvis man skal bekymre sig om liv-efter-død-spørgsmålet. Det mere naturlige svar er bare, at alt sker hele tiden, og at der i øvrigt findes overgange mellem alle personligheder et eller andet sted i multiverset, så alt i alt kan man se det som, at man vil leve alle mulige liv i multiverset for evigt. *(Nå ja, og jeg kan også lige tilføje, at \emph{hvis} man nu bekymrer sig for, om man skulle have ``trukket et dårligt lod'' her i den samlede tilværelse, så kan det måske i nogen tilfælde hjælpe lidt at tænke på, at man i så fald bare tager denne smerte for andre.\,. Dette kan måske lette nogens sind omkring det --- ikke at jeg nødvendigvis tror at nogen rigtigt vil bekymre sig om disse detaljer overhovedet. De fleste vil nok som mig bare være glade for, at verden er uendelig og indeholde uendeligt mange afskygninger af dem selv og til evig tid --- og hvor ``uendeligt mange afskygninger'' så endda også vil medføre, at man der vil være overgange til alle mulige andre liv, og at man derfor kan se det som, at man selv kommer til at leve alle liv for evigt.)

Nå, men lad mig prøve at vende tilbage så til min opsummering. .\,.\,Nu hvor jeg har tænkt de tanker med, at den fundamentale logik også løbende kan tage alle mulige beviser for at komme til samme punkt, så kan jeg mærke, at jeg egentligt også synes meget godt om den hypotese nu. Så alle de tre omtalte muligheder kunne altså være gode nok efter min (nuværende) mening. Ok, men jeg bør nu fortsætte med endeligt at skrive om `sandsynlighed.' Tja, det burde egentligt være klart for disse og mine tidligere noter.\,. Pointen er bare, at spørgsmålet om, hvordan man skal udregne sit univers og sin oplevelses prior-sandsynlighed, så kan være lidt tricky. Det er endda ikke sikkert, at man overhovedet kan opstille et argument for, at der findes en specifik prior. For hvem siger, at man kan give alle sætninger i den fundamentale logik en ordning, og hvem siger, at man så skal følge denne ordning, når man udregner prior-sandsynligheden. Hvorfor kan man med andre ord ikke bare lave om på rækkefølgen og dermed få prior-udregningen til at konvergere mod noget andet? Jeg har stadig ikke noget klart svar på dette problem, udover at man i det mindste altid kan sige følgende: Det går aldrig at regne sin prior ud, hvis man bevidst prøver at konstruere en rækkefølge af universerne, så sandsynligheden konvergerer imod noget, man søger efter. Hvis man kan få det til at konvergere, så må man altid kun bruge rækkefølger, der har udgangspunkt i en simpel teori. Man må således ikke bare tilføje arbitrære parametre og ændringer til teorien for at få det til at konvergere mod noget specifikt, man søger. Men hvad gør man så, hvis to simple teorier virker lige gode men konvergerer mod noget forskelligt? Ja, det har jeg så ikke svar på. Ikke andet end at man selvfølgelig altid bare kan sige: hvad betyder dette også? Så længe der ikke er nogen paradokser i prior-sandsynligheden, så vil det jo aldrig betyde noget. Og hvis man finder en teori, der medfører et paradoks i prior-sandsynligheden, hvilket enten kan være, at der er 100 \% kaos i det samlede multivers, og at alle tilsyneladende love derfor bare må være usandsynlige tilfælde (og illusioner, om man vil), som vil forsvinde hvert øjeblik, det skal være, eller at der kun er ét eller få universer, der har ikke-forsvindende frekvens, og at ingen af disse kan indeholde vores. Men disse paradokser kan man efter mine overvejelser aldrig konkludere: Jeg er kommet frem til, at der helt klart findes simple teorier, hvor der ikke er de para.\,. Hov, jeg fik ikke færdiggjort mit `hvis' fra før (fra ``Og hvis man finder en teori''). Hvis man gør dette så må man jo bare udelukke denne, for så vil den simpelthen bare ikke passe med, hvad vi observerer. Og som jeg var ved at skrive, så findes der også altid simple teorier, som ikke medfører disse paradokser (som i øvrigt minder lidt om hinanden, for 100 \% kaos-paradokset vil, som jeg ser det, også tit opstå, når for simple universer får al frekvensen i multiverset). 

.\,.\,Okay, jeg tror egentligt, jeg vil stoppe for nu. Det blev alligevel mere brainstorming end opsummering, og jeg fik tænkt over og skrevet om de ting, jeg gerne ville. Og jeg har altså ikke mere at sige på stående fod, som man ikke kan læse sig til i mine tidligere noter. .\,.\,Nå jo, lad mig dog lige påpege, at hypoteserne, hvor vi kunne betragte alle oplevelserne som noget, der kun opleves én gang, dem er jeg så nu gået lidt væk fra. De var ellers gode nok, for så ville den gennemsnitlige levetid, man har tilbage som sjæl og/eller jeg-person, være uendeligt, og.\,. Hm tja, måske ville man alligevel kunne argumentere sig ud i nogle mulige bekymringer der også, men never mind. Disse hypoteser virker altså ikke længere så fornuftige for mig; nu er det umiddelbart mere bare de tre, jeg lige har snakket om her, som er interessante (indtil jeg eventuelt kommer på noget nyt, der giver mening (eller ombestemmer mig, selvfølgelig.\,.)). 

%Sandsynlighed.. (ok tjek)
%Hvad vi kan måle eksperimentielt.. Lad os se på vores univers.. (..nvm)
%Uendelighed og overgange mellem personligheder.. (..nvm)

(05.01.22) Lad mig egentligt lige fremhæve det billede, hvor den fundamentale logik kan ses som en graf (og med flere forskellige kanter imellem hver knude). Det kan så som sagt være, at denne graf ligesom udvider sig i en uendelighed fra et globalt begyndelsespunkt for al eksistens, og her kan det så enten være, at eksistens af universer og af bevidste oplevelser forekommer udelukkende i randen af denne udvidende graf, eller at de forekommer i den indre del af grafen. I det sidste tilfælde er antagelsen altså, at så snart en eksistens er ``opdaget'' af/i den fundamentale logik, så vil den forekomme hele tiden (som noget statisk i den globale eksistens). Og med den sidste antagelse kan vi så i princippet også have endnu en mulighed for dette graf-billede, og det er, at grafen bare selv er statisk og altid fylder det hele (og altid har fyldt det hele så at sige), således at der slet ikke er nogen global tid, og at alt bare sker ``hele tiden.''




\subsection[M.o. QED]{Mere om min QED-teori}
(26.11.21) %(kl. 16:11 nu btw..)
Okay, spændende!! Jeg begyndte i tirsdags (d.\ 23/11) at se lidt på mine QED-noter igen, så jeg kunne blive klar til at skrive om det til min udgivelse, som jeg er i gang med at arbejde på nedenfor. Jeg tænkte, at det kunne være en god eftermiddagsaktivitet, til når jeg gik i stå med det andet arbejde. I tirsdags fik jeg så bare lige gennemgået de grundlæggende noter omkring Lag\ldots\ Nå nej, hedder det ikke en Laplace-transformation\ldots? *(Legendre, self.) Anyway, bare lige noterne omkring $H \leftrightarrow L$ i positionsrummet. I onsdags (eftermiddag) fik jeg så gennemgået noterne for at gå til Fourier-rummet og læste også videre derfra og begyndte at komme så småt tilbage ind i stoffet. Men jeg kunne nu ikke huske, hvordan jeg kom frem til, at $\hat \Pi_V = 0$ og $\hat \Pi_{\boldsymbol A} = k^2$ (hvis man tillader sig at udtrykke det ret upræcist) kunne være løsninger (og heller ikke, hvordan man så viste at disse løsninger var Lorentz-invariante). Så i går d.\ 25/11 tog jeg så en hel tænkedag. Selv hvis mine gamle argumenter holdt, ville jeg ikke have noget imod, hvis jeg fandt nye, så jeg har bare prøvet at tænke lidt i forskellige baner, og har derfor ikke kun gået og prøvet at huske, hvad jeg kom frem til. I dag --- for jeg har nemlig også fortsat i dag med en tænkedag (og selvom jeg taster nu, så skal jeg sikkert også stadig tænke videre, og der kan det godt være, at jeg går væk fra tasterne igen (he, jeg kan forresten mærke, at jeg falder lidt tilbage i mine vaner fra dengang med at gå og have det meget i hovedet, og så ikke arbejde/tænke så meget på papiret eller på tasterne)) --- har jeg endda overvejet, om man måske alligevel skulle prøve med en mere konventionel måde at eliminere gauge-symmetrien på og ligefrem begynde at skære i feltintegralet. For jo mere jeg har tænkt over mine gamle argumenter --- som jeg dog stadig ikke kan huske helt! --- så synes jeg ikke rigtigt de holder. Som jeg husker det, så kunne jeg i første omgang udlede at $\hat \Pi_V$ kunne sættes til 0 overalt, og endda ret uafhængigt af resten. Men nu kan jeg altså ikke helt se hvordan. Jeg kan halvt huske og halvt læse mig til, at min løsning havde noget at gøre med at betragte, hvad der sker, hvis man forskyder feltet i integralet med $\square^2 \varphi = 0$-bølger og så ser på, hvordan det så samler sig sammen, men jeg kan nu stadig ikke få dette til at give mening. Men nu kom jeg så lige til at tænke lidt mere over $\nabla \cdot \boldsymbol A$- / $A_\parallel$-delen igen nu her, og så blev jeg med det samme optimistisk omkring $A_\parallel \sim k^2$-løsningen (eller hvad potensen nu skal være, men allevegne har jeg skrevet $k^2$) igen! Og jeg begyndte så ret hurtigt at tænke over, hvad $\nabla V\, \hat \Pi_{A_\parallel}$-leddet gør, og så kom jeg altså til at tænke på, at dette kunne udgøre et slags energi-led, der hele tiden stabiliserer $\psi(x)$ (elektronens/ernes bølgefunktion) ved at udligne den relative fase, der allers vil komme over et (lille) tidsinterval på baggrund af $\nabla V$\ldots\ Hov, nu tænkte jeg jo godt nok på\ldots\ hm, never mind, men lad mig nu lige tænke lidt mere over det\ldots\ Hm tja, på den anden side kan jeg jo bare tænke over det, når jeg alligevel skal det. Indtil videre synes jeg i hvert fald bare, at det er spændende, at der kan være en mulighed for, at $\nabla V\, \hat \Pi_{\boldsymbol A}$-leddet kan være stabiliserende for $\psi(x)$ set i forskellige positioner/lattice-konfigurationer af $V$. Og lige inden jeg så begyndte at gå til tasterne nu her for at skrive ned, hvad jeg kom frem til, kom jeg så også lidt frem til, at $(\nabla \cdot \boldsymbol A) \hat \Pi_V$-leddet måske så godt kan blive 0 overalt, netop på baggrund af $\hat \Pi_V = 0$ som jeg jo har tænkt for, men nu med det argument (og jeg husker det altså ikke umiddelbart helt som om, at dette var mit oprindelige argument, selvom der dog er nogle vage klokker, der ringer ift.\ disse tanker omkring $\nabla V\, \hat \Pi_{\boldsymbol A}$ og $(\nabla \cdot \boldsymbol A) \hat \Pi_V$\ldots\ldots), at en amplitudeændring i $V(x)$ ikke ændrer på randen af $V$-feltet, og hvis $\psi$'er nu vil have den samme udvikling, hvis bare randen af $V$ holdes konstant, jamen så vil $\hat \Pi_V = 0$ også være bevaret/invariant over tid i alle de indre punkter af $V$. Så det er altså disse (indtil videre løse) argumenter, jeg står med nu, og som jeg glæder mig meget til at undersøge nærmere. (He, hvis bare jeg altså finder en nogenlunde tilsvarende løsning igen, så er dette altså bare en gave; jeg nyder virkeligt at få lov til at løse problemet igen --- faktisk så meget, at det næsten overskygger frygten for, at jeg ikke kan løse det, men samtidig så er spændingen der dog stadig (så altså sjovt nok den samme spænding, men uden frygt; som om jeg aldrig havde gået og troet, at jeg allerede kendte en løsning). He, det er dog ikke fordi, jeg går og er ekstatisk eller noget, men jeg kan bare mærke en følelse i maven omkring det, som virkelig vækker nogle nostalgiske følelser i mig; ja, nærmest som om: jeg \emph{får lov} til at få oplevelsen af at løse (7, 9, 13, selvfølgelig, men på et rationelt plan er jeg dog på ingen måde sikker på, at jeg \emph{kan} løse det igen (og/eller få det samme), men ja, som sagt vækker dette af en eller anden grund ikke rigtigt nogen frygt i mig, som tingene er nu\ldots) opgaven igen.)

(28.11.21) Ha, jeg tror faktisk jeg nærmer mig min gamle løsning og mine gamle argumenter. Jeg må have husket forkert, at jeg først udledte $\hat \Pi_V = 0$ og så bagefter kunne udlede løsningen for $\hat \Pi_{A_\parallel}$. For jeg se i mine (papir-)noter (som bare er én side x) (..der kan dog være nogle andre noter et andet sted, men jeg ved nu ikke, om det er sandsynligt..)), at jeg så netop endte med at ville have $\nabla V\, \hat \Pi_{\boldsymbol A}$-leddet til at udligne den normale $V$-elektron-energi. Så jeg må jo have gået den vej rundt også dengang. Sjovt nok kom jeg også på idéen omkring at lade $\nabla V\, \hat \Pi_{\boldsymbol A}$-leddet gøre dette nu her (selvfølgelig ``snyd'' fordi jeg allerede har tænkt i alle de baner førhen), og sjovt nok var der ikke lige nogen klokker, der ringede med det samme ift.\ min oprindelige løsning (og de ringer stadig ikke rigtigt..), men ja, jeg kan jo se på mit papir, at dette også netop var en tanke, jeg tænkte dengang. Jeg tror i øvrigt også, jeg har fundet frem til mit argument for Lorentz-invarians igen. Nu vil jeg så gå tingene igennem ift.\ at se $\nabla V\, \hat \Pi_{\boldsymbol A}$-leddet virkeligt kan udligne $V$-energien, eller hvad. Jeg skal også derefter se på, hvad coulomb interaktionen bliver (jeg husker det som om, at $k^2$ og/eller $k^{-2}$ på en eller anden måde Fourier-transformerede til $x^{-1}$). Men nu må jeg se. 

Jeg tror, jeg vil gå ned til min dertil indrettede ``QED notes''-part og så begynde at lave arbejdet der.


(29.11.21) He, jeg brugte hvaldelen af i går på at stirre mig blind på normaliseringsfaktoren i feltintegralet, indtil jeg ikke orkede mere. Og så efter at have lagt det fra mig, kom jeg i tanke om, at den nok skal være rigtig nok, og at den alligevel ikke er vigtig. Og så kom jeg til at tænke på, om det mon ikke bare så \emph{er} $\hat \Pi_{A_\parallel} \sim k^{-1}$ i stedet, som jeg også lige var kommet frem til ved et hurtigt overslag. Så spiser $k^{-1}$ jo bare $k$ i $\nabla V\, \hat \Pi_{\boldsymbol A}$, og alt er godt. Jeg har forresten ikke kunne slippe tanken om, at felt-amplituden kunne afhænge af $k$ for felt-bølgerne, men det er selvfølgelig forkert. Jeg kom vist til at tænke på positions-operatoren for en harmoniske oscillator (udtrykt via hæve/sænke-operatorerne), men det er jo noget helt andet. Og ja, da jeg indså dette, kunne jeg jo se at pengene passede. Jeg fik så også summet over, hvad Coulomb-energien så vil blive, og fik faktisk (ret meget i hovedet) fundet frem til, hvad integralet måtte blive, og fik løst det. (Faktisk kiggede først i min Shaum's (som jeg nu ikke har fået brugt så meget ellers) efter det, men der stod det ikke. Så tænkte jeg over, om jeg selv kunne integrere det, og indså jo lynhurtigt, at det har en simpel stamfunktion.) Og ja, ved lige at have haft slået $\int k^{-1} \sin(k x) dk$ op, så kunne jeg se, at det faktisk lige netop var proportionelt med $x^{-1}$, som jeg huskede det. Og ja, jeg har også hele tiden gået at husket det som om, at jeg Fourier-transformerede $k^{-2}$ til $x^{-1}$ (hvilket altså passer i tre dimensioner), og $k^{-2}$ er nemlig lige netop, hvad det drejer sig om, når $\hat \Pi_{A_\parallel} \sim k^{-1}$. Og det giver i øvrigt også fint mening, at jeg har skrevet forkert (og har hele tiden også tvivlet lidt på, om nu det også var rigtigt), for jeg kan jp sagtens bare lige have skrevet $k^2$ på mit (ene!) notepapir, og så har jeg jo bare kopieret den kommentar efterfølgende. (Også fordi jeg jo netop (åbenbart!) ikke har tænkt så meget tilbage på denne del af det hele efterfølgende, men har bare ladet det være et lukket kapitel i mit hovedet (hvilket er grunden til, at jeg nu har skulle bruge så lang tid på at huske/genopdage det igen).) 

Så ja, jeg må hellere lige sørge for at indsætte rettelser: $\hat \Pi_{A_\parallel}$ skal gå som $k^{-1}$, og ikke gå som $k^{-2}$, hvad jeg ellers har skrevet i mine ovenstående noter. \ldots\ Sådan, nu har jeg indsat en rettelse ovenfor.

Så selvom jeg altså lidt har tvivlet på mit fortidige selv nu her, så er jeg altså nu kommet frem til den samme løsning. Og jeg er faktisk også rimelig sikker på, at mine argumenter nu er de samme som dengang. Jeg har godt nok husket det som om, at jeg først viste $\hat \Pi_V = 0$, men dette må bare komme sig af, at jeg har forvekslet det med, at jeg først \emph{satte} $V = 0$ for at se løsningerne for dette tværsnit (og hvor $\hat \Pi_V$ jo så også bare blev sat til 0 / blev ignoreret). Og så gemte jeg nemlig bare $V$-dimensionen (eller dimensionerne, rettere) til sidst (hvor løsningen så altså er, at $\nabla V\, \hat \Pi_{\boldsymbol A}$ kommer til at udligne $V$-energien, og at $V$-afhængigheden dermed forsvinder). 

Den eneste ting er nu, at jeg dog stadig ikke er sikker på, at Coulomb-energien lige bliver ($2\times$) $e^2 / 4 \pi r$, som den vel gerne burde.. Det må jeg altså lige se på nu.

(29.11.21) Ah, jeg fik lavet et overslag på den omtalte udregning her i middags (se nedenfor i QED-parten), og dette gav faktisk $e^2 / 4 \pi r$, som det gerne skulle.! Det er ret fedt, for jeg lavede nemlig aldrig den udregning helt igennem, dengang da det var, så på en måde er det lidt en ny ``opdagelse,'' nærmest. Og dengang tænkte jeg også, at jeg måske alligevel ville skulle re-normalisere elektronmassen og/eller noget andet i den dur, så jeg tænkte heller ikke, det var vigtigt, at det lige blev Coulomb-energien nøjagtigt. Men det tror jeg altså faktisk, det er nu, og så er det jo bare \emph{så} fedt, at det nu ser ud til, lige netop at \emph{blive} lig Coulomb-energien.!

Så nu tror jeg egentligt bare lige, jeg vil læse op på, hvordan resten af beviset (omkring konjugeringen/omfortolkningen af de negative energi-løsninger og Lorentz-invariansen derfra), og så tror jeg godt bare jeg kan tage det stille og roligt igen og arbejde lidt på begge ting (i.e.\ på fysikken og på web-idéerne). 

\ldots\ Mit argument for, at Range$(H \pm iI)$ er tæt, holder (men der behøver ikke at tale om en ``basis;'' jeg skal bare finde en procedure for at finde en vektor (med arbitrær norm og med arbitrær vinkel ift.\ de andre vektorer) for hver $N$-partikel-$k$-tilstand, der tilnærmer sig denne, når man her opereret på den med $H$). **[Nej, det holdte forresten ikke, og jeg har vist ikke nævnt dette i disse noter, men se mine \texttt{qed.tex}-noter for en opdatering på, hvor jeg står med dette problem.] (Og når man har vist at Range$(H)$ er tæt i nabolaget af samtlige vektorer i en basis, såsom altså vores $N$-partikel-$k$-basis, så følger det at Range$(H)$ er tæt overalt.) Hm, og hvordan kommer man så til Range$(H \pm iI)$ til slut?\ldots\ ..Hm, hvis man nu også viser, at der ikke bare findes vektorer, men også arbitrært små vektorer, der sendes arbitrært tæt på den vektor, vi vil tilnærme os, så kan man jo om ikke andet konstruere en række, hvor man bliver ved med at\ldots\ nå nej, man kan så bare direkte argumentere med, at $\ket{\psi}$ så kan vælges arbitrært lille, således at $H\ket{\psi} \pm i \ket{\psi}$ kommer arbitrært tæt på $H\ket{\psi}$, og samtidigt kan man jo så altså også vælge $\ket{\psi}$, så $H\ket{\psi}$ kommer arbitrært tæt på den ønskede $\ket{\psi'}$. Ja, så det er om ikke andet én mulighed. 

Jeg har forresten ikke endnu gennemgået udregningerne for at vise, hvad $\hat \Pi_{A_{||, k}}$ (og måske bør jeg forresten heller denotere dem med $\hat \pi$\ldots\ hm, eller\ldots?) skal være lig, men det gør nu sikkert nok alt sammen. (\ldots Hm, jeg kunne også bruge stort $Q$ og $P$ for felterne, men det bliver måske så ikke så pænt\ldots\ Hm, nej så vil jeg næsten hellere bare holde mig til $q$ og $p$ (det må også være fint nok, tænker jeg)\ldots\ Hm, på nær at $\hat p_{A_{||, k}}$ bliver lidt forvirrende\ldots\ Hm, men om ikke andet kan jeg jo bare gøre, som jeg har gjort, og skifte til $\hat \Pi$ eller $\hat \pi$, når dette bliver pænere (i.e.\ når vi begynder at se på elektrodynamikken)\ldots)

\ldots Hm, jeg kan ikke helt lige se, hvad jeg mente med prop.\ 10.14 i Hall, og ja, uanset hvad, så mangler jeg altså stadig lige at se på, om mine argumenter omkring Lorentz-invarians af den endelige $H$ holder (og at finde frem til dem og/eller at finde på nye). Jeg kan jo lige se lidt mere på mine noter ovenfor, og så ellers summe lidt over det til i morgen. 


(02.12.21) Okay, jeg har lige haft et par tænkedage (inkl.\ mandag aften og i dag, formiddag og middag). I mandags aften kom jeg til at tænke over den gyromagnetiske ration, og tænkte lidt over, om den mon kunne stamme fra, hvordan Coulomb-potentialet interagere med Dirac-elektronerne, nu hvor halvdelen af overgangende (fra elektron til hul/positron) er omfortolket. Jeg tænkte også lidt over, hvordan man mon kunne opstille et tankeeksperiment, der udleder den gyromagnetiske ratio. Angående dette, så har jeg også tænkt lidt videre siden (hvor jeg kom på at prøve at se på et eksempel, hvor $A^\mu$-feltet tages som værende klassisk (i stedet for at se på\ldots\ Ja, jeg kan lige så godt sige, at de to andre ting, jeg havde i tankerne var at se på, hvordan en elektron bevæger sig i forskellige/visse $\int \rho(x) dx/4\pi r$-potentialer, eller at se på en kohærent foton-bølge)), og jeg kan i øvrigt stadig godt tænke mere over dette. Og uanset hvad, kom jeg frem til, kan jeg også tænke mere over, hvordan Coulomb-potentialet kan pertubere elektron-/positron(/hul)-tilstandende og/eller deres energier. 

I tirsdags tænkte jeg så særligt over, hvordan jeg viser Lorentz-invariansen (hvilket jeg jo også lagde op til i mandags (som var d.\ 29.)). Jeg nåede faktisk at overveje en del, om man egentligt ikke alligevel burde konjugere de nye egentilstande i stedet for at konjugere de bare $\psi$'er. De har jo altid været lidt spørgsmålet; bør man gøre det ene aller det andet? Og jeg kom faktisk lidt kortvarigt frem til, at man nok alligevel bør prøve at konjugere til allersidst. Men så slog en anden løsning mig: Hvad med om man bare konjugere, inden man overhovedet eliminerer gauge-symmetrien (og altså $A_\parallel$ og $V$)?! Jeg undrede mig så en del over, og gør det stadig, hvorfor jeg enten ikke har tænkt noget mere på denne mulighed, og/eller hvorfor jeg dog gik væk fra den? Jeg sluttede så dagen af med at indse (7, 9, 13), jeg jeg \emph{måske} kan finde løsninger til Yukawa-teorien! (Og jeg kom også lige på den nævnte tænke om at se på et klassisk $A^\mu$, når det kommer til den g.m.\ ratio.) Men ja, dette vil jeg være vildt spændende (hvis min løsning holder)!

I går fik jeg så tænkt mere over L.-invarians-delen, og fik faktisk tvivlet på, at min idé holdte to gange. Men nej, jo; det ser ud til at holde\ldots! Og nu forstår jeg altså stadig ikke, hvordan jeg dog i alverden har gået så meget væk fra den mulighed. I mit hovedet kan jeg fornemme en tanke/følelse af, at jeg blev nødt til (som jeg erindrer det) at konjugere efter gauge-eliminationen, fordi jeg jo skulle\ldots\ at det ligesom var måden at få hele resultatet med at $A_\parallel$ giver Coulomb-energien osv\ldots\ Tja, det er lidt svært at finde ud af; det virker som nogle tanker, jeg absolut bør have tænkt over, men ja\ldots\ Jeg kunne lige gennemgå mit kladdehæfte fra dengang, for der bør være nogle sider (mere end bare én måske x)), hvor jeg overvejer dette, men hvad fa'en. Det betyder jo ikke noget, og det er nu heller ikke fordi, at jeg på noget tidspunkt har set dette som et vildt stort problem; der var altid langt større problemer at takle, og jeg følte hele tiden, at problemet ikke ville være så svært, når det endelig var (selvom jeg dog selvfølgelig \emph{har} overvejet det seriøst i hvert fald i nogen tid\ldots). Og da jeg endelig kom tilbage for at tjekke den del af beviset, var det i sommeren '19, hvor jeg bare lige gik med det i hovedet lidt i løbet af min ferie på Bornholm. Så ja, det er også fair, at jeg har overset og/eller gået væk fra denne mulighed, men jeg synes nu stadig, det er en smule underligt (ikke mindst fordi mit fortidige selv kun positivt har overrasket mig, nu hvor jeg har gennemgået\ldots\ Tja, og dog, for jeg har muligvis også lige fundet en løsning på et problem (som vi kommer til om lidt), som jeg bestemt har brugt lang, lang tid på dengang, så\ldots\ både og\ldots (men nu må vi også se, om jeg har ret i det --- det handler om at føre min \emph{mulige} løsning på Yukawa-teorien over på QED-teorien, og jeg er altså ikke nået at til tjekke det ordentligt endnu)). 

Men ja, så en rimeligt simpel løsning (som ikke kræver alt muligt hurlumhej med Trotter-ekspansioner *(eller jo, der kommer en Trotter-ekspansion i beviset, men det er så en simpel én) og ultraviolette cut-offs (hvilket jeg ellers havde et meget godt argument, ja nærmest et lemma, omkring)), er faktisk bare at holde hele $A^\mu$-feltet klassisk og så udlede at Dirac-teorien på et vilkårligt klassisk $A^\mu$-felt er Lorentz-invariant. Hertil kan man så (og det er min nuværende strategi) bruge, at $\hat S = \exp(-i \int \hat{\mathcal{H}}(x) dx)$ kan approksimeres med en diskret version --- bl.a.\ fordi man kan ignorere ikke-kommutationen imellem to naboområder, når områdernes størrelse går imod 0. Man kan så vise, at resultatet i alle inertialsystemer --- og vi kan i øvrigt lade $A^\mu$-felt være 0 på nær i et indre område --- kan skrives som den samme variable sum, som kan varieres i alle indre punkter i det diskrete $A^\mu$-felt. Denne sum vil så i øvrigt bestå af en blanding af $\psi(x)\psi^\dagger(x)$'er, samt af propagatorer (hvilke man dog også kan opsluge, hvis man ser det i interaktions-billedet). Ved at antage antisymmetri og ved at bruge kommutations-relationerne --- samt at bruge, at de konjugerede antipartikel-$\psi$'er, hvor fortegnet på energien også flippes, propagerer på samme måde som den ikke-konjugerede udgave men med negative energier --- vil man så ikke bare kunne vise, at resultatet stadig bliver det samme i alle inertialsystemer efter konjugeringen, men endda også at matrixelementerne af $\hat S$ også bliver det samme i den konjugerede version, bare hvor de selvfølgelig er omfortolket til en ny overgang. Dette sidstnævnte resultat kan man så bruge, når man går videre og ser på, hvad der sker / kan ske for det kvantemekaniske $A^\mu$-felt. For så kan man nemlig stadig argumentere for, at der må være samme $\hat \Pi_{A_\parallel} \sim k^{-1}$-løsning, bare hvor man selvfølgelig argumenterer for alle partikel-antipartikel-konfigurationer. Argumentet kan nemlig gøres ved så at se på start og sluttilstanden, hvilket så er givet ud fra $\hat S$, og fordi $\hat S$ så har samme overgangsamplituder (bare omfortolket), så må der også være den samme gauge-symmetri for teorien, og løsningerne må altså også være på samme måde (selvfølgelig bare med fasen flippet for positronerne). Og herfra går proceduren så på samme måde: Man argumenterer så for, at Dirac-ligningens $V$-energi vil blive spist af $(\nabla \cdot V)\hat \Pi_{A_\parallel}$-leddet, og at $\hat \Pi_V = 0$ derfor er en løsning. Ved at se på Lorentz-transformationer, og særligt på hvordan $\square^2 \varphi = 0$ -bølger (hvor $\partial^\mu \varphi$ lægges til $A^\mu$) kan forskyde start- og slut-tilstandende (og lad mig lige forklare dette lidt nærmere her i disse noter inden længe), kan man så udlede, at $\hat \Pi_{A_\parallel} \sim k^{-1}, \hat \Pi_V = 0$ -løsningen også er L.-invariant, og altså også transformerer til løsninger af samme type/domæne, når man Lorentz-transformerer. Hermed får vi, at teorien (med omtalte løsninger/løsningsdomæne) er Lorentz-invariant, og man kan så udlede $H$ og vise, at den minder om den gængse $H$, men bare hvor der altså nu fremkommer et Coulomb-potentiale (som det kendes fra standard kvantemekanik) oveni\ldots\ Nå nej, eller ikke helt; det var sandt for den ikke-konjugerede teori, men nu hvor vi har konjugeret teorien, så bør $V_{Coloumb}$ altså nu skrives i $k$-rummet, hvor det (vist nok) nogenlunde kommer til at blive $V([ k_1, k_2] \to [k_1', k_2']) = s / (k_1' - k_1)^2 \delta(k_1' + k_2' - k_1 - k_2 )$ (og her kan man altså lige selv kan forestille sig fede $k$'er, hvis man vil), hvor $s\in\{-1, 1\}$ så afhænger af, hvor mange af $k$'erne er fra partikelbølger, og hvor mange er fra antipartikelbølger. Og dette bliver altså så matrixelementet for en $\psi_{k_1'}\psi_{k_2'}\psi_{k_1}^\dagger\psi_{k_2}^\dagger$-overgang, hvis f.eks.\ alle $k$'er er fra normale partikler, og eller kan daggerne altså fjernes og sættes, og ``flippes'' så at sige, hver gang vi erstatter et elektron-$k$ med et positron-$k$ i subscripterne. 

Nå, men ikke nok med dette kom jeg så også i går på\ldots\ eller rettere, jeg kom vist på det i sengen i forgårs nat, men fik så tænkt videre over det i går, nemlig om ikke min \emph{mulige} (`mulig' som i `muligivis') Yukawa-løsning ikke også ville kunne overføres til QED-teorien!?. I øvrigt så var jeg kommet frem til, og er også kommet lidt frem til det samme tidligere i dag i øvrigt, at min mulige løsning så \emph{ikke} ville have et $1/4\pi r$-potentiale. Dette vil være et super interessant resultat i sig selv (fordi jeg jo netop har en teori, hvor Coloumb-potentialet kan forklares uden behov for et sådant!.), og det vil jo så være særligt interessant, hvis jeg endda ligefrem kan finde samme løsning (hvormed samme argument altså så kan overføres), for QED-teorien. 

Jeg regnede så lidt (bare i hovedet, hvilket jeg jo åbenbart har for vane at gøre meget med fysik (og så har jeg måske bare lige slået nogle få formler og resultater op hist og her --- men altså ikke skrevet så meget ned --- hvilket jo nok er lidt en dårlig vane, hvis jeg tænker på, hvor meget jeg lige har skulle rekonstruere fra ét stykke papir. xD)) på energi-forskydningen, der nok vil komme fra min mulige løsning, og også på, hvor meget vektoren/tilstanden forskydes i Hilbert-rummet. Jeg kom umiddelbart frem til, at begge konvergerede til noget konstant, men i dag tvivler jeg nu meget på dette resultat for energien især. Men dette er altså noget, jeg jo bør se nærmere på. 

Det nåede at blive sent i går, og om natten, inden jeg gik i seng, fik jeg så en idé til, hvordan man nok (hvis min indskydelse altså holder, hvad der jo slet ikke er sikkert) måske kan gribe løsningen an, når man skal have $\boldsymbol \sigma \cdot \boldsymbol A$ -operatoren til effektivt set at lade Dirac-spinoren være, når den opererer på den. Jeg har en indskydelse om, at dette vist nok kan lade sig gøre (men 7, 9, 13, og selvfølgelig ikke sikkert)\ldots\ 

I dag har jeg så tænkt lidt over\ldots\ Ja, for jeg indså nemlig, at min løsning (og lad os bare overveje selve Yukawa-teorien her til at starte med) ikke vil holde alligevel for flere partikler, hvis man får, at energiforskydningen afhænger af $r$ imellem de to partikler. Men tidligere i dag kom jeg så faktisk lidt frem til, at energiforskydningen (selvom den umiddelbart faktisk eksploderer (hvilket én-partikel-energien i så fald så klart også må), hvis man ikke kan regularisere/renormalisere den på en måde (hvad jeg jo nok kommer til med min vante diskretisering)) faktisk er uafhængig af $r$. Og hvis min hovedudregning er rigtig, jamen så \emph{er} dette lige præcis det spændende resultat, jeg håber på.! (Der er flere resultater, jeg kan håbe på, men dette simple resultat er nu nok det, jeg håber mest på\ldots) 

Så det er altså her, jeg står nu! (Jeg glemte lige at nævne, at én af de grunde til, at jeg tvivlede lidt på den nye strategi omkring L.-invarians der i starten af i går, var, at jeg fik den tanke, at jeg ikke ville kunne håndtere alle par-produktionerne og sådan, men det er jo noget pjat. (Men måske kan jeg altså have tænkt noget tilsvarende engang før i tiden\ldots\ Nå! Slut med at kigge mere tilbage!.)) Nu er den faktisk blevet lidt sent igen, inden jeg fik skrevet alt dette, men det er jo også ok. Nu skal jeg så have skrevet hele min udledning af min QED-teori ned, og jeg kan måske endda lige sørge for lidt tidligt at se på $\square^2 \varphi = 0$ -delen igen, som jeg nævnte, at jeg burde. Dette vil jeg jo gerne putte i forlængelse af min note-udgivelse, som jeg jo arbejder på nu her. Jeg tænker nu at gå i gang (fra i morgen som det første på arbejdsdagen (bemærk, hvordan jeg lige sneg mig uden om at love, at jeg vil stå tidligt op)) igen med at skrive mine web-noter færdigt, og så tænker jeg nemlig nu, at fysik-arbejdet måske kan gå hen og blive aften- (og/eller sen eftermiddags-)arbejde. Og noget af det, jeg helt sikkert gerne lige vil arbejde på, inden jeg udgiver, er altså mine Yukawa-løsninger (og på også muligvis at overføre disse til QED-teorien). For det vil jo som nævnt bare passe vildt godt i forlængelse af min teori-udledning, hvis jeg f.eks.\ ligefrem kan sige, at Yukawa-teorien ikke egentligt giver et Yukawa- (/Coulomb-)potentiale.! Og \emph{hvis} jeg virkeligt kan finde løsninger for QED-teorien (når man altså lige ignorerer par-produktionen plus måske nogle andre ting), så vil dette jo også bare være fantastisk. Det svære ville så næsten blive at stoppe derfra, for jeg vil jo gerne have udgivet mine noter snart. Men der må jeg så bare være god til selv at sætte grænsen, og så bare skrive i mine noter, at jeg vil arbejde videre med dét og dét, når jeg får tid efter udgivelsen. Jeg vil selvfølgelig lige gøre bare \emph{nogle} tanker, omkring at forsøge at undersøge den gyromagnetiske ratio, men jeg skal altså virkeligt prøve at passe på, at jeg ikke bliver opslugt af det. Ok! Var det det hele, jeg skulle skrive? (Jeg vender jo tilbage til tingene igen, og bl.a.\ om, hvad mine løsningstanker indebærer\ldots) Ja, det bliver det for i dag! Og så vil jeg altså i første omgang fortsætte på web-idé-noterne i morgen. (02.12.21) 


%%(03.12.21) Okay, lad mig prøve at redegøre for mit $\square^2 \varphi = 0$ -argument. Lad os se på ...
%(04.12.21) Ha, jeg prøvede at huske mit ``$\square^2 \varphi = 0$ -argument'' i går, men det er ikke lige lykkedes (så ved faktisk ikke pt., om det holder, eller om jeg mon bare har tænkt noget lidt usammenhængende (og jeg gider ikke lige at prøve at bruge tid på at granske min hukommelse grundigere)). Til gengæld gør dette ikke noget, for der er et langt bedre argument! :D I går (tidlig) aftes kom jeg således frem til, at man bare kan argumentere ud fra $\psi(x)$ (/$\psi^\dagger(x)$), som jo netop findes som en lokaliseret operator i $x$-rummet. Denne vil så samle en fase op af $\exp(-i \varphi(x))$, når %$\tilde A_{\parallel\, k\, s} = k \tilde \varphi_{\parallel\, k\, -s} \neq 0$ 
%$\nabla \cdot A \neq 0$, og denne vil så nemlig bare være\ldots\ (bevaret under L.-transformation\ldots) Hov, hvad med $\nabla \cdot A$ selv, den vil jo ikke bare sådan lige transformere til sig selv\ldots? Hm nej, der var jeg lige lidt for hurtig\ldots\ 
%\ldots Hm, nu overvejer jeg endda, hvordan jeg egentligt konkluderede, at $V=0$ kunne medføre at $k \tilde \varphi_{\parallel\, k\, -s} = \tilde A_{\parallel\, k\, s}$ (gange $q_{\mathrm{coupling}}$), for ville dette egentligt ikke også kræve et mere kompliceret argument\ldots? \ldots Hm, for et konstant $A_\parallel$ i tid (for et specifikt inertialsystem, vi så kigger i), der vil der bare komme omtalte fase til forskel, når man sammenligner forskellige $A_\parallel$\ldots\ 
%\ldots Hm tja, men på den anden side behøver man måske ikke at argumentere ud fra felt-integralet; hvis man kan se at $\hat \Pi_{\tilde A_{||\, k}} = q/k$ løser Schrödinger-ligningen, så kan man jo bare gå videre til at vise Lorentz-invariansen, hvis man kan dette\ldots\ 
%%Hm, måske \emph{skal} jeg hellere bruge store P og Q for lattices.. ..Ja, mon ikke det faktisk er bedst..
%\ldots Ah, måske kan man argumentere den vej; se at det er en løsning for et specifikt inertialsystem ud fra $\hat H$, og så konkludere noget om $\psi(x)$ (/$\psi^\dagger(x)$) ud fra dette\ldots(!\,\ldots) 
%%
%%%Eksperiment:
%%\ldots
%%
%%.\,.\,. %Ah nice, det er samme spacing.. 
%%
%\ldots Hm, gad vide om jeg egentligt ikke startede med at gøre dette oprindeligt, altså se på et $A_\parallel(x)$, der er konstant i tidsretningen (og et $V$, der er 0 overalt), og så gå videre derfra?.  
%%
%%%Hm:
%%\ldots
%%
%%?.? %Hm, kun næsten samme spacing, men det er også ok.. 
%%
%Ja, det kan sagtens være, og det virker umiddelbart også fornuftigt nok\ldots\ \ldots Nå, selvom det er super vigtigt, det her, så tror jeg lige jeg vil gemme overvejelserne til senere, for jeg var lige i gang med et hot streak omkring (brainstorming til) web-idéerne. 
%
%\ldots\ Ah, det har altså virkeligt været nogle gode ting, jeg har indset, og nogle gode idéer, jeg har fået, i dag omkring hjemmeside-systemet der. Nu håber jeg bare sådan, at jeg også kan finde frem til et bevis igen på, at mit løsningsdomæne her har sig selv som billede, når man Lorentz-transformerer.\,.\,! 
%
%\ldots Hov, kan man egentligt ikke nå alle slutpunkter, hvis man starter med en given $A_\parallel(\boldsymbol x)$ (eller $\nabla \cdot A( \boldsymbol x)$ med andre, mere konventionelle ord) og $V(\boldsymbol x)$ og skal nå en vis $(V(\boldsymbol x'), A_\parallel(\boldsymbol x'))$ til slut, hvis man har lov til at lægge $\partial^\mu \varphi$ til feltet, oven i det originale $(V(x), A_\parallel(x))$ (hvor jeg ikke lige har gidet at gøre $A$ fed)? For vi må da kunne ændre vilkårligt $\nabla \cdot A( \boldsymbol x)$'s start- og slutpunkter, og ved også at ændre $\nabla \cdot A( \boldsymbol x)$ i midten må man da også kunne opnå en vilkårlig slut-$V(\boldsymbol x)$, også selvom man skal opfylde $\square^2 \varphi = 0$?\ldots\ Hm, så vis vi først ser på integralet, hvor de to felter er konstante i tid\ldots\ Hm, eller hvor $V$ endda er 0, hvis vi gerne vil undgå noget bidrag til $\nabla \times A$, når vi L.-transformerer\ldots\ Hm, så kan vi tilføje en vilkårlig $\varphi$ som starter helt flad, men hvor $A_\parallel$ så\ldots\ Hov, kom jeg til at forveksle $\square^2 \varphi = 0$ med $(\partial^\mu \varphi)^2 = 0$\ldots? Nå nej, det er vel lige godt, for jeg er vel stadig fri til at variere $\nabla^2 \varphi(x)$ til\ldots\ nå nej, pjat\ldots\ Ja, jeg har i stedet forvekslet det med $\partial_\mu A^\mu$, hvor der er den frihed til at sætte $\boldsymbol A$ vilkårligt. Nå, jeg må vist lige hellere tænke lidt væk fra tasterne\ldots\ \ldots Hov, nu returnerer jeg lige. Man kan vel stadig nå vilkårlige sluttilstande ved at lægge en $\varphi$, $\square^2 \varphi = 0$, til? \ldots Hm, nej ikke med $A$ og $V$ samtidigt\ldots\ Hm, jeg går lige ud i kommentarerne\ldots\ 
%%Okay, lad mig lige tænke tænke lidt.. ..Hov nej, jeg har jo nærmest ingen frihed (når vi kigger på at specifikt k, og vi L.-transformerer).. (..Hov, L.-trans er forresten tvetydigt, men jeg har jo selvfølgelig ment 'Lorentz-'.. ..Nå ja, det er jo endda to andre at forveksle med..) ..Ah ja, for fasen har jo endda bare noget at gøre med, hvor meget vi så skifter fra cosinus til sinus og omvendt. ..Ja, men jeg kan altså nå en vilkårlig slut-tilstand, og nu kan jeg huske, at mit argument så har været at se på, hvordan A, V = 0 må gå til forskellige slut-felter.. ..Hm, hvilket jo nok vil være ret sådan.. ja, altså: Det eneste man nok kan konkludere er vel bare, at.. Hm ja, nærmest at vi altid kan trække A_\parallel og V fra via \varphi-bølger og få den samme overgang, når det kommer til elektroner (og positroner) og fotoner, vel?.. ..(I øvrigt kan vi bare vælge et punkt, så cosinus-bølgerne går rent til cosinus, og tilsvarende med sinus.) Hm, hov.. (he, måske ikke det bedste at lave denne type arbejde på dette tidspunkt xD.. og dog; det gode ved typen af arbejdet er, at det ikke koster noget, at går forkert (især ikke, når jeg bare skriver herude i kommentarerne)) jeg har da nærmest helt i al denne tid glemt, at phi-fasen skal afhænge af elektronernes positioner. xD Åh ha.. (Klokken er bare ti over 19 forresten, men alligevel..x)) ..Tja nej, jeg tror jeg holder med tasterne for i dag. Det har også allerede været en givende dag (..endnu en). ^^ 
%%(05.21.12) Okay, nu har jeg husket mit gamle argument nogenlunde, men det er vist lige knap så simpelt. Så nu tænker jeg altså over det. (Jeg tror, at en af grundene til, at jeg kom væk fra mit oprindelige argument, var, at jeg kom til at tænke på \square^2 phi = 0 -bølgerne som mere komplicerede, end de er (når man skal transformere) (fordi jeg jo bare lige har gået med det lidt i hovedet, og fokuseret mest på nogle andre ting). Og nu hvor jeg så kom frem til mit gamle argument, og at det holdt, der har jeg nok kommet til at se dem som mere simple, end de er; de er vist lidt en mellemting..) Nå, lad os så se. Jeg vil gerne fortsætte med at se på cosinus- og sinus-bølger, det har jeg det altså bare langt bedst med; de er både nemmere at tænke over og i sidste ende også nemmere at argumentere (selvom det dog kræver lidt ekstra arbejde i ligningerne). Ved at indlægge en phi = a cos/sin(|k| t + k \cdot x) -bølge kan vi forskyde feltet med V,A_\parallel = a|k| sin/cos(-"-). Så vi kan kun eliminere én af de to ved denne forskydelse. Jeg er klart mest interesseret i at eliminere V, så.. Hm, jeg var ved at sige, "så man ikke får bidrag til B i sluttilstanden," men det kan man jo godt sådan set gøre fra A_\parallel, fordi den jo kan ændre sig i tid.. ..Hm, nu kom jeg lige til at tænke på, at en phi = a sin/cos((|k| + b) t + k \cdot x) -bølge må give, lad os se, \square^2 phi = a (b^2 + 2|k| b) cos/(-)sin(-"-), hvilket.. hm, nå ja, man integrerer så godt nok dette i anden i Lagrange-funktionen, så man får vel bare noget der divergerer som \int a^2 (-"-)^2/2 dx.. ..Ja, lad mig lige gå tilbage igen.. ..Okay, så jeg kan forskyde V, hvis jeg bare også forskyder A_\parallel.. ..Hm, nu kom jeg lige på, om man mon kunne få gavn af at blande de to felter sammen på en måde, men det kan jeg lige holde i mente.. ..Hov, det skulle lige have været phi = a sin/cos(|k| t - k \cdot x), og så bliver V,A_\parallel = \pm a|k| cos/-sin(-"-).. Hm, og hvordan kan man egentligt blande dem sammen, hvis man skulle..? ..Tja, lad mig lige putte en nål i det.. ..Hm, det går måske slet ikke at se på.. klassiske.. hm.. ..Hm, lad mig egentligt lige tænke over, hvad proceduren helt præcis er, når man skal Lorentz-transformere.. ..Hm ja, den burde jo bare være det og det, men hvordan ved jeg nu, at det og det giver en unitær transformation..? ...Hm, man må kunne argumentere med, hvordan k-bølger i hyperplanen sendes til andre k-bølger i samme hyperplan, men hvor man har Lo.-transformeret rum- og tids-koordinaterne (m.m.).. ..Hm, eller fungerer dette argument?.. ..Ja, for det er jo ikke trivielt, at dette \emph{er}, hvad man kan gøre under en Lo.-trans.. ..Hm, hvis vi nu ser på en hyperplan bare en lille bitte afstand væk fra en anden hyperplan, således at vi kan ignorere \hat H og/eller \hat H_{Int} i dette mellemrum, hvordan bliver transformationen så..? ..Hm nej, dette kan man ikke; man skal have hele \hat H med.. ..Og rumtids-spacingen skal gå imod nul for at få den rigtige transformation.. ..Doh.. Åh, er jeg ikke kommet til at tænke.. ups! Jeg kom til at tænke, at f.eks.\ \partial_t A kunne give resultere i en forskel på, hvilken slut-tilstand, man går til, når man transformerer..! x) Åh.. He åh, det ville da være et mareridt. x) Ej, hvor fjollet.. Jeg kom til at tænke på det, som om E og B var de underlæggende.. Hm, men vent, for \partial_t A kan.. Nå ja, doh: Ja, den kan have indflydelse på (og altså bidrage til) \nabla \times A i den transformerede tilstand, men den bidrager ikke til A_\perp..! x) Okay, nu kan jeg forhåbentligt komme lidt bedre back on track, så. Og ja, så bliver Lorentz-transformationen mere triviel ift., hvad proceduren er, og man må så i øvrigt kunne se, som jeg lagde op til, at k-bølger sendes direkte ind i tilsvarende k-bølger, når man omfortolker koordinat- (og spinor-/vektor-)retningerne i omtalte hyperplan. ..Hm, ah, gad vide, om så mine tanker omkring bare at se på psi'erne og sådan måske holder alligevel..? Lad mig nu se.. ..Ah ja, det holder måske..!.. 
%%... Ah, nu tror jeg endelig, jeg kan se strategien. Man kan forskyde til enhver slut-tilstand af V, A_\parallel ved først at forskyde hen til den tilsvarende start-tilstand for V, A_\parallel, nemlig ved at påføre en fase i henhold til løsningsdomænet, og via et \square^2 phi = 0 -argument kan man så se, denne løsning så bare må samle en hvis \exp(-i phi)-fase yderligere op i forhold til en nulte sluttilstand. Så forskellen på den givne slut-tilstand og den nulte slut-tilstand må så være en vis fase. Og fra denne fase må man så direkte kunne udlede den transformerede V- og A_\parallel-afhængighed fuldt ud (uagtet måske en samlet overordnet fase-forskel, som man jo dog bare kan ignorere i argumentet). Fedt nok. Så må jeg bare lige lave den udregning.
%
%(05.12.21) Okay, nu er jeg klar igen, og tror altså, at jeg har fat i bevisstrategien igen. (Hvis man vil læse lidt om, hvad jeg har tænkt galt indtil nu (i denne lille periode), og prøve at følge min tankegang lidt, så kan man læse, hvad der står ude i kommentarerne (imellem denne og forrige paragraf).) Jeg bør, hvis jeg har, tage og se, ikke på felt-integralet (heller ikke delvist), men bare på $\exp(-i \int \mathcal{H}(x)dx)$-integralet. Jeg kan så se på en specifikke positions-tilstande af elektronerne, start- og slut-, og se på specifikke felt-tilstande *(i hvert fald specifikke start-felt-tilstande), når det kommer til transformationen (så altså hvor jeg ser på et specifikt matrixelement, på nær at jeg bare ser på hele foton-rummet på én gang). Pointen i bevisstrategien er så, at man i første omgang, hvis man starter på f.eks.\ $V(\boldsymbol x)=0, A_\parallel(\boldsymbol x) = 0$ for den indledende, ikke-transformerede tilstand, kan forskyde dette til en vilkårlig anden $(V(\boldsymbol x)', A_\parallel(\boldsymbol x)')$ ved at opsamle en fase i henhold til løsningsdomænet (givet ved $\hat P_V = 0$ (hvad jeg har plejet at kalde $\hat \Pi_V$) og $\hat P_{A_\parallel\, k} \propto 1/k$). Når man så endelig udfører transformationen, så kan man argumentere for, at forskellen på den samlede slut-tilstand, som den specifikke $(V(\boldsymbol x)', A_\parallel(\boldsymbol x)')$ sendes til, må ligne den samlede slut-tilstand, som  $(V(\boldsymbol x) = 0, A_\parallel(\boldsymbol x) = 0)$ sendes til, bare hvor der opsamles en fase til forskel, som kun afhænger af fermion-positionerne samt af forskydningen af $(V(\boldsymbol x)', A_\parallel(\boldsymbol x)')$ ift.\ $(V(\boldsymbol x) = 0, A_\parallel(\boldsymbol x) = 0)$. Eller rettere, det tror jeg, man må kunne. Og idéen er altså så at betragte forskellen i de to felt-integraler, hvor man\ldots\ ah, hov\ldots\ Her tror jeg igen, jeg løber ind i, at man ikke bare kan lægge en $\square^2 \varphi = 0$ -bølge oveni for at sammenligne de to tilstande. Jeg fik lige tænkt det sådan, at der altid finde én modpart (som vi kunne kalde $(V(\boldsymbol x')'', A_\parallel(\boldsymbol x')'')$.\,. hvis ikke dette så var lidt forvirrende med de forskellige mærker\ldots\footnote{For mine første mærker betød nemlig \emph{ikke} Lorentz-transformeret, men bare forskudt i forhold til $(V(\boldsymbol x) = 0, A_\parallel(\boldsymbol x) = 0)$.}) til en given $(V(\boldsymbol x)', A_\parallel(\boldsymbol x)')$, når man transformerer (hvor den samlede slut-tilstand så vil centrerer sig om denne modpart på samme måde, som at $(V(\boldsymbol x) = 0, A_\parallel(\boldsymbol x) = 0)$ vil centrere sig om sin modpart). Men denne konklusion bygger vist på, at man kan forskyde sig til et vilkårligt $(V(\boldsymbol x)', A_\parallel(\boldsymbol x)')$ med en $\square^2 \varphi = 0$ -bølge, og det kan man jo nok ikke\ldots\ 
%%Ej, klokken er allerede 15; jeg har brugt næsten hele dagen (i hvert fald hvis man ser på, hvor effektiv jeg plejer at være på forskellige tidspunkter) bare på dette.. ..Nå, men jeg kan nu mærke, at jeg har mere gods i mig.. ..Hm, men lad mig så lige hurtigt nedfælde nogle web-tanker fra i går som det første... 
%%Okay, så er det gjort.:) ..Hm, men orker jeg at arbejde mere på dette lige nu..? 
%
%%\ldots Hov vent, havde jeg ikke fat i noget her? %..Ah, hvorfor er det her så flygtigt for mig..! x)
%%\ldots Tjo, tja\ldots\ 
%
%%Hov, kan man ikke godt nå et vilkårligt (V_k, A_{\parallel k}) med et phi?..! For hvis jeg kigger på f.eks. .. phi = a cos(k t \pm k x), så får man \Delta V = -a|k| sin(k t \pm k \cdot x) og \Delta A_\parallel_k = \mp ak sin(k t \pm k \cdot x)..! Og disse to phi-løsninger for k kan så sættes i superposition til at opnå \Delta V = -(a_1 + a_2) k sin(-"-) og \Delta A = -(a_1 - a_2) k sin(-"-), hvilet spænder over alle amplituder for (\Delta V, \Delta A) (hvor jeg har udeladt subscripts)! Yes! Åh, godt jeg endelig fik indset denne (nærmest selvfølgelig x)) ting..! Ja, og nu holder min løsning så, og den er i øvrigt nærmest identisk med den, jeg førhen, og også ad flere omgange nu, har haft i tankerne. x):D Det eneste er, at jeg nu har tænkt, at det nok er smart at argumentere henholdsvis med Lagrange- og med Hamilton-integralet, når det kommer til henholdsvis lemmaet om, at hver $(V(\boldsymbol x)', A_\parallel(\boldsymbol x)')$ ligesom har én modpart, der svarer til hvordan $(V(\boldsymbol x) = 0, A_\parallel(\boldsymbol x) = 0)$ relaterer sig til en sluttilstand omkring $(V(\boldsymbol x') = 0, A_\parallel(\boldsymbol x') = 0)$ (..eller noget i den stil), og så det samlede argument, hvor man altså nok skal argumentere ud fra $\exp(-i \int \mathcal{H}(x)dx)$-integralet i stedet.. Hm, lad mig lige tænke over, om dette endeligt holder, og om man skal dele de to nævnte argumenter op således.. (He, gad vide, om ikke det er meget godt, at jeg ikke rigtigt førhen har tænkt fysik på tasterne, for det er da næsten pinligt, det her. ;)xD (Ej, bare for sjov; jeg er egentligt ikke pinligt berørt over det, men det er nu lidt sjovt, at jeg har skulle bruge så lang tid på at genfinde en bevisstrategi, som jeg egentligt allerede havde genfundet i denne periode nu her, men som jeg så kom væk fra igen. x)xD)) (..Og jeg kan da godt mærke, at jeg er lidt rusten på nogle ting. ;)xD) ...Nå, jeg kan ikke mere for i dag, så jeg vil holde fri, og så bare lige summe lidt over spørgsmålet til i morgen.
%%(06.12.21) Okay, i går aftes kom jeg endelig frem til argumentet igen, og kom vist også på en god måde at formulere det på. Hvis man nemlig starter med vilkårlig start-tilstand for V og A_\parallel, så har man i første omgang en fase på denne i henhold til løsningsdomænet. Denne fase svarer til \exp(-i phi(\sum x_{fermions})), hvor denne phi er identisk med den, man skal lægge til nul- ((V=0, A_\parallel=0)-) tilstanden for at komme til den vilkårlige tilstand. Når vi specifikt lægger en phi til nul-tilstanden med \square^2 phi = 0, så vil den transformerede vilkårlige tilstand (som jeg har kaldt (V', A')) blive til en slut-tilstand, som svarer til den transformerede nul-tilstand, bare centreret om et andet punkt, og nu med en fase på, der afhænger af phi(\sum x_{fermions}_{fin}). Min argumentations-idé (hvilket jeg jo nok i sidste ende bare oversætter til matematik, hvis jeg kan) er så, at man ser på en transformeret base af slut-tilstanden, så slut-fasen, der afhænger af phi(\sum x_{fermions}_{fin}) altid bare går ud med denne base-transformation. Nå, hvis vi så går lidt tilbage, er vi altså frie til at lægge en \square^2 phi = 0 -bølge til nul-tilstanden, hvis.. Hm, \emph{skal} jeg mon ikke transformere basen i begge ender for det mest overskuelige argument?.. Tja nej, måske skal jeg egentligt bare argumentere med, at slut-tilstanden bliver den samme bare forskudt og med en fase på \exp(-i (phi(X_{fin}) - phi(X_{init})), men at man så i første omgang også samler \exp(-i phi(X_{init})) op pga., hvordan den samlede start-tilstand ser ud (og altså pga. løsningsdomænet). Så i sidste ende for man en slut-tilstand, der bare er forskudt (i henhold til (V', A') og til \square^2 phi=0 -bølgen (som også afhænger af (V', A'))) og som altså får en fase på sig oveni, lig \exp(-i phi(X_{fin})) (fordi de to andre faser spiste hinanden), hvor phi er i form af \square^2 phi=0 -bølgen. Og når jeg siger "centreret," så behøver det altså slet ikke at være der, hvor slut-tilstanden er størst omkring eller noget. Vi kan med andre ord vælge nul-tilstandens "modpart" frit, og kan altså vælge (V=0, A=0) her også. Det vil sige at dette "center-punkt" vil forskydes med phi-bølgen (og kun denne) i det Lo.-transformerede tilstandsrum også. Dette vil sige at den samlede sluttilstand vil være en sum, eller et integrale rettere, af en masse næsten identiske tilstande bare centrerede om forskellige punkter og med en fase hver især som den eneste forskel, som er lig \exp(-i phi(X_{init})), hvor \partial^\mu phi = A^\mu = "centerpunktet" for det pågældende slut-tilstands-bidrag. Det kan godt være, at jeg bare vil argumentere matematisk, men intuitivt kan man så herfra argumentere med, at vi kan lave et baseskift af sluttilstanden, som går fra løsningsdomænet og til \hat P_V = \hat P_{A_\parallel} = 0 -løsninger, hvorved dette så for alle slut-tilstands-bidrag vil spise vores \exp(-i phi(X_{init})). Så nu har vi altså bare en masse identiske slut-tilstande, bare centrerede om forskellige (V, A_\parallel)-punkter, lagt sammen til den endelige slut-tilstand (efter Lorentz-transformationen). Og så skal man bare lige vise, at determinanten af Jacobianten fra phi_{init} til phi_{fin} er 1 for sådanne \square^2 phi = 0 -bølger, og herved får man så, at slut-tilstands-bidragenes "centerpunkter" vil være fordelt jævnt over hele tilstandsrummet, og herved får man, når man summer/integrerer dem samme, en samlet slut-tilstand, der ikke afhænger af V eller af A_parallel. Til sidst skal vi så lige transformere slut-basen tilbage, og så opnår vi altså en løsning indeholdt i vores løsningsdomæne. Fedt! Rart endelig at have styr på den del af beviset (det tog godt nok lidt længere, end jeg havde regnet med ^^). ^^ 
%%Nu tror jeg sådan set bare, jeg vil udkommentere hele denne brainstorm --- ikke at jeg skammer mig over den; slet ikke --- men bare så guldkornene kommer til at stå lidt mere kompakt i denne sektion for den renderede tekst (og så en læser ikke unødvendigt lokkes til at læse noget indviklet rod uden at det betaler sig --- ikke at jeg generelt har været god til at sørge for dette, men jeg har da prøvet lidt hist og her, haha.) Og så vil jeg altså skrive argumentet ind i den renderede tekst der. Jeg tror dog lige jeg holder en pause pt. og lige skriver om nogle andre ting først.



%... %Hm, jeg tror egentligt bare, jeg venter med at skrive det her ind som renderet tekst, til jeg alligevel skal brushe op på det, når jeg skal skrive argumentet i udgivelsen. ..Hm, jeg kan jo lige indsætte kommentar-teksten i mellemtiden:

%(10.12.21) Okay, min plan var/er at udkommentere ovenstående par paragrafer, hvor jeg alligevel ikke når nogen vegne, og så indsætte det argument, jeg nu er kommet frem til (og kom frem til d. 6/12), men indtil videre tager jeg bare lige og kopierer mine noter fra kommentarerne ind her: ...Hm, nej det er nærmest mere besværligt..


*(10.12.21) Okay, jeg har nogle nu udkommenterede noter fra d.\ 4.\ til den 6.\ i 12., hvor jeg brainstormer mig frem til mit ``$\square^2 \phi = 0$ -argument'' (og hvor jeg ender med at finde frem til svaret ude i kommentarerne). Men nu skriver jeg lige resultatet her i stedet, og så kan man jo bare læse disse udkommenterede noter, hvis man vil se processen. :) I argumentet ser vi på, hvad der sker for hvert felt-integrale, hvor $V$ og $A_\parallel$ holdes konstante, og hvor vi altså starter med at integrere over alle de andre frihedsgrader først. Nå ja, og integralet er altså fra ét inertialsystem ind til et andet, så vi integrerer altså imellem to (rumlige) hyperplaner i rumtiden. Vores start-tilstand er så en vis foton-fermion-tilstand (dog med fermionerne i en basistilstand fra den rumlige basis), der kun afviger med en fase for forskellige $V$ og $A_\parallel$ i henhold til $\hat \Pi_V=0$ og $\hat \Pi_{A_{\parallel\,\boldsymbol k\,s}}=q/|\boldsymbol k| \times \sum_{\boldsymbol x\in X}(\exp(i \boldsymbol k \cdot \boldsymbol x) + s \exp(-i  \boldsymbol k \cdot \boldsymbol x))/2$, hvor $X$ er mængden af positioner i den pågældende (rumlige) fermion-basistilstand. 
Lad os så starte med at betragte en start-tilstand, hvor $V$ og $A_\parallel$ er 0 overalt. Denne vil så efter sti-/felt-integralet sendes ind i en slut-tilstand, som vi kan kalde %...
\ldots Hm, jeg kan mærke, at jeg ikke helt gider dette, så jeg tror måske jeg udsætter det lidt alligevel.\,. (Og indtil videre står argumentet altså også lige ovenfor her i kildeteksten som en udkommenteret note.)


*(29.12.21) Her var det meningen, at jeg ville udkommentere noget af den ovenstående renderede tekst her og skrive en ny tekst, som forklarer mit ``$\square^2 \phi = 0$ -argument,'' men det dropper jeg altså bare nu, og så må man indtil videre (indtil jeg for skrevet noterne til ``first draft'') bare læse det i omtalte brainstorm-noter ude i kommentarerne.


(06.12.21) Okay, jeg har i skrivende stund lige færdiggjort brainstormen omkring det, der bliver den ovenstående paragraf, men som jeg altså ikke lige pt.\ har skrevet endnu. Jeg er nemlig ivrig efter at skrive (her) om, hvordan det står til med mine Yukawa-tanker. Jeg har tænkt lidt mere over det, og jeg tror heller ikke umiddelbart at min ``mulige løsning'' holder, heller ikke bare for én partikel.\footnote{Og dette er selvfølgelig klart nok; det er \emph{klart} det mest sandsynlige udfald, for det ville virkeligt være usandsynligt, hvis ikke andre så havde fundet samme løsning, og at dette så ville have været nævnt i læretekster omkring Yukawa-teorien. Men dette kan man på den anden side sige om en masse ting, og sådan arbejder jeg bare generelt ikke: Jeg lader mig nærmest aldrig holde mig tilbage, fordi der for en mulig opdagelse eller idé ``må være andre i så fald, der har tænkt/opdaget det samme.'' For ellers var jeg simpelthen ikke kommet nogen vegne med noget, og desuden lærer jeg altid en masse (og har lært så meget gennem tiden, at det nærmest ikke er til at forholde sig til, hvem jeg havde været foruden denne tilgang) ved at prøve kræfter med disse (store) ting. Så jeg har absolut på ingen måde lyst til at ændre denne tilgang --- også selvom jeg på papiret kan og har kunne spare en masse tid og kræfter; jo, men dette er bare langt fra hele udregningen.} Men tilgengæld er jeg så kommet til at tænke på en anden ting, og det er, at man jo næsten (7, 9, 13) må kunne komme helt vildt langt med perturbationsteori.\,.\,! For når $k$-spacingen går imod 0, så går koblingen til den enkelte foton-$k$-tilstand også gå mod nul. Og jeg tænker altså lidt: Selv hvis man af en eller anden grund ikke så bare kan bruge første-ordens-perturbationen nærmest som om, det var den eksakte løsning (hvis man altså kan vise, at den eksakte løsning vil nærme sig denne, når $\Delta k \to 0$), så må det i så fald næsten du med anden-ordens-pertrubation (\ldots eller trejde; hvad ved jeg?). Og hvis dette virkeligt passer (hvad der dog godt kan være nogle ting i vejen for), så vil det jo virkeligt være fantastisk! % :D ..(7, 9, 13..)
Det ville så faktisk slet ikke undre mig, hvis der kom et Yukawa-agtigt potentiale, men som jeg ser det umiddelbart, bør dette så næsten blive tiltrækkende og ikke frastødende. Samtidigt har jeg svært ved at forestille mig, at det bliver lig $1/4\pi r$, især ikke for QED-teorien (hvor man måske nok bare lige sørger for at ignorere par-produktion og positroner i første omgang). Og tænk(!), 7, 9, 13, hvis nu det gav et lille negativt Coulomb-/Yukawa-potentiale, således at vi måske i virkeligheden gør og ser en $e$, der er en lille smule mindre end den egentlige, ``bare'' $e$ (eller $q_{\mathrm{coupling}}$ rettere).\,.\,! For så vidt jeg ved er den eneste ændring til Coulomb-potentialet ifølge gængs QED-feltteori-lærdom en delta-funktion som følge af par-produktion (hvad der jo i øvrigt alt andet end lige stadig kan være i min teori også). Og hvis vi så går og ser en anden $e$ end den bare $e$ (/$q$) i teorien, jamen så kunne det jo muligvis (7, 9, 13!) forklare, hvorfor den gyromagnetiske ratio for elektroner er lidt over 2. Så ja, det må jeg virkeligt tage og regne på, når jeg for tid! Og lad mig lige i øvrigt pointere, at en sådan ``ændring af den bare $e$'' ikke vil føre til, at $e$ vil se stærkere ud for $E$-felterne i forhold til for $B$-felterne. Godt nok kan man sige, at der stadig vil være den bare $e$ stående i koblingen mellem fotoner og elektroner i Hamilton-operatoren, og så kunne man jo umiddelbart tænke, at dette så vil få $B$-felterne til at blive kraftigere sammenlinget med $E$-felterne (f.eks.\ hvis man måler $E$ af et klassisk Coulomb-felt fra en ladet sfære), men dette kan ikke passe, for så ville vores klassiske verden (altså når vi zoomer ud til vores niveau, hvor tingene ser klassisk ud) tydeligvis ikke kunne være Lorentz-invariant. Og dette er den jo, og det vil den også nødvendigvis være, hvis den er beskrevet ved en Lorentz-invariant kvanteteori. Men der er dog så vidt vi ved (for ellers ville man jo måbe over de eksperimentielle resultater) ingen ting i vejen for, at den gyromagnetiske ratio ikke kan have en lidt\ldots ``ikke-ballanceret'' værdi, kan vi sige, så her kan man altså sagtens forvente, at en reducering i styrken af Coulomb-feltet fra elektroner set i forhold til, hvad koblingskonstanten er i teorien, godt kan ændre den gyromagnetiske ratio, vi så observerer. (.\,.\,!\,!\,.\,.) % Og 7, 9, 13 igen.. ^^:D
Og uanset hvad, så ville det også bare være så kæmpe stort, hvis man virkeligt kan bruge perturbationsteori så effektivt på QED-teorien, som jeg håber på. Det ville være sådan et effektivt værktøj til at besvare spørgsmål omkring teorien! Ja, så selv hvis ikke min ovenstående ønsketænkning her holder, men at ønsketænkningen omkring effektiviteten af perturbationsregningen dog stadig holder, jamen så vil det stadig bare være et spørgsmål om (relativt kort (7, 9, 13)) tid, før man kan regne sig frem til, hvad forudsigelserne er --- især for sådan en ting som den gyromagnetiske ratio, som jo nok ikke bør afhænge så meget af par-produktionen.\footnote{Den kan dog i princippet godt, som jeg ser det, afhænge af, hvordan de tilsyneladende 2-spinorer så i virkeligheden kommer fra Dirac-4-spinorer, bare hvor elektron-til-positron-overgangene alt andet end lige vil kvæles af, at positronernes hæve-operator og omfortolkes som en sænke-operator (og derfor alt andet end lige vil forsvinde i høj grad i udregningerne).} Så ja, jeg er altså virkeligt spændt på, når jeg får tid på at regne på noget af alt dette.(!) 

Nå jo, og en anden lille ting er, at jeg da også helt klart på et tidspunkt lige må gennemgå den gængse QED-Hamiltonian, for det må da næsten være sådan, at man får noget forskelligt, hvis man Trotter-ekspanderer denne (på samme måde som jeg gør det, når/hvis jeg skal argumentere for, at Dirac-teorien på et klassisk $A^\mu$-felt stadig er Lorentz-invariant, efter at man har konjugeret positron-$\psi$'erne.) og så udregner resultatet i to forskellige inertialsystemer. De må jeg lige få set på engang. For dette ville så netop være fremgangsmåden, hvis man skal vise, at denne teori ikke er Lorentz-invariant; man kunne også godt gå baglæns af min vej og vise, at Lagrange-funktionen ikke er Lorentz-invariant, men dette beviser sådan set ikke nødvendigvis noget i sig selv. 


(07.12.21) Det er tidlig eftermiddag nu og disse noter kommer lidt i forlængelse af noget, jeg har skrevet på papir, men nu er jeg lige løbet tør for plads på det ark, og jeg er også kommet frem til nogle ting, jeg alligevel skal skrive her, så nu har jeg altså lige bevæget mig herind. Jeg kom nemlig til at tænke lidt videre over perturbationsregningen i går aftes og jeg kom frem til, at ligningen måske er separabel, hvis man betragter et fast antal fermioner/elektroner samt et vilkårligt antal foton-$k$-tilstande oveni. Så hvis man kan løse problemet med én foton-($k$-)tilstand kan man med andre ord (hvis det er rigtigt) løse det for et arbitrært antal foton-tilstande. Jeg kom så i tanke om, at dette ikke nødvendigvis alligevel holder generelt for en Yukawa-teori, for jeg brugte i mine overvejelser, at den kinetiske del af H, hvad fermionerne angår, kunne distribueres ud over faktorene i et produkt, hvilket $\hat p$ nemlig kan, men $\hat p^2$ kan det ikke. Men i QED-teorien er det nemlig $\hat p$, der indgår, og så vidt jeg kan se, så er den kinetiske operator her distributiv! Jeg stopper lige mig selv engang nu, for imens jeg skriver dette, kommer jeg jo så lige i tanke om massen; skal masse-delen ikke så også være distributiv, før det fungerer.\,.\,? .\,.Ah, nå ja, masse-delen er jo allerede distributiv, fordi den er et ydre operator-produkt! *(Hov, eller `-sum' retttere (tensor-sum)) Ja, never mind! Ok, så jeg har altså stadig mit samme håb. Her til formiddag har jeg så tænkt lidt om overordnet perturbationsregning for tilstandene i QED-teorien. Jeg kunne sige lidt forskellige ting om det, men det korte af det lange er bare, at man nok skal sørge for at være omhyggelig og tage alle de højere ordner med i betragtning, og umiddelbart har jeg endda fået det til, at f.eks.\ tredje orden af energien vil eksplodere, og så går det hele jo (måske) nok ikke rigtigt. Så her skal jeg altså være omhyggelig. Her lige omkring middag kom jeg så frem til, at ligningerne bare er separable (hvad jo på en måde også var mit forrige resultat, men jeg var for træt i går til at tænke det helt til ende); for vilkårligt mange foton-($k$-)tilstande og for vilkårligt mange elektroner i spil. Og ja, dette er bare helt vildt stort, hvis det passer.\,.\,!\,! For så vil elektronerne faktisk opføre sig nærmest som frie partikler! Dette er så på nær, hvis, og dette er altså så, hvad jeg har tænkt over nu her i mellemtiden, hvis man har f.eks.\ et Coulomb-potentiale i spil (hvad jeg jo netop har). For hvad sker der med disse (separable) løsninger, når vi booster en perturberet elektron-$k$-egentilstand til med en $k'$. Jo, når jeg lige tænker over det (og jeg har nemlig tænkt lidt over, hvordan den perturberede tilstand bliver (til første orden)), så får jeg umiddelbart (og altså bare i hovedet), at man får næsten den samme tilstand som den perturberede $(k+k')$-tilstand, bare de adderede tilstande fra perturbationen (overtone-tilstandene, har jeg nærmest lyst til at kalde dem) ikke lige akkurat for de rigtige amplituder på sig. *(Uh, og det er *(meget) vigtigt lige at nævne, at jeg betragter sinus-/cosinus-bølger for foton-tilstandene, så når jeg siger en foton-$k$-tilstand, så er det altså lidt en misnomer; vi snakker egentligt en foton-cos/sin-$k$-tilstand, hvor $k$ så ligger i det positive halvrum.) Og som jeg kan se det, sker forskellen så bare pga.\ elektronmassen, hvilket i så fald vil sige, at denne uoverensstemmelse går imod 0, når $k$ bliver stort nok, og jeg kunne faktisk sagtens forestille mig, at det så nemlig også vil integrere til noget endeligt. Og hvis det passer, så vil dette jo være enormt spændende! For så må man næsten kunne sætte sig ned og se på, hvad vi har, nemlig nogle i udgangspunktet nærmest frie elektroner, som så dog får lagt Coulomb-potentialer imellem sig ind over, og hvor der så altså lige kommer ekstra ting ind over (i hvert fald inden vi begynder at betragte positroner og par-produktioner alt for meget): Coulomb potentialet vil nemlig så ikke helt få sin $1/|k_2-k_1|^2$ -karakter i $k$-rummet, men vil for hver $(k_1, k_2)$-overgang få trukket noget fra, hvis man ser på overgangen mellem de to (perturberede) egentilstande. Og foruden egentilstand-til-egentilstand-overgangene (hvor vi altså med egentilstand snakker om de perturberede tilstande, men inden at Coulomb-delen kommer ind i billedet), så giver det resulterende, ændrede Coulomb-potential altså nu også nogle overgange, der exciterer en foton (eller hvad der ift.\ de perturberede tilstande vil komme til at svare til en (fri) foton (for de perturberede egentilstande har jo f.eks.\ tilsyneladende også fotoner exciteret i sig, men disse vil jo ikke få fysisk karakter ligesom frie fotoner; de vil nemlig bare følge med elektronerne, så at sige --- men det vil dem, der exciteres af $V_{\mathrm{Coulomb}}$ altså nok nemlig ikke)). Så ja, alt sammen vildt spændende.\,.\,!\,!


(08.12.21) Okay, jeg var alt, alt for hurtig i går. For det første, så kan man ikke ignorere, at der er flere spin-tilstande i Dirac-ligningen, hvilket jeg også kom frem til i går aftes i sidste ende. Men jeg har i det hele taget også været alt, alt for hurtig, da jeg kom frem til, at ligningen er separabel, når det kommer til fermion-løsningerne (for ja, de deler jo det $A^\mu$-felt, de hver især er entangled med). Og hvis den kinetiske energi i Yukawa-teorien var lig $k$ for de frie/bare fermioner, så opnår man ingen gang, hvad jeg ønsketænkte her til sidst i går, nemlig at de kommer til at opføre sig som frie partikler, selv på trods af Yukawa-leddet. Det \emph{eneste}, man opnår (måske; jeg kan også sagtens stadig tage fejl, hvad dette angår), er, at løsningen bliver separabel ift., hvor mange foton-$k$-tilstande, man indfører udregningen. Med andre ord bør udregningen altså kunne gøres for hver foton-tilstand separat, hvis man f.eks.\ ser på to elektroner, og så kan man lægge (eller rettere gange) disse løsninger sammen til sidst til en endelig løsning. Dette er jo stadig lidt interessant, for for relativistiske partikler i Yukawa-teorien bliver energien tilnærmelsesvist $k$, når $k$ bliver stor nok, eller hvis man ser på tilstrækkeligt små fermion-masser. Men for at kunne sige noget interessant, så kræver det så stadigt, at jeg (eller andre) løser ligningen for en vilkårlig foton-$k$-tilstand (eller $k$-tilstandsrum, rettere; vi kigger på hele rummet af tilstande for $k$-fotonen på én gang). .\,.\,Hm ah, men det store ved sådan en konklusion er da for øvrigt så også, at man netop så absolut \emph{kan} bruge perturbationsregning.\,.\,!\,.\,. Ah ja, så jeg kan måske sagtens regne på Yukawa-teorien for masseløse fermioner og konkludere noget interessant (eller forhåbentligt interessant; det kan jo også være, at jeg bare for et $V=1/4\pi r$ -potential). Jamen det må jeg da se på at gøre så.

Hm, lad mig lige tænke en anelse mere over mulighederne i dag så, når det kommer til Yukawa-halløjet (synes nu også jeg har fortjent en lidt rolig dag efterhånden.\,. også selvom i går egentligt også var ret stille og rolig, når det kommer til stykket.\,.), men ellers bør jeg dog prøve at holde, hvad jeg har lovet mig selv og ikke blive for opslugt. 

Nå, men noget andet er, at jeg her til middags kom på en muligvis ret god idé, når det kommer til at vise, at den gængse Hamilton-operator (hvilken man må kunne udlede baglæns fra $\hat S$, også selv $\hat H$ ikke er opskrevet i den lærebog eller tekst, man nu tager det fra) for QED \emph{ikke} er Lorentz-invariant. Idéen er at se på en meget stor kohærent (foton-)bølge (altså såsom en laser), og se hvordan fermionerne/elektronerne interagerer med denne. Så dette tror jeg altså, godt kan være værd at gøre, inden jeg udgiver teorien, så jeg måske kan slutte af med dette (foruden mit resultat omkring Yukawa-teorien, hvis jeg opnår et). 

\ldots\ Nå nej, det får man vel egentligt ikke noget ud af, altså det med at se på en kohærent bølge, for medmindre Coulomb-potentialet bør spille ind i udregningen, så bør det jo ikke ændre noget, hvis man ser bort fra det (og man bør altså stadig få et Lorentz-invariant resultat).\,. .\,.\,Ej ja, så tror jeg altså, jeg opgiver *(for nu i hvert fald) at prøve at regne på det (for ja, så er der nok en god grund til, at det har kunne gå under radaren, hvis den gængse Hamilton-operator (eller $\hat S$, om man hellere vil se på denne) rigtignok ikke er Lorentz-invariant, som jeg forudsiger). 

\ldots\ Uh, det er lige før jeg tror, at man må kunne finde frem til noget interessant, hvis man analyserer Yukawa-teorien først for $m=0$, og så ser hvad der sker, hvis man perturberer med et $\hat p^2$-led for fermionerne i Hamilton-operatoren. Som jeg umiddelbart lige tænker over det her, hvis man ser på $\exp(i \boldsymbol x \cdot \boldsymbol k)$-basen for fotonerne i stedet for på cos/sin-basen, så får man måske, at der ikke sker nogen ændring i første orden af $\hat p^2$-koblingskonstanten\ldots\ Men ja, det må jeg så lige regne på en gang, for selvom det hele dog bliver mere kompliceret (sandsynligvis.\,.) i Dirac-ligningen, så vil det bestemt stadig være arbejdet værd, hvis man kan vise, at man kan komme langt med perturbationsregning for Yukawa-teorien. 

\ldots Hov vent, man må da egentligt kunne argumentere på samme måde i Dirac-ligningen, eller hvad?\,!\,.\,. Hvad sker der, når $m\to 0$ i Dirac-ligningen (eller i QED-teorien rettere, men bare hvor vi dog stadig lige ignorerer antipartikler inklusiv par-produktioner)? Kan vi så også her separere løsningerne, når det kommer til forskellige foton-$k$'er (hvis vi ser på et fastholdt antal elektroner såsom to styks)?\,.\,. .\,.\,Hm nej, det går jo nok ikke så nemt her, fordi alle foton-$k$-kanalerne stadig her skal dele spinor-tilstand\ldots\ .\,.\,Hm, men der er noget om, at spinor-tilstandene også går imod noget simpelt, når massen går imod 0, men kan man bruge dette til noget? .\,.\,Hm, så må foton-interaktionen vist bare flippe ``helicity'en'' i denne grænse, er det noget med.\,. .\,.\,Tja, men det bringer mig vist ikke rigtigt videre.\,. \ldots Hm, hvis bare man kunne vise, at energi-egenværdierne af løsningerne til 2-elektron-1-$k$-foton-oscillator-rummet var ens for begge.\,. nå nej, der må blive fire med to elektroner, selv hvis vi regner dem for at være 2-spinorer hver især.\,. for alle fire resulterende spin-egentilstande (og altså at der er en fire-dobbelt degeneracy (kan ikke huske det danske ord) for grund-tilstands-løsningerne).\,. For så ville vi jo have frihed til at vælge en vilkårlig fire-spinor, hvortil løsninger så kan findes (antaget at $m\to 0$) for alle $k$-foton-oscillator-kanalerne (som så ville kunne ganges sammen til en samlet løsning).\,. .\,.\,Uh, hvis man ville vise dette, så handler det jo bare om at jagte en symmetri i ligningen.\,.\,! Det må jeg lige få set på.\,. .\,.\,Nå jo, det gav jo egentligt sig selv, men jeg tænkte nu nærmere på, kan man ikke så vise en sådan symmetri, ved bare (hvis man kan finde en) at vise en kommutation med en operator? Er dette ikke en mulig vej til en sådan konklusion (jeg er jo en anelse rusten, som man nok kan høre)? Lad os se, en kommutation med $\hat H$ fortæller, at der findes en base af samtidige egenvektorer for de to operatorer, så meget kan jeg da huske. Og kan dette så medføre en degeneracy ift.\ energierne?\,.\,. .\,.\,Ah ja, for det betyder jo, at vi kan transformere systemet i henhold til den givne symmetri, som operatoren genererer, uden at dette så ændre på en løsnings energi (og korrekthed). Og jeg ved jo at spin-operatorerne netop genererer spin-rotations-symmetrier, så der kunne altså være en mulighed her, hvis man kan vise en sådan kommutation! Hm, det må jeg da bestemt lige se på engang.\,. \textasciicircum\textasciicircum 


(10.12.21) Nå, nu tror jeg altså, jeg opgiver at tilføje mere end min teori til min udgivelse. Min idé til, hvordan man måske kunne løse Yukawa-teorien, holder ikke, selv ikke med $m\to 0$. Jeg kam nemlig til at tænke, at så ville energien gå imod $k=i\partial_{\ldots x? y?}$, men det holder jo selvfølgelig ikke i tre dimensioner. Så never mind alt det med Yukawa-teori osv.; jeg giver op på alt det for nu. I går fik jeg så også tænkt lidt videre over, hvad man kunne gøre med pertubationsregning (og måske så bruge på en måde, at vi er frie til at lade $\Delta k \to 0$ (svarende til $L \to 0$)). Men det bringer mig ikke umiddelbart noget. Det ser ud til, at man ville skulle bruge en masse kræfter på at opnå et svar --- og måske computerkræfter i virkeligheden. Så de tanker lægger jeg altså på hylden.

Der er så lige en anden ting, som jeg også kom i tanke om, og det er, at jeg i min ovenstående argumentering for Lorents-invariansen af min teori mangler et argument for, at vi kan approksimere $\hat H$ arbitrært godt med en diskret version (med begrænset $L$ og med begrænset $k$ (og altså med et ikke-infinitesimalt $\Delta k$ og et ultraviolet cut-off på $k$ med andre ord)). Det er nemt at argumentere for et begrænset $L$, for hvis denne begrænsning nærmer sig en Lorentz-invariant teori, når $\Delta k \to 0$, jamen så vil man for et tilstrækkeligt indre system få nøjagtigt den samme dynamik (når $\Delta k \to 0$) uanset $L$'s størrelse efter en hvis grænse, og dermed kan vi altså sende $L$ til uendeligt uden at det vil bryde Lorentz-invariansen. Hm, ja måske behøver man ingen gang at argumentere således (altså via lokaliteten af en Lorentz-invariant teori, hvis ikke det var klart, at det var det jeg mente), for hvis en forstørrelse af $L$ aldrig gør teorien mindre Lorentz-invariant (og altså ikke bryder den), så kan vi jo bare sende $L$ til uendeligt, punktum. Ja, det må nok være fint. Nå, og hvordan viser vi så, at vi kan lave et cut-off på $k$? Jeg tænkte lige kort over mit flux-argument, som jeg fandt på der i sommeren '19, men jeg er faktisk ikke umiddelbart sikker på, at det holder. Men som jeg også har lagt op til ellers, så kan man vist sagtens bare tage en Trotter-ekspansion af $\hat H$ (hvor vi bare kan lade $L$ være begrænset allerede, hvis det hjælper), og ved et argumentere for, at hvis integralet er endeligt, så må vi også kunne sætte en høj nok grænse på $|k|$ i integralet, uden at det ændrer resultatet, og dermed må der altså gælde, at for en given tilstand må vi altid kunne finde og sætte et tilstrækkeligt højt cut-off, uden at det ændre dynamikken. Og dermed bliver $\hat H$ ikke.\,. mindre Lotentz-invariant af at $k$ går mod $\infty$.\,. Hm, eller kan vi også springe dette argument over i virkeligheden, ligesom vi lige så, vi måske kan gøre for $L$ og $\Delta k$?\,.\,. .\,.\,Nå nej, humlen er, at vi jo lige bare helst skal vise, at dynamikken (/fysikken) konvergerer, når $L$ og $k_{\mathrm{max}}$ sendes mod uendeligt. Og her kan man altså bruge de to nævnte argumenter, nemlig om rummelig lokalitet og om, at for enhver given tilstand vil integralet fra Trotter-ekspansionen konvergere, når $k_{\mathrm{max}}\to \infty$. Fint.


(11.12.21) Åh, der er lige sket noget vildt. Okay, lad mig lige opsummere nogle ting, inden jeg kommer til det spændende. Jeg var egentligt totalt klar på at stå op og gå i gang med min web-sektion til min udgivelse (hvor jeg nu har fundet ud af, at jeg vil starte med en masse ultrakort beskrevne idéer (ultrakorte pitches, kan man sige)), men jeg fik alligevel lige tænkt over lidt fysik i går nat, og det sag lidt i tankerne til morges. Og så har det sneet så fint i går aftes og i nat, så jeg ville gerne nå lige at gå en tur i det, hvis nu det skulle regne eller smelte væk inden længe (hvad det nemlig godt kan, som jeg ser det --- læste også, det skulle blive en del varmere i morgen). Nå, men på turen kom jeg for det første til at tænke lidt over gængs QFT, og til dette vil jeg gerne lige sige noget. Nu hvor jeg jo godt kan se, at man altid må kunne lave et ultraviolet cut-off, så er der egentligt rigtig meget regularisering i gængs QFT, der må give fin mening, når det kommer til stykket (selvom jeg mangler nu stadig en bedre forklaring på dim-reg, hvis jeg skal købe dette (men en sådan kan nu sagtens eksistere og gør det sikkert; jeg er jo bare så skeptisk af natur, så min tilgang er selvfølgelig dog stadigvæk: jeg vil se forklaringen først, før jeg tror det)). Så det største problem er vel egentligt bare det med, at man jo laver perturbationsregning ved at se på de bare start- og slut-tilstande. Men da de egentlige (perturberede) egentilstande jo godt kan afvige væsentligt fra de bare, så vil der altid være en mulig fejl på alle sådanne udregninger, der kun er begrænset af $2\Delta\psi$, hvor $\Delta\psi$ er forskellen på en bare og den egentlige, fysiske tilstand, som jo vil være den man i sidste ende måler i et laboratorie. Og ja, jeg har for et par dage siden tænkt på, om ikke jeg kunne finde et godt eksempel, hvor man kan vise, at dette ikke holder, men så er jeg kommet i tanke om, at der allerede er $\varphi^3$-teorien lige foran næsen på os, hvor alle og enhver ved, at denne perturbationstilgang ikke må holde, når det kommer til stykket. Så ja, de fleste er nok godt et eller andet sted klar over det.\,. eller nogen er i hvert fald måske. Og ja, det er egentligt også forståeligt nok, for hvad skal man ellers gøre? Man kan jo ikke umiddelbart rigtigt gøre andet, når man ikke kender de faktiske egentilstande (andet end selvfølgelig så at prøve at lægge mere arbejde i at finde disse.\,. ja, det kunne måske godt være en ting, der var værd at påpege). Men ok. Og så er jeg lidt kommet frem til, at jeg nok bør tage på et tidspunkt (men selvfølgelig først når jeg har tid! (.\,.\,hvis ikke jeg kan få klemt det ind om aftnerne.\,.)) at læse op på gængs QFT igen, og prøve så at se det med lidt nye øjne. *(Nå ja, og særligt bør jeg prøve at forstå den gængse forklaring på den gyromagnetiske ratio noget bedre, hvilket jeg jo længe har været (og stadig er, må jeg sige) meget skeptisk overfor (men ja, denne skepsis vil være bedre underbygget, hvis jeg rent faktisk satte mig ned og prøvede at forstå argumentet bedre).) Ok. 
Inden jeg kommer til det spændende, vil jeg også lige sige.\,. nå ja, måske vil jeg også sige én eller to ting mere, men jeg vil for det første gerne sige, at det bare ville være så rart, hvis lige jeg kunne finde bare selv en enkelt lille ting at sige udover: ``her er en QED-teori med et Coulomb-potentiale forskelligt fra den gængse (men som den gængse alligevel bare vil tage med som en ``approksimation,'' når folk skal regne på ting, hvor det bliver betydende (og jeg mener så altså bare, at det kommer fra et direkte led i $\hat H$, og altså ikke bare er en approksimation til, hvordan to ladede fermioner generelt interagere med hinanden via fotoner)), som jeg kan vise er Lorentz-invariant (i modsætning til, hvad man måske skulle tro).'' For hertil vil mange sagtens kunne spørge, ``er det virkeligt noget nyt,'' og så er det ikke så sikkert, at svaret, `ja,' kommer til at klinge så skarpt igennem. Så hvis jeg bare lige havde en lille forudsigelse, eller en lille ting, hvor jeg kunne sige, ``og her holder disse gængst accepterede argumenter altså ikke alligevel,'' så vil det bare booste udgivelsen \emph{så} meget. Og jeg har nemlig en tilsvarende, eller rettere ikke helt så stor men næsten, skepsis omkring gennemslagskraften af mit blockchain-angreb og -forsvar. For selvom nyheden her sikkert vil sprede sig, så er det dog ikke ensbetydende med, at mange folk vil tage at kigge på mine andre idéer. Men hvis jeg bare lige kunne finde den lille ting, der ville booste min fysik-udgivelse helt vildt meget, så kunne det være så dejligt, for så tror jeg, jeg sikkert ikke ville skulle kæmpe nær så meget om at få folks opmærksomhed efter at have udgivet det. Så naturligvis har jeg lige givet mig selv lov til lige at tage et par dage (som jo hurtigt bliver til flere og flere, men jeg har jo lovet mig selv at være vaks og stoppe mig selv, inden jeg bliver opslugt (hvad jeg jo nærmest også havde gjort her i går)), hvor jeg lige ser på, om ikke jeg lige kunne finde den lille ekstra ting, der kunne booste nyheden så meget mere. Og nu tror jeg faktisk måske, jeg lige fik den kæmpe store idé her på min gåtur, der ikke bare kan føre til en ``lille ekstra ting,'' men en kæmpe ekstra ting. Så jeg er bare \emph{så} glad for, at jeg valgte, ja, at gå den tur, men ikke mindst også valgte lige at tænke bare en anelse mere på fysikken i denne omgang.\,.\,!\,! Jeg er så spændt på at fortælle om det, men lad mig lige se to sekunder, hvad de to andre ting var, jeg måske gerne ville nævne.\,. 
Nå jo, for det første, så har jeg overvejet lidt på turen, hvad man mon skulle gøre for at finde en egentilstand (til en fri elektron med foton-interaktionen), og/eller for at få perturbationsregningen til at give mening. Men tanke var så, at mon ikke løsningen kommer til på en måde at indebære, at de foton-elektron-del-tilstandene på en måde får en forøget energi, der vokser stærkere end $|k|$ for store $k$ (og så pga.\ at denne energi nemlig selv er ``perturberet'' af interaktionen). For hvis disse del-tilstande kan komme til at få nogle hurtigere egensvingninger, så kan tilstandene med lavere $|k|$ godt have overgange frem og tilbage med disse, uden at dette vil medføre et flux fra lav-$|k|$-tilstande og ud til alle (de uendeligt mange) høj-$|k|$-tilstande. Og hvis man så skulle prøve at løse dette problem, så kunne man så måske starte med at gætte en perturbation af energien for høj-$k$-tilstandene og se, om dette gæt vil være i overensstemmelse med sig selv, hvilket så nemlig vil sige, at høj-$|k|$-tilstandene, grundet samme perturbation af høj-$k$-energien, heller ikke vil bløde ud / fluxe ud til andre høj-$k$-tilstande selv (og så med én mere foton i sig for hver gang). Og for at finde dette gæt, ville jeg så bare i første omgang ignorere $m$, for når vi først ser på den perturberede energi fra en tilstrækkelig energirig tilstand, jamen hvad er så sandsynligheden (eller rettere amplituden (.\,.\,eller overgangsfluxen)) for, at den overgår til en lav-energi-tilstand, hvor $m$ bliver betydende igen; den må være lav. Dette kan så måske gøre ligningerne lidt pænere, så man nemmere kan gætte en løsning. Så ja, dette må jeg (eller andre for den sags skyld) jo se lidt på engang. .\,.\,Hm, jeg kan ikke lige komme på, hvad nr.\ to ting var.\,. 
.\,.\,Hm, det var nok ikke det her, men jeg vil også gerne lige huske mig selv på, at jeg gerne skal inkludere i min udgivelse, at der måske også kan være andre valgmuligheder, når det kommer til spørgsmålet om, hvorvidt man skal konjugere (fra huller til positroner) før eller efter, man går fra felt-integrale til Hamilton-operator. Ok, lad os bare sige, at det var det for nu. Og så vil jeg nemlig rigtig gerne forklare om, hvad jeg lige har indset.(!) Det kommer her:

Jeg har indset, for det første at den gyromagnetiske ratio godt kan komme direkte fra Coloumb-interaktioner og Dirac-partikler, fordi dele af $(k_1, k_2) \to (k_1', k_2')$-overgangene så nu bliver til elektron-hul-overgange, hvilke nu vil ``brænde op i helvede,'' som min lektor, Anders Sørensen, altid sagde i Kvant1. Det er sjovt, for jeg har jo egentligt tænkt på dette før. Det slog mig bare på gåturen, hvor stort denne formodning egentlig er.\,. og måske slog det mig også først rigtigt nu, at jedet måske ville være mere nemt at regne på, end jeg lidt ellers havde tænkt. Om ikke andet, så slog det mig i hvert fald, at jeg bestemt ikke var færdig med at lede efter et ekstra resultat oven i mit eksisterende, og at jeg bestemt bør prøve at se, om jeg kan regne mig frem til en korrektion herved at Coloumb-potentialet (for de forskellige spin-tilstande). Jeg kom så hjem her og slog op i QFT-for-(``gifted'')-amatører-bog på Dirac-siden og tænkte på, hvad man mon kunne udlede. Og efter lidt tid slog det mig så, og dette har jeg ikke nævnt endnu, at man i første omgang ikke bare vil få et ændret potentiale, man man vil faktisk få en spin-spin-interaktion ud af det! (Altså en $\hat{\boldsymbol{S}} \cdot \hat{\boldsymbol{S}}$-interaktion (som jeg lige hurtigt kan se / gisne om --- men selve argumentet, selvom det nok skal holde, bliver nu en anelse svært at fuldføre (men den tid, den sorg)).) Og med det samme kunne jeg så også tænke mig til, at der lige netop må blive e spin-uafhængig korrektion, der for den effekt at mindske størrelsen af potentialet, hvilket lige netop vil føre til en forskel på den umiddelbart iagttagede $e$ og den teoretiske (bare) $e$, hvilket altså netop vil give en gyromagnetisk ratio forskelligt fra 2. Og denne korrektion bliver så, så vidt jeg lige hurtigt kan se, endda meget lettere at udregne! Og argumentet/argumenterne for at der vil komme disse korrektioner er bare så stærke.\,.\,! (7, 9, 13!) Ja, jeg skal selvfølgelig lige regne på det ordentligt engang, før jeg overhovedet kan sige noget, men jeg kan altså ikke lade være alligevel med allerede at være totalt blæst bagover af det her.\,.\,! Ikke mindst, fordi jeg jo lige havde nået at give op, og så lander der bare med det samme sådan en bombe, bare fordi jeg lige alligevel tænker lidt mere over det hele. .\,.\,Så ja, nu er jeg altså bare vildt spændt på, hvad mine udregninger vil give mig, når jeg får dem lavet.\,.\,!

Nå ja, lad mig hurtigt lige forklare min strategi for udregningerne. Idéen er først at ekspandere Dirac-løsningerne (og her snakker vi altså bare Dirac-teorien uden et $A^\mu$-felt) til anden orden, og så se på, hvordan det ændrer $\tilde V_{(k_1, k_2) \to (k_1', k_2')}$ (som førhen bare var lig $1/\Delta k^2$ (hm, måske burde jeg forresten bruge $p$ og ikke $k$ om elektronernes impuls.\,.)), når altså nu at hver af disse overgange kan opdeles i en elektron-til-elektron- og en elektron-til-positron-del, og hvor vi så altså kun bør beholde elektron-til-elektron-delen (som derfor vil få en mindre amplitude end bare $1/\Delta k^2$). Jeg tror så, at den spin-uafhængige korrektion, som kan give anledning til en effektiv gyromagnetisk ratio forskelligt fra 2, kommer fra 2.-ordensledet, hvilket måske så faktisk (men det ved jeg nu ikke helt endnu) vil give en korrektion for hver overgang, der kun afhænger af $\Delta k$.\,. Tja, det er i hvert fald værd at håbe på. Det er sikkert en kort udregning, men jeg udskyder den altså lige lidt. Men ja, hvis nu der er en korrektion proportionelt med $\Delta k^2$, så vil dette jo give et Coulomb-potentiale tilbage, bare ændret fra den bare version (og altså med mindre størrelse). Og når det kommer til $\hat{\boldsymbol{S}} \cdot \hat{\boldsymbol{S}}$-interaktionen, så handler det om at se på, hvad 1.-ordenskorrektionerne giver for de fire sammensatte spin-tilstande, nemlig $(\ket{\uparrow\downarrow} - \ket{\downarrow\uparrow})/\sqrt{2}$, $(\ket{\uparrow\downarrow} + \ket{\downarrow\uparrow})/\sqrt{2}$, $\ket{\uparrow\uparrow}$ og $\ket{\downarrow\downarrow}$. Og grunden jeg indtil videre har til at tro, at dette vil give en spin-spin-interaktion (hvad end denne så lige er proportionel med $\hat{\boldsymbol{S}} \cdot \hat{\boldsymbol{S}}$ eller ej.\,.), er et lidt løst argument om, at for en elektron med en afstand, $\boldsymbol r$, fra en anden må den mest betydende del af $\tilde V$ være den, hvor $(\boldsymbol{k}_1'-\boldsymbol{k}_1)$ har samme retning som $\boldsymbol{r}$. Og hvis man så kigge på første-ordens-korrektionerne, så vil man altså, mener jeg, sikkert finde, at to elektroner med samme spin-retning langs $\boldsymbol{r}$ (hvad man jo f.eks.\ kunne tage til at være langs $z$-aksen), får en anden korrektion end dem med forskelligt spin. Og med det løse argument om, at lige netop disse $\boldsymbol k$-overgange vil være mest betydende for potentialet, så kan man altså lidt løst argumentere for, at energien må være ændret på nogenlunde tilsvarende vis (altså ift.\ hvad korrektionen var her). Så det er altså min strategi for udregningerne indtil videre. \texttt{:D}\textasciicircum\textasciicircum


(12.12.21) Nå øv, der var nogle ting, jeg havde overset. For det første tænkte jeg\ldots\ ja, jeg fik ikke tænkt så grundigt over, hvordan en amplitude-ændring ændrer $1/k^2$-potentialet, og på en eller anden måde fik jeg altså slet ikke tænkt over (indtil i går nat), at man jo skal \emph{gange} denne ændring på. Og så er det lige præcis en konstant ændring, der skal bruges, hvis man skal håbe på et nyt $1/k^2$-potential med justeret kobling. Men anyway, der er stadig et par interessante ting at se på, når det kommer til alt dette, og ikke mindst (og det har jeg det på en måde lidt blandet med \texttt{;)}) har det også ledt til, at jeg har opdaget, at jeg bliver nødt til at se nærmere på, hvordan man viser, at $\exp(-i \varphi)$-løsningen stadig giver ligeså god mening, når man har konjugeret positronerne, og jeg bør så også se på mere præcist, hvad potentialet imellem dem så bliver. Så det vil jeg altså lige overveje nu her\ldots\ 


(13.12.21) Jeg har nogle ting fra i går, som jeg skal have skrevet ind her, men jeg vil gerne lige starte på mine web-noter, så det er ikke sikkert jeg får det gjort i dag.
%Men det handler altså for det første om, at jeg har set på de rigtige løsninger til Dirac, og kan se at de opfylder noget meget rart. Og så er jeg kommet frem til noget vigtigt: Jeg tror man skal konjugere lige netop positions-egen-spinor-tilstandene. Og ja, jeg kan se, at jeg faktisk har overset denne detalje om, hvad man skal gøre her helt præcist i mit Lo.-inv.-bevis (ovenfor) indtil nu.
%Andre stikord er: Har set på, hvad der sker når V booster. Heraf må man næsten få spin-orbit kobling, men ifølge Abers er det et gængs resultat at få spin-orbit-kobling fra Dirac + Coulomb-felt, så dette vil der ikke være den store tjubang i. Jeg har tænkt lidt over, hvorfor vi ignorerer protonens spin-orbit-kobling.. Og tænkte så også lige, gad vide om folk husker at bruge den reducerede masse i stedet for elektronens (frie) masse?.. Jeg har overvejet, om boost-argumentet også kunne føre til en reduceret V, men umiddelbart så må.. hov, vent.. Kunne effekten faktisk ikke.. Nå nej.. Eller jo.. Hvis nu p_z væk fra V's centrum vil være propoertionelt med \nabla V.. Hm, eller vil den egentligt ikke være \propto 1/r for en egentilstand (og l=0)?.. ..Hm.. Tjo, men i sidste ende får jeg nok stadig ikke en korrektion proportionelt med 1/k^2, som jo kunne forklare g.m.-ratio, måske. Det vil nok nærmere blive noget a la Darwin-termet, og dette er jo også forklaret ved analyse af Dirac + Coulomb.. Ja, og jeg giver altså op. Det sjove, og det er et andet stikord, er også, at jeg på en måde har fået mere tiltro til idéen om at se på de bare start- og slut-tilstande, og på gængs QFT i det hele taget. Og det er sjovt nok min teori, der ligesom får det hele til at give meget mere mening (frem for det modsatte, som jeg havde forestillet mig! xD) Men ja, det er vist endeligt blevet tid til at være lidt rationel (hvad jeg jo i lige netop, når det kommer til sådan noget som dette, prøver lidt at undgå) og så sige: jo, folkene har sikkert ret i meget af det. Det er så bare lige Yukawa-teori-argumentet om, at fotoner vil skabe et Coulomb-felt, hvor jeg så kan bidrage og sige, hey, det behøves ikke! (Og jeg kan komme med et bedre framework omkring, hvad der skal til for at en teori er Lorentz-invariant.. Hvilket kunne føre til ændringer i andre teorier.. Uh, dette er da en vild pointe, egentligt!).. Ej, ja! Så min teori er altså ikke bare en.. halvsnoppet.. se-nu-her-rettelse til det gængse, men hvis den er sand, så antyder den også, at der måske kan opdages interessante ting om andre gauge-teorier (via teoretisk arbejde)..! (Så det kunne føre til et midlertidigt brud på den stilstand, der ellers hersker, hvor det hedder sig at teoretisk fysik venter på den eksperimentielle fysik..!) Ej, den pointe må jeg fremhæve! Fedt, at jeg indså den! Men ja, lige for at slutte af ellers, så må jeg lidt overgive mig (om ikke andet så for nu), og sige, at, på nær Yukawa-teori-argumentet, hvor jeg jo tror på det mere rene Coulomb-potentiale i stedet, så har gængs QFT sikkert ret i mange forudsigelser (for hvis de egentlige tilstande i høj grad ikke er langt fra at være parallele med de bare egentilstande, så virker udgangspunktet for path-integralerne, og meget af den lidt uldne matematik ellers vil jo nok vise sig at holde, idet man jo i det mindste altid må kunne regularisere via et ultraviolet cut-off eller tilsvarende). ..Nå ja, og i øvrigt kommer spin-orbit-koblingen så heller ikke fra foton-elektron-interaktionen, det kan jeg også nævne. Jeg ville ønske, jeg kunne sige noget om spin, men jeg har lovet mig selv ikke at falde ned i et hul og blive opslugt af fysikken nu her, så alle sådanne udregninger og overvejelser må nok vente til efter min udgivelse. ..Hov, men er det ikke netop Darwin-termet, der giver spin-spin-interaktionen for grundtilstanden? Og dette kommer jo også (ifølge Abers, E. S.) fra Dirac-ligningen (måske + Coulomb-felt). I så fald kan man jo allerede være rimelig rolig, for så får vi faktisk, at alle de mest gængse spin-interaktioner forklares uden foton-elektron-interaktionen..! 
%... Åh, nu fik jeg lige taget et lidt mere ordentligt kig på, den sektion i Lancaster & Blundell (hvor jeg kan mærke at jeg nu endelig er ved at være klar til tage de gængse ting seriøst), og nu fik jeg endelig givet mig selv tid til at forstå, at udregningen egentligt bare handler om at se på interaktionen med et klassisk felt, og i sidste ende kommer resultatet endda bare af, at elektroner ikke bare interagere som de bare elektroner, men at der følger nogle fotoner med i udregningen. For hele resultatet kommer nemlig ved at se på de fotoner, der er udsendt før elektronen rammer det pågældende punkt i det klassiske felt, og se hvordan de to interagere med elektrontilstanden efter den (lokale) klassiske interaktion har fundet sted. ..Hvilket jo kun giver super god mening. Så dette virker pludeselig bare som en total legit udregning --- måske endda netop fordi den bare ser på et klassisk felt! Jeg tror på en eller anden måde, jeg har været blind overfor denne kendsgerning før, fordi jeg bare har tænkt: jamen hvis det er et QFT-resultat, så må det bestå af QFT-feltintegralerne, og så må alle felter komme fra fotonerne. Og når så det klassike felt i diagrammet er skitseret lidt som en foton.. ja, jeg tror bare jeg har været blind overfor, at der kunne være tale om en halv-klassisk udregning, når nu resultatet er gjort til sådan en stor ting for QFT. ..He.. Jeg er en anelse måløs; det er lidt svært lige at overskue, hv.. Hm ja, hvad er det, der er svært at overskue..? Hvorfor jeg ikke har gidet at læse mig ordentligt frem til, at det bare var dette, det hele handlede om (og/eller hvorfor jeg har smidt den indsigt ud igen, hvad jeg nok har, og hvad der nok giver mere mening, for.. ja, der er alligevel mange indsigter, der gør, at jeg forstår det nu, og jeg ikke lige har haft disse indsigter klar i hukommelsen/bevidstheden, da jeg førhen har læst det (..hvis jeg overhovet på noget tidspunkt har prøvet at forstå det..(?..)))?.. Hm tja, dette giver egentligt ok mening.. ..Hm, det skulle egentligt ikke undre mig, hvis jeg, første gang jeg har læst det, simpelthen har været sur på, at der brugtes et klassisk felt, og at jeg så efterfølgende har glemt dette igen, og bare ikke orket at læse om en gængs-QFT-udregning, fordi "jeg sikkert ikke kan forstå det alligevel, uden at læse op på en hel masse igen, og det orker jeg ikke." Ja, og når man kigger på det, så ser det også ud som om, der er en masse resultater i det, som man bør kende, før man kan forstå udregningen. Så ja, det giver egentligt god mening. Men nu fik jeg lige set på det med friske øjne og et skarpt hovede (klokken er først lige slået tolv nu), og hvor jeg lige er begyndt at tage pertubationsregning seriøst (og i øvrigt hvor jeg ikke har noget imod at se på et klassisk felt i udregning --- for pr. mit eget resultat kan V-feltet jo i \emph{høj} grad regnes som klassisk, og derudover kan man jo også have kohærente bølger), og kunne altså ret hurtigt læse mig til, at det hele egentligt bare handlede om at se på, hvordan start-foton-feltet (altså det felt, der følger med elektronen kan man sige (også selvom man jo egentligt ser på en bar start- og slut-elektrontilstand (som altid i gængs QFT), men dette er jeg jo også lige begyndt at tro på det fornuftige (nok) i at gøre)) interagerer med slut-tilstanden, efter at den har haft en lokal interaktion med et klassisk felt. Ja, og som nævnt gør mit resultat jo om noget bare denne udregning mere fornuftig, fordi man med mit resultat bare i endnu højere grad kan tillade sig at se på, hvad der sker, når der bruges klassiske felter i udregningerne (særligt når det kommer til V-feltet, man da teorien er Lo.-inv. så er der ingen grund til at tro, at man kun kan gøre det for V-feltet (og desuden har man jo også kohærente bølger til at lave "klassiske" B-felter)). Så altså, i hvert fald hvis spin-spin-interaktionen i høj grad også kan forklares af Dirac-ligningen (måske plus Coulomb, men måske er dette ingen gang nødvendigt), jamen så har jeg ingen grund til ikke at tro på, at der er styr på alle de gængse spin-interaktioner. Og det er så kun g-korrektionen af disse (hvis spin-spin er accounted for), der tilsyneladende kommer fra foton-elektron-interaktions-perturbationen, men dette er også fuldt ud fair. Jeg kan personligt ikke lide, hvis det mere stærke interaktioner skulle komme fra foton-feltet omkring de fermionerne, for som jeg ser det, så bør pertubationen fra fotonerne ikke kunne være særligt stor (for ellers ville vi, som jeg ser det skulle få alt for mange udsendte fotoner, hver eneste gang en elektron m.m. accelereres.. og ja, det ville i det hele taget være mærkeligt.. oh well, men det er nu mest bare en subjektiv fornemmelse, jeg har, og selvfølgelig er jeg jo nu også biased, nu hvor jeg har mit Coulomb-resultat fra min QED-teori).. 
%Nå ja, og et andet stikord er så nemlig også, hvis ikke jeg har nævnt det, at huske at nævne i det udgivne, at der jo er lidt et paradoks i både at regne med, at hele Coulomb-potentialet kommer fra en pertubation fra foton-elektron-interaktionen (og fra "foton-feltet," der følger med elektroner), men så samtidigt have det helt fint med at se på de bare start- og slut-tilstande i path integral-pertubationsregningen.. Ja, det kan jeg vist godt nævne.. 
%Ah, det er lidt vildt, at jeg nu er nået hertil. Så jeg har nu ikke umiddelbart rigtigt noget, jeg har lyst til sådan at modbevise i gængs QFT --- og jeg har lidt lagt mit had til det i jorden, endeligt. ..Jo det kunne selvfølgelig være rart med et ordentligt modargument mod Yukawa-teori-argumentet også, i stedet for bare: det behøves ikke. Nå jo, jeg har forrsten et lille et, og det er bare: hvordan skal fotoner med impuls i retning x lige nå at propagere hen til et punkt i retning -x; \emph{imod} deres impulsretning(!), for at afgive deres impuls og tiltrække.. Nå jo, men her var modargumentet lidt, at man enten nok kan behandle alle positivt ladede fermioner som huller, og hvis man ikke kan, jamen så skal man nok bare konjugere teorien efter, at man har udledt \hat H fra Lagrangianen, således at man \emph{vil} kunne se dem akkurat som huller (i et Dirac-hav). Jeg føler jo, at jeg kan vise, at teorien, hvor jeg konjugerer positron-løsningerne, \emph{inden} jeg transformerer fra L til \hat H, er Loretnz-invariant, og umiddelbart så tror jeg på, at jeg har fat i den rigtige her, men det kan gå hen og vise sig, at denne teori er forskellig fra, hvis man konjugerer, efter at L er omdannet til \hat H (og efter at vi altså har indført guage-symmetri-løsnings-restriktionerne (i.e. mit "løsnings-domæne") og har integreret felterne i felt-integralet ud, så vi opnår \hat H), og i så fald kan det jo være, at det vil være nummer to løsningen, vi i virkeligheden er ude efter. Ja, så never mind for nu med det (indtil videre ret løse) mod-argument, jeg var ved at skitsere.. 
%Nå, jeg havde egentligt regnet med hurtigt at gå igang med udkastskrivning omkring mine web-idéer, men nu gik der så fysik i hele formiddagen-middagen alligevel. x)^^ Ha, men hvor fedt, at der gjorde det..! Hvor er det nogle fede indsigter, jeg lige fik her (altså bl.a. accept af gængs g.m. ratio og ikke mindst også min idé om at reklamere med, at der jo kan være nye teoretiske resultater at opdage for andre gauge-teorier). 
%Hm, og det kan jo så måske være, at man kan udlede noget ny fysik ved at tage position-konjugeringen som en mere fundamental ting, når man regner på forudsigelser fra Dirac-ligningen. Men hvis dette derimod fører til forkerte forudsigelser, jamen så kan det også være, at man i stedet bør prøve at konjugere teorien, efter at man har transformeret fra L til \hat H. Hvis dette skulle vise sig at være tilfældet, så skal jeg nok melde mig på banen igen, for jeg har da også overvejet, hvordan man kan vise Lorentz-invarians i dette tilfælde (og har før følt, at jeg har løst det, men denne løsning har jeg ikke lige kunne og/eller haft tid til at genskabe (fordi jeg indtil videre er tilfreds med resultatet, hvor jeg konjugerer før)), og måske kan jeg godt fuldføre dette, hvis det skal være. 

(29.12.21) Nå lad mig endelig lige prøve at færdiggøre denne sektion. Nå ja, jeg har også godt nok et hængeparti ovenfor, men det tror jeg bare jeg lader være. Sådan, nu har jeg indsat en paragraf om, at man bare skal kigge i de udkommenterede brainstorm-noter i stedet. Og jeg vil legeledes også anbefale, at man læser noterne ude i kommentarerne imellem denne paragraf og den forrige. Men lad mig dog lige hurtigt knytte nogle ord her til, hvad disse handlede om. 

For det første indså jeg, at jeg ikke har dykket så meget ned i, hvordan $\psi(x)$ (og $\psi^\dagger(x)$) skulle defineres i beviset om Lorentz-invarians. Men jeg kom så frem til, at $\int\psi_\mu(p) dp$ faktisk er lig  $\psi_\mu(x)$ i form af en delta-funktion, som vi kender dem (og som altså kan defineres som grænsen til en tyndere og tyndere gauss-funktion). Og så kom jeg altså frem til, at det simpelthen bare er disse helt simple $\psi_\mu(x)$'er, man bare skal (eller i hvert fald kan, om ikke andet) bruge (nemlig som bare er en delta-funktion i én af spinor-indgangene og nul i resten). Og så vil to af disse $\psi_\mu(x)$ bare repræsentere antipartiklerne, og skal så konjugeres i felt-/path-integralet. Så det var jo ret vigtigt lige at få styr på; bedre sent end aldrig.

Så fik jeg også endeligt afsluttet tænkeriet omkring den gyromagnetiske ratio og alt det. Men det må man hellere læse sig til i omtalte kommentar-noter. Jeg kan dog lige forklare, at mine tanker omkring at se på, hvad der sker, når $V$ ``booster'' partiklerne, det handlede om, at en overgang fra $\psi(p_1) \to \psi(p_2)$ jo nu, når det er Dirac-spinorer, hele tiden må miste lidt overgangsamplitude til antipartiklen. For når $\psi_{k_1}$ boostes uden at ændre forholdet mellem spinor-indgangene, så vil, f.eks., $\psi_0(p_1)$ jo i stedet blive til $a\psi_0(p_2) + b\psi_1(p_2) + c\psi_2^\dagger(p_2) + d\psi_3^\dagger(p_2)$. Og alle antipartikel-overgangene vil så ikke bidrage med noget, for her vil f.eks.\ $\psi_3^\dagger(p_2)$ jo bare ramme vakuum og forsvinde. Og dette ledte mig så til (i øvrigt lige ved at betragte et ordentligt sæt af Dirac-spinorer, og ikke det fra Lancaster \& Blundell) at der vil komme en spin-orbit-kobling. Men jeg kan så se i Abers, at dette er et gængs resultat. Og jeg tror i øvrigt også at spin-spin-koblingen ligeledes er forklaret (altså i teorien, hvor man bare ser på Dirac-partikler i med en Coulomb-interaktion imellem sig (som de jo har i min teori)). Jeg kunne ikke komme nogen vegne med mine tanker om, hvordan der måske kunne komme en gyromagnetisk ratio, men så fik jeg også kigget ordentligt i den gængse udregning beskrevet i Lancaster \& Blundell, og jeg kan se, at det er udledt fra en helt simpel (path-integral-)kanal, hvor man ser på et klassisk felt, og en sådan udregning vil altså bestemt også være gyldig i min teori. Så det eneste punkt, hvor min teori siger noget andet, er bare ift., at jeg jo så ikke tror, at Coulomb-interaktionen bæres af fotonerne og kommer fra, hvad argumenteres for i Yukawa-teorien (se Lancaster \& Blundell), men at Coulomb-potentialet kommer fra $V$- og $A_\parallel$-felterne, som man altså ikke bare kan se helt bort fra. 

Og i øvrigt har jeg så netop også endelig indset, at min teori jo faktisk ikke modsiger gængs QFT men snarere retfærdiggør den, fordi det nu giver meget mere mening, at regne på path-integraler, hvor man ser på det bare start- og slut-tilstande i stedet for de virkelige egentilstande (som også vil bære nogle fotoner med sig så at sige). For med min teori giver det nemlig god mening, at elektronerne ikke behøver at ``bære særligt mange fotoner med sig'' --- altså at de bare fotoners amplituder i de virkelige egentilstande ikke behøver at være særligt store. Det ville jeg jo ellers mene, de burde være, hvis de skal kunne bære hele Coulomb-interaktionen (og endda på sådan en glat og sømløs måde, at ingen ligger mærke til dem, når vi f.eks.\ betragter atomer (og deres orbitaler)). Men hvis de ikke er det, så giver det jo pludselig fin mening at regne meget på de bare start- og sluttilstande. Så jeg er altså endeligt klar til at begave mit ben i siden på gængs QFT --- i hvert fald når det kommer til noget af det helt grundlæggende ved QFT; der er umiddelbart stadig mange ting, der virker totalt langhårede, men dem må jeg jo sørge for at studere nærmere, inden jeg kan kritisere dem (indtil videre tænker jeg altså bare på ting, jeg lige har hørt i forbifarten, og ikke umiddelbart syntes virkede overbevisende (jeg har btw ikke noget specifikt i tankerne.\,. det hele var nu bare lige for at understrege, at jeg begraver stridsøksen, men det betyder selvfølgelig ikke, at jeg siger god for det heller)).

Okay, og så fik jeg også en rigtig dejlig tanke, som jeg også lige kan gentage kort her. Jeg bør helt klart påpege i min udgivelse, at min teori (selvom den jo måske ikke lige selv fører nogen nye, testbare forudsigelser med sig med det første) jo måske kan åbne op for, at andre gauge-teorier også kan blive behandlet. Og hvem ved, her kan det jo godt være, at man kan opnå nye forudsigelser ved at behandle dem, som jeg har gjort i mit arbejde. 




\subsection[M.o. Økonomiske ting]{Mere om nogle af mine økonomirelaterede idéer}

(03.12.21) Jeg har lige lyst til at brainstorme/opsummere lidt over nogle af mine økonomirelaterede tanker. Jeg tror muligvis, jeg har nogle nye ting at tilføje, men ellers gør det jo heller ikke noget lige at gentage og opsummere nogle ting. 
Jeg tænkte nemlig lidt over min idé til ``kundedrevne virksomheder'' i går aftes, samt lidt andre ting i den dur (bare lidt småtanker; jeg var lidt udmattet efter alt fysik-tænkeriet (og i øvrigt også skriveriet i går sen eftermiddags-aftes)). 

Hvor skal jeg lige starte?\ldots\ Jeg kom lidt frem til, at min idé jo svarer lidt til, at en virksomhed og dets investorer underskriver en kontrakt om, at virksomheden vil tilbagekøbe dets aktier på sigt med det formål at reducere dets priser eller bruge pengene på andre ting, der kommer kunderne til gavn. \ldots I princippet kunne man jo så hævde, at dette vil en virksomhed allerede gøre, for dens omsætning afhænger af, hvor meget den kommer kunderne til gavn, og den vil derfor automatisk søge efter at maksimere denne gavn, for ellers vil andre virksomheder jo bare komme til og gøre et bedre job. Dette virker bare ikke helt i princippet, eller rettere det virker ikke, som tingene er nu. Men det kommer, som jeg ser det, til at ændre sig med ``civil-foreninger'' (eller hvad der svarer til) m.m., og særligt også med ``forbrugerforeninger.'' Jeg tror således på, at forbrugere i fremtiden vil få og/eller udvikle foreninger og værktøjer til at granske, hvordan de som diverse grupper bedst muligt får diverse virksomheder til at arbejde for \emph{dem} (og ikke for andre parter), nemlig ved at investere i, donere til og købe produkter og services fra de rigtige steder. Og når den tid kommer, så \emph{vil} markedskræfterne igen begynde at fungere i forbrugernes favør. Og så bliver det ikke ligeså nødvendigt at underskrive kontrakter, som jeg foreslår det for de ``kundedrevne virksomheder,'' som skal sikre sig, at virksomheden i sidste ende vil sigte efter at gavne så mange kunder (måske inden for et specifikt område, men det kan jeg vende tilbage til) som muligt og så godt som muligt. Men uanset hvad, så \emph{er} sådanne kontrakter stadig smarte --- både for virksomheder, der gerne vil være første til at ride på en ny økonomisk bølge, men også for virksomheder, der startes (eller omlægges) efter at bølgen har fundet sted. For det vil nemlig altid være lidt kosteligt for kunderne (i gennemsnit) at skulle irettesætte virksomhederne, hvis de drift'er\footnote{`Driver' på dansk, men der er nogle skarpere konnotationer på engelsk, synes jeg, til at drive væk (tilfældigt eller som følge af underlæggende drivkræfter) fra en fastlagt kurs.} væk fra, hvad der gavner kunderne bedst. Så derfor vil det være mere tiltrækkende for kunderne i første omgang bare at bakke op om virksomheder, hvor der allerede er indbyggede foranstaltninger, der holder virksomheden på ret kurs (nemlig så man sparer fremtidige irettesættelsesprocesser). 

Og mine idéer til, hvad sådanne foranstaltninger kunne indebære, handler altså om, at virksomheden\ldots\ Ja, for det første er en vigtig ting jo, at aktierne ikke automatisk skal give nogen stemmeret i virksomhedens ledelse --- ikke i det grundlæggende lag. Og samtidigt skal der ligge en kontrakt om, at virksomheden kan og skal tilbagekøbe dele af alle dens aktier, således ejerskabet ikke bliver på investorernes hænder, men at det med tiden vil brede sig ud til kunderne (mere om det om lidt). Så aktierne bliver dermed bare et krav på et vist samlet afkast, som enten kan betales løbende (hvilket sikkert er bedst) eller på en gang ved en deadline, eller når omstændighederne er til det (hvilket man så skal beskrive i kontrakterne, hvornår de er det). Dette afkast kan så godt afhænge af faktorer for, hvor meget virksomheden f.eks.\ vokser i perioden, og der må også gerne f.eks.\ være minimums- og maksimumsgrænser på afkastet. Men det korte af det lange er altså, at virksomheden køber sine værdier tilbage løbende til en vis aftalt pris fra investorerne, og at dette så bliver prisen for at investere. Og hvem skal så få de tilbagekøbte aktier? Det skal kunderne og nye investorer. Hele pointen er nemlig, at virksomheden skal betragte kunder langt mere som en slags investorer, end det er tilfældet i gængse virksomheder. Så selv hvis der ikke kommer nye investeringer (og altså eksterne pengeindsprøjtninger i virksomheden), og at alle pengene til virksomheden kommer til at komme fra kunderne, så skal aktierne altså løbende skifte hænder (på forceret vis ud fra kontrakterne (men selvfølgelig hvor man betaler en pris for det; den der er fastsat på forhånd ud fra kontrakterne)) fra de gamle investorer og kunder til den nye kundebase. Og her snakker vi altså aktier, som giver krav på andele af værdierne ved salg af virksomheden ved en konkurs (plus altså det ekstra krav om, at aktien skal sælges til en fastlagt (men ikke nødvendigvis konstant; den kan som sagt afhænge af virksomhedens vækst) pris). Så man sørger altså for, at aktierne i høj grad følger kundebasen --- og også inkl.\ nye investorer, hvis der er sådanne. 

Og hvad skal alt dette så til for? Jo, det eliminerer behovet for, at aktionærerne nødvendigvis skal have medbestemmelse over virksomheden. For hvis en aktie var en bestandig ting, så ville det ikke være rart som investorer (eller rettere som samlet gruppe af investorer) ikke at have medbestemmelse/bestemmelse over virksomheden. Og samtidigt, hvad der er endnu vigtigere, så skal kunderne ikke bekymre sig om, at en opbakning omkring en virksomhed måske bare i høj grad ender i nogle aktionæres lommer. Jo, en større opbakning og en større vækst som følge heraf vil gavne investorerne, men kun til en vis grad. Og fordi investor-gruppen med tiden i større og større grad vil være kunderne selv, jamen så bliver det jo så meget mere desto værd for kunderne at bakke op; fordi de så selv vil være investorerne. Og det er altså her værdien af det hele virkeligt kommer fra. Når først man når et punkt, hvor kunderne i høj grad selv er investorerne (og hvor de i øvrigt også i høj grad har stemmemagt over virksomheden), jamen så vil kunderne bakke op i langt højere grad og vil ikke have grund til at sanktionere/irettesætte virksomheden eller ligefrem begynde at starte en mere kundevenlig konkurrent, som så kan udkonkurrere virksomheden. Men der skal jo noget til for at nå det punkt, og dermed vil kunder, som jeg ser det, helt sikkert også være interesserede i at opbakke en virksomhed, hvor dette er fremtidsudsigten, men hvor der selvfølgelig er nogle indledende investorer, der skal selvfølgelig skal have deres pris for at have været med til at starte virksomheden. Og fra investorenes side kan det være smart at investere i sådanne kundedrevne virksomheder, hvis der alligevel er en vis udsigt til, at der vil komme en ny økonomisk bølge og skylle de virksomheder, der ikke har foranstaltninger til at eliminere drivkræfter væk fra kundernes interesser i fremtiden, væk. Og hvis virksomheder har udsigten til måske at skylle væk i fremtiden, jamen så vil --- fordi gængse aktier jo også er bestandige --- aktierne her jo falde i henhold til denne risiko. Men hvis man investerer i kundedrevne virksomheder, så er man altså under en paraply ift.\ en sådan bølge. Og det er i øvrigt ikke fordi, at der ikke kan være mange penge at tjene som investor. Medmindre alle virksomheder transitionerer på én gang, så vil der jo kunne blive stor vækst i en kundedrevet virksomhed, og aktierne kan så sagtens give udsigt til et stort afkast til investorene. (Jeg gider ikke lige, at prøve at regne på det, men det kan godt være, at man i princippet ikke vil kunne forvente ligeså stor fortjeneste, som hvis man var investor i en mere gængs virksomhed med en tilsvarende vækst. Men pointen er jo så, at væksten jo kan være så meget desto større, \emph{fordi} det er en kundedrevet virksomhed (og altså fordi denne så kan tiltrække mere opbakning fra kunderne).) 


Nå ja, og det er en helt essentiel del at nævne, at virksomheden skal regne på, hvor meget hver handel, hver type af betaling fra en kunde, bidrager med til overskuddet (så altså hvor man trækker omkostningerne fra). Og virksomheden skal så betragte dette som en ren og skær investering, og skal give kunderne en aktie, der svarer til dette beløb. Man kunne så tænke, at jamen så bliver der jo slet ikke noget overskud, for dette giver man jo så bare kunden i hver handel. Men aktierne skal ikke betragtes på lige fod, hvis de er uddelt/udstedt på forskellige tidspunkter. Disse ``aktier'' er altså ikke nogen, der skal eksisterer forud for handlen, og som virksomheden så skal købe eller skaffe et sted fra for at give dem med. De er i stedet bare nogle kontrakter, der udstedes som en del af handlen (selvfølgelig på en måde, så de bare tilføjer til en eksisterende kontrakt med kunden, hvis denne er oprettet som medlem (hvad alle kunder selvfølgelig skal kunne blive nemt)). Og de nye kontrakter/aktier skal altså så love et afkast, der afhænger af væksten i den følgende periode --- og som i øvrigt også kan afhænge af, hvor meget man ser ud til at skylde de gamle investorer på nuværende tidspunkt ift., hvor meget virksomheden har af værdi- og/eller likvid-beholdning. Så de gamle aktionærer skal altså i første omgang nok få, hvad de er lovet, men kunder skal altså også behandles som nye aktionærer. Og det duer så i øvrigt slet ikke, hvis man lader retten til udbytte falde for meget med tiden; de første investorer må gerne måske få lidt mere favorable aktier, end de efterfølgende kunder/investorer, men kun i det omfang, at de gamle investorer altså stadig skal udkøbes inden for en vis tidsramme, og at det på et tidspunkt flader helt ud (og gerne ret hurtigt), således at alle aktier efter et vist tidspunkt vil være fuldstændigt ligeværdige (for altid; dette må gerne være noget som virksomheden har en vis kontrakt omking med alle dets investorer (inkl.\ kunde-investorer) til hver en tid, nemlig at alle aktier efter et vist tidspunkt skal være formuleret på samme måde ift.\ afkastet), forstået på den måde at afkastet skal være det samme til to forskellige tider, hvis væksten var de samme i perioden (fra udstedelsen til tilbagekøbet). 

Nå, den næste vigtige del af det hele er så, hvordan stemmerne skal fordeles i den kundedrevne virksomhed. Vi snakker altså stemmerne omkring, hvordan virksomheden skal ledes. Her kunne man jo eventuelt bare fordele det ud fra aktierne, men jeg kan se to grunde til, at man gerne vil gøre noget andet og altså opdele de to fordelinger (i.e.\ over værdiejerskab (ved konkurs) og stemmeret). En god grund til ikke at fordele det ens kan være, at man som investor kan have investeret mere i virksomheden, end hvad der svarer til, hvor stor en kunde man er (og altså forbruger af dets produkter og services). Og ved så at afgive mere af sin stemmeret til de egentlige kunder (inkl.\ nytilkomne kunder), så kan dette måske i højere grad tiltrække flere kunder (fordi de jo så for mere medbestemmelse over virksomheden med i købet ved deres handler). Herved kan væksten måske så stige yderligere, og man kan få et større afkast i sidste ende som investor. Som investor, eller som gruppe af investorer, vil man selvfølgelig altid gerne beholde en del af magten selv, så man bedre kan sikre sig, at der kommer en vækst i firmaet, men som sagt kan det altså muligvis alligevel betale sig, at benytte en model, hvor nytilkomne kunder får større stemmemagt end ellers, fordi dette så kan lokke kunder til og på den måde forøge væksten. Så dette vil jeg altså foreslå, at man gør: Jeg vil foreslå at man lader stemmeretten skifte hænder om til nye kunder, endda i endnu højere grad end man gør det for værdi-aktierne, fordi dette altså også kan lokke flere kunder til. (Så på samme måde som for værdi-aktierne, så giver man altså afkald på en rettighed som tidlig investor, men med den opvejning, at man så formentligt kan lokke flere nye kunder til (samt få større opbakning fra eksisterende kunder).) 

Bemærk i øvrigt, at ``virksomheden'' selv jo bare er en række ansatte (inkl.\ virksomhedens ledere), samt en hvis mængde likvider, %*(Og også anlægskapital, eller hvad det nu hed, men det tænkte jeg måske åbentbart ikke på her --- eller også tænkte jeg bare det som en del af likviderne.. ..Ja, det gjorde jeg nok..)
som lederne ud af de ansatte så kan råde over midlertidigt til de daglige handler. Så når jeg taler om stemmeretten, så snakker jeg altså om en samlet ret fuldstændig magt over virksomheden; investorer (inkl.\ kunde-investorer) har altså magt til selv at fyre folk fra ledelsen og ansætte nye, hvis man vil det. Hm, dette giver nok egentligt lidt sig selv, men nu har jeg sa nævnt det, hvis nogen skulle være i tvivl. 

Den sidste store ting, der er værd at overveje i for kundedrevne virksomheder, og som jeg vil skrive om her, er så, hvordan man kan åbne op for, at virksomheden kan opsplittes, hvis det kommer til at give mening. Nå ja, og en ting mere: Jeg vil også sige noget omkring kunde-domænet (måske\ldots). \ldots Nå nej, det giver faktisk sig selv, når man kan opsplitte det; så er der ingen grund til at overveje, om man skal prøve at begrænse sit domæne (hvad angår hverken geografisk set eller set ift.\ produktdomænet) til at starte med. Ok. Og det korte af det lange er her bare, at\ldots\ Ja, det giver faktisk lidt sig selv: Man skal som aktionærer (inkl.\ kunder) kunne vælge at splitte virksomheden op, og så opsplitte aktierne på tilsvarende vis. Og her kan man så enten gøre, så at aktionærene kan handle sig mere til den ene type aktier frem for den anden som en del af opsplitningsprocessen, eller også kan man bare dele det, og så bare lade aktionærene handle med hinanden eksternt.


Så dette kunne altså være en måde, at danne en virksomhedstype, der kan ride på den bølge, der nok vil komme, når forbrugerne bliver mere bevidste om den magt, deres forbrugsvalg egentligt har mulighed for at give dem. Og dette mener jeg altså vil komme bl.a.\ ved at begynde at oprette ``civil-foreninger,'' hvor folk altså kan hyre folk (og særligt, synes jeg, med stort fokus på, hvad jeg har kaldt ``bagud-belønning'' i mine noter her; nemlig hvor folk kan bidrage frit, og hvor medlemmerne, der får gavn af disse bidrag så sørger for at belønne bidragsyderne, sådan at det bliver ved med at være attraktivt for folk (både nye og gamle bidragsydere) at komme med nye bidrag) til netop at analysere de økonomiske situationer (og i øvrigt situationer inden for andre områder --- det kan også sågar f.eks.\ være ting i privatlivene, og ja, alt muligt andet) og komme med handlings-(/opførsels-)forslag til medlemmerne. Og sådanne handlings forslag kunne altså muligvis i økonomisk øjemed komme til at indebære, samlet set at investere og så bakke op om virksomheder, hvor omtalte så selv (som kunder) får del i aktierne, så man dermed kommer til at spare penge på sigt, som ellers ville gå til at give et overskud for andre mennesker. Og særligt kunne en handlingsvej så her være, specifikt at bakke op om --- og skabe, hvis der ikke allerede eksisterer nogen --- sådanne ``kundedrevne virksomheder,'' som jeg lige har beskrevet. Og ja, civil-foreningerne kunne også gå hen og føre mange, mange andre gode handlingsforslag inden for diverse områder med sig (tror jeg på). 


I min idé omkring civilforeninger lægger jeg jo i høj grad op til, at bidragene bestemt kan have vidensmæssig karakter, og altså være i form af diverse (løsnings)forslag til forskellige ting (også inkl.\ velarbejdede forslag, som f.eks.\ ligefrem specificere en hel forening eller virksomhed, og/eller en række kontrakter). Men man kan nu bestemt også inkludere andre former for bidrag med i billedet. Dette kunne f.eks.\ være i form af open source-programmering eller i form af kreativt arbejde, f.eks.\ på en web-platform (eller hjemmeside (prøver jeg jo at vende mig selv til at sige i stedet)) såsom gængse web 2.0-platforme, eller som den/de, jeg har i tankerne (og er ved at udarbejde noter omkring). Og ja, det kan også være selv sådan nogle ting som at komme med nyttige kommentarer til ressourcer på internettet (og specifikt måske på en hjemmeside som den, jeg har i tankerne), som andre brugere kan have glæde af. Hvis vi først tænker os om, så er der \emph{så} mange ting, vi kan gøre for hinanden i det samlede fællesskab, men hvor der pt.\ ikke rigtigt er en god måde for folk til at belønne sådanne gode bidrag. Men ved altså at oprette og deltage i civil-foreninger (som man også kunne kalde ``donationsforeninger,'' især hvis de specifikt retter sig imod denne del, og ikke så meget imod den vidensmæssige del (omkring handlingsforslag til at forbedre folks liv/livssituationer)) så kan folk altså blive en del af en forening, hvor man kan sørge for at ``belønne folk bagud'' for diverse ``bidrag,'' og altså donere penge til folk, der har gjort et godt stykke arbejde, som kommer medlemmerne til gavn (hvorved man så kan opretholde en forstørret lyst fra folk for at gøre sådanne bidrag). 
Finten bliver lige, at få styr på, hvordan pengene skal fordeles, og få medlemmerne til alle at betale i sådanne civil- og/eller donationsforeninger, men det kommer jeg til om lidt. 
Og der \emph{er} mange tilfælde (mener jeg), hvor folk har mulighed for at gavne en masse andre mennesker med relativt lidt arbejde, og hvor hele fællesskabet derfor kunne få meget gavn af, hvis man blev bedre til at donere penge (bagud) til folk, der gør et værdifuldt stykke arbejde. 
Nå ja, lad mig i øvrigt også nævne en særlig ting: Det kan også bruges til at belønne virksomheder for at træffe beslutninger, der i højere grad kommer foreningsmedlemmerne til gode, frem for de modsatte beslutninger. Dette kan altså være en lidt blødere måde at bevæge sig i retningen af at få virksomheder til at arbejde mere for dets kunder (og her særligt altså hvis disse kunder er medlemmer af civil-/donationsforeningen), i.e.\ ved at give dem økonomiske incitamenter for at bevæge sig i denne retning. (Så her bruger man altså guleroden frem for pisken, men man kan selvfølgelig gøre begge ting, og også på en gang; man kan godt lokke med en gulerod og samtidigt også true med en pisk.) 


Nå, nu kommer jeg så til at snakke om afstemninger osv., hvilket bl.a.\ skal bruges i civil-/donationsforeningerne, når man skal fordele belønningerne, men også andre steder: 
Hver eneste gang vi snakker `stemmeret' og `afstemninger,' så mener jeg for det første, at vi klart bør gå over til at bruge en mere konstant og direkte forbindelse imellem folk med ``stemmeret''/``stemmemagt,'' således at folk ikke bare giver deres stemme tilkende i form af nogle få stemmer i løbet af et år, og hvor tingene så har det med at køres op i en spids, så man bliver tvunget til at stemme på mange ting på én gang, og med få effektive valgmuligheder i praksis. I stedet bør vi gå over til mere digitale afstemninger og mere flydende/kontinuerlige ``afstemninger,'' hvor \ldots\\ (06.12.21) \ldots hvor folk løbende kan ændre deres mandater til forskellige personer og/eller ledelsesretninger for virksomheden/partiet/foreningen/etc. Så dermed altså i princippet et mere direkte (e-)demokrati, men dog kan man så sige, at et mere direkte demokrati i sig selv ikke nødvendigvis vil føre noget bedre med sig til forskel fra mere repræsentative og statiske udgaver, for hvem siger, at medlemmerne har tid til at følge med i, hvad der sker? De har jo deres egne liv at leve, og sine egne ting at gå op i, og mange af ledelsesbeslutningerne kan måske afhænge af ting, der kræver meget tid at sætte sig ind i (og især for medlemmer, der ikke er så interesseret i stoffet, og/eller som ikke har flair for det). Men hertil vil jeg så foreslå noget andet, som jeg finder ret vigtigt, nemlig at man i pågældende forening, eller hvad det er, gør det til absolut central og vigtig ting for denne, at medlemmerne skal tilskrive sig visse analyse-/vagthunde-instanser. Om disse instanser bare i princippet skal være tredjeparter, og om medlemmerne så selv altså skal finansiere dem, eller om de er mere en del af foreningen og finansieret gennem foreningen, det må man så lige finde ud af. Men selv hvis det er tredjeparter, så skal vigtigheden af dem stadig fremmes stærkt af foreningen, og denne skal altså helt sikre sig, at alle medlemmer er meldt til nogen, og at der i øvrigt er masser af plads til et stort udvalg af vagthunde-/analyse-instanser, så alle medlemmer kan finde nogle, de stoler på i høj grad. Hele opgaven af disse instanser er så, at holde øje med ledelsen og dens beslutninger (og med de forhold, der laves beslutninger ud fra eller omkring) --- hvor alt denne materiale i øvrigt bør være tilgængelig for offentligheden; jeg går klart ind for åbne og fuldt gennemsigtige foreninger/virksomheder/etc --- og skal præsentere oversigter overfor instansen medlemmer, om hvordan ledelsen kunne handle / have handlet bedre (i henhold disse medlemmers interesser). Hvis en sådan instans gør et dårligt arbejde, og altså ikke får formidlet forhold, som kunne have været interessante for dens medlemmer, jamen så må disse medlemmer jo gå over til andre instanser (eller fyre folk i den og indsætte andre). Og dette fik jeg så ikke nævnt, men der skal være et økonomisk incitament for instanserne om at tiltrække mange medlemmer, hvad end de er tredjeparter, som medlemmerne finansiere på egen hånd, eller at finansieringen på en måde går igennem foreningen (/\ldots/etc.). Så hvis en konkurrerende instans (hvad der så netop gerne altid \emph{skal} være) ser, at en given instans kunne gøre et bedre arbejde, så kan denne jo med fordel påpege dette, hvorved den andens medelemmer så sandsynligvis migrere væk fra den. I øvrigt må disse instanser så helst ikke rigtigt varetage andre funktioner over for dens medlemmer (og dette bør selve foreningen altså også gå op i), for så snart der er andre perks ved at blive hos sin nuværende instans, som ikke har med dens årvågenhed at gøre, jamen så kan den jo pludselig godt slække så meget desto mere på denne årvågenhed. Med dette system behøver medlemmerne altså ikke alle sammen være årvågne (hvilket ikke kan lade sig gøre i praksis; folk har deres egne liv at gå op i), men kan altså bare betale en instans (måske virkende i form af nogle få deltidsstillinger), som de har tillid til, til at være det i stedet. 

Et kritikpunkt til denne idé kunne være at: ``jamen i gængse foreninger m.m.\ vil man jo også høre om det af den ene eller den anden vej, hvis ledelsen gør noget, der ikke kommer en stor del af deres medlemmer særligt meget til gode --- f.eks.\ i medierne, om ikke andet.'' Men nej, det tror \emph{jeg} ikke rigtigt på, gør sig gældende for de fleste foreninger m.m., og uanset hvad kan man også sige, at der jo ingen grund er til at tage dette for givet --- ikke hvis man kan gøre noget bedre og uden de vilde ekstra omkostninger (hvilket der ikke vil være, hvis bare foreningen er stor nok). Og hvis vi snakker foreninger, der er så vigtige som donations-/civil-foreningerne (eller som politiske partier eller som andre typer af investeringsforeninger), jamen så vil det helt klart være det værd, at have mere direkte demokrati (og mere e-demokrati i øvrigt), og så bare sørge for at have sådanne instanser til gengæld, som folk kan benytte (også selvom sådanne vagthunde-instanser vil give en lille ekstra udgift for medlemmerne). \ldots I øvrigt vil man sikkert også kunne komme rigtigt langt med også at have frivillige foreninger (som så også kan være meget lokale) som vagthunde-instanser, så længe der bare er også er mere kommercielle instanser på markedet, som kan konkurrere med de frivilligt drevne instanser, og som de frivilligt drevne instanser altså derfor er tvunget til at konkurrere med.  

Angående det med at ``ændre mandater til forskellige ledelsesretninger,'' så kan dette altså være, hvis nu man i første omgang giver mandater til en ledelse, som ikke har et fast (forenings-)politisk program i udgangspunktet, men som i stedet er åbne for, at folk (løbende) kan stemme på forskellige ledelsesretninger (som altså svarer lidt til ``politiske programmer''), og hvor omtalte ledelse altså så forventes (og ellers ville man nemlig ikke stemme på dem i første omgang) at følge disse ledelsesretninger, så godt som de overhovedet kan. 


(07.12.21) Og så kan jeg lige berøre spørgsmålet omkring, hvad man dog kan gøre, hvis man har f.eks.\ en donationsforening, der donerer penge til folk, der gør en god indsats for andre folk, deriblandt forenings medlemmer, og man gerne vil gardere sig imod, at folk hopper af og stopper med at donere, og så bare nyder godt af de samme indsatser/bidrag men uden at donere penge selv. Her kunne vi jo eksempelvis snakke om en donationsforening med det formål at støtte open source programmeringsbidrag eller andre bidrag på internettet (f.eks.\ hjælpsom/gavnlig adfærd på den hjemmeside, jeg har i tankerne). Kan man gøre noget for at få folk til at ville donere i højere grad og/eller blive ved med dette, i stedet for at de bare freeloader på bidragene? Tjo, desværre har jeg ikke nogen magisk løsning på dette forhold. Udover den lille ting, at donationsvillige medlemmer kan gøre så at deres donationer afhænger af, hvor meget andre giver, så kan jeg umiddelbart ikke rigtigt se andet for, end man enten må prøve at sørge for, at der kan blive nogle fordele for medlemmerne, der donerer, og ellers må man lidt bare håbe på, at der er nok gavmilde og moralske personer i fællesskabet, således at de gavnlige bidrag kan få doneret løn for sig. Jeg tror faktisk i mange tilfælde, at man kan komme langt bare på folks gavmildhed og moralfølelse (hvis man ser på, hvordan mange andre ting i denne verden fungerer pt.\ (omkring donationer osv.)). Men jo, det vil der være rart, hvis man kan skabe fordele for sine donerende medlemmer. Dette kan lade sig gøre, hvis man som forening ejer rettigheder (f.eks.\ IP-rettigheder, hvis vi snakker programmering m.m.) til det, der arbejdes på i fællesskabet. For så kan man jo give medlemmer særlige tilbud. Det samme gælder, hvis man på en eller anden måde har en indkomst i foreningen i det hele taget; så vil der være ting, man kan gøre. Ellers\ldots\ Nå jo, og hvis bidragene i fællesskabet godt kan rette sig meget imod enkelte individers behov, så kan man jo også indføre en konvention om særligt at lytte til og hjælpe medlemmer, der har været faste donorer igennem længere perioder. (Dette kunne måske også bare være sådan noget som villigheden til at besvare vedkommendes spørgsmål i gruppen, og villigheden til at lytte til og diskutere med vedkommende måske.\,.) Hvis vi snakker mine civil-foreninger eller lignende, så kunne det også være sådan noget som at få lov at komme først i rækken, når der er investeringsmuligheder, hvor man altså kan/skal bruge et antal investeringsparate medlemmer.\,. Men ja, udover disse små tanker har jeg altså ikke nogen magisk løsning, der kan gælde for alle mulige foreninger på én gang. Man må lidt bare se på det, når det kommer dertil, og prøve at tænke over, hvad man som donations-/civilforening (etc.) eventuelt kan gøre i fællesskabet for bedre at holde på donorerne samt lokke flere donorer til. 


(09.12.21) Hov, der er faktisk én løsning på spørgsmålet fra ovenstående paragraf, som jeg kan komme i tanke om (og som nærmest også er lidt ``magisk'' \textasciicircum\textasciicircum). Jeg kom nemlig til at tænke på (i forbindelse med at jeg lige kom til at tænke lidt om min simplificerede donationskæde-idé, som jeg jo også gerne vil nævne i min udgivelse), at man jo evt.\ kan vedtage for sådanne typer af foreninger (der kommer samfundet til gavn), at de bliver statsstøttede (i den pågældende stat de befinder sig i). Og i forbindelse med min donationskæde-idé, i øvrigt, så var det altså fordi jeg tænkte på, at jeg for det første i denne forbindelse har tænkt mig at påpege, at folk med tiden kunne oprette foreninger til at opkøbe og destruere bidrags-tokens. For dette vil nemlig så være en måde at indføre en cyklus omkring donations-tokens'ne, således at deres værdi i sidste ende afhænger af hvilke nogle støtte-foreninger, der vil være i fremtiden til i sidste ende at indkøbe og destruere tokens'ne (og så afhænge af, hvad disse så vil betale for dem da). Tanken er altså, at folk altså herved kan komme til\ldots\ %Hm, jeg tror bare, jeg skriver dette i morgen tidlig; kan ikke lige samle tankerne ordentligt.

(10.12.21) Jeg tror lige, jeg prøver at starte lidt forfra og forklare om min nuværende donations-kæde-idé, som jeg også vil nævne i min udgivelse, og så kan jeg lige vende tilbage og runde af omkring skat / offentlig støtte. Det var nu ikke helt rodet, det jeg skrev ovenfor, men det kan ikke skade og gentage det. Så hvis jeg skal opsummere idéen, så er det altså, at man har en kæde ret meget tilsvarende de eksisterende kæder, hvor folk kan komme til at ``eje'' billedfiler osv., men bare hvor folk i stedet ejer signerede erklæringer om specifikke ``bidrag,'' som den undertegnede person (altså den person, der har givet signaturen, hvis ikke jeg bruger ordet rigtigt) har ydet. Og disse ``bidrag'' kan så i princippet være i form alle mulige (dokumenterbare) handlinger, som personer kan gøre, der kan gavne andre mennesker. Og man kan sagtens have mere end én af sådanne kæder, så visse kæder kan også sagtens bare vælge at fokusere på et specifikt område. Hvis bidrag er duplikerede på tværs af forskellige kæder, så bliver fællesskabet dog bare nødt til ret hurtigt at finde en god protokol (hvilket bare bliver i form af en decentral vedtagelse om, hvad fællesskabet synes) for at bestemme sig for, hvilken én af duplikaterne er den ``autentiske,'' for ellers vil det skabe nogle dårlige situationer, hvor investorer ikke kan vide, om deres token overhovedet er noget værd. Men dette giver lidt sig selv; enten må man benytte en god alsidig kæde, eller også må de to kæde-fællesskaber finde en god måde at dele det hele op på --- sådan må det også være for forskellige billedfil-kæder, og hvad har vi?. En vigtig ting ved idéen er så, at det altid skal være bidragsyderen selv, der starter med at eje token'et (hvilket er nemt at sørge for, for man kan jo bare kræve at signaturen sker med en nøgle, der tilhører en token-pung, eller noget i den stil). På den måde får kæden/kæderne nemlig den funktion, at de kan hjælpe med at støtte pågældende bidrag. Og hvis der i fællesskabet så kan opstå en konsensus om, at tokens'ne har værdi ud fra, hvor meget og hvor mange mennesker, bidraget har gavnet, så ville dette jo ikke være dårligt. Så langt så godt, men hvis idéen virkeligt skal blive bæredygtig (udover at man selvfølgelig bør vælge en teknologi, der ikke kræver så meget mining, hvilket \emph{kan} lade sig gøre, for det kræver bare, at fællesskabet er effektive til at soft-forke væk fra angrebskæder), så skal der gerne kunne blive en slags cyklus af tokens'ne, så de ikke bare bliver ved med at eksistere som penge for evigt (for det ville jo så alt andet end lige give en uendeligt stigende pengesum, hvilket ikke rigtigt kan lade sig gøre; så vil der blive inflation, og til sidst vil kæden så blive ubrugelig til at støtte nye gode bidrag i samfundet). Så der skal altså gerne være en måde, hvorpå tokens'ne enten kan falde i værdi automatisk løbende og/eller kan udløbe til en vis dato. Men hvis en token udløber i fremtiden, så vil den i princippet ikke være noget værd, medmindre der er et afkast fra den i mellemtiden. Og hvis der skal være det, så skal der altså umiddelbart være nogen instanser, der har lyst til at betale disse afkast. Og hvem skulle have lyst til det? Jo, det kunne eventuelt jo de mennesker, som bidraget repræsenteret ved token'en har gavnet. Og det er altså næste del af idéen: Investorer i tokens'ne kan håbe på, at der i fremtiden vil opstå foreninger eller andre instanser, der vil være villige til at købe tokens'ne, ikke for at at samle på dem selv, men for endeligt at bringe dem ud af cyklussen ved at give dem en endelig betaling for så at destruere token'et (hvilket kan gøres ved at sende dem til en falsk pung og/eller give dem til en smart-kontrakt, der aldrig kan udløses). Men for at dette skal kunne lade sig gøre, skal der altså være en ordentlig grund til, at sådanne foreninger vil gøre sådan. Som sagt kan det altså bruges som en måde %at 
%... Uh, jeg kom lige til at tænke på, at det hedder jo bare: "at træke det fra i skat." Så det er det eneste, man skal opnå for at sikre sig, at folk ikke for noget ud af bare at freeloade på andres donationer, og det er, at donationerne skal kunne trækkes fra i skat! Nemt! Ej, det er da nærmest lidt "magisk," det her; vores samfund er jo nærmest gearet til denne udvikling allerede. ^^
som forening at støtte bidrag bagud, hvilket så selvfølgelig ikke bare gøres, fordi man synes, man skylder det, men også for at tiltrække nye bidragsydere i samtiden for, hvornår denne endelige betaling gøres. Men her kan man så spørge, hvorfor til den tid ikke bare donere til bidrag direkte i samme forening; hvorfor gå igennem omtalte kæde-tokens for det? Jo, måske fordi dette kan medføre, at bidragsydere kan blive betalt for deres arbejde hurtigt efter, det er gjort, men uden at foreningen så skal tage risikoen ved dette. For det kan jo være, at det bidrag, man donerer til, slet ikke når at blive til noget særligt. Men ved at lade bidragsyderne kunne sælge deres rettigheder til betaling for bidraget videre til andre (og hvor foreningen så klart har sat sig for 100 \% at anerkende denne rettighedsoverførsel), så får foreningen altså automatisk en mulighed for at lade først bidragsyderen selv, og eventuelt de personer, vedkommende sælger sin token til, tage den risiko, der kan være forbundet med, om et bidrag i sidste ende viser sig at blive gavnlig for samfundet eller ej, og i hvor høj grad. Så dette er altså pointen. Så kunne man så spørge: ``hvorfor skal sådanne foreninger så lige benytte en given kæde, som man får sat i gang; hvorfor vil disse foreninger ikke bare til hver en tid sige: `lad os egentligt bruge en anden kæde i stedet og glemme alt om den gamle kæde i de tokens, der var på den?'\," Og hertil er der et enkelt svar: Fordi det altid vil være mest gavnligt for foreningen, hvis målet er at tiltrække flere bidrag, at anerkende en hvilken som helst handel fra fortiden, hvor der var en klar forståelse på daværende tidspunkt, imellem sælger og køber, at handlen ville overføre rettighederne til den endelige betaling (altså fra foreningerne / fra samfundet). For hvis først man pisser på sådan en fortidig konsensus, jamen hvorfor skulle man så ikke gøre det igen? Så dette vil jo selvsagt ikke gøre det mere tiltrækkende for samtidige bidragsydere, tvært imod. Selv hvis foreningen sagde, hov nu ændrer vi lige reglerne, og betaler de oprindelige bidragsydere (hvis de stadig er i live) i stedet for de nutidige ejere af tokens'ne, så vil dette jo heller ikke tiltrække flere samtidige bidragsydere. For hvis der er et samtidigt behov for, at bidragsyderne kan sælge deres tokens videre til tredjeparter, så vil man jo bare ødelægge dette ved at pisse på den gamle konsensus (imellem de handlende) for, hvad tokens'ne repræsenterede. Og hvis der på et tidspunkt ikke bliver et behov for tredjeparterne, jamen hvad får man så alligevel ud af, at betale de oprindelige bidragsydere mere, hvis disse alligevel indvillede på daværende tidspunkt at sælge deres rettigheder til den endelige betaling videre? Hvordan i alverden skulle dette tiltrække flere samtidige bidragsydere? Nej, jeg tror aldrig det vil kunne betale sig med for fremtidens parter ikke at anerkende en fortidig handel, hvor der var en ærlig forståelse er, at en person solgte nogle rettigheder videre til en anden\footnote{Ikke medmindre noget virkeligt drastisk skulle ske, hvor omtalte parter vil forsøge at omfordele pengemængden i samtiden, men dette er jo totalt langt ude; i så fald kunne vi vinke farvel til alle nutidige fortolkninger af, hvad værdipapirer er værd.} --- og dette er altså uanset om disse rettigheder var juridisk bundet til pågældende token (hvad de jo nok ikke kan blive i vores tilfælde), eller om personerne bare har underskrevet et udsagn hver især, om hvad handlen repræsenterer, og hvordan den bør fortolkes af den daværende samtid og fremefter. Så hvis der altså bare kommer en klar konsensus for en kæde af, hvad tokens'ne repræsenterer, og hvis man kan se, at bidragsyderne ikke forsøger at duplikere deres bidrag på flere forskellige tokens og/eller på forskellige kæder, jamen så tror jeg altså ikke man skal være bekymret for, at fremtiden skulle gå hen og pisse på en nutidig handel og på den fortolkning af den, som begge parter har tilskrevet sig (og underskrevet på en eller anden måde, enten som en del af handlen, eller i forbindelse med at de oprettede sig på kæden). 

Det vil heller ikke rigtigt betale sig for en fremtidig donations-forening (hvilket vi jo godt kan kalde dem) at pludseligt at stoppe med at anerkende en kæde og dens tokens for at investere direkte i samtidige bidrag, for her kan man jo ligeså godt bare købe tokens'ne i stedet. Og man vil ikke rigtigt spare noget ved pludseligt at stoppe med at betale for de gamle tokens, hvis man stadig vil have det sådan, at nye bidragsydere kan se (hurtig) betaling for deres bidrag, for hvis man stopper med at betale for de gamle tokens, så vil tredjeparterne med det samme stoppe med at købe nye tokens for samtidige (nye) bidrag, og så ville foreningen bare skulle dække disse betalinger i stedet. Så der bliver altså heller ikke som sådan nogen vedvarende frist for foreningen til pludseligt at droppe betalinger for gamle tokens. 

Hvad bliver så den egentlige værdi af tokens'ne? Jamen, donationsforeningerne kan jo til sammen beslutte sig for, hvad man vil købe (eller ``tilbagekøbe,'' kan man nærmest tænke det som (hvis man vælger at se alle oprettede tokens som et slags lån fra bidragsydere og til de mulige nyttehavere, hvor tilbagebetalingsprisen for ``lånet,'' så bare ikke er sat på forhånd, men afhænger af, hvor godt bidraget viste sig at blive)) tokens'ne for. Hvis ingen foreninger vil betale over en vis pris for en token, og det ikke ser ud til, at de på noget tidspunkt vil komme til dette, jamen så må man jo bare i sidste ende sælge den for den pris, de kan gå med til. Så hvis man har købt token'en for mere end denne pris, jamen så vil man sandsynligvis tabe penge i sidste ende. Hvis man dog har købt den for mindre, vil der sandsynligvis være penge at indtjene, for donationsforeningerne er heller ikke interesserede i at købe tokens'ne til en pris, der ligger særligt meget under et vist niveau, for så vil nye bidragsydere jo kunne se dette, og vil så ikke blive ligeså tiltrukne til at gøre nye bidrag. Så der er altså med andre ord ikke noget incitament fra foreingernes side om at presse den endelige pris ned under et niveau, der ikke stemmer overens med, hvad bidraget egentligt var værd (set ift. hvor gavnligt for foreningen det er at tiltrække flere af sådanne bidrag). Så dette vil altså i sidste ende bestemme tokens'nes endelige værdi.

Jeg har ikke lagt så meget vægt på det indtil nu, men man kunne også lige nævne, hvor brugbart det også vil være for bidragsyderne, at hvis der virkeligt kan opstå en situation, hvor de ikke behøver at gøre andet end at dokumentere deres arbejde for at kunne forvente retfærdig betaling for det, og at de med andre ord ikke skal bruge en masse krudt på at sikre sig de rette rettigheder og aftaler omkring deres arbejde først. Så med sådan en situation ville folk kunne give bidrag meget mere frit, og der kunne opstå en meget mere fri kultur, når det kommer til deling af viden og idéer, og når det kommer til at yde hjælp til andre. Dette vil jo især være smart, hvis mere og mere arbejde i fremtiden kommer til at handle om at skabe indhold og yde hjælp til folk på/over internettet. Her vil en sådan økonomisk situation med ekstremt høj frihed for arbejdere/bidragsydere jo blive helt vildt gavnlig. 


Nå, og den sidste ting, jeg vil nævne her angående min donations-kæde-idé (i dens nuværende simple form), er, at man så også kunne spørge: ``Hvad forhindrer folk, lad os sige fra en hvis samfundsmæssig gruppe som eksempel, i at undlade at melde sig til, eller melde sig ud af, en donationsforening (og/eller altså stoppe med at donere) og så alligevel bare drage nytte af de samme bidrag i høj grad?'' Mange bidrag vil jo ramme en stor gruppe mennesker på én gang, hvor det ikke kan lade sig gøre kun at gavne de individer, der donerer til foreningerne --- og især ikke når det kommer til sådan noget som indhold og hjælp på internettet (\ldots eller her bliver det i hvert fald besværligt at skulle begrænse adgangen). Men til dette kan der faktisk være en utroligt simpel løsning, og det er, hvis donationerne kommer til at kunne trækkes fra i skat. Så vil der pludselig ikke være nogen grund til ikke at melde sig til en forening og donere til denne, for dette vil jo bare i så fald betyde, at man for højere magt over, hvad de penge, man alligevel skulle betale, går til. Så potentielle tidlige investorer i tokens'ne på en sådan ``donations''-kæde, gør altså klogt i først lige at overveje, hvad sandsynligheden er for, at donorer i fremtidige donations-foreninger vil kunne trække deres beløb fra i skat, for dette vil jo så betyde noget for, hvor effektivt sådanne foreninger kan kære, og dermed i sidste ende også for sandsynligheden for, at man kan forvente at blive betalt ordentligt for den token, man har investeret i.

Så ja, dette er den idé, jeg også vil præsentere i min udgivelse nu her. Jeg skal så også lige i den forbindelse huske at nævne, at det altså bare er en idé, og at den bør gives meget mere overvejelse (og altså helst i et stort fællesskab, så man ikke overser noget), før at folk begynder at investere, for der kan jo være ting, jeg har overset i min lille analyse her. 


Og nu kan jeg så gå tilbage til min idé om, hvordan man måske kunne sikre sig, at folk ikke vil melde sig ud af (eller aldrig melde sig til) donations-foreninger m.m., og at de altså ikke bare vil freeloade, men ja, nu har jeg faktisk forklaret idéen allerede ord for ord: Hvis man kan opnå, at folk kan trække deres donationer fra i skat (selv måske endda bare delvist på en eller anden måde), så kunne dette altså være en kæmpe løsning på netop dette problem. Bum. 


(11.12.21) Jeg kan også lige tilføje dette, for man kan jo spørge: ``hvilket starttidspunkt skal man så vælge at udstede tokens fra?'' Og her er det simple svar jo, at det hele handler om at forudsige, hvad folk (og altså foreninger/grupper af folk) i fremtiden vil donere penge for. Og disse penge doneres så både pga.\ en vis lyst til at give gengæld for bidrag, der har kommet en til gavn (hvad man nemlig kan finde hos os mennesker; vi er ikke rene homo economus) og ikke mindst altså også som en måde at tiltrække nye bidragsydere på i samtiden for donationen. Så det vil være naturligt, at der vil blive en vis konsensus om nogenlunde at starte fra, at hele bevægelsen gik i gang, og folk begyndte at sælge ``rettighederne'' (dog nemlig måske ikke på en juridisk bindende vis, men bare på en måde, hvor man regner med at aftalen bliver respekteret af andre, også i fremtiden) til donationer for bidrag til hinanden. For inden da kan man jo altid sige, jamen folk må jo have haft deres grunde allerede der til at gøre de bidrag, de nu engang gjorde, og så ikke tænke så meget mere over dette. Men på den anden side, kan det da sagtens tænkes, at nogen i fremtiden på et tidspunkt vil betale / donere penge for tokens af ældre bidrag. Så dermed giver det jo stadig mening at samle på sådanne, også selvom disse så ikke vil være helt så eftertragtede at samle på. Og især hvis der er tale om bidrag, hvor man virkeligt tror på, at fremtiden vil se det som\ldots\ tja, på den anden side.\,. Nej, lad mig ikke bekymre mig så meget om, hvad der her kunne være værd at samle på. Det må folk selv finde ud af, og det er jo altså slet ikke sikkert, at præ-bevægelses-bidrag vil få doneret penge for sig i fremtiden overhovedet. Ok. Hvad gør man så, hvis den mest populære kæde slet ikke vil godkende tokens fra gamle bidrag? Jo, men bevægelsen er jo ikke afhængig af en enkelt kæde. Det eneste den er afhængig af, er at folk nogenlunde kan finde ud af at opdage og slette duplikat-tokens i tide, så folk ikke kommer til at føle sig snydt ved at købe noget, der så ikke ender med at repræsentere det, de troede. Så hvis bare man sørger for altid aktivt at kommunikere imellem de forskellige kæder (eller hvad end database-systemer, man ellers bruger udover blok-kæder), så man kan annullere duplikat-tokens, jamen så kan der altså være ligeså mange kæder, man har lyst til at have. Ok, jeg tror det var de ting, jeg lige ville nævne. 


(29.12.21) Måske var dette allerede klart, men folk skal jo bare kunne vente med at udstede tokens for deres arbejde/bidrag, så længe de har lyst. Donationerne må altså gerne kunne afhænge af tid, men ikke særligt gerne af, hvornår bidragsyder udstedte token'et. Og bidragsydere bør i øvrigt kunne udstede tokens, der kun gør krav på en procentdel af donationen. 

Hm, hvad gør man egentligt, hvis en person lader som om, at denne har gjort et stykke arbejde, men i virkeligheden er det en anden, for så kan de måske begge sælge tokens, først den falske og så den rigtige (efter sandheden er kommet frem), og den falske arbejder kan så bare betale den rigtige halvdelen eller mere af, hvad denne fik udbetalt (så de begge tjener på det).\,? Sådanne ting gør jo nok, at det er ekstra vigtigt, at arbejdet er dokumenteret korrekt. Men der vil altid være folk, der kan finde en vej at snyde, så måske vil det faktisk være gavnligt med et lidt rigidt system.\,. %Ah vent, man kan jo bare sige.. nej..
%\ldots Okay, i princippet bør man nok starte med at sige, at donationsforeningerne ...
\ldots Hm, skal man så bare starte med kun at lade det handle om programmeringsbidrag? Og så kan programmørerne bare sørge for at vedlægge en slags licens, som fællesskabet udformer, hvor denne så erklærer, at alle donationer for bidraget skal gives, til dem der ejer et gyldigt token, der gør krav på en procentdel af donationen (og fordelt i henhold til omtalte procentdele). Så skal man bare sørge for at udforme denne licens, så den ikke afhænger for meget af én specifik kryptokæde, men bruger mere generelle begreber om, at der har fundet en række dokumenterbare handler sted, hvor donationsretten har skiftet ejerskab.\,. Hm.\,. Ah, der er desværre lige en del, jeg bliver nødt til at tænke over her. 

Hm, måske er det ikke så galt. Måske skal man netop bare starte med at fokusere på digitale bidrag. Det er sikkert også meget sundt et eller andet sted. Og angående ``licensen,'' så kan den bare erklære, at der skal være tale om en offentlig database, hvor alle transaktioner kan uploades til (muligvis imod en lille fee), og hvor der er en sikker måde at opnå konsensus omkring rækkefølgen af alle transaktioner inden for en rimelig tidsgrænse. Systemet begrænses dog ikke kun til programmører i starten; det skal også rumme web 2.0-skabere med mere. I så fald skal skaberne jo bare uploade en licens til deres ``kanal,'' eller hvad der svarer til, hvor de erklærer det samme. Og hvis de fremtidige donationsforeninger kan se, at licenser blev uploadet et sted, hvor ingen andre end skaberen selv typisk har adgang, og hvis den ikke blev annulleret inden for en tid, hvor man må kunne forvente, at skaberen vil opdage det, hvis nu andre skulle have hacket sig ind (måske folk de kender, men hvor de selv ikke kender til den fare endnu og dermed stoler på personerne til at have den adgang) og skrevet en falsk licens med deres egen nøgle, så må licensen bare gælde. Og her kan man så benytte.\,. Hm nej, det kan godt komme et lille høne-æg-problem med, at kæden skal være populær, før folk er vakse, men den bliver måske først populær, når handler begynder at ske (hvilket de først kan, når de ikke bare kan annulleres af den ene part). Okay, men man kan nu bare sørge for, at sikre sig at størstedelen af fællesskabet kender til systemet, inden man som fællesskab bliver enige om en start-dato for licensen. Så kan folk nå at forberede sig, og sørge for, at klargøre deres egen licens og/eller sørge for at overvåge, hvad andre folk med adgang uploader.\,. Tja, eller måske bør fællesskabet også bare udforme en vente-og-se-licens, der fratager skaberen retten til at lægge en licens op i en vis start-periode (sådan at andre med adgang heller ikke kan gøre det). Ok.:)

Ah, nu kom jeg igen i tanke om en rettelse, som egentligt var den største grund til, at jeg begyndte at skrive videre på dette afsnit nu her, og det er, at folk ikke skal \emph{sælge} deres tokens til donationsforeningerne. I stedet skal donationsforeningerne bare registrere, når de har doneret penge til en person pga.\ dennes ejerskab over et token. Så folk kan beholde og handle med deres tokens, så længe de vil, også selvom der allerede er betalt rigeligt donation af på tokenet. For efterfølgende kan det jo stadig have værdi af et collectible. Så folk, der gerne vil eje NFT'er, som har en historie ift.\ at finansiere et vist bidrag til verden, kan så gøre det. 
\\


(04.02.22) Jeg har nogle nye idéer og rettelser omkring kundedrevne virksomheder, som jeg har skrevet om ude i kommentarerne til ``Web 2.1''-sektionen under ``Draft to first\ldots'' nedenfor. Jeg har også en ny idé til en lykke-/gavn-valuta / donationskæde/-forening, som står under ``(old)A donation economy around open source contributions''-sektionen lige over nævnte ``Web 2.1''-sektion. Sidstnævnte idé er ret interessant, og jeg burde egentligt skrive den ind her i den renderede tekst. Men det synes jeg dog ikke, jeg har tid til nu. Og det er nemlig heller ikke en idé, jeg umiddelbart vil gå videre med, for jeg synes alligevel, jeg har bedre og mere simple løsninger til de samme problemstillinger, som den idé prøver at løse. Men hvis disse andre idéer nu skulle fejle, så kunne det dog muligvis være værd at støve nævnte idé af (den fra ude i kommentarerne lige over ``Web 2.1''-sektionen) og se på, om dette så kunne blive en idé, der var værd at prøve at implementere. Men pointen er lidt, at en sådan implementation vil kræve en hel del, og desuden kan idéen også føre uforudsete ting med sig, så jeg ville altså personligt langt hellere satse på mine andre idéer først, hvis jeg skulle vælge. 

Ok, og lad mig så lige gå tilbage til mine nye idéer omkring kundedrevne virksomheder (kd.v.'er). For dem vil jeg nemlig \emph{gerne} lige prøve at opsummere her. \\
(05.02.22) Den nye version af idéen minder meget om den, jeg skrev om ovenfor i denne sektion, men nu skal kd.v.'en ikke længere ``tilbagekøbe'' aktier, som jeg formulerede det. I stedet udløber aktierne bare efter en vis periode, og stemmemagten, der følger med aktierne, daler også løbende mod 0, jo tættere man kommer på udløbsdatoen. Men ellers skal aktierne i hele den periode, hvor de er gældende, bare svare ret meget til gængse aktier, hvor man altså får ret til en tilsvarende andel i alle potentielle afkast, der måtte komme fra overskuddet, og man kan også købe og sælge dem frit. Men nye aktier kan (på nær eventuelt i en indledende fase, som jeg vil vende tilbage til) kun blive udstedt til kunder, der har betalt for et produkt eller en service, og aktiens størrelse skal kun afhænge af (og være proportionelt med) den betalte pris (hvilket virksomheden jo skal have styr på, bl.a.\ fordi der jo skal betales skatter af denne pris, alt efter hvorhenne i verden salget sker). Til sidst bliver det altså sådan, at enhver generation af kunder så altså kommer til også at blive en generation af aktionærer. Perioden af aktierne skal så gerne være så store, at den samlede anlægskapital *(eller de samlede `anlægsaktiver' rettere) kan sælges for væsentligt færre penge, end hvad aktionærerne (i.e.\ de tidligere kunder eller dem, de har solgt deres aktier til) samlet set har i vente ved at fortsætte forretningen. *(Og perioderne skal i øvrigt også gerne være så lange, at det sagtens kan betale sig at investere som aktionær, fordi man selv vil nå at få udbytte af sådan investering (jeg tænker f.eks.\ selv at den kunne starte med at være noget a la tyve år).) *[(10.02.22) Hm, følgende del af denne paragraf er ikke længere helt sand i henhold til min seneste version af idéen, men jeg lader det lige stå, for det beskriver alligevel meget godt tanken bag hele idéen:] Og aktionærerne er altså helt frie til at drive virksomheden som de vil, og kan i princippet drive den helt som gængse virksomheder. Men selv i dette tilfælde vil det så være sådan, at al ``udnyttelse'' af kunderne og al ``presning af citronen'' (i.e.\ hvis nu virksomheden ikke har nok andre konkurrenter på markedet til at forhindre dette) så opvejes af, at kunderne så får helt den samme magt med tiden. Så hvis vi f.eks.\ tager en forretning, hvor efterspørgslen ikke ændre sig, og hvor tingene kan køre på samme måde, så vil kunderne altså med tiden få lejlighed til at ``presse'' helt de samme penge ud af de næste generationer af kunder, og på den måde vil man næsten kunne sige, at ingen således bliver udnyttet rigtigt (i hvert fald i princippet). *(Nå ja, og en vigtigt ting at nævne er så også, at fordi mængden af aktionærer med tiden vil blive rigtig stor og indeholde rigtig mange (vedvarende) kunder, så vil der dog sikkert blive lidt mindre incitament samlet set til at ``presse citronen'' helt så meget som forrige generation, selv hvis der fortsætter med at være manglende konkurrence på markedet.) Men hvis det nu går særligt godt for virksomheden eller særligt dårligt for virksomheden i en periode, så kan pågældende generationer af aktionærer altså tjene hhv.\ lidt flere eller færre penge tilbage, hvilket jo er sundt, for så bliver de gode beslutninger fremmet, akkurat som for en gængs virksomhed. *[(10.02.22) Og sidste del af denne paragraf %er heller ikke længere helt sand 
passer heller ikke længere rigtigt 
i min seneste version:] %...Nå, måske skal denne del faktisk alligevel netop med i idéen..
Og for at det ikke kan løbe løbsk, og at der ikke går pyramidespil i den, så skal aktieperioden automatisk justeres løbende ud fra, hvor stor salgsværdien af den samlede anlægskapital er, samt hvor store overskuddet/afkastene er på samme perioder, sådan at man sikrer sig, at den værdisum, som aktionærerne samlet set har i vente altid vil være tilnærmet en vis faktor større end 1 ganget med værdien af anlægskapitalen. Selvfølgelig vil den faktiske faktor så kunne variere en smule, men på denne måde vil det altså tilpasse sig sådan, at man kommer i nærheden af den ønskede faktor. %*Hm, men måske kan det være meget smart i nogen tilfælde med en ret stor faktor.. Hm.. 

Så langt, så godt, men hvad når der lægger andre værdier end bare anlægskapitalen i virksomheden? Hvad hvis virksomheden f.eks.\ har visse IP-rettigheder, som en del af dens assets (det danske navn for det glipper lige.. *(`aktiver,' må det være..))? Tja, hvis nu virksomheden har et ret begrænset område, så kunne man jo oplagt bare gøre det, at man sørger for også at udregne (og dette skal virksomheden være juridisk bundet til at gøre, og til at det skal gøres lødigt (og gerne med tredjeparter involveret)) ikke bare anlægskapitalens salgsværdi men også IP-rettighedernes. Og så kan de to ting bare lægges sammen, og virksomheden kan justere aktieperioderne automatisk (igen noget som den er juridisk bundet til at gøre) efter denne sum i stedet. (Og bare for lige at præcisere, så er virksomheden også juridisk bundet til nævnte ting såsom kun at udstede aktier til kunder og at størrelsen af dem på lødig vis skal sættes ud fra prisen.) Men for en ikke nær så begrænset virksomhed virksomhed har jeg også en anden idé angående IP-rettighederne, så bl.a.\ særligt vil gøre idéen velegnet til en Web 2.1-/3.0-virksomhed. Men nu vil jeg faktisk lige afbryde mig selv, og vende tilbage til denne idé senere, for jeg skal vist lige *(nævne nogle flere ting)\ldots %..Hm, eller var det egentligt ikke bare lige konceptet om et delay på aktieudstedelsen, jeg mangler at nævne (før vi når til idéer omkring IP-fonden osv.)..? 
%... Okay, bemærk, at der også stadig er spørgsmålet, som jeg indsatte ovenfor, nemlig: "Hm, men måske kan det være meget smart i nogen tilfælde med en ret stor faktor.."(?) Og nu har jeg også tænkt lidt mere. Jeg er lidt kommet frem til en måde, jeg vil foreslå, for at virksomheden/erne kan opsplittes (nemlig ved at foreslå og indstemme en slags binært opdelende (opsplittende) service-definitioner..), men nu har der også lige meldt sig nogle andre lidt flydende spørgsmål, som jeg lige skal "summe" over en gang.. 
%(06.02.22) Ah, faktoren er ikke så vigtig, for nu kom jeg lige i tanke om, at kunderne jo altid bare kan sælge deres aktier med det samme, hvis de vil.. Hm ja, så den kan vel altså godt bare være ret stor til at starte med..? Og så kan man jo altid bare.. hm, købe flere aktiver.. Nå, der er lige flere ting, jeg lige skal tænke over en gang.. ..Hm, kunne man måske bare gøre, så at alle undervirksomheder kan stemme om at forøge eller formindske faktoren, og at det så bare kun må ske rigtigt langsomt..? (..Lige for at præcisere, så handler det altså om "faktoren" for, hvor lange aktieperioderne ligesom skal være sammenlignet med aktiverne vs. overskudsafkastene..) ..Hm vent, det er da ikke nok bare at sørge for, at perioden tilpasser sig, for da den jo vil gøre det med et vist delay, så kunne aktionærerne vel lave pyramidespil ved at sætte priserne mere og mere i vejret..?.. Så skal der også være et cap på priserne på en eller anden måde..? ..Ah, eller på afkastene fra overskuddet? ..Hm ja, der kunne måske godt.. vent lidt.. ...Skat! Der får vi vores ønskede naturlige dæmper på disse pyramidespilsmuligheder!:D Det kan ikke betale sig at pumpe prisen op, så at kunderne kommer til at betale en højere og højere del af prisen til aktierne, basalt set, og ikke til, hvad produktet/servicen egentligt er værd, for skattetrykket gør jo, at dette ikke vil kunne betale sig! Sjovt, at skat på disse forskellige måder (der var jo også den med, at det ikke kan betale sig for aktionærerne at "hyre sig selv") faktisk kommer og løser problemerne på en simpel måde. Og skatten skal jo betales alligevel (tror ikke, der findes et land uden skat.. tjo, eller nogen virksomheder skulle eftersigende slippe, men der kan dårligt være tale om skatter på salg (eller på lønninger), kan der?. ..), og så er det jo bare dejligt, at den faktisk også har en positiv virkning på virksomhedens stabilitet. ..Okay, så vi behøver altså ikke yderligere dæmpninger/caps på priserne eller afkastene.. Og hvad med den (omtalte) faktor der..? ..Og hvad i øvrigt med tanken omkring "forælder-services," når det kommer til den opdeling, jeg vil vende tilbage til på et tidspunkt (bare så jeg lige husker det)?. ..  ..Hm, angående omtalte faktor, vil priserne alligevel ikke bare hele tiden ligesom skulle sættes ud fra den nuværende faktor..? 
%... Ah, måske har jeg det endeligt.. Jeg ved det ikke endnu (om jeg har den), men jeg fik lige den idé, at man måske bare kunne sørge for, at perioden mere bare sættes ud fra omsætningen (og altså stadig fra aktiverne), så man bare altid sørger for, at et salg af aktiverne kun vil købe en værdi for hver af aktionærerne, som kun er en vis mindre procentdel af, hvad de har bidraget med til omsætningen.. Jeg skal lige tænke lidt mere over det, men måske dette kunne være en nem løsning.. 
%... Nå, nu tror jeg muligvis, jeg har en nemmere løsning. Man sørger i stedet bare for, at alle nyudstedte aktier får et minimumskrav.. Hm, vent lidt, for skal man så lægge de penge til side nødvendigvis, eller hvad..? ..Hm, eller kunne man lave noget med, at det kravet skal betales, men hvor man også får muligheden for at forlænge aktierne, hvis de ikke kan betales.. Hm.. Nå, jeg troede, jeg havde den, men jeg skal så tænke lidt mere.. ..Tjo, men skal man ikke bare sige, at det er op til virksomheden, om pengene lægges til side, eller om man vil satse og investere dem? ..Det ville være godt, hvis virksomheden ikke tog store risici, men det må man vel så bare regne med, at aktionærerne også er enige i, eller hvad..? ..Jo, ved du hvad.. Det må man bare.. Okay, så tror jeg umiddelbart, jeg holder fast i idéen. Og den er altså, at givet den periodelængde og den aktiv-formue, der er på tidspunktet for en akties udstedelse, skal der automatisk fastlægges et minimumsbeløb, som langsomt \emph{skal} betales af i løbet af perioden. Man kan så i princippet regne dette som en ekstra fast afkastgevinst udover hvilke afkast, der ellers måtte være, og denne vil så alt andet end lige skulle lægges direkte til prisen for den/det købte service eller produkt. Og i princippet kan virksomheden så bare lægge de penge til siden, medmindre de har et sikkert sted at investere dem henne. Dette minimumsafkast skal så sættes ud fra en forskrift, som sørger for at aktionærerne altid vil være mere investerede i virksomheden, end hvad de ville kunne få ud af at sælge virksomhedens aktiver. Aktieperioden skal gerne være relativt lang, måske tyve år eller noget i den dur, og i øvrigt må der også gerne være et lille delay på --- måske et enkelt år eller lignende, fra at man har gjort købet, og til at aktien træder i kraft (så det måske bedre kan betale sig at træffe gode investeringsvalg, uden at skulle dele). Men både aktieperioden og også det eventuelle delay skal kunne justeres løbende, men bare meget langsomt, af aktionærerne. Så skal f.eks. kunne stemme om at forøge aktieperioden, men så skal dette ske gradvist og meget langsomt (således at den ikke kan nå at stige særligt meget relativt til den nuværende aktieperiode). (Og hvis man justere delayet, så skal justeringens fart også måles relativt til aktieperioden; det skal nærmest alt, kan man sige..). Desuden kan aktionærerne også stemme om ændringer i de tidligere vedtægter for firmaet (bl.a. f.eks. hvilke regler og hvilke tredjeparter man bruger til at vurdere aktivernes størrelser osv.), man så skal sådanne ændringer bare først kunne træde i kraft lang tid efter. Og her taler vi altså om vedtægterne, der danner fundamentet for virksomheden. Ja, og for det første så skal sådanne ændringer først kunne ske, når man er stort set helt inde i fase 2. Og tiden før de kan træde i kraft skal altså være væsentligt mere end en aktieperiode --- hellere (som udgangspunkt, for dette kan jo også ændres) noget a la to aktieperioder, skal der gerne gå. Og hvis man indstemmer en sådan ændring i fremtiden, så skal der selvfølgelig også være en god tid efter, hvor den ændringen kan anulleres igen. Men ja, dette gør altså, at virksomheden kan ændre i selv dens fundamentale sætninger (altså efter vi er helt inde i fase 2), hvilket jo sikkert vil være ret sundt for den. Og som jeg har lagt op til, hvad angår delay- og aktie-perioder, så kan der altså også være mindre fundamentale ting ændringer, der ikke behøver helt ligeså lang ventetid, men nok stadig skal ske så langsomt, at alt for store ændringer ikke kan ske på én aktieperiode. ..Så jeg tror altså nu, at dette bliver det grundlæggende i, hvordan virksomheden kommer til at fungere. Jeg har også nogle flere ting, jeg skal nævne, bl.a. omkring virksomhedsopdeling og om, at jeg nu ikke længere er sikker på, at man behøver den centrale fond, selv ikke til min tænkte web-virksomhed.. Men det kan jeg nemlig bare skrive om i den renderede tekst. ..Jo, jeg kan også lige nævne, at til denne idé er det jo smart, at man som regel kun betaler skat af givinsterne fra aktier, for dermed behøver kunderne ikke at betale mere skat, selv hvis man altså regner det som en aktie, de køber, altså denne obligatoriske aktie, der giver krav på et minimumsafkast. Og den skat, der så er, er bare rigtig smart, for det gør at tingene ikke vil løbe løbsk; på et tidspunkt vil priserne sættes så lavt som muligt, sådan at kunderne-og-aktionærerne (når disse mængder er ret ens) skal betale mindst muligt skat. Og det kan man så altid have i baghovedet, når man sætter priserne. Men kan således ikke f.eks. bare presse priserne op og sige, "nå ja, I kan jo bare selv sætte de samme priser, når jeres tid kommer," til kunderne, for kunderne ved jo så, at dette ikke passer, for skattetrykket gør, at det ikke vil kunne lade sig gøre i længden (ikke når der ikke samtidigt er ny innovation, der kan retfærdiggøre de forhøjede priser). 
%(07.02.22) Okay, jeg skal lige komme med en rettelse angående minimumsafkastene. I reglen skal pengene investeres, for ellers vil de jo bare tilføje til aktiverne, og så vil det ikke gå op. Tanken er jo, at folk skal være investerede i virksomheden med dem, så i princippet skal pengene jo gå til at betale for aktiv-delen af aktierne. Og det er meningen, at afkastpengene skal tages fra indkomsten ved fremtidige salg. Og så skal jeg så bare lige overveje, hvad der sker, når anlægsaktiverne forøges eller formindskes.. ..Hm, og særligt skal jeg overveje faren ved, at de bliver mindre værd, enten fordi efterspørgslen på dem går ned, eller hvis de bliver beskadiget, f.eks. i et uheld.. ..Hm, men nu fik jeg lige pointeret, imens jeg skrev nu her, at pengene jo i \emph{princippet} går til at købe aktiv-aktierne, så et oplagt spørgsmål er da, hvad med at mit nyligt foreslåede minimumsafkastkrav i stedet bare erstattes med et krav om, at kunderne, når de køber varer eller services, simpelthen bare bliver tvunget til at købe (eller rettere virksomheden bliver tvunget til at sælge) en del af aktiv-aktierne med i prisen.. Hm, men det er jo allerede det, der skete i forvejen, så lad mig lige se en gang.. ..Hm, men kommer det så til i virkeligheden at give sig selv?.. Det ser lidt sådan ud.. (For priserne vil så bare kunne opjusteres helt naturligt, fordi brugerne også får aktier med, der giver ejerskab over aktiverne (i fase nr. 2)..) ..Hm, og hvad så med perioden, \emph{kan} man så bare sætte denne frit..?.. ..Altså ved udstedelsestidspunkt.. ..Det kan den jo i princippet.. Men okay, så skal vi nok egentligt bare sige, at aktie-perioden er fastlagt ved udstedelsestidspunktet, og at den kan stadig kan justeres stille og roligt af aktionærerne, men det behøver ikke at gå nær så langsomt, som jeg lagde op til i går. ..Cool.:) Forresten vil jeg også lige nævne, at stemmeretten jo godt kan have et andet (måske kortere) delay, end hvad retten til del i afkastet er. Og selvom jeg på et tidspunkt i går beskrev det som om, der var flere forskellige aktier i spil, så skal der i virkeligheden altså kun være den ene (for ejerskabet skal selvfølgelig følge med stemmemagten (alt andet end lige)). Ok. Så er jeg vist klar til at skrive dette underafsnit (uoformelle; uden overskrift) færdigt..:) ..Hov, men lad mig lige hurtigt tænke over en gang, om det skal være sådan, at aktierne skal udløbe før man sælger dem videre, eller om størrelsen af andelen, aktierne har krav på, bare kan variere alt efter, hvor mange der bliver udstedt.. ..Hm, eller udløber aktierne bare i takt med, at der bliver solgt/udstedt nye.. nå ja, mon ikke det bliver svaret..(?) Hm.. ..Altså de skal udløbe efter en fast forskrift uanset hvad, men spørgsmålet er, om man på en måde kan få det til at passe med udstedelsen via salget..? ..Hm, mon ikke bare man sådan set skal lade dem udløbe, og så have en buffer af udløbne aktier, som man så har pligt til at sælge en andel af hele tiden sammen med vare-servicerne..? Og hvis man så sælger.. Ah, man kunne måske bare medregne bufferen i aktierne, så hvis aktiverne sælges, så vil man altid få 1 \% eller hvor meget, det nu er, oveni, fordi bufferen af udløbne aktier ikke tæller med og dermed overgiver deres krav til resten af aktierne. Dette system kræver så bare, at man meget nøje kan tvinge virksomheden til bedst muligt at tilpasse det, så bufferen har samme størrelse hele tiden.. ..Hov, men nu glemmer jeg lidt, at aktierne jo gerne skal udstedes proportionelt med prisen, så dermed kan prisen jo ikke direkte afhænge af aktieværdien.. Hm, kunne man så fastsætte aktiestørrelsen ud fra de tidligere handler i samme periode på en eller anden måde..? ..Hm, det må vist lige blive lidt gåturstænkeri, det her.. ...Nå, det blev ikke en særligt lang tur, men det er også fint. Jeg fik tænkt over, at pointen med denne virksomhedstype jo er, at aktierne, der gør krav på overskudsafkast alle skal have en passende udløbsdato, samt at retten i første omgang går videre til kunderne. Vi kan i princippet (om end alt andet end lige ikke i praksis (men det kan jeg i øvrigt lige genoverveje om lidt også)) dele aktierne op i to, nemlig i afkastrettigheder og i aktivejerskab.. Ja, og fordi man altid kan sælge aktivejerskabet videre, selv hvis man ikke rent faktisk sælger aktiverne i virksomheden, så vil aktiens værdi altså blive en sum af værdierne fra de to dele af aktien.. ..Hm, og nu fik jeg så den naturlige idé at spørge mig selv, om man mon så bare netop kan opsplitte de to aktier, og ligesom inkudere aktiv-aktien i prisen, og lade den anden (altså afkast-aktien) afhænge af den resulterende pris sammenlignet med, hvad andre kunder betaler.. Hm..? ..Hm, det kunne da egentligt være, det ikke var en helt dum idé.. ..Tja, men man vil jo nok gerne have, at de to ting skal følges ad, og at aktivaktierne også bare er fastsat proportionelt med prisen, i det mindste for en lille periode.. ..Hm, men kommer det så ikke også til at passe, for hvis alle priserne sættes op for en periode, så vil.. ja, så vil der ikke være problemer med cirkulære afhængigheder, for så vil aktiestørrelsen for en kunde altid bare være proportionel med, hvor meget denne giver.. Ah, og nu kom jeg også lige i tanke om, at idéen omkring et aktie-delay jo vil blive ret tam i min ellers seneste idéversion (i.e. som den var i går): Det er faktisk \emph{rigtigt} gavnligt for investeringslysten, at aktiernes pris-proportionalitetsfaktor kun sættes for små perioder (eller måske ud fra en varierende kurve), for hvis så aktionærerne lige for en markedsidé, der giver helt vild (og langvarig) vækst i virksomheden, så skal de ikke dele gevinsten på samme måde, med alle de kunder, de tiltrækker, så snart den korte delay-periode er udløbet. Men i dette billede er det i højere grad netop de aktionærer, der fik indført den vækstskabende idé, der får gevinsten. Så jeg vil altså gå tilbage og holde fast i min tidligere idé om, at aktieudstedelsen skal variere så den er nogenlunde konstant for en given lille periode. ..Hm, hvad skal vi kalde disse små perioder for ikke at forveksle dem med f.eks. aktie-perioderne?.. Eller skulle det som nævnt hellere være ud fra en slags kurve (som måske kunne fungere efter et slags moving average..)..? Hm, et slags moving average nok bedst, så man ved, hvad man køber. Ok.. ..Hm, og så var det måske ikke dårligt, det med at have den lille buffer, så procenterne altid går op.. Tja, eller man kunne også gøre andre ting, men er der nu nogen ting, der kunne være smartere..? ..Tja nej, det er sikkert meget smart med den buffer der. For så kan dette faktisk blive en.. eller lad mig lige se.. Kan det mon blive en naturlig måde at skabe de små perioder på..? (Fordi man ligesom så kan forøge prisen, jo større bid der tages af bufferen..) ..Hm, jeg kom lige til at tænke på, at hvis man laver priserne alt for tidsafhængige, så kunne det ske (og det har jeg slet ikke forstand på), at der kunne være nogle lovmæssige krav, der gør at man så virkeligt skal kunne servicere/eksekvere handlerne hurtigt og effektivt, sådan at alle får fair handelsmuligheder.. (Så måske det ville være smartere at holde sig til de "små perioder" i starten..?) ..Ja, det er sikkert bedst med små perioder med faste priser og faste aktie-proportionalitetsfaktorer.. ..Og hvis man så hurtigt for solgt ud af, hvad man regnede med, så må man så bare tage hul på en ny omgang varer/servicer, og så sætte en ny aktie-proportionalitetsfaktor og nye priser som gælder fra den nye omgang af. ..Hm, og "aktie-proportionalitetsfaktoren" er så i virkeligheden triviel, for det handler så bare om, hvor meget man lige netop regner med at sælge. Idéen om en aktie-buffer er så stadig god, og hvis man så skal have hul på en ny vare/service-omgang tidligere end regnet med.. ah, og i øvrigt bliver det dermed heller ikke faste "små perioder," for hver periode afhænger så bare af, hvor hurtigt man får omgangen solgt --- eller indtil man erklærer den slut og sætter nye priser.. Men ja, hvis man får solgt varerne/servicerne hurtigt, så vil man altså tage lidt af bufferen.. Hm, jeg kom lige til at tænke på, at man kunne have en lille automatisk variation i aktie-perioderne, som gør, at der altid vil blive frigjort lige mange aktier pr. tidsenhed (groft set) i den anden ende, når aktierne udløber (og altså så der hele tiden udløber lige mange aktier ad gangen). ..Nå tilbage til bufferen.. ..Hm, men man sørger jo for det første bare for altid at sætte en brøkdel mindre end 1 til salg af bufferen for en periode. Og samtidigt skal der dog sørges for at sælge så meget af den for hver periode, at dens procentdel holdes nede.. Hm, det bliver måske lidt tricky med den del, faktisk.. (Det må jeg lige have tænkt godt og grundigt over på sigt..) ..Hm, og dog: Man kan vel bare lave en grundsætning om, at størrelsen af den næste "omgang" altid bliver fastsat automatisk, så den kommer til at følge med størrelsen af omsætningen, og at man så også bare kræver, at en stor nok del af bufferen klistres på denne omgang, således at hvis omsætningen fortsætter i samme gear eller mere, så vil bufferen være inden for den ønskede (flydende maks-)grænse, når omgangen er færdig. ..Og aktionærerne kan, så vidt jeg kan se, bare selv finde frem til en politik for, hvor stor en del af bufferen man sælger i hver omgang, og også hvor meget man sætter til salg, når den er formindsket (når salget er gået stærkere end forventet). ..Hm, og virker virksomheden så efter disse principper?.. (Og ift. både overskudsafkast og til aktivsalg skal den samtidige buffer så bare regnes fra, så kravet dermed bare fordeles til de andre aktier. Og bufferstørrelsen kan bare være endnu en ting, som man kan justere, men \emph{kun} meget langsomt sammenlignet med en aktieperiode). ..Jeg kan jo lige summe lidt over det en gang.. 
%... Hm, jeg bør forresten også lige brainstorme over det med at opdele/opsplitte virksomheden en gang. Jeg havde nogle nye tanker om at lave "service-definitioner" beregnet til at opdele mængden af services (også her inkl. produkter) i to. Men nu tror jeg muligvis, jeg går lidt væk fra den idé igen.. ..Jeg tror bare, jeg hælder til at sige, at hvis man kan splitte en (under)virksomhed (yderligere) op i nul eller flere "forælder-servicer" og en eller flere "børne-servicer," hvormed jeg altså snakker om servicer (børn), der bruger mere grundlæggende servicer (forældre) for at fungere, og hvis en lødig udregning projekterer, at dette ikke vil koste det store ift. omsætningen, så skal der automatisk kunne vedtages afstemninger om sådanne opsplittelser. Og afstemningerne kræver så ingen gang en majoritet for at opslittelsen bliver godkendt, men hvis de projekterede omkosninger altså er lave nok, så skal det være nok, hvis bare.. ja, og nu bliver den udregning, jeg lige har i hovedet lidt kompliceret: Jeg tænker nu, at man skal trække den største børne-gruppe.. okay, det bliver meget kompliceret selv at forklare.. Jeg må næsten kunne finde på et mere simpelt princip, end det, jeg skulle til at skrive om.. ..Tjo, et bedre princip kunne sikkert bare være at se på, hvad omkostningerne (de projekterede (i.e. forudsete)) er ift.\ den smalede omsætning, at så bare sige, at jo lavere omkostninger, desto mindre procentdel af stemmerne kræves det for at udføre opsplittelsen. ..Og igen kan alle disse regler være noget, man kan ændre for hver (under)virksomhed, selvfølgelig bare hvor man endnu en gang altså ikke kan ændre det med et trylleslag, men først (medmindre altså man også ændrer dette faktum på sigt) kommer til at indføre det efter lang tid. ..Ja, det var da en god, simpel løsning at fremføre for opdelingerne.. (..Og det er en rigtig god ting at have med, sikkert for mange tings skyld, men bl.a. især for at visse kundegrupper ikke potentielt kan tyranisere andre kundegrupper, fordi de er i overtal.) ..Nå ja, og så kan jeg ligeså godt også lige nævne her (og ikke bare i den renderede tekst), at når en opdeling så sker, så starter aktierne bare med at være ligeligt fordelt på tværs af alle involverede (eventuelle børne- og forælder-)servicer, og herefter kører virksomhederne bare uafhængigt af hinanden efter det fundametnale kd.v.-princip --- eller rettere det eventuelt modificerede kd.v.-princip, hvis nu den pågældende virksomhed har ændret i dens grundsætninger i mellemtiden. Og hermed vil f.eks.\ en mindre kundegruppe, der overvejende bruger én af servicerne, altså langsomt automatisk købe de andre kunder mere og mere ud af den undervirksomhed, der nu står for denne service efter opdelingen, så aktierfordelingen i sidste ende kommer til at passe med, hvem der forbruger hvor meget af servicen. Umiddelbart fint.:) 
%... Ah, det kunne faktisk være meget godt med en central enhed omkring IP-rettigheder, især når det kommer til et web 2.1-3.0-firma, for fordelen kunne nemlig lige nøjagtigt være, at det kan gøre det langt mere attraktivt for nye skabere at tilføje ting, der bygger videre på andres IP. For hvis nu kd.v.'en skulle gå ned på en eller anden måde, så ville man ellers komme ud i en situation, hvor man skal til så at forhandle på kryds og tværs ift. at sælge IP-rettighederne igen (dem der ikke er udløbet) til andre virksomheder. Og hvis disse forhandlinger så endda går helt i kludder, så kun indholdet gå hen og blive låst. Men hvis man sørger for at have en central i toppen af virksomheden, der kun varetager IP'en, og som undgår enhver fare for at kunne gå konkurs, så kan man måske herved opnå en situation, hvor skaberne (især altså dem, der bygger oven på andres IP) ikke behøver at frygte konkurser særligt meget --- eller i hvert fald ikke skal bekymre sig om en masse forhandlinger og potentielle deadlocks, hvis de sker. Centralen kan så fungere som en slags forælder-service for resten af virksomheden, men muligvis dog en, der fungerer efter nogle andre, mere simple grundsætninger end resten af virksomheden (inkl. afstikker-/under-virksomheder).. ..Hm, spørgsmålet er så, om denne central så bare nærmest skal være i en evig fase 1 nærmest, eller.. eller/og i øvrigt også om undervirksomhederne selv har gavn af at varetage IP'en. Okay nu steg det mig lige lidt over hovedet, men jeg havde alligevel lyst til at skrive, hvad jeg tænkte, for det er rart, når nu jeg skal tænke videre over det..  ...Hm, måske ikke helt dumt med en evig, og særlig, fase 1, hvor man udelukkende sælger aktier til nye skabere.. Og hvis skaberne heller vil have penge, så må de sælge deres.. skabninger.. til kd.v.'en/erne i stedet for til omtalte central.. Hm.. (..Indskudt: Reglerne omkring, hvor mange stemmer der skal til for en splittelse, kan nok godt gå hen at blive ret tricky, men heldigvis behøver man ikke have dem på plads fra starten af. Kd.v.'en kunne således godt i princippet bare starte med en simpel regel, såsom at det f.eks. bare kræver 33 \% af stemmerne til en splittelse, og så kan man bare ændre denne grundsætning over tid til noget mere sofistikeret.. *(Hm, og i øvrigt kunne man måske så have en lidt mindre procentsats for at kunne stemme et forslag op til afstemning..)) (Og lige hurtigt indskudt også: Jeg kunne fint stoppe med at kalde det forælder-/børne-servicer, ikke bare fordi det er et forvirrende navn, men det duer heller ikke, fordi "servicer" (eller rettere undervirksomheder) sagtens kan handle på kryds og tværs af hele samlingen. Så man kan i stedet bare se det som spaltninger, hvor der så bare ofte kan være et asymmetrisk afhængighedsforhold.)
%(08.02.22) Nå, jeg bliver ved med at få en god følelse af, at den er der om aftenen, og så vende tilbage om morgenen med skepsis. I går aftes/nat var der nogle klokker, der begyndte at ringe, og her i badet til morges kom jeg frem til, at min seneste version fra i går af idéen også er lidt for usikker. Jeg tror ikke længere, at skat er helt nok til at stabilisere virksomheden.. Så nu tænker jeg lidt over dette.. ..Og lige inden jeg begyndte her på tasterne, kom jeg så frem til, at det måske heller ikke bliver så nødvendigt, når vi når godt ind i fase 2, at aktionærerne faktisk prøver at opnå et godt afkast.. For tanken er jo, at stemmemagten gerne i høj grad skal ende på kundernes hænder, og hvis dette lykkedes, så har de jo allerede motivationen for at tage de gode beslutninger. Og mit mål er jo netop at sigte efter noget, hvor kunderne ender med at få den hovedsagelige magt (og sådan set gerne ret hurtigt i fase 2), men derfra behøver man jo så ikke så meget aktiespekulations-motivation for at ledelsen kan træffe de gode beslutninger.. Ja, kunde-aktionærerne kunne jo vel bare sørge for at virksomhedsledelsen belønnes efter gode beslutninger, og så er det ret meget det.. ..Hm, vigtige tanker. Lad mig lige tænke over det hele en gang.. ..Hm okay, den interessante del bliver vel nu, hvad vi kunne se som den nye fase 2, men som er en mellemfase imellem første fase og den gamle "fase 2.." For i fase 1 kører virksomheden bare efter gængse regler på et fundamentalt plan, og i sidste ende skal man nok bare indsætte en regel, der simpelthen bare siger, at overskuddet skal være 0.. *(..eller måske noget fast..?) Hm, og hvis overskuddet kommer fra salg af aktiver, skal de gå til aktionærerne, og hvis det kommer fra indtægter, skal det helst i denne (endelige) fase gå tilbage til kunderne igen, så de ender med kun at betale for omkostninger inkl. lønninger.. ..Men hvad så med mellemfasen..? ..Hm, skrev lige "..eller måske noget fast..?" ovenfor.. ..Nå ja, der skal jo være et fast afkast, for om ikke andet skal selve anlægsaktiv-delen (..eller bare "aktiv-delen"..) af aktien jo betales tilbage løbende i takt med, at den udløber.. ..Nå ja, noget andet, jeg kom i tanke om, var, at systemet med de små perioder godt måske kunne ende med at presse priserne op og op, for jo færre kunder, jo mere vil salgene bare blive værd.. Hm, passer dette, eller kan systemet godt holde..? ..Hov, kunne idéen mon egentligt ikke bare blive noget med at sørge for, at afkastet altid bare er proportionelt med omsætningen..? Hm, det svarer jo også lidt til noget, jeg har tænkt før; jeg har bl.a. tænkt, at det skulle være proportionelt med lønningerne.. ..Hm, men skal afkastet så ikke bare være proportionelt med indtægterne, og så er det det..? (Lad mig lige prøve at se tilbage på, hvorfor jeg tidligere gik væk fra idéen om fastsat afkast (forhåbentligt bare fordi jeg fik en "bedre" idé..)..) ..Tja, jeg har nok dengang bl.a. bare tænkt, at det var smartere med noget, der fungerede lidt mere frit, hvis det alligevel sagtens kunne lade sig gøre.. Og så har jeg måske også overset nogle ting, som jeg ved nu.. ..Men ja, det bør jeg nok hellere tænke videre over på et senere tidspunkt, hvis det er, for nu er det nok bedre bare lige at overveje, hvordan denne version af idéen så bliver.. ..Nå ja, der \emph{er} jo faktisk gode grunde til, at afkastene ikke bare kan være helt fastsat --- og de må på den anden side heller ikke kunne varieres helt frit af ledelsen --- men det er jo netop det jeg føler, jeg har fod på, nu hvor aktionærerne bare kan justere det relativt langsomt (og når vi alligevel allerede er i en fase, hvor aktionærerne er ret meget kunder selv, og derfor selv alligevel vil have mest gavn af, at overskuddet bliver så lille som muligt (så der ryger mindst muligt til skat)).:) ..Så tanken er altså lidt nu, at kunderne i mellemfasen bare får et afkast i vente, som fastsættes af fase 1-aktionærerne, og at denne så derefter er rimeligt konstant, men kan justeres op og ned af (kunde-)aktionærerne efterfølgende.. Uh ja, og her tænker vi specifikt på overskudsafkast-delen af afkastet; "afkastet" der kommer fra, at aktiv-delen af aktien skal betales tilbage, det bliver bare en separat ting i udregningerne.:) Ok.. 
%Det er sikkert ikke dumt at sørge for, at "de små perioder"/"salgsomgangene" eller moving average-kurven ikke bliver alt for korte og/eller følsomme for fluktuationer.. (Det skal nødig blive sådan, at der går for meget spekulation i for kunderne at købe, når salget ellers er lavt..) ..Hm, men nu behøver man vel ikke "salgsomgange" i helt samme grad, gør man..? Der er jo stadig aktiv-delen af aktierne, som nok gerne skal fungere, som jeg tænkte det i versionen i går.. ..Hm, og skal man stadig have den buffer der, eller \emph{kan} man mon egentligt gøre noget, der er mere simpelt..? ..Tja nej, det er måske egentligt et udmærket (og heller ikke vildt kompliceret) system.. 
%..Hm, jeg er umiddelbart ret glad for denne nye version af idéen.. Det har jeg så dog været for alle de seneste versioner (som nærmest har været én ny for hver af de sidste dage..), så det er jo nok også meget klogt, at jeg lige venter med at udgive den økonomiske side af web 2.1-idéen (som jo netop alt andet end lige bliver min kd.v.-idé i den version, jeg nu kommer frem til (som jo måske kan være denne)).;) ..Men ja, jeg føler bare, at jeg får ret godt styr på det, nu med mine seneste idéer (og inkl. tankerne fra i dag).. ..Så det er jo godt. Og så må tiden vise, om det holder, eller om jeg når at blive skeptisk igen inden samme tid i morgen.x)^^ Det føles dog lige som om, der er nogle små løse ender, hvor jeg lige skal beslutte, hvordan tingene så kommer til at være, men muligvis noget, jeg kan klare her i middag/eftermiddag. 
%... Det skal være muligt for virksomhederne at sælge produkter og services på forskud (som en slags pre-order eller kupon). Så skal aktierne udstedes til købstidspunktet, præcis som var produkter-servicerne leveret da, men omkostningerne i udregningerne må man så vente på at få svar på, til de rent faktisk er gjort. Og disse omkostninger skal altså så regnes som var de betalt ved købstidspunktet. På den måde kommer folk til at kunne investere, hvis nu de kan se, at virksomheden(s omsætning) vil vokse meget i den kommende periode, men hvor der ikke bliver udbudt nok produkter/services i nutiden, til at disse kunder kan investere alene ved køb af disse. Og det gør ikke noget, hvis virksomheden så ved dette får sig en pukkel af skyldte produkter/servicer, for så må man bare lige investere en anelse i produktionsmidler m.m., så man kan begynde at levere nok til, at puklen kan "betales af." Og alle parter må jo så bare være klar over, at man selvfølgelig skal passe på med at investere for meget i ekstra produktionsmidler, for hvis produktionen kommer til at overstige den egentlige efterspørgsel (i den nære fremtid) for meget, så vil man jo skulle sælge dem igen og nedskalere, når puklen er ved at være "betalt af." Men dette fænomen skal alle parter (i.e.\ virksomheden og kunderne/investorerne) jo bare gøre sig klart, så alle er klar og kan foudsige denne effekt (og justere deres ledelses- eller handelsstrategi efter det). ..Og ja, så ser idéen altså umiddelbart ud til at holde..:) (Krydser fingre, for det er jo (beviseligt) slet ikke sikkert.^^) Men det virker altså bare ret fornuftigt, det her..:) ..Jeg ville også lige nævne/understrege, at nye kd.v.'er i samlingen ikke behøver kun at komme fra spaltning af gamle. Der kan sagtens løbende opstå nye virksomheder, som selv starter med deres egen fase 1. Dette kan jo oplagt være, hvis man finder på nye opfindelser eller anden IP, og derfor gerne vil sørge for at starte fra fase 1 som opfindere/skabere (for dermed ligesom at tjene mest på det).:) 
%(09.02.22) Okay, her til morgen/formiddags/middags har jeg tænkt over nogle tanker omkring, hvordan man gøre tingene bedre ift. IP-rettigheder, og er vist på kanten af at have fundet frem til en god løsning.. Jeg kan for det første sige, at løsningen indebærer, at der nu bliver en slags specielle aktier, der gives til folk, der bidrager med IP (hvilket i starten vil sige, at de sælger deres IP til virksomheden, og altså så får aktier i betaling). Og så forestiller jeg mig, at disse aktier netop kan vare længere end de normale (kunde-)aktier. For kunde-aktiernes varighed skal jo bare afhænge af, hvor mange aktiver er i spil. Og ja, nu vil jeg altså muligvis lidt tilbage til \emph{ikke} at blande IP-aktiver sammen med de andre aktiver.. For tanken er ligesom, at det meget vel kunne blive mest naturligt, hvis IP-aktierne fik lov potentielt set at vare længere end de normale aktier. Og folk der bidrager med IP til virksomheden, vil jo nok gerne også have en vished (især måske f.eks. i en web 2.1/3.0-virksomhed) om, at grundsætningerne ikke bliver ændret til noget, der bryder principperne i kd.v.'en. For man må nemlig gerne regne med, at én motivation fra skabernes side for at bidrage til kd.v.'en også er, at denne bevarer sin integritet. Så derfor mener jeg altså, at det kan blive smart at kunne give IP-aktionærerne, som jeg vil kalde det her og i det følgende, en forstærket vetoret, når det kommer til foreslåede ændringer i virksomhedes grundsætninger. Og for at kunne implementere dette, så er det jo smart, at IP-skaberne kommer med i systemet som specielle aktionærer. Og dette har sikkert også andre fordele, som jeg vil prøve at argumentere for nu.. ..Ja, jo, det vil jo bare være rigtigt smart bl.a. i den tidlige periode og særligt i, hvad jeg har kaldt "fase 1," hvis IP-skaberne kan belønnes med aktier, så deres løn kan afhænge af virksomhedens vækst (og ikke f.eks. skal betales med det samme). (Selvfølgelig kan man finde på andre måder også, men at betale i aktier kunne være én god, naturlig måde at opnå dette på.) Og det kan så også være meget naturligt, tror jeg, hvis IP-aktierne i reglen bliver mere længerevarende end de normale (for denne virksomhedstype (ikke "normale" ift., hvad eksisterer i verden pt.)) aktier. Og for at slå tre fluer med ét smæk kan de altså som nævnt også gives en forstærket vetoret, når det kommer til ændringer i kd.v.'ens grundsætninger. Nu tænker jeg så, at forholdet imellem, hvor mange "normale" aktier og hvor mange IP-aktier, der er, bare kunne være.. en af de ting, der varierer langsomt relativt til én periode.. ..Hm, og man skal jo så huske, at perioden for de aktier i udgangspunktet vil være en del længere.. ..Ah vent, kunne man ikke gøre sådan, at forholdet sådan set ender med at blive frit, og at det samtidigt også ender med at være udelukkende bestemt af kunde-aktionærerne?:D For jeg tænker nemlig lidt nu, at "fase 1" (som jeg måske ikke har forklaret super meget endnu, i hvert fald ikke her i denne omgang --- og som jeg heller ikke har tænkt så meget over endnu i denne omgang) måske kan smelte sammen med "fase 2+," men nu kom jeg så lige på, at forskellen måske netop kunne ligge i, hvem der bestemmer over, hvor mange IP-aktier skal udstedes og til hvem..(:D..) ..Ja, okay, så jeg tror altså muligvis ikke "fase 1" bliver, at virksomheden kører som en gængs virksomhed, men bare at IP-aktierne (og (inkl.) start-aktierne, som de første iværksættere starter med..) i starten har ligeså stor medbestemmelse over, hvordan nye IP-aktier skal udstedes og sælges (for IP), og så skal det jo nævnes, at der jo vil være meget få kunde-aktionærer i starten pr. design, så dermed har de første iværksættere og skabere altså dermed den hovedsagelige magt her. ..Og den tanke, jeg altså lige har i hovedet nu her umiddelbart, er altså, at IP-aktiernes magt over udstedelsen/salget af nye IP-aktier så skal formindsket efter en fast forskrift i tid (og altså ikke bare i og med, at der kommer flere kunde-aktionærer til, hvad der jo dog også vil..).. Hm.. (..Og ja, så skal IP-aktionærerne dog altså alligevel beholde en anseelig vetoret, når det kommer til grundsætningerne, selvom de altså mister den anden magt..) ..Hm, og jeg kalder det \emph{IP}-aktier, men i virkeligheden er man jo også fri til at belønne skaberbidrag, der ikke har stærke IP-rettigheder omkring sig. Dette kan bl.a. handle om de første iværksætterbidrag, og det kan også handle om skaberbidrag, der kommer på et tidspunkt, hvor man har udviklet et godt og troværdigt system omkring "bagudbelønning." ..Hm, og lad mig lige tænke en gang over, hvad jeg ellers skal sige, og hvad jeg ellers skal tænke over.. ..Hm, det gør så, at de første kunder jo dermed kommer til, når de køber varer/servicer, at spekulere i (indirekte), hvor gode IP-aktionærerne er til at drive virksomheden, og ikke mindst også hvor mange IP-aktier.. de når at udstede.. men vent, skal man ikke bare sørge for, at IP-aktionærerne kun kan sælge af deres egne aktier..? ..Hm nej, det giver ikke vildt god mening, medmindre at man på en eller anden måde kan vide, at kunderne også så vil være tilsvarende motiverede for at sælge af deres aktier.. ..Hm, det bliver jo måske nok bare noget med, som jeg var ved at sige, at de første kunder så bare ikke kan vide, hvor meget IP-aktionærerne har lyst til at sælge af deres egne og kundernes aktier.. ..Hm, men kan et vigtigt problem så ikke være potentialet for, at IP-aktionærerne bare sælger til sig selv(..!)..? Hm.. ..Hm, men man \emph{kan} da egentligt forresten godt så sagtens bare sige, at nyudstedte IP-aktier ikke må formindske kunde-aktiernes krav, og at det dermed bare svarer til (hvad man så også kan implementere det som), at IP-aktionærerne bare kan sælge af deres egne aktier.. Hm, men det ville nu stadig være rart, hvis kunde-aktionærerne så også vil være ligeså motiverede for at sælge af deres aktier.. ..Hov, det vil da egentligt altid kunne blive et problem, at majoriteten kan sælge til sig selv, også bare hvis man ser på IP-aktionærgruppen i sig selv.. ..Så skal alle aktionærer ikke bare kunne.. vetoe udstedelser.. Hm..? ..Hm, der er godt nok der ved det, at selve aktiernes værdier nok vil falde på markedet.. og kunder vil bakke mindre op, hvis der sker sådanne korrupte ting, men er det mon nok..? ..Især hvis virksomheden først råder over mange IP-aktiver..(?) ..Hm, det virker umiddelbart som et svært problem at løse, andet end at man dog, måske egentligt med ret stor løsningsgrad, bare kunne sige, at de tidlige (IP-/iværksætter-)aktionærer bør prøve på officielt at kortlægge alle sådanne "korruptionsvektorer," og bør så simpelthen bare love rent ud, at de vil gøre alt hvad de kan for at undgå sådant. Hm, og man kan selvfølgelig også forsøge at binde sig juridsik til ikke at korrumpere på visse måder, hvor det er til at gøre (uden al for stort et arbejde).. Så de første IP-/iværksætter-aktionærer bør altså opskrive et manifest for, hvordan de vil belønne nytilkomne IP-skabere (med aktier), og disse skal så også selv gerne skrive under på samme manifest (og det skal alle eventuelle købere, som aktierne bliver solgt videre til).. ..Hm, jeg skal vist faktisk lige tænke en del over dette.. Går mig vist også lige en eftermiddagstur nu her.. 
%... Okay, for det første må man nok bare sige, at IP-aktionærerne ikke har ret til at sælge af kunde-aktionærenes aktie(-krav) og omvendt. (..) Og hvis nu en af grupperne er mere tilbageholdende med at investere i ny IP, så må begge grupper jo bare tage dette med. Sådan er det bare. Og ja, apropos "sådan er det bare," så må man også bare tage de nævnte "korruptionsvektorer" med, som jeg også lagde op til. Man kan altså med fordel prøve lige at kortlægge dem offentligt, således at parterne officielt kan love, at det ikke vil ske, og så må man ellers bare sige, at hvis det sker, så må kundeopbakningen og dermed aktieværdien falde, og samtidigt skal man ikke overleve disse vektorer særligt længe, for når kunderne alligevel hovedsageligt styrer virksomheden, vil disse vektorer ikke længere være et problem / en risiko. I øvrigt kom jeg også på gåturen på, at hver (under-)kd.v. jo selv skal kunne bestemme, hvordan det vil fordele omkostningerne på deres produkter. Det tror jeg nemlig er smartest. Så en virksomheden skal altså hermed bare udregne sine samlede omkostninger, og så have bestemt på forhånd, hvordan de skal fordeles pr. produkt/service. ..Og dette må jo så bare være en fastsat (for en "salgsomgang") pris på hver vare/service.. ..Og man kan så eventuelt benytte sig af, at man jo godt kan sørge for at skyde lidt over ift. omkostningerne, for så kan man bare betale pengene tilbage i form af et tilsvarende større aktieafkast.. ..Hm, og jeg ville nævne, at jeg tror, det er en meget god idé, hvis man lægger op til, at begge aktionær-grupper hver især skal beslutte sig for protokoller/forskrifter for, hvor mange IP-aktier der udstedes på en lille periode og til hvad. Dette kan så blive starten på "bagudbelønning," og så kan man dog altid udskifte det med et andet system, hvis det er.. I øvrigt skal jeg så lige se på, hvad begge grupper har af muligheder for at \emph{udstede} nye IP-aktier, og altså ikke bare sælge gamle, hvad kun IP-aktionærerne kan.. Jeg tænker jo lidt, at det skal køre lidt på samme måde med, at der hele tiden bliver nye aktier til rådighed for udstedelse for begge grupper, og at de så altså her ikke bare skal udstede dem til kunderne på triviel vis alt efter de betalte priser, men hvor det i stedet er en demokratisk proces, hvor de skal gå til.. ..Ja, og også hvor stor en buffer man får samlet sig osv.. ..Det er dog ikke sådan her, at summen skal være konstant; mængden/størrelsen af IP-aktierne må f.eks. gerne aftage samlet set ift. kunde-aktierne i omløb.. ..Hm, mon man skulle gøre noget med, at IP-aktionærernes udstedelser ikke forøger IP-aktiemængden, men at kunde-aktionærernes godt kan..? ..Eller at de ikke bare \emph{kan}, men bare \emph{gør} dette..? ..Hm, det virker jo egentligt ret fornuftigt, og så bliver det også ret nemt at sikre sig, at tallene går op.. ..Ja.. ..(I øvrigt skulle jeg lige slå fast, at IP-aktierne også bare giver et vist krav på en procentdel af omsætningen over deres gældende perioder.) ..Hm, er systemet så nu, som det bør være..? (Lad mig summe over det..) (..IP-aktionærerne skal forresten altså selvstændigt kunne styre også (og endda hvor dette kan justeres/varieres over relativt kort tid sammenlignet med andre ting (og det samme gælder for kunde-aktionærgruppen)), hvor mange nye IP-aktier udstedes.. hm, da det udstedelsen jo ikke(?) ændrer på de eksisterende IP-aktiers perioder..?) Hm, skal det være, som jeg sluttede forrige (..eller altså den lige før denne sætning) parentessætning af med?.. ..Ja. Ja, det skal bare være sådan, at nyudstedte IP-aktier fra begge aktionær-grupper bare æder af de krav, som pågældende aktionærgruppe har. ..Hm, det virker altså faktisk, som nogle ret nice idéer, de her..:) Det kunne godt gå hen at ende med, at dette bliver det system (eller måske tæt på), jeg vil foreslå..:) 

\ldots

(10.02.22) Okay, jeg har lige tænkt en hel del videre over idéen, hvilket kan læses om i noterne ude i kommentarerne imellem denne paragraf og forrige fra d.\ 5. Lad mig har prøve at præcisere og forklare om min seneste version af idéen (som jeg umiddelbart er ret glad for\,:)). For det første skal aktieperioden ikke længere justeres automatisk, men skal bare kunne justeres ud fra en demokratisk proces blandt aktionærerne (måske via stemmerepræsentanter i virksomhedsledelsen eller måske mere direkte), men kun langsomt. Så man skal derfor bare sørge for at starte med lange aktieperioder, så man kan forøge anlægskapitalen uden at skulle forøge priserne vildt meget. Men om ikke andet kan man altid gøre dette: Virksomheden bliver i denne version bare bundet til at sælge aktier sammen med alle deres produkter og services (hvor aktiestørrelsen til et givent tidspunkt altid er proportionelt med prisen sammenlignet med andre køb på samme tidspunkt) i sådan en grad, at det altid vil passe nogenlunde med, hvor mange aktier udløber på det tidspunkt. %..Nå ja, så bliver der faktisk en lille korruptionsvektor.. hm, i at man giver kunderne større og større perioder.. hm, men er dette ikke en rimeligt vigtig/alvorlig korruptionsvektor..? Hm.. ..Hm, for så kan aktionærerne jo hæve priserne ved at hæve perioden så meget som muligt, og dermed kan de selv tjene et større overskud, og kd.v.'en vil så samtidigt bevæge sig mere og mere mod gængse forhold, idet aktieperioderne så vokser og vokser.. ..Hm jo, det er en ret alvorlig ting, og det bliver i øvrigt ikke voldt meget bedre, hvis man sætter et delay på, hvornår i fremtiden perioden først kan justeres fra, for en stigning her vil alligevel også føre til en højere mulig salgspris (for de nyudstedte aktier på pågældende tidspunkt).. ..Nå, jeg bliver virkeligt bare ved med at finde hager hver eneste dag, hva.. (..Selvom i går godt nok ikke så meget var en fejltagelse, men mere at jeg lige skulle have bedre fat om IP-delen af systemet..) ..Hm, og en løsning kunne ikke bare være, at man ikke må regne aktiesalget med som en del af overskuddet..? ..Hm tja, det bliver nu nok ikke så simpelt, nej.. ...Hm, måske skal man bare sørge for, at perioden følger aktivernes samlede værdi..? ..Hm, men vent, måske ville det da være en god idé netop at trække aktiernes værdi fra overskudsudregningen?.. ..Hm.. ..Hov! Åh, jeg har jo faktisk tænkt på disse ting, har jeg ikke?xD Der skal jo være en aktiv-del af aktien, som bare skal justeres op og ned, og som altid bare betales tilbage eksakt, medmindre virksomheden går konkurs (hvorved de samtidige aktieindehavere altså vil kunne gøre krav på deres del af konkursboet). Og så skal aktie-periodens maksimale variation jo ellers lige netop bare være meget lille sammenlignet dens størrelse. ..Ja, og måske kan man så også gøre klogt i.. hm, at sætte en maskimal størrelse, man hvad med at man bare i stedet starter stort? Så kunne man altså ikke bare have en fast periode fra start af..? ..Hm, og hvis prisen for aktiv-delen bliver for stor, så kan man jo bare sælge aktierne til.. Hm, men en vigtig del af idéen er jo, at kunderne gerne skal have råd i almindelighed til at kunne beholde aktierne selv og ikke bare sælge dem videre med det samme.. ..Hm, jeg ville ønske, jeg vidste hvad aktiv-pr.-indkomst-ratioen generelt.. hm, men må vel.. hm, nå nej, det afhænger vel meget af, om man regner aktierne med i aktiverne.. Hm, lad mig lige prøve at se, om jeg kan finde svar på sådanne spørgsmål.. ..Hm, nå ja, man skal jo bare regne IP-rettigheder (samt også "brand" og etablerethed generelt (også hos andre virksomheder I guess)) osv. med, og så er det egentligt nogenlunde simpelt derfra. Men hvad kan man så sige om aktiverne vs. forventet overskud..? ..Tja jo, hvis man netop regner IP og brand osv. med, så vil de følges ad.. ..Hm ja, pointen er vel faktisk lidt, at der (muligvis) er noget cirkulært i udregningen, hvorved at aktierne jo kun får mere værdi, hvis der er et højere overskud at tjene, men det overskud skal komme fra indtægterne, så på den måde er aktieværdierne jo altid begrænset af indtægterne.. ..Tja, eller der er godt nok lige anlægsaktiverne, som nok godt kan variere meget ift. overskuddet (i hvert fald alt efter hvad jeg ved, hvilket nemlig er stort set ingenting).. Hm, men man må nu næsten kunne forudsige det alligevel.. ..Tja, eller måske ikke (også fordi "økonomien" (i et land) jo kan svinge meget..).. ..Hm, og det ville ikke give mening, hvis man bare kunne opsplitte kravene lidt på aktivsalg og overskudsandel, så de kan opdeles på forskellige aktier, og hvor man så bare kan lade stemmeretten følge overskuds-aktierne i høj grad..? ..Hm, det var da egentligt en fin idé, umiddelbart.:)(..!..) ..Hm, skulle man så kun gøre det for anlægsaktiver, eller bare for aktiver generelt..? ..Hm, jeg tænker faktisk, at det bare kunne være for anlægsaktiver, for de andre værdier i virksomheden må vel netop som nævnt være stærkt korrelerede med det forventede overskud..?:) ..Ja!:D ..Okay, lad mig prøve at fortsætte med den renderede tekst, så.
\ldots

%*[Jeg udkommenterer lige de følgende tre paragrafer, for de bidrager ikke længere rigtigt til forklaringen. ..Hm, jeg indenterer dem også lige, så man bedre kan se, hvad de var. (Så fra "Nå, nu tænker jeg dog lige ..." var teksten altså allerede udkommenteret til at starte med.)]
	%Nå, nu kom jeg lige i tanke om endnu et problem, som jeg tror, jeg fik løst (se evt.\ ude i kommentarerne). Nu tror jeg faktisk, jeg vil lade anlægsaktiverne (specifikt) være en separat del af det hele. .\,.\,Hm, eller skulle man bare, som jeg også lagde op til ude i kommentarerne, sige, at aktierne nu bare kan deles over i to, hvor stemmeretten så følger med overskuds(afkast)-delen (plus andre aktiver) af aktien, og hvor anlægsaktiv-delen så bliver sin egen del for sig selv.\,.\,? .\,.\,Ah ja, og så kan måske bare ved denne spaltning så også få muligheden for at annullere (eller forlænge) udløbsdatoen på aktien, og så samtidigt bare droppe (eller formindske) afkastene på den samtidigt.\,. Ok.\,. 
	
	%Og så vil jeg nemlig nu bare sige, at aktieperioden (for ``kunde-aktierne,'' som altså kan være forskellige fra visse ``IP-/iværksætter-aktier,'' som jeg vil vende tilbage til) bare vil være helt konstant i udgangspunktet for virksomheden (og dennes ``grundsætninger,'' så at sige, som den er juridisk bundet til at følge). Samtidigt gør hver af aktierne også bare krav på en konstant del af omsætningen, hvilket altså også bare skal fastsættes fra starten af. Og pointen er så, at man sagtens bare kan vælge en anseelig procentdel, så som f.eks. 5, 10 eller 20 \%, hvad ved jeg, for selv hvis man sætter den ret højt, så vil det aldrig blive helt unfair.\,. i hvert fald ikke på sigt, for så snart væksten flader ud for virksomheden, så vil kunderne alligevel bare selv tjene ca.\ det samme på deres aktier, som de selv betalte til de daværende aktionærer i sin tid. 
	
	%Udover at gøre krav på en hvis procentdel af overskuddet (ganget med størrelsen på aktien (hvis afhængighed jeg har nævnt kort, men som jeg også vil komme tilbage til) og med aktieperioden (hvis ``overskud'' altså måles i penge pr.\ tid)), så gør aktierne altså udmiddelbart også krav på del i pengene, når aktiver sælges af virksomheden. Og kravet her skal også bare være proportionelt med aktiens ``størrelse.'' For hvert salg skal virksomheden så vurdere, hvor meget aktiverne er værd på pågældende tidspunkt og lægge dette (ganget med ``størrelsen'' af aktien sammenlignet med den samlede mængde) til købsprisen. I takt med at aktien så udløber skal dette selvsamme beløb dog så løbende betales tilbage. Så man kan derved altså ikke rigtigt spekulere i denne del af handlen (på nær måske hvis man ser på risikoen for konkurs.\,.), men det svarer bare til et depositum, som betales tilbage igen. 
%Nå, nu tænker jeg dog lige lidt igen over, om skat virkeligt også er nok til, at det ikke løber løbsk. Jeg har lige læst mine ovenstående kommentarnoter fra de sidste dage igennem, men kunne ikke lige finde det helt tydelige svar på, hvorfor jeg mente, at det gik.. Det kan være, der er noget jeg lige har glemt, eller også fik jeg bare aldrig tænkt tanken nok til ende. (Der var godt nok det med, at aktierne alligevel relativt hurtigt ville komme over på "kundernes hænder," men...) Nå men, solen skinner virkeligt dejligt nu, og nu har jeg jo en god undskyldning for at gå ud i den og tænke lidt over dette. ...
%(11.02.22) Okay, det blev alligevel til en del tænkeri i går, men om aftenen (tidligt) kom jeg så endelig frem til, at man jo bare kan sige, at det altid bare er en hvis procentdel af overskuddet, der skal gå til afkast til aktionærerne. Så vil aktionærerne herved være interesserede i at øge overskuddet, og samtidigt tror jeg ikke, der bliver problemer med positivt feedback på priserne, for.. ..Jo, for lad os sige, at alle kunder og aktionærer bliver enige om at sige, at prisen lige hæves med en faktor nu og fremover.. Kunderne vil så betale x mere for deres aktier og overskuddet vil stige med x. Men samme aktionærer kan så kun få del i den faste procentdel af x i form af afkast og resten vil virksomheden fortsat skulle råde over. Disse penge kan så bruges til at investere eller købe varer for, og hvis man gør dette klogt og har muligheden, så kan man muligvis få dem til at yngle, så der bliver et større overskud. Men ellers kan man også nedjustere sine priser, så man får underskud. Men hvorfor skulle man gøre det? Jo, det kunne man gøre, hvis man alligevel har noget grænsen for, hvad der giver mening at investere, og \emph{hvis} man er nået et punkt (og dette var min indsigt der i går aftes), hvor virksomhedens aktionærer i høj grad selv er kunderne, og derved vil få gavn af dette. Jeg kom så også frem til i sengen i går nat, at man jo nok bør indføre skarpe regler, der gør at virksomheden ikke må forsøge at lave favorable handler til aktionærerne specifikt. Så derfor vil det faktisk være en god idé, hvis hver (undervirksomhed) har en offentligt kontrakt om, hvilke services og produkter, den vil levere, og hvad den kunder er, og hvem der må få specielle tilbud (\emph{hvis} altså nogen overhovedet må det), og så skal virksomheden holde sig til dette og altså aldrig lave favorable handler med aktionærerne (hverken ift. anlægsaktiverne osv. eller ift. indkøb og salg af varer/services). Og sådanne regler vil nemlig være gode uanset hvad, man gør, for der vil ellers altid kunne findes korruptionsvektorer, hvor en majoritet af aktionærerne pludselig spiller fejt spil og vinder en fordel ift. en minioritet af aktionærerne. Men tilbage til de grundlæggende aktier: Jeg skal så lige tænke over en gang, hvad man gør med anlægsaktiv-aktierne, og ved salg af aktiver.. ..Af men ved salg tjener virksomheden bare penge til virksomhedskassen. Så så længe virksomheden ikke må kunne lave favorable handler med aktionærer, så kan virksomheden altså bare enten miste værdi, få værdi, eller beholde samme værdi ved en handel med aktiverne. Aktionærerne kan altså godt lave dårlige handler, men det kan de ikke få noget ud af (hvis de juridisk bunde regler om ikke at favorisere altså følges godt), medmindre handlerne kommer kundeskaren generelt til gode, og hvis aktionærmængden altså er lig denne i høj grad. Og ellers vil det jo udelukkende være motiverede til at lave gode handler, der kan forøge overskuddet. Fedt. Ej, det virker egentligt som en rigtig simpel løsning (som man kunne have været kommet til ret hurtigt, hvis man var heldig/heldigere), men det tog mig alligevel noget tid at nå hertil..^^ He. Men til gengæld indeholder idéen jo også nogle mere komplicerede ting, når vi når til IP-aktierne, den tidlige fase, og til spaltningerne osv., så det har selvfølgelig været sundt, at jeg har måtte tænke over mange forskellige ting for at få det til at virke.. ..Nå, 7, 9, 13, for dette kan jo ligeså godt bare være endnu, endnu en gang, hvor jeg lige om ikke særligt længe finder ud af, at denne version af idéen heller ikke går.x).. Men sådan er det jo bare.:) ..Idéudvikling tager bare lang tid, må man erkende. ..Heldigvis er det jo noget af det, jeg nyder allermest, men nu kunne jeg dog også snart godt bruge at blive færdig og få en pause for denne omgang.. (..Men nu kan der ikke være mange uger igen (..7, 9, 13)..) ..Hm, ikke at på en måde nogensinde har haft en pause rigtigt (i mit voksne liv), hvilket er sjovt at tænke på.. ah, der har lige været nogle få (korte) stille perioder, men eller har jeg altid haft gang i et eller andet, som jeg har følt kunne blive til noget stort.. ..Men ja, nu har jeg haft et helt år med ret fokuseret idéudvikling (på tasterne), så nu glæder jeg mig altså også ret meget til (og det har jeg også gjort i ret lang tid nu) at blive færdig med det..! ..Nå, det var et sidespring.. ..Jeg kunne også lige sige, at jeg kom på i går, at man også eventuelt kunne lave en regel om, at der går en aktie-skat (måske altså hvor aktien formindskes tilfordel for alle andre aktier i spil), hvis en kunde-aktionær vil sælge sin aktie videre, og man kunne i øvrigt så også (og det kunne man også gøre alligevel) måske lave et system med, at stemmeretten også bliver formindsket med en faktor på videresolgte aktier. Disse få tiltag (og skatten og/eller stemmefaktoren kunne jo sagtens være små) kunne så hjælpe til at sikre, at aktierne i sidste ende netop \emph{vil} findes meget på kundernes hænder, hvilket jo lidt er hele pointen. ..Nå, lad mig prøve så at skrive videre på den renderede tekst med denne seneste version af idéen.. ... Nå, jeg ved godt, det er ret hovmodigt, især når man tænker på den seneste historik, men jeg tror faktisk, at jeg nogenlunde har styr på det nu, og jeg vil gerne hurtigst muligt i gang med blockchain-skriveriet, så jeg tror faktisk bare, jeg prøver at delegere dette til aftenarbejde (ikke at det nødvendigvis vil tage særligt lang tid), og så vil jeg altså gå videre nu med det næste. ..Nå, det tager jeg straks i mig igen, for jeg kan mærke, at jeg ikke så godt kan koncentrere mig om at sætte min hjerne ind i de baner igen, når nu jeg endnu ikke har afsluttet dette helt. Så det må jeg bare prøve at gøre nu med det samme.. ..Nå, eller også har jeg bare i det hele taget lidt svært ved at koncentrere mig i dag.. (sov også ret dårligt i nat..) Nå, koncentrationen kommer sikkert til mig igen om lidt.. 
%(12.02.22) Nå, min koncentration til at skrive kom ikke rigtigt igen, men om aftenen fik jeg tænkt lidt tanker omkring det og fik fundet nogle tilføjelser. Det ville også være smart hvis aktionærerne skulle betale den samme procentdel, hvis de laver et underskud. Og her tager man jo så bare af depositumspengene. Hvis pengebeholdningen så på et tidspunkt når til, at man kun lige kan betale "depositummerne" minus procentdelen tilbage, så må virksomheden melde konkurs og gøre dette. Hvis man på en eller anden måde får tabt flere penge end dette, så må virksomheden jo bare betale tilbage, hvad den kan. Men normalt vil virksomheden altså stoppe alle foretagender, når man når til kun at have depositummet tilbage minus procentdelen. ..Hm, kunne depositummet egentligt ikke godt variere, for hvad kunder for tilbage afhænger jo bare af overskuddet (og af den fastsatte procent)?.. Jo, det kunne man nok godt sige: Så depositummet pr. handel (og altså pr. pris) fastsættes bare i små handelsperioder. Og størrelsen af det afgører så egentligt bare, hvornår virksomheden vil gå konkurs, skulle den få underskud. En anden ret vigtig tanke, jeg fik i går aftes, var at man faktisk nok bare \emph{skal} prøve at regne IP-rettigheder m.m. med som aktiver, så man derved regner dem med som en del af overskuddet. Alternativt sagde man jo bare, at værdien alligevel kan ses på det efterfølgende overskud, så hvis bare aktieperioderne er lange nok, så gør alligevel ikke en vildt stor forskel. Men pointen er dog så, at det jo altid kun vil være \emph{mere} fair, hvis man alligevel sætter sig for at prøve at måle IP-aktivernes værider. Og dette kan nemlig bare gøres afterfølgende. Det gør således ingen gang noget, hvis selv aktier, der er "udløbet," først bliver afregnet noget tid efter, og at det skyldte beløb bliver betalt der. Man kan med andre ord altså sagtens sætte et delay på ubetalingerne, så man bedre kan vurdere IP-aktivernes værdi. Og jo, man vil aldrig kunne gøre dette 100 \% præcist, men pointen er altså, at det alligevel vil være \emph{mere} fair end ellers, og man kan jo sige for det første, at en fejlvurdering heller ikke vil gøre det \emph{vildt} store stadigvæk, hvis perioderne bare er lange nok, og samtidigt kan man også sige, at alle jo alligevel har de samme forhold. Men ja, "mere fair" er jo kun godt, kan man sige.. Det eneste direkte negativ er, at det så lige vil koste lidt at lave de analyser, men det tror jeg på, er til at betale. Og nu får man så en situation, hvor "overskuddet" i princippet kun stiger eller falder, hvis virksomhedens ledelse (som styres af aktionærerne) laver en god eller en dårlig handel (eller hvis analyseenheden på en eller anden måde ikke kan vurdere ordentligt bagefter, om det var en god handel eller ej, hvilket nok egentligt er ret usandsynligt i det store hele). Og så bliver aktionærer altså ikke længere "straffet" kortvarigt for at have lavet en investering, heller ikke i IP (m.m. (bl.a. PR)). I princippet kunne man derfor også nu godt droppe aktie-delayet, men jeg beholder nu alligevel den idé, for det kunne nu godt ske, at den vil være tiltalende for folk alligevel.. Men det er dog rart, at man har muligheden for at droppe den. Så kan man nemlig også få en situation, hvor kunder for mere lyst til at joine et opsving.. tja.. Måske. Om ikke andet så beholder jeg nok idéen, for man kan jo altid bare justere delayet til 0.. ..Tja, og ellers kan man jo altid bare foreslå den.. Ok, så det var altså nogle af de ret vigtige idéer, jeg lige kom frem til i går aftes (tidligt på aftenen). Så kom jeg også lige frem til, at man måske skulle gøre, så et stort nok underskud kunne udløse mulighed for grundsætningsændringer.. ..Hm ja, og så skal man jo også bare kunne ophæve denne regel igen.. ..Tja, måske skal jeg lige tænke noget mere over dette. Men en tidligere idé, som måske er endnu mere vigtig, er, at aktionærerne skal have lov til at ændre i grundsætningerne --- gerne stadig med et kæmpe stort delay på, hvornår de indtræder --- når først aktionærmængen og kundemængen er har tilstrækkeligt stort et overlap, og at de to altså er tilstrækkeligt meget lig hinanden.. ..Sikkert slet ikke dumt.. ..Ok. Lad mig så prøve, at skrive den renderede tekst igen.

\ldots

(12.02.22) Okay, igen har jeg tænkt idéen lidt om, hvilket man evt.\ kan læse om ude i kommentarerne imellem denne og forrige paragraf. Lad mig prøve at starte lidt forfra også, og altså opsummere de ting, jeg har slået fast hidtil (og som stadig er beholdt i min nuværende version af idéen). 

Det skal altså gælde at alle aktier udløber efter en (lang) periode, og at nye aktier skal udstedes med samme rate, som de udløber, men \emph{kun} til kunder, der har købt en af de udbudte varer eller services, som virksomheden officielt leverer. Størrelsen på hver udstedte aktie skal så være proportionel med den betalte pris fra kunden sammenlinget med alle andre kunder på samme tid. Jeg vil om lidt vende tilbage til, hvordan man får dette til at fungere, samtidigt med at udstedelsesraten holdes (nogenlunde) konstant. Men inden da vil jeg bl.a.\ pointere, at virksomheden altså skal have en officiel beskrivelse af, hvilke varer og servicer, den leverer. Disse beskrivelser kan godt laves, så de er rigtigt inklusive, og man kan i øvrigt også gøre dem åbne overfor visse justeringer og tilføjelser, men i bund og grund skal de altså bare beskrive, hvem der er virksomhedens kunder, og hvad disse kunder tilbydes af produkter og services overordnet set. Hm, lad mig lige sige allerede, at grunden til, at man skal have disse beskrivelser altså bare er, så man f.eks.\ kan vælge, at der ikke udstedes aktier til instanser, man f.eks.\ sælger udgåede anlægsaktiver til osv.\ (for der vil jo også være mange handler, som ikke gøres med ``kunderne'' specifikt i en virksomhed), og derudover er beskrivelserne også for, at man bedre kan sikre sig, at aktionærerne aldrig bliver favoriserede i de handler, virksomheden gør, hvilket nemlig meget gerne skal være forbudt. Eksempelvis må virksomheden ikke give aktionærerne favorable tilbud som kunder osv. *(Og de må heller ikke favoriseres i nogen som helst ikke-kunde-handler, bare lige for også at slå det fast.) Og dette skal virksomheden altså være bundet juridisk til (fra start af) ikke at kunne gøre. 

Kunder der får udstedt aktier til sig, skal også betale et lille beløb oven i prisen, som fungerer som et slags depositum. Virksomheden kan selv vælge at justere dette depositum løbende, og de betalte depositummer skal altid regnes fra, når man skal udregne, hvad virksomhedens overskud er (hvilket man skal). Depositummerne bliver betalt tilbage løbende i takt med, at udløbsdatoen for aktien kommer tættere og tættere på. Samtidigt bliver der dog også betalt eller opkrævet et beløb til/fra aktionærerne løbende, som er proportionelt med, hvor stort et overskud eller underskud man har opnået i perioden. Hvis aktionærerne altså er lykkedes med at forhøje virksomhedens samlede værdi (alle penge og aktiver medtaget i udregningen), den råder over, så skal de altså have udbetalt en vis procentdel af overskuddet i form af afkast på aktien (og proportionelt med aktiens størrelse). Procentdelen af overskuddet, som betales i afkast, skal være fuldstændigt fastlagt fra virksomhedens start af. Og det samme skal længden af aktieperioderne. På den måde bliver alle kunde-generationer nemlig lige meget værd (i hvert fald hvis man dividerer med størrelsen af pågældende kundemængde) set ift., hvor meget af over-/underskuddet, de har krav på (at få eller at betale). Bemærk så, at underskuddet jo bare betales fra depositummet, og at størrelsen af depositummet derfor i bund og grund bare bliver afgørende for, hvor stort et underskud virksomheden kan tillade sig at få, før den bliver nødt til at dreje nøglen om og melde konkurs. Dette vil så ske, når virksomheden kan se, at den vil nå til et punkt, hvor den kun akkurat kan betale depositummet minus den fastlagte procentdel tilbage til alle aktionærer. Bemærk også at en lang aktieperiode medfører, at kunderne kan betale et tilsvarende lille depositum hver især, uden at det ændre på, hvornår virksomheden går konkurs (.\,.\,eller hvad.\,.\,?).\,. .\,.\,Hov nej, det passer ikke helt. Der gælder i stedet bare, at med et stort nok depositum, så kan virksomheden godt nå at gå lidt i minus, ift.\ hvad den skylder i depositumtilbagebetalinger, for det underskud, der bragte den i minus, vil jo skulle betales delvist af aktionærerne (hvorved omtalte skyld jo derfor også falder i takt med at underskuddet vokser (men selvfølgelig ikke ligeså hurtigt, så de to vil mødes på et punkt, hvilket så altså bliver konkurspunktet)). Ok.

I min allerseneste version af idéen her, mener jeg så, at man endda skal prøve at regne selv IP, og endda også PR osv., med som en del af virksomhedens samlede værdi, når man udregner under-/overskuddet for en periode. Det kan så godt være, at man så skal sørge for at bruge en pessimistisk udgave af disse vurderinger, når det kommer til konkurspunktet, så man er mere sikker på ikke at overstige den faste procentdel (som trækkes fra i dette tilfælde), når kunderne/aktionærerne skal have deres depositummer betalt tilbage i dette tilfælde. (Ikke at man selvfølgelig regner med, at virksomheden vil gå konkurs.) Men i alle andre tilfælde, hvor man nemlig har tiden til at regne godt på det, så tror jeg altså, det vil være gavnligt, hvis man havde en (muligvis uafhængig (og i hvert fald uafhængig af aktionærerne --- på nær at de måske kan have magt til at ændre den med en (meget) lang periode til at ændringerne så træder i kraft)) instans til at analysere virksomhedens værdihistorik. Instansen skal så f.eks.\ ikke regne det som et negativt bidrag til overskuddet, hvis virksomheden f.eks.\ gør en handel, der skaffer den mere IP, men som koster i andre aktiver, hvis IP'en alligevel var mere værd for virksomheden. Tværtimod kan det muligvis regnes som bidragene positivt til virksomhedens værdi, hvis det (efterfølgende) vurderes, at handlen forøgede virksomhedens samlede værdi (i.e.\ i form af de penge, den sandsynligvis kan tjene på IP'en). Selvfølgelig vil sådan en løbende analyse koste virksomheden lidt, men på den anden side vil jeg hævde, at sådan analyser alligevel generelt vil være sundt for virksomheden rent økonomisk også. Og selvfølgelig vil sådan en instans ikke kunne udregne alting eksakt, selv ikke hvis den altså bare kan udregne tingene efterfølgende. Men min pointe er dog, at dette tiltag alligevel altid kun vil gøre det hele \emph{mere} fair end det mere simple alternativ, hvilket nemlig bare er i bund og grund at ``straffe'' alle aktionærer end smule, hvis de investerer i noget, der ikke har en konkret værdi, der er nem at regne på (såsom IP). Selvfølgelig vil aktionærerne stadig kunne tjene på disse investeringer alligevel, hvis bare perioden er lang nok --- og \emph{hvis} deres aktie altså ikke snart udløber --- men det vil nu være langt mere fair, hvis alle handler blev vurderet mere fair ift., hvor meget de øger virksomhedens værdi. Og med denne version af idéen er det i og for sig altså kun dårlige handler (og altså dårlige beslutninger), der kan føre til et tab for aktionærerne. Og det lyder jo ikke helt dumt.\,. Jeg skal så også nævne, at med dette system, så kan aktionærerne altså godt regne med, at deres afkast for en lille periode først bliver betalt noget tid efter, når analyseafdelingen/-instansen har haft tid at regne på tingene. Dermed vil aktier altså heller ikke være helt døde nødvendigvis, bare fordi de er ``udløbede,'' fordi der altså stadig kan være afkast til gode på dem, som endnu ikke er betalt. 

Lad mig forklare om, hvad jeg mener, man kan gøre for at opnå en konstant udløbsrate for aktierne, og bagefter kan jeg så forklare om, hvem de første aktionærer er, og hvordan deres aktier skal være. Den idé, jeg er kommet frem til for at gøre udløbsraten konstant, handler om at have en buffer af nyligt udløbne aktier, som man fornyer ved at udstede dem igen til kunder, og hvor man så hele tiden justerer aktiestørrelsen pr.\ købspris for en lille periode (eller rettere faktisk for en mængde af salg), så bufferen altid holdes nogenlunde konstant givet omsætningsprojektionerne. Men lad mig lige starte med at slå fast, at man sikkert også kan finde på andre idéer. Jeg synes bare at denne idé virker fornuftig, og den er også lige akkurat simpel nok til, at den opfylder alle de kriterier, jeg ønsker for den. Så jeg er altså med andre ord tilfreds med den (og jeg kan ikke lige finde på nogen bedre idé umiddelbart). 
Idéen er mere præcist, at virksomheden er bundet til at se på omsætningen og på, hvad størrelsen af bufferen er (som altså er mængden af udløbne aktier, som ikke er blevet fornyet ved at blive udstedt igen), og så fastsætte en aktie-pr.-pris-værdi, som så skal gælde for den næste lille salgsperiode, eller indtil man får solgt en hvis mængde varer/servicer. Det skal så gerne være sådan at bufferen ud fra projektionerne \emph{skal} ramme indenfor en hvis grænse (i form af en vis lille procentdel af de samlede aktier, som så altså kommer til at blive bufferens omtrente størrelse), når salgsperioden er ovre. Så derved bliver virksomheden altså bundet til at holde bufferen nogenlunde konstant, og får altså samtidigt frihed til at sætte aktieprisen, nogenlunde som de vil, for hver lille periode. .\,.\,Hov, jeg skal lige se en gang, om dette nu også kommer til at gå så glat.\,. \ldots Tja, det er så meget sagt, at de selv kan justere prisen, for det vil de blive meget begrænset med. Men dette er også helt fint, og der er stadig frihed til at sælge så meget, de kan, for hvis man får ædt meget af bufferen, så kan de jo bare (og bliver også nødt til) at sætte prisen (eller rettere aktie-pr.-pris-værdien) højere næste gang. Ja, for jeg fik ikke nævnt det eksakt, men salgsperioden skal også automatisk stoppe, når man får solgt en hvis maksimal del af bufferen ift.\ hvor stor den var i begyndelsen af perioden. Ok, og så burde dette system altså fungere, som jeg kan se det, så at virksomheden altså tvinges til at holde bufferen nogenlunde under et vist niveau, og den kan i øvrigt heller ikke bare sælge hele bufferen på en ultra kort periode.\,. Ah, dette bør man nok muligvis sikre yderligere ved også sætte en fast øvre grænse på, hvor meget aktie-pr.-pris-faktoren må sættes til givet omsætningen og bufferens størrelse, således at bufferen heller ikke når ned under en vis nedre grænse i salgsperioden ud fra projektionerne. Herved kan man så nemlig yderligere opnå, at disse salgsperioder typisk vil have samme længde, medmindre efterspørgslen stiger meget på kort tid. Og hvis den gør, så sikrer den anden regel, jeg nævnte, at man dog ikke alligevel kan få solgt bufferen helt, for inden da vil salgsperioden automatisk slutte, og virksomheden skal så igen sætte en ny aktie-pr.-pris-værdi ud fra samme regler. 

Nå, jeg skal som nævnt også skrive om, hvem der har de første aktier. For virksomheden kan jo ikke starte med kunde-aktionærer. I stedet vil man have iværksættere og/eller indledende IP-indehavere og/eller pengeinvestorer, som starter med aktierne. %For ikke at det 
Man kunne så sige, at for at det ikke 
skal blive unfair overfor de første kunder, så må man simpelthen bare sørge for, at disse aktier formindskes og udløber over helt samme periode, sådan at de første kunder ikke kan mærke forskel på, om de eksisterende aktionærer er de indledende aktionærer, eller om det er ``kunde-aktionærer.'' %Og hvis man så er bange for, at dette vil gøre %..Hm, men skal jeg så virkeligt efterlade idéen helt med, at der kan være specielle sideløbende aktier, som har længere perioder? (Hvad så med tankerne omkring "IP-aktier"..?) ...Ah, på den anden side kan man sige, at det jo på en måde godt kan være "fair" alligevel, for man vil jo typsik regne med, at det store opsving i virksomhedens værdi (hvilket er lig "overskuddet," som jeg regner det her) vil komme mest i starten.. ..Ja, så de indledende aktionærer vil godt i mange tilfælde kunne slippe fint af sted med at give sig selv nogle lidt længere aktieperioder på bekostning af.. Ah, men det behøver vel ingen gang være specielt på bekostning af de første kunder, for man kan jo også bare sætte andelsprocent-kravet ned på aktien, samtidigt med at man forlænger den periode..:).. ..Yes, og herved kan de indledende aktionærer altså sagtens give sig selv længerevarende perioder, og de behøver endda ingen gang at nedjustere aktie-kravet på dem tilsvarende, for dette vil nemlig slet ikke nødvendigvis gøre, at det kommer til at føles unfair for de første kunder.:).. ..Ja, cool nok.:) 
%((13.02.22):)
Dette \emph{kunne} man sige, men på den anden side kan man tilgengæld sige, at det store opsving i virksomhedens værdi (og dermed altså i ``overskuddet,'' som her jo bare er værdistigning pr.\ tid i bund og grund) sandsynligvis må ske mest i første del af virksomhedens historie. Så dermed er det ikke nødvendigvis helt unfair som indledende aktionærer at tage en bid af de første kunders aktier, så at sige, især ikke hvis den indledende IP er meget værd. Hvis \emph{ikke} den indledende IP er særligt meget værd (måske fordi man ikke har så mange rettigheder til det grundlæggende (eksempelvis hvis andre har opfundet det)), så \emph{kunne} man endda i princippet gøre det modsatte og give de første kunder en store andel i aktierne end de efterfølgende. Men hvis nu de indledende skabere og iværksættere har nogle værdifulde rettigheder samlet set, så kan de altså sikkert sagtens tillade sig at give sig selv større og/eller mere længerevarende aktier, uden at de første kunder føler, at dette er unfair. Og bemærk særligt det med, at de netop kan vælge at gøre dem mere længerevarende og altså ikke bare gøre dem ``større.'' Det er forresten lidt forsimplet sagt; i virkeligheden er de to muligheder mere præcist at formindske aktiemængden, der skal udstedes i starten af virksomheden (for på den måde at gøre de indledende aktier ``større'' i sammenligning), og denne mulighed åbner så også op for, at de indledende aktiers perioder dermed også frit i princippet kan gøres længere end kunde-aktiernes perioder. Og når de indledendes aktiers perioder kan forlænges, kan det have to fordele. Det ene er, at de så potentielt kan gøre det endnu mindre ``unfair'' over for de første kunder, fordi kurverne i spil bliver blødere og mere længerevarende, således at reglerne er mere lige for de forskellige ``generationer'' af kunder. Og for det andet så kan man sige, at lange aktieperioder måske også bare er mest fair uanset hvad, for de indledende aktionærers fortjeneste behøver jo ikke at afhænge af, at virksomheden nødvendigvis vokser i den første periode af virksomheden (og jeg tænker forresten, hvis jeg ikke har nævnt det, at kunde-aktiernes periode nok gerne må kunne være noget så langt som måske tyve år eller noget i den stil (der er nemlig mange gode grunde til, at gøre aktieperioderne lange, så længe man bare ikke gør dem så lange, at udsigten om en ægte \emph{kundedrevet} virksomhed bliver for langt væk)), men man kunne måske argumentere for, at så længe virksomheden bliver stor inden for en levetid, så har de indledende aktionærer fortjent en god fortjeneste. Men alle sådan nogle tanker er dog i bund og grund et spørgsmål om, hvad de indledende aktionærer selv synes, og hvad de så tror at den potentielle kundemængde ved kunne gå med til uden at miste opbakningen. Og her skal man selvfølgelig tænke på, at hele princippet i denne virksomhedstype set fra aktionærernes synspunkt er at få kæmpe stor goodwill og opbakning fra kunderne, så dermed betaler unødvendig grådighed sig sikkert ikke. 

Jeg har også nogle andre vigtige ting at nævne, men nu hvor vi er ved de indledende aktier, så vil jeg også lige pointere, at virksomheden ikke bare kan bruge en fast løbende udstedelse af aktier til at lokke kunder til, men også kan gøre noget lidt tilsvarende for at tiltrække IP-bidrag. Dette kunne måske især være gavnligt, hvis virksomheden f.eks.\ er en tech-virksomhed (såsom en ``Web 2.1/3.0-virksomhed,'' som jeg har i tankerne) eller en anden virksomhed, der baserer sig meget på intellektuelt materiale, og især oven i købet hvis man måske netop gerne vil åbne op for, at folk kan bidrage ved at indsende/uploade ting online på en rimelig fri måde, og hvor indsenderne alligevel kan blive rigeligt belønnet. Tanken er så, at man for perioder ad gangen sørger for også at udlove en mere eller mindre fast (og det behøver nemlig ikke nødvendigvis at være ligeså fast som for kunde-aktierne) udstedelse af, hvad jeg kalder IP-aktier. I starten vil det så være de indledende aktionærer, der styrer, hvor mange nye aktier skal udstedes til nye IP-aktier. Og mængden skal så aldrig æde noget af, hvad kundeskaren har krav på at få udstedt for en given periode, for sidstnævnte skal nemlig være fastlagt fra starten af, og må ikke kunne brydes af de indledende aktionærer igen på hverken den ene eller den anden måde. Så de indledende aktionærer kan altså med andre ord vælge at benytte et system, hvor de i stedet for at købe IP-bidrag på mere sædvanlig vis, prøver at tiltrække det ved at udlove en vis del af deres (indledende) aktier som belønning. Det pæne ved denne løsning er så, synes jeg, at dette jo vil blive en naturlig måde at sørge for, at belønningen til (IP-)skaberne kan komme til at afhænge af, hvor godt bidragene hjælper virksomheden med at vokse. Selvfølgelig vil der være forskel på, hvor meget hvert bidrag i virkeligheden hjælper, men her må man jo bare prøve at forudsige dette, når man beslutter sig for, hvor stor en aktieudstedelse det enkelte bidrag fortjener. Samtidigt lægger denne idé også op til, at man kunne lave et system, hvor udstedelsen af IP-aktier af tvunget til at gøres konstant over en længere periode (hvis man midler over denne periode). Jeg tror på, måske især for en sådan ``Web 2.1-virksomhed,'' som jeg har i tankerne, at dette i høj grad kan lokke skabere til. For ligesom at idéen ved denne (``kundedrevne'') virksomhedstype generelt er, at anse kunder på mere lige fod med investorer, så kunne denne tilføjelse altså også gøre, at efterfølgende skabere kommer til automatisk at blive behandlet mere på samme niveau som de første skabere. Og hvis man oven i købet tvinger virksomheden (juridisk) til at blive ved med at opretholde dette system i lang tid, jamen så behøver nytilkomne IP-skabere heller ikke frygte, at deres peers vil blive grådige og skabe stagnation i det hele ved at holde på værdierne selv. Med dette system kan man være forsikret om som skaber, at hjulene vil blive ved med at dreje rundt, og at nye skabere selv hele tiden (i hvert fald så lang tid, som systemet er sat til at skulle køre) vil blive belønnet ligeså fair for deres bidrag, som de tidligere skabere blev det. Jeg håber det er nogenlunde forståeligt, det jeg skriver her. Lad mig lige præcisere, at den der udsteder IP-aktierne stadig kan vurdere hvert enkle bidrag nogenlunde, men de er dog bare tvunget til alligevel, at der skal udstedes lige mange aktier, hvis man midler over en vis fastsat længere periode. Ok, men indtil videre har jeg så kun snakket om et system, hvor de indledende skabere tvinger sig selv til at udstede IP-aktier på deres egen bekostning (dog som betaling for nye IP-bidrag), men dette er jo ikke nødvendigvis fair, hvis man netop er en virksomhed, hvor man regner med at få brug for IP-bidrag i lang tid (og ikke bare mest i starten af virksomhedens historie) (såsom f.eks.\ en ``Web 2.1/3.0-virksomhed vil). Så i dette tilfælde vil det altså være mere fair, hvis man fra start af også bare fortsætter med det samme system for kunde-aktionærerne, og at disse altså også bliver tvunget til helt det samme. I øvrigt kan jeg lige skynde mig at nævne, at det dog altid nok er meget godt med en udløbsdato for systemet, for hvis alle er glade for det selv i fremtiden, jamen så skal fremtidens aktionærer nok selv sørge for at reinstituere et tilsvarende system til den tid. Og lad mig nævne, at for virksomheder, hvor man ikke nødvendigvis regner med, at en fast strøm af intellektuelt materiale (som jeg også bare kan finde på nogen gange at kalde IP) er vigtig for virksomheden, så kan de indledende aktionærer passende lade det være mere op til kunde-aktionærerne selv, hvilket system de vil vælge at fortsætte med, når det bliver mere og mere deres tur til at bestemme. Nå ja, og lad mig også lige lynhurtigt pointere noget helt andet, nemlig at mit koncept om, hvad jeg har kaldt ``bagudbelønning'' i disse noter, jo vil være ret naturligt for dette system, hvor man skal udstede en fast mængde aktier over en længere periode; hvorfor så ikke gøre sådan, at man lige venter og ser, hvor godt det går, inden man bestemmer sig for den endelige belønning til det enkelte bidrag? Og hvorfor ikke, når man er i gang, sørge for så at skrive nogle gode og forståelige erklæringer om, hvilke nogle retningslinjer man sigter efter (hvilket man så også gør klogt i at gøre), når man beslutter lønnen, så bidragsyderne bedre kan vide, nogenlunde hvad de har i vente? Så ``bagudbelønning'' vil altså blive meget naturligt at indføre i dette system. .\,.\,Var der andet, jeg skulle sige om dette?\,.\,. .\,.\,Tjo, jeg skal for det første slå fast, at idéerne i denne paragraf altså ikke hører med til det helt grundlæggende i idéen om ``kundedrevne virksomheder,'' men bare er noget, man eventuelt kan tilføje til virksomhederne. Og derfor behøver idéerne jo heller ikke nødvendigvis at være implementerede fra starten af. .\,.\,Jo, eller hvis man vil indføre systemet i den version, hvor det også gælder for kunde-aktionærerne, så \emph{skal} man godt nok selvfølgelig gøre virksomheden åben for dette i starten, for de indledende aktionærer må jo som nævnt ikke bare gå ind og ændre i, hvor meget kunderne har i vente af aktier. Men udover dette så kan man dog sige, at finansiering af IP jo langt hen ad vejen kan anses for at være et virksomhedsanliggende, som man kan finde strategier for løbende i virksomheden. Så dette skulle jeg lige nævne, og så vil jeg også godt lige nævne en anden vigtig motivation som IP-skaber til at bidrage til en virksomhed, der benytter et sådant system. Jeg har allerede nævnt, at det for mange skabere sikkert vil være betryggende, at deres peers ikke har mulighed for at lade grådighed føre til stagnation i det hele (på fremtidige skaberes bekostning). Noget andet er.\,. Hm, eller måske giver dette først mening at snakke om, efter jeg har snakket om ``spaltning.''.\,. %Lad mig se, \emph{er} denne tanke overhovedet værd at nævne..? ..Det var, at det kan være rart at vide, at man ikke kommer ud for at skulle genforhandle med sine peers.. 
.\,.\,Ja, det vil jeg nok vente med at nævne, til jeg har snakket om ``spaltning'' af virksomhederne. *(Nå jo, jeg kunne også lige nævne hurtigt, at virksomheden også eventuelt på sigt jo har muligheden for at prøve at udvide systemet, så at IP-skaberne ikke bare får en simpel aktie, hvis størrelse er besluttet inden en vis (mindre) fast periode udløber, men hvor de også kan få en aktie, hvis afkast specifikt kan afhænge af parametre omkring, hvordan det går for det enkelte bidrag i fremtiden (i løbet af aktiens periode). Eksempelvis kunne dette jo så være sådan noget som, hvor meget bidraget (i.e.\ f.eks.\ koden, grafikken, teksten, videoen m.m.) bliver brugt/læst/set og/eller liked/ratet, eller hvor godt det går med salget af den/det specifikke service/produkt, som bidraget handler om.)

Og lad mig lige slå fast \emph{igen} igen, at den ovenstående paragraf her kun skal ses som en tilføjelse til idéen, hvilket jo også klart, da den som sådan ikke har noget med \emph{kunderne} at gøre, og da idéen jo er til en ``\emph{kunde}drevet virksomhed.'' Men idéen er altså til en måde, hvor man også (særligt for virksomheder hvor en stor mænge af IP-skabere (og en stor opbakning fra disse) er vigtig) kan bruge lidt samme princip, som man bruger til at opløfte kunderne, til også at opløfte IP-skaberne noget mere. 

Nå, nu tilbage igen til nogle (ret vigtige) idéer, som er lidt mere centrale for den overordnede idé. For det første mener jeg, at det bliver en ret vigtig ting for en god kundedrevet virksomhed, %der gerne vil have mulighed for at kunne vokse sig rigtig stor, at implementere regler for, hvordan kunderne/aktionærerne kan stemme om, at indføre en opdeling/spaltning af virksomheden. 
%... Og hvorfor var det nu lige præcist, at dette er så vigtigt for kunderne?.. Det var jo noget med bl.a. bedre at sikre sig, at majoritetsgruppen af kunderne ikke kan få så meget magt over mere specielle kunder.. ..Og nu har jeg også "service-definitionerne" med som en mere fast del af grundidéen.. ..Tja altså, det \emph{vil} være en rigtigt smart idé, men hvad er lige argumenterne for, at det også endda er en virkligt vigtig del af idéen..? ..Tja, og det er ikke nok det med at sikre sig, at.. Hm, men nu hvor aktionærerne ikke længere \emph{må} kunne lave favorable handler til dem selv, kan de egentligt gøre så meget korrupt ift. at rotte sig sammen mod kunde-minoritetsgrupper..? Hm, lad mig lige tænke en gang.. ...Og nu er der ikke længere noget med at fordele aktieprisen ud på forskellige varer/servicer: Aktier pr. pris skal være det samme for alle kunde-køb for enhver salgsperiode.. ..Hm, men det er jo altså stadig en rigtig god ting, at kunderne kan blive aktionærer mere specifikt i den virksomhedsafdeling, de selv er kunder hos.. ..Hm, lad mig nu bare foreslå idéen, og så lad mig ikke gå så meget i dybden af, hvor \emph{vigtig} den er, og altså hvor meget man kan undvære den eller ej..
hvis den også kan opsplitte sig selv. For hvis virksomheden vokser til at omfatte et stort nok marked, så vil der sandsynligvis også komme lidt forskellige grupper af kunder (med forskelligt forbrug og forskellige interesser). Og i overensstemmelse med hele ånden i en kundedrevet virksomhed, vil det jo så være bedst, hvis virksomheden kan splittes op, så de forskellige kunder mere specifikt kan blive (kunde-)aktionærer i lige præcis den del/afdeling af virksomheden, som de selv er forbrugere af. Så helt generelt mener jeg altså, at det gerne skal være sådan, at hvis der bliver en anseelig forskel på forskellige kundegrupper, og hvis spaltningen ikke koster virksomheden det helt store, så skal virksomheden gerne i reglen splitte sig op, så de forskellige kunder løbende kan blive mere og mere aktionærer specifikt efter deres forbrug hos de resulterende undervirksomheder. Jeg har jo allerede nævnt, hvordan virksomhederne helst skal have klare beskrivelser af, hvad de leverer. For at undgå konkurrence imellem nyligt fraspaltede (under)virksomheder, så kan man jo så måske med fordel vedtage, at de hver især også binder sig til i en vis periode ikke at gå ind og sælge konkurrerende servicer/produkter, som en af de andre.\,. spaltningssøskende, kan vi jo kalde det. For omtalte beskrivelser kan jo sagtens ændres, så de juridisk er bundet i en vis form i en periode. Ok, så det er altså den overordnede tanke. Bemærk, at aktionærerne jo så bare alt andet end lige skal forblive som de er umiddelbart under en spaltning, således at deres aktier altså efterfølgende har samme størrelser som før, men nu bare er spaltet til $x$ antal kopier, som hver især gælder for én af hver af de resulterende virksomheder. Og så vil det altså bare være med tiden, at kundegrupperne automatisk mere og mere bliver aktionærer i den.\,. lad mig hellere kalde det `delvirksomhed' for nu, som de forbruger mest af. Og derudover kan hver aktionær selvfølgelig også bare forsøge at handle deres aktier til andre, hvis de er mere interesserede i at holde én type frem for en anden (så hvis kunderne i høj grad allerede \emph{er} aktionærerne kan de altså potentielt set fremskynde processen selv). Nå, så langt så godt, men jeg mener tillige, at man endda fra start af i den kundedrevne virksomhed bør prøve at give nogle bestemmelser for, hvornår sådanne spaltninger kan ske, og ikke mindst hvornår de \emph{skal} ske. Mit forslag er så, at man for det første har en demokratisk proces, hvor man kan stemme forslag igennem til afstemning i første omgang. Hvis et forslag bliver stemt til videre afstemning, så skal en lødig tredjepart så gerne udregne, hvor meget det vil koste virksomheden ca., hvis spaltningen skal ske. Og alt efter hvor stor en andel af den samlede virksomheds værdi dette er, samt hvor omsætningen er for.\,. lad os sige den andenstørste del af servicerne/produkterne, som kommer til at udgøre en selvstændig delvirksomhed efter spaltningen ifølge forslaget, og ud fra dette og ud fra en fast forskrift bestemmes så, hvor stor en andel af aktionærernes stemmer (hvor de selvfølgelig har stemmeret proportionelt med deres aktiers størrelser ligesom for alt andet), der skal bruges til at stemme spaltningen igennem til udførsel. Selve datoen for, hvornår man skal udføre den skal så i reglen også gerne ligge lidt ude i fremtiden fra afstemningen af, så aktionærerne har mulighed for potentielt at annullere den igen (tænker jeg). I øvrigt kan der også være andre grupper involveret med specielle veto-/annulleringsretter, men det vender jeg tilbage til (og det er dog heller ikke super relevant for lige dette). Nå ja, og det er værd lige at slå helt fast, at (selvom jeg har kaldt dem ``undervirksomheder'' og pt.\ kalder dem ``delvirksomheder'') de fraspaltede virksomheder gerne skal blive helt selvstændige virksomheder (selvfølgelig dog med de samme regler som gjaldt for den virksomhed, de stammer fra) i bund og grund, på nær altså at der kan være vedtagelser i starten om, at de ikke må begynde at konkurrere med deres søskende omkring salg af visse produkter eller servicer i en vis periode efter spaltningen. Nå ja, og det er bestemt også værd at nævne, at spaltningssøskende godt kan være kunder hos hinanden, og særligt kan det meget vel forekomme, at visse delvirksomheder i høj grad er kunder hos én eller flere andre virksomheder (så altså et asymmetrisk kundeforhold), fordi der kan være tale om en mere grundlæggende service, som denne/disse virksomhed(er) leverer. Vi kan således f.eks.\ forestille os en hypotetisk virksomhed, hvor der leveres to forskellige primære services, og hvor begge disse benytter det samme grundlæggende system, som virksomheden måske har IP-rettighederne til. Kan man så splitte sådan en virksomhed op? Ja, det kan man sagtens i princippet. Man kan nemlig splitte den op i tre i så fald, hvor man så får to delvirksomheder til hver af de to primære servicer (som vil have de samme kundemængder efter spaltningen) samt en delvirksomhed som leverer det omtalte grundlæggende system som en service. Sidstnævnte får altså så de to andre delvirksomheder som dens kunder (plus eventuelle andre kunder, den med tiden kan tiltrække). Bemærk, at der ikke er noget til hinder for, at virksomheder kommer til at optræde som kunder i en virksomheds definition/beskrivelse af, hvad den betragter som kunder (i.e.\ hvem der har krav på kundeaktier). Og hvis en kundedrevet virksomhed er kunde hos en anden kundedrevet virksomhed (hvormed jeg altså mener en virksomhed med de regler, som jeg foreslår her i disse noter), så vil de kommende afkast fra aktierne, som udstedes fra sidstnævnte jo bare indgå som en del af overskuddet i den førstnævntes regnskab. Jeg tror en sådan spaltning som denne, hvor en mere grundlæggende del af virksomheden også ``faktoreres ud'' som sin egen virksomhed under en spaltning, vil blive meget gavnlige og meget typiske for sådanne virksomheder. 

%(14.02.22):
Lad mig lige gentage/understrege, inden jeg går videre til de sidste ting, at idéen om at virksomheden skal kunne spaltes, hvis nok kunder ønsker det (set ift.\ omkostningerne), altså \emph{er} ret vigtig, når man tænker på hele ånden i idéen. For hvis en kundedrevet virksomhed bare vokser og vokser og breder sig ud på flere områder og at spalte sig, så vil visse kundegrupper nemt kunne komme ud for, at de ikke føler, at de har udsigt til blive bestemmende over den afdeling, de er de hovedsaglige forbrugere af. For hvis de andre afdelinger er meget større, så vil de aldrig kunne få en reel magt i sammenligning. Så for at bevare ånden i idéen og føre den helt i mål, så \emph{skal} virksomheden altså binde sig til automatisk at igangsætte spaltninger, hvis en stor nok kundegruppe ønsker det (og hvis det altså ikke er for kosteligt at gøre). 

Jeg fik vist heller ikke nævnt det i denne omgang (altså fra d.\ 12.\ af (og det er d.\ (14.02.22) forresten)), men det skal stadig være sådan, at stemmemagten, der følger med aktierne, skal aftage løbende (jævnt) i takt med at aktien udløber (og at flere og flere afkast-perioder er overstået). 

Nå, den sidste (vistnok) ret store ting, jeg skal nævne, er omkring ændringer af virksomhedens grundsætninger. Dette er jo bl.a.\ sådan noget som, hvor lang tid aktie-perioderne skal vare for kunde-aktierne, og hvor stor en del af overskuddet/underskuddet skal gå til afkast/depositumsafgifter, hvilket er to af de nok mest faste grundsætninger (samt selvfølgelig alle reglerne omkring, at aktierne aftager og udløber, og at de så skal udstedes igen med samme rate til kunderne osv.). Men hvad nu hvis man med tiden finder ud af, at man ikke satte disse værdier optimalt (og at forskellen betyder noget)? Hvad hvis der opdages andre børnesygdomme i virksomhedens grundsætninger? Skal man så bare leve med dem, eller er der noget man kan gøre for at justere dem. Problemet er, at hele virksomheden på en måde er en stor aftale imellem en masse parter, hvoraf de fleste af disse netop bakker op om og investerer i virksomheden, \emph{fordi} den har de grundsætninger, den har. Og det nytter således ikke eksempelvis at give aktionærerne lov til at ændre i disse grundsætninger, for så kan kunderne jo dermed blive snydt. Så hvad kan man mon gøre? Jo, det handler selvfølgelig om, at man skal finde en måde, hvor alle parterne involveret, hvilket i øvrigt involverer kunder, kunde-aktionærer, indledende aktionærer (hvis de stadig findes) og eventuelt også IP-skabere, kan få en vis passende ret til at vetoe nye planlagte ændringer i grundsætningerne. Alle disse parter skal altså med andre ord gerne kunne have en stærk vished om, at der ikke bliver indført nogen ændringer i grundsætningerne, som strider klart mod denne gruppes interesser (hvorved et flertal af gruppen jo vil stemme imod forslaget). Nu nævnte jeg IP-skaberne, og lige netop her kan det jo være op til den enkelte virksomhed, hvor meget vetoret man vil give denne gruppe, men hvis vi f.eks.\ ser på en mulig ``Web 2.1/3.0-virksomhed,'' som jeg tænker den, hvor jeg jo netop mener, at man også har behov for (eller i hvert fald kan have meget gavn af) en forhøjet opbakning fra skaberne (ved også at betragte disse på mere lige fod med ``investorer''), så er det klart, at disse også vil fortjene deres egen vetoret. Desuden skal alle ændringer i grundsætningerne (pr.\ de indledende grundsætninger selv, for alt dette kan man selvfølgelig også justere og lave om på) gerne have en meget lang tid, før de træder i kraft, så der er god tid til at kunne annullere dem igen. Nu her taler jeg så særligt om de halt basale grundsætninger som en ``kundedrevet virksomhed'' skal binde sig til pr.\ denne udgave af idéen, men der er også nogle knap så basale sætninger, hvor det også kan være værd, hvis diverse parter kan vetoe ændringer. Her tænker jeg særligt på ``beskrivelserne'' af, hvilke services og produkter er inden for virksomhedens område, og altså dermed også hvem præcist er deres \emph{kunder}, og hvem er bare andre instanser, der handles med. Her nytter det jo ikke noget, hvis aktionærerne f.eks.\ bare pludseligt kan ændre helt på, hvem \emph{kunderne} er i beskrivelsen, og dermed hvem har ret til aktier, og så f.eks.\ sige, at.\,. Ja, måske sige, at de nu er en meget lille gruppe af mennesker, der tilfældigvis.\,. enten \emph{er} aktionærerne selv, eller som har en (måske særligt konstrueret) gæld til aktionærerne. Selvfølgelig har vi også allerede reglerne om, at virksomheden ikke må kunne lave favorable handler, så man kunne jo oplagt udvide dette til også at lægge grundlæggende bånd på omtalte ``beskrivelser,'' men uanset hvad vil det sikkert heller ikke være dumt, hvis man også giver vetoretter til de forskellige parter, så beskrivelsen ikke kan ændres til noget, der går én part (eller flere perter) klart imod. Ok. 

En tilføjelse til denne idé, som jeg selv umiddelbart er ret glad for, er så, at man måske også kunne lade ``fremtidige kunder'' indgå som en ``part,'' når det kommer til vetoretter. Tanken er, at da man jo alligevel gerne skal have en vis (gerne ret lang) periode fra, at en ændring vedtages, og til at den træder i kraft, så kunne man måske også lade alle nye kunder i denne (gerne ret lange) periode få lov at stemme om ændringen også, og altså med andre få retten til potentielt at vetoe den. Jeg har faktisk ikke tænkt mig at argumentere vildt meget for idéen, men vil nok mest bare sige, at den altså tiltaler mig ret meget. Og så vil jeg dog lige pointere, som også er et rimeligt godt argument for den, at den særligt også er brugbar, når det kommer til nyligt spaltede virksomheder. For så vil de fraspaltede virksomheder i reglen hurtigere kunne ændre sine grundsætninger, så de passer med den pågældende kundegruppes ønsker (fordi man så ikke nødvendigvis behøver at vente på, at denne kundegruppe kommer til at dominere som aktionærerne for virksomheden, fordi de ligesom får andel i vetomagten hurtigere). Og dette vil så særligt gøre sig gældende, hvis man endda i høj grad lader vetomagten sidde hos de ``fremtidige kunder'' (eller man kunne jo egentligt også tænke på dem som de nutidige kunder). For hvis man gør det, og altså dermed i høj grad giver dem magten over, hvilke ændringer i grundsætningerne må vedtages, så kan det jo på den måde blive mere fleksibelt. .\,.\,Ja, man kan nok godt høre, at jeg ikke har finpudset disse argumenter, og det har jeg egentligt heller ikke tænkt mig at prøve på. Jeg synes bare, at det helt klart var en idé, der var værd at nævne. 

Hermed tror jeg så, jeg har været igennem de vigtigste ting. Jeg har dog lige et par mindre idéer endnu, jeg også gerne vil nævne. For det første har jeg tænkt på, at man måske kunne gøre, så at hver kunde-aktie står fast i kundens navn i udgangspunktet, og at det så koster en lille afgift til virksomheden, hvis vedkommende gerne vil ``frigøre'' aktien, så den kan handles med. Dette skulle være for lige at gøre det en anelse mere attraktivt for kunderne at beholde deres aktier, hvilket jo er ønskeligt ifølge ånden i idéen, for hele idéen er jo, at magten efter noget tid gerne overvejende skal findes på kundernes hænder. Og noget andet man så i øvrigt også kunne gøre, er at sætte en tilsvarende afgift, men i form af at den pågældende aktie, der ``frigøres,'' nu mister en del af sin medfølgende stemmemagt. Så disse idéer kan man altså (som startende kundedrevet virksomhed) overveje, om man vil tage med. 

Og de sidste ting, jeg vil skrive her, er så hvad jeg lovede at vende tilbage til omkring IP-skaberne, samt noget jeg har glemt at pointere om samme. Sidstnævnte er, at jeg har glemt at pointere, at spaltninger, som i det eksempel, jeg nævnte, hvor virksomheden ``udfaktorerer'' en delvirksomhed bestående af alle de underliggende IP-rettigheder, faktisk potentielt kunne gøre det endnu mere tiltrækkende for IP-skabere at bidrage (hvis altså vi f.eks.\ tænker på en Web 2.x-virksomhed eller lignende). For med sådan en adskillelse kan man nemlig gøre selve IP-delen af virksomheden meget mere robust og altså sørge for, at den ikke kommer i risiko for at gå konkurs. Og jeg mener altså at dette vil kunne være meget mere tryghedsskabende for IP-skaberne, for så risikerer de ikke, at de skal til at omforhandle deres aftaler med kunder, og ikke mindst med hinanden, hvis nu virksomheden går fra hinanden. Og ja, når man tænker på dette, så kunne man måske endda med fordel starte virksomheden i flere dele allerede fra start, sådan at alle IP-bidragere kan være mere trygge ved, at den (del)virksomhed, de bidrager til, også vil bestå og blive ved med at fungere. Dette kan således både være betryggende økonomisk for skaberne, og jeg tror i øvrigt også, det vil være rart at tænke på, at deres intellektuelle bidrag (som jeg jo tillader mig at kalde IP-bidrag for nemhedens skyld) ikke kommer til at blive låst for omverden af, at først virksomheden går i vasken, og at omforhandlingerne sidenhen går i kludder mellem IP-skaberne (i.e.\ dem der bidrager med intellektuelt materiale *(Man kunne måske så kalde dem IM-skabere i virkeligheden.\,.)) --- måske fordi visse parter prøver at benytte lejligheden til at forhandle sig til endnu bedre aftaler på bekostning af andre. Jeg tror altså på, at det vil være rart at vide, at sådanne fremtidige deadlocks ikke vil forekomme, når man investerer sine bidrag i en virksomhed. Nå, og så var der også lige den ting.\,. Hov! Ha, det var jo forresten én og samme ting. Den ting jeg snakkede om, som gav mest mening at nævne, efter at jeg havde snakket om spaltning, der var jo netop denne ting. He. Nå, men så har jeg ikke mere at sige i denne omgang?\,.\,. .\,.\,Jeg tror, det var det.\,:)\textasciicircum\textasciicircum\ (Og ellers må jeg jo bare vende tilbage her og skrive det, hvis der er andet, jeg kommer i tanke om.:))


%Buffer. (tjek)
%Første aktier *(tjek..) (måske lidt allerede her om løbende IP-aktier..).. (tjek)
%Spaltning. *(..Husk IP-"forælder-undervirksomheder".. Og husk at tale om sammenslået IP generelt.. Hm.. ..Hm, det var jo det med ikke at skulle genforhandle med andre skabere.. Var der ikke også en anden ting..? ..Hm, så var det nok bare ligesom at forhindre stagnation pga. grådighed..)
%Grundsætningsændringer.. (tjek)
%Delay, vetoret.. (andet..?) *(Ja, lille afgift ved at videresælge sin aktie i første omgang som selve kunden..) *(Hm, jeg gider faktisk ikke alligevel at nævne "delayet" (på at nye aktionærer har ret til afkast), for det er bare ikke længere rigtigt relevant (og gavnligt så vidt jeg kan se) for denne seneste version af idéen.) (tjek i så fald)
%Idé: Uh, hvad med vetoret til fremtidige kunder også, hvor disse altså kan stemme om at annullere grundsætningsændringer, der blev indstemt før deres aktier udstedtes?:).. ..Nice.. (tjek)
%... Ah, og med denne idé, så kommer det egentligt også meget naturligt, at grundsætninger kan ændres som en del af spaltninger, fordi den kundemængden så også kan dele sig, og de kunder, der bruger den nye undervirksomhed mere kan jo så lade ændringerne gå igennem uden at stemme dem til annullering.:) (tjek)


**(27.03.22) Jeg tror muligvis min idé om, at aktiernes værdi skal være betingede af, at den originale ejer (kunde) ikke har solgt den videre, således at kravet på aktien falder med en signifikant andel, hvis dette gøres, er mere vigtig end jeg måske har tænkt den før.\,. Jeg er ikke helt sikker, men det virker altså umiddelbart sådan.\,. Jeg forestiller mig i øvrigt et fald i kravet på omkring 10 \%, men hvem ved, om det i virkeligheden er bedre at gøre dette tal højere eller lavere. (Men idéen med omkring 10 \% (eller mindre) er, at så sprænger det ikke budgettet helt for folk, som ønsker at købe produkterne, men ikke har råd til at holde fast på investeringen.) Nå, men lad mig så lige prøve at forklare idéen lidt, for det kan jo egentligt virke paradoksalt, at man prøver at fremme en virksomhed ved at give sine kunder begrænsninger (nemlig i form af at de straffes lidt, hvis de ikke holder fast på aktierne selv). Men fra kundernes synspunkt \emph{er} der faktisk en fordel, netop fordi de samme krav gælder for alle andre kunder også. Og dermed får man så rigtig stor vished om som kunde, at virksomhedsejerne ikke bare vil opkøbe sine egne aktier hele tiden i en uendelighed (for dette kan ikke være rentabelt --- kun hvis de allerede ejer næsten alle aktier, men så vil der blive enormt meget at miste for den tilbageværende lille mængde originale kunder (og ja, for alle nye kunder dermed også) ved at sælge sine aktier, for så giver man jo afkald til rettigheden til alle straf-pengene). Ja, så måske vil 5 \% faktisk sagtens kunne være rigeligt.\,. (For når aktierne nedsættes i værdi, så må dette jo ske til gavn for alle dem, der så ikke har solgt (eller købt, rettere) deres aktier.) Og hvad er fordelen ved at have ``vished om, at virksomhedsejerne ikke bare vil opkøbe alle aktier igen hele tiden?\,.'' Jo, her er det så vigtigt at forstå, hvad jeg mener med at kunder ``investerer med deres forbrug.'' I min idé investerer de så også i en mere gængs forstand, fordi de betaler nogle (ekstra) penge, som de så forventer et afkast på, men selv inden/uden sådan et system, så kan man stadig se kunder som en slags investorer. Når man køber et produkt og gør en handel, så ændrer man nemlig ikke bare sine egne forhold, men man ændrer også verden en lille smule. Derfor kan det ofte betale sig at være politiske forbrugere, hvis man altså kan samles nok om det, og hvis der er nok gevinst ved det --- nemlig hvis forskellen på to virksomheders fremtidige aftryk på verden (lokalt eller globalt) er stor i forhold til prisforskellen. Så der kan være alle mulige gode grunde til gerne at ville forbruge hos en virksomhed, hvor man ved at kunderne selv på et tidspunkt for ledelsesmagten, selv også, og dette er en vigtig pointe, bare på et fundamentalt økonomisk plan. For når man vælger at være forbrugere hos en virksomhed som kunde, eller som en mængde af kunder, så indvilliger man i at deltage i en forretning, hvor en stor del af overskuddet går til bestemte personer, nemlig dem der har været hurtige til at investere (og har kunne købe aktierne billigt) eller iværksætterne. Dette er ofte en helt fin deal, især hvis der er noget banebrydende iværksætteri i firmaet, som kommer kunderne til gavn; så vil man da gerne som kunne betale lidt til at iværksætterne får løn som fortjent. Men mange firmaer etablerer sig selv, ikke fordi de gør noget på en ny og iværksættende måde, som andre firmaer ikke også ville kunne finde ud af eller ikke har patent på. I stedet etablerer mange virksomheder bare sig selv på reklame og på, ja, at de allerede er etablerede (også hos deres handelsparter), og at det ville være et sats at prøve at starte noget nyt op for at konkurrere med dem. Men vis skyld er det, at en virksomhed er etableret?\,. Er det iværksætternes skyld, der har fundet på de gode reklamer og de gode branding-strategier, eller er det de kunder, der har valgt at være kunder hos virksomheden, som har skyld i, at virksomheden er veletableret?\,. Tja, det er i hvert fald svært at argumentere for, at det ikke også er kunderne, der har en del af denne ære. Og hvis vi således zoomer ud og ser tingene fra denne synsvinkel, så bliver det pludseligt svært at forsvare, at det kun er branding-menneskerne og de folk, der tidligt øjnede forretningsmulighederne, der skal have del i al overskuddet, frem for de kunder, der har valgt at bakke op om alt dette. Og videre kan man så også sige, at jamen hvis kunderne så får krav på en del af dette overskud (som de altså selv helt klart har en stor del af æren for pga.\ deres valg om at være kunder her frem for andetsteds, bare lige for at gentage den pointe), og endnu vigtigere at de får krav på en del af ledelsesmagten i virksomheden, jamen så kan de også bedre sørge for, at virksomheden bedst muligt fortsætter med at fungere ud fra kundernes interesser. Meget specifikt kan de forhindre en stor tendens, der ellers er i vores nuværende kommercielle system, hvor virksomheder ofte bliver ``korrupte'' i den forstand, at de ender med træffe beslutninger, der er til fordel for deres ejere men på bekostning af deres kunder, nemlig ved at udnytte deres grad af etablering. (Og hvis man prøver at kaste nogle arkaiske økonomiske teorier efter dette som modargument og prøver at sige, at dette fænomen omkring, hvor meget ``etableringsgrad'' betyder for (og kan udnyttes af) virksomheder, ikke er betydende, så forstår man bare ikke nok og vores nuværende kommercielle system (jeg er ikke ret vidende på området, og selv jeg forstår dette). Der er en grund til, at reklamer og PR er så enormt meget værd at investere i som virksomhed (især virksomheder, der har privatpersoner som kunder).) Der er desværre denne tendens, hvor virksomheder i vores nutid ofte ser ud til alle at følge den model, der går på, at de først forsøger at få opbygget så meget momentum som muligt (ved at komme deres kunder til gode), men når så først tilstrækkeligt med momentum er opbygget, så begynder de så at presse citronen og udnytte indtjeningsmuligheder, der sker på \emph{bekostning} af deres kunder, når man ser det ift.\ deres overordnede interesser. Så selv på dette meget grundlæggende økonomiske plan kan det derfor (når det kommer til nogle områder/hverv) muligvis betale sig for kunder at bakke op om en virksomhedsmodel, såsom jeg forslår her (i.e.\ med hvad jeg her har kaldt ``kundedrevne virksomheder''), hvor det sikres, at det med tiden mere og mere bliver kunderne (alt efter hvor meget de har investeret i den nære fortid, hvis man vil følge min model her), der får magten i virksomheden. Okay. Det næste oplagte spørgsmål bliver så, jamen vil der så også med denne nye forretningsmodel blive ekstra incitament for kunder i at investere i den, fordi de så ikke længere bare ``investerer som forbrugere'' i den normale forstand (som forstås ved det gængse begreb om ``politiske forbrugere''), men også nu investerer i en mere gængs forstand, fordi de jo så har et faktisk afkast i vente?\,. Er det her en idé, hvor de første kunde-investorer så bliver spidsen af en pyramide, og hvor de så kan forvente store afkast heraf (på bagrunden af virksomhedens efterfølgende vækst)?\,. Måske, men ikke nødvendigvis. Og dette er ret vigtigt at pointere, for ellers var der nok tale om et pyramidespil. Men kunde-investorerne har altså \emph{ikke} nødvendigvis en masse at vinde, for deres afkast afhænger kun af, hvor meget virksomheden vokser. Hvis således alle bare med det samme kan se, at her er en god ny idé, der er værd at investere i, jamen så vil iværksætterne muligvis bare kunne finde nok kapital til at starte med, så virksomheden allerede starter nogenlunde med den (kapital-)størrelse, den forventes at ende med at få, minus det man så skylder tilbage til dem, man har lånt kapital af. (Og nu kan jeg faktisk dårligt huske, om jeg besluttede, at det var en dårlig idé at låne kapital eller ej, men det kan man bare læse ovenfor; det er ikke så vigtigt nu.) Men ja, der er altså selvfølgelig kun noget at vinde for de første kunder, hvis der er tale om en virksomhed, der starter med lav kapitalbeholdning og en begrænset kundeskare, ift.\ hvad virksomhedens egentlige potentiale er. Og den eneste grund til at starte lavt er enten, hvis der af en eller anden grund bare ikke kan findes startkapital nok, så man bliver tvunget til at tjene den løbende selv (hvilket ikke burde forekomme i en globaliseret verden, medmindre måske virksomhedsmodellen starter op over hele verden i alle mulige afskygninger på én gang.\,.), eller hvis der simpelthen bare er mange folk (og dermed potentielle kunder), der ikke er overbeviste om idéens værdi til at starte med. Og hvis således en stor del af den potentielle kundeskare, og også en stor del af mulige investorer, dermed er langsomme til at hoppe med på bølgen, kun i så fald \emph{kan} der være mange penge at tjene på så at være en af de tidlige (kunde-)investorer. %Nå, tilbage til fysikken!.. Shit er klokken allerede to.x) ..Godt nok er det lige blevet sommertid, men alligevel.x) 








\subsection[K.o. ITP-idé]{Kort kommentar om min ITP-idé}

(08.12.21) Det er nok egentligt ikke så nødvendigt, det her, men jeg tror lige jeg vil sige et par ord om min ITP-idé. Det føles jo lidt som om, jeg har efterladt idéen, fordi jeg ikke har skrevet om den siden foråret, og fordi jeg ikke tager den med i min udgivelse, som jeg arbejder på nu her. Men jeg tror nu altså stadig på, at idéen virkeligt kan blive kæmpe stor i fremtiden. Jeg tror bare ikke nødvendigvis, at den har sådan en gennemslagskraft, som jeg godt kunne tænke mig af de idéer, jeg vil udgive først. Men ja, jeg føler dog for det første, at jeg har fat i noget godt med mit grundlæggende design, hvor man altså i første omgang har en række meta-antagelser (som er ret forskelligt fra \emph{aksiomer} til diverse matematiske teorier og modeller, man vil se på, og som kan varieres og tilpasses fra computersystem til computersystem). Disse meta-antagelser beskriver så i et formelt sprog, hvad man kan forvente af de præ-installerede programmer i mappen, samt giver en opskrift for, hvad man kan forvente af nye programmer, som man beviser korrektheden af med ITP'en, og som man så installerer i mappen med ITP'en. Og ja, mere om dette kan man så læse ovenfor under ITP-sektionen. Men en anden rigtig vigtig ting i idéen er så altså, at man systemet i høj grad skal lægge op til, at man kan tilføje meta-aksiomer løbende (og forskellige brugere kan frit selv vælge, udskifte og opdatere sine meta-aksiomer løbende, især når det kommer til antagelser, der vides at være korrekte, og som bare gør ITP'en og/eller installerede programmer fra den mere effektive, for i så fald kan man altid bare om-kompilere beviserne til de mere ineffektive versioner (men som altså så kan bygge på færre meta-antagelser)), som så kan give mulighed for at foretage handlinger/operationer, hvor man har en vis tiltro til korrektheden af disse operationer. Så hvis nu en masse kyndige folk har analyseret en operation og har bevist matematisk, den holder, jamen så kan man jo, som individuel bruger og/eller som fællesskab af brugere, acceptere denne operation via en meta-antagelse. Hvilke nogle meta-antagelser et resultat har bygget på, skal så fremgå direkte af propositionerne, som ITP'en spytter ud, således at folk i sidste ende altid kan justere graden af tiltroen til visse meta-aksiomer (i form af en sandsynlighed for korrektheden af hver). Og dette lægger så op til et paradigme, jeg virkeligt tror bliver stort en dag, nemlig at alle resultater man når frem til, i første omgang hvad angår matematik og programmering (f.eks.\ i forbindelse med beviser af programkorrekthed som et vigtigt eksempel), alle får et mærkat, som fortæller, hvilke nogle antagelser der skulle til for at nå frem til korrektheden af resultatet. Og da meta-antagelser kan inkludere selv sådanne ting som at stole på alle udsagn af en vis karakter, der er underskrevet af en vis instans, så er der altså ingen grænser for, på hvor højt et niveau, man kan nå op med meta-antagelserne. Og ja, jeg tror, som jeg også har skrevet i pågældende sektion.\,. nej, sektion\emph{er}.\,. jeg tror også, at dette kan udbrede sig til mange andre områder end bare matematik og programmering. Det kan f.eks.\ også (hvis ikke noget andet gør det først) lede til mine kære ``prædiktive modeller,'' som jeg jo også har så store forhåbninger til. Med andre ord kan det altså lede til, at vi også får sandsynligheder for korrekthed, samt også anvisninger til, hvilke nogle antagelser man har gjort, for at nå frem til givne svar, til at blive en fast følgesvend til, hver eneste gang vi ser et resultat fra en eller anden form for deduktion på internettet, hvad end denne er gjort formelt af en computer eller mere uformelt måske af en gruppe mennesker. Så ja, selvom der også er en del mere eller mindre skrald, når man bevæger sig op til de lidt tidlige noter i min ovenstående ``Summary/brainstorm''-sektion, så mener jeg altså bestemt alligevel, der er noget at komme efter, når det kommer til nogle af mine ITP-relaterede idéer (også selvom idéen måske ikke ville have den gennemslagskraft, som jeg tænkte, da jeg skrev om det, hvis den man lancerede den pt.).






\subsection[M.o. P-modeller]{Mere om mine ``prædiktive modeller''}

(21.12.21) Idéen kan egentligt forklares ret simpelt, og det tror jeg lige, jeg vil gøre her. Lad os sige, vi allerede har en hjemmeside og/eller database, hvor folk bl.a.\ kan uploade udsagn til, og hvor folk også kan rate (i.e.\ vurdere) ethvert udsagn ud fra et rating-prædikat, der betegner, hvor sandsynligt (brugeren tror) det er, at udsagnet er sandt. Herfra skal vi så bare bruge en måde, hvor vi også kan implementere en slags implikationsudsagn, som så kan bruges til at lave en slags sandsynligheds-Modus Ponens. Så hvis en bruger stemmer, at et implikationsudsagn har en sandsynlighed $p_1$, og også stemmer at et udsagn, der kan matche antecedenten hos dette (hvilket gerne må kunne formuleres generelt, så det kan matche mange forskellige slags udsagn på en gang), har en sandsynlighed $p_2$, så vil dette give en speciel pointparameter til den konsekvent, der fremkommer efter matchet, med værdi $p_1 p_2$.\,. (Jeg improviserer en anelse her, men det går jo nok (og det gør jeg jo tit).) Ok.\,. En `prædiktiv model' handler så om, at en brugergruppe sørger for for det første at give en række sandsynligheder til en masse udsagn, og særligt til en række implikationsudsagn (men også gerne til en række udsagn generelt, hvis dette er nyttigt). Og for det andet skal der også gives korrelationsvurderinger imellem hvert udsagn --- i det mindste i princippet, men for mange par af udsagn i modellen vil korrelationen sikkert være triviel (i.e.\ lig 0), og disse vil man så bare kunne lade være. Men der skal altså gerne gives korrelationsvurderinger imellem alle par, der kan have korrelerede sandsynligheder (pr.\ brugernes vurderinger). Så hjemmesiden (eller tilsvarende), hvor det foregår, skal altså også kunne implementere muligheden for sådanne korrelationer (samt hvad man så får ud af disse.\,.). Uh ja, og det er vel egentligt her mit begreb, som jeg har prøvet at formulere omkring ``koncepter,'' kommer ind i billedet, nemlig hvis man gerne vil spare lidt arbejde angående korrelationerne samt gøre det hele mere overskueligt (og mere lærerigt). Hvis man således har en masse udsagn, som er korrelerede, fordi de alle hører til en underliggende sammenhæng på en eller anden måde, så vil det være rigtigt smart, hvis brugerne kan finde frem til netop disse underliggende sammenhænge. .\,.\,Herved bør brugerne så på en eller anden måde altså kunne få muligheden for bare at oprette sådanne ``koncepter'' som udsagn, og så bare nøjes med at indsætte korrelationerne imellem hvert udsagn i den egentlige model og så med.\,. alle udsagn i laget bestående af disse ``koncepter.'' Ja ok, så det handler altså om, at brugere skal kunne organisere alle udsagn i mindst to lag, hvor, hvis vi starter med at tænke os tilfældet med netop to lag, alle (ikke-trivielle) korrelationer skal indsættes kun for par på tværs af de to lag. (Så udsagn i det samme lag skal altså ikke have sat korrelationer imellem sig.) I øvrigt må der godt være gengangere i nr.\ 2 lag (hvilket vi kunne kalde ``konceptlaget,'' hvis vi vil fortsætte med at bruge det term (hvad vi i det mindste bare kan gøre for nu, for jeg har ikke lige et bedre et)). I princippet kan man så klare sig med to lag, for hvis to udsagn i konceptlaget er korrelerede, så kan man jo bare erstatte dem i princippet med nogle, der ikke er. Men i praksis kan det måske være bedre, hvis man i stedet bare kan tilføje et nyt lag. For på den måde bliver det måske lidt nemmere at genbruge arbejde, både sit eget og andres. 

Ok, nu mangler jeg at nævne certifikat-metoder, og så ellers hvordan man opnår de endelige svar ud fra sandsynlighederne og korrelationerne. Angående det første så er kan dette forklares rigtigt kort. Det er nemlig bare, hvad jeg i høj grad har lagt op til, da jeg skrev om ITP og om formel programmering, nemlig at man jo ved at antage udsagn omkring visse underskrifter, så kan man herved komme til at lave ``matematik'' over selv vilkårligt abstrakte metoder. Jeg føler ikke, jeg behøver at gentage denne del, for man kan bare læse omtalte sektioner. Men måske bør jeg lige overveje, hvordan man får det til at virke her i denne forbindelse.\,. .\,.\,Ja, det kan godt være, at man lige så skal have nogle specielle syntaks-prædikater (der kan eksekvere en regex), og at man så også skal have en slags mappe eller database-relation (alt efter hvordan man ser det), hvor dokumenter, der aktiverer syntaks-prædikaterne, kan uploades til. 

Og hvordan får man så sine endelige sandsynlighedsvar fra modellen? .\,.\,Hm, og noget andet er: Hvordan justeres modellen løbende ved at brugerne uploader data om faktiske udfald i den virkelige verden (således at modellen kan justeres ifølge Bayes' sætning)?\,.\,. .\,.\,Hm, hvis vi starter med at se på et udsagn, der kun kan konkluderes ud fra én implikationssætning i modellen, så er det rimeligt trivielt, hvordan man skal opnå den resulterende sandsynlighed (og i øvrigt også de resulterende korrelationer til konceptlaget (som jeg næsten burde finde et bedre navn til.\,.)). Men hvis udsagnet kan udledes på flere måder ud fra modellen, så bliver det jo lidt mere indviklet, hvad man lige gør der.\,.\,? .\,.\,Tjo tja, men hvis nu gør det meget klart, hvilke domæner af hver implikationssætning må besvare, så man let kan finde frem til alle de implikationssætninger, man må (og skal) bruge for ethvert givent udsagn, man efterspørger svar omkring, så bliver udregningen også rimeligt ``triviel'' (med hvilket jeg altså bare mener, at svaret kommer direkte fra, hvad vi kender fra sandsynlighedsregning). Så hvis nu de forskellige implikationssætninger, der tages i brug, ikke er korrelerede, ``lægge sandsynlighederne sammen'' mere eller mindre (hvilket altså nærmere bestemt vil sige $p=p_1 + p_2 - p_1 p_2$, hvis vi eksempelvis taler om to sandsynligheder). Og hvis de \emph{er} korrelerede, jamen så giver vores teori for sandsynlighedsregning også svaret på, hvad man gør her (hvilket jeg dog ikke lige kan og/eller gider at tænke mig til på stående fod, for jeg har ikke været udsat for så meget regning med korrelationer i min tid hidtil). 

Ok, og som nævnt skal man så også i sidste ende kunne tilføje, hvad der svarer til eksperimentelt data (men som ikke behøver at komme fra laboratorier; det kan være alle mulige hændelser fra det virkelige liv, som hører ind under modellens domæne af hændelser, som den giver forudsigelser omkring). Jeg har heller ikke selv før set på, hvordan man generaliserer Bayes' sætning, når der også indgår korrelationer som en del af prior-sandsynlighederne, men mon ikke det er rimeligt ligetil?\,.\,. Hm.\,. Hm, i princippet så kan man jo altid bare konvertere alle sine ``prior-sandsynligheder'' fra modellen (inklusiv de vurderede korrelationerne) til sandsynlighedskurver i stedet.\,. Ah, men vil man måske ikke egentligt kun gøre dette, når det kommer til korrelationerne, og så lade de andre ``prior-sandsynligheder'' (som altså bare er alle udsagn, som brugerne har vurderet, før eksperimentel data / data fra den virkelige verden begyndte at blive uploadet til modellen) være.\,.\,? Hm, måske skal man bare sørge for, at alle korrelationer i udgangspunktet gives ud fra sandsynligheds kurver i stedet for bare sandsynlighedsværdier. Ja, det må i hvert fald kunne løse det. Hvis der findes en bedre måde, så kan man jo bare bruge denne, men ellers vil man jo altid helt sikkert kunne løse problemet (så man opnår en protokol for, hvordan modellen skal justeres som følge af ``eksperimentelt data'' (/ udfald fra den virkelige verden)). Ok. :)

Så sådan kan mine såkaldte ``prædikative modeller,'' som jeg har snakket om før i det ovenstående, altså implementeres. Jeg tror så, at dette vil udgøre et kæmpe teknologisk fremskridt, for når først folk begynder at kunne nørkle med sådanne modeller i store (bruger-)grupper, så tror jeg, folk vil lære \emph{så} meget af dette. Jeg tror virkeligt man vil kunne komme frem til mange nye indsigter som gruppe ved at arbejde på sådanne p-modeller. Jeg tror virkeligt, at denne teknologi vil blive en game changer på mange punkter.\,. Men ja, det kan jeg jo sagtens sidde her og prale med. Det vil tiden alt sammen vise. 

*(23.12.21) Hov, jeg mangler vist at tilføje noget her. Tja, jeg kunne starte med at tilføje, at min idé til en hjemmeside, hvor folk kan rate forskellige prædikater om ressourcer, måske ikke er helt nok til at implementere disse p-modeller. For de kræver jo gerne lige, at brugerne selv kan vælge nogle input-parametre, som altså sætter alle prior-sandsynlighederne og -korrelationerne i en p-teori, og så skal serveren gerne ud fra disse kunne udregne endelige sandsynligheder for hver udsagns-ressource, brugeren hiver fat i. Og dette kræver, som jeg kan se det, altså lige lidt mere af serverne og af systemet. Men når vi tilgengæld først begynder at få det, så at brugere selv kan bygge server-algoritmer, som kan indstemmes og implementeres i server-netværket (altså i et Web 2.1-netværk), så vil man jo kunne bygge systemet til p-modellerne her. Ok. Og så skal jeg også tilføje, at hvis man har flere implikationssætninger, der kan konkludere på de samme propositioner.\,. Hov nej, vent nu lidt.\,. Hm nej, dette giver sig selv, når først man allerede har begyndt at inkludere korrelationer i det hele. Jeg ville have skrevet noget med, at hvis to implikationssætninger ender med at konkludere noget meget forskelligt, så kan man sørge for at give et klart signal til brugeren om dette.\,. Tja, måske kan denne funktionalitet faktisk godt give mening uanset hvad, men om ikke andet så jo, så kunne dette så være i en tidlig version, inden man når så langt som til at arbejde med korrelationer. Man kunne altså i så fald bare nøjes med at vise et gennemsnit for de to udregninger, og så give et klart signal, hvis de to resultater (eller flere) ligger langt fra hinanden. Fint. 






\subsection[M.o. Blockchain]{Mere om blockchain}
(23.12.21) Jeg har nogle noter omkring en ny idé til en blockchain-løsning, som ikke egentligt er en blockchain i virkeligheden, men løsningen går på en decentraliseret database, hvor rækkefølgen af transaktioner kan bestemmes (på en decentral måde), og hvor angribere (umiddelbart, måske; hvis alt altså går, som jeg håber) heller ikke kan komme og ændre i historikken (for det vil, som jeg umiddelbart ser det, være for let for offentligheden at opsnappe, hvis nogen prøver at ændre historikken). Jeg har ikke lige tid til at arbejde videre på det, og heller ikke til at skrive det ind nogen andre steder. Så indtil videre kan noterne (som heller ikke er super klare, som de er nu; man skal nok læse dem mange gange for at forstå, hvor jeg vil hen --- men det bør man nu nok kunne i sidste ende) læses ude i kommentarerne, under hvad der pt.\ er den sektion, der bare hedder ``Web ideas''/``Ideas for websites'' under ``Draft to first publication,'' men som om ikke andet kan findes, hvis man søger på ``22.12.21'' i dette dokument (ikke det første resultat, man finder, men det næste). 



***(03.06.22) Okay, jeg kan ikke lige helt finde ud af, hvor det er bedst at placere dette, så jeg placerer det bare her. Jeg ved, der er et andet sted i disse noter, hvor jeg forklarer, at jeg nærmest har droppet alle mine blockchain-idéer. Der burde jeg næsten placere dette, men jeg gider ikke lige at lede efter det, så det bliver bare her. Men ja, det er altså den status, jeg sluttede af med. Men nu har jeg så for nyligt (et par dage siden eller tre) kommet frem til, at jeg faktisk godt kan udgive min ``ny angrebsvektor''-idé, for der er nok alligevel lidt nyhedsværdi i det --- og i øvrigt også på en ret positiv måde. For selvom et angreb ikke vil kunne fungere i praksis, i.e.\ fordi folk kan se, at det vil være for hurtigt at stable et forsvar på benene, så er det stadig interessant, for man kan jo så bare fokusere på at sige: Jamen, kryptofællesskaber, hvis et sådant type forsvar alligevel skal implementeres så snart nogen forsøger sådan et angreb, hypotetisk set, hvorfor så ikke bare indføre det før snarere end siden?\,. For et sådant forsvar vil nemlig også gøre pågældende kæde/valuta mere sikker i det hele taget, da forsvaret også med ret stor effektivitet vil kunne bruges mod almindelige 51 \%-angreb. Så valutaerne vil altså blive mere sikre, og især for mindre kæder. Og desuden vil man så også kunne gøre kæderne meget mere ressource-effektive, så man kan spare rigtigt mange energi-/regne-ressourcer. Og dette vil jo løse et stort (bl.a. PR-)problem for blockchain-teknologien. (Ja, og hvis man skulle være lidt fræk, hvad jeg dog ikke har tænkt mig, så kunne man også næsten sige, at det er ret uansvarligt ikke at tage sådan et skridt som fællesskab (hvilket især handler om købernes tilgang i fællesskabet i princippet (men i praksis hænger det sammen med alle i fællesskabets tilgang --- det er bare køberne det kommer an på i sidste ende i princippet)), når nu muligheden jo ret klart er der.) Men ja, jeg behøver nu ikke at løfte pegefingeren på den måde; jeg kan bare udgive en lille tekst, som forklarer angrebet, og som siger: Hey, skal vi så ikke tage det skridt og gøre teknologien mere sikker og mindre ressourceforbrugende?\,. 















\chapter[Draft to first publication]{Draft to the first notes I will publish (27.10.21--20.02.22)} %(Which will just be a set of notes with a somewhat loose structure and tone. :))

%This draft is a continuation of...
%
%All the sections are written as (relatively short) summaries of the relevant ideas... The details...



*\textit{(20.02.22) I noterne fra i dag ude i kommentarerne under ``Trying to outspend the attackers...''-undersektionen kan man læse om, hvorfor jeg afslutter denne ``Draft to first \ldots''-sektion og tilmed også hele dette dokument, og om hvad mine nye planer er nu her fremadrettet.\,:) .\,.\,Lad mig også lige fremhæve, at der i denne ``Draft \ldots''-sektion altså er mange noter, der bare står i kommentarerne (i kildeteksten), som dog alligevel kan være værd at kigge på, hvis man er interesseret. Der er godt nok meget brainstorming, som sikkert ikke er så interessant at læse, men der står også en del idéer omkring mine web-tanker, min ``lykke-valuta''-idé, og om mine planer fremadrettet, som egentligt er ret vigtige (for mig i hvert fald) --- især når det kommer til web-idéerne --- men som jeg desværre ikke har tid til at skrive ind og opsummere her i den renderede tekst.}\\




*(07.02.22) Jeg kom frem til her d.\ 31/01, tror jeg, det var, at jeg jo har kød nok på min fysik-opdagelse til at udgive den, også selvom jeg ikke lige selv kan nå at vise selvadjungerethed. Og dette gjorde mig så overbevist om min nye plan, hvilket er bare at starte med at udgive min attack vector-idé (og idéen til forsvaret) først, mere eller mindre for sig selv, og så bare skrive nogle teasende noter, som opsummere nogle af de idéer, jeg vil skrive om og udgive efterfølgende. For det første vil jeg tease mine web 2.1-tanker, og her vil jeg nok faktisk være ret uddybende (ift. de andre). Jeg vil dog ikke lige forklare om den økonomiske side af idéen til at starte med; kun tease, at der er nogle idéer på vej omkring det. Og så har jeg også tænkt mig også at tease UX- og ML-gevinsterne ved sådan en side og nævne, at jeg også har nogle idéer, der tager udgangspunkt i (brugerdrevet) faneopdeling af kommentarer og i kontinuerte rating hhv.\ (plus det løse), og har idéen om et mere simpelt system til bruger-kategorisering end f.eks.\ FOAF. Og ja, at disse idéer altså kan booste ML-mulighederne for det første, og self.\ også bare UX'en ved at blive hørt bedre af dem, ens uploads har relevans for. Jeg vil også kort nævne, at der findes en god måde at sikre sig anonymitet på, selv på en måde så man får nærmest fuld udbytte af ML-potentialet (nemlig simpelthen ved at lave en god opdeling af kontoerne). Ah jo, og så vil jeg jo ikke mindst altså også nævne, at jeg har opdaget en Hamiltonian for fuld QED (over fotoner og elektroner/positroner), som jeg mener at kunne vise, er Lorentz-invarient, også selvom den indeholder et led, der tilsyneladende ikke ser Lorentz-invariant ud (ud over de forventede led, som svarer til interaktionerne Feynman-diagrammerne i feltteorien), \emph{hvis} altså bare, det kan vises, at den er selvadjungeret. Og jeg kan måske nævne helt kort, at dette led altså ser ud til at udgøre Coulomb-interaktionen for lave energier, og at min teori altså dermed ikke tyder på, at denne er båret af fotonerne, i hvert fald ikke særligt meget og især altså ikke for lave energier. Så det er altså planen nu. Jeg vil skrive mine attack vetor-noter i nedenstående sektion, hvor jeg allerede (pt.) har arbejdet lidt på dispositionen (selvom jeg også lige har noget, jeg skal tilføje til den). Hvor jeg skriver de andre (teasende) noter, kan jeg lige se ad. Når jeg ved det, kan jeg lige indsætte en notits om det her. 
%... Spørgmål: Skulle jeg mon sørge for også at nævne, at en web 2.1-side jo gerne må være stærkt lagopdelt i API'er, og særligt hvor de.. nederste eller øverste, om man vil.. yderste, kunne jeg sige.. API'er skal være rigtigt nemme for brugerne at lære og bruge (og pointen er meget, at de gerne skal være nemmere og nemmere at bruge)..? (Og helt yderst har man så også muligheden for nogle personlige brugerindstillinger i selve app-brugerfladen, som selvfølgelig gerne skal være \emph{rigtigt} nemme at bruge (og uden noget som helst kodning, ingen gang WYSIWYG-kodning (hvad ellers måske kunne være meget smart i nogen af de aller yderste lag før selve app-brugerfladen)), det er klart..) ..Tja, det skal i hvert fald nok i så fald bare nævnes i en helt kort sætning eller to højst.. 


%(12.02.22) Omkring web 2.1 osv., så tror jeg bare, jeg vil sige de overordnede ting om, at en open source web 2.0-side, hvor skaberne ikke bare bidrager med indhold men også design(-forks) og algoritmer osv., selvfølgelig vil være bedre i fremtiden.. Så ja, en fremtidig side, som i langt højere grad er i brugernes egne hænder.. Og jeg vil så påpege, at økonomien omkring dette kan blive en kæmpe stor faktor for, om det vil virke. Og så kan jeg pointere, at hvis man går væk fra reklamer, så er der bestemt penge nok. Og jeg kan jo så tease, at jeg selv har en specifik forretningsidé. Og så vil jeg også lige pointere angående algoritmerne, at det jo langt hen ad vejen bare handler om at finde korrelationer. Og her tror jeg altså, der er meget at vinde ved at inddrage brugerne selv i dette arbejde. For blandt andet kan brugerne så selv studere korrelationerne og prøve at finde frem til, hvad de skyldes --- og prøve at navngive det fænomen, de repræsenterer. Tillige kan de så finde på, hvad man kunne lave af spørgeskemaer for bedre at adskille korrelationsvektorerne ad og finde frem til nye, formodne korrelationer. Brugerne kan således arbejde med hypoteser omkring, hvilke korrelationer, der må kunne findes (givet generelle psykologiske overvejelser), og de kan så altså arbejde med bl.a. at udforme spørgeskemaer osv. for at prøve at eftervise (eller forbedre) disse hypoteser. Så denne del af det vil jeg altså også tease. Og i denne forbindelse må jeg så altså hellere også lige kort opsummere min anonymitetsidé, som jeg jo også har tænkt mig. ..Og så har jeg jo også tænkt mig lige at nævne, at jeg har nogle idéer, der tager udgangspunkt i henholdsvis noget med at indføre brugerlavede kommentarfaner og med at bruge kontinuere ratingskalaer i langt højere grad, bl.a. til tags.. (..og måske nævne at at dette så kan komme til i sidste ende at implementere de relationer, som tripletterne i gængs w3.0 skulle udgøre..) Hm.. Nå, men dette er også knapt så vigtigt at tease (måske behøver jeg det endda slet ikke), så det kan jeg jo bare lige se..  
%(18.02.22) Nå ja, og idéen omkring kommentarfaner har jo udviklet sig --- eller splittet sig i to, kunne man nærmest sige --- så der altså også i høj grad er idéen, om at alle ressourcer på hjemmesiden skal inddeles i brugerlavede faner, hvilket så kan ses som en slags mængder/klasser, og fanerne selv kan altså ses som en relationer/prædikater. Og så er en stor del af idéen jo dermed, ligesom at starte, hvad der svarer til det semantiske web (og hjemmesidens ressourcetyper skal nemlig meget gerne være meget alsidige), men altså drevet af (i starten), at brugere gerne vil vurdere og klassificere ressourcer på en Web 2.1-hjemmeside (for andre brugeres skyld, og måske også lidt for deres egen). Så i starten vil omtalte relationer/prædikater altså i høj grad handle om, at sortere og vurdere ressourcer på hjemmesiden, og om derved at skabe bedre (brugerlavede) feed-/søge-algoritmer på siden. Dette tror jeg altså, kan blive kæmpe stort, og jeg tror dermed, at dette faktisk kan blive vejen til det semantiske web (nemlig at det kan startes via, hvad der svarer til en Web 2.0-hjemmeside (men rettere hellere skal være en "Web 2.1"-hjemmeside)). Og hvis man nemlig i øvrigt også har tilsvarende faner/relationer/prædikater for kommentarer (og annotationer osv.), så kan resten af det semantiske webs visioner nemlig også følge med (..antaget at hjemmesiden også i høj grad er open source og decentral, så at rettighederne (med tiden i hvert fald --- og helst inden for en kort tidsramme) kommer ud på alles hænder). ..Nå ja, og så vil jeg også gerne lige nævne, at en ret vigtig idé omkring dette, altså er, hvordan brugere skal kunne rate ressourcer ved at (rykke rundt) og placere dem på en akse/liste, og hvor man gerne skal kunne "zoome ud," så at sige på listen/aksen, så man kan se en stor del af den, men hvor der så kun vises enkelte ressourcer, som er kendte for brugeren. Så brugerne skal altså på den måde have en rigtig nem og lækker måde lynhurtigt at vurdere diverse ting, ved at placere dem ift.\ andre ting, de allerede kender. (Og her snakker vi jo så ratings med hensyn til det underliggende prædikat, som fanen repræsenterer (udover at fanen dog også kan være sammensat af flere prædikater, hvoraf resten af disse altså bare kan indgå som "filtre" for visningen).)




\section{An attack vector on PoW chains and how to defend against it}

%(15.02.22) Jeg har faktisk ikke rigtigt fået set på min gamle disposition (fra 27/12 af) endnu, men har fået tænkt lidt en ny disposition (altså bare i hovedet), som jeg tror jeg egentligt bare vil gå i gang med at skrive ud fra. Og så kan jeg altid kigge på det nedenstående bagafter for at se, om der er noget, jeg mangler. Desuden kan det nok være, at jeg også bare skifter dokument, inden jeg begynder at rette det til, så på den måde kan følgende tekst altså på en måde også halvt ses som lidt dispositionsarbejde. 

\subsection{Introduction}
I believe I have discovered a new attack vector on Proof of Work (PoW) blockchains (such as Bitcoin and Ethereum). It involves initiating a 51\% attack on a side chain of the blockchain in full view of the public and then trying to recruit other attackers publicly to help finance the attack. The recruited attackers could first of all be users who have recently sold crypto coins on the blockchain and who then stands to gain from a reversal. But in more severe versions of the attack, every user who has sold or bought crypto coins in the recent period can be targeted for recruitment, namely by employing a divide-and-conquer strategy where users need to join the attack quickly in order not to become victims themselves. 

Luckily I have also discovered a solution to defend efficiently against such attacks, which can probably even be implemented in a much shorter time span than what it will typically take for such %public 51\% 
an attack to be carried out. 

In the following sections, I will give more details on how the different versions of the attack vector work and then go on to explain the defense solution and discuss its implications.

%\subsection{The basic vulnerability in play}
\subsection{The underlying vulnerability}
The attack vector%
	\footnote{I have deemed the question of whether to classify these attacks as several versions of the same attack vector or just as several different attack vectors to not be in the scope of this paper. So throughout the text, I will simply be referring to it as \emph{one} attack vector with several versions, even though this might not be... the most correct thing to do... ***(Needs revision.) *Hm, og jeg tror faktisk muligvis, at jeg næsten bør snakke om i hvert fald to vektorer (også fordi nr.\ to jo faktisk potentielt set kan imødegås lidt ved at have en blochchain, hvor kontrakterne også skal indeholde en reference til en (ikke alt for gammel) blok.\,.)}
exploits the fact, that a blockchain such as Bitcoin or Ethereum %, in the present moment at least,
typically carries a significantly greater volume of transactions than what is mined in the same period of time (also when including the mining fees). 
For normal (nonpublic) 51\% attacks, where the attackers are limited to very few users, the typical mining cost for a period larger than the confirmation time is too great for the attack to be beneficial, at least if we only consider the well-established blockchains. In theory the attack could work if the attackers sell a large enough amount of crypto coins but in practice the community will probably be alerted by such big transactions and remind the buyer(s) to require an extended confirmation time. But when attackers are recruited publicly from the whole userbase, neither the buyers nor the community at large will be able see an attack coming if the transactions from the attackers thus does not differ from the... bla bla; needs revision anyway...

bla bla... typically several hundreds times larger (or however to express that..)...

bla bla... The vulnerablity thus lies in the fact that otherwise normal users can be recruited at any time in theory..


\subsection{Public 51\% attacks via side chains} %Jeg overvejede kort, om det ikke burde skrives som "51-\% attack," men konklusionen bliver jo så nok bare, at man i stedet bare bør slette mellemrummet..
To exploit this vulnerability, initiating attackers can create a new side chain on which they, as well as all who are subsequently recruited to take part in the attack, sign smart contracts to reward miners of the side chain. The idea is then to choose exactly the same mining protocol as the main chain uses and to chose a dead fork from an earlier point 
	***(Hov nej, ``temporary forks'' er nok ikke lige så almindelige, som jeg tænkte her, så dette skal skrives lidt om. Jeg skal i øvrigt måske droppe at bruge ``forks'' rigtigt, og måske altså bare holde mig til at snakke om ``main chain'' og ``(attacking) side chain.''...) 
of the main chain as the starting block of the side chain. In other words, the side chain should be a fork (initially deemed as a dead fork) of the main chain itself. If enough attackers can be recruited to finance large enough rewards of the side chain (and for a long enough time), then the side chain has the potential to overtake the main chain in time and become the longer of the two. If this happens, then by the normal consensus of PoW blockchains, it will then become the new main chain and the 51\% attack will have succeeded. Of course one might then point out that if the miners then also looses all the rewards mined on the side chain at this point, how can the attackers ever convince them to complete the attack fully? But all the attackers need to do to eliminate this problem is to make sure that the same smart contracts used to finance the side chain is also copied over and added to the side chain itself. 

%Let us do a quick calculation on how much money such an attack will cost, where we for simplicity make the naive assumption that the original main chain will retain all its mining power, even when the attack is in an progressed stage where it seems it will succeed. And let us also assume the mining power is proportional to the reward money with the same constant of proportionality for both forks, i.e.\ the main chain and the (attacking) side chain. (This might also not be completely true in reality.) ..bla bla... ...Hov, eller denne paragraf giver måske ikke så meget mening, for med den antagelse, så vil svaret jo bare blive lig, hvad main chainen har kostet siden fork-punktet, men dette vil det jo aldrig blive i praksis, for dette kræver så ubegrænset udbud af regnekraft.. Så mon ikke jeg bare lige udkommenterer dette.. 

%But in reality the cost needed might be even lower than this since miners might potentially abandon the original main fork %..Nvm..

%Hm, jeg tror måske bare, jeg vil skrive en lille paragraf, der snakker om, at det kan betale sig for angriberne at sætte rewarden så højt som muligt i starten, hvilket så alt andet end lige vil få minere (i teorien altså) til at migrere til sidekæden. Og når det sker.. Hm, men angriberne skal jo stadig vise, at de også har pengene, før det argument, jeg var ved at skrive holder, ikke..? Altså de skal først vise, at de har pengene til at udføre angrebet, og så vil folk begynde at tro på, at midlertidigt ugyldige coins minet ud fra den basale mining-protokol på et tidspunkt vil blive gyldige (når angrebet altså lykkes), hvilket så vil gøre, at angriberne herfra kan begynde at eksperimentere med at skrue den nu "ekstra" sidekæde-reward lidt ned.. Hm ja, og for at dette argument holder, så skal de dog vise, at de har pengene, hvilket altså vil sige, at man stadig behøver at rekrutere ligeså mange angribere.. Men man kan dog sige, at pågældende angribere jo ikke behøver nødvendigvis at \emph{ofre} alle de penge, de stiller til rådighed (og "viser"), for man kan jo gøre det sådan, at rewarden kan skrues ned igen efter en automatisk og/eller demokratisk protokol og/eller at de kan abortere angrebet helt, også ud fra en automatisk og/eller demokratisk protokol.. 
%Så hvad skal jeg mon skrive..? ..Hm, det må næsten så komme som et senere punkt i stedet, nemlig at de rekruterede angribere potentielt står til at miste lidt, hvis angrebet mislykkes, men altså ikke nødvendigvis alt, hvad de har stillet til rådighed for sidekæden.. 

%(16.02.22) Jeg sidder og skal lige til at i gang med at finde ud af, hvad jeg skal fortsætte med at skrive, men nu kom jeg så lige frem til, at jeg måske faktisk har fundet en separat sårbarhed i Lightning-netværker (LNs), der ikke har med public 51\% attacks at gøre.. ..Hm, jeg ved det godt nok ikke helt endnu, så jeg skal lige læse op på det hele, men måske er der altså en sårbarhed der.. Det kom sig af nogle tanker i badet i morges (den er tyve over ti nu, btw), som handlede om ovenstående vektor, men det er så lige ført til noget (og forresten ikke til andet; mine oprindelige tanker "virkede ikke"), der er uafhængigt af det andet.. ..Men jeg tænker dig, at jeg bare venter med at undersøge denne sårbarhed til efter, jeg har udgivet denne. For \emph{hvis} denne nye sårbarhed er, som jeg tror, den er, så bliver konklusionen alligevel bare: Vær venlig at hold lidt inde med LNs, hvis det er muligt, for så vil jeg nemlig snart (derefter) udgive, hvad jeg mener, er en sårbarhed ved dem (som dog vil kræve nyoprettede kanaler, hvis en angriber pludeligt vil prøve at udnytte denne sårbarhed efter at være blevet klar over muligheden (og i øvrigt vil det også kræve en hel del ressourcer..)).. ..Ja, jeg kan godt sagtens tillade mig at vente med at undersøge dette, og det vil jeg så også gøre. 
%Nå ja, og lad mig lige forklare også, at sårbarheden altså går ud på, at oprette flere LN-kanaler anonymt fra flere forskellige konti til at modtage penge, og så bruge dette til at finansiere et 51 \%-angreb (hvor man dog alligevel også først skal have ressourcerne på rede hånd (og dette kan man, som jeg kan se det, ikke komme uden om)). Hm, men mon dog at LN-netværket overhovedet indeholder nok ressourcer, som det er nu, til at dette overhovedet kan lade sig gøre --- og hvis man så som klargørelse til et angreb får disse ressourcer til at stige meget, så kan brugerfællesskabet jo nå at opdage det herved. Okay, så hvis dette altså passer, så er der måske altså slet ikke nogen overhængende fare her, så længe LNs'ne bliver på det niveau, de vist (og det bør jeg lige læse op på, måske senere i dag) har nu.. ..Okay, fint nok. Det får jeg lige læst op på, og hvis jeg ikke vender tilbage hertil, så er det fordi, at de samlede ressourcer ad gangen på LN'et/s'ne ikke rigtigt er store nok. *(De er måske store nok, men idéen holder ikke af andre grunde. Eller rettere, det er ikke LNs, der giver sårbarheden. Sårbarheden er i stedet bare, at man ved at lave mange handler potentielt kan opnå flere penge at vinde ved et 51\%-angreb. Se nedenfor (ude i kommentarerne også) under "Trying to outspend the attackers..."-sektionen.)

Since the attack happens in full view of the public, any user who has sold more crypto coins total than they have bought, and who is opportunistic enough, can join the attack. The initiating attackers can even design the relevant smart contracts in a way such that the money of the attackers only gets locked to the side chain once a threshold is reached, after which the rewards for mining the side chain will be automatically given. The initiating might also implement a way to abort the side chain, perhaps by a democratic process, such that all attackers will be able to get their remaining money back if the attack is countered somehow before it succeeds. Such fail saves could of course help persuade more attackers be recruited. 


\subsection{A more advanced version with... A divide-and-conquer strategy for recruitment...}
...



\subsection{Trying to outspend the attackers...}
%Hm, jeg har før tænkt, at denne løsning ikke holder så godt, for så kan minerne bare selv starte angreb, men nu bliver jeg så lidt i tvivl.. Mon egentligt ikke dette alligevel kunne være ret effektivt til at afværge angreb, nemlig bare et løfte om, at fællesskabet vil gå sammen om støtte ofrene til et modangreb, og hvorved man i øvrigt kan vente indtil angriberne allerede har brugt mange af deres penge, for at straffe dem.. ..Mon ikke det faktisk vil være nok til at overtale langt de fleste brugere til ikke at deltage (selv hvis vi altså antager, at de ikke har nogen moralske kvaler med det)..? For selvom man i princippet kunne have, at "miner'ne bare er lig angiberne," så vil dette jo ikke rigtigt være til i praksis, eller vil det?.. ..Hm, vigtige tanker; godt at få dem nu, og ikke efter at jeg har udgivet det..(!..) ..Hm nej, det \emph{kan} jo faktisk ikke rigtigt lade sig gøre i praksis, for hvis miningen overvejende sker ret decentralt, hvad den alligevel alt i alt gør, så vil omverden hurtigt få nys om, hvis en stor gruppe minere prøver at konspirere. Og hvis vi kun snakker en meget lille gruppe af minere, så vil de ekstra mining-penge jo overvejende alligevel gå til alle mulige andre end lige dem. Hm.. ..Hm, og hvis ikke der er noget, jeg har overset her, bliver counter-angrebsstrategien så ikke nærmeset for triviel (når man ikke længere tænker, som jeg gjorde, at denne bare vil medføre et nyt sæt af problemer) til, at angrebsvektoren overhovedet bliver særlig meget værd at interessere sig i..? ..Hm, der var dog også noget med, at man jo kunne shorte kæden.. ..Ah, men dette hjælper ikke på.. eller vent.. Hvor meget vil crypto-kursen mon falde, hvis en counter-strategi bliver udført? For dette vil jo.. Hm, der bliver cirkulært, så mon ikke kursen vil bevares rimeligt meget..? Lad mig se, man får jo så afværget angrebet, hvilket også bør sende et signal til andre (fremtidige) angribere.. ..Så kun hvis angriberne alligevel kan få noget ud af det, så må man regne med, at andre (fremtidige) angribere så også vil blive afskrækket, hvilket så vil bevare kursen, for så er der lille chance for, at man skal bruge disse penge igen (til at afværge et nyt angreb).. Og vil angriberne altid (i reglen) tabe på det?.. (..Altså også selvom de shorter og gør ved..(?)) ..Hm, jeg er bange for, at svaret bliver ret simpelt, men lad mig alligevel lige tage en pause og summe lidt over det, for det fortjener det jo alligevel (det er jo ret vigtigt/vægtigt ift., hvordan jeg skal fortsætte herfra).. ... 

%(17.02.22) Okay, der er en del ting, jeg skal skrive. For det første, ja, der var et ret simpelt svar på ovenstående (hvor jeg sluttede i går middags).. Jeg må på en eller anden måde have tænkt (tror jeg; og jeg gider ikke lige prøve at gå tilbage at læse om min tankegang dengang), at det faktum at folk kan shorte kryptovalutaen var nok til at gøre, at angriberne altid vil kunne få noget ud af det.. Hm tja.. Nej, never mind; idéen, som jeg var ved at skrive om, falder stadig til jorden, uanset hvad, for hele idéen handler jo om, at det er et \emph{offentligt} 51\%-angreb, men man kan jo ikke forvente, at de rekruterede angribere så også kan nå at gardere sig ved at shorte valutaen. Så ja, dette må altså ryge ind under kategorien af endnu en blockchain-idé, som ikke holdt i sidste ende. ..Nå, lad mig så lige snakke kort om, at dette jo var en anelse nedslående, men på en måde forventede jeg jo også et eller andet sted, at det kunne ske. Det er jo derfor, jeg også så gerne ville arbejde på fysikken sideløbende, så jeg også havde den idé at slå igennem med, hvis den anden skulle være en fuser (hvad jeg så nu kan se, at den måtte blive). Og selvom jeg burde have dykket ned igen i, hvor god faren for et sådant kapløbsmodangreb egentligt kunne være til at afværge angreb, og altså nu nok burde have overvejet dette igen noget tidligere, så er det jo bare sådan, det er. Ofte fanger man først sådanne huller, når man prøver at formulere den endelige sammenhængende tekst. Og \emph{så} forfærdelig lang tid har jeg heller ikke brugt på lige denne idé.. Nå, så det var altså en anelse nedslående i går, men ikke vildt meget. Og jeg kom så i første omgang bare frem til, at jeg så bare måtte satse mere alene på min QED-teori og begynde at lave en lille opsummering og udledning af denne --- hvor jeg så altså bare dropper helt at vise selvadjungerethed selv og siger, at den chance har jeg haft, og at nu er det bare bedre at få det ud, så andre kan deltage i at løse det problem, hvad jeg i øvrigt også stadig har tænkt mig nu; det bliver nok planen, når jeg når til det. *(Og hvis det så er en eller anden nem ("indlysende") løsning, som jeg har overset, så er det bare ærgerligt; sådan er det jo bare.) Men jeg fik dog også alligevel nogle flere blockchain-idéer i går aftes, og nu tror jeg faktisk alligevel, at jeg har noget jeg kan gå videre med der. Jeg tror således at jeg muligvis har nogle nye idéer til/omkring angreb, som er værd at dele, og som nærmest nok (forhåbentligt) kan du helt som en stedfortræder for min gamle angrebsvektoridé. Hm, men inden jeg når til at diskutere disse idéer, så kan jeg også lige sige, at min idé omkring LNs ikke rigtigt holder --- eller man kunne også sige, at den bare reducerer til en idé om, at hvis man laver et 51 \%-angreb, hvor man også inkluderer \emph{nogen} (men ikke andre) transaktioner, så kan man ved at lave mange handler (der så ikke har noget med LNs at gøre) altså potentielt opnå mange penge at vinde ved et angreb, nemlig endda flere, end hvad man har ejet på noget tidspunkt. (Lad mig også lige nævne dette ovenfor.. ..Sådan..) ..Okay, så nu har jeg altså nogle idéer fra i går aftes, jeg gerne vil skrive om, og her kort inden jeg startede skriveriet kom jeg så også på en supplerende idé, omkring at man også bruger min "divide and conquer"-idé til at forøge gevinsten som (indledende) angriber. ..Ah, og der er også en anden idé/tanke fra i går aftes, som jeg lige vil nævne først, og der er, at jeg nok bare skal hive det hele en tak eller to ned ift. den måde, jeg formulerer det på, og ift. hvor.. sensationelt jeg ligesom gør det til. Der er ingen grund til besynge opdagelsen selv, for der \emph{vil} bare altid være en stor risiko, at jeg tager fejl (jeg er jo bestemt heller ikke særlig vidende eller erfaren på hele området (om blockchain)). Og desuden bliver mine nyeste idéer, hvis de holder, nok også bare mere low-key end de forrige (da jeg tænkte, at de holdt). For de nye idéer er alligevel nogen, som jeg tror, andre sikkert sagtens også kan have tænkt potentielt set.. ..Ja, så jeg tror mere bare min artikel (eller hvad det bliver) kommer til at handle om, at jeg vil gøre opmærksom på nogle ting og nok bruge dette til at argumentere for en soft fork, hvor man indrekte får en lidt mere "Proof of Public History" (som jeg lidt har kaldt det før) -agtig kæde.. ..Ja, så jeg tror altså lidt, at jeg vil udforme udgivelsen mere som nogle argumenter for, hvorfor man hellere bare burde begynde at bruge soft forks, hvor man bare sørger for at have et decentralt system af instanser, som hver især kan meddele, hvilke nogle.. ja, i dette tilfælde så hvilke nogle forks, de ser er fremme i offentligheden, og så må der bare være en indirekte aftale om, at nye købere, medmindre de gerne vil miste deres penge, altså sørger for kun at købe mønter fra forks, som har været ude i offentligheden, hvad der i praktiske henseender svarer til "hele tiden," efter begyndelsespunktet fra soft forken, og hvor man derfor aldrig køber mønter på en kæde, der tydelighvis indeholder et 51\%-angreb i sig efter pågældende tidspunkt (medmindre igen at man altså har lyst til at miste sine penge). (Og så skal kunderne (nodes'ne) også bare opdateres, så brugerne (\emph{aktivt}; det skal nemlig gøres uden nogen central stillingstagelse, i hvert fald ikke på et grundlæggende, teknisk plan) kan fravælge forks med angreb i sig.) 
%Nå, nu til at prøve at forklare de her idéer fra i går aftes (og her i formiddags).. ..Jeg har allerede kort nævnt, at handlende godt potentielt set kan opnå mere end bare at fordoble sin formue på et 51\%-angreb. Dette er simpelthen fordi, de kan oprette en masse uafhængige punge, hvor de først køber kryptovaluta til og med det samme sælger det videre. Så snart de har solgt dem videre og har fået pengene (altså ikke-krypto-penge) tilbage kan de så gentage processen for en anden pung. Hvis vedkommende forholder sig anonym, så skal den indledende formue altså bare være større end, hvad der bliver tjent af minerne på to gennemsnitlige confirmation times, før at aktøren dermed kommer til at få mere og mere at vinde ved et 51\%-angreb (hvor alle transaktioner til omtalte punge bliver medtaget, men ikke dem væk fra dem igen). Og aktøren kan i øvrigt godt dele disse handler ud på mange punge ad gangen, så det er altså langt fra sikkert, at fællesskabet kan opdage denne aktivitet (af hvad jeg ved). ..Hm, jeg skulle til at sige, at aktøren i teorien ikke taber noget på disse forberedelser, hvis kursen er rimelig stabil, men der er dog mining fees'ne, som vedkommende godt nok vil tabe.. ..Nå, men hvis angrebet altså er forestående, så kan dette altså alligevel være en måde at forberede sig på, hvor man altså for meget at vinde til sidst.. ..Min tanke var så, at man derfor måske især egentligt skal passe på med centrale handelscentre, som ligenetop har sådan en aktivitet, hvor de køber og sælger kryptomønter fra alle mulige forskellige hele tiden. Måske er det således godt enten at sørge for, at disse altid køber og sælger via de samme adresser, så denne sårbarhed formindskes, eller også kunne man gå over til mere at bruge handelscentre, der formidler handler imellem sælgende og købende parter (hvis de ikke allerede fungerer på denne måde), således at man altså ikke lægger op til / frister til 51\%-angreb fra sådanne centraler af. Nå, min næste tanke var så på en hel anden sårbarhed, nemlig at 51\%-angreb måske kunne være ret rentable for rige nok aktører (muligvis grupper af aktører (så en slags angrebs-fonde..)), for hvis bare angriberne kan vise, at de har mere end rigeligt til at betale for udførslen af angrebet, og at de endda også underskriver kontrakter, der gør at de faktisk også i værste fald vil \emph{bruge} disse penge, skulle folk forsøge med et kapløbsmodangreb, jamen så kan man jo opnå en situation, hvor de simpelthen bare ikke kan betale sig for ofrene samt sympatisører at forsøge at bekæmpe angrebet, for så vil de bare selv skulle af med den samme store mængde penge. Jo, det vil så gøre virkeligt ondt på angriberne, men det vil jo gøre tilsvarnede ondt på alle involveret i modangrebet. Og hvad værre er, så holder tanken om, at kryptovalutaen så måske vil falde som følge af et angreb, heller ikke. Dette kunne ellers både give mere motivation for et modangreb og kunne også give mindre motivation for et angreb. Men alene af den grund at der jo faktisk findes en forsvarsløsning, som kan ligeså godt kan implementeres efter det første angreb som før (så længe folk bare holder sig til løsningen, når først aftalen er indgået), så er der altså intet rigtigt argument for, at valutakursen vil falde specielt. Hm, nu kom jeg lige til at tænke på, om dette egentligt ikke også var, hvorfor jeg troede, at mine tidligere idéer (som denne sektion skulle handle om) holdt, nemlig fordi dette så kunne afholde sympatisører fra at gribe ind..?.. ..Hm, det kan være, men idéen virker nu alligevel ikke, for selve ofrene har jo i princippet rig mulighed for selv at finansiere et modangreb (for de har jo ikke noget at miste, hvis angrebet ellers alligevel vil ske, og så kan de ligeså godt overbyde angriberne (hvad de nemlig også burde have råd til) og straffe disse, samt muligvis også beholde nogle af deres penge i sidste ende).. Ja, så den idé holdt nu altså nok ikke hele vejen.. ..Men ja, så en ret væsentlig sårbarhed er nu altså nok, at en tilstrækkelig stor angrebsfond faktisk vil kunne slippe af sted med et angreb, der kan gå hen at blive næsten gratis for dem (minus eventuelle fees, hvis deres forberedelserne altså bruger mange mindre transaktioner i stedet for få store (nemlig hvis de gerne vil maskere deres intentioner --- nå ja, eller hvis de altså kan formå at rekrutere en handelscentral som en del af angrebet (hvilke jo ikke typisk selv vil betale transaktionsfees'ne))).. ..Og selvom der vil være en risiko i at stille så mange penge på højkant, så vil sandsynligheden for, at det mislykkes, sikkert være ret lille, og desuden så kan der også blive mere at vinde, jo flere penge man også har at gøre forberedelserne med. Og som sagt så er det ikke sikkert, at kryptovalutaen vil dale i værdi efter angrebet --- og slet ikke permanent --- og selv hvis angrebet skulle føre til et stort krak, så maner dette ikke alle farer bort, for i princippet har angriberne jo, alt andet end lige, også muligheden for at shorte valutaen inden, så de ikke selv mister på dette kursfald. ..Og hvad jeg så kom i tanke om i formiddags, er, at man i øvrigt også kunne supplere dette angreb ved at blande divide-and-conquer-idéen ind i det, for så kunne man for det første muligvis tjene ekstra penge, fordi man så kunne kræve et gebyr af alle rekruterede angribere (som altså ikke er en del af de indledende). Men hvad der næsten er endnu værre, er, at dette også psykologisk kunne blive en ret effektiv måde at skabe splittelse i føllesskabet og få en væsentlig portion til bare at sige: "Oh well, surt for jer, der ikke nåede at joine. Sådan er det bare. I kunne jo bare have været mere vakse, eller nednu bedre, I kunne have sørget for at indgå aftale om en defensiv soft fork noget tidligere." Og det er nemlig bestemt ikke ufordelagtigt, at man måske lige netop kan vinde (en del af) den mere vakse del af fællesskabet over på sin side som angriber, for et modangreb kræver jo lige netop en hel del gåpåmod. ..Og de rekrutterede angribere vil også være anonyme, hvilket også vil være en psykologisk fordel, både når det kommer til at lokke en del af fællesskabet, og når det kommer til den efterfølgende diskussion omkring iværksætning af et modangreb.. Hm, ja så konklusionen bliver jo nok, at en (lidt PoPH-agtig) defensiv soft fork vil være rigtig god at indføre med det samme, inden et sådant angreb.. ..Hm, jeg tror måske lige, jeg vil holde en pause nu (..skal også snart afsted mod tandlægen..) og så summe lidt mere over disse ting.. 
%... Okay, jeg tænker, at hvis jeg skulle udgive noget om det, så skal det jo nok umiddelbart bare tage meget udgangspunkt i pointen i, at hvis man har nok penge (at "flashe" / stille på højkant / gå all in med), så kan man med en vis sandsynlig faktisk få et 51\%-angreb næsten helt gratis (sammenlignet med udbyttet). Og så kunne jeg nævne, hvordan sådanne angribere eventuelt ville kunne forberede sig meget i smug (og endda med måske begrænsede penge involveret i forberedelserne..) ved altså at lave flere små handler i stedet for få store.. Hm, og så kunne jeg altså måske bare tilføje, at handelscentraler herved også kunne udgøre en (ekstra) sårbarhed.. ..Hm, og så er der vel bare lige tilbage at nævne, at man også potentielt set kan forøge udbyttet via en divide-and-conquer-strategi, og/eller man kan bruge dette til at skabe mere splittelse i fællesskabet (grundet at mange så selv får gavn af angrebet i sidste ende), hvilket altså kan formindske chancen for, at fællesskabet vil spare sammen og gå sammen om et modangreb. Hm, nå ja, og så er der også lige kommentarerne om, at man ikke bare kan være betrygget ved, at kursen ved sådan et angreb vil dale, for det forudsætter, at der ikke findes en forsvarsløsning, der godt sagtens bare kan implementeres efterfølgende, hvilket der nemlig gør.. Ja, jeg kunnem måske netop bare nævne dette efterfølgende, her i sidste ende af det hele.. Ja, for det leder jo også meget fint videre til, at man så kunne forklare om den forsvarsløsning, jeg har i tankerne, hvilken man jo heldigvis kan implementere ret hurtigt.. ..Hm, bliver det nærmest ikke det, der bliver planen så..? (..Og så vil jeg bare gå direkte videre til også at lave og udgive nogle noter om min QED-teori..) 
%Nå jo, LNs kunne faktisk også repræsentere lidt en sårbarhed, i hvert fald hvis de vokser sig større.. For det er sikkert sjældent indre knuder i netværket, der betaler fees'ne, når der laves en kæde af transaktioner. Så hvis angribere overtager en stor mængde af LN-knuderne i længere tid, så kan dette potentielt (og altså måske også alt efter LNs' udvikling i fremtiden) blive en måde for dem at lave en masse handler, hvor de ikke selv betaler fees'ne. (Og på den måde kan de måske så generere mere og mere at vinde ved et 51\%-angreb.) ..Ja, så det kunne man også lige nævne, når man alligevel nævner det med handelscentralerne.. 
%(18.02.22) I går aftes kom jeg også frem til, at de indledende angribere jo godt kan maskere, hvem der egentligt er de første ofre.. Nå ja, og så hører det nemlig med, at man jo godt sagtens bare kan give et krav om (og dette kan lade sig gøre via en protokol, hvor folk kan satse penge og gøre indvendinger, hvis dette ikke gælder), at alle andre transaktioner skal inkluderes på angrebs-forken. Så ja, på den måde vil der altså kun være en begrænset mængde indledende ofre, og dermed vil det blive langt sværere for fællesskabet at samle sig om et modangreb. Og ja, altså især, når hurtige brugere kan joine angrebet og tjene en hel del penge selv (imod et lille gebyr måske til de indledende angribere). ..Hm, men det er da næsten lige før, at et angreb mere a la det, jeg havde tænkt mig før, godt kunne lykkes, også selvom man ikke sætter en hel masse penge "all in" på det.. ..Hm tja, måske, men det kan nu også meget vel være, at de fleste folk heller vil gå sammen om at bekæmpe angrebet, for det vil alligevel i gennemsnit være bedre for dem, ikke mindst fordi kursen også endda ville kunne komme til at falde, skulle det lykkes. Men hvis man altså \emph{går} "all in," så kan disse idéer altså supplere.. 
%... Nå, tanken er så faktisk lidt nu, at jeg vil begynde først at skrive en lidt løs tekst om alle de ting, jeg vil "tease" her til at starte med, inkl. omkring disse angreb og løsningen på det. Og måske vil jeg faktisk skifte dokument til et friskt et og skrive det der. (Jeg har også lige skrevet et hængeparti, jeg havde, så nu mangler jeg bare lige at gøre færdigt her, og så mpåske lige indsætte nogle små noter om, hvad man kan læse i kommentarerne i denne "first draft"-sektion..) Hm, men inden da vil jeg nok lige summe lidt over disse blockchain-idéer igen.. ..Hm ja, og lad mig lige gå en middags-/eftermiddagsgåtur og summe over tingene der.. ..Nå nej, lad mig først lige skrive, at jeg i går aftes også kom frem til, at jeg nok ville skulle tænke mere over min idé til en alterativ blochkchain, hvis man virkeligt ville søge efter at udvikle en virkeligt energieffektiv-men-stadig-decentral blockchain. Men pointen er lidt, at det vil jeg faktisk ikke gøre (for når man gør det, så forsvinder meget af mystikken også bag teknologien, og.. ja.. Tja, og i virkeligheden svarer det jo også meget til PoS, selvom PoS, som det er nu, lige har nogle problemer, der skal løses..).. Så nej, det vil jeg ikke bruge tid på, og det er heller ikke vigtigt. Pointen med min løsning er nemlig, at den kan fungere som en \emph{soft fork} i praksis, hvorved man altså slet ikke skal bekymre sig om alle mulige spørgsmål som disse. Og så er det bare dejligt, at denne soft fork også vil kunne føre til en \emph{mere} energieffektiv blockchain (om ikke andet). Og lige for at præcissere, så er pointen altså her, at man, når først forsvarsløsningen er implementeret, godt i princippet kunne bygge videre med en ny soft fork (eller hard fork, men den \emph{kan} faktisk implementeres som en soft fork (ved brug af en slags colored coin-teknologi, hvis jeg altså har gættet rigtigt på, hvad "colored coin" indebærer..)), der også nedsætter mining-lønnen en anelse. Og så længe man ikke nedsætter den så meget, at de kan betale sig for folk at prøve at holde liv i en anden fork (for netop at ødelægge "soft"-kvaliteten hos konsensus-forken), så kan man altså netop godt bare betragte udvidelsen som en \emph{soft} fork. ..Nu vil jeg gå mig en tur.. 
%... Nå, det var vist rigtig fornuftigt lige at summe over det, for nu er jeg faktisk kommet frem til, at min idé om at gå "all in" nok heller ikke holder helt i praksis. For hvad jeg ikke har tænkt på helt, det er, at når så angribere er gået "all in," så vil der bare blive en kæmpe (skadefrydelig/hævngærrig) frist til virkeligt at få ram på disse i meget af resten af fællesskabet, og så vil de derfor faktisk med stor sikkerhed vælge at committe til en soft fork væk fra angrebet, \emph{inden} at angrebet lykkes. Så hvor jeg altså før, nu her, har tænkt, at man i princippet ikke kunne vide, om fællesskabet vil lave en soft fork undervejs.. Hov, men det tænkte jeg jo netop, at man kunne.. Hm ja, pointen var jo netop: Tag den med ro, for man kan jo altid nå at lave soft forken imens første angreb er under opsejling. Og så er spørgsmålet jo, falder.. nyhedsværdien.. i idéerne så lidt til jorden, eller er det stadig værd, fordi, som jeg har tænkt før, at man netop kan afværge et angreb, når man altså lige netop kender til soft fork-løsningen..?.. Nå, men jeg lader lige de tanker hænge lidt, for på gåturen, vistnok inden jeg rigtigt kom frem til, at mit "all in"-angreb ikke vil virke i praksis, kom jeg også på nogen tanker (..men det kan jeg faktisk ikke huske helt.. jeg mener dog, at jeg fik idéen om de nye angreb, inden jeg nåede helt frem til, at det forrige ikke holdt), som så førte til en idé om et nyt angreb.. Idéen er et 51\% angreb, hvor man gør det meste af mining-arbejdet i skjul (i modsætning til alle mine andre angrebsidéer), og hvor man så bare tjener pengene ved at offentliggøre et bevis på, at man har en lang kæde (der overgår den offentlige fork), uden at man afslører den (og det har jeg ikke tænkt så meget på endnu, hvordan man helt præcist gør), og så lader folk købe sig ind på angrebet ud fra en divide-and-conquer-strategi, og hvor man altså så tjener pengene efterfølgende.. ja altså udover de ny-minede mønster på angrebs-forken, som man jo her kan give til sig selv (hvilket \emph{i teorien} faktisk nærmest kan betale for angrebet selv..).. men ja, hvor man altså så tjener de ekstra penge ved at lokke og afpresse folk til at joine angrebet imod et gebyr. ..Hm, nå ja, og det skal siges, at måden man så udfører sidste del af angrebet på, er ved så at mine videre i skjul, hvor man tilføjer alle de transaktioner, som man ønsker, hvilket altså kan være.. tja, i praksis vil man jo nok bare, for at det ikke skal trække totalt i langdrag, nøjes med at føje alle de transaktioner til, der kommer de indledende angribere samt de rekrutterede angribere til gode.. ..Hm, og hvordan kunne man bevise, at man har en lang kæde..? Man kunne vel muligvis enten bruge en tredjepart, eller man kunne måske også bruge et obfuskeret program --- som også leveres til angriberne ret hurtigt, og hvor de så skal sende svar tilbage ret hurtigt --- som tjekker længden på en kæde *(og at den har det opgivne begyndelsespunkt) og sender en længden tilbage i et kodeformat, hvor angriberne altså ikke kan nå.. hm, hvor de ikke kan nå at reverse programmet, og hvor koden i øvrigt er for kort til at indeholde et helt hash, så man på den måde ikke kan opsnappe dette.. Nå, men imens jeg skrev dette, kom jeg så til at tænke på, hvorfor man overhovedet skulle handle med angriberne på denne måde.. Nå jo, hvis de også betaler for ulejligheden.. Hm.. ...Hm, man kunne også udgive kun dele af alle hashes, og så lave ne procedure, hvor offentligheden kan stemme om, hvor de vil efterspørge de fulde hashes (for to på hinanden følgende blokke).. Og ja, desuden kan angriberne også putte samtidigt data ind og afsløre en god del af starten af kæden, således at der her er fuldt bevis for, at angriberne har gjort dette arbejde for noget tid siden (hvorved de jo må have haft en grund til det).. ..Hm, men den protokol, hvor offentligheden kan indsende stemmer.. ah, eller endnu bedre at man bruger tilfældige tal fra den oprindelige fork, for nye blokke der tilføjes den.. den strategi er faktisk rigtig god, for hvis f.eks. bare man offetliggør fem tilfældeige blok-par, og de alle holder, så vil dette jo være afsindigt usandsynligt, hvis ikke langt det meste af kæden er gjort korrekt (og man kunne sagtens offentliggøre mange flere end fem.. hvis de bare ikke lige ligger i enden af kæden.. Ah, men angriberne kan bare sørge for, at deres kæde på det tidspunkt allerede slår den oprindelige fork med rigelig margin), og hvis man allerede har lavet alt det arbejde, så kunne man jo ligeså godt (eller rettere langt bedre, altså) sørge for at gøre den hele korrekt. Så ja, obfuskerede programmer behøves altså slet ikke. ..Ok, så det kan være, hvis dette angreb holder, at det så bare bliver det, jeg primært vil skrive om.. ..Lad mig lige tænke over det forrige angreb noget mere, og ja, så kan jeg også lige summe over dette nye også.. 

%(20.02.22) Nå, der er sket en del nye ting. I forgårs, efter jeg slap sidst her, kom jeg faktisk lidt frem til, at ingen af disse angrebsidéer rigtigt vil have den store nyhedsværdi i sig. For man kan også sige det samme om det mere regulære 51\%-angreb, som jeg skrev om da, nemlig at enten skal angriberne være gode til at shorte kæden inden, eller også skal de forvente, at kursen igen vil stige pga., at en forsvarsløsning bliver indført, men når først man kender til sådanne, så kan ofrene (plus sympatiserer og stake-holdere i kædens integritet) bare indføre disse med det samme for at afværge angrebet. Og jo, måske \emph{kan} man godt shorte en kæde så meget i princippet, og i øvrigt kan man også altid bette imod den (og altså nærmest indirekte shorte den, kan man sige), f.eks. ved at investere i konkurrerende blockchains. Og ja, jeg læste lige et afsnit om Ethereum i går, og jeg kan jo se, at her \emph{er} man klar over, at PoW har problemer, og at man bør gå over til noget andet såsom PoS. Samtidigt kunne jeg i øvrigt også læse, at de problemer, jeg har set ved PoS, er kendte og hedder "long-range attacks." Og løsningen på sådanne virker umiddelbart også rimeligt god, så vidt jeg kan se, nemlig at man altså bare har tænkt sig at give knuderne/noderne en hukommelse, så de lige netop vælger "angrebs-forks," som jeg kalder dem, fra. ..Og ja, er dette ikke umiddelbart en fin løsning, som tilsvarer ret meget, hvad jeg muligvis tror, man kan opnå med mine PoPH (eller hvad man skulle kalde det) -tanker?.. ..Tjo, og om ikke andet, så er man da på sporet af stort set det samme.. Nå, jeg fik også i går nogle tanker omkring, at for Bitcoin i fremtiden, når mining-rewarden går imod 0, og hvis så altså al mining betales med fees, jamen så vil det blive barnemad at lave angreb lidt a la mine "public side chain"-angreb, fordi man så altså kan lave side chains, hvor minerne kan betales ligeså meget som på hovedkæden, men hvis man så pludeslig begynder at skrue mining-lønnen lidt i vejret, så.. Hm, så ville hovedkæden skulle gøre det samme.. ..Tja, whatever. Det er også lige meget. For ja, det korte af det lange er faktisk, at jeg opgiver blockchain idéerne helt. Kun \emph{hvis} ingen af mine andre ting bryder igennem (hvad der ærligt talt ville være lidt mærkeligt), så kunne jeg måske lave en lille (blog-agtig) artikel, hvor jeg.. Tja, hvor jeg skriver argumenter imod PoW?.. ..Man kan jo godt sagtens konstruere gode argumenter imod, at det kan blive en ægte fremtidsvaluta, altså uden skelsættende soft/hard forks.. Forresten virker det også til, at folk på internettet bekymrer sig meget om private nøgler osv., der kan brydes med Shors algoritme, men i virkeligheden er der en anden stor fare, som mange vist overser, og det er at den første kommende gode nok kvantecomputer bare kan bruge Grovers algoritme til at omskrive hele kædens historik og mine alle mønter til sig selv (og/eller mine nogle og så lave divide-and-conquer-angreb oven i for at tjene endnu flere).. Tja, det tænker jeg i hvert fald lige umiddelbart, må blive en fare. Nå, men det korte af det lange er altså stadigvæk, at jeg lægger det på hylden, og muligvis endda for good denne gang. Det har været meget sjovt (..til tider..) at overveje blockchain, og det har da ført mig til min "lykke-/gavn-valuta-idé," som i sin seneste version især da måske godt kunne virke (men dog nok kun, som jeg ser det, hvis der ikke sker fremskridt med mere simple ting såsom mine kundedrevne virksomheder, "civil-/forbrugerforeninger" eller noget i den stil). Men udover dette så har alle andre idéer lidt krakeleret, inden de rigtigt kom til at virke, og nu tror jeg altså, det var det. Det er ikke sikkert, at jeg overhovedet vil tage dette emne op igen; jeg tror i hvert fald, jeg er færdig for nu samt for den nære fremtid.
%Så ja, det blev til mange tanker og meget tekst i dette dokument omkring blockchain/kryptovaluta (og nogle lidt ældre dokumenter, hvis ikke den tekst faktisk også kan findes nedenfor, fordi jeg ikke har gidet at slette det..), men det førte ikke til så meget udover min lykke-valuta-idé, og selvfølgelig erfaringerne, hvilket jo heldigvis altid er en del værd. 
%I går kom jeg så heldigvis frem til nogle nye planer fremadrettet, som jeg faktisk er blevet rigtig glad for. Planen er, at jeg faktisk starter med at skrive min QED-artikel, og endda lægger den ud på arxiv osv. --- det hængeparti der har hægt over mig snart en evighed. Og det gode er, at når jeg alligevel lader selvadjungeret-problemet stå åbent, så kan jeg jo også godt skippe lidt hen over nogle af de andre detaljer. Så det vil jeg faktisk nok gøre. Særligt vil jeg måske altså skippe over detaljer omkring argumentere for diskretiseringen (og hvorfor man må kunne lave et cut-off osv.) og også hvorfor at teorien må fortsætte med at være Lorentz-invariant, selvom man konjugerer og omfortolker positronløsningerne. Dermed tror jeg muligvis dispositionen bare kan blive noget i retning af, først at opskrive teorien i form af Hamilton-operatoren, omskrive det til et felt-/path-integrale, opskrive det tilsvarende path-integrale for gauge-elimineringen, hvor de specielle løsninger for V og A_\parallel antages og vise baglæns (men forlæns ift., hvordan jeg selv fandt frem til det, og ift., hvordan man altså nok typisk ville motivere det), at de to er lig hinanden. Derefter kan jeg så gå videre til at vise, at løsningerne selv er Lorentz-invariante og konkludere, at teorien derfor må være L-invariant (givet antagelserne). Inden jeg laver antagelserne, kan jeg så snakke om, hvad de betyder og sådan, og at der kunne være andre fysiske opdagelser (f.eks.\ i sammenspil med andre teorier (muligvis ved høj energi), eller hvis man opdager en mindste afstand i universet (så det altså er diskret til at starte med), eller bare hvis man kan få en meningsfuld teori ud af en ikke-selvadjungeret operator, hvilket også kan være muligt), der kunne få det til at blive well-behaved, men at det jo ville være totalt awsome, hvis man kunne bevise, at operatoren er selvadjunget og/eller at teorien i det hele taget altså er self-contained. Hm, eller dette kunne også evt. komme i en diskussion efter, hvor jeg i stedet så bare lige giver en kort opsummering eller noget. Nå, men tanken er så ellers, at jeg vil gå videre til at snakke om, hvorfor teorien ellers indeholder alt, hvad man kunne øsnke sig, nemlig at alle de basale forventninger til en QED-teori er opfyldt, og at den også forudsiger spin-orbit-kobling, spin-spin-kobling, Lamb shift (såvidt jeg ved), og også selvfølgelig gyrromagnetisk ratio. Og så er der altså bare lige den dejlige forskel, at den ikke længere behøver fotonerne til Coulomb-potentialet, hvilket altså godt nok modsiger noget gængs viden, men hvilket jo bare gør visse andre gængse (og mere grundlæggende) felt-teori-antagelser så meget mere troværdige. Og så har jeg tænkt mig (og dette er btw en del af den disposition, jeg tidligere har skrevet et sted her i dokumentet, at jeg havde i tankerne, men at jeg ikke har (og stadig ikke har før nu) skrevet ned) lige at skrive, at det måske kunne være interessant med dobbeltspalteeksperimenter eller lignende til at undersøge, hvor meget dekohærens en elektron har med sit foton-felt som funktion af vinkel og hastighed, når den skal inteferere med sig selv med en anden vinkel (og måske også anden fart, hvis man på et tidspunkt vil prøve at undersøge den parameter også). Og så vil jeg nok slutte af med (medmindre jeg rykker anden diskussion ned under dette) at påpege, at der måske, hvem ved, kunne være ny fysik at opdage, ved at prøve at give andre gauge-teorier lignende behandling; måske man også herved vil finde faste (frastødende/tiltrækkende) felter, som altså ikke er båret på ryggen af bosonernes interaktioner. Her kunne vi jo eksempelvis tænke på den stærke kernekraft som et oplagt eksempel at undersøge (\emph{hvis} dette da \emph{er} en gauge-teori, hvad jeg dog tror, men altså hellere lige må undersøge først;)). :)
%Så det er altså planen nu her, og så vil jeg også skrive en tekst, jeg kan ligge ud på min github, som siger, at jeg har nogle web (2.1 bl.a.) -tanker, og at jeg også har en idé til en forretningsmodel, som kunne være brugbar især for en web 2.1-side (dog uden at komme nærmere ind på, hvad det er). Så vil jeg gå direkte videre til at brygge på en kort artikel om min kd.v.-idé, imens jeg prøver at udbrede mig om min fysik-idé. Og planen er så, at \emph{hvis nu} min fysik-idé er rigtig lang tid om at slå igennem, så kan det være, at jeg vil vælge ikke at gå stille med dørene med den kd.v.-artikel, og i stedet forsøge at motivere den med nogle ret høje ord omkring, at det kan blive en løsning på, hvad man førhen har prøvet at løse med diverse foreslåede ideologier (som så ikke er ført til noget --- i hvert fald ikke det, man håbede på, det kan roligt siges..). Så ja, planen er altså, hvis der går stor stilstand i fysikken, så at udgive noget ret bombastisk omkring kd.v.'erne (i modsætning til at være mere stille og rolig og ydmyg med det (hvilket jo nok ville være mere ærligt, for jeg \emph{kan} jo immervæk bare ikke vide --- har jeg jo lært mange gange --- om jeg har overset et eller andet vigtigt (eller om der bare for sen sags skyld vil være mange små ting, der gør, at det ikke vil holde i længden..))). Så ja, det er altså planen, og jeg er umiddelbart ret glad for den. :)
%Og noget andet er, at dette så faktisk også, så vidt jeg kan se, nok kommer til at markere slutningen på dette dokument. Jeg ved godt at det er "continuous note collection" (og jeg har sjovt nok aldrig i al den tid rigtigt fået tænkt på et bedre navn til konceptet..^^), men hvis jeg skal tex'e det, så er det alligevel bedst, jeg starter i et nyt dokument igen, når jeg engang skal udvilke nye idéer (eller supplere til de gamle). Så derfor slutter jeg nok dette dokument nu. Og jeg har nemlig ingen hængepartier rigtigt tilbage. Der er godt nok lige nogen ting i denne "Draft---"-sektion ude i kommentarerne (såsom denne tekst er), der måske egentligt næsten har fortjent at blive opsummeret og skrevet ind som renderet tekst, men nej; det er slet ikke vigtigt nok. Så nu tror jeg altså bare, jeg vil indsætte slutdatoerne, og så måske lige skrive en lille afslutningsvis opdatering et sted i den renderede tekst. Og ellers vil jeg så altså slutte dokumentet af.:) ..Sikke en omgang, det blev til.:) 
%..Nå ja, og jeg kan også lige nævne helt afslutningsvist, at jeg nok starter med at fortsætte skriveriet i mit 'qed.tex'-dokument, og efterfølgende vil jeg jo selvfølgelig lave et frisk dokument til min QED-artikel, og jeg vil nok også bare lave nogle mindre små (friske) dokumenter til hver af de andre ting, jeg vil arbejde på her i den nære fremtid (og så må jeg se, hvornår jeg laver en ny "continuous note collection" (eller "tankearkiv," som jeg har kaldt det tidligere) igen:)). 












%(27.12.21) Forfra (brainstorm-disposition):
%I might have discovered a new attack vector agianst PoW blockchains. It is essentially a kind of 51 \% attack, but where miners are paid and convinced to join the attack on-chain via smart contracts.. (In full view of the public.. instead of being coerced in secret.. or..)
%The attack is initiated by creating side chain and persuading users to join it. The side chain is a previoulsy abandoned fork from the smae blockchain..
%If enough people join, the contract initiates a second phase where nobody can withdraw their money again unless enough people signs on to aborting the attack.. or unless a goal of mined blocks are reached.. 
%Miners of this side chain will then be payed by the pool of money for each block they have mined at the end of the side chain.. So regardles of how the side chain ends, there is a phase where the longest chain is decided and where the miners are then paid according to this..
%If the side chain overtakes the original fork of the blockchain, it will become the new main fork of the blockchain and the attack will have succeeded.. 
%The attackers, however, first of all need to make sure, that the miners are still payed on the alternative fork. This can be done by including the same contracts that initiated the side chain at the beginning of the side chain itself. This way the miners will be able to cash the same rewards regardless of whether the attack succeeds or not. And if it does, they will also get an additional reward in the form of the mined coin on the side chain.. This fact can perhaps be used by the attackers as an end reward to make sure, the miners see the attack through to the end..
%The attackers can add any additional restrictions to the side chain as they want which means they can in principle control what transactions will appear on it and what will not.. They can for instance require that no transaction is included that moves money away from any of the attackers' wallets.. By doing so, they might be able to attract any party who has sold coin since the fork point to join the attack along them..
%What what I have gathered, the amount of money transferred compared to the cost of mining is typically in an order of hundreds or a thousand times as much.. This makes the attack dangerous.. Let us say that enough people join so the total amount of coin sold by the attackers is, say, 50 times the mining costs since the fork point. Let us also say they each offer to lock 20 \% of that money on the attack (which they may or may not have to then rebuy first). If the side chain continuously lets the attackers vote to change the rewards (beside the standard mining reward), they can start by setting the reward high to gain speed of the attacking fork compared to the main one and then potentially adjust it downwards in order to make the funds last the whole attack. This will only needed if there is a limit on how much minig capacity can be bought at a time. Otherwise (if there is an unlimited pool of computing power to buy from), the attackers can just keep the rewards very high so that the attacking fork overtakes the original one as quickly as possible. (Because then we can assume that the difficulty pr. block will rise accordingly.) When they attack is underway, the attackers can see if they will make it or not and if it does not look like the attack will succeed, they can abort and still get most of their money back. This is why having attackers offering to lock something like 20 \% may not be unrealistic. Because it chould be clear quite early whether the attack will suceed or not, and each attackers should be able to trust that if the attack does not seem likely to succeed, a majority of the attackers will probably vote to abort it. 
%Attackers can not only require that no money leaves their own wallets, but can require that no money is transfered to any others (unless money is transfered to multiple wallets in the same transaction and one of those wallets is an attacker). This means that not only people who has sold coin since the fork point but also people who have bought coin can be tempted/forced to join, namely to ensure that they \emph{keep} their money. 
%And in a particular nasty version of the attack, the atackers might even device a protocol to divide and conquer the users by making it so that wallet cannot join the attack if that wallet has transactions going to or from a wallet of an existing attacker. This might potentially cause a panic where users try to join with there wallets as quickly as possible in order not to be the ones to loose all the money they have recieved from transactions in the period. But since joining also means canceling all outgoing transactions this means more and more people are negatively affected by the attack and more and more people will then try to join in time. 
%
%The defense strategy:
		%My defense soultion is for the community to agree to soft fork away from any such attacking fork. This is easier said than done but I am convinced it can work.. 
		%The soft fork just needs to be planned ahead of time to avoid a split community if an attack actually takes place.
		%The soft fork I have in mind is not one that can be implemented by a software update alone. It also requires a spoken agreement in the community at large not to buy coins from a fork which has overtaken another fork as part of an attack of the type I have just described, and to only buy coins on the previous (innocent) fork..
		%A software update would still be for the best so that all nodes in the mining network have a chance to tell their program never to adobt the attacking fork, even before the attack is successful. If a majority of the mining network does this, the original fork will soon enough overtake the attacking one again and all can be forgotten about the attack (except for the participants of the side chain that will either loose or gain coin from it). (This is why I see it as a soft fork.)
		%And if the community, espacially inculing potential future buyers of coins, can convince the mining network, that they will only buy mined coins on the original fork rather than on the (temporarily longest) attacking fork, the miners will benefit from banning the attacking fork in advance. 
%My defense soultion is for the broader community to make an agreement not to treat an attacking fork as invalid, even if it becomes the longest fork. This means commiting to not buying coin on the attacking fork and to buy them on the original fork instead.. 
%This solution is easier said than done since we are dealing with a decentralized community which needs to remain that way for the technology to be interesting.. But I am convinced that it is possible and I hope I can convince the reader as well. 
%First of all, let us note that the solution does not require all of the community to agree at once, which would otherwise make it impossible. It only requires a majority of users, and buyers in particular, to use the original fork, and only for the duration it takes for that fork to then overtake the attacking fork again. Because if the original overtakes the attacking fork again eventually, even users who did not agree to the defense solution will have to switch back unless they want to keep using a shorter fork of the blockchain for no reason. 
%So if the defense works, we can see at as essentially a soft fork of the block chain. The only difference from a regular soft fork is that this soft fork cannot be implemented by a software update alone, as there is probably no easy way to make the nodes able to recognize such attacks on their own. Instead it would probably be best to roll out a software update where miners as well as regular users can tell their program to ban a certain fork. In particular they can then ban an attacking fork so that their node will not switch to this in case it overtakes the original fork.
%
%Why the defense strategy will work if the community prepare in time:
%I am sure that this solution will work but only if the broader community voices a commitment to the strategy ahead of time. Let us first explore what will happen if the community manages to do this, and then I will get back to the danger should they not afterwards.
%Let us assume that the broader community agrees to the soft fork and then look at what would happen if an attack occurs and a majority of buyers are not convinced to implement the soft fork and switch back to the original fork. If no other defense solution is known, this would immedately show the public that the defense does not work and that there is a major attack vector on the PoW blockchain. This would scare of most future bayers of the coins and the value of the coins would tank immensely. The very same buyers whole failed to choose the original fork over the attacking fork will then lose a big portion of the money spent. This is why new buyers in the time when the attacking fork has overtaken the original one in particular will be interested in the defense working. For else the will lose the money the just spent buying coins. But the buyers are exactly the ones who have the power to choose which fork will win in the end since miners will only be interested in mining coins on the fork where there are buyers. 
%Note that this argument relies on the fact that the coin value is already as high as it is due to the community not being scared of the attack vector despite knowing of it. This is what means that the value would tank should the buyers somehow work against there own interrest and dominantly buy coin on an attacking fork. But if the attack vector was unknown just before an actual attack, or if the broader cummunity simply has ignored it up until that point, then we cannot necessarily conclude that the value will tank afterwards. In that case, users might argue that the current attack, should it suceed in the end, will just be the wake-up call the community needed to implement the defense strategy from there on. And given that the attack might have been the kind which divedes and conquers the userbase, a big part of the community might have been swept up in the panic and felt forced to spend money to be safe. This would potentially result in a community that is split on what to do. And since both sides in this case would be able to argue that a future defense should for the fork they support, both forks have the potential to go on --- and perhaps at the same time for that matter, resulting in a permenent fork of the blockchain. 


%Final thoughts:
%So I believe that it is only a matter of time before the defense solution will be implemented. Furthermore I believe the defense can be easily set in place before the first attack is potentially carried out. Even though a software update might be required, it just needs to be in place before the attack succeeds, not before it is initiated. So as long as the community have already voices an overall agreement that the soft fork should be implemented in case of an attack, we should be good. And if the value of the coin does not tank immediately after the overall agreement is made, which it should not, we already have the situation where buyers during the endgame of an attack would lose money if they chose to support the attacking chain. So once the agreement is reached in the broader community and the value of the coin is (or becomes) reasonably stable, any attack of this new type should fail after that point.
%(Hm, måske behøver jeg ikke den sidste paragraf her (om at RW-kontrakter også skal bruge soft-forken), for det giver nok lidt sig selv i virkeligheden:)
%In order not to make this defense strategy less secure, ...

%Cool..:)





%(30.12.21) Hov, jeg må da også hellere lige overveje, og skrive om!, hvad angrebet kan betyde ift. at koble real-world-kontrakter osv. til blockchains..!.. ..Ah, men det gør heldigvis ikke så meget, netop idet der ikke er så mange af sådanne kontrakter i spil pt. ..Nå jo, men hvad med i fremtiden, var pointen jo.? Kan man lave kontrakter, der.. ja.. Ja, så en del af løsningen er også lige, at man skal sørge for i fremtiden ikke at gøre strategien mere ustabil ved at ballancere kontrakter ovenpå den originale fortolkning, men skal i stedet sørge for at disse også indeholder soft-forken i deres definition af kæden.
%Hm, noget andet er, at jeg jo lidt antager, at der ikke findes andre løsninger for at min løsning virker effektivt (gør jeg ikke..?), men kan der ikke godt være det?.. Kan det ikke lade sig gøre at implementere en soft fork med en opdatering, hvor programmet selv kan opsnappe alle side chains og automatisk banne dem?.. Hm, men dette vil så heller ikke være en regulær soft fork, fordi den så vil afhænge af historiske fakta og ikke bare af dataen. Ja, jeg kan ikke rigtigt forestille mig nogen andre løsninger.. 





















%The first idea I want to share is actually an idea for an attack against Proof of Work (PoW) blockchains. I also have an idea for how to efficiently defend against this attack, which is why I see it as the best course of action to just share these ideas as widely as possible. If I had only found the attack vector but not the defense, I would of course have to take much greater care in how I shared the knowledge. A lot of people have a lot of stake in PoW blockchains and a new attack without a known defense would be devastating to a lot of people. I could in theory also just stay silent about it but then it would only be a matter of time before someone else discovers the same attack vector. And if that discovery is either done by or if the knowledge is sold on to a party who has the resources to carry it out, then we would potentially get the devastating scenario. And since the defense works better the longer time the community have to prepare and the more people are in on the preparation, I find that the best course of action is to share it all as widely as possibly.


%... Jeg fik lige gået en god eftermiddagstur (også selvom jeg stod så sent op), hvor jeg kom til at tænke: Idéen er da egentligt næsten for simpel til ikke at være opfundet før. Og senere kom jeg så på turen til at tænke over, at angriberne endda må kunne udnytte lightning networks vildt meget ved at lave et slags prisoner's dilemma for hver LN-kanal. Dette gør virkeligt angrebet effektivt og ja, ret modbydeligt.. Så ja, nu er jeg ikke rigtigt i tvivl overhovedet, om at idéen er ny.. 


%Ok, lad mig lige se: Angrebet er egentligt ret simpelt. Det handler om at betale for et replay attack. For enkelte partier kræver dette for mange penge, men hvis man ser på samlede antal overførsler ift.\ mining-omkostninger i det samme tidsinterval, så er der alligevel (måske ca. en faktor 1000 for BTC..) mange flere penge i overførslerne. Så hvis angriberne nu finder ud af at gå mange sammen, så kan det måske betale sig. Og i princippet kan rekruteringen af medsammensvorne ske via smart-kontrakter på kæden.. 
%Hm, alle dem på kæden, der har solgt mere kryptovaluta, end de har købt siden pågældende fork-tidspunkt kan så købe en lille smule KV tilbage og deltage i angrebet.. ..Hm, man melder som som angriber ved at oprette en smart-kontrakt på hver af de to forks, som betaler angrebs-minerne løbende (for alle de minede blokke i angrebsforken, som opfylder visse krav).. ..Hm, der er vel egentligt ikke så meget mere i den grundlæggende version, for nu kan man så sige, at angrebs-forken kun bliver bygget...
%..Hm, man kan vel også bare lave nogle multi-sig-kontrakter, og endda kode det i kontrakterne, at de først træder i kraft, når mange nok har oprettet kontrakter (eller rettere når mange penge nok er tilføjet disse kontrakter).. ..Hm, og man kan endda gøre, så at deltagerne altid kan holde en afstemning om at afbryde kontrakterne og angrebet. Og i øvrigt kan angribere godt altid booste et angreb, hvis der er brug for det, ved at oprette en ny samling kontrakter af samme karakter og med samme fork-tidspunkt og det hele, men med et nyt mål-beløb, før de udløses.. ..Ja, det må da næsten være god simpel version af den grundlæggende angrebsidé at præsentere som eksempel.. Og når det så kommer til også at udnytte LN, så handler det vel.. nå ja, jeg skal også huske, at der skal sættes krav til angrebskæden om, at ingen overførsler væk fra de punge, som.. Hm.. Tja, man kan jo bare sige, at der ikke må forekomme nogen transaktioner.. Ah, andet end transaktioner, som indeholder et hash af noget data, der er defineret ud fra angrebet, f.eks.. ah nej, man kan bare bruge et af de nyeste block-hashes fra den normale fork. For på den måde opnår man, at ingen af de gamle transaktioner / kontrakter kan indsættes igen på angrebsforken. Ja. ..Hm, men når vi så når til det angreb, der også udnytter LNs, så vil man jo muligvis gerne kunne indsætte gamle transaktioner, der åbnede/åbner kanaler til et LN.. Skal man så bare her undtage disse fra kravene..? ..Tja, det kan jo lade sig gøre, om end det måske ikke er så effektivt, men man kan jo sige, mine eksempler behøver jo ikke overhovedet at være gode opskrifter på angrebene; de skal bare overbevise folk om, at angrebene er mulige. Ok.. 
%Og ja, pointen i at man endda kan udnytte LNs, er altså at man netop kan opnå mere ved at sætte nogle mere ikke-trivielle krav til angrebs-forken. Her kan man så for det første bekendt gøre, at alle LN-transaktioner godkendes, hvis og kun hvis de kommer fra en af angriberne. Og ja, det kan så være, at man også kan opnå enddu mere, hvis man tillader alle LN-kanal-åbnende transaktioner, særligt også fra den normale fork (og fra halv-gamle blokke i denne). På den måde kan alle angribere høste alle pengene fra de kanaler, der har været oprettet siden angrebs-forkens starttidspunkt, og hvor modparten også havde de penge nok i deres pung til at oprette LN-kanalen ved dette starttidspunkt (for ellers går det jo ikke så let). Så herved kan alle der har haft LN-kanaler åbne (og ikke kun har modtaget alt på disse kanaler på den normale fork) nu blive potentielle angribere.. Og ja, her tidligere i dag fik jeg så altså den idé, at angriberne også kan tage dette endnu videre potentielt set, og faktisk også semi-afpresse andre folk generelt, der har deltaget i LN-kanaler (men som ikke indtil videre deltager i angrebet), ved.. Hm, måske skulle man i den forbindelse lave et koncept om, at angrebpagten ikke må indeholde to angribere, der har haft LN-kanaler åbne med hinanden, for så ville man vel lidt prisoner's-dilemma-afpresse folk til at være den første, der melder sig til angrebet ud af de forbindelser, hvor de har haft LN-forbindelser til og fra..? ..Hm, eller måske endnu mere potent kunne man sige, at folk bare ikke må være med, hvis de har haft LN-forbindelse til en anden angriber, der ved optagelsen af pågældende nye potentielle angriber så kommer under et fastsat antal LN-forbindelser (måske to eller tre *(tja, eller måske endda bare én, for alternativet er jo, at man selv bliver udnyttet..)), de så kan udnytte.. Hm.. Ja.. Ja, og jeg kan nok godt tegne et lidt mere overordnet billede af, hvad denne ekstra sårbarhed går ud på (nemlig at man tvinge en anseelig brøkdel af alle brugere, der har haft LN-kanaler åbne, til at deltage i angrebet, hvis ikke de vil miste alle pengene låst i udgangspunktet til disse kanaler (samt alle de penge, de eventuelt har fået overført på disse..)).. 

%Inden jeg når til mit forsvar, kan jeg vel så sige, at first line of defense i princippet kan ses at være det faktum, at et succesfuldt angreb vil foringe kursen/møntværdien, og herved bliver der så mindre at vinde.. Tja, men det hurtige modargumetet er jo for det første, at angriberne ikke selv i princippet behøver at være stake-holdere i udgangspunktet for angrebet, så de kan potentielt set slippe af med at betale ret lidt for angrebet, og så skal kursen jo falde meget, før det ikke kan betale sig.. Ja, så spørgsmålet er, om jeg overhovedet skal nævne denne del..
%Og ellers er der jo det modangreb, der simpelthen hedder "at overbyde angriberne." Problemet er bare her, for det første at angriberne kan være minere, hvorved disse så kan få meget at vinde, hvis en stor part pludselig bliver tvunget til at overbyde et angreb.. Hm, det burde man nok egentligt regne noget mere på.. Tja.. Og ja, ellers er der også den fare, at angriberne kan lykkes med at shorte KV'en forinden, således at disse kun ender med at tjene flere penge på at kursen tanker.. Tja, så hvis jeg skal nævne disse ting, så skal det måske heller ikke være som en slags first options, men mere bare for at redegøre, hvorfor disse options måske nok ikke holder.. ..Hm, jeg kan jo måske bare lige nævne disse ting i nogle hurtige vendinger, inden jeg går videre til mit holdbare forsvar.. 

%Og mit holdbare forsvar er så, at gå sammen og blive enige om i fælleskabet ikke at samle på mønter fra forks, som har overhalet en anden fork som følge af et sådant angreb som det, jeg lige har beskrevet, hvor angribere altså (tydeligvis) har betalt for, at det bliver mere lukrativt for minere at arbejde på angrebs-forken frem for den normale fork, indtil at angrebs-forken har overhalet. Med andre ord kan fællesskabet altså vedtage en soft fork af PoW-kæden, hvor sådanne forks, som midlertidigt får overholet den normale fork på baggrund af sådanne angreb, ikke er gyldige. Og grunden til, at dette bare vil være en soft fork, er jo, at den "normale kæde" så alt andet end lige så på et tidspunkt må overhale angrebs-forken igen og blive den gældende fork selv efter de gamle kæde-vedtagelser. Bum.. Man kunne så (måske kortvarigt) overveje, om fællesskabet virkeligt vil kunne nå til enighed om en sådan soft fork? Og her må man bare med det samme erkende, at alle mønt-købere må være mere interesseret i at købe KV-mønterne, hvis en sådan soft fork virker som en klar beslutning iblandt størstedelen af fællesskabet --- og ikke mindst iblandt potentielle fremtidige mønt-købere. Og når mønt-købere, nutidige og fremtidige, alle har denne interesse, jamen så vil det også være alle i fællesskabets (måske på nær eventuelle angrebsoppertunister, der håbede på det værste) interesse, at give klart signal om, at soft forken er vedtaget. For alle (eller i hvert fald størstedelen af folk), der er interesserede i PoW-kæden, vil jo også være interesserede i, at den bliver mest mulig lukrativ for mønt-købere.. Tja, og samtidigt kan man også bare sige, i princippet vil de fremtidige mønt-købere jo altid i sidste ende bestemme værdien (på en måde), så hvis man ved at det altid vil være i købernes interesse, at kæden har vedtaget soft-forken, jamen så vil den også automatisk det, for denne vedtagelse er jo netop i sidste ende faktisk op til mønt-køberne.. Ja, jeg kan sikkert sige dette på en mindre indviklet måde, men ja, det må vel nærmest bare være det, jeg skal sige til denne sektion.. 

%(20.12.21) Nå, jeg fandt ud af i går aftes, at lightning-netværket ikke er så stort igen, så dette forøger ikke egentligt truslen særligt meget. Men så kom jeg faktisk lige på i badet her til morgen, at man jo faktisk kan lave det samme beskidte trick, når det kommer til transaktionerne generelt, og altså halv-tvinge folk til at deltage i angrebet ved et slags prisoner's dilemma-agtigt spil. Så dette gør jo måske nok hele truslen meget større ved mit angreb, så derfor skal jeg måske også passe ekstra meget på med, hvordan jeg udgiver det.. Hm, jeg stoler rigtigt meget på mit forsvar, men måske er det nu alligevel klogt lige at gå til nogen først, inden jeg udgiver.. Det tror jeg, jeg vil..



%%The first idea I want to share is actually an idea for an attack against Proof of Work (PoW) blockchains as well as for a defense against that attack. 
%I might have discovered a new attack vector on Proof of Work (PoW) blockchains. Luckily I have also found a defense strategy against this attack that seems to be efficient. 
%%I now of course have to be a bit careful with how I am sharing it. 
%%A lot of people have a lot of stake in PoW blockchains at this point in time and thus a discovery of a new attack could potentially be devastating to many. 
%%Luckily, however, 
%%the attack vector is only severe if the knowledge of it becomes widely spread. Otherwise the initiating attackers would need to control 51 \% of the mining capacity, which makes it just a standard 51 \% attack. And since my defense strategy is most likely easier to implement than the attack, it should be the safest course of action to just share them both at the same time and as widely as possible. 
%%Jeg kan bare vente med denne linje, til jeg rent faktisk udgiver det..:
%%I will still try to find some experts first before I make the discoveries public,\footnote{So by the time that you are reading this, I will have done so.} just in case, but I am sure they will agree with me.
%%And the attack vector is only %..Hm, nej det kan man vel egentligt heller ikke sige..
%...
%%Jeg genskriver nu nok det meste af dette (altså hele denne tekst), for jeg skal nok alligevel "skrive mig varm," imens jeg udformer denne udgivelse, så jeg går nok bare løbende tilbage og skriver ting om. 
%
%\textit{(Jeg genskriver nu nok det meste af dette (altså hele denne tekst), for jeg skal nok alligevel ``skrive mig varm,'' imens jeg udformer denne udgivelse. Så jeg går nok bare løbende tilbage og skriver ting om.)}
%
%
%%The attack vector is a kind of 51 \% attack but where the initiating attackers recruit more attackers on chain
%The attack vector is a kind of 51 \% attack where malicious users bribe miners to mine blocks for an otherwise abandoned fork. If the malicious users have recently sold a lot of coin on the main fork, they might stand to gain coin if the other fork overtakes it as a result of the attack. If they stand to gain more than the cost of bribing the miners, the attack might be beneficial for them. The bribed miners will stand to gain not just the bribes but also the coin they mined on the new fork if the attack is successful.
%
%This is the overall strategy of the attack but so far it not very severe. Let us say that some attackers tried to bride miners into being part of the attack via some closed lines of communication. How well would this attack work? In theory you could say that since the miners still get the mined coin on the new fork, let us call it the \emph{attacking fork}, then any significant amount of money should be enough to make them switch. But this first of all assumes that you can convince the miners that the attack will work in the end. Otherwise the miners will lose all the coins they have mined on the attacking fork. And since any parties who do not want the attack to succeed can just pay an equal amount %(or perhaps even less in practice) 
%of money to hinder the attack, this assumption does not hold.
%
%The attackers would thus need at least an amount of money comparable to the mining rewards (plus fees) unlocked on the main chain since the chosen fork point of the attack to even get it started. We can then ask ourselves: Is this possible at all use this kind of money on an attack while still standing to gain from it? Well, even though the mining costs/rewards are huge for a well-established PoW blockchain, the amount of money transferred on a given interval can be at least about a thousand times more.\footnote{This is at least from what I have gathered looking at current blockchain statistics (for BitCoin in particular).} From what I have gathered ***(mind the repitition, but probably insert a reference as well instead.\,.), however, the blockchain communities are already aware of this potential risk and if a large amount of money have been transferred from the same wallet, users are likely going to alarm each other, that extra time should be given before a transaction is considered cleared. 
%%
%%So can we feel safe that such attacks will not happen; that miners cannot be bribed into switching to a different fork and start working on this? No, I do not think so, because there are still more options to consider for the attackers.
%%
%Does this then mean that we can now safely assume that such attacks will not happen? No, I do not think so, unfortunately, because there are still more options for the attackers that we have yet to consider.
%
%Malicious users might use smart contracts to try an recruit other attackers publicly at the initial stage of the attack. This way they make sure that the attack is only initiated if enough users join in and split the cost of bribing the miners. This means that any user who has sold coin recently can be potential attackers in theory. Such an attack will therefore require a minimal amount of planning beforehand as any user who notices the smart contract and stands to gain from it can in principle just jump in on the opportunity. And if we then look back at the fact that the total amount of transactions is in the order of hundreds times the total mining cost of an time interval, this means that at any given point there might be plenty of potential attackers who can help split the bribes and make the attack affordable. 
%
%The fact that a given attack can have the effect of reducing the value of the given cryptocurrency can often help as a detergent against that attack. But I do not think this is enough of a detergent in this case. If we thus look at a group of users who have recently sold all\footnote{If they have sold all their coin, the can always just buy a little of it back in order to participate in the attack.} or most of their coin on a given PoW blockchain and assume that a significant number of those would be willing to carry out an attack if they stand to gain, then let us say that they succeed in doing so and that the value of the coins fall as a result. Since all those users will have sold the coins that they have now regained in the near past, we can assume that they all have money on hand apart from the regained coins. These users can then in principle, as far as I can see, now just buy more coins and drive the value up again (at least somewhat). 
%%Skal nok erstattes:
%%At this point they can then hold on to that stake and wait for a solid defense against future attack of the same kind to be implemented. At that point the value of the coin should at least in theory be roughly the same as at the beginning, and the attackers might thus be fortunate enough to nearly double their money (assuming that the costs of bribing the miners are small in comparison). I am nowhere near an expert in these matters (quite the opposite, in fact) and I could of course be wrong about this whole analysis, but so far, I cannot see why this would not be a possible scenario. 
%...
%
%%And the worst part is that these users do not even have to be current stakeholders in the blockchain; they only have to be recent stakeholders. A common detergent against such attacks is otherwise
%
%
%\subsubsection{Implementing the attacking fork as a side chain}
%
%An attack of this kind be implemented as a side chain to the main chain. Any wallet who wants to participate in the attack can log on to this side chain and lock some coin to it. If enough users join before a certain deadline, the attack is initiated automatically. Otherwise every user will have their coin unlocked again. During the execution of the attack, the side chain could allow for the participants to vote for aborting the attack, in which case all the unspent coin is retrieved. The side chain of course consists of the attacking fork, i.e.\ a previously abandoned fork to the main chain, and at the end of the attack, regardless of whether the attack is carried all the way through or aborted at some point, each participant will pay an equal amount of coin to all the miners who build the fork. There is, however, one more thing, the attackers need to consider in order for the attack to work, and that is how the same amount of money is also transferred on the attacking fork. Because if the money is not earned on the attacking fork as will, they will stand to lose what the earned if the attack is successful. There is several way in which to ensure that the miners are paid on both forks, and one very simple solution might, as far as I can see, just be to make sure that the same contract that initiates the side chain is added at the beginning of the attacking fork as well. If the attack is successful, all the miners will automatically earn a bonus in the form of the mining rewards for the blocks mined as part of the attack. 
%
%The attackers might also add certain restrictions to the side chain. They might for instance add the restriction, that no transaction which moves coin away from any wallet claimed by an attacker is to be included (or the miners will not get their extra pay for their work). Such a restriction will then last until the attack is completed. They can also in principle just block all transactions if they want to. And in fact the can control all transactions in general. This means that they might also target other side chains or side channels. They can for instance target all lightning networks and if the duration of the attack is long enough, they can in principle distribute the money in all open payment channels how they like. In general, any state channel, side chain or smart contract which is theoretically vulnerable to DoS attacks can in principle be targeted in this attack as well. 
%
%
%
%
%
%
%\subsubsection{The defense solution}
%
%The fact that this kind of an attack only works if a large number of the community is aware of it (which means that almost everyone will be aware of it in practice), as well as the fact that an attack would most likely take a long while to carry out, luckily also makes it so much more easy to defend against. It means that such an attack can be easily spotted and documented when it happens. An all it takes to counter the attack is simply to decide on a soft fork in the community, where all fork that clearly only own its length to such an attack as described above is deemed invalid. In this case, even if the attacking fork overtakes the original one, if a majority of the community then does not want to trade (and to buy in particular) coin on that fork, then at some point the original fork will overtake the one from the attack again. So if an attack seems to be under way (and it will as mentioned be easy to spot and easy to react to), all the community needs to do is for a majority of its members active in trade to promise not to trade the coins of the attacking fork (and in particular to buy them). Then there is nothing to gain for the attackers, and they will probably even just abort the attack in that case. 
%
%%Hm, lad mig egentligt lige tænke lidt mere over situationen.. 




%Skal jeg skrive noget om at shorte KV'en..?..
%Kan godt være, at jeg er for træt nu ((22.12.21)), men jeg kom frem til i går, at jeg bare egentligt burde inkludere det med, at angriberne lidt kan tvinge brugere med tidligt. For i så fald kan man også let opnå et splittet fællesskab, hvor halvdelen ikke gider at resette, for så vil de selv \emph{miste} penge frem for at vinde penge (så de vil miste dobbelt). Og hvis ikke fællesskabet af en eller anden grund var klar med forsvaret i tide, så vil det også være for nemt at argumentere for, at så var "de langsomme selv ude om det i denne omgang." .. 

%(23.12.21) Okay, lad mig lige prøve at foklare faren her forfra som en kommentar-brainstorm-note. Hvis bruger-fællesskabet ikke er klar på faren, så kan angribere oprette en side chain.. eller rettere i første omgang er det bare en state channel, hvor man erklærer angrebet og prøver at lokke flere angribere til, muligvis med semi-tvang. Så angriberne viser altså herved intentionerne og håber på, at folk vil lokkes til. Hvis angrebet så er så groft, at det oven i købet bruger en proces, hvor netværket deles op i to, og hvor alle transaktioner på main-forken så (ved et succesfuldt angreb) enten vil gå igennem eller annulleres alt efter, om modtageren eller afsenderen nåede at blive blandt angriberne, jamen så kan der jo virkeligt gå panik i folk. Og hvis det så bliver kosteligt, for de angribere, der lokkes med på angrebet, jamen så vil man stå tilbage i en situation, hvor fællesskabet vil blive splittet i sidste ende, når man skal beslutte, om man skal soft-forke væk fra angrebs-forken igen. Og dette vil så alt andet end lige betyde, at sådan en soft fork ikke vil blive en realitet. (Og ja, medløber-angriberne vil jo have mistet bestikkelsespengene på main-forken, hvis soft forken gennemføres.) Så man bliver altså nødt til at planlægge soft forken inden da, sådan at der kan blive større enighed i fællesskabet om at soft forke væk fra angrebsforken bag efter (for så kan alle i offentligheden nemlig klart blive enige om, at angriberne, inklusiv medløber-angriberne, også var for dumme at blive narret med på (eller starte) projektet). Men hvis man derimod ikke har sådan en beslutning først, så vil den generelle offentlighed meget vel ligeså godt kunne sige ``that's life'' til dem, der blev ofre for angrebet. Især fordi at en sådan holdning så ikke vil sætte nogen precedens for, at en aftale ikke ville kunne lade sig gøre i fremtiden. Og dermed kan kryptovalutaen altså godt bevare sin værdi i rimelig høj grad (i princippet i hvert fald (hvilket vil sige, at det \emph{kan} ske, for alt kan ske i praksis, når det kommer til sådanne systemer, hvor intet tæller udover folks holdninger til, hvad tæller)). Men hvis der derimod virker til at være en generel "aftale" i fællesskabet om (og altså en generel forståelse af), at man vil soft-forke i tilfælde af sådan et angreb, og man så ender med \emph{ikke} at gøre det alligevel, jamen så vil fra den tid af være precedens for, at sådanne angreb altid har en chance for at virke, hvilket med det smame vil få værdien til at falde helt vildt. Heldigvis betyder dette faktum dog så, at folk der ønsker at investere i KV'en umiddelbart efter et angreb, absolut ikke vil have interesse i at investere i andet end soft-forken, for ellers vil de jo selv, ved at investere i den midlertidigt længere angrebs-fork, være med til at legitimisere omtalte precedens for, at et angreb sagtens \emph{kan} lade sig gøre, hvorved værdien af den KV, de lige har investeret i så med det samme vil falde (og altså direkte som følge heraf). Hm, jeg kom lige i tanke om, at noget sjovt (..og på nogen måder knapt så sjovt..!..) er, at forsvaret så et eller andet sted beror lidt på, at det ikke findes en anden type forsvar mod samme angreb.. For hvis der gør, så kunne man i princippet offentliggøre denne nyhed lige efter et angreb, og så.. så kunne man splitte fællesskabet over, om man så skal soft-forke eller ej.. På den anden side, så vil man i praksis aldrig kunne udføre angrebet, hvis ikke folk så kender til denne staretegi i forvejen, for ellers vil ingen jo hoppe med på angrebet. Så dermed kan man altså heller ikke i praksis gøre dette. Og desuden så vil fællesskabet jo nok i så fald alligevel bare soft-forke alligevel, om ikke andet så fordi det generelt sætter en god precedens for, at sådanne soft-fork-forsvar virker (til hvis nu der skulle opdages andre angrebsvektorer, der kræver et soft fork-forsvar imod sig).. Ok! Så dette er altså den gode forklaring på, hvorfor angrebsvektoren er farlig.  









%\section{title: A donation economy? / NFTs for open source contributions?...}
%eller:
\section{(old)A donation economy around open source contributions?...}
%Se noter fra d. 2/1-22 til d. 3/1-22 nedenfor for min nye (bedre!) version af idéen. (Altså indtil jeg får skrevet det ind på en bedre måde.) (03.01.22) *(Nej, den gik ikke. Læs noter nedenfor fra d. 4/1.)



This idea is also related to blockchain in some way though the idea can also be implemented without blockchain technology in principle. 
\ldots


%(30.12.21) Disp-arbejde *(brainstorm-arbejde i stedet):
%This idea is also -"-..?
%Hm, lad mig prøve at starte mere fra midten:
%..Hov nej, lad mig da lige sørge for at starte med en brainstorm, inden jeg går videre til paragraf-disposition.
%Brain:
%Tja, jeg kunne jo godt tage udgangspunkt i NFT'er, og så sige: Hvorfor ikke lave NFT'er med open source-bidrag?.. Der er jo bare lige det ved det, at jeg rigtigt gerne vil have det som en indbygget (eller ret central i hvert fald..) del af det, at kravet på donationer også skal føjes til.. I princippet kunne man så bare bygge denne idé ovenpå efter, man lad mig nu lige tænke over, om det vil være godt så, at introducere en idé, der ikke indeholder det, jeg gerne vil have, som en selvstændig idé... ..Hm, kunne man mon også i stedet introducere idéerne mere hver for sig, og så foreslå en blanding efter..? ..Hm, eller hvad med bare at fokusere udelukkende på donations-rettigheds-kontrakter, og så bare nævne, at det fungerer en anelse lig NFT'er..? ..Ja, det er måske det bedste.. ..Hm, skal jeg så altså tage lidt udgangspunkt i min nr. 2 titel ovenfor ("A donation economy around open source contributions"..), eller..? 
%(31.12.21) Som jeg kom frem til i går aftes, så er idéen jo bedst, hvis folk også kan samle på tokens'ne efterfølgende. Så ja, vi skal meget gerne bruge NFTs. ..Men ja, så skal jeg nok lige tænke lidt mere over, hvad man så rent faktisk skal gøre for at implementere systemet. Lad mig bare gøre det her i dag. (Jeg kunne også have starten dagen med at arbejde på den egentlige tekst til forrige sektion, men nu tillader jeg lige mig selv at udskyde det arbejde en dag eller to. :)) 
%Jeg kan forstå at normale NFTs kører over BitCoin ellet Etherium.. (Jeg skal egentligt lige sætte mig ind i hvordan..) ...Okay etherium.org/en/nft forklarer det fint nok; jeg kan godt se, hvordan det må kunne lade sig gøre. Det kræver jo bare en public ledger, ja.. ..Okay, det er rigtigt positivt, med hvad der står på den side, for min idé er så stort set bare at sige: Hvad med NFTs, der gør for open source-skabere, hvad andre NFTs gør for kunstnere?:) ..Hm ja, og kan se, andre også har haft tanken om at content creators kan bruge NFTs.. ..Ja, så mon ikke bare min idé bliver specifikt at bruge de NFTs som klare aftaler om, at skaberen sælger sine rettigheder for fremtidige donationer videre med den token, og så at dette kan føre til en donations-økonomi omkring internet-skabere..(?) ..Sikkert..:D Okay, så mon ikke min sektion her kan blive noget i retning af: NFTs gør allerede sådan og sådan, og (hvis ikke det allerede findes, det skal jeg lige prøve at læse om) i fremtiden kan det også blive brugt mere og mere af content creators --- og også open source-programmører! Men nu skal i høre en idé, som tager alt dette op til et helt nyt niveau! (Og så bare skrevet helt tonet ned selvfølgelig.:)) *(Udråbstegnene var mere til mig selv.:)) Hm, det vil jeg lige tænke lidt over.:) 
%Hm, og man skal nok bemærke, at NFTs ikke følger de samme koncepter som cryptovalutaer. Systemet kræver f.eks., at folk holder øje med og husker (/gemmer), hvilke creators/artists har hvilke offentlige nøgler osv. For kendte kunstnere/creators i systemet behøver vi almindeligvis ikke være bekymret for, om offentligheden pludselig skulle glemme, hvad den originale offentlige nøgle var for visse kunstnere. Men hvis systemet også skal kunne optage bidrag fra alle mulige ukendte personer, så kræver det jo lige pludselig en vis infrastruktur, hvis offentligheden skal kunne holde styr på hvem og hvad. Og det skal de jo, hvis fremtidige donationsforeninger skal kunne finde ud af, hvem har krav på donationerne for et bidrag. Hvis en skaber nu bare beholder sins tokens, så kan denne jo bare vente med at sælge dem, men hvis små skabere gerne vil sælge deres.. Hm.. Jeg kom også lige på, at skaberne jo dog i mange tilfælde bare kan lægge et vandmærke ind med deres offentlige nøgle i, hvad de uploader, så ja.. Løser dette egentligt ikke problemet.. ..Ah jo, måske, for selv hvis de afslører deres offentlige nøgle, så betyder dette jo ikke noget. For nøglen i vandmærket kan i princippet gøres forskellig fra gang til gang. Så skaberen skal i så fald så bare lige selv sørge for, ikke at bruge den afslørede nøgle i næste vandmærke, for så kan andre muligvis minte tokenet i forkøbet af dem. Jamen er det så bare det!..? ..Og angående det med at give rettighederne videre, så skal NFTs'ne jo bare netop inkludere et klart udsagn om dette, som så underskrives klart af skaberen. Nå ja, og vandmærket, eller hvad vi skal kalde det, bør også indeholde eller referere til en klar underskrift om, at donationsrettighederne til pågældende indhold skal ses som blivende solgt videre, når tokenet skifter hænder. ..Hm, og kunne man også egentligt gøre, som nævnte side lægger op til, og gøre det frit for skaberne at inkludere en forskrift, der giver dem royalties for salgene? Tjo, hvis NFT'erne alligevel dannes på en blockchain, der kan understøtte dette, hvorfor så ikke. Og så kan alt dette jo bare indgå i kontrakten/vandmærket/lincensen. Ah, og det kan jo i det hele taget også bare indgå i vandmærke-licensen/-kontrakten, hvilken blockchain, der må bruges til at udstede og handle med tokens'ne på, og, hm, mon ikke man i øvrigt også kan inkludere her, hvad man skal gøre i tilfælde af, at denne blockchain skulle bryde sammen..? ..Jo, for dette behøver jo ikke at være formuleret som en eksakt protokol, men kan bare være sige, at alle handler der så kan ses at være sket, imens blockchainen havde integritet, de skal så bare kopieres over på en ny ledger, som den sidste indehaver så selv kan vælge. Det bliver herved så sidste indehavers ansvar i princippet selv at sørge for at melde sig på en ny blockchain (eller anden ledger) og udstede en tilsvarende token her, og fremtidige donationsforeninger skal så respektere dette som den herfra gældende token. Og hvis der så bliver tvivl om, hvem der er den rigtige indehaver, så bliver dette et valg i sidste ende, som donationsforeningerne skal tage. Og det er så derfor på eget ansvar, hvis man vælger at handle tokens på en blockchain, som har en stor risiko for at bryde sammen omkring tidspunktet for handlerne. ..Cool..! I øvrigt føler jeg mig altså bare mere og mere sikker på, at min donationskæde-idé vil virke. Især når tokens'ne aldrig købes tilbage, for hvis så de første donationsforeninger fucker op på en eller anden måde, eller hvis de er lang tid om at komme i gang, jamen så gør det ikke rigtigt så meget egentligt, for fremtiden vil altid (7, 9, 13, he) være der til at rode bod på sagerne og gøre, hvad retfærdigt er. Og selv hvis man får oparbejdet en pukkel af manglende donationer, jamen så kan de fremtidige donationsforeninger altid bare starte fra en ende af. Det eneste der kan gå galt er nærmest bare, at donationsforeningerne ikke kommer hurtigt nok (eller godt nok) op at stå, og at potentielle bidragsydere så ikke føler sig sikre nok til at gøre arbejdet, hvilket så bare vil koste verden for nogle ellers mulige bidrag, men som ikke som sådan vil koste token-stake-holderne noget.. :) 

%Okay, så hvis jeg så lige skulle prøve at opridse en mulig disposition..:
%NFTs kan allerede sådan og sådan. De kan hjælpe digitale kunstnere med at finansiere deres arbejde. Og der bliver også spået om, at content creators kan gøre det samme i høj grad i fremtiden. 
%NFT'er fungerer sådan og sådan sagt helt kort, og man kan læse om dem her..
%Man kunne også forestille sig det samme for open source-programmører i øvrigt.
%Det ville være dejligt, hvis skabere og open source-programmører i fremtiden i høj grad ville kunne finansiere deres arbejde via tokens. Jeg tror jeg har en idé, der kan gøre dette muligt..
		%Min idé går ud på, at skabere skal underskrive et klart udsagn om...
		%Min idé bygger på et økonomisk system, som jeg tror kan blive kæmpe stort i fremtiden, især når det kommer til digitale "bidrag" (eller hvad vi skal kalde det).. 
%Min idé går ud på, at skabere skal underskrive et klart udsagn som en del af udstedelsen, at deres fremtidige donationer for at belønne deres arbejde i stedet skal gå til holderen af pågældende token.. Der kan også være andre udgaver af dette; f.eks. kan man vælge at udstede tokens, der gør krav på en bestemt procentdel af donationerne (eller hvor der er royalties ved at handle med dem, men det kan jeg selv lige tænke lidt videre over --- om måske bare nævne senere i sektionen..).
%Jeg tror nemlig på, at et økonomisk system (omkring digitale bidrag især (eller rettere: det kan hurtigere og nemmere blive stort for sådanne, men jeg kan bare skrive 'især,' og så lidt undlade at skrive om andre områder)), hvor folk donerer bagud til skabere for deres bidrag, kan blive kæmpe stort i fremtiden..
%Lad mig forklare hvordan og hvorfor sådan et system kan fungere.. (Det har er bare et oprids; ikke en paragraf-disposition..)
%Brain: Hm, jeg vil gerne lave en paragraf nu, hvor jeg fortæller om, hvad fordelene ved systemet kan være.. tja, og måske skal jeg også lige forklare grundigere, hvordan systemet fungerer først. Og så vil jeg nemlig gerne fortsætte med så at forklare, hvorfor det er holdbart, og derefter forklare, hvorfor det kan og vil fremkomme (fordi det vil være nogen (samrte) investorer, der kan se pointen og potentialet i det hele..). (Og jo flere investorer, desto mere interesse, og jo mere interesse, desto flere, der kan se poitnen og potentialet..) (Og også desto større sandsynlighed så for at opnå en situation, hvor man kan trække donationerne fra i skat..) Men ok, så lad mig altså lige brainstorme over, hvad jeg skal skrive til de to første nævnte punkter, nemlig om hvordan det virker, og hvor godt kan være.
%
%... Ah, jeg har mulighvis fundet på en vigtig ny tilføjelse til idéen! Det er forresten blevet d. (01.01.22), og det var så på en eftermiddagsgåtur nu her. Jeg tror muligvis man skal indføre en faktor i systemet, som siger, at bidrags værdier også afhænger af, hvor gode den mængde af folk, bidraget kommer til gavn, har været til at give igen for tidligere bidrag, der er kommet dem til gavn, og som det har haft råd til at give gengældelse for. ..:D.. Denne faktor vil så naturligvis starte på en, fordi folk jo ved systemets opstart ikke skylder hinanden noget. Men hvis nu en gruppe af befolkningen fejler og ikke donerer særligt godt igen, ift. hvor meget gavn de har fået af systemet, jamen så bør faktoren falde til en vis grad. Der kan så godt være et minimumspunkt for faktoren forskelligt fra nul, men det må fremtiden beslutte, hvad denne så skal være. Hm, inden jeg vender tilbage til, hvad et minimumspunkt forskelligt fra nul kunne gøre godt for, så lad mig lige tænke over, hvordan man bedømmer, hvad faktoren skal reduceres med.. For det vil jo alt andet end lige også blive udefineret, hvornår præcis gælden skulle have været betalt tilbage. Fremtiden kan altid bare beslutte noget, men så bliver det jo svært for investorer at vide, hvad prisen bliver. Hm.. ..Hm, kunne man mon gøre noget med at sige, at det bare er relativt til andre grupper i samtiden hele tiden..!..? ..Ah, det var muligvis en virkeligt vigtig idé også, jeg kom på der, og jeg fik den ret hurtigt, efter at jeg skrev "Hm..". :D^^ Det føles i hvert fald, som om det bliver noget a la det, der bliver svaret!:) ..Hm, men hvad gør man så, hvis alle bare udskyder deres donationer?.. ..Hm, kan man mon bare sige noget i retning af, at hvis folk gør dette, og at dette så kommer til at give problemer, så har fremtiden ret til at straffe dem for det ved at give dem en lavere faktor..? Dette vil så straffe dem indirekte, men det er vel også fint, for folk bør vel kunne se, hvis sådan dårlig opførsel er på randen af at skabe problemer.. Hm.. ..Man kunne også overveje at få noget afstemning ind i billede, så lad mig lige summe over den tanke også.. ..Okay, jeg kom også lige på, at man kunne gøre noget smart, hvor folk omkring gruppen kan stemme om en deadline, hvis en gruppe menes allerede at have haft tid til at betale tilbage (/donere) og ikke rigtigt har nogen undskyldning for ikke at have gjort det endnu. Hvis så fremtiden vil være enige i det udsagn, så bør de så acceptere den indstemte deadline. (Og folk kan så stemme om flere deadlines senere hen, hvis der er en vis uenighed, og fremtiden kan så bare vælge den første, de er enige med.) Fra dealinen af kan det så noteres i systemet, at gruppen har skyld. Hvor stor den skyld er, og hvor meget de skal straffes for at have denne skyld (i form af en nedsat donationsfaktor, for bidrag der kommer dem til gavn (i.e. for donationer til folk, der yder bidrag til dem)), det må fremtiden i princippet så bare bestemme. Men i øvrigt behøver alle sådanne bestemmelser ike at være et sort hul, som man ikke kan gisne om, for nutiden bør bare selv diskutere og stemem om, hvad de (eller rettere hvad forskellige grupper) synes om, hvad alle satserne osv. skal være. Og det vil så nemlig være oplagt for fremtiden at rette sig efter, hvad samtiden mente om, hvad satserne skulle være; hvorfor ikke!? Og hvis fremtiden så har nogle lidt andre meninger, så mener jeg, at det vil være naturligt for dem så bare at justere tingene løbende. Så hvis en samtid er kommet frem til nogle fornuftige parametre i systemet, så vil det give mest mening for fremtiden allerhøjest kun at ændre parametrene med fremadrettet kraft, hvis denne ender med at have lidt andre meninger. Cool. Desværre bliver systemet bare en anelse kompliceret nu, så jeg bliver altså lige nødt til at tænke over, hvordan disse principper kan formuleres mere enkelt (/ hænges op på noget mere enkelt). (01.01.22) 
%(01.01.22) (stadig) Ah, måske er der en mere simpel løsning! Man kan måske bare dele folk op i foreninger og/eller grupper af befolkningen, og så kan det måske lidt bare være et politisk spørgsmål grupperne imellem, hvor meget man vil betale donationer.. hm, til andre grupper, og hvad man vil sætte faktorerne til for andre grupper.. Hm, jeg skal vist lige tænke noget mere over dette.. ...Ah, men bliver der ikke bare automatisk så et ansvar ligesom overfor sine arvtagere i en gruppe, hvis man deler det op sådan, og grupper kan jo så være alt fra beboelsesgrupper til faggrupper osv., for, ja, enhver gruppe kan jo selv vælge, hvordan den vil dele resten af verden op i princippet.. 

%(02.01.21) Nej, det var faktisk meget godt, det jeg havde med faktorerne og sådan. (Og også en god idé, det med at stemme om deadlines i samtiden..) Nå, men i går aftes blev jeg faktisk helt i tvivl, om idéen kunne lade sig gøre, eller om jeg stod over for en stor nedbrydelses- og genopbygningsproces igen.. Men så kom jeg frem til, at jeg har tænkt for meget på lykke/gavn --- altså hvor meget gavn en handling giver andre mennesker --- hvor jeg også meget mere burde tænke i løn ift. arbejdstimer. Så nu tror jeg, man skal sige, at hvis et arbejde er gjort ærligt, og hvis fremtiden kan sige god til, at arbejdet var værdifuldt og at arbejderen ikke helt klart burde have brugt sin tid på noget andet (med dennes viden! (SÅ med andre ord er det helt okay at arbejde på noget, der ikke bliver brugt, hvis man ikke kan vide, om det vil blive brugt eller ej, og at arbejdet derfor alligevel hører med som en sund og naturlig del af det samlede arbejde)), så skal der som minimum gives en fair time-løn. (Og hvis arbejdstiden ikke er indført, så må man bare sammeligne arbejdet med tilsvarende arbejder --- og måske man også kan gøre dette, hvis nu arbejdet ser for sølle ud ift., hvad personen burde kunne opnå). Derudover bør der så være en bonus (som fremtiden i sidste ende bestemmer, men som samtiden som nævnt sikkert godt kan stemme om og få indflydelse på i praksis) alt efter, hvor mange mennesker det kommer til gavn. Her kunne man måske endda love hinanden at bruge et system, hvor folk simpelthen bare stemmer på, hvad de anser som vigtigt.. Hm, tja.. Nå, det kan jeg vende tilbage til. Denne bonus må så også meget gerne afhænge yderligere af, hvad det har kostet arbejderen/bidragsyderen at bidrage. Dette kan f.eks. være i form af personlige omkostninger, det kan være i form af risici (hvad der f.eks. vil være forbundet ved at sætte sin lid til dette system i starten, inden det vides helt, om det vil holde), og det kan også være i form af at give afkald på mulige patent-chancer og at dele sine idéer i stedet for at udnytte chancen ved at holde dem tæt til kroppen og jagte patenter m.m. I princippet kunne man også se arbejdstid som en sådan omkostning, men jeg tror umiddelbart, det kunne være en meget god opdeling, denne jeg lige har nævnt.. Og så vil der hermed blive et rimeligt sikkert bundniveau for tokens'ne, som kan estimeres ved at se på tilsvarende samtidigt arbejde (og hvor man så bare lige skal kunne svare ja til, at fremtiden sikkert vil værdsætte arbejdet nogenlunde ligeså meget (eller mere), som hvis vedkomne havde lavet tilsvarende arbejde for en gængs arbejdsgiver i stedet (og fået normal løn)). Og derudover vil der så være gavn- (og set ift. de parsonlige omkostninger) bonussen, som man så kan spekulere i: Hvor store vil fremtiden gerne have at sådanne bonusser skal være?. (Og fremtiden skal selvfølgelig være ærlig og bedømme fortidige bonusser ret meget på lige fod med, for dem, nutidige. (Og kun have en forskel, hvis nu der var en generel stemning for et vist niveau i samtiden, og at fremtiden gerne vil ære denne stemning (og derfor kun ændre parametrene med fremadrettet kraft)).) Så det bliver nok det, og så er der altså også lige samlings-værdien af tokens'ne, som jeg bare afhænger af folks samler-gener/samlerlyst. Og så skal systemet i øvrigt også indeholde det med, at hvis en gruppe mennesker får skyld, så vil.. hm, det vil så være bonusserne umiddelbart, der vil falde.. Hm, vil dette så skabe en naturlig nedre grænse for faktoren, nemlig hvis arbejdstime-lønnen ikke ændres..? Eller vil dette ikke være ønskeligt..? ..Ah, man kan jo altid bare sætte en faktor på skyldnernes egen arbejdstime-løn også. Ja, så man kunne sætte en faktor deres egen arbejdstime-løn, og så en, måske tilsvarende, men måske anderledes (..), faktor på, hvad bonussen stiger med som resultat af, at bidraget har kommet skyldnerne til gavn. Ok, og hvordan skal disse faktorer så ligesom beregnes og/eller forudsiges..? ..Hm, hvad med at sige, at faktoren bare skal være ens for de to ting, og at.. Tja, måske er det egentligt bedre, hvis den overvejende er på arbejdstime-l.. eller..? Men pointen var så, at man måske bare kunne sørge for, at gruppen kommer til at betale lidt mere i sidste ende, end hvad de skylder.. ..Og måske bare noget a la 10 \%.. ..Men ja, problemet så er bare lidt, at så kommer tokenværdier til at afhænge af, hvor meget arbejde de gør, og hvor meget gavn de får fra andet arbejde i fremtiden.. ..Hm, er det overhovedet nogen smart måde at gøre det på..? ..Hov, nu fik jeg lige en anden tanke i øvrigt: Kunne man mon gøre noget a la mine 11/6-idéer, bare med donationsforeningerne..(!..)? Det må jeg lige sørge for at tænke over også..  
%... Nå, jeg har tænkt en del mere, og nu er jeg så lidt kommet frem til, at det bare sådan set er "næste generation," så at sige, der bestemmer skyldspørgsmålet, og sådan set også bare kan implementere det ved at nedvurdere værdien af tokens fra skyldnerne tilsvarende.. Så det er altså lidt hvor jeg står nu, men jeg skal lige tænke det hele igennem noget mere.. 
%...Der er den model, man opgiver (som eventuelt har mangler eller mangler helt), den man faktisk følger i sidste ende, og den som de næste generationer, så at sige, synes man burde have fulgt (som så kan afhænge af, hvad man opgav).. 
%Og ja, det er stadig godt at dele det op (og også at kunne dele tokens'ne op) i arbejdstime-løn-tokens og gavn-til-andre-mennesker-kontra-personlig-ofring-og/eller-risiko-tokens. 
%Ah, og man kan stadig bruge det der dealine-(samtids-)afstemnings-koncept..
%Hm, og man kan vel stadig have de der faktorer med (i modellerne (se tidligere noter angående, hvad jeg mener med "modeller" btw)), hvor bidrag der giver gavn til skyldnere så bør nedjusteres i værdi.. (Og alle sådanne ting; alt hvad hjertet kan begære..) ..Så ja, der behøver vel nærmest ikke være nogen indbyggede regler om modellerne: Folk i samtiden kan bare prøve sig frem, og hvis ikke de handler ulødigt og selvisk, men gør et rimeligt arbejde, så vil næste generation sikkert respektere tokens'nes værdier, som de er/var, og altså bare sørge for at lave ændringer fremadrettet.. 

%(03.01.22) Det er jo virkeligt fantastisk, hvis jeg har fundet en version af idéen, der holder totalt godt vand. Det er lidt sjovt, for jeg har jo opgivet idéen før, men tog den op og godkendte den, fordi jeg syntes den var noget værd, også selvom der var huller i den, hvor menneskelig grådighed kunne komme ind og ødelægge det. Men nu hvor jeg så har fremhævet idéen, som noget stort jeg gerne vil fremføre som noget af det første, og som jeg faktisk, med de planer jeg har i sinde, gerne vil have til at være udgangspunktet for web-idéerne, så kræver det jo pludselig igen en meget mere stabil idé. Så jeg kunne have forudset (men gjorde det ikke), at jeg ville nå til (hvad jeg gjorde i forgårs aften), at idéen måtte gennemgås og omtænkes. Det har jeg så gjort lidt nu, og det er lige før, jeg tror, at jeg nu endelig har en robust idé (som altså ikke holdes fuldstændig oppe af en moralfølelse blandt folk om ikke at pringe over, hvor gærdet er lavest (hvilket nemlig især er farligt for store grupper, for mentaliteten, "hvis de gør det, så gør vi det altså også," vil jo altid være der)).
%Idéen, jeg har nu, kan vel helt kort siges som, at folk i foreninger (eller i foreningsløse grupper) sørger for at love en model (og ja, gerne hvor man altså har modellen opskrevet offentligt) for, hvordan de vil donere penge til diverse tokens i systemet. Foreningerne må dog godt lade være med at donere hele det beløb, bidragene vurderes til, især lige netop, når der kan være åbne spørgsmål omkring det, for så kan man nemlig bare beholde skyldsspørgsmålet i form af tokens, som fremtiden så i sidste ende får til opgave at vurdere endeligt (og betale de sidste donationer af til). Ydermere kan fremtiden også holde øje med, om samtidens folk var for nærige med at betale den mere faste del af donationerne af, hvorved der nemlig så kommer en pengeskabelse (nemlig hvis en generation donerer færre penge end den forrige, og derved tvinger næste generation til at donere mere). Fremtiden kan så modvirke dette ved simpelthen at straffe forrige generation, hvis de var for nærige ift., hvad de havde af informationer på daværende tidspunkt, ved simpelthen at nedvurdere tokens, der kommer fra pågældende genrations bidrag. Eller dvs., i virkeligheden er der flere grupper i en "generation," og fremtiden kan så straffe de nærige grupper specifikt heri. Og måden dette mere nøjagtigt sker på, er, at samtiden påpeger sådan nærighed, hvis den forekommer, og holder en afstemning om en deadline. Hvis ikke skylden er betalt i høj nok grad inden den deadline, så skal alle tokens efter denne deadline altså potentielt nedvurderes af fremtiden, hvis denne ender med også at dømme gruppens handlinger for for nærige. 
%Hvis fremtidens grupper så stadig er uenige, så lader man altså bare spørgsmålet stå til næste generationer. Så der skal derfor være plads til en vis pengeskabelse i systemet, hvilket netop vil komme i form af kontroversielle tokens, som endnu ikke har fået den sidste gæld betalt af på sig (måske; det bestemmer fremtiden så). Og derfor er det også vigtigt, at folk ikke sløser med at betale af, hvor der er enighed, så pengeskabelsen \emph{kun} vil komme fra de åbne spørgsmål, og ikke fordi folk af nærige med at betale af (og altså lader som om, spørgsmålet er åbent, selvom det ikke er det). 
%I går aftes kom jeg i øvrigt på, at foreningerne med fordel kan lave betalinger samtidigt i "runder," hvorved foreningerne herved med det samme kan regere på en forenings brudte løfter (nemlig ved så selv at donere fære penge til denne). Herved kan man så undgå mange små fejder, især når eksport er nogenlunde lig import (så for alle rimeligt stabile økonomier), for så vil det ikke kunne betale sig, end ikke kortvarigt, ikke at holde sig til sine (donations-)løfter. Samtidigt må foreningerne gerne offentliggøre en fortolkning af andre foreningers skyld-modeller, og særligt altså offentliggøre, hvad de forventer at få af donationer ved hver runde, så der på den måde bliver god til at afstemme eventuelle uenigheder. 
%Og ja, hvis nogen ikke donerer/betaler, hvad der forventedes af dem (ud fra hvad de har lovet --- og, hvis de simpelthen har lovet alt for lidt, hvad der er fair (for hvis en forening pludselig lover noget helt unfair, så kan fremtiden faktisk gå ind og rette disse løfter med tilbagevirkende kraft (hvad der dog kun skal gøres i ret drastiske tilfælde))), så kan andre foreninger altså i første omgang bare holde igen med deres donationer/betalinger, og hvis ikke man kan forhandle sig frem til en aftale efter sådanne tilbageholdelser, så må man så bare gå over til at erklære, at spørgsmålet står åbent for fremtiden at afgøre (hvorved man så altså aftaler at lade pågældende tokens stå (u-afbetalt) indtil eftertiden).
%Systemet er holdbart, fordi det er et godt, virkeligt smart og praktisk, økonomisk system, hvor man bare kan afregne løn endeligt, når man kan se tilbage på, hvordan alting gik, og hvor bidragsydere ikke behøver at sælge deres arbejdskraft, før de bruger den, men i princippet bare altid kan bidrage ad libitum, og således vil et kollaps af systemet altså gerne undgås af størstedelen af befolkningen. Samtidigt vil et kollaps også medføre (nærmest pr. definition), at en masse mennesker ikke får som forskyldt/fortjent, og der er vi dog altså for store tilhængere af retfærdighed som mennesker (også selvom vi er selviske på mange områder --- retfærdighedssansen har vi trods alt i den grad (generelt)), så hoveddelen af befolkningen vil altså også græmmes over denne tanke. Og så længe hovedparten af folk ikke har lyst til at et kollaps skal ske, så sker det heller ikke, for så vil tokens'ne være det værd, som hoveparten synes, de skal være. Og i øvrigt er det ingen gang folk i nutiden, der skal afgøre disse værdier, men folk i fremtiden, så selv hvis folk i nutiden, eller "samtiden" skulle jeg hellere sige (for i min nutid, i.e. i skrivende stund, findes systemet jo ikke), pludselig skulle forlade systemet kortvarigt, jamen hvis så den efterfølgende genration vil, så kan de bare indføre systemet igen og lade som om, det aldrig var stoppet, og så kan det altså bare straffe forrige generation for de ting, den gjorde forkert.. Og her har vi så også grunden til, at systemet kan opstå ret let, for det kræver bare en tro på, at fremtiden vil synes om systemet. Og jo flere begynder at tro på dette, jo flere bidrag til fællesskabet vil komme i samtiden, hvilket vil få flere og flere med på idéen, hvilket vil få flere og flere til at forudsige, at systemet vil blive til noget stort, og at fremtiden vil synes om systemt i sidste ende. 
%Ja.. ..Ja, værdier vil bestemmes, måske ikke af en majoritet af mennesker nødvendigvis, men stadig en majoritet af værdihavere om ikke andet. 
%... Ja, jeg tror altså, det holder. Og hvis en generation føler sig forkert bedømt af en efterfølgende generation, så kan man altid bare igen anke indtil næste generation igen, og lade tokens'ne stå u-afbetalte. Og i øvrigt kan man også komme pengeskabelsesproblemet ret godt til livs ved så at gå sammen i grupperne (og på tværs af grupper) om at låse tokens til konti, hvor kun meget få af tokens'ne på kontoen kan handles ad gangen. Bum. ..Hm, og hvad gør man, hvis man nu gerne vil betale lidt mere af, end hvad den nedre grænse på ens estimering af den endelige skyld er..? ..Ah, om ikke andet kan man jo bare udstede en speciel token til dette formål, hvor fremtiden så kan betale tilbage på denne, hvis man har givet for meget.:) ..Hm, nå nej, for så vil fremtiden jo bare skulle betale den ekstra sum for en; det er jo ikke mere fair.. Så hvad gør man lige..? Hm, kan modtager ikke bare give.. hm, tja.. Skyld-tokens giver vel ikke specielt god mening.. For skyld skal vel bare holdes til at være noget, man bestemmer i sine modeller..? ..Hm, men kunne man så give skyldnere plads til at betale mere netop ved at love i sine modeller, at man vil betale det tilbage i fremtiden? Ja, det må da bare være svaret..(?) Jo, ja. Fint.:) 



%Okay, lad mig så prøve at gå i gang med ny dispositions-brainstorm..: (stadig (03.01.22) btw)
%Hm, hvis jeg skulle lave (ikke at jeg vil det umiddelbart) en disposition uden at snakke NFT, hvad ville jeg så starte med at forklare..? 
%Det er jo ret essentielt det med, at man ligesom udsteder tokens (som jo så kan kaldes NFTs, hvis man vil) på for sit arbejde, og så simpelthen forvente at folk i fremtiden vil donere penge til en ift., hvad der er "fair." Og dette kan så ligesom føre en hel lille økonomi med sig.. ..Hm, skulle man mon så allerede dele det op her og snakke om, at forskellige grupper af mennesker så kan være indirekte i gæld til hinanden, hvis en gruppe har modtaget gavn fra bidrag fra den anden gruppe..?.. ..Selvfølgelig forstyrrer det lidt, at sådanne grupperinger ikke er skarpe i praksis, men kan man ikke hurtigt bare lige forklare dette, og så gå videre med at antage, at samfundet alligevel kan inddeles i grupper..?.. ..Så kan jeg nemlig måske starte med ligesom at sige: "Lad os antage at folk har ydet bidrag uden at nogen har købt deres arbejdskraft først, og at de har udstedt tokens på baggrund af det. Vil nogen så betale dette tilbage? Ok, lad os så se på samfundet som værende inddelt i grupper. Så vil spørgsmålet altså blive, om de andre grupper vil kunne forventes på et tidspunkt at donere penge til sådanne bidrag, der har kommet dem til gavn." ..Og hvad skulle jeg så sige efter dette..? At jeg tror på, at det kan de meget vel komme til, for jeg tror på, at man kan opbygge en hel lille (og meget praktisk!) økonomi på baggrund af dette..? Hvor diverse grupper involveret i systemet altså betaler hinanden tilbage bagud.. Og ja, hvor disse donationer alt andet end lige retter sig imod enkle bidragsydere endda.. ..Hm, sådan kunne man måske godt strukturere det.. ..Altså som første del af forklaringen.. Og så ville man så skulle til at argumentere for, hvordan og hvorfor denne økonomi så virker.. ..Her kunne man måske bare vente med "generations"-snak og starte med bare at give et billede af de forskellige grupper, der har et smart system kørende, som så måske er for smart til, at nogen af dem vil bryde systemet.. ..Hm, og kunne jeg så sige noget i retning af, "jo, det kunne man godt forestille sig, men der er lige nogle forhold, som vi mangler at tænke på," og så fortsætte med at påpege et problem ved dette system.. Og hvad er problemet så lige, som man skal påpege?.. ..Hm, noget med pengeskabelse, eller..? ..Der er vel også et problem, at det måske så ikke er så nemt at løse konflikter (..i hvert fald som hvis man havde "generation-snak"-delen med også..), men skal man virkeligt tage udgangspunkt i dette, eller skal man så bare nævne dette efterfølgende..? ..Tja, og dog.. For det er vel nærmest mere centralt ved idéen, det med at fremtiden altid løser uenigheder, mere end hvordan man undgår pengeskabelse, der løber løbsk.. ..Ja, det er nok rigtigt: Det er nok bedre at starte med at påpege, at der umiddelbart så kan være problemer med, hvordan man løser uenigheder.. Hm, men kan man overhovedet komme uden om, at det er fremtiden, der afgør spørgsmålene, så længe? Ligger det ikke ligesom i det at bruge tokens..? (..Ah, dispositioner er svære..x)) ..Hm, man kan nok godt komme udskyde ligesom at tage hul på, at en gruppes meninger og interesser ligesom kan udvikle sig med generationerne, og at samme gruppe derfor godt ligesom kan være mindre partisk, når den ser tilbage på et spørgsmål fra gamle eller havl-gamle dage.. ..Ja, så derfor kan jeg nok godt bare snakke om "grupper," som om de ikke ændrer sig over tid, til at starte med.. ..Hm, men behøver man overhovedet så at se "'generations'-snakken" som en udvidelse af idéen; kan man ikke bare se det som en forklaring af, hvorfor det virker, og altså nærmere bestemt hvorfor uenigheder kan opklares?:).. ..Jo, det kan man vel.. ..Og så kan jeg så komme ind på, at fremtidens grupper (nu hvor jeg har snakket om, at de kan have andre interesser) også kan straffe de gamle, hvilket kan bruges til flere ting. For det første, og dette er ret vigtigt, kan det bruges til at holde folk i ørende, så de betaler af på tokens. For hvis de iikke gør, så ryger denne betalingsbyrde jo bare videre til næste generation, hvilket jo ikke er fair. Og man kan nemlig heller ikke bare ignorere fortidig gæld, for ellers får man større og større pengeskabelse i systemet, og det går ikke. Så kan jeg så forklare, hvordan fremtiden kan straffe samtiden, hvis de ikke betaler sin skyld/gæld, og hvordan dette sikkert med fordel kan involvere samtidige afstemninger, hvor man beslutter deadlines for andre grupper (og hvis fremtiden så er enig med præmisserne for en sådan afstemning, så bør de så godkende pågældende deadline-tidspunkt og begynde at nedjustere værdien fra bidrag fra skyldner-gruppen herefter). Og så kan jeg også sige, at sådanne straffe også kan bruges, hvis nu en gruppe simpelthen var al for nærrig med deres donationsløfte-model i første omgang, eller selvfølgelig hvis de ikke fulgte den (men dette sidstnævnte hænger jo ret meget sammen med ikke at betale sin gæld --- forskellen er jo bare, om samtiden selv er enige i denne gæld eller ej..). ..Hm, kunne dette mon allerede være en nogenlunde disposition..? ..Hm, de forskellige typer tokens, som jeg skrev om i går, kan jeg vist rigtignok bare forklare om efterfølgende.. Og hvad med.. Hm, hvad var det nu..? ..Hm, det var underligt: Jeg følte, jeg tænkte på noget andet, men enten var det ikke vigtigt, på den måde at det bare var en ting, man sagtens kunne komme ind på efterfølgende, eller også.. ja, eller også kommer det jo til mig igen. Men ja, der er nemlig nogle flere få ting, der skal nævnes, men dem jeg lige kan tænke på, de kan altså sagtens bare tilføjes efter denne overordnede gennemgang af idéen.. ..Ah, det var vist "modellerne," jeg tænkte på.. Hm ja, hvor skal man putte dem ind henne?....Jo, jeg må jo forklare om dette, når jeg har beskrevet billedet (eller som en del af dette billede, kunne man også sige) med grupper, der betaler hinanden tilbage / donerer til hinanden for diverse bidrag, og altså inden jeg så går over til at snakke om tidslige forskelle i grupperne (og hvordan man kan udnytte dette).. Ja, så her skal modellerne altså på banen.. Og måske kan man også forklare om "runder" osv. her, det ville måske ikke være dumt.. Men det kan jeg jo se på.. Ok, og så har jeg altså muligvis en god overordnet gennemgang af idéen (i min nye version her), og så kan jeg så fylde på med små noter efterfølgende.. Cool.:) (..!)  

%(04.01.22) For det første: Pis, pis, pis! Den gik ikke. Det var egentligt en rigtig god idé, jeg kom frem til her i forgårs og i går, men som jeg kom frem til i går aftes, så har jeg altså overset en vigtig ting. Jeg kom til at antage, at systemet relativt hurtigt vil blive udbredt til alle grupper (måske på et par generationer bare), men det kan jeg jo ikke.. (Det er i hvert fald min holdning nu..) Og ja, jeg synes så faktisk, at det hele falder lidt til jorden. Jeg tror egentligt stadig på, at systemet kan holde, men.. ..Det er ligesom om nu, at det så ikke rigtigt vil være besværet værd; ikke ift. andre løsninger (såsom hvad civilforeninger, donationsforeninger og/eller kundedrevne virksomheder (eller tilsvarende ting) kunne bringe med sig (og sådan idéer er jo alligevel generelt en hel del simplere..)).. Men ja, jeg kom altså nok ligesom til at tænke, at internet-bidrag ville fylde mere af den overordnede økonomi, ligesom, i fremtiden.. Og selvom den sikkert vil komme til at fylde meget mere, så kan jeg på ingen måde regne med, at den kan komme til at fylde særligt meget. Og problemet er, at systemet ikke kommer til at fylde så meget, så kan man ikke straffe folk (eller rettere straffe "den forrige generation"), og så vil systemet bare ikke give mening ift., hvor kompliceret det er.. ..Hm, jeg kan nu virkeligt godt lide idéen (eller denne version af idéen, rettere), og det er næsten lige før, jeg har lyst til at skrive den ind i den renderede tekst, men det gør jeg nu nok ikke. Jeg lader den nok bare stå her ude i kommentarerne, og går nok heller ikke videre med den.
%Det gode er så til gengæld, at jeg lige akkurat nåede i går nat at komme til at tænke på, om ikke det bare ville give god mening i stedet at fokusere på en simpel idé om, at skabere bare får stemmeret i en open source web 2.1/3.1-forening, alt efter hvor meget de har fået doneret til sig. Nu har jeg så tænkt videre over denne tanke, og er faktisk umiddelbart ret optimistisk omkring den..! Så selvom det var surt i går aftes, så kan alt dette nok alligevel ses som en virkelig positiv udvikling.:) (Altså selv set fra et subjektivt POV; al udvikling er jo positivt fra et objektivt POV.) Og inden jeg lige forklarer nærmere om den idé, kan jeg også lige sige, at jeg nu også mener, at jeg kan bryde donationskæde-idéen ned til en meget simpel version, som så bare går ud på at sige: Man kunne nemt forestille sig, at folk vil være interesserede i at eje NFTs over bidrag, der har kommet dem og deres peers til gode, især hvis man oven i købet får en vis konsensus om, at når en person køber tokenet til et tidligere bidrag, så er det næsten lige før, at han/hun herved overtager æren for det --- især hvis købsprisen oversteg, hvad bidragsyderen almindeligvis skulle have for at gøre det (hvis denne altså påtog sig en noget-for-noget mentalitet, og derved ikke ville gøre bidraget gratis overhovedet). Og hvis en ny køber køber samme token til en købspris, der også overholder dette koncept, så kan æren altså ligesom gå på omgang, nærmest. Selvfølgelig vil prisen skulle falde over tid, for man må regne med, at der kommer flere bidrag til. Men her kan fællesskabet så bare blive enige (på decentral vis) over hvordan sådanne tokens typisk falder i værdi. Og dette vil altså være en naturlig følge af, at folk ikke vil synes, det er så interessant at samle på gamle bidrag, der alligevel ville have været kommet hundrede gange siden det blev gjort, hvis ikke den originale bidragsyder havde gjort dem. Så tokens'ne vil falde helt naturligt i samlerværdi --- i hvert fald hovedparten af dem --- og så kan man jo bare se det som, at "æren" stadig bliver overført lidt ved hver handel, også selv hvis prisen har dalet noget. Og lige inden jeg begyndte at skrive her, kom jeg så lidt frem til, at jeg måske endda bare kan nævne dette (og især den simple version af det, men måske kan jeg også nævne det med "æren" ligesom) i forlængelse af web 2.1/3.0-noterne; ikke som en stor selvstændig idé (for andre må også have tænkt noget ret tilsvarende, når det kommer til NFTs), men altså mere bare som en ekstra note, der kan bidrage til web 2.1/3.0-idéerne og give dem mere styrke. :)
%Nå, og nu til at skrive nogle flere ord om idéen, der på en måde bare er en "skaber-styret" virksomhed, hvad jeg helt sikkert har tænkt over før. I modsætningen til mine "kundedrevne virksomheder" så virker denne idé for en virksomhed / et fællesskab omkring open source bidrag, hvor man altså ikke opnår IP-rettigheder og/eller aktier/værdipapirer, men hvor man bare prøver at tiltrække donationer. Og modsat så virker min idé om "kundedrevne virksomheder" lige netop på virksomheder, hvor der er klare værdier involveret i firmaet, og hvor nogen altså skal eje disse på en eller anden måde. (Og sidstnævnte vil så også være meget mere afhængig af investeringer, og kan ikke nær så godt klare sig med bagud-donationer.) 
%Idéen er bare, at man for en sådan web 2.1/3.0-forening, som jeg opsummere i mone noter fra d. 29/12, sørger for, at brugerne alle sammen melder sig til en form for donationsforening (hvilke gerne bare skal være en del af siden). Og man kan bare starte med helt simple "foreningerne," hvor donationerne f.eks. bare automatisk går til de skabere, der har flest likes og views, eller noget i den stil. Men man skal selvfølgelig altid kunne skifte til en anden (bedre) donationsforening. Når skaberne får deres donationer, så får de også en tilsvarende forøget stemmeret til de næste valg i foreningen. Der er så skaberne, der indgår i disse valg, hvilket i øvrigt er meget nemmere, end at hver bruger og/eller donor skal registrere sig selv. Jeg føler så, at jeg kan argumentere for, at skabernes interesser og de donerende brugeres interesser alt i alt må være stort set på linje med hinanden. De vil nemlig alle sammen gerne have flere og bedre bidrag til foreningen samt få flere andre til at donore også. Og brugerne donerer altså bare til de skabere, de gerne vil støtte for ligesom at tiltrække flere bidrag af samme type, og kan i øvrigt også rette deres donationer til, så de skabere, der er politisk enige med brugerne får mere, end dem der ikke er. På denne måde vil skabere stort set af sig selv komme til at rette sig ret meget efter brugernes --- specifikt de donerende brugeres --- behov. Dog vil tidligere skabere beholde deres stemmeret i noget tid, hvilket altså muligvis giver en lille forskel, for hvis det var brugerne, der stemte, så kunne de jo bare stemme for nul penge til tidligere skabere, også selvom deres bidrag stadig er relevante og i brug (og altså bare ignorere dem, simpelthen på baggrund af, at de ikke ser ud til at ville komme med flere bidrag). Og det er jo meget sundt, at man automatisk slipper for sådan en fristelse, og at gamle bidragsydere altså nok skal få, hvad de har fortjent for deres bidrag, uanset om de fortsætter med at være aktive efterfølgende eller ej. På den anden side kunne man dog frygte, at skabere vil bruge deres stemmemagt til at fremføre dem selv på hjemmesiderne, for så at tiltrække endnu flere donationer. Men brugerne skal bare være vakse over for dette, og sørge for ikke at støtte skabere, der stemmer for ting, der unødigt er for at fremhæve dem selv og deres egne indtægter. For i sidste ende vil brugerne have magten med deres donationer, også selvom skaberne altså ligesom kan genbruge deres stemmemagt over flere afstemninger, også efter at folk har stoppet med at donere til dem. Er der mere, jeg skal sige lige nu..?
%Ellers så vil jeg altså bare nævne denne idé i forlængelse af mine web 2.1/3.0-idéer (som vistnok nogenlunde opsummeres meget godt i mine noter fra d. 29/12..).. Ja, eller rettere, som en del af dem.. 
%Jo, jeg skal lige sige, at skabere ikke kan sætte betalingsmure op for indholdet direkte, for al imdkomst skal nemlig gerne komme fra donationer, men til gengæld så kan de godt sætte mure op, når det kommer til serverne i fællesskabet. For det er nemlig lige præcis disse, som skaberne har stemmemagt over.. Hm, hov vent, holder dette nu også..? ..For folk kan jo bare kopiere dataen, hvis man altså holder sig til, hvad jeg skrev der i 29/12-noterne.. 
%... Hm, måske man godt alligevel kan lave noget med donationrettigheds-tokens til web 2.1/3.0-foreningen.. Og så kunne investorer måske bare have det med i satsningen, at fællesskabet vil finde en måde at lokke de fleste til at betale, hvad de moralsk set skylder for deres forbrug.. Hm... 
%(..Tanken er altså lidt, at første bidragsydere så bare kan gøre det fuldt open source, men at fællesskabet så forventer af sig selv på et tidspunkt og gå over til også at inkludere closed source ting, sådan at man kan lokke penge fra den brede brugergruppe og på den måde kompensere de første donorer (men som så netop har kunne handle med tokens indtil da, så hvis folk generelt har tillid til, at dette vil ske, så mister de første donorer altså ikke meget værdi på donationerne (fordi de jo altså kan sælge tokens'ne videre med det samme i princippet))..)
%(Og brugerne kommer så effektivt til at kunne bruge hjemmesiderne rimeligt gratis, og så bare love fremtidig betaling i stedet.. hm, selvom de nu også burde melde sig til en ikke-nul-betalings-donationsforening til at starte med..) ..Ah, men en god ting, at man så heller ikke behøver funktionelle donationsforeninger fra start af.. 
%Folk kan også måske vælge at købe tokens \emph{og} betale dem af (til sig selv), således at de på den måde opnår æren ved at blive den, der endeligt betaler prisen af.. Hov, dette viser da i øvrigt et hul.. For det koster jo ikke noget at betale.. nå nej.. Nej, never mind. Man skal selvfølgelig bare ikke kigge så nøje på de endelige priser, der betales af, for det kan jo være, at folk bare betaler til sig selv (eller til venner/"sammensvorne").. 
%Hm, men ellers er jeg så nogenlunde tilbage, ca. hvor jeg var før nytår..? ..Tja, på nær måske også, at min web 2.1/3.0-idé og min donationstoken-idé nu nok hænger tættere sammen.. ..Tja, men dog ikke at begge idéer ikke godt kan bruges uden / i andre sammenhænge end den anden.. ..Hm, men det er nu næsten lige før, at jeg stadig kan komme med token-idéen bare i forlængelse af den anden, som jeg også skrev nu her ovenfor.. ..Ja:).. 
%Nå, men vil det så sige, at skaberne ikke skal få stemmeret over serverne (og altså alt efter, hvor meget doneres til dem)?.. ..Hm, det var jo ikke umiddelbart en helt dum idé.. ..Hm, men problemet bliver jo netop så, at så skal man til at håndhæve rettigheder som en central del (og altså også gøre det rimeligt meget fra start..).. 
%Hm, men er det egentligt så svært at putte en IP-licens på alle uploads?.. Og med denne kan man jo så nemt nedlægge alle server-netværker, der ikke overholder fællesskabets retningslinjer.. ..Og så kan man jo netop godt have systemet, hvor skabere har stemmeret alt efter donationerne.. ..Hm, vil det så sige, at idéen så holder, og at der i princippet ingen gang er brug for donations-tokens?..!.. (Er jeg så tilbage til nogenlunde, hvor jeg var i eftermiddags? ;)xD) ..Hm, for så kan jeg jo netop sige, at skaberne kan stemme om at udelukke visse brugere (som er medlemmer af tarvelige donationsgrupper på hjemmeside-netværket) fra visse ting hos serverne.. ..(Men så kunne man nu altså stadig sikkert godt have en "use now, donate later"-tilgang til tingene, og så altså bruge tokens som en måde at tage forskud på de fremtidige (forhåbentlige) donationer på..) ..Hm, og hvad hvis nogle skabere gerne vil gøre deres bidrag mere offentlige, end de andre (stemmeangivende) skabere vil..? Hm, så må de vel bare kunne dette, så længe der er tale om deres originale arbejde. Og hvis der så er uenighed omkring det, så kan skaberen jo bare uploade det til et andet fællesskab, og så bliver det således et lovligt spørgsmål, om dette så har lov til at give deres brugere afgang til det (og ellers kan de få copyright strikes). Men hvis en skaber udgiver sine ting til et vist fællesskab af min type og under dennes licenser, så er det altså i første omgang en demokratisk proces, der afgør, om arbejdet/dataen er original(t), og om skaberen derfor har lov til selv at bestemme, hvem det skal vises og ikke vises til.. Hm, og det kan i øvrigt også være, at skaberen har givet samtykke til, at andre kan bygge videre på det, og så kan de heller ikke bare selv bestemme over det derefter.. Hm, disse sætninger er lidt prøvende; jeg er ikke sikker endnu på disse ting, men skal fortsat lige tænke over dem.. 
%..Hm, i princippet kunne man også lade donationsstørrelserne bestemme, hvad er originalt og ikke-så-originalt arbejde, men det er måske lidt drastisk.. Men en tanke, der i det mindste er værd lige at summe lidt over --- også fordi den i øvrigt også indeholder noget med, at folk som har samarbejdet om et værk så kan få stemmeret ud fra, hvad er doneret specifikt til deres enkeltvise bidrag.. 
%..Men ja, sidstnævnte vil være lidt drastisk og lidt mærkeligt, for så ville skabere kunne købes ud af deres egne værker, og endda uden at få nogen betaling (tvært imod).. ..Så det er nu nok bedre, at det ligesom bliver op til licenser og jura.. 
%(05.01.22) Min idé her til et token-økonomi-system kunne nok godt ubrede sig langt nok til, at det ville give mening, når først vi når til mine "civil-foreninger" og alt det..
%(Indskudt: Jeg skal i øvrigt prøve at finde et bedre navn til den idé. "Civil-foreninger" er ligesom bare det, der har sat sig fast i mit hoved. Men 'civil' accocierer jo mere til ikke-militær, så det er jo nok et fjollet navn. Jeg ved bare ikke hvad jeg ellers lige skulle kalde idéen.. Og "folke-foreninger" eller "borger-foreninger" virker også bare lidt ved siden af på en måde.. Nå, det må jeg jo lige sørge for at tænke over..)
%Men ja, så idéen kunne måske blive brugbar på det tidspunkt --- bl.a. til at få gang i iværksætteriet omkring "civil-foreningerne" og det hele.. ..Det er i hvert fald en idé, man kan overveje til den tid, om ikke andet.. 
%*(Og når jeg siger "civil-foreninger" her, så er de egentligt især mine "forbrugerforeninger," jeg tænker på (som jo lidt kan ses som en underdel af, hvad jeg forestiller mig, man kan opnå med "civil-foreningerne..).)
%Jeg kom i tanke om i går nat, at jeg nok skal passe lidt på med at snakke om arbejdstime-løn for token-systemet. Nu er det ingen gang sikkert, at jeg vil fremføre token-systemet særligt meget, men hvis jeg gør, så skal det altså ikke forstås som, at alle internet-skabere pludselig skal betragtes som timelønsarbejdere, slet ikke.. Men hvis nu de laver et så vigtigt stykke arbejde (og dette vil jo især være programmørerne så), sådan at det sagtens kan sammenlignes med tilsvarende arbejde gjort for gængse (private) firmaer m.m., så vil det retfærdige jo netop være, at de ligesom i hvert fald ikke skal mindre, bare fordi de vælger at bidrage det arbejde til open source-fællesskabet frem for at gøre det for en arbejdsgiver. Bare lige en hurtig note. 
%... Hm, min hjerne er gået lidt i stå, så nu prøver jeg bare at starte tænkeriet på tasterne og så se, hvor det bringer mig hen. (Så undskyld hvis denne paragraf bliver lidt ligegyldig, i hvert fald i starten.) Jeg var nået til et punkt i går, hvor jeg følte mig nogenlunde afklaret.. Web 2.1-fællesskabet kunne i høj grad bare køre på direkte donationer, og så kan skaber-fællesskabet altid på sigt gøre noget for, at opsætte, hvad der svarer lidt til betalingsmure, så længe dette bare ikke koster dem støtten (og donationer) fra de donerende brugere.. I princippet kunne nye skabere forsøge at lade de gamle skabere i stikken på det tidspunkt, og altså kun ligesom "opkræve" betalinger til dem selv, men det ville der være ret dårlig karma i, så jeg tror ikke de første skabere behøver at være bekymret for, at de får unfair behandling ift. nyere og mere kontinuert aktive skabere.. ..Ja, og det er kun rart for alle skabere at føle sig sikre på, at hvis deres værker f.eks. pludseligt får et kæmpe opsving i popularitet, så kan de også regne med at få et opsving i donationer, og det kan donor-fællesskabet altså sikre ved at holde fast i en god, fair ånd omkring donationerne, hvor man aldrig bevidst prøver (på oppertunistisk vis) at kante gamle skabere af vejen, bare fordi det på papiret ville kunne betale sig med helt kyniske "økonomiske" øjene.. ..Og dermed tror jeg altså på, at første skabere altså sagtens kan bidrage frit og så bare regne med, at skulle det blive nødvendigt, så vil fællesskabet på et tidspunkt opsætte nogle betalingsmure, så man tvinger en eventuel gruppe af totale freeloaders til i det mindste bare at give en smule, og at denne smule så ikke mindst også går til de tidlige skabere.. 
%Hm, og hvad med reklamer..? Jeg tænkte nemlig over i går aftes, at måske er der bare så mange penge i det (fra virksomhedernes side i første omgang), at det næsten vil kunne betale sig for folk at gå med til at se reklamer, hvis de bare er sikre på, at skaberne så får en anseelig del af pengene også i sidste ende (og så man dermed kan spare disse penge på donationer).. ..Hm, men så kan fællesskabet jo eventuelt bare indføre dette som en ting, som donationsforeningerne så kan vælge at slå til.. Ja, fint nok. (Håber næsten, at det ikke kan betale sig, for jeg kan selv virkeligt ikke særligt godt lide reklamer, men whatever..) 
%Og nu er jeg så som nævnt i tvivl om, hvor meget man skal fremføre token-idéen, og sågar om man overhovedet skal det.. ..Tja, jeg kan jo om ikke andet altid fremføre, den version jeg også skrev om i går eftermiddags, hvor det altså bare rent handler om samleri, og hvor folk måske muligvis kunne vælge at se det lidt som, at man holder æren, nærmest, for at have finansieret bidraget, når man holder tokenet.. ..Hm, og så tænkte jeg jo lige på, hvordan det så kan spille sammen med donationsforeningerne i netværket, men måske man simpelthen så kan sige, at skaberne/bidragsyderne kan vælge ikke at ville modtage donationer direkte, men \emph{i stedet} vil udstede tokens og så prøve at sælge dem. Donationsforeningerne bør så købe tokens, hvis deres pris (divideret med den procentdel, de repræsenterer) er lig (eller under), hvad foreningen ellers ville donere (for samme procentdel). Men ellers kan skaberen altså også sælge dem videre til private, hvorved donationsforeningerne herved med det samme kan se skylden (eller rettere den procentdel af skylden, der så er solgt til private) for betalt. ..Hm, men hvis først dette system kommer op og køre, så kan donationsforeningerne jo ligeså godt også bare købe tokens, når de donerer helt generelt.. ..Ja, og så vil foreningerne også herved kunne signalere med, hvad de så forventer af andre, og/eller hvor meget de ligesom eventuelt har flottet sig ved at donere.. ..Lyder umiddelbart ret nice.. Og så kan det altså være op til foreningerne, om de vil uddele tokens til medlemmerne løbende, eller om de først og fremmest bare bliver beholdt som et fælles eje i foreningerne. Hm, det lyder da meget godt, alt sammen.. 
%..Ah, men hvis donationsforeningerne så sælger dem videre igen, skal dette så ikke betyde, at de så giver afkald på retten til at sige, at de har doneret deres del (og til at blive betragtes som, at de har doneret deres del ikke mindst)..?!.. Hm.. ..Ah, og så kommer tokens'ne også direkte i spil, når fællesskabet begynder at sætte små mure op.. Hm, interessant.. ..Hm, nu tanker jeg så, om man så skulle gøre noget med, at en ejer af en token ligesom kan omskabe den, så det svarer til, at alle ejer den lige meget, men hvor vedkommende alligevel stadig beholder dem, som et token for, at vedkommende har udført den handling.. ..Ah, dette kan faktisk godt gå hen og blive en rigtig vigtig idé.. For så svarer det på en måde lidt til gængse NFTs, men bare mere alsidige, og hvor der rent faktisk bliver et (stærkt) fællesskab af skabere, der til sammen bliver i stand til at straffe dem, der bryder systemet, fordi de for det første kan lukke andre hjemmesider ned (således at folk skal prirate ting, hvis de vil have adgang til dem), og selvfølgelig kan de også lukke brugere ude, hvis de kan bevise, at pågældende har brudt aftalerne.. Hm, ikke fordi al dette bliver sindsygt vigtigt, men bare det, at det ligesom er muligt, det giver bare (måske) ligesom en større vægt til hele dette NFT-system.. Nå, jeg kan mærke, at jeg lige skal tænke lidt.. 
%..Hm, men jeg kan dog allerede se et muligt problem, for hvis en donationsforening transformerer nogle tokens, så skal de jo bare købe flere igen med det samme, og det virker da næsten lidt fjollet.. Hm.. 
%..Og det bliver også muligvis lidt fjollet, hvis folk i smug kan købe monopol på en ressource; det vil blive mere træls i længden, end det vil blive fedt.. ..Ja, så det bliver nu nok ikke helt den vej, jeg skal gå med idéen.. 
%Hm, kunne man mon gøre noget med, lige for at vende lidt tilbage, at foreninger kun ligesom overfører resten til fællesskabet, når de transformerer tokens.. ..Og det samme kunne man så sige om enkelte brugere.. ..Hm, nu kom jeg så til at tænke på, om alt overskydende ikke bare automatisk skal transformeres; for altid.. 
%Så spørgmålet er altså, netop om skaber-fællesskabet ikke altid automatisk skal betragte procenterne som fordelt (jævnt) ud på resten af fællesskabet, når en donationsforening (eller enkelt bruger, hvad der så typisk gælde) ejer flere procenter, end hvad de ligesom skylder (hvilket vel vil svare til, at det altså ejer flere procenter, end den procentdel, de selv udgør?.. eller måske rettere den procentdel af server-ressourcerne, de kræver.. (det kan jeg også lige tænke lidt over..)).. 
%Hm, men er det ikke nærmest bare det, så..? På den måde så kommer man altså heller ikke ud for det med, at nogen så kun købe NFTs og derved komme til at lukke for adgangen til en ressource fra andre. Det kan gå gå den anden vej, og man kan således godt købe NFTs og så lette adgangen til ressourcer for andre.. Virker altså umiddelbart som et ret cool system..!
%... Hm, og dog: Hvad er den store pointe med det så?.. Nu er det lige før, jeg bare dropper at nævne tokens helt, og lader det blive ved web-fællesskabet og tilhørende bruger-/donationsforeninger.. 
%(For tokens vil ikke være noget specielt værd i det system, jeg forestillede mig her, og alt hvad der hedder samleri.. Det kan man jo altid bare indføre i et lag oven over.. ..Skabere kan altså altid selv lave NFTs i et helt andet system; der er ingen grund til at prøve at blande dette sammen med donationsforeningerne og alt det.. Så nej, jeg blander nok NFTs helt udenom..)
%Så skal jeg så bare have en enkelt sektion imellem blockchain-angreb- og QED-sektionen, hvis man ser på mit nuværende udkast over hovedidéer, som jeg har i tankerne, som så bare egentligt ret kort skal handle om mit web 2.1/3.0-fællesskab her, men altså uden nogen tokens (direkte) invovleret i idéen..?
%..Ja, det er lige før, at det må blive sådan.. 
%(06.01.22) I går aftes kom jeg lidt frem til, at fællesskabet da godt kan have interne tokens, således at skabere og brugere (og investorer) tilsammen bruger en konsensus om, at skabere og investorer kan få lov at handle deres donationsrettigheder fra dem. Fællesskabet kan så bare bruge sine egne databaser til at holde styr på alle transaktioner, og bare løbende sørge for at tjekke, at der ikke bliver gemt modstridende transaktioner. ..Og så får man nemlig den situation, hvor det ikke kan betale sig ikke bare at afbetale skylden fra en ende af, for så længe der bare ikke er for stor en pukkel, så kan samfundet sagtens håndtere den pengeskabelse, der vil komme ved at alle nye skabere så med det samme kan handle med, hvad brugerfællesskaberne i sidste ende vil skylde dem.. Hm, man kunne i øvrigt også give lov til at dele tokens'ne op i tid, sådan så det er meget nemmere at regne med: Så skal man ikke forudsige, f.eks. hvor populær en ressource bliver i fremtiden.. Ja, det var da en god idé. Så skabere kan altså udstede tokens, som så f.eks. gælder for, hvad brugerfællesskabet har haft af gavn af ressourcen indtil udstedelsestidspunktet. Ja, fint. ..Og ja, ville dette system ikke netop gøre, så det ikke kan betale sig at lade gamle skabere (inkl. programmører), og investorer i øvrigt, i stikken? ..Hm, det eneste er, at.. Nå nej, donationsforeninger kan jo løbende estimere deres egen skyld, og dette vil så være næsten ligeså godt som at udbetale pengene, på nær at der lige vil være en risiko for, at systemet og/eller donationsforeningerne ikke vil holde.. ..Men hvis donationsforeningerne skulle ændre sig, jo, så kan skylds-beløbet godt ændre sig i princippet, men det vil stadigvæk bedst kunne betale sig for nye donationsforeninger at betale skylden af fra en ende af.. Hm, og jeg skal også huske, at de tidlige skabere vil have stemmeret, samt også IP-rettigheder.. Ja, så det kommer ikke til at komme på tale at lade tidlige bidragsydere i stikken. Og ja, hvis man så udlover donationer på forhånd, så kan der blive plads til, at folk kan få tidlig betaling, som dog vil have en risiko på sig, hvis skaberne kassérer den med det samme (og altså sælger sine donationsrettigheder videre). ..Hm, men skaberne kan jo endda godt sætte sig for ligesom at håndhæve, via potentielle mure, at brugerne gerne skal holde sig meget til de løfter, der blev gevet, også selvom der er kommet flere brugere og/eller donationsforeninger til siden; så må de nye tilkommere bare.. hm, men nye tilkommere.. hm.. Hm, hvis bare brugere kunne identificere sig selv som en del af det, at melde sig til donationsforeninger, så ville det blive så meget nemmere, for så kunne skylden jo bare følge en bruger, og skaberne kunne endda splitte tokens op ift. brugerundermængde.. Brugere kunne så sige, at de skylder så og så meget.. men betale de tidlige skabere først.. Hm, det hele er vist lige blevet en anelse rodet nu, så lad mig lige tænke lidt.. ..For hvis bare der først er en god ramme omkring skabernes IP-rettigheder.. ..Hov, hvad med et system a la mit brugergruppe-vægtfordelingstoken-system, men bare for programmør-skaberne..?(..!..?) Og ja, sikkert også for de andre skabere..? ..Hm, det var da egentligt virkeligt en interessant idé umiddelbart..!!... 
%... Hm, kunne man mon så gøre det, så at der stemmes om løbende, hvor meget skal deles ud af stemme-rettighederne (løbende), og så kan skabere i øvrigt bare melde sig til del-foreninger til at dele deres rettigheds-token ud..(?) 
%(Og så er tanken, at det hele bare skal sigte mod, at stemmerettigheds-token-fordelingen kommer til i sidste ende at svare meget til, hvad det ville være, hvis skabere fik stemmeret ud fra, hvor meget de har fået doneret til sig. Så det skal ligesom bare være målet..)
%... Hm, det svarer vel lidt til en kundedrevet virksomhed, bare mere med IP-rettigheder i stedet for aktier, og hvor "investorerne" så i bund og grund investerer med IP-rettigheder og ikke med penge (og hvor de så altså oplagt selv kan være dem, der har skabt IP-bidraget). Og så kan, hvad der svarer til "aktierne," så heller ikke opdeles ligeså nemt, men her kan man dog bruge nævnte system i stedet, hvor "medejere" kan (og skal) sælge deres "aktier" lidt videre løbende til nye skabere(/investorer).. Hm, det lyder rigtigt nok, men lad mig lige sammenligne de to idéer lidt grundigere... 
%(Indskudt: I princippet kunne man jo også bare starte med donationsforeninger, og så udvikle det hele derfra, men det er ikke så realistisk, for det bliver ikke så let at få den brede befolkning med på en idé, for at idéen er bevist: Det er jo meget mere realistisk, at kun nogle få vil se potentialet til at starte med, og at en delmængde af disse så kan vælge at investere i at iværksætte projektet.)
%..Hm, man skal så lige huske at beslutte, hvad afkastet i "aktierne" skal afhænge af for en forskrift, hvis man også vil bruge den del af kd.v.-idéen.. Og jeg skal i øvrigt også tænke på, hvordan man sikre sig, at det ender ud med, at kunderne/brugerne driver det og/eller at skabere får stemmeret alt efter donationer.. ..Hm, men det kommer vel ret naturligt, hvis man har et princip om, at "aktierne" skal afgives løbende til nye, i dette tilfælde ikke kunder/investorer, men til nye skabere/investorer.. ..Hm ja, er det egentligt ikke ret meget det?.. Nå nej, det bliver jo også svært at definere en forskrift for afkastet, når dette netop ikke bare handler om at finde en faktor ift. det investerede beløb.. Så her skal man altså også lige være en tand klogere.. ..Tja, men kan det nu ikke bare på en måde være et krav en en del af de samlede donationer.. Hm, eller måske netop en del \emph{af en del} af den samlede mængde af donationer?.. Umiddelbart interessant.. ..Hm, men nu skal jeg lige tænke på, at indtjeningen netop godt kan være mere end fra bare donationer.. Men ja, så er spørgsmålet stadigvæk: Hvordan får man det nu til at gå imod, at skaberne får stemmeret ud fra donationer.. Hov, men er det egentligt ikke i det hele taget et problem med den idé, at skaberne ligesom kan købes ud af deres egne rettighder, netop ved \emph{ikke} at få nogen betaling?.. 
%..Hm, det kan da egentligt køre ret meget på samme måde som en "kundedrevet virksomhed," hvor "aktierne" så bare er et krav på en del af indtjeningen for den periode, de gælder i (og altså også ud fra en forskrift), og i øvrigt kan brugernes betalinger så ligeledes ses som en investering på samme måde som for kd.-virksomhederne..?(..!..) 
%..Og jeg kan bare nævne den del med at betragte brugerne/kunderne.. / de donerende brugere.. på lige fod med investorer som en ekstra muligt punkt for idéen.. ..Hm, og skulle man netop så bruge dette til at få bedre plads til ikke-betalende/donerende brugere; fordi disse så bare heller ikke får stemme- og IP-rettigheder herved.. ..Det var da umiddelbart ikke en dårlig idé heller.. 
%(Og det er nemlig vigtigt at bruger-investorerne får vished, netop ved at trække på idéerne fra kd.v.'erne, om at det, de investerer i, bliver godt og frit i sidste ende (og at der ikke bare bliver nogle fede (og grådige) katte på toppen i sidste ende).)
%(07.01.22) Jeg kom også i tanke om i går aftes, at skaberforeningerne jo bare kan starte uden licenser for, hvordan IP-rettighederne deles i gruppen (og ikke deles; en skaber skal gerne beholde alle rettigheder til ressourcen uden for hjemmeside-netværket i så højt omfang, som det kan lade sig gøre), men bare starte med, at alle skabere har licens til sit eget, og så regner man bare med, at fællesskabet kan blive enige om en licensen, som så ligesom bliver rygraden for foreningen/virksomheden, når man har fået juridisk kyndige til at hjælpe med at udforme denne. Og noget andet, jeg kom til at tænke på, er, at skaberforeningerne jo også kan indføre et bagud-belønning-koncept, ligeså vel som brugerforeningerne kunne det. Så dette er der altså stadig mulighed for. Og desuden kan man også gøre, som jeg lagde op til her sidst, da jeg skrev om kundedrevne virksomheder, nemlig at man kan lade kunderne/brugerne få mere stemmeret i tilgift med deres aktier, end hvis det hele var en-til-en. For således kan man jo netop tiltrække flere nye kunder/brugere. 
%Okay, så nu bør jeg vel egentligt bare lige nævne min nye version af donationskæde-idéen under Andre opfølgende noter, og så bør jeg vel bare gå i gang med dispositionsbrainstorm for web 2.1/3.0-sektionen.. Så denne sektion udkommenterer jeg enten, eller skriver "(old)" foran som i de nedenstående udgåede sektioner.. Ok. Og hvis jeg får nogle andre opfølgende tanker og idéer, så kan det være, at jeg vender tilbage her og skriver dem, men hvis de kommer i forbindelse med dispositionsarbejdet (eller bare efter at jeg er ligesom er kommet godt i gang der), så skriver jeg det jo bare der. Fint. ..Never mind, for Web 2.1-sektionen er jo lige nedenfor her, så jeg fortsætter bare udelukkende der.
%(14.01.22) I går kom jeg frem til, at det nok er vigtigt at virksomheden kan deles op i forskellige services, og at folk så måske kun får magt over den afdeling, de er kunder hos. I går aftes og nat kom jeg så frem til, at hvis vi betragter alle services i en graf, hvor alle blade er en service som kunderne køber, og alle indre knuder er services, som andre services bygger på, hvor der kræves arbejde og/eller andre omkostninger for at vedligeholde denne service, så skal det nok være sådan, at aktionærerne i første omgang får sat en forskrift for, hvor meget de forskellige børne services hver især skal betale for deres forbrug. Kunde-investorers kommende aktier skal så ikke kunne bruges direkte til at ændre forholdet imellem, hvor meget børne-servicerne hver især skal betale til at opretholde forælder-servicen. Men de skal kunne bruges til at stemme om en sluttelig faktor for, hvor meget der skal gå til forælder-servicen. Når de indledende aktionærer er henfaldet, så skal det nok bare, er jeg lidt lige kommet frem til, kun være ikke-kunde-investorerne, der får lov at bestemme betalings forholdet imellem barne-servicerne. Da aktionærerne i starten også vil være direkte investorer, så bliver reglen jo simpel: Kun direkte investorer bestemmer forholdet.. Hm, lad mig lige tænke lidt.. ..Hm, måske holder det.. Så når kurverne er henfaldet til, at kunderne står for hovedparten af at betale lønninger og andre omkostninger, men ikke meget mere end det, så kan de forskellige kundegrupper jo vælge at investere lidt ekstra alligevel og så bruge dette til at flytte forholdet. Men hvis dette så resultere i lavere omsætning, så vil de jo miste penge på det.. Og i takt med at den anden gruppe får betalt mere og mere til forælder-servicen, hvis forholdet bliver ændret, så kan får de til gengæld også mere magt over den sluttelige faktor.. hm, og tanken her var så, at hvis denne anden gruppe ikke behøver servicen helt så meget, så kan de jo bare nedjustere den sluttelige faktor, og så går hele øvelsen måske mest ud over første gruppe.. ..Og ja, plus de mister penge, hvis omsætningen går ned.. ..Hm, mister de egentligt ikke penge alligevel, hvis omsætningen går ned, for de har jo alligevel et depositum, når de er faste kunder..(?) ...







%Jeg skal lige huske at forklare, hvordan donationsforeninger skal forholde sig, hvis de fastlår én værdi for en token, der er fordelt på flere hænder (og med forskellige procentdele på sig), men hvor andre allerede har doneret til en undermængde af token-holderne. Foreningerne skal nemlig i så fald bare udregne, hvad de selv synes, bidraget er værd, og så kun betale, hvis holderne ikke allerede har fået dette beløb eller mere. 

%Hm, anonymitet er måske også meget smart at have i systemet (hvorfor det også kan være smart at implementere over en blockchain)..

%Kunne være fedt, hvis det udbredte sig til energi-sektoren også især.. 

%I starten kan det jo nærmest ikke betale sig ikke at love noget.. 


\section{Web 2.1}
...

%Kopieret nede fra i bunden af Web ideas(1)-sektionen (og fra d. 29/12):
%"Som jeg tænkte på i går aftes, så kunne en simpel version af web 2.1 vel bare være et åbent fællesskab af hjemmesider og skabere (og interesserede, I guess..), hvor hjemmesider og skabere har en aftale om, at hjemmesiderne må og skal dele al data imellem hinanden pr.\ request. Fællesskabet skal som nævnt være åbent, så hvis en ny hjemmeside kan sige, at den vil følge samme principper, så skal den altså optages i fællesskabet alt andet end lige. Hjemmesiderne skal også bare vare open source, og det forventes derfor ikke, at de skal hente deres profit ved at designe hjemmesiderne godt. Det skal brugerne alligevel gerne ende med i høj grad at stå for, hvis målet er web 2.1, hvad det gerne skal være. De skal i stedet bare tjene på fees fra at stille deres servere til rådighed, og så ellers bare donationer og donationstoken-handler ved både at opstarte hjemmesiden og også ved at stille servere til rådighed (hvis nu fees'ne ikke dækker omkostninerne helt). Og skaberne skal også bare forvente at få løn via donationer og donations-token-handler (eller -salg rettere). Og hvad med brugernes data? Nå jo, det skal forresten også være en del af konceptet bag fællesskabet, nemlig at brugernes data f.eks.\ også må og skal deles pr.\ request (med visse undtagelser, som jeg kan vende tilbage til). Og her er det så bare, at man skal bruge mit anonymitets-system, hvor man kører det hele over et anonymt VPN, og hvor hver bruger sørger for at have (typisk mindst tre --- hhv. til person-afslørende, til potentielt pinligt for f.eks. folk, der måtte kigge brugeren over skulderen, og så til normalt, ikke-pinligt og ikke-personafslørende brug) forskellige konti/profiler, så dataen er adskildt, således at ingen kan opsnuse de data-forbindelser, de ikke må se. (Og hvis man så f.eks. gerne vil gøre brug af ML fra den normale profil/konto til søgning på person-relevante og/eller "pinlige" ting, så kan man også gøre nogle ting, for at folk skal kunne dette. Man kan nemlig lave nogle protokoller, hvor folk kan knytte de ikke-"normale" (hvis vi kalder dem det) profiler/konti til en hel gruppe af "normale" konti, hvori de selv indgår.) Så brugerne kan nyde fuld anonymitet samtidigt med at de kan få ML-funktionaliter stort set ad libitum, og skaberne og serverne kan alle sammen få løn og betaling, endda uden at skulle genere brugerne med clickbait og/eller reklamer.(!)   
%Nå ja, og eventuelle web-udviklere, der ikke uploader arbejde via web 2.1-vejen, men som måske er med i en mere tidlig fase, som bl.a. bygger systemet, der gør web 2.1-funktionaliterne mulige (eller som bygger første udgave af hjemmesiden i det hele taget), de kan så også få betaling via donationer og donations-token-salg."


%(07.01.22) Man skal kun lige bladre et par scrolls op fra sektions-headeren og til, hvor jeg ligesom slap sidst med mine noter omkring Web 2.1/3.0. Jeg har også lige kopieret mine noter fra d. 29/12 ind lige ovenfor, men disse passer nu ikke helt, for nu vil jeg jo have (i hvert fald som tingene ser ud for mig lige nu), at systemet alligevel skal operere lidt mere som en virksomhed (og nærmere bestemt ligesom en "kundedrevet" en af slagsen).. ..Jep. Okay, så lad mig prøve at brainstorme over disp'en. 

%Disp.-brain.:
%Jeg skal jo gerne starte mere med web 2.1-konceptet i sig selv, og så ikke tænke så meget på den økonomiske side af det. 
%Jeg kunne jo måske bare starte direkte med, "hvad med at vi lader web 2.0-hjemmesiders deign selv blive en del af, hvad brugerne kan bidrage med, desuden også inkl. feed- og filter-algoritmer osv.?." 
%Skal jeg så ikke med det samme gå ind i, at dette så meget naturligt kunne gøres ved, at brugerne får forskellige programmeringslag at arbejde med, og hvordan man altså kan give flere og flere muligheder løbende.?.. Og så kunne jeg måske også hurtigt komme ind på, at hjemmesiden kunne bevæge sig over på apps (mobile og desktop) også.. Hm, eller bliver dette for teknisk et sted at starte..? ..Lad mig bare fortsætte med denne dispositionsform for nu.. 
%..Hm, men hvis jeg nu ser på den økonomsike del af det, så kunne dette næsten ligeså godt bruges for bare en web 2.\emph{0}-side.. 
%Tja, men jo bedre (og jo flere varianter), der er på designet og på funktionaliteterne/mulighederne, jo mere er ressourcemængden også værd.. Så det hænger vel lidt sammen, det hele.. (Jeg kan mærke, at det går lidt langsomt for mig i dag..) 
%... Men ja, jeg skal jo nok bare starte med.. Hm ja, med konceptet om en mere fri web 2.x-side, og specifikt som inkluderer web 2.1-konceptet.. ..Ej, hvor går det bare langsomt i dag af en eller anden grund.. Det bliver ellers dejligt at få styr på, det her, for når jeg får det, nærmest bare skrivning derfra, som det ser ud nu. (For dette er jo mere end bare skrivning: Der kan jo potentielt ske lidt det samme, som der lige gjorde for "donationskæde"-sektionen..) 
%...Hm, hvad med egentligt at gemme selve det med "web 2.1" til senere i sektionen, og så bare starte med at fokusere på idéen om et frit og åbent fællesskab, som dog kan formå at sætte betalingsmure op for noget af indholdet, når tiden er til det..? 

%(08.01.22) I går aftes kom jeg på, at "aktierne" jo også kan deles ud til pengeinvestorer, og så kommer virksomheden til at være nærmest helt lig min "kundedrevne virksomhed," bare hvor aktieindehaverne også kan vælge at give aktier direkte for IP-rettigheder, og hvor det hele altså ikke behøves at måles i penge.. Hm, og dette kræver jo sådan set bare, at man kan sælge sine aktier.. medmindre der jo også skal være en regel om, at man \emph{skal} "sælge" (/uddele) lidt af sine aktier løbende.. ..Men ja, dette kan så enten være en del af det grundlæggende lag, eller noget man aftaler efterfølgende. Uanset hvad vil det sikkert være meget smart, hvis man kan aftale, hvor mange aktier skal gives ud hver måned eller hvert kvartal.. Og her snakker vi altså aktier, der gives ikke for penge, men for IP-rettigheder.. og ja, nu kan jeg jo godt se, at det hele jo bliver nemmere, hvis man bare regner det hele i penge alligevel. For selvom det måske kan betale sig at købe IP-rettigheder for aktier, især i starten, så kan dette jo sangtens bare omregnes til penge, hvis aktierne handles på det frie marked. ..Ja. ..Aktionærerne kan så bare med fordel opsætte en model for, hvorhen aktierne.. skal gå.. Hm, vent nu lige lidt.. ..Nå jo. Jeg tænkte lige, at man så bare kunne snyde og give dem til sig selv, men der skal jo stadig være et fundamentalt princip om, at alt i princippet kun koster, hvad det har kostet at producere, på nær at priserne godt kan være højere, men så skal de overskydende penge altså betragtes som en investering. Og "aktierne" gør så altså ikke krav på en del af overskudet (og måske heller ikke på boet, men det skal jeg måske faktisk lige tænke lidt mere over..), men i stedet så gør de krav på et vist afkast, der kan afhænge af virksomhedens udvikling. Hvis virksomheden nu f.eks. så står stille i lang tid nok, så vil dette bare betyde, at kunderne efter noget tid vil få de ekstra betalte penge tilbage igen i form af afkast. Men hvis der skulle blive brug for pengene, eller hvis der stadig er nogle gamle investorer, der har krav på en bid af afkastet, så kan kunderne altså ikke forvente, at afkastet kommer op og matche, hvad de har betalt. ..Tja, eller rettere, det handler jo egentligt ikke så meget om, om der er brug for pengene eller ej, for hvis der er brug for pengene til at udvide, så skal aktierne jo gerne holde i så lang tid, at man når at få afkast ud fra, hvor meget det så lykkedes at vokse med. Så ja, aktierne skal altså meget gerne holde over lang tid. ..Og "størrelsen," når vi snakker om at "vokse," det handler vel så om den samlede omsætning.. ..Hvilket vel i øvrigt også så kan måles i udbetalt løn.. ..Og her kommer "produktionsomkostninger" så også til at give mening for en web 2.1-virksomhed, for dette bliver så den udbetalte løn til skabere og programmører (plus det løse --- bl.a. jurister). Og ja, i starten kan virksomheden godt tiltrække arbejde på normal vis i princippet, men især når nu vi snakker web 2.x, så vil det sikkert give mening hurtigt at oprette en "bagudbetalings"-model i virksomheden og følge den nøje (og løbende stemme om ændringer). Og hvis der opstår noget uforudset, som modellen ikke kan håndtere, så må man bare have en klausul om, at aktionærerne også sætter nogle penge til side til særlige belønninger, især i starten, hvor det så bare forventes af dem, at de kun giver af disse, når der er et særligt behov, og at de ikke bare kanaliserer dem tilbage til dem selv.. ..Men ja, alt dette er i princippet i et lag over det grundlæggende. I det grundlæggende lag skal aktierne bare give en stemmeret til.. Hm, til hvordan virksomheden skal betale for arbejdskraft og produktionsmidler-ressourcer, men måske man lige bør overveje, hvordan man sikre denne magt.. Umidddelbart tænker jeg, at det bare kan gøres ved, at man får en stemme både til en model, og til ansættelser/fyringer af den ledelse, der skal udføre modellen. ..Hm.. Og hvis man så er gerne vil have en stemmeproces for modellen, der ikke bare er majoritetsvælde, men måske er mere nuanceret end det, så kan man måske også binde ledelsen juridisk til at følge modellen. For hvis en minoritet så får stemt noget igennem i modellen, og ledelsen ikke følger det, så kan minoriteten så sagsøge. Ja, så den mulighed er der også. Men majoritetsvælde kan nu sikkert også gå langt hen ad vejen.. Ok, er der ellers andet, jeg skal tænke over..? ..Så ja, afkastene kunne altså meget vel afhænge af, hvor meget virksomheden vokser målt i lønninger over en længere periode fra at aktien udstedtes. Hm ja, og folk kan handle med aktier, men det ændrer så ikke, hvilken periode de gælder for. Og de udstedes altså i forbindelse med (formelle) investeringer samt med alle handler, hvor en kunde/investor betaler for et produkt eller en service. 
%*Hov, det bliver da egentligt lidt svært for en kundedrevet virksomhed at håndtere salg af services, hvor omkostningerne ikke ligger så meget i at.. give dem.. Hm, to sek.. ..Hm ja, kommer man ikke potentielt lidt i klemme ved ikke at kunne udnytte efterspørgsel, eller er dette mere bare en feature og ikke en bug..?.. Lad mig se.. ..Ah, men virksomheden er jo stadig fri til at sætte alle priser, også selvom.. en vis del af dem regnes for investeringer.. hm, men problemet er lidt: Hvilken del, hvis der er tale om en service? Og ja, især altså hvis denne service ikke kræver arbejdstimer direkte, og derfor ikke direkte kræver løn.. ..Hm, og det er ikke bare noget med, at aktionærerne (som jo er en dynamisk mængde i øvrigt, bare så det lige er på plads (for stemmeretten skal nemlig også falde i løbet af aktiens periode og forsvinde til sidst)) så bare beslutter, hvordan alle omkostningerne skal fordeles ud på priserne for diverse services..? Hm, men det naturlige spørgsmål vil jo så med det samme blive, hvorfor ikke så bare gøre investering og køb til to forskellige processer; hvorfor.. hm.. Fordi man \emph{kan}? (Og fordi aktionærerne nemlig \emph{kan} drage nytte af at tvinge kunder til at investere..?) Hm, kan det hele på en måde ses som en speciel form for non-profit organisation (men dog ikke rigtigt, for aktionærerne \emph{får} afkast i sigte), hvor priserne og produkterne bare kan komme med et krav om investering (og desuden også vil have en del af sig, som går til at betale omkostninger, man hvor virksomhedens ledelse/aktionærer dog selv kan vælge, hvordan omkostningerne skal fordeles på diverse services og produkter i første omgang (og altså inden der så kommer en ekstra (investerings-)pris oveni))..? ..Ja.. Hm, og i praksis vil dette så bare sige, ift. hvad jeg havde før, at det nu skal understreges, at aktionærerne/ledelsen har magt til at fordele deres omkostninger ud på priserne for deres produkter og services, som de har lyst, hvilket så desuden også toppes af med et "krav om investering," som disse også er frie til at sætte, som de vil. Ok.. ..Hm, nu fik jeg lige en ny tanke: Hvordan undgår man i øvrigt, at en "generation" af aktionærer ikke bliver pyramidespils-grådige..? ..Nå ja, afkastet regnes jo altid ud fra lønnen.. nå ja, men netop stadigvæk: Hvordan sikrer man sig, at de ikke kræver for mange penge af kunderne for så at betale for meget i løn til.. he.. Lidt den omvendte verden.. Hm, kan dette virkeligt blive et problem?.. For hvis priserne stiger, så vil kunderne jo bare.. hm, måske kan det godt blive et problem, for måske kan de kanalisere lønnen over til sig selv.. ..Ah jo. Det er en korruptionsvektor, hvis aktionærerne kan komme til at hyre dem selv eller nogen, der har gæld til dem (..måske med renter, der gør gælden konstant..), således at de nærmest indirekte hyrer dem selv. ..Ja, og man kan jo altid konstruere sådan en skræddersyet "gæld" til formålet, så man skal jo nok passe meget på her.. ..Ja, for konceptet virker kun, hvis aktionærerne ikke kan finde en måde at komme uden om, at al overskud skal komme i form af en formel investering.. Hm.. ..Hm, ah, men der er jo det ved det, at aktionærmængden og kundemængden efter noget tid vil blive mere og mere ens, og på den måde får man altså ikke den situation, hvor en generation af aktionærer vil prøve at snyde kunderne. Nej, så derfor er der faktisk ingen grund til at bekymre sig om sådan en situation; så vil det kun være i ekstreme tilfælde, at aktionærerne ville have lyst til at lave beskidte kneb i den halvtidlige fase af virksomheden og så slippe af sted med det (for kundebasen kan jo altid bare skifte, hvis de opsnapper korruption). Ok.

%Disp.-brain. igen:
%Jeg overvejer lidt at starte med at motivere ved simpelthen at komme med noget kritik af gængse web 2.0-sider.. ..Og denne kritik, måske inden den bliver for tyk, kunne så måske gå direkte over i at komme med forestillinger om en mere fri og åben side (eller et netværk af sider).. ..Hm, og kunne jeg så pointere, at det ville være smart, hvis man kunne bringe selve designet/programmeringen ind som noget mere.. socialt-agtigt.. på siden, hvor programmører/designere altså så kan få følgere og likes.. Hm.. 
%(Indskudt: Hører ikke til her, men jeg tror i øvrigt ikke jeg vil lave ét første dokument med disse udgivelser, som jeg ellers lidt havde tænkt, men deler dem nok bare op i flere dokumenter fra starten af.. (Og så behøver de ingen gang at være skrevet i samme tone eller til samme niveau af udførlighed.)) 
%..Hm, men det med følgere og likes, det bliver jo først \emph{rigtigt} interessant, når man får forklaret, at der bliver penge i det også, for (web 2.1-)programmørerne.. ..Men her kunne jeg eksempelvis bare nævne den tanke, og så netop med det samme fortsætte med: "men det bliver først rigtigt interessant, hvis vi kan finde en udgave af netværket, hvor der er penge i det, og hvor programmørerne og designere kan få løn efter, hvor populært deres arbejde er.".. Hm ja, måske kunne dette faktisk give mening.. Og så kunne man måske med det samme også påpege, at dette måske kunne lade sig gøre bare ud fra donationer.. ..Hm, skulle jeg så lave en hel lille afstikker ved at forklare kort om donationsforeninger også?.. ..Hm ja, det skulle man da næsten.. ..Og så kunne jeg altså gå videre til at sige, at jeg også har.. hm, eller jeg kunne starte med at sige, at problemet kan blive at få udbredte nok donationsforeninger, og samtidigt er det måske ikke ligeså spændende, hvis man så er en af de første programmører.. Hm.. ...Hm ja, måske kunne dette give mening. Så kunne jeg også sige, at første donorer måske i virkeligheden også hellere vil fungere som en slags investorer og altså få noget igen, hvis det hele letter og flyver til vejrs.. Og så kunne jeg måske sige, "ja, og her kan man så finde på rigtigt mange forskellige ting at gøre," og så fortsætte med: "Jeg har fundet frem til en virksomhedsmodel, som jeg synes kunne være rigtig interessant at undersøge".. 
%Hm, det hele bliver jo lidt løst med denne form for disposition, men måske den kunne være meget god..(?) ..Ja, det lyder faktisk ikke helt dumt.. Og jeg behøver vel ikke at komme så meget ind på alt muligt om donationsforeninger, for det giver vel lidt sig selv at belønne bagud, når vi netop snakker om donationer, og så kan det jo siges meget kort. 
%Hm, og når vi får hul på min virksomhedsmodel, hvad skal jeg så igennem der?.. Hm, har jeg egentligt i det hele taget fået behandlet ejer- (og konkurs-)spørgsmålet i denne omgang? Det skal jo gælde, at det skal være så godt som umuligt for aktionærerne at fremtvinge en konkurs og løbe med boet (inkl. de delvise (for skaberne skal jo også altid bevare sine egne rettigheder så meget som muligt) IP-rettigheder), ikke?.. Hm, men hvem.. ah.. Skal man ikke bare sige, at alle værdier skal gå til offentligheden under en konkurs..!..? For så vil aktionærerne jo altid bare have ét formål (især hvis vi ser bort fra potentiel økonomisk korrelation med arbejderne (og vi behøver ikke at se bort fra korrelation med kunderne, for dette er jo faktisk meningen skal komme med tiden)), nemlig at få et godt afkast som følge af deres aktier og tilhørende perioder og forskrifter.. ..Ja, fedt nok. Så al ejendom går ligesom til en fond, hvor der gælder, at den skal uddeles.. f.eks. til hver af.. nej.. Hm.. Okay, IP-rettigheder er nemme at give til offentligheden, men hvad med andre værdier, for virksomheden må jo meget gerne kunne være international. ..Hm, men kan fonden ikke bare fortsætte.. indtil den ikke har flere værdier og ikke kan finde flere investorer..? Kunne fonden så være juridisk bundet til altid at forsøge at.. hm tjo, eller også kunne man bare fordele værdierne i henhold til alle tidligere investorer, hvor man \emph{ikke} tager højde for, at stemmeretten er faldet eller forsvundet helt for aktierne.. Ja, dette ville også være en oplagt mulighed.. Men hvis det kan lade sig gøre, så kan den underliggende fond måske også bare være juridisk bundet til at prøve at opstarte en ny virksomhed af samme type, skulle aktionærerne blive tvunget til.. Hm, men hvorfor skulle de så nogensinde det.. Hm, hvis de får gæld, men skal det ikke være imod virksomhedens politik at få gæld..? ..Jo.. Og hvis den skulle blive lukket ned af andre udefrakommende kræfter, ja så kunne man måske netop bare sørge for, at det underliggende ejerskab, som ellers ikke kommer i spil, bare fordeles ud fra, hvad investorerne, inkl. kunderne self., gennem tiden har betalt (og uagtet om deres aktie sidenhen er udløbet ift. dens normale funktion).. Hm, men det kan da så blive lidt farligt, hvis folk så kan sælge disse, umiddelbart rimeligt værdiløse (i praksis), udgåede aktier, og at en køber så kan købe sig til en stor interesse i, at virksomheden går "konkurs" (hvad end dette så kun kan ske af politiske veje eller ej).. ..Hm nej, så er det næsten bedre, at elle værdier i så fald bliver doneret til en ny fond, og hvor man så laver en prioriteringsliste, således at man starter med en fond magen til, men hvis dette af politiske grunde ikke kan lade sig gøre, fordi den så vil være ulovlig, så går man så bare et trin ned i prioriteringsrækkefølgen (osv.). Ja, det lyder meget fornuftigt. 

%Disp.-arbejde (08.01.22):
%Så kunne man altså skrive noget a la følgende?
%Jeg er meget interesseret i fremtidens internet.. og Web 3.0.. Der er flere forskellige visioner omkring, hvad Web 3.0 bliver helt præcist, og jeg har også nogle af mine egne, men overordnet set så handler det om bla bla.. ..Hm, måske kunne jeg faktisk godt starte sådan her med at fortælle om Web 3.0-visioner, og så om mine visioner specifikt, hvor jeg også kan inkludere web 2.1 lidt i det. Og dette kunne altså bare lige gøres som en hurtig liste over ting.. Og hvad så med kritikken af gængse Web 2.0-sider..? ..Tja, man kunne jo altid bare indsætte en sektion med det (lige efter..).. 
%...Ah ja, nu tror jeg næsten, jeg ved det. Jo, jeg kunne godt starte med at beskrive de kendte Web 3.0-visioner (som jeg jo lige kan finde nogle kilder på), og så måske undervejs eventuelt tilføje nogle små bemærkninger, som f.eks. at bruge linked data bare til browsing også, og kan i sidste ende også sørge for at få tilføjet de ting, der eventuelt mangler i de kilder, jeg kan finde, så mine Web 2.1-tanker også kommer til at fremstå klart der allerede. *(Og jeg vil i øvrigt også istemme mig, at ML-mulighederne også bare bliver glorious(!) for Web 3.0.) Og ja, når man så netop når til at snakke om mere frie algoritmer og sideindretninger, så kan jeg så tangere med en hurtigt nævnt kritik af gængse web 2.0-sider (..eller kun halv-hurtigt..). Og så vil jeg vil allerede have introduceret emnet og motiveret de efterfølgende idéer (omkring at opstarte netværk til at tage det næste skridt), eller hvad..? ..Tja, og om ikke andet, så kan jeg jo bare indlede det så følgende afsnit med at motivere det en tand dybere.. ..(Eller jeg kan slutte motivationen af med denne dybere tand, hvis der er en..) ..Og skulle jeg så stadig lige starte med at fedte lidt rundt omkring at opstarte det med donationsforeninger bare..? ..Tja tjo, det ville måske ikke være nogen hel dum idé lige at beskrive formålet helt generelt med projektet.. Ja, nej. ..Hm, og måske kan jeg så bare starte med at lave en antagelse om, at det kan køre bare på donationer, for så at beskrive idéen mere fuldt --- så kan jeg jo nemt nemlig også lige komme ind på bagud-belønning, hvilket som nævnt vil være oplagt, når vi alligevel snakker donationer. ..Og ja, så kan jeg så indlede (til?..) næste afsnit med at påpege, at der måske vil være nogle mangler ved den strategi, og nogle ting som diverse parter måske kunne ønske sig, som ikke nødvendigvis bliver opfyldt med donationsstrategien/-antagelsen.. Så kan jeg sige, at man sikkert kan finde på mange ting, men jeg er selv kommet frem til én løsning, som jeg synes er rigtig interessant, og som jeg gerne vil fremføre her.. ..Ja, det må da næsten blive min (overordnede) dispositions..:) 


%Paragraf-disp.*(-arbejde) (09.01.22):
%Jeg tror lige, jeg starter med at lave nogle placeholder-paragrafer for første sektion især. Og så kan jeg lave dem mere grundigt, når jeg lige finder nogle kilder.
%
%I have become very interested in the future of the web over the recent years and want to take part in the research of how we get to the next step. 
%The first big step happend when.. Hm, eller kan jeg ikke bare henvise til en anden kilde?..
%..Ja, lad os sige, at jeg har foklaret de tidligere steps..
%The visions for the future web, Web 3.0, include having more linked data.. Hm ej, og det er nok også bedst, hvis jeg bare lader kilderne forklare disse ting, for så sparer jeg en del arbejde, og jeg kan helt sikkert finde kilder tilgængelige på internettet, der kan sige det bedre, end jeg kan.
%Så lad mig sige, at jeg har refereret til dem og lige opsummeret lynhurtigt, hvad de fortæller.
%Okay så:
%
%I have become very interested in the future of the web over the recent years and want to take part in the research of how we get to the next step. *(Og det har selvfølgelig ikke været meningen, at jeg ville formulere det, som jeg skrev det her; jeg gad ikke at overveje formuleringerne så grundigt her; bedre bare at få nedfældet noget.)
%It is currently considered to be in the Web 2.0, which differ from the original web by allowing users and not just the web developers to provide a big part of the content people engage with. So in other words, with Web 1.0 we mainly logged on to read the HTML documents web developers had put up, but with Web 2.0 we mainly log on to read what other users have uploaded. There have then been severeal vision for what the next big step, i.e.\ Web 3.0, will look like. 
%I will also link to some other ressources that can explain the visions better than me, but overall the vision includes a more \emp{semantic web} as it is called. This was a very early vision by Tim Berners-Lee when... It means... linked data to... For a better explaination see... I will would also add to the vision described here, that semantically linked data also can be used to make browsing, not just searching, easier... 
%As so and so says, the vision has since then also gone on to include... a more decentralized.. community... 
%Comments about ML.. *(Nå nej, dette skal jo bare komme i en sen sektion.):
%The source also speaks about ML... I can only agree with this... And I would add a specific reason why the ML possibilities will be glorious with Web 3.0. Imagine a web where all services happens on an anonymous VPN and where all user accounts are essentially public. At first this might seem to severely limit the user's experience of anonymity but the idea is then for each user to divide their activity into different accounts. In particular each user could have a accounts for activity related to their own private person. Whenever looking up data about their local area, their local community, their friend network, their occupation and so on, the users should use seperate account for such activity. For anything that might be considered embarrassing by others, the users should also preferably use seperate accounts. This is especaily so users might use their main anonymous account freely even when there happens to be on-lookers, i.e.\ if friends, family, coworkers or random strangers can look over their shoulders. And for everything else, users can use their main anonymous account, whose data does not reveal their identity but whose content is still not comprimising enough that onlookers need to go away. And if a user wants semi-embarrassing data to be part of their main account, they can always just create a copy with some data removed which they can log on to in certain companies. 
%The idea here is that data is only sensitive if it is connected. Even the most potentially embarressing data is fine for an account if that account is deviod of data revealing the users identity, and even the most identity-revealing account is fine, as long at other data on is strictly professional.
%Of cource everyone can make a slip and search or click on the wrong things with the wrong account. But the decentralized network I have in mind should not be as public and decentralized ass for instance blockchain where everyone can see all the data. It should only be decentralized in the way no party (or group of parties) has exclusive right to set up servers. But the server networks should still be run by entities who have the ability to promise their users to be able to erase any unwanted data --- and also to not make any data public before the user has a chance to review new updates to it. ..Ah, dette er sådan set rigtigt godt, men det skal jo selvfølgelig bare komme i en senere sektion. I starten kan jeg bare skrive noget a la:
%The source also speaks about enhanced ML possibilities and I agree that these will be great on Web 3.0. I will expain why this is in a later section..
%
%Design and algorithms as part of Web 2.0 (or 2.1..)..:
%I will also underline a vision that i see for the future web, which is not really mentioned in the given references.. (..Hm, jeg bør næsten lige søge under Web 2.0-definitioner/-artikler og se, om jeg kan finde noget om mine "Web 2.1-visioner" her..) (..Nå, lad mig bare antage for nu, at jeg ikke kan finde disse visioner gentaget nogen steder..) \
%A defining feature of Web 2.0 is as mentioned that users provide a lot of the \emph{content} accessed on the web pages. I hope to see this extended to the design of the web pages themselves and to the various search and feed algorithms in the future. I think this will be a relatively easy step to take compared to some of the other visions, and it might even be considered more of a "Web 2.1" rather than part of a Web 3.0.. 
		%
		%Some criticism of the current Web 2.0..:
		%The current popular Web 2.0 sites does a lot of good for their users. They provide a framework for the users to engage with each others in and they often even provide their services free of charge.. (Hm, eller måske skal jeg bare fortsætte forrige sektion således:)
%A problem that I see with current Web 2.0 sites is that the users' contributions get locked to the same framework of the site's design and its algirithms.. 
%But on a Web 2.1, as we might call it, this is no longer the case.. 
%...
%Hm, jeg føler lige, at jeg skal samle tankerne om, både hvad man helt præcist kan sige (bare kort) om Web 2.1-konceptet og, vigtigere, hvad min kritik mere præcist går på.. Og ja, jeg tror altså lige jeg prøver at gøre det her på tasterne: ..Der er for det første lidt et spørgsmål, om jeg skal fokusere på det med, at fremhæve ramme-skaberne på samme måde som indholdsskaberne.. Hm, lad mig lige læse, hvad jeg skrev ovenfor.. ..Hm, om ikke andet kan jeg huske, at jeg jo tænkte på noget med så at fortsætte denne tanke direkte med: "Og det kunne jo være en mulighed, men det bliver først rigtigt interessant, når vi får penge ind i systemet".. Hm, men gængse sider har jo også penge involveret, så er det ikke bare det, jeg skal fremhæve, nemlig at ramme-skabere også skal kunne få følgere og likes, og ligeså vel som indholdsskabere skal kunne tjene penge på deres arbejde (og her skal man så gerne finde en måde, hvor dette kan lade sig gøre)?.. ..Jamen er det så ikke bare det: "Jeg forestiller mig, at ramme-skabere gerne skal have profiler/kanaler på siden ligesom almindelige (indholds)skabere, så man kan like og følge. Og særligt skal disse også gerne i ligeså høj grad kunne tjene penge på deres arbejde som indholdsskabere." Og så kunne jeg måske tilføje, at jeg har en idé nedenfor til et netværk, hvor det ikke skal finansieres igennem reklamer eller salg af brugerdata, og her bliver det derfor ret naturligt, at ramme-skaberne også kan blive lønnet --- også selvom de ikke inkluderer nogen reklamer i deres rammer (som jo ellers kunne være et lidt kedeligt men muligt scenarie).. ..Fedt nok..:) *(Nå ja, jeg skal også huske at understrege "you can get it as \emph{you} want it"-pointen, inkl. at det jo er for dumt, at brugernes indhold skal fængsles til rammer, som kun web-udviklerne bestemmer (..Og så er jeg godt nok allerede lidt i gang med næste punkt..)..)
%Og hvad skal jeg så komme med af kritik?.. ..Hm, og man skulle ikke bare sige det i vendinger a la: Vi bør altså også undersøge, om ikke der kan være et alternativ til denne situation med reklame-finansiering og salg (ofte sådan lidt i smug) af brugernes data..(?) ..Nu skrev jeg lige '*'-parentesen i slutningen af sidste kommentarparagraf.. ..Men jo, sikkert rigtigt fornuftigt at putte det i de vendinger.. ..Ah, men jeg skal da så bare netop antage, at jeg allerede så har berørt punktet omkring, at det er for dårligt med en situation, hvor web 2.1-siderne låser indholdet til de samme algoritmer og designindstillinger.. Og så kan jeg måske bare netop følge det punkt op med, at man også (måske) kan kritisere finansieringsmodellen --- eller om ikke andet kan man pointere, at det dog er værd at begynde at kigge på andre finansieringsmuligheder.. ..Lyder umiddelbart ikke dårligt.. ..Ja, bliver det ikke nærmest bare det (og så videre til punktet, hvor jeg leger med tanken om, at alt bare kan komme fra donationer)..?:).. ..Hm, men misser jeg så ikke muligheden for lige at nævne, at ramme-skabere jo så bare skal arbejde i forskellige lag af sprog med forskellige.. verifikationsniveauer.. Hm, men det ville nu være dejligt, hvis denne snak bare kunne udskydes.. (Det bliver jo nok et lidt længere punkt, hvis jeg først skal tage hul på det..) ..Hm, eller jeg kan måske bare nævne, at det svarer lidt til en slags browser-udvidelser, som vi kender dem, bare hvor de knytter sig til selve siden i stedet, og også hvor man så sikkert bare starter med et (eller nogle få) meget begrænset(/ede) sprog. Og så kan man bare opgradere sprogene mere og mere, når det kan lade sig gøre. ..Hm, og skulle man så lige fortsætte med at sige: I øvrigt kan disse sprog måske også på et tidspunkt gøres ret magtfulde/lav-niveaus, men hvor man så bare sideløbende udvikler et godt system og netværk til at verificere sikkerheden af nye udvidelser.. ..Og så kunne man jo også gå videre og sige, at så er der faktisk i princippet heller ingen grænser: Man kunne måske nemlig endda på sigt få det sådan, at "siden" også omdannes til appds (desktop og mobile) og så endda give ramme-skaberne mulighed for både at lave forskellige forks af disse, samt at lave udvidelser inde i hver app-fork.. Hm, og så ser man lidt, hvorfor jeg er bange for at tage hul på dette emne, for det eksploderer jo en anelse.. ..Tjo, men ellers kan det oplagte svar vel bare være, at referere til en senere sektion, og så køre tangenten ud der.. Så kan jeg måske bare cutte af ved, at man skal starte med nogle simple og sikre sprog.. Men ja, et muligt problem bliver så bare, hvis denne senere sektion så ender med at blive så lille, at det kommer til at se fjollet ud at vælge at trække det ud for sig.. ..Tja, på den anden side kan jeg vel godt finde på nok at sige, så sektionen lige akkurat bliver lang nok til, at det ikke virker fjollet..(?..) ..Ja, det tror jeg, jeg vil gøre, for det er alligevel et meget godt emne lige at dykke lidt ned i, så folk bedre kan forholde sig til, hvad web 2.1 kommer til at indebære mere i praksis..  
%..Nå ja, og selvom jeg nok bør putte det i forsigtige vendinger, så kan jeg nu godt slutte af med at sige, at det altså godt lidt kan virke som en fælde, nærmest, når man tænker over det: Det virker lidt som om markedsmodellen kan handle om at lokke brugere til, og når så brugerne er afhængige af dette mødesspunkt, så gør ejerne efterfølgende hvad de kan for at genere brugerne så meget, som det føler, de kan slippe af sted med for at tjene penge, uden at brugerne forlader siden igen.. Hm, måske er dette nu også allerede lidt for groft/hårdt sagt, men det kan jeg jo lige tænke over.. 

%(10.01.22) Nå, jeg tænker lidt over nogle ting og prøver lidt at samle op på de forskellige idéer, jeg har arbejdet med nu her (som førte frem/tilbage til min "kundedrevne virksomhed" her i sidste ende). Én ting, jeg vil starte med at sige, er at det hele nok \emph{ikke} skal "måles i penge," for det er jo netop ret smart, at programmører kan være de første iværksættere. Jeg skrev så, at hvis aktierne kan handles frit, så går det ud på et, men det passer vel nemlig ikke helt alligevel?.. For så skal de næstførste programmører jo have penge på rede hånd.. Ja, for man kan ikke regne med, at markedet reagerer hurtigt nok, ift. hvad man har brug for, hvis det i starten bare er programmører, der er iværksættere. Men alt dette kan bare løses ved at indføre en indledende fase, hvor aktierne handles frit, og hvor man ikke behøver at notere pengebeløbende for hver handel.. Hm, men dette svarer vel egentligt bare til, at man bare starter med aktier, der, jo, tilsammen udgør 100 \%, og så handler man bare frit med dem, og behøver ikke at notere handlerne som nogen formel investering.. Ja, så gav det faktisk ikke sig selv alligevel..? ..Jo.. ..Tjo tja, for hvordan skal man så sammenligne de første iværksætteres aktiers afkast-krav med det for de efterfølgende (formelle) investeringer?.. ..Ah, men der kan så netop lige præcis bare være en fase 2, hvorfra man begynder at kunne komme med formelle investeringer, hvilke så kan bruges til at bl.a. at betale den løn til arbejdere/skabere, der jo i sidste ende bliver et mål for virksomhedens vækst. 
%... Okay, jeg har også overvejet en del, hvordan dælen jeg lige skal introducere det, for jeg ver ikke tilfreds med min tidligere idé. Men nu tror jeg bare, jeg egentligt kan spore det ret hurtigt hen, så jeg får udtalt problemstillingen på en måde, så det naturlige svar lidt bliver: Aktier med fast afkast-forskrift (og ikke ejerskab, men dog stemmeret), og at kunderne skal have mere og mere del i det, hvrtil et (rimeligt) naturligt svar så også bliver, at de bare skal betragtes helt som investorer. Og jeg behøver vist i øvrigt ikke rigtigt at komme ind på, om virksomheden kan du for andre områder. Jeg kan bare lægge ud med at sige, hvorfor der er et særligt behov for at løse problemstillingen for et web 2.1/3.0-netværk/firma/fællesskab, og så lad det blive ved det. Og jeg kan så bare lige lægge ud med at sige, at det dog kræver meget mere analyse, før folk bør begynde at investere i start-ups, der implementerer idéen. 

%Ok, så hvis jeg skal braine over, hvordan det næste i dette paper skal lyde, så kunne jeg jo lige starte med at gentage det, hvor jeg begynder at kritisere gængse web 2.0-sider, og så tage den derfra.. (10.01.22):
%
%Okay, så lad mig antage, at jeg har forklaret om "Web 2.1" (om så jeg vil kalde det det eller ej..).. Så kan jeg vel sige, at jeg synes, det er et problem ved gængse w2.0-sider, at de låser brugernes indhold til de samme designrammer og de samme algoritmer..
%Og et andet problem ved gænse w2.0-sider er, at de ofte kan anklages for at ende med at bruge en forretningsmodel, hvor der i starten bliver kælet så meget for brugere og skabere som muligt, for at lokke en stor mængde til, men når så først en side har etableret en kæmpe brugermængde, så begynder virksomheden at forsøge at malke dets brugerskare mere og mere for penge ved at genre dem mere og mere med reklamer eller sælge deres data videre. Der er jo som bekendt en såkaldt netværkseffekt, når det kommer til w2.0-sider, hvor værdien af siden stiger stærkt med, hvor mange der bruger dem. Og brugerne er derfor selv afhængige af at siden har gjort sig "etableret," og kan altså ikke bare sådan lige skifte til en anden side, der kan love nogle bedre forhold. Til sådanne virksomheders forsvar kan man så sige, at hvis nu de har investeret en masse penge, så skal de jo gerne tjene noget for deres investeringer, og dette kan måske så netop kun ske efter den indledende fase, når virksomheden først er etableret. ..Hm.. Skulle man så fortsætte med at pointere, at dette hvis virksomheden kun akkurat sørgede for, at den fik en vis retur på sine investeringer, så ville man jo forvente, at der så på et tidspunkt vil blive skruet ned for reklamerne igen. Men det kan vi jo alle regne ud ikke vil ske, når først malkningsprocessen er i gang, også selv hvis man i starten bare "malkede" det, man har ærligt fortjent ved sin investering, jamen så vil denne malkningsproces fortsætte og fortsætte så længe aktionærerne kan holde den kørende.. Hm, er det denne vej, man skal gå i sin kritik..? Hvad er alternativerne..? ..Tja, det er måske ikke helt dumt, hvis jeg bare lige formulerer det lidt mere sobert.. ..Og ja, bare påpeger, at jeg ser dette som en mulig faldgrube, der kan resultere i en virksomhed, der ikke ender med at fungere optimalt i kundernes favør (når vi snakker sådanne web 2.0 companies).. 
%Hm, og så kunne jeg jo muligvis fortsætte og sige: Der er også et andet muligt kritikpunkt, og det er, om reklamer og salg af brugernes data virkeligt er den optimale finasieringsstrategi set med brugernes øjne. Selvfølgelig er der rigtigt meget værdi i, at man bare kan åbne op for internettet og bruge frit af diverse w2.0-hjemmesider uden at skulle til at tænke på at betale noget. Og ja, man kunne dog netop sige, at bare det, at man ikke skal \emph{tænke} på at skulle betale, det gør det til den optimale "handel" fra brugerens synspunkt, også måske selv hvis en længerevarende nu alligevel skulle vise, at det ville være bedre, hvis brugerne bare betalte et beløb om måneden for hvor meget, de har brugt servicen.. ..Hm, måske skal jeg egentligt lige selv tænke lidt, om dette ikke kunne blive et alvorligt modargument mod min kd.v.-idé, når det kommer til internetvirksomheder.. (Altså måske inden jeg skriver videre på denne brainstorm, lad os nu se..).. ..Nå nej, det kan aldrig blive et rigtigt stærkt modargument, for en kd.v. kan jo altid bare selv gøre brug af reklamer. Men det kunne godt nok godt potentielt set tage luften meget ud af kritikpunktet her..  ..Hm, det \emph{er} faktisk ret mange penge, der tjenes pr. reklame.. ..Det virker som en grotesk stor andel, som Google tager (fra YT-skaberne), men ikke så stor, at det betyder vildt meget fra brugernes synspunkt direkte.. ..Hm, men på den anden side, ift. til forrige sætning, så er det vel stadig ikke mange penge pr. reklame, ift. hvad brugerne ville give for at slippe for se samme mængde reklamer (for en given mængde, især hvis denne mængde bliver stor).. Der er selvfølgelig det, at et alternativ så vil tvinge folk som mig, der alligevel bare bruger add blockers til så at betale, hvilket bliver en stigning i omkostninger fra 0 til noget ikke-forsvindende.. Til gengæld kan jeg så sige, at jeg gerne vil betale, hvis jeg kan gøre det med penge og ikke reklametvang (hvilket btw understreges af, at en af mine få faste abonnomenter har været til Spotify igennem alle årene).. Men det vil "gå ud over" folk, der ikke har lyst at betale, og som godt kan finde ud af et bruge en add blocker.. ..Oh well, det behøver jeg vel ikke som sådan bekymre mig over.. (eller hvad..?) ..Nej, selvfølgelig ikke: Selvfølgelig er det kun en fordel, hvis alle brugere bedes betale lige meget for deres brug af servicerne. ..Ok, men har jeg så ikke et godt nok argument bare i, at mange brugere nok vil foretrække, at betale løbende alt efter deres forbrug og så undgå en tilsvarnede mængde af reklametids-tvang.? ..Jo, så dette kan altså muligvis blive mit andet kritikpunkt, som jeg også lagde op til.
%..Og mere skal jeg vel ikke bruge. Nu bliver det naturlige spørsmål så: Hvordan får vi et system, hvor ejerne ikke kan udnytte netværkseffekten til at malke fællesskabet for penge, selv hvis der kunne være muligheden for det potentielt set?.
%..Hm, vent lige: Hvad med argumentet om, at modellen især ikke giver mening, når brugerne endda selv bliver programmørerne..? ..Hm, men det kan jeg jo bare påpage som et tredje "kritikpunkt," som så til gengæld ikke vil være et nutidigt kritikpunkt, men et kritikpunkt af ikke at finde en bedre løsning, når w2.1 indtræder.. Ok.
%..Med andre ord leder vi efter et system, hvor investorerne stadig får en fortjeneste i vente, men hvor vi ikke bare lader aktierne hænge og dingle, så folk, der har lyst til at købe en hovedpart og så tvinge virksomheden til at malke fællesskabet, ikke bare kan komme og gøre dette.. 
%Jo, jeg har en mulig idé til dette. Den handler lige præcis om i første omgang bare at låse alle aktier til en fond, og så i stedet danne et nyt framework for, hvordan folk kan investere i virksomheden, og hvordan de kan få stemmeret i virksomheden. 
%Denne fond vartager så bare alle rettigheder og værdier. Ledelsen af virksomheden har mulighed for at sælge af disse værdier, men de har ikke ret til at tage lån (eller oprette skyld på anden vis) på baggrund af fonden.. Hm, men kan en generation af magthavere i virksomheden så ikke bare sælge alle værdierne og rettighederne til sig selv?.. ..Hm.. ..Tja, så fonden må altså ikke kunne sælge IP-rettighederne videre, når den har købt dem (delvist) fra skabere (inkl. programmører).. ..Hm, er dette nok..? Og hvad i øvrigt med min normale kd.v.?.. ..Hm, bliver dette bare endnu en "korruptionsvektor," man lidt bare må tage med, fordi der så forhåbentligt alligevel ikke går lang tid, for aktionærmængden ca. er lig kundemængden (groft set)..? 
%(11.01.22) Det er jo sådan set bare alle IP-rettigheder, der skal (\emph{skal}) være låst til fonden.. Så ja, det er helt fint..
%Mine tre kritikpunkter nævnt i går var malkning af netværkseffekt, bedre værdi for pengene/tiden for brugerne og at det er dårligt forsvarligt, at en central enhed skal tage et cut, når nu endda selve brugerne er programmørerne og designerne.. Og i går aftes kom jeg så også på, at der jo også bare er det ved det, at reklamer er svære at få til at fungere, når nu ikke alle skabere er indholdsskabere. Desuden kom jeg i tanke om, at reklamer jo også sætter en øvre grænse på, hvor meget hver enkel skaber tjener, der ikke nødvendigvis er fordelagtig for fællesskabet. Og desuden værdsætter systemet specielle ressourcer, f.eks. clickbait-ressourcer osv., hvor det egentligt ikke giver særligt god mening for fællesskabet. Men ved i stedet at have en meget kunde-/bruger-styret virksomhed, hvor brugermængden i høj grad har stemmeret til at bestemme lønningerne, så vil sådanne ting være fortid, og lønningerne vil kunne optimeres meget (meget!) bedre.. 
%Så medmindre jeg cutter nogen af dem, så bliver disse vist indtil videre mine kritikpunkter. Bemærk at jeg med dette egentligt heller ikke behøver at nævne noget om mine "donationsforeninger" og "lønningsmodeller" som et btw-punkt, for jeg præsenterer jo alligevel en idé, hvor stemmeretten er fordelt meget ud til kunderne, og så vil bagudbelønning blive naturlig nok at nævne her. Den sætning gav ikke særligt meget mening, men jeg lader den stå, for jeg tror godt, jeg selv ved, hvad jeg prøvede på at sige..
%I øvrigt, kan jeg lige hurtigt sige, kom jeg også frem til i går, at man netop sagtens kan starte virksomheden uden særligt meget kapital ved bare at have en fase 1-ledelse, som gives ansvar for at belønne fase 1-skaberne og fordele aktierne, og hvor denne ledelse i øvrigt så selv har nogle faste aktier og ikke selv har nogen baises, når det kommer til fase 1-skaberne. Dette vil måske endda være fordelagtigt, for så kan man nemlig lige præcis belønne bagud allerede næsten helt fra start af og således tiltrække de gode og kreative programmører fra start, og ikke bare lige dem, man kan finde og ansætte.
%
%Og dette er alt sammen bare skønt, for nu synes jeg bestemt, jeg har nok til at motivere min kd.-idé her for et/en w2.1/3.0-fællesskab/netværk/virksomhed. Og uden at jeg behøver at perspektivere til alt muligt andet. Og i sidste ende føler jeg også, jeg godt kan hente pointen hjem omkring det store fremtidspotentiale i idéen. Så det er bare \emph{så} fedt.
%
%Ok, lad mig prøve så at forsætte med forklaringen af kd.-systemet..:
%Hm, i stedet for at sige:
%"Med andre ord leder vi efter et system, hvor investorerne stadig får en fortjeneste i vente, men hvor vi ikke bare lader aktierne hænge og dingle, så folk, der har lyst til at købe en hovedpart og så tvinge virksomheden til at malke fællesskabet, ikke bare kan komme og gøre dette."
%Kunne man mon ikke så sige..:
%Med andre ord leder vi efter et system, hvor investorerne stadig får en fortjeneste i vente, men hvor vi samtidigt også har en god sikkerhed for at ledelsen overdraget mere og mere til brugerne/kunderne.. Noget af det første, man så bliver nødt til at sørge for, er at ejerskabet og magten over ledelsen ikke bare får lov til at hænge og dingle på det frie marked, så enhver oppertunist med kapital nok kan komme og opkøbe firmaet og forvandle det til en malkningsmakine.. Selvfølgelig skal investorer gerne få en vis magt over virksomheden, men denne magt skal så gerne aftage og difundere over på kundernes hænder, når afkastet på deres aktie (hvilket så er en fase 2-aktie..) er mere og mere betalt. 
%Så for min løsning vil jeg altså som det første punkt forslå, at man tager og opretter en fond til at eje alle IP-rettigheder for web-virksomheden, samt alle lincenser til at gøre brug af diverse skaberes IPs..
%..Hm, samtidigt opretter man så (måske; hvis det er sådan, det skal gøres) en virksomhed, der har ret til at bruge af al denne IP, så længe det bare holder sig til følgende regler..
%Efter eventuelt en indledende fase med et fastsat sluttidspunkt, hvor virksomheden kan få lov til bare at køre som en normal virksomhed, så skal alle aktionærerne stemme om.. Hm, et beløb pr. aktieprocentdel..? Men hvad så med forskellige penge kurser..? For man vil vel i øvrigt ikke særligt gerne binde det hele op på en enkelt valuta.. Så hvad gør man i stedet..? ..Hm, skulle man bare så sørge for at definere en værdienhed, som så kan afhænge af mange ting..? ..Og så kunne jeg bare nævne dette, når det lige passer.. ..Ja, det må næsten bare blive min løsning.. 
%Ok, men efter denne første fase er min idé så, at alle aktier i virksomheden skal veksles (med den underliggende fond) til nye aktier, der giver krav på et afkast, som kan afhænge af virksomhedens vækst i den gældende periode, og som også giver en vis stemmeret over virksomhedens ledelse, hvilken dog kan aftage over tid. Samtidigt skal der fra det punkt af gælde, at alle salg af services (og i princippet også produkter, men lad os bare gå ud fra, at vores web-virksomhed her kun sælger services i reglen), der bidrager til et overskuddet også skal.. Ah shit, jeg ser en potentiel korruptionsvektor..! Så lad mig lige tænke over dette en gang:
%Hvis kunderne, når disse selv er blevet "aktionærerne," frit bestemmer, hvilke services der bidrager til overskudet, og hvis kunderne kan splittes op, så en majoritet af kunderne særligt er forbrugere af én mængde af services, så kan de jo sørge for, at overskuddet bliver regnet for at lægge hos disse services og ikke hos dem, minoriteten forbruger.. Og på den måde kan man ende ud i en potentielt meget unfair situation.. Så er der noget man kan gøre, for at omkostning-udregningen bliver mere lødigt bestemt..? ..Så hvad, skulle man bruge et uafhængigt parti, måske som en del af fonden selv, til at udregne omkostningerne..? ..Hm, et eller andet sted så er problemet jo lidt, at omkostningerne vil lægge meget i udviklingen.. Hm, men kan man ikke tit også dele det op der..? ..Hm ah, hvad med, om man måske gjorde noget med, at kunderne på en måde selv betaler for de investeringer, de har stemt for..(!)..? ..Hm, men vil dette så ikke bare svare til.. ..hm, at kunderne ligesom skal betale en fast udgift på deres aktie.. Hm, jeg må lige tænke lidt..
%..Hm, så skulle man mon prøve at sige noget med, at overskud pr. pris bare normeres, så det regnes for at være det samme forhold for alle servicer og alle produkter..? ..Hm, eller at man kun trækker de faktisk omkostningerne fra for en service eller et produkt, og ikke regner lønninger ind i det..? ..Ja, det må jo næsten blive det, der bliver svaret.. ja, det er klart.. Ok..
%Så for at fortsætte, hvor jeg slap, så skal alle salg efter det punkt altså regnes som en investering, efter at de har fået de faktiske omkostninger trukket fra sig..
%Dette vil sige, at den betalte penge sum, der så går til, hvad der svarer til "overskuddet" og til lønningerne i samme periode, og altså hvor (kun) de direkte omkostninger trækkes fra, skal købe kunden en aktie helt magen til, hvad alle andre investorer vill få, hvis de investerede det samme beløb direkte. Og disse aktier skal alle sammen være af den type, hvor der gives et afkast, der kan afhænge af virksomhedens vækst (og muligvis også af, hvad man skylder forrige "generation" af aktionærer). 
%..Hm, og nu kommer jeg vel så lige straks til at foreslå, at væksten kommer til at udregnes specifikt efter lønningerne givet, men så må der da i øvrigt også gælde, at man først skal have sikret sig, at man også kan betale det ekstra penge, der går til de forstørrede aktieafkast, før man har lov til at skrue op for lønningerne..? ..Ja, det må jo næsten være sådan, for man vil jo meget gerne undgå, at virksomheden kan gå konkurs.. Hm, og det vil jo heller ikke nødvendigvis have så meget kapital, der ikke netop er låst til fonden.. Så ja, måske kan man gøre, så at virksomheden til en vis grad gerne må "tage et lån" i dets ikke-likvide kapital, som kan bruges til at godkende stigende lønninger, selvom man ikke allerede har likviderne til at betale de så forstørrede afkast, men.. Tja, det bliver under alle omstændigheder ikke så relevant, når det er et web 2.1-firma.. hov, medmindre man jo ejer servere og infrastruktur.. Hm.. Tjo, men det lyder nu alligevel klart mest fornuftigt, at en ny generation af aktionærer ikke har lov til at gamble med dette ikke-likvide kapital. Ja, ok. Så alt andet end lige, skal virksomheden altså have likviderne, før den har lov til at skrue op for lønningerne. Hvis man så mangler penge, kan man jo netop bare søge efter investorer, der jo så selv vil få gavn af de efterfølgende stigninger i ansatte og/eller i deres lønninger. Ja, det virker da netop så totalt oplagt på den måde. (For så forhindrer det også ledelsen i at prøve først at finde de investorer efterfølgende, når lokummet brænder, og evt. i et grådigt forsøg på ikke at skulle skylde samme investorer så meget.) Cool. 
%Men inden da bør jeg vel næsten lige skrive om, hvorfor at idéen om at behandle kunder som investorer vil føre til, at kunderne får mere og mere andel i aktierne.. ..Eller hvad; jeg kunne jo også bare skrive om det efter de pointer omkring lønningerne og det..(?) ..Og hvorfor \emph{vil} det så for det første føre til, at kunderne vil få en større og større andel?:
%Det vil det vel, fordi.. Hov vent, kan en generation ikke også bare skide hul i, hvor godt et arbejde, der bliver gjort.. især hvis de selv endnu ikke stort set er selve kunderne, eller hvis de er stoppet med at være kunder..!..? ..Hm, vel ikke hvis aktierne bare fungerer over lang tid nok, og hvis de aftager med en kurve, således at stemmemagten også overvejende er fordelt til de aktionærer, der er interesserede i lønningerne om lang tid også.. ..Ja. Og her er det altså lige præcis stemmemagten, der så skal aftage.
%Ok, så er det lige før, at jeg faktisk kommer med mit forslag omkring, at vækst regnes i lønninger, før jeg så begynder at forklare, hvorfor aktionærene, givet at aktierne så konstrueres rigtigt, samlet set har interesser i de lange udsigter, og så hvorfor at dette alligevel vil føre til, at kunderne får mere og mere andel i virksomheden.. 
%..Og ja, fordi en interesse i at forhøje lønningerne på længere sigt vil medføre en interesse i at lønningerne går til noget, man så kan sælge videre igen, så man kan få penge til at udvide, så vil virksomheden bevæge sig mod at give så store lønninger som muligt, hvor servicerne og produkterne så dog satdig kan betale for både lønningerne og for aktieafkast (for før må man heller ikke skrue op for lønningerne). Så den eneste måde aktionærerne altså kan tjene mere i sidte ende, end de har investeret, er hvis der bliver en omsætning, der kan betale for større lønninger. Og i denne forbindelse vil aktier jo også skulle udstedes til kunderne. Så aktionærerne vil altså søge imod, at der skal blive en stor omsætning i firmaet, hvilket også medføre, at flere og flere kunder skal have aktier. ..Så alt i alt kan man sige, at aktionærerne netop tjener penge, kun hvis aktierne samtidigt også bredes mere og mere ud på kundernes hænder over samme periode. ..
%... Hm tja, det kræver vel bare lige, at man netop sikrer sig, at en stigning i lønninger ikke medfører en højere afkast-stigning, end hvad den samlede typisk vil udgøre i forhold til de samlede lønninger.. ..Hm, eller..(?..) 
%(12.01.22) ..Ja, man kan jo sige, at det kun kan betale sig at give løn til arbejde, der betaler sig, for ellers bliver man hurtigt nødt til at fyre og/eller skrue ned for lønniveauerne. Hvis det kun lige akkurat kan betale sig, så kan man have et stabilt punkt, hvor alle investorer bare får de samme beløber tilbage igen, men dette kan afhænge af, om man lader afkastet afhænge af tidligere skyld til investorer: Hvis virksomheden således lige er vokset inden dette stabile punkt, så kan det være, at forskrifterne (hvis man tillader dette), vil tvinge en til at fyre lidt igen, så man lige kan betale den skyld.. nå nej, dette bliver så ikke et problem i praksis alligevel, for vi har jo en vedtagelse om, at pengene skal være der inden lønstigningen. Ja, så man kan sagtens nå disse stabile punkter. ..Hm, men hvis omkostningerne inkl. lønningerne trækkes fra, så kan man jo netop se, at.. Ja, at kundernes indflydelse kun vil stige med en vækst i virksomheden, og det er jo ikke helt godt nok.. Der skal gerne lige akkurat være en stigning af kundeindflydelse også i det stabile punkt.. Hm, ville det mon i stedet give mening bare at se på priserne uagtet, hvad går til lønninger..? ..Hm, og vil dette egentligt ikke også på en måde være mere fair..? ..Hm, hvad er mest fair..? ..Hm, og skal man i øvrigt ikke sørge for, at virksomheden på en måde kan lave bløde opsplittelser af sig..? ..Hm, en grund til at det ikke er fair bare at se på overskuddet, er, at det er unfair så ikke at omkostningerne for at udvikle servicen eller produktet tages højde for.. ..Hm, men bliver svaret næsten ikke så bare at se på de totale priser og ikke andet..? Og så kan kunderne i sidste ende jo selv stemme for, hvad der skal investeres i, når det kommer til udviklingsarbejde.. Hm..? ..Hm ja, og man kan vel netop ikke argumentere for, at nogen kunder får mere i bytte for prisen, for priser skal jo netop bare afspejle, hvad kunden får ud af det, især når den samme procentdel af prisen går til aktier uanset hvad.. ..Hov, gad vide, om dette ikke også var, hvad jeg kom frem til i går/forgårs.. Nå, det gider jeg ikke lige at tjekke. Men jo, aktiekøb pr. pris skal bare være det samme for alle varer/services til et tidspunkt --- i hvert fald inden for samme virksomhedsafdeling, hvis den er splittet op, men det vil jeg prøve at vende tilbage til.. --- og så kan dette forhold så til gengæld bare afhænge af det samlede overskud ift. lønninger og andre omkostninger.. ..Nå, men hvordan dælen skal den så afhænge af overskuddet, for den må jo netop ikke bare være lig forholdet, for kundeandelen skal gerne vokse en smule selv for stabile punkter.. ..Ah, kunne man mon gøre noget med, at.. hm, lad mig lige tænke lidt... 
%(13.01.22) Jeg fik tænkt lidt videre i går og fik et par små idéer, og nu fik jeg også den tanke, at man jo godt kan gøre, så det bliver mere tidsinddelt ift. stemmemagten, således at en aktie vil have samme magt over den gældende periode uanset, om rigtigt mange nye kunder/investorer kommer til. Og når man så først benytter dette koncept, så kan man jo også næsten ligeså godt droppe at se på overskudsforholdet og bare regne med den samlede pris, ikke? I øvrigt kom jeg også i tanke om, at det at alle lønninger skal beskattes, det eliminerer faktisk nogle korruptionsvektorer, jeg ellers har tænkt på, hvor aktionærer betaler kunstig høj løn til visse arbejdere, hvor pengene hurtigt kan veksles om og investeres igen.. Eller hvor en kunde betaler en kunstig vildt høj løn, som denne så bare får igen med det samme som arbejder (eller med en sammensværgelse med en arbejder).. Men så længe alle lønninger beskattes, så giver sådan nogle schemes ikke så meget mening. ..Hm, og jeg kan lige nævne en anden idé, hvilket var ikke at sætte et delay på, hvornår en aktie begynder at regne væksten fra. Det virker nemlig også mest fair, så at nye kunder/investorer ikke bliver belønnet for beslutninger, der kun har kunne være taget, inden de kom om bord, ligesom. ..Ok, kommer det så ikke rimeligt meget til at give sig selv nu..? ..Hov, men det er jo ikke så fedt, hvis investorer så.. Ja, man bliver enten nødt til at se på overskuddet, når man sammenligner kundeinvestorer og almindelige investorer, eller også skal man regne det som to forskellige kategorier..(?) ..Og kan man ikke godt finde overskuddet, eller var problemet, at man kunne fuske med dette tal..? For man skal jo gerne kunne investere overskuddet, og gerne også med det samme, men hvordan skal man så lige kende forskel på, hvad er de normale omkostninger og lønninger, og hvad er ny investering..? ..Hm, og man kan ikke bare se på, hvad summen af omkostningerne og lønningerne er til forskellige tider..? ..Og muligvis hvad den samlede indkomst fra salg er (før omkostningerne trækkes fra)..? ..Hm, kan man ikke bare sige, at hvis sum af omkostninger og lønninger stiger, så skal det betragtes som en investering, og skal derfor regnes med som en del af overskuddet?.. ..Jo, og når man så ser på en lille periode, så kan man så, hvor stor en del af den samlede indkomst gik til dette overskud, og så putte denne procentdel på alle salg, således at kunderne får aktier ud fra denne "investering." ..Hm, der bliver så bare lidt med, at hvis man udvikler en service, så kommer det (nogne gange) til at koste penge inden at selve servicen udbydes, og så kan man komme ud for, at kunder ses som.. Hm, nå nej, for procentdelen uddeles godt nok ens på alle priser, så det vil dog kun blive alle kunder i den periode, der får den fordel, men.. Er dette ikke også et problem..? ..Hm ja, investorer bør næsten hellere belønnes, hvis deres investeringer tager endeligt tid at bære frugt, og at lønningsudgidfterne ved dem ikke bare varer ved.. Hm.. ..Hm, men skulle man så ikke netop også bare måle væksten via overskuddet..? ..Hm, eller hvad med primært via overskuddet, men så med en bonus der afhænger af lønningerne..? 
%Hm, jeg skal vist alligevel tænke en del over denne idé: Der er ligesom flere problemer, der melder sig, som jeg skal have opklaret...
%... Okay, jeg tror endelig, jeg har det. Man skal sådan set bare finde en måde, hvorpå man kan opnå, at priserne som børne-servicerne skal "betale" til en forælder-service kommer til at blive sat, så den optimerer omsætningen. Dette kan man så sikkert finde flere måder at gøre på: Måske kan man bare indføre en kontrakt om, at ledelsen skal forsøge at sætte priserne så de optimerer.. men ellers er der også mange andre ting, man kan gøre. Man kan måle på det ved at lade parametrene fluktuere periodisk over perioder, der er længere end den typiske tid, det tager en kunde at købe et nyt produkt eller service og/eller fornye et abonnoment på.. ..Hm.. Og ellers kan man indføre et system, hvor folk kan vædde på, at en parameterændring vil give en stigning i indkomst. Hvis der så væddes nok i én retning, så skal parametrene så skiftes og væddemålet kan så tjekkes efter en fastsat periode, medmindre at en af parterne vil indrømme tabet og indgå en settlement, hvor den anden part vinder nogle penge, og hvor man så igen kan vædde om. Dette er især brugbart hvis flertallet tager fejl og kan se det i tide, for så kan man settle væddemålet tidligt uden at parametrene behøver at tvinges i den gale retning i længere tid.. Et sådant system kunne nok være værd at foreslå, hvis ikke jeg har eller finder en simplere løsning.. ..Hm, men måske er dette i virkeligheden den mest simple løsning.. (Altså af dem, jeg kan tænke mig til..) ..Hm ja, for selv hvis man har en kontraktbunden ledelse, så skal man jo alligevel tjekke det på en måde.. Ja, denne slags væddemål kunne godt ende med at blive det bedste svar.. Nå, men jeg kan i hvert fald antage, at man godt kan finde løsninger, og at man godt kan sørge for, at priserne for barne-services bliver sat, så de --- i hvert fald tilsyneladende (ud fra data) --- optimerer indtægten nogenlunde bedst muligt. Og så skal der ellers bare gælde det, jeg har lagt op til, nemlig at barne-servicerne jo så i sidste ende kommer til at få magt over den sluttelige faktor, ud fra hvor stor andel de betaler til den ift. de andre.. ..Hm, hvad gør man i øvrigt, hvis en service ikke kan bruge de penge, de får på en fornuftig måde, hvis de enten får for meget eller for lidt..? ..Hm, så er det vil bare kontraktbundet (formentligt) til at lade pengene være i en pulje, som så kan gå til at give rabat på de næste betalinger.. Eller også bare sende dem retur, hvis det på en måde er muligt.. Fint nok. Nå, men tilbage til at stemme om den endelige, samlede pris betalt til en forælder-service.. ..Hm, jeg tænker jo bare at bruge et slags gennemsnit.. Var der andet, jeg skulle sige omkring dette emne..? ..Hm, eller giver det allerede meget sig selv bare med denne lille tilføjelse..? (Jeg tror i hvert fald på, umiddelbart, at dette var en rigtig vigtig idé, der ligesom åbner op for en rigtig god og endda stadig simpel version af idéen..:)..) ..Lad mig lige summe en lidt mere; jeg troede langt om længe, jeg var ved at være færdig med summeriet, men.. Tja, den er jo også sent nu (ti i seks), og jeg kan mærke, at min hjerne er ret grødet, selvom det jo egentligt har været en af de lettere arbejdsdage (sådan summe-tænkning, hvor jeg lader tankerne køre i periferien meget af tiden, plejer ikke at tage særligt meget på ressourcerne (sammenlignet med mere fokuseret tænkning (såsom hvad skrive-tænkning især er) eller med skrivning (generelt))). Så jeg tror bare jeg summer videre her i aften og så vender tilbage i morgen. Men fedt, at jeg endelig ser ud til, at være ved at få styr på idéen ("igen," ville jeg sige, men det er jo så meget sagt..).. :) 
%(15.01.22) Ah, nu tror jeg, jeg har det. I går aftes kom jeg på nogle rigtigt gode ting. Nu tror jeg, jeg har en ny, mere simpel version af en kd.v. Der var nemlig nogle rimeligt væsentlige problemer med min version hidtil, som jeg var ved at prøve at (finde frem til og) løse.. Jeg fik nogle idéer, der måske kunne have ledt til en løsning, men nu fik jeg så også disse nye idéer, der sikkert ville slå de løsninger, jeg kunne have fundet på, alligevel. Nå, lad mig se.. I den nye version af idéen skal kunderne faktisk allerede lige efter den indledende fase gives fuld stemmemagt --- eller også kan der være en mellemfase, hvor magten overdrages mere og mere, men stadig ud fra en fast forskrift, der kun afhænger af tid (..ja, det er nok bedst, så folk bedre kan forberede sig på overgangen) --- til at styre alle investeringshandler, hvor investorer får mulighed for at komme med penge til en opgradering af en 'service' eller til dannelsen af en ny, og hvor disse så til gengæld får en kontrakt, der giver dem ret til et vist afkast betalt over tid (eller bare betalt ved en slutdato), hvilket så kan afhænge af parametre for, hvor godt det går med den opgraderede eller den nye 'service.' Og her skal det altså sådan set ret meget bare være de aktive kunder til den 'service,' man vil opgradere eller fornye helt (hvilket i sidste tilfælde så kan være en vilkårlig 'service'). Henfaldstiden på kundernes stemmemagt, ift. deres tidligere køb kan altså godt være noget, der kan måles i år eller årtier.. lidt alt efter, hvor lang tid der går mellem betalinger fra de normale, "aktive" kunder.. Ok. Og ja, investorerne fra første fase skal selvfølgelig bare selv prøve at sætte tilsvarende afkast-forskrifter, så de kan få mest muligt, men hvor de stadig kan tiltrække kunder på baggrund af, at virksomheden i en nær fremtid vil bevæge sig over i en tilstand, hvor kunderne kun betaler omkostningerne inkl. lønningerne til de 'services,' de forbruger. Nå, og jeg har så prøvet at fremhæve 'service' lidt her i denne tekst, for jeg føler, at det faktisk er en vigtig kilde til, hvorfor denne idé kan fungere, nemlig at jeg er begyndt at forestille mig et træ af 'services,' hvor der endelige services, barne- og forælder-services.. (Hm, jeg burde skrive 'servicer' i stedet..) Og pointen med at fremhæve forælder-servicer er så, når disse altså har nogle faste omkostninger, som gerne skal finansieres fra kunderne til de barne-servicer, der gør brug af dem. Og ja, hele virksomheden kan så ses som opbygget af moduler af 'servicer.' Og her lader jeg så også bare produktioner indgå under begrebet 'service' (så vi altså her for nu i denne tekst bare kalder det en "service" at bygge noget for nogen). Nå, og som jeg ladge op til i min forrige version af idéen, så skal kunderne altså have stemmemagt over en "sluttelig faktor," som jeg har kaldt den ovenfor, som bestemmer, hvor meget 'servicen' skal have at gøre godt med, ud fra hvor meget de betaler til "servicen" (og med en vis (varierende) henfaldstid, når deres forbrug er faldet fra et tidligere stadie). Og samtidigt skal der også gælde, at service-ledelsen, eller hvad end modul der helt præcist skal styre dette, på en eller anden måde blive tvunget til at sætte parametrene i servicens forskrift for, hvilke kunder betaler hvor meget i forhold til hinanden, så omsætningen maksimeres. De idéer tror jeg også, fortsat bliver ret vigtige. Hm, og lad mig i øvrigt lige gå videre en gang og nævne et par idéer, jeg kom på i går nat, nemlig at 'service'-kundegrupperne så bør gå sammen rigtigt mange (og gerne alle for så vidt muligt) og aftale nogen fælles bestemmelser for, hvordan udefrakommende opfindere skal belønnes, hvis de kommer med en god idé til en opgradering eller til en ny service. For hvis dette bare er op til de enkelte 'service'-kundegrupper, så bliver det for fristene bare at hapse idéerne uden at give igen. Men hvis mange nok går sammen om at følge de samme principper, så kan man sikkert opnå et rigtigt godt system, hvor nye opfindere kan føle sig sikker på, at de bliver rigeligt betalt for deres bidrag. Belønningerne til opfindere kan jo så oplagt være i form også af kontrakter, der giver krav på et afkast, der kan afhænge af, hvor godt det går den nye 'service' (og nærmere bestemt hvor godt det går for den nye opfindelse). Og noget tilsvarende kan man så i øvrigt eventuelt også gøre for at sikre fair lønninger for arbejderne, så kundegrupperne ikke bliver fristet til at bruge at presse citronen for de arbejdshold, som de betaler til. Her kan man altså også gå sammen på tværs af mange 'service'-kundegrupper og så aftale at holde sig til nogle bestemmelser for, hvor meget arbejdere af forskellige typer skal gives i løn for deres arbejde. En specifik kundegruppe vil så ikke være nær så fristet til at presse citronen for deres arbejdere, for dette kan så gå ud over deres egen arbejdsløn. Men ja, og ellers så kan arbejder jo også bare tilmelde sig fagforeninger og på den måde tvinge arbejsgivere til at betale en fair løn. Men tværgående aftaler internt i virksomheden kan altså også potentielt være hjælpsomme, vil jeg umiddelbart mene (for jeg vil mene, at fristelsen bliver mindre til at presse citronen, når man så indirekte herved vil komme til at forværre forholdene for så mange flere mennesker..). Ok, men jeg hoppede måske lidt af kæden ift. den hoved-idéen (dog dermed ikke sagt, at disse idéer ikke er en vigtige; idéen om at belønne opfindere tror jeg faktisk, bliver rimeligt essentiel), så var der ellers noget, jeg mangler at forklare her?.. ..Ah jo, jeg kan for det første sige, at 'servicerne' hver især skal holde en slags 'definition,' som både kan indeholde noget IP, og som ellers skal indeholde bestemmelserne for, hvordan 'servicen' skal operere. Hver 'service' kan således ses som en lille virksomhed, som operere ud fra en 'definition' og så selvfølgelig de overordnede bestemmelser i kd.-virksomheden, som gælder for alle 'services.' IP'en kan så forresten komme fra opfindere af den type, jeg lige har snakket om, og ellers fra opfindere, som arbejder for en af de 'servicer,' som den pågældende 'service' udsprang fra (for ja, mon ikke godt man også kan gøre, så at én 'service' kan søsættes af flere andre 'servicer' --- så skal jeg bare passe på, for disse vil så ikke nødvendigvis være "forælder-servicer" med min terminologi hidtil, hvilket er lidt forvirrende --- det tror jeg, man kan). ..Ok.
%Nå, var der ellers andet, jeg mangler at forklare..? ..'Servicernes' 'definitioner' kan jo i øvrigt også bare indeholde bestemmelser for, hvornår penge kan spares sammen, og hvornår de kan og skal betales tilbage.. Ah ja, og det bringer mig så til at nævne, at alle 'definitioner' jo selvfølgelig skal konstrueres, så nye kunder ikke bare kan få 'servicen' til at sælge værdier og "tilbagebetale" (..ha, deja vu..) dem, og på den måde blive i stand til bare at løbe med boet.. Hm, er der ellers andet, jeg skal nævne..? Hm, jeg kan jo lige summe lidt over, om jeg har overset nogen huller, nu her i eftermiddag (og jeg vil jo også gerne gå en eftermiddagstur), og så kan jeg ellers prøve at se, om jeg kan nå til at genoptage dispositionsarbejdet senere i eftermiddag/aften... (..Nå jo, inden jeg gør det, så lad mig lige udtrykke, at det virkeligt virker som en god version af idéen, det her. :D^^)
%... Nej, jeg har vist ikke så meget mere at sige. :) Måske bare lige, at sidste punkt omkring, at de nye kunder ikke kan løbe med kapitalen, og derfor kun selv har gavn af, hvis det går virksomhedsmodulet/'servicen' godt, er en ret vigtig pointe, der gør, at det hele kan fungere..
%*Nå jo, jeg skulle også lige nævne, at en "investering" også kunne komme i form af et IP-bidrag. Så (udefrakommende) skabere og/eller opfindere, der allerede har IP-rettigheder over deres bidrag, de kan så vælge at sælge deres bidrag til 'service'-gruppen i stedet for bare at stole på at få bagud-belønning.
%
%Cool,:) og hvis jeg så endelig skal begynde at tænke i disposition igen, så kan man vel for det første sige, at idéen jo stadig tager godt afsæt i de to punkter, om at det kunne være godt, hvis afkast blev fastsat ud fra forskrifter fra starten, og hvis at kunderne vil få mere og mere stemmemagt.. Og hvordan man så lige tage hul på at få forklaret idéen..? ..Hm, ville det mon ikke være oplagt at starte med at introducere, netop at stemmemagten overdrages (måske løbende) til kunderne automatisk efter den indledende fase, og forklare hvordan man stadig kan sikre sig, at de kun har gavn af, at det går virksomheden godt (og så kunne man altså udskyde at snakke om 'service'-modulerne en omgang..)..? *(Og her snakker vi så i øvrigt i helt første omgang magt til at bestemme, hvor meget virksomheden skal have at gøre godt med til at betale omkostninger/lønninger..) ..Ja, og inden jeg når til 'service'-modul-opdelingen, så kan jeg også forklare om, hvordan kunderne så får den endelige magt (selvom de nu godt kan have en ledelse ansat til på egen hånd også at træffe nogle mindre investeringsbeslutninger), ift. videre investeringer. ..Og det med udefrakommende opfindere/skabere kan så også komme derefter, og inden jeg begynder at snakke om 'service'-modulerne.. ..Tja, i hvert fald den del om, at de også kan belønnes eller betales med afkast-kontrakter (ikke den del om at "gå sammen i mange 'service'-grupper (omkring bagud-belønning)).. (Indskudt: Det skal forresten også lige siges, at det jo hører med, at en ny service ikke må betales ved at tage fra kapital fra en gammel service, får så ville nye kunder jo ellers sagtens kunne hacke sig til en måde at løbe med kapitalen på..) ..Nå, og så er det vil nærmest bare at foklare, hvordan der gerne skal kunne oprettes nye services (og opgraderes gamle i øvrigt), og.. Hm, men dette kommer jo til at hænge sammen med nye investeringer.. Hm, på nær at investeringer jo netop også kan gå til at opgradere.. ..Ja, så jeg kan vel bare snakke om opgraderinger indtil jeg kommer til 'service'-modulerne, og så kan jeg jo her måske starte med at sige, at investeringer også skal kunne gå til at oprette nye virksomhedsmoduler, og at virksomheden på den måde kan blive mere og mere opdelt.. ..Hm, skal man også definere en proces til at opsplitte én 'service,' så den bliver flere..? Det kan jeg lige tænke over.. Men ja, og så skal der jo forklares, omkring forælder-services, og om at pris-parametrene skal sættes.. eller i hvert fald meget vel \emph{kan} sættes..(..?) efter, hvad der optimerer omsætningen, men hvor kunderne så har stemmemagt til at sætte den "sluttelige faktor." ..Ja, og så er jeg vel næsten nået hele vejen rundt, på nær det der med at slå sig sammen om f.eks. (særligt) bagud-belønning til udefrakommende opfindere/skabere..? ..Tja, eller jeg skal jo også sørge for at uddybe omkring 'definitionerne,' det er klart.. ..(Og jeg skal jo også komme med forslag omkring det med at sikre sig, at ledelsen vil optimere omsætningen ift. prisernes forhold..) ..Hm, men ja, det virker umiddelbart som om, dette kunne blive en fin gennemgang af det..(..:)) 
%(16.01.22) Okay, i går aftes kom jeg frem til, at der lige manglede lidt mere, for det kunne være rart, hvis det stemmehavende kunder lige kunne blive en anelse mere investerede i virksomhedsmodulet, end at de bare er aktive kunder. Og jeg kom så også på en mulig løsning. Jeg tænker, at der generelt i kd.v.'en skal være et krav om, at kunder skal købe nogle.. jeg kan se, det vist er det, der hedder 'futures..' Ja, så kunder skal altså købe futures med i alle deres handler, som de så selv kan sælge videre, når de har lyst (og endda bare med det samme). Dem der ejer en future på et givent tidspunkt har så en stemmeret i virksomhedsmodulet, der dog aftager, når man kommer tættere og tættere på futurens dato. Og bum, så tror jeg næsten, idéen holder.. :) 
%Jeg kom også i tanke om, at man selvfølgelig også kan starte nye fonde, og dermed nye grundlæggende kd.v.'er, og måske skal man også kunne gøre dette som en intern del af kd.v'en.. Hm, men uanset hvad, så bliver dette ikke så vigtigt. Man kan nok godt regne med, at én fond kan varetage alle.. 'aktiver'.. Hm.. ..Hm ja, det er ikke så vigtigt.
%*Nå ja, og det skal så også siges, at det så gerne skal fremgå klart af virksomhedsmodulets 'definition,' hvad servicerne er, som sælges, sådan at det altid er ligetil at udstede futures på baggrund af disse servicer. Og hvis dette kommer til at begrænse kd.v.'en en anelse, så gør det faktisk ikke noget; det er faktik helt fint, hvis f.eks. virksomheden således begrænses til produktion og underlæggende services, og at der så i nogen tilfælde bare lige skal et ekstra lag over, før det kan komme helt ud til kunderne.. ..Hm.. Ah, man kan sige, det kan være, det kommer til at koste kd.v.'en noget fleksibilitet, som et lag af handelsinstanser så måske kan fylde ud, men dette vil nok altid kun være midlertidige behov, for med tiden vil alle sådanne services også blive forudsigelige, og man vil altid i sidste ende kunne udbygge 'definitioner' (med helt faste, formelt definerede services) til at varetage de samme funktioner. Fint.:)

%Dispositions-brainstorm forfra :) (16.01.22):
%Okay, lad mig se.. Helt overordnet set er planen vel så først at introducere Web 3.0-visioner og supplere med mine Web 2.1-visioner. Så vil jeg komme med visse kritikpunkter, hvorved jeg gerne ligesom skal nå frem til, at det vil være ne god idé, hvis de første investorer fik som fortjent, men at magten på en eller anden måde blev overdraget mere og mere til kunderne. Og så vil jeg komme med mit nuværende bud, hvilket handler om netop at overdrage stemmemagten efter en hvis indledende fase (og løbende i løbet af en mellemfase). Men hvordan sørger man for, at kunderne ikke bare beslutter sig for at sælge alle aktiver og give penge til sig selv? Jo, netop ved at have en fond ved roden af firmaet, som i princippet har ejerskab over alle aktiver, men som er kontraktbundet til.. Hm ja, lad mig lige se.. ..Hm, men skal nye virksomhedsmoduler egentligt ikke ofte bare eje deres egne aktiver, og så bliver hvert virksomhedsmodul sådan set bare en kd.v. i sig selv, hvor den altså er bundet til at følge kunderne --- eller rettere ejerne af futures --- ift. hvad de.. Hm, men skulle kunderne ikke kun have magt over "sluttelig faktor" og over salg til nye investorer i form af.. Hm.. ..Tjo, men hvis alle moduler i sig selv bare er kd.v.'er, så behøver man netop ikke.. hm, lade nye moduler (/'servicer') udspringe af eksisterende, eller hvad?.. For nye moduler kan jo bruge af gamles IP-rettigheder m.m.. ..Ja, hvem skal helt præcist have rettighederne til at bygge videre på de gamle IP-rettigheder m.m.? Kunne denne magt eventuelt gå, ikke til de aktive kunder, men til fortidige kunder generelt, hvor man altså ikke lader stemmemagten henfalde særligt meget..? ..Ah, men nu afhænger 'sluttelig faktor'-stemmemagten jo også af futures, så ja, måske ville dette blive en naturlig nok opdeling.. Lad mig lige se.. ..Hm, mon det ville være mest fair, hvis det ligesom er de kunder, der bruger en given IP eller et givent andet aktiv meget, der får mest stemmemagt over, hvilke nye 'servicer' kan startes, som kan.. bruge af den.. Hm, jeg kan mærke, at der findes en simpel løsning på det her, men min hjerne skal lige nå dertil.. ..Lad mig se på, hvad er fair, når først vi er i et sent punkt, hvor kunderne har fuld magt over priser og lønninger i princippet.. ..De vil vel så kun være glade for, hvis de samme aktiver kan bruges til andre ting også.. Ja. Så man må ligesom sige, at når først afkastene er betalt af, jamen så er det helt fint, at IP-rettighederne m.m. bare tilhører hele offentligheden, i hvert fald lige netop når det kommer til oprette andre kd.v.'er/kd.v.-moduler på baggrund af dem. Hm, så hvad siger vi..? ..Hm, og indtil afkastene er afbetalt, skal det så bare være pågældende aktionærer, der har den fulde magt, og ikke kunderne..? Eller skal man have seperate kurver over magten til at oprette nye 'servicer'..? ..Hm ja, er det ikke bedst, hvis man bare har IP-rettigheds-kurver med som en seperat del af aftalerne omkring 'servicerne'..? ..Hm, og andre aktiver end IP-rettigheder kan vel et eller andet sted bare regnes sammen som en del af "omkostningerne".. ..Og når jeg siger "IP-rettigheds-kurver," så kan der altså godt være tale om step-funktioner (osv.), så at rettighederne f.eks. også bare kan frafalde ved én bestemt dato.. ..Okay, så er min nye kd.v.-idé overvejende bare (hvis vi lige ser bort fra ting så som at pris-forholdene imellem forskellige kunder skal optimeres ift. omsætningen og sådan..) at afkast- og stemmemagt-kurver (over ("sluttelig") omkostningsfaktor) for de indledende investorer og opfindere skal være fastsat (ud fra forskrifter) fra starten, og at den restende stemmemagt over omkostningsfaktoren så skal komme fra ejerskab over futures, hvilke i øvrigt gerne skal sælges som en tvungen del af alle handler (medmindre jeg skifter mening..).. Og ja, i øvrigt skal der også være fastsatte kurver for stemmemagt, når det kommer til at opstarte nye kd.v.'er på baggrund af samme IP-rettigheder, hvor opfinderene/aktionærerne altså kan sidde helt eller delvist på denne magt i noget tid.. Hm, men vent, bliver dette ikke bare noget eksternt, nemlig at opfinderne bare liciterer IP-rettighederne til kd.v.'en/kd.v.-modulet, men jo beholder rettighederne til at sælge licenserne til andre virksomheder (evt. kd.) eller andre kd.v.-moduler inden for samme kd.v..? Hm.. ..Men man kunne jo så netop godt måske købe licens til rettigheder til også at starte nye 'servicer' op på bagrund af samme IPs.. ..Hm, men ødelægger det ikke lidt hele pointen, hvis ikke IP-rettighederne på et tidspunkt bliver frie, det synes jeg umiddelbart, det gør. Jo, en del af idéen, er at kunderne kan bakke op om servicen i vished om, at det fører til, at de selv for mere og mere rettighed over kapitalen, inkl. IP-rettighederne. Så IP-rettigheder skal altså have en dato, hvorefter de helt overgår til kd.v.'en, og denne skal nok bare fra samme dato også gøre, så at alle er frie til at oprette kd.v.'er, der benytter sig af IP-rettighederne (hvor kd.v.'erne jo måske så netop bare skal opfylde nogle krav, der gør dem til en "kd.v."). 
%Okay, så fonden i midten skal vel så kun handle om IP-rettigheder, og om at sørge for, at det kun er kd.v.'er, der får adgang til dem, når de er helt overdraget til fonden.. Og i mellemtiden.. Ja, så kan der jo være en kontrakt om, at kun visse 'servicer' må bruge af licensen, men det behøver selve fonden nok ikke bekymre sig så meget om, for der kan de enkelte kd.v.'er nok bare hives i retten, hvis de bruger noget, de ikke har licens til.. Og er en kd.v. så lidt bare en virksomhed med nogle meget formelt definerede udbudte servicer og/eller produkter, hvor der så, efter en indledende fase, kommer til at gælde, at det bliver future-holdere, der kommer til at bestemme, hvilken faktor skal puttes på priserne i sættet af alle servicer/produkter. Virksomhedens ledelse er så kontraktbundet til at følge 'definitionen,' hvilken i øvrigt gerne, som jeg ser det, skal indeholde noget med at prisfoholdene skal sættes, så de maksimerer indtægten (givet den "sluttelige faktor," der nu er indstemt på det tidspunkt), og selvfølgelig også til at rette sig efter future-holdernes afstemning, når det kommer til den "sluttelige faktor." I starten af virksmoheden (som kan ses som modul i en slags makro-virksomhed) kan der så også være nogle aktionærer/opfindere, der har en vis stemmemagt over den "sluttelige faktor", og desuden kan disse have krav på et fastsat afkast, der kan afhænge af virksomhedens omsætning.. Hm, og hvad mangler jeg så..? ..IP-licenser må ikke kunne trækkes tilbage.. ..Hm, og ellers skal jeg jo lige tænke lidt mere over, hvordan nye 'services' så skal dannes --- og ikke mindst hvordan man skal opgradere en 'service'.. ..Hm, og det kan ikke bare være en del af 'definitionen,' hvordan en 'service,' eller rettere en sådan formel (mikro-)virksomhed, kan "opgraderes"..? (Så man i princippet altså kun har konstante mikro-virksomheder/kd.v.-moduler, men hvor deres ("konstante") definition måske bare er åben over for ændringer..?) (..For hvad skulle man overhovedet eller gøre..?) ..Cool, det bliver mit cue til en eftermiddags(gåturs-)pause...
%Okay, jeg kom på flere ting på min gåtur. For det første kom jeg lidt frem til, at det også bare kan være kunderne ligesom, der bestemmer, hvor meget virksomheden må låne og med hvilke rente-forskrifter.. Så kom jeg frem til, at futures måske ikke alligevel er det rigtige at bruge. Jeg kom så på, at man måske kunne bruge et krav på en andel af den fremtidige omsætning eller samlet løn i stedet for futures eller noget i den stil.. Og så kom jeg på, at man måske bare kunne gøre, så at kunderne stemmer på en fremtidig ("sluttelig") "faktor".. Og senere igen kom jeg så lidt frem til, at måske skal jeg egentligt bare tilbage til, at det bare er de fortidige kunder (med en aftagende kurve), der har stemmemagt over den nutidige "faktor" (og så også over, hvor meget må lånes og med hvilke rente-kurver).. Så det er lidt der, jeg er nu..(^^..) ..Nå ja, og så overvejer jeg dog også, om man mon kunne blande det lidt, så at de stemmende også skal have aktier i f.eks. den fremtidige løn.. ..Hov, hvis det jo kun alligevel er den "sluttelige faktor," som kunderne har magt over, så.. så kan virksomheden vel bare stoppe.. "produktionen," før den sælger nogen aktiver.. eller ja, den behøver vel ikke at sælge aktiver.. hvilket så enten bare er godt, men måske også kan blive et problem.. Lad mig lige se på det hele.. 
%..Hm, deja vu, men kan man ikke bare sætte noget inerti på..? (Sikkert ikke, men alligevel..?)  
%
%(17.01.22) Ah, jeg tror, jeg har fundet svaret. I går aftes kom jeg på den idé, at man efter den indledende fase, og efter at skylden til de oprindelige opfindere/iværksættere er betalt, bare skal lade gamle kunder have både stemmemagt og faktisk nærmest også ejerskab, sådan at de faktisk \emph{er} frie til at tjene penge på forretningen, som de har lyst. Så på den måde kommer det til at fungere som en helt normal virksomhed, men bare lige med den ændring, at de kunder, der på et givent tidspunkt betaler til virksomheden og til de nuværende "ejere," så selv vil blive "ejere" i fremtiden (og over en lang periode i øvrigt), og vil få mulighed for at tjene de samme penge (plus eller minus), som de ligesom kunne have sparet, hvis nu "ejerne" var minimalt grådige til at begynde med, da de selv var aktive kunder. ..Ja, om man kan jo sørge for også at sætte en øvre grænse på, hvor stor en andel må gå til ejerne, sådan at man ikke kan ende ud i et pyramidespil, og så "ejerne" dermed er nødt til ikke at afpresse deres kunder for meget. Jeg tænker umiddelbart, at der bare skal være lige stort "ejerskabsandel" for hvert tidsinterval, man kigger på (så hvis der er få kunder i en periode, så får disse så bare så meget desto større ejerskab hver især).. Hm, eller vil dette mon skabe nogle underlige dynamikker..? ..Ellers kunne man vel også bare gøre det, så.. Hm.. Tja, måske kan man bare tage nabo-intervallerne lidt med (og nabo-nabo-intervallerne i mindre grad osv.), når man skal regne gennemsnittet ud, sådan at det ikke har potentiale til at flukturere så meget, det kunne nok være en god idé.. Ok, og hele idéen er så, at man nu ikke længere behøver at tænke på, at folk kan løbe med boet, og altså at de kan få virksomheden til at sælge anlægsaktiver, som det vist kaldes, for nu vil ejerne jo således være investerede. Og tanken er så, at jo flere anlægsaktiver, og jo mere kapital der således ligger i forretningen, desto længere skal perioden bare være, før aktierne henfalder. Og bum, så kan man pludselig regne med, at "ejerne" ikke bare vil få virksomheden til at sælge aktiverne, selv hvis de handler grådigt.:) Desuden får man også en måde at belønne de rigtige beslutninger, som svarer nærmest helt til, hvordan almindelige virksomheder fungerer, men bare hvor man så samtidigt sørger for, at kunderne i fremtiden selv for ret til en tilsvarende del af overskuddet (og hvis dette så f.eks. er det sammen, jamen så kan de få de samme penge tilbage). I øvrigt skal kunderne gerne frit kunne sælge deres aktier med det samme. Aktierne skal \emph{kun} udstedes til kunderne, og den tilhørende stemmemagt \emph{skal} følge med her, men efterfølgende skal kunderne være frie til at sælge det igen. Så på den måde kan selv kunder, der har brug for penge nu og her måske komme til at få rimeligt favorable priser alt i alt, fordi de så nok kan komme til i sidste ende kun at betale for omkostningerne samt at købe sig af med risikoen på deres aktie.. Jeg er totalt glad for denne løsning.^^ På en måde er den lidt bare et skridt tilbage til, hvor jeg var før, men dog ikke helt efter min mening: Jeg føler at idéen lige er tweeket nu, så den holder meget bedre. Der er bl.a. smart, at man nu godt kan separere de indledende aktier og så kunde-aktierne, så de kan fungere på to forskellige måder. Apropos dette, så kan kunde-aktionærerne jo godt tillade sig at kompensere sig selv til en vis grad for at skulle have betalt de indledende iværksættere, sådan at denne byrde fordeles mere jævn ud i tid. Og så kan de jo bare vælge lige at skære nogle procenter fra hver gang, sådan at den samlede.. hm, den samlede kunde-aktie-værdi passer med anlægsaktivernes værdier i et vist forhold.. eller skal man bare undlade at tænke over dette?.. Eller skulle man faktisk gå ind og gøre, så at aktie-værdierne kommer til at følge anlægsaktiverne i et forhold..? Hm, eller lade det være..? ..Hm, mon ikke markedskræfterne plus foreninger til at forhandle priser ned, så de passer bedre på et princip om, at "ejernes" samlede investering henfalder imod et niveau, der er en vis del større end anlægskapitalen, kan klare det fint? Og man har jo en foranstaltning så priserne ikke kan løbe løbsk.. Jo, mon ikke dette er fint.:) I øvrigt vil jeg lige nævne, at alle IP-rettigheder altså stadig \emph{ikke} ejes af (mikro-)virksomhederne selv, og at "ejerne" altså aldrig har mulighed for at sælge disse. Og for at vende tilbage til de foreninger, jeg lige nævnte, så giver dette jo mening, for aktionærerne vil jo selv være aktive kunder alle mulige andre steder, så det vil sagtens kunne give mening at lave en bred aftale på tværs af (mikro-)virksomhederne om at lade priserne henfalde til et niveau, hvor aktieværdierne kommer til at passe med kapitalen. 
%..Hm, var det det hele, jeg skulle sige..? ..Nå jo, jeg kan lige sige, at virksomheden stadig gerne må kunne deles op, men nu behøver man så ikke (nødvendigvis) alt det der med "formelle 'definitioner'" osv. Men i og med at der jo stadig gerne skal være en fond til at.. basalt set fratage kunde-aktionærerne ejerskabet over IP-rettighederne (sådan så andre kd.v.'er også er frie til at bruge dem --- i hvert fald hvis de enten har købt licens fra opfindere/skabere, eller hvis altså licensperioden er udløbet, sådan at de ikke længere behøver at købe den, men at den er fri for alle kd.v.'er). Men ja, så vi kan altså stadig se alle kd.v.'er som en del af et større hele, især hvis de deler samme fond eller hvis fondene arbejder tæt sammen, hvad der gerne skal være tilfældet (for man vil jo gerne have det sådan, at IP-rettighederne til diverse ting ikke er adskilte på flere makro-virksomheder i sidste ende, men kan bruges frit sammen med hinanden).:) Fint! Og hvis der er andet, jeg mangler at sige, så kommer jeg jo nok hurtigt på det.:)
%*(Nå jo, lad mig da bare lige nævne, hvis ikke jeg har nævnt det et sted, at man jo kunne starte med et ret simpelt bagudbelønningssystem i den indledende fase, og om ikke andet kunne man endda gøre det så simpelt, at de eksisterende aktionærer kan udlove et fast beløb hvert kvart kvartal, hvor de så fordeler belønninger til folk (gerne i form af aktier med afkast-kurver), så det når om på dette "beløb.")

%Disp-brain forfra igen (17.01.22):
%(Kopieret ovenfra:) "Helt overordnet set er planen vel så først at introducere Web 3.0-visioner og supplere med mine Web 2.1-visioner. Så vil jeg komme med visse kritikpunkter, hvorved jeg gerne ligesom skal nå frem til, at det vil være ne god idé, hvis de første investorer fik som fortjent, men at magten på en eller anden måde blev overdraget mere og mere til kunderne." Og lad mig lige prøve at ridse disse ting op igen (og evt. supplere).. ..Kopieret ovenfra: "Mine tre kritikpunkter [...] var malkning af netværkseffekt, bedre værdi for pengene/tiden for brugerne og at det er dårligt forsvarligt, at en central enhed skal tage et cut, når nu endda selve brugerne er programmørerne og designerne.. Og i går aftes kom jeg så også på, at der jo også bare er det ved det, at reklamer er svære at få til at fungere, når nu ikke alle skabere er indholdsskabere. Desuden kom jeg i tanke om, at reklamer jo også sætter en øvre grænse på, hvor meget hver enkel skaber tjener, der ikke nødvendigvis er fordelagtig for fællesskabet. Og desuden værdsætter systemet specielle ressourcer, f.eks. clickbait-ressourcer osv., hvor det egentligt ikke giver særligt god mening for fællesskabet. Men ved i stedet at have en meget kunde-/bruger-styret virksomhed, hvor brugermængden i høj grad har stemmeret til at bestemme lønningerne, så vil sådanne ting være fortid, og lønningerne vil kunne optimeres meget (meget!) bedre.. "
%Jeg kan i øvrigt også tilføje, som jeg kom til at tænke på i går, nemlig at man også kan mindste den alt-eller-intet-agtige netværks-effekt, der lidt er for gængse web 2.0-sider, hvor en stor skaber kan blive større på bagrund af dennes storhed hidtil.. så ikke som sådan 'netværkseffekt,' men også en effekt, hvor et positivt feedback gør at større bliver større. Og her kunne man med det nye system sørge for evt. at give lidt flere penge i retning af mindre, måske nystartede, skabere. Man har i hvert fald muligheden. Jeg kom i øvrigt også frem til, at jeg gerne også skal nævne / gøre det klart, at hele denne virksomhedstype jo kan være vejen til noget stort, og at der derfor kan potentielt kan være store penge i spil for de første Web 2.1-programmører.. eller rettere de programmører, der udvikler en Web 2.1-platform, hvilke jo så netop ikke kan betegnes som Web 2.1-programmører endnu.. Ja, jeg må lige være lidt forsigtig med det(/den?) term. 
%Hm, mange af mine punkter handler egentligt mest om at skifte lønningsmodel for skaberne og gå væk fra reklamer, men mit første punkt om "malkning af netværkseffekt" og mit tredje punkt om "at det er dårligt forsvarligt, at en central enhed skal tage et cut, når nu endda selve brugerne er programmørerne og designerne," synes jeg faktisk i sig selv er god nok grund til at komme med et bud på en virksomhedstype, hvor der ikke sidder nogle fede katte centralt og spiser sildene. Og især når det også gerne skal være sådan, at grænsen imellem oprindelige programmører og Web 2.1-programmører gerne skal være så flydende som muligt, så man inddrager web 2.1-programmørerne mest muligt, og så det kommer til at fungere så meget som en open source-virksomhed som muligt.. ..:) 
%
%Ok, så hvis jeg går tilbage til dispositionen.. ... Hm, kunne jeg starte med at påpege, at det så ikke så godt kan være reklamebaseret, for det virker ikke nødvendigvis så godt for programmering. Dette leder så til, at man nok burde lave en virksomhed, som kan købe og håndhæve IP-rettigheder fra skabernes bidrag. ..Hm, skulle jeg så vente med at nævne andre fordele ved at gøre dette og ikke at bruge reklamer og så gå direkte til at pointere problematikken i at have et firma, der i praksis gerne skal fungere som en open source virksomhed, men hvor der så altid er nogen, der kan sidde centralt og malke penge (ud af netværkseffekten)..? ..Jo, det kunne vel meget vel være, og jeg skal så lige huske på, at det altså ikke gør noget, hvis det virker som et en anelse tyndt argument til at starte med, for jeg tror på, jeg nok skal få gjort det tykkere alt i alt i sidste ende.. ..Hm, og man kunne ikke lige finde en eller anden oplagt pointe, der siger at de første skabere også gerne skal være aktionærer, og at dette koncept også bedst passer sammen med et kd.v.-system..?.. ..Hm, det relaterer sig måske ikke så meget til kd.-idéen i sig selv, men jeg har jo dog en version af den, hvor skaberne kan blive aktionærer.. Men giver det så mening at introducere kd.-idéen med det koncept, det gør det vel ikke.. ..Hm, men måske kunne man i stedet lige fokusere lidt på, at kunderne skal have mere indflydelse, sådan at skabere kan få større lønninger, giver det mening..?.. ..Hm, det bliver vel nok lidt svært at argumentere for, eller hvad..? ..Tja, på nær netop hvis man argumenterede ud fra netværkseffekt-malkningen og sådan..(?).. ..Hm, nu kom jeg lige til at tænke på, om man også kunne nævne, at gængse virksomheder vil være interesserede i at vise clickbait og i at få algoritmer til at få folk til at blive afhængige, bl.a. ved at lede dem ind i ekkokamre, og sådan nogle ting, hvor brugernes interesser egentligt strider imod virksomhedens.. ..Hm, nu kom jeg lidt til at tænke på, kunne man eventuelt bare starte med at tale lidt om det, der svarer til "den indledende fase," som jeg helst vil have den, hvor skabere i høj grad gives aktier..?.. ..Hm, og nu tænkte jeg på, om man ikke bare kunne starte med at skrive lidt om at have en IP-baseret Web 2.1-virksomhed (ikke med reklamebaseret indkomster), og så pointere, at dette dog ikke helt kan nå Web 3.0-visionerne, medmindre det altså bliver mere åbent og uden en kornfed central.. Hm.. (..Hm, jeg havde lidt håbet på, at jeg allerede kunne starte på fysikken igen som en sen eftermiddags-/aften-aktivitet, men nu bliver jeg nok lige nødt til at summe over disse tanker først..) ..Hm, men er den helt store pointe ikke, at hvis man som skaber skal vælge at give.. eller licitere sin IP, men på en rimelig permanent måde, så vil man vel gerne have.. at det er en åben og langtidsholdbar virksomhed, der ikke ender med at finde en måde at tage røven på brugerne og/eller skaberne.. Hm, kan dette komme til at udgøre et tilstrækkeligt argument..? ..Og måske kan jeg så supplere med, at ellers kan det heller ikke forventes at føre til Web 3.0.. Hm.. ..Hov, hvis jeg nu starter med at rette en ("malknings"-relateret) kritik mod gængse w2.1-firmaer, som jeg jo også lidt havde tænkt mig førhen, så er det da vel ikke et tyndt argument at sige: "Det vil vi nok gerne undgå; kan dette lade sig gøre? Ja, det kan det..."..? Hm.. ..Hm, og det er da også egentligt et næsten i sig selv holdbart begyndelsespunkt at sige: "Disse visioner kommer så ikke nødvendigvis til at passe så godt på w3.0-visionerne, fordi vi så får en lidt for fed central. Er det en måde at undgå dette på? Ja..."..!.. ..Hm, nice..(?).. ..Hm, og kunne jeg om ikke andet bare sige "jeg tror," når det kommer til, et udsagn om, at skabere nok vil være meget mere villige til at bidrage til en virksomhed, der er åbent og har gode langtidsudsigter (fordi korrupte mennesker ikke bare kan købe det i sidste ende og presse citronen): "Det tror jeg, de vil" (hvad jeg jo altså rigtignok gør). ..? ..Ah, og her kunne jeg jo oplagt tilføje: "Ikke mindst fordi skaberfællesskabet jo nok er i ligeså stor fare for at lide last som brugerfællesskabet, hvis centralen skulle begynde at presse citronen." :) Ah, nu tror jeg altså, jeg har rigeligt med "skyts.":D.. ..Og så gør det ikke noget, at jeg bare udskyder selve den snak, hvor jeg lægger op til, at der også kan være ekstra store penge i så at være en del af de første skabere. Så jeg behøver altså nok ikke at tease dette (/ have det med i introduktionen), men kan bare lige nævne de ting sådan lidt til sidst.. ..:) 
%
%Okay, så jeg vil altså stadig gerne starte med "at påpege, at det så ikke så godt kan være reklamebaseret, for det virker ikke nødvendigvis så godt for programmering. Dette leder så til, at man nok burde lave en virksomhed, som kan købe og håndhæve IP-rettigheder fra skabernes bidrag." Og så vil jeg så prøve at komme ind på, at der jo så er en fare for.. nej vent.. Nej, måske kunne man netop lige argumentere for nu, hvorfor det godt kan virke ikke at have det reklamebaseret. Og så har jeg på den måde introduceret idéen om en rimelig gængs w2.x-virksomhed, som dog bruger IP-rettigheder (og diverse abonnomenter) frem for kun at bruge reklamer, og bare hvor x'et her så bliver til et 1.. Hm, men vent lige, for man kunne da sagtens basere indkomsten på reklamer som virksomhed, men så bare fordele pengene over til programmør-skaberne også.. Hm, ja.. ..Hm, så skulle man måske endda starte på en anden måde.. Nu, kom jeg så også til at tænke på, at jeg måske også lige skal overveje igen, hvordan virksomheden/fællesskabet/systemet ville klare sig, hvis det bare var mere eller mindre ren open source.. ..Tjo, meget ville potentielt fungere fint open source, men der vil også være mange muligheder, man misser.. Og ja, så skal man til at finde ud af også, hvordan man finansierer servere osv.. ..Ja, så der er helt klart meget at vinde ved at give den fuld gas, men jeg skal så bare lige have i mente, at et simpelt open source-system sikkert også kan komme langt.. Så jeg skal måske derfor sørge for netop at komme med nogle pointer, hvor ren open source ikke kan klare det..?.. ..Ah ja, og måske skulle jeg specifict nævne denne version lige før eller efter versionen med at implementere det som en meget gængs virksomhed.. ..Ah, og måske kan jeg simpelthen bare starte min idé med så at sige: "Men lad os nu sige, at vi gerne vil udforske, om man kan starte en virksomhed, som ikke bare er ren open source, men som heller ikke bare implementeres via en gængs virksomhed, hvor der så bliver fare for, at denne på et tidspunkt udnytter bruger- og/eller skaber-fællesskabet..."..(?) ..Ja, det ville da umiddelbart ikke være nogen helt dum idé..!.. (Og så kan jeg i øvrigt lige nævne donationsforeninger med bagud-belønning, når jeg alligevel snakker om o.s.-versionen..) ..Hm, det lyder meget godt, umiddelbart.. ..Hm, og jeg kan måske bare nævne kritikpunkterne af gængs-firma-versionen, når jeg snakker om denne (i stedet for at nævne dem efterfølgende som tilløb til min egen version), det er sikkert bedst sådan.. ..Cool.:)  

%(19.01.22) Jeg tænkte fysik stort set hele dagen i går, men jeg har lige nogle hurtige ting at pointere her, og det er bare, at der nok gerne må holdes nøje styr med, hvad f.eks. anlægskapitalen er specifikt, og også hvor meget af omsætningen går til de normale omkostninger og lønninger, og der skal så nok være et krav om det, jeg også har tænkt, nemlig at priserne skal henfalde langsomt ned (hvis de altså er over) til et niveau, hvorved kapitalen i form af det forventede afkast kommer til at blive sammenligneligt, og gerne lidt større, end anlægskapitalen. Og noget andet jeg gerne ville nævne, er at man jo passende også kunne lokke kunder til i den indledende fase, simpelthen ved at de jo også (automatisk) får aktier i virksomheden. Og hvis dette så (lykkeligt) giver så høj efterspørgsel, at produktionen ikke kan følge med (også fordi man kun kan opnå aktierne som kunder (investeringspenge skal nemlig gerne lånes som noget separat.. med en fastsat rente-kurve, som ikke behøver at afhænge på samme måde af virksomheden vækst..)), jamen så er det jo bare smart, hvis man sørger for at regne med store nok tidsintevaller og/eller (endnu bedre) sørger for at tage et slags gennemsnit, hvor man også bruger nabointervallerne (som jeg snakkede om ovenfor), således at kunderne ikke behøver at få stress for at skynde sig at købe ting. Så kan produktionen nemlig nå at stige med efterspørgselen, uden at kunderne sidst i køen så får krav på særligt meget mindre afkast. 

%(20.01.22) Jeg overvejer lidt, om man mon skulle forklare om den genrelle kd.v.-idé (i dens nye version) i en sektion for sig, og så bare sige, at jeg jo ikke kan sige noget på egen hånd, men jeg håber at andre vil være interesseret i at kigge på, om jeg kan have ret i noget af det. Og grunden til at jeg lidt bliver nødt til at nævne det tidligt, er jo så, at jeg rigtigt gerne vil fortsætte med web-innovationsforskningen, og hvis idéen holder, om ikke andet så bare specifikt for et w2.1-firma, jamen så kunne dette jo muligvis give et kæmpe (nyt) startskud imod Web 3.0. (I denne forbindelse kan jeg så nævne, at jeg har en del andre spændende idéer, som jeg også er ved at skrive om..) ..Og nu tænkte jeg så lidt på at nævne, at det måske især er forretninger, hvor der kan være store mark-ups, hvor det kunne blive vildt gavnligt enten at starte en kd.v. eller at konvertere til en kd.v. som eksisterende virksomhed, for på den måde effektivt set at kunne give nogle meget lavere priser, selvom de faktiske priser egentligt forbliver det samme, og hvor ens umiddelbare indtjening også forbliver det samme (hvis ikke mere selvfølgelig (pg.a. højere efterspørgsel)).. ..Hm, og ja, der er jo flere gode grunde til at gøre det lige med en w2.1/w3.0-virksomhed, bl.a. fordi IP-ejerskab (af hvad jeg ved af) tager så langt tid om at ophøre, så det er virkeligt vigtigt at overdrage det til det rigtige sted, men selvfølgelig også bare fordi, det ville stride vildt meget imod hele web-tanken, hvis man endte med en virksomhed, hvor et lille centralt parti fik for meget indflydelse og magt.. (:)) 


%(05.02.22) Nå, jeg kom lige på en ting, jeg skal tænke over. Jeg havde forestillet mig, at man bare kunne have en forskrift for, hvordan overskuddet på en periode for en fase 2-aktie gerne skal svare til, at aktiehaverne til sammen har flere penge i vente via aktierne, end hvad anlægskapitalen er værd, og at forskriften altså på en måde skulle sørge for, at de to værdisummer kommer til at matche nogenlunde med tiden.. Men nu tænkte jeg så på, at dette jo bliver svært at måle ift. IP-værdier.. ..Nå nej, virksomhederne skal jo netop \emph{ikke} have magt over IP-rettighederne. D'oh. Nå, så er der ikke et problem her alligevel, andet end at jeg måske godt lige kunne genoverveje, hvad jeg nu har tænkt for at implementere denne "forskrift," eller hvad vi skal kalde det.. ..Nå ja, jeg vil jo bare sætte et cap på.. Nej, vent nu lige lidt.. ..Hm jo, var idéen ikke bare netop, at der skal være et krav om, at man udregner anlægskapitalen løbende, og at aktie-perioden så løbende automatisk justeres (muligvis med en minimumsgrænse også), så værdisummen kommer til at nogenlunde at blive lig en faktor større end 1 ganget med anlægskapitalens salgsværdi.(?) ..Jo, det må vist være sådan..  










%Nævn "korruptionsvektorer" og forklar, at i det mindste så bliver disse korruptionsmuligheder mindre og mindre sandsynlige, jo mere aktionærerne bliver lig kundemængden.. 
%Nævn "konkurs."
%Nævn investerings-prioritet og runder..
%Nævn at der ikke skal være fordel i at lade væksten svinge op og ned..

\section[(old)QED]{(old)QED...}
\ldots
%(19.01.22) Som nævnt ovenfor, så tænkte jeg fysik hele dagen i går. Jeg startede faktisk i forgårs aften med lige at sætte mig ind igen i sætningerne i Hall. Der var nogle ting, der undrede og bekymrede mig, men jeg nåede at synes, at jeg fik det opklaret, inden jeg gik i seng. (Havde selvfølgelig rigitgt svært ved at falde i søvn efter sådan en aften, så jeg valgte også bare at sove længe.) Da jeg så stod op, så havde jeg faktisk fundet et problem, imens jeg lå i sengen, som så ud til at ødelægge min strategi for at vise, at Range(H \pm iI) er tæt. Så i går gik altså på at overveje fysik hele dagen. Til slut kom jeg frem til, at min strategi for at viste nævnte rigtignok ikke holder alligevel, men at det måske nok stadig passer, at jeg altså kan finde (forhåbentligt tæt) et domæne, hvorover H er (defineret og) symmetrisk. Nu vil jeg så smøge ærmerne op og gå i gang med at analysere problemet. Jeg vil ikke gøre det her, for jeg vil gerne bruge minimale marginer, og jeg vil gerne gøre det i et frisk dokument i det hele taget, så jeg ikke skal vente på, at det renderer. Så jeg starter altså en ny tex-fil, hvor jeg analyserer problemet. Og det bliver også bare her, at jeg skriver udkast til min udgivelse. Så lad mig lige putte "(old)" foran denne sektion, bare for at vise, at den er udgået, og at der ikke kommer til at ske mere her. Jeg kalder nok bare min nye fysik-tex-fil for qed.tex.
%..Nå jo, og jeg kan lige sige, at jeg nok tror, mit bedste bud på en løsning bliver at prøve at finde frem til Dom$(A^*)$ og $A^*$, og så enten prøve at vise at Dom$(A^*)$ = Dom$(A)$ direkte, eller (måske mere sandsynligt) prøve at vise at kernen af $A^* \pm iI$ er lig $\{ 0 \}$..





\section[Web ideas]{Ideas for websites}%{Web ideas}

{\itshape
Om ikke andet indtil videre, så står det meste fra denne sektion og de næste sektioner (altså mine ``(old)Web ideas''-sektioner) ude i kommentarerne. Lige neden for denne tekst står dispositioner over, hvad jeg har tænkt mig at skrive om mine web-idéer. Det er ikke sikkert at jeg ender med at inkludere nogen af disse (på nær dem specifikt om ``Web 2.1,'' som jeg kalder det, og måske dem om et open source server-netværk samt om anonymitetsprotokoller). Hvis jeg gør, så bliver det nok bare mine første punkter omkring vidensdelingssider (se ``take 2''-dispositionen i kommentarerne). Og så vil jeg bare udgive resten i et separat dokument. Men det kan altså også være, at jeg også udgiver vidensdelingsidéerne i et separat dokument selv, og at resten så vil komme i et tredje (eller flere) dokument(er). Jeg kunne i øvrigt have inkluderet mine ``p-modeller'' som en idé omkring vidensdeling (m.m.), men jeg synes næsten, at denne idé fortjener at lade vente lidt på sig. Den er nemlig ret.\,. heftig på en måde, og jeg vil gerne holde det hele lidt mere nede på jorden lige til at starte med. 
}



%(17.12.21) Disp.-arbejde:
%
% ------ Take 1: ------
% - Wiki idea..
%     - Knowledge levels..
% - Discussion site..  (Husk btw at nævne p-modeller i fodnote eller note på dansk..)
% ---- (Måske en undersektionstitel her, som kunne være: Ideas for web 2.0 sites..)
% - Tags with ratings --- Easy to add and to rate.. --- Of all different kinds!.. (Very short version of the idea.)
%     - Can be shown in a comment section tab when visiting the resource.. (..But menu should also be expandable, preferably..)
%     - Use mainly the median in the shown result.. 
%     - Users should be able to search on tag ratings --- sorting by ratings and also with the ability to see specific intervals.. (similar to how you can often show specific price intervals for product pages of online stores..)
% - User-driven organization (as in: to organize..) of resources into subjects.. (also done via different kinds of ratings..)
% - Similar organization for rating predicates.. (which could be article predicates and/or tags...)
% - Rating tabs.. --- Rating a list of resources by moving them around..
%     - (Hm, spørgsmål: Måske man kunne blande rating tags og tabs ved at vægte tag-rating lavere, jo tættere rating resultatet af på den værdi, brugeren har givet..? ..Ja, det er da egentligt en god idé..!)
%     - ..Special tags for agreement and probability.. Hm, eller måske bare propability, og så kan agreement bare vises via denne..? (..Altså specielle ved, at deres tag-rating regnes som præcis.. Hm, men.. Hm..).. ..Hm, uanset hvad, må dette nok bare flyttes ned under additional ideas.. %"...Hm, kan prædikater ikke bare erklæres som relative som standard?.. Og så kan nogen måske erklæres med en specifik fortolkning (af "rating-aksen")..? ..Ja.. "
% - Organized comments..(?) (Very short, perhaps..) ..Yes..
%     - Aksing for links (including sources), annotations, explainations, discussions etc. by upvoting comment tab..(?)
%     - Getting an overview of diffenrent reaction and/or discussion topics..
%     - Having comment ratings plit up into importance/relevancy and agreement..
%     - Implementing discussion graphs via comments and comment tabs.. 
% - A website that puts all this (or at least most of it) together..
%     - Predicates (and tabs) for subject and predicate tree organizations themselves as well..
%     - Cross links between subject and predicate tree and inheritance..
%     - The potential in having a site with a lot of users where resources can be rated according to all kinds of different predicates..
%     - The potential if users can feel quite anonymous and feel ownership over their data, and if the website can then support user-driven machine learning (or statistics if you will)..(?) ..Or maybe later..? ..Maybe just mention it here very shortly..(?) ..Yes..
%     - I have some further implementation ideas that I will write some other place (later here or somewhere else)..
% ----
% - Web 2.1.. 
%     - ..Not having valuable content and user data of a site imprisoned within the confines of that sites particalur design (and its algorithms).. 
%     - You can get it as \emph{you} want it!..
%     - Why would you even use anything else? --- a couple of examples why..
%         - Web shops and other sites, where you need to deal directly with a party..
%         - If the website is corporate owned and thus the corporation need to disturb you to make a lot of money (as they do).. Later on I will discuss the possibility for an open source website to get around this problem..
%         - If the userbase / user community is not to your liking. Here I also have some good ideas: We can actually make it so that users can see the site pretty much as if only the part of the community they like is really active.. (I can word this differently for sure..) 
% - With web 2.1, the above ideas can be even more justified, since mixing the two ideas together must yield a giant userbase (in my mind this must be true..).. 
% ---- (?)
% - Open source servers.. Open source businesses to implement a framework for web 2.1 sites.. Hm..(?)
% - Effective ways to ensure anonymity of users..
% - ..Eller måske (soft) user groups heroppe..? 
% ---- (Additional ideas to suplement the above ones.. ..part I)
% - (More on discussion if I've left something..)
% - ..Hm, måske min "idéudviklingsplatform"..(?)
% - ..Og måske om idéen om at bruge ratings til dynamiske artikler/dokumenter(/programmer) i kollaborationer..(?)..
% - ..Hm, og skal jeg mon også lige nævne web 3.0 og snakke lidt om forbindelsen til/fra dette..?
% - User groups and weights
%     - "Soft" user group / user weight algorithms..
% - Syntax predicates for a web 2.0 site like the one a have proposed above..
	%%% - ..Special tags for agreement and probability.. Hm, eller måske bare propability, og så kan agreement bare vises via denne..? (..Altså specielle ved, at deres tag-rating regnes som præcis.. Hm, men.. Hm..).. ...Hm, kan prædikater ikke bare erklæres som relative som standard?.. Og så kan nogen måske erklæres med en specifik fortolkning (af "rating-aksen")..? ..Ja.. 
% - ..Automatiske (algoritme-)point..(?).. ..(Og særligt nok omkring propability-værdi..)
% - ..Wish resources..(?).. (altså lidt mine "ønske-ratings"..) 
% ---- (Additional ideas part II)
% - Implementation ideas for a website with the ideas from section 2 above and with web 2.1 ideas implemented as wel.. (..)

%Umiddelbart ikke en dårlig disposition so far..! :D Nu vil jeg lige summe lidt mere over denne, og så vil jeg måske lige rette den i morgen (måske i et "take 2"), og ellers så forhåbentligt gå i gang med at skrive en (brainstormy) udgave af den. :) (17.12.21) 

%Hm, hvad med "opgave-ontologi" (og andre "nyttige ontologier")..?.. ..Og "idéudviklingsplatform"..?
%Hm, men måske skal jeg bare nøjes med de vigtigste idéer at få fremført først, og så kun append'e andre idéer, hvis de løfter førstnævnte..?.. 

%(18.12.21):
%Jeg tror lige jeg lader mine første disp. her ovenfor (i.e. "take 1") stå, og så kopierer jeg den bare ind her. Og så vil jeg bare rette i denne (nedenstående) disposition løbende (indtil jeg måske kopierer igen).
% ------ Take 2: ------ 
% - Wiki idea..
% - Knowledge levels.. (in different areas..)
% - Discussion site..  
% - Creative version of discussion site..  
		%%%     - (Husk btw at nævne p-modeller i fodnote eller note på dansk..) *(Ikke nødvendigt længere; nu har jeg skrevet om dem andetsteds.)
% ---- 
% - Tags with ratings --- Easy to add and to rate.. --- Of all different kinds!.. (Very short version
	%of the idea.)
%     - Can be shown in a comment section tab when visiting the resource.. (..But menu should also be
		%expandable, preferably..)
%     - Use mainly the median in the shown result.. 
%     - Users should be able to search on tag ratings --- sorting by ratings and also with the ability
		%to see specific intervals.. (similar to how you can often show specific price intervals for product pages of online stores..)
% - User-driven organization (as in: to organize..) of resources into subjects.. (also done via
	%different kinds of ratings..)
% - Similar organization for rating predicates.. (which could be (wiki-)article predicates and/or
	%tags...)
% - Rating tabs.. --- Rating a list of resources by moving them around..
%     - (Hm, spørgsmål: Måske man kunne blande rating tags og tabs ved at vægte tag-rating lavere, jo
		%tættere rating-resultatet er på den værdi, brugeren har givet..? ..Ja, det er da egentligt en god idé..!) (Så altså hvis man gerne vil have ens rating til at tælle, når det kommer til detaljerne, og altså når man zoomer ind på et element og dets naboer i listen, så man man først navigere til den liste, hvor man også kan se elementet side om side med dets naboer. Hvis man derimod bare rater en ressource via et enkelt klik på en rating-akse, imens man lige bladre forbi den, så kommer den rating altså kun til at tælle, når det kommer til den grove inddeling af ressourcerne/liste-elemneterne.) (Og tanken med næste underpunkt er nemlig så lidt, at dette netop ikke skal gælde for bl.a. 'probability' (of being a true statement), hvor der jo er en klar (absolut) fortolkning af rating-værdien, og hvor denne altså ikke bare er relativ.)
%     - Special tags for agreement and probability.. Hm, eller måske bare propability, og så kan
		%agreement bare vises via denne..? (..Altså specielle ved, at deres tag-rating regnes som præcis.. Hm, men.. Hm..).. ..Hm, uanset hvad, må dette nok bare flyttes ned under additional ideas.. %"...Hm, kan prædikater ikke bare erklæres som relative som standard?.. Og så kan nogen måske erklæres med en specifik fortolkning (af "rating-aksen")..? ..Ja.. "
% - Organized comments.. (Very short, perhaps..) ..Yes..
%     - Asking for links (including sources), annotations, explainations, discussions etc. by upvoting
		%comment tab..
%     - Getting an overview of diffenrent reaction and/or discussion topics..
%     - Having comment ratings split up into importance/relevancy and agreement (and/or probability of
		%being true).. (For man kan godt synes en kommentar er vigtig, selvom man ikke er enig med den (eller "synes godt om den" i form af den holdning, den giver udtryk for (men hvor man måske netop så synes, at det er vigtigt at starte en diskussion og så argumentere imod den holdning)).)
%     - Implementing discussion graphs via comments and comment tabs.. 
% - A website that puts all this (or at least most of it) together..
%     - Predicates (and tabs) for subject tree and predicate tree organizations themselves as well..
%     - Cross links between subject and predicate tree. (So that e.g. users can navigate to 'movies' as
		%a subject and then be shown relevant predicates such as 'enjoyable to watch' (which might inlude other subjects than 'movies,' but then users can just remove that subject restriction once a fitting predicate is found) or 'good for relaxation in the eventing after work' or 'good for watching together with friends' etc. (but of course there can't be too many options, because the userbase will only rate the most popular ones)..)
%     - Inheritance.. (By which I mean a special tab, where the top-rated ressources (or subjects) of
		%that tab will tell the site that the given resource should inherit its predicate recommendations from those resources (or subjects).. This way, what predicates the site should recommed for different ressources belonging to different subjects can be more easily updated..)
%     - The potential in having a site with a lot of users where resources can be rated according to
		%all kinds of different predicates.. (And here, I am refering especially to the statistical (ML) possibilities.. Her snakker vi altså om, at folk så ikke bare kan rate en ting, om den er god eller dårlig, men også bl.a. kan gå ind og give forskellige rating, der måske siger, "hvis du skal bruge tingen til det og det, og/eller hvis du bruger den sådan og sådan, så er tingen så og så god." Man kan altså dele ratings op, så de kun hver taler om en specifik måde at bruge en ting på. Herved kan man også sige, "hvis man ved, at man kan gå ind og tilføje en linje i skill.cfg, der sætter damage-take-skalaen op, og at man dermed kan få sværhedsgraden ligeså høj, som man vil have den, så er HL2 sådan og sådan et godt spil.(!)" Man kunne måske også finde på at dele ratings op i moods; "hvis du er i humør til det og det, så er tingen så og så god." Eller persontræk: "hvis du er den og den type, så vil denne ting falde i din smag til den og den grad." I øvrigt er det også bare generelt smart, at alle ting kan blive rated med et overordnet prædikat, der giver mening for domænet (og hver alle rating akser (som regel) er relative), så man aldrig nogensinde skal "sammenlinge pærer og æbler," eller rettere deserter og hovedretter, film og fritidsaktiviteter, holdninger og haveredskaber, osv. osv. Og alt dette kan så komme til at betyde så meget, når man i sidste ende skal lave ML over folk i fællesskabet. Jeg følte, at jeg manglede at have skrevet dette nogen steder, så nu kom det lige her.)
%     - The potential if users can feel quite anonymous and feel ownership over their data, and if the
		%website can then support user-driven machine learning (or statistics if you will)..(?) ..Or maybe later..? ..Maybe just mention it here very shortly..(?) ..Yes..
%     - I have some further implementation ideas that I will write some other place (later here or
		%somewhere else)..
% ----
% - Web 2.1.. 
%     - Not having valuable content and user data of a site imprisoned within the confines of that
		%sites particalur design (and its algorithms).. 
%     - You can get it as \emph{you} want it!..
%     - Why would you even use anything else? --- a couple of examples why..
%         - Web shops and other sites, where you need to deal directly with a party..
%         - If the website is corporate owned and thus the corporation need to disturb you to make a
			%lot of money (as they do).. Later on I will discuss the possibility for an open source website to get around this problem..
%         - If the userbase / user community is not to your liking. Here I also have some good ideas:
			%We can actually make it so that users can see the site pretty much as if only the part of the community they like is really active.. (I can word this differently for sure..) (Pointen er her, at man så bare kan transformere alle ratings via "brugergrupperne" (i.e. via bruger-vægtfordelinger)..)
% - With web 2.1, the above ideas can be even more justified, since mixing the two ideas together must
	%yield a giant userbase (in my mind this must be true..).. 
% - And (ML) the statistical possibilities with such a site will be great..
% ---- 
% - Open source servers.. Open source businesses to implement a framework for web 2.1 sites.. Hm..
% - Efficient ways to ensure anonymity of users..
% - User groups and weights
%     - "Soft" user group / user weight algorithms.. (Som altså er det her med eksempelvis at have 
		%muligheden for at oprette brugergrupper formelt på siden, hvor moderatorer i et eller flere niveauer kan uddele vægte til dem i niveauerne under, og hvor dette så i sidste ende for til en vægtfordeling fordelt på en slags "brugergruppe." Gruppen kan så følge sine egne protokoller for, hvordan vægten fordeles, og det er dette, jeg mener med "bløde algoritmer," for en sådan løsning som denne kan så bruges af trejdparter til at oprette vilkårlige vægtfordelingsalgoritmer på siden. For en trejdpart kunne jo f.eks. bare oprette en sådan brugergruppe, hvor der kun er én moderator, som uddeler vægten (nemlig trejdeparten selv, og så dikterer denne altså vægtfordelingen og kan bruge sine egne algoritmer.) Så min idé om moderatorer og vægtfordelinger er altså i stand til at bære vilkårlige trejdeparts-vægtfordelings-algoritmer (og det ville andre tilsvarende idéer også kunne).) (Og \emph{min} idé er så også gavnlig, fordi den også kan bruges til at lave brugergrupper, hvor der bare er skrevet et manifest (altså bare i et naturligt sprog) for, hvordan brugergruppen skal fungere, og hvordan vægten skal fordeles. Og jeg mener, at sådanne brugergrupper sagtens kan blive ret effektive, fordi medlemmerne sikkert ofte vil være gode tilsammen, overordnet set, at følge sådanne manifester..)
% ---- (Additional ideas part I)
% - Using ratings as an automatic way to "moderate" the compound articles.. (Altså mine dynamiske
	%dokumenter..)
%     - This can perhaps be extended to other areas (of collaborative work).  
% - (More on discussion site if I've left something..)
		%%% - Hm, måske min "idéudviklingsplatform"..(?) *(Ja, den bør faktisk endda komme lige efter diskussionssiden.)
% - Hm, og skal jeg mon også lige nævne web 3.0 og snakke lidt om forbindelsen til/fra dette..?
% - Syntax predicates for a web 2.0 site like the one a have proposed above..
% - Automatiske (algoritme-)point..(?).. ..(Og særligt nok omkring propability-værdi..)
% - Wish resources..(?).. (altså lidt mine "ønske-ratings"..) 
% ---- (Additional ideas part II)
% - Implementation ideas for a website with the ideas from section 2 above and with web 2.1 ideas
	%implemented as well.. (..)





%I have spent a lot of time thinking about ideas for the web of the future. I am very excited about the prospect of Web 3.0\footnote{...} and therefore I wanted to try and see if I could figure out some idea that might help bring this future about. 
%This process has ...


%Okay, nu overvejer jeg lige noget nyt: Før jeg skrev ovenstående disposition i går, kom jeg frem til, at web 2.1-idéen godt måtte komme rigtigt tidligt. Men da jeg skrev dispositionen, så syntes jeg alligevel, det ville give god mening at forklare nogle af mine andre idéer (se omtalte disposition) først, for jeg mente nemlig, og det mener jeg faktisk stadigvæk i øvrigt, at jeg vil kunne skrive dem rigtigt kort. Planen er altså bare at skrive måske en 1--5 paragrafer (så altså måske to-tre i gennemsnit) for hver oversektion i dispositionen ovenfor, før vi når til Web 2.1-idéen. Jeg føler stadig ikke, at denne plan er helt dum, men nu kom jeg i tanke om en anden mulig strategi. Jeg kan nemlig ikke lade være med alligevel at synes, at jeg ikke rigtigt har fortjent læserens opmærksomhed, når det kommer til at lire.. 100 web-idéer af på vedkommende (og det kan jeg jo nok godt have ret i..). Og nu virker det jo lidt som om, at Web 2.1-idéen faktisk bliver det store omdrejningspunkt (he, den ære har godt nok gået meget på skift.. xD).. For de andre idéer er jo langt hen af vejen sådan nogle lidt "hej, jeg hedder Mads, og her er nogle af mine iværksætter-idéer" idéer.. (Og dette er endda foruden alle de idéer, der bare handler om implementation..) ..Og hvis man tager sådan en idé som min "brugerdrevet ML/statestik"-idé, så kan den, selvom jeg faktisk stadig på en måde ser den som en af mine vigtigste idéer, ret meget forkortes til: "De statistiske muligheder for Web 3.0 blive kanone; når det hele bliver mere åbent, og når folk kan eje al deres eget data og alle direkte digitale bidrag (adskildt fra, hvilken \emph{hjemmeside} de nu lige \emph{gav} den data/aktivitet \emph{til} (hvilket Web 3.0 altså gerne skal gøre op med, så al aktivitet kan genbruges på tværs af webbet)), jamen så bliver machine learning-mulighederne også bare totalt forøgede." Ja, alt dette er super vigtigt, og jo, jeg tror jeg har noget at bidrage med i denne forbindelse, men i bund og grund er dette nu ikke nogen ny tanke. Jeg læste lige en kort artikel her tidligere, fordi jeg lige ville se, hvordan den Web 3.0-definition var, som kunne inkludere blockchain i sig som en "Web 3.0"-teknologi, og her blev bedre fremtidig ML også nævnt som en vigtig ting ved Web 3.0. 
%Så nu vil jeg altså lige overveje, om ikke jeg alligevel kan sige noget ret slående omkring at starte Web 3.0 via en open source web 2.0-side (og altså en web 2.1-side), og hvis jeg kan det, så kan jeg måske bare tease mine andre web-idéer (altså tease deres eksistens rettere..) og så dele det op, så jeg laver et nyt dokument med alle disse (eller måske endda med en yderligere opdeling).. Umiddelbart kan jeg ret godt lide denne nye strategi, hvis den holder (og jeg skal altså lige tænke lidt mere over den), for selvom den forrige også nok er ok, så kan jeg altså ikke lade være med at føle, at jeg lynhurtigt vil få læsere til at løfte øjenbryn over den "spredehagl af idéer," som det jo nok let ville kunne virke som.. ...
%Ja, jeg kan altså ikke rigtigt komme uden om, hvor god denne strategi-idé virker.. Jeg har nemlig også tænkt over, hvor oplagt det så ville være at fremhæve synergien imellem denne idé og så donations-kæde-idéen.. Og i sig selv er der jo så altså også Web 2.1-argumenterne, nemlig at en Web 2.0-side, der begynder at inkludere designet og rammerne generelt for indholdet i, hvad brugerne kan skabe og uploade, virkelgt må kunne udkonkurrere alle andre Web 2.0-sider (hvis ikke disse følger trop). Og så synes jeg nemlig endda også, at man kan argumentere for, at en open source-version af en sådan side vil udkonkurrere alternativerne, for her vil man jo klart kunne lokke flest (open source-)programmører til.. Ja, man må her klart kunne lokke mest arbejde til fra brugerne, nemlig når brugerne ved at dette arbejde ikke bare (nærmest) går ned i ejerne af hjemmesidens lommer.. Og ja, med min donationskæde, der kan booste denne opbakning omkring open source-projektet, så synes jeg altså, at jeg har nogle ret holdbare argumenter.. Jeg skal forresten huske at nævne, at man ligeså godt også kan udstede tokens på baggrund af donationer, for selvom alle donationer ikke herved kan blive refunderet, så vil det stadig give god mening lidt at behandle de første donorer i fremtiden som en slags aktionærer på projektet, og altså donere penge i sidste ende til de tidlige donorer, hvis det bedømmes, at deres aktivitet (i form af deres tidlige støtte og opbakning) var en vigtig hjælp til at sætte gang i projektet. ..Ah.. Hvor er det bare dejligt, hvis det her kan give mening..! Og det synes jeg, det ser ud til..! Og så vil jeg så som sagt bare lave et ekstra dokument.. ja, eller flere, rettere, og altså med resten af de web-idéer, jeg gerne vil nævne. Og jeg kan nemlig fint bare starte med de mest overordnede i første dokument, og så bare skrive resten efterfølgende. Fedt, fedt, fedt..!... 

%*Der er flere web 2.1-noter længere nede. (Omkring 6-7 paragrafer længere nede..)


%Okay, jeg synes lidt jeg har nogle hængepartier i disse noter, som jeg lige bør uddybe noget bedre. Dette kan jeg så bare gøre her om aftnerne imens jeg arbejder på udkastet til den primære ("første") udgivelse. Foruden min fysik-sektion, hvor jeg mangler at skrive et par ting, som ligenu bare står ret løst i nogle kommentar-noter, så synes jeg også lige at jeg bør tage at uddybe mine p-modeller engang hurtigt. Det gør jeg nok bare under "Andre opfølgende noter".. Jeg vil også gerne skrive videre på det, jeg har påbegyndt om mine idéer til at implementere en side, der indeholder et emne-træ, et prædikat-træ, og hvor alle prædikater i reglen vurderes ud fra de prædikat-faner, man finder frem til. Vi snakker altså lidt mit sidste punkt i min disposition lige her ovenfor, men måske bør jeg også netop starte med lige at give den mere abstrakte version først, hvilket så svarer lidt til nogle af punkterne i Web 2.0-sektionen i dispositionen (altså lige over Web 2.1-sektionen). Så det vil jeg nok gøre. Derudover bør jeg forklare mere om punkterne "Open source servers.. Open source businesses to implement a framework for web 2.1 sites.." og "Efficient ways to ensure anonymity of users..", men det kan være at jeg alligevel skal inkludere disse i den "primære udgivelse," så måske skriver jeg bare om dem der.. Det må jeg lige se.. Og ellers så mangler jeg måske bare lige at tænke over og uddybe, hvad jeg skal gøre med "algoritme pointene," som altså lidt handler om bl.a. (særligt) at åbne op for, at brugere kan implementere p-modeller på hjemmesiden. For selvom man kan meget med mine "brugergrupper," så skal der alligevel lidt ekstra til (som jeg umiddelbart ser det), hvis brugere skal kunne indsætte deres egne variable i udregningerne og så få serverne til udregne resulterende pointværdier, der altså så kan afhænge af deres variabel-inputs.. Så her har lige altså også lige en lille ting, jeg lige skal have tænkt over og så skrevet om her i disse noter.. Men ja, så umiddelbart kan jeg altså bare lige komme på tre punkter foruden fysikken, som jeg mangler at skrive om, og som ikke hører til den "primære udgivelse," og så er der altså også lige to punkter, som kun måske gør (og som jeg vist mangler at skrive nogle ord om).. ..Så det er jo ikke så galt.. :) (18.12.21)
%(19.12.21) Ah, der er forresten også lige idéen om en PoPublicHistory(/Record..)-kæde, som på en måde er en mere decentraliseret version af en PoS-kæde.. Det handler om at hvem som helst kan melde sig som instans (og uden noget bestemt formalia for at gøre dette) og begynde at underskrive udsagn om, hvilke blokke har været offentlige til bestemte tider. Jeg fik også lige en tanke her til formiddags (er stået op ret sent i dag, ligesom i går (næsten.. ah, minus en lille time)), om at brugere så selv skal melde deres punge til visse instanser, således at hvis det skulle ske, at der bliver en uenighed mellem flere instanser, hvilke transaktioner gik igennem før andre, så er det.. brugernes instanser, der i sidste ende får lov at bestemme.. hm.. Hvad hvis der dog nu er mange handlende i en enkelt blok..? Hm, hvordan gør man det mere decentraliseret uden at åbne op for en forstørret risiko for (hard) forks..? ..Hm, hvad med at de handlende bare får lov at stemme..? ..Ah, men hvem gider at støtte en instans, der laver rav i den?. .. ..Hm, kan godt være, at jeg bare kan tænke over dette senere i dag, men lad mig lige prøve at slå fast, at spørgsmålet vel er, hvad man gør, hvis nu to transaktioner dukker op til offentligheden, som er i strid mod hinanden, og hvor man ikke tydeligt kan se, hvilken en kom først.. ..Hov, jeg kom lige på en tanke, der måske kan føre til en anden idé til et blockchain-angreb.. Det må jeg også lige tænke videre over.. Okay, lad mig bare tænke videre over disse ting senere. ...
%(..Hm, jeg kan se, at sidst jeg tænkte over en idé som PoPublicHistory, der kom jeg frem til, at det åbnede for mange spørgsmål, og at det ikke var arbejdet værd at tænke videre over..)

%(22.12.21) I går aftes kom jeg frem til, at PoPublicHistory-kæde godt kan lade sig gøre. :) Når det kommer til spørgsmål om, rækkefølgen af, hvornår modstridende kontrakter blev uploadet, så kan man nelig bruge et system, hvor parterne kan settle spørgsmålet ved et spil, hvor den, der har den længste kæde af.. jeg mener, at det er logaritmer, der bl.a. indgår i time locks (for at løse dem), sandsynligvis vinder. Hele kæden fungerer så i praksis meget som en centraliseret kæde, bortset fra, at der kan være arbitrært mange "centre" (i.e.\ knuder som kører kæden) i gang på samme tid.. Hm, og pointen er så bare, at knuderne til sammen skal arbejde på at gøre det meget klart for eftertidem, hvilke transaktioner/kontrakter har været, og i hvilken rækkefølge. Det vil sige, at alle knuder skal synkronisere (eller hvad vi lige skal klade det), medmindre de har opdelt sig i grupper, der midlertidigt varetager disjunkte mængder af punge. Hvis en knude går rouge og/eller ikke synkroniserer rigtigt, så bliver dem sparket ud af de andre. Hvis mange knuder nok (skarpt under halvdelen) stopper med at fungere på én gang, så bør netværket genstartes.. Hm, man må endda nok kunne lave nogle helt formelle parametre, som netværket skal opfylde for ikke blive genstartet. ..Hm, jeg kan mærke, at der er nogle ting, jeg skal tænke over.. Det ville være rart, hvis man havde et eller andet system, som man ved vil kunne blive ved med at ekko'e tilsendt data til offentligheden.. Her kunne man i princippet bruge en eksisterende kæde, hvis ikke dette ville ødelægge formålet.. ..Hm, men er det ikke netop bare at sørge for, at alle potentielle købere bør sørge for, at opsnappe alle offentliggjorte underskrevne del-kæder.. Hm.. ... Ah, men man skal da bare bruge et netværk af ekko-knuder (som egentligt bare svarer til nogle database-servere med åben læse-adgang og åben, på nær PoW-begrænset skriveadgang), og så kan offentligheden nemlig bare løbende teste disse. Vil dette ikke fungere..? ..For disse vil så automatisk kunne synkronisere med resten i (del)-netværket, medmindre det har DoS af en eller anden grund.. ..Og alt dette vil så kunne testes af offentligheden.. ..Og man vil ikke kunne snyde og beskylde en server for ikke at fungere, hvis den gør det.. ..Så bliver enhver overvågningsinstans ikke nødt til at erklære sig enige med andre ov.-instanser, når det kommer til alle funktionelle knuder..? ..Og hvad med angreb, der bruger.. hm, der overtager størstedelen af knuder.. hm ja, her vil offentligheden så meget let kunne se lynhurtigt, hvilke nogle instanser lyver.. ...Hm, jeg tror altså næsten, det holder.. Så man har bare et netværk af knuder, som gemmer tilsendt data, hvis det indeholder lidt PoW, og ved så at sende time-locked data rundt til hinanden kan knuderne altså automatisk synkronisere i netværket. Ovenover dette har man så arbitrært mange overvågningsinstanser, hvilket kan være hvem som helst, der kan sende data til knuder og tjekke, hvis en knude ikke synkroniserer (hvilket kun kan ske ved en (indre eller ydre) DoS), om der virkeligt er en DoS til stede.. Hvis de også finder, at der er en DoS bør de afskrive knuden fra netværket. Folk der har pending transaktioner, som har brugt omtalte knudes (eller knuders, hvis der er tale om flere på én gang), bør så sørge for at sprede transaktionerne til flere knuder (der ser funktionelle ud). Hvis to overvågningsinstanser i sidste ende bliver uenige om et resultat, så.. Hm.. ..Ah jo, det kan stort set ikke lade sig gøre, medmindre den ene lyver så vandet driver.. For den eneste måde, at der kan komme rav i transaktionshistorikken, er hvis to eller flere grupper af alle offentlige knuder ikke synkroniserer. For ellers kræver det, at et knude-nætværk i hemmelighed bygger en "kæde," der efterligner den lødige, men dette vil være alt alt for let for fællesskabet at opsnappe. Alle interessehavere (stakeholders og potentielle fremtidige stakeholders (i.e.\ folk der følger med i "aksienkursen" og holder øje med, om de mon skal købe eller ej)) i kæden kan så let følge med i, hvilke nogle offentlige knuder er i spil. Og hvis to eller flere grupper af knuder i samme (del-)netværk ikke synkroniserer, så vil dette kræve aktiv og vedholdende (og ret fuldkommen) DoS hos nogle af grupperne, og dette vil man så let kunne udpege. Så det vil altid være nemt at finde frem til den "rigtige" (og altså oprigtige) gruppe af knuder.. ..Ja. Og pointen er så, at hvis to grupper kortvarigt ikke synkroniserer, og hvis nu de har modstridende transaktioner optaget i sig hver især, hvorved rækkefølgen (midlertidigt) vil være udefineret, så skal der nemlig bare være en protokol (og dette kan man nemlig godt gøre), hvor man får bestemt den endelige rækkefølge, når de begynder at synkronisere igen. Og måden denne protokol kan fungere på, er nogenlunde bare (men måske ikke helt..) ved at se, hvilken en af transaktionerne har den længste logaritmekæde forbundet med sig. (Jeg kan tage fejl, men jeg husker det som nævnt som om, at time locks fungerer ved, at folk skal tage logaritmer for at løse koden (og hvor koden så nemt kan genereres via eksponetielfunktioner), og det er altså derfor, jeg kalder det logaritmekæder. Vi snakker altså kæder af data, som kan bygges med en hastighed, der (så vidt vi ved) er begrænset af, hvor hurtig ens processor(-kerne) kan køre.) Protokollen kan så indebære, at det to parter, hvis nu der af en (sjælden! (for dette kan næsten kun ske, hvis nogle bevidst ofre penge for at prøve at trolle kæden)) eller anden grund er delt interesse for de to parter, om hvad vej resultatet skal gå (hvilket kun sker, hvis de har gjort en favorabel handel, som de er bange for ikke vil gentage sig ellers). Hvis der er dette, kan disse altså så kæmpe om rækkefølgen ved løbende at uploade længere og længere udgaver af en logaritme-kæde til netværket, hvor logaritmekæden skal tage udgangspunkt i noget data (som altså indgår i roden af kæden), der indeholder et hash af den ønskede transaktion. (Hm, indskudt: Vi kunne også bruge hash-på-hash-kæder i stedet for logaritme-kæder.) Protokollen stopper så, når man opdager, at en af de uploadede kæde-udgaver, har en logaritme-kæde (eller hash-kæde), der efter en bestemt længde rammer et target. Så på den måde opnår vi altså et kapløb, hvor sluttidspunktet først findes efterfølgende (hvor hvor "de løbende" altså ikke med det første kan vide, hvornår "slut-skudet" (hvis der fandtes sådan et uden for denne metafor) har lydt). Men man kan så finde frem til slut-tidspunktet efterfølgnede, og så kan man her bare måle på, hvem der har ført mest i gennemsnit i løbet indtil da. Hvis folk virkeligt har meget på spil, så kan de jo leje en hurtig processor-kerne, og dette er fint nok. Sådanne settlements bliver så sjældne, og som regel så ligegyldige, at det ikke gør noget. Og den enste grund til, at der skulle være noget at kæmpe om til at starte med, er hvis en sælger er kommet til at lave en vildt favorabel handel for to forskellige købere på én gang, hvilket nærmest kun kan ske pga. en lyst til at trolle, og hvorfor skulle resten af fællesskabet så bekymre sig om resultatet her. Vedkomne kunne jo også bare have fået de to potentielle købere til at hoppe og danse efter handlen på anden vis. Ok. Nå, jeg skal nok lige tænke lidt mere over idéen, før jeg kan sige, at den virker, men umiddelbart ser det ud til at virke. Nå ja, og grunden til at alt dette i sidste ende virker, er så også, at alle de interesserede parter fra start har gået med til, at det hører med som en "regel" for kæden, at det kun er tansaktioner, der har indgået i offentlige netværk, der duer. Alle transaktioner, der kun findes på forks, hvor knudenetværks-gruppen tydeligvis har haft et længerevarende DoS (eller har kørt ikke-offentligt), og hvor disse så også efterfølgende har nægtet at synkronisere med resten på den rigtige måde, gælder ikke. Og hvad skal tiden så i øvrigt være for, hvornår en gruppe får lov at synkronisere igen med resten af netværket? Ja, dette skal jeg jo lige besvare (og ret præcist endda, desværre..).. ..Ja, det er her, det hele ligger nu.. Hm, og hvis man altså bare kunne gøre, så at friske transaktioner på DoS-knuderne vil få mindre og mindre værdi i det efterfølgende (potentielle) kapløb, jo længere DoS'en varer.. (Men hvordan?!...) ..Ah, men kunne man ikke netop bare måle på, hvor meget synkroniseringsarbejde, de har været bagud med, før gensynkroniseringen..?!.. ..Jo. :) Så man kan sørge for at definere et mål før, hvor synkroniseret en knude er med resten, hvilket så ikke nødvendigvis kan måles i øjeblikket, men hvor udregningen konvergerer med tiden (og for alle parter, for det er jo som sagt nemt for fællesskabet at sørge for løbende at pege på, hvilken delmængde af netværket er "den rigtige"). Og når man så efterfølgende kan se, at et gruppe af knuder har haft lav grad af synkronisering på et tidspunkt, så vil kontrakter fra det tidspunkt, og som sidenhen har indgået i et kapløb, skulle vægtes så meget desto mindre i dette kapløb (og altså når man skal udpege den endelige vinder). Nice! Uh, jeg håber dette holder. :) (Ikke at jeg tror, idéen bliver kæmpe stor sammenlignet med andre database-løsninger, men jeg tror nu alligevel, den kan blive (om ikke \emph{kæmpe} så) \emph{rigigt} stor.. :)) (Tja, men den kunne i princippet godt blive \emph{den} nye kryptovalutta, hvem ved? :)) Det ville under alle omstændigheder være en rigtig dejlig teknologi at vide eksisterer.
%... Ja, jeg tror altså, det virker. Så med andre ord skal vi altså bare bruge et servernetværk, hvor serverne hver især lover at gemme time-locked data (hvis det lige kan proof'e lidt work for at undgå for høj kapacitet) og offentligøre det gemte data med det samme. Alle servere synkroniserer så løbende ved at at oploade time-locked data til hinanden (over et VPN-netværk med høj anonymitet, så serverne ikke kan se, hvem kilden til dataen er), som indeholder og underskriver servernes egen tilstand ved det givne tidspunkt. Man skal så lige finde en god forskrift, hvor man kan søge i servernes historik og nu frem til en værdi for, hvor "synkroniserede" de var til et givent tidspunkt. Denne værdi skal altså så stige, jo mere servernes tilstand til det givne tidspunkt kan finde underskrevet hos de andre servere (i sidenhen oplukket time-locked data), og jo mere serveren selv indeholder de andre serveres tilstande (altså dem der indeholdt sereveren tilstand, så værdien altså afhænger af, hvor mange referener der er frem og tilbage imellem tilstande hos serveren). Denne værdi kan så i sidste ende komme til at afhænge af, hvilken gruppe af knuder, som interessehaverne beslutter sig for er den rigtige (hvilket de gør ud fra nogle klare principper, nemlig om at skille sig af med knuder, der åbentlyst nægter at synkronisere ordentligt, hvilket vil sige at de opretholder et (måske indre) DoS i en længere periode). Så hvis en masse knuder lige pludselig vælger at operere i hemmelighed og bare kommunikere med hinanden, så kan de godt midlertidigt nå frem til, at de har en "høj grad af synkronisering," men denne værdi vil så falde drastisk (og sandsynligvis til 0, medmindre de skynder sig at synkronisere med resten igen), når de så i sidste ende vil prøve at synkronisere med det offentlige fællesskab. Og i sådanne forbindelser vil det altså ikke være nogen sag for fællesskabet at finde ud af, hvilken knude-/server-gruppe, man skal tro på, hvis de nu begge kræver, at de bliver anset for den oprigtige gruppe. Det vil altid være nemt at nå frem til, hvem der opfører sig uoprigtigt, for folk vil nemlig have rig mulighed for at teste spørgsmålet selv ved at sende data til de to servergrupper, og se hvem der fungerer korrekt. Og hvis den ene gruppe så stopper med DoS, så vil der ikke ske andet, end at man så skal i gang med en protokol (nemlig det ovenfor nævnte "kapløb"), som sørger for at bestmme den endelige rækkefølge over transaktioner i netværket. Nå ja, og transaktionsrækkefølgen bestemmes også normalt ud lidt fra samme procedure, hvor to parter, der måtte have modsatrettede interesser ift. rækkefølgen så må sørge for at uploade time-locked data til netværket løbende i et slags kapløb, hvor man så kan se i sidste ende (afhængigt af hvilken undermængde af servere, man har udpeget som den "rigtige"). Og her gælder det så om at uploade til servere, der har en (tilsyneladende) høj grad af synkronisering med resten, for så tæller ens data mest i kapløbet. Men normalt vill dette ikke blive nødvendigt, for hvis en modstridende transaktion først uploades til servernetværket i nogen tid efter, at den originale transaktion blev det, så vil kapløbet ikke være til at vinde. For når serverne selv synkroniserer, så før transaktionerne automatisk mere vægt i servernetværket, skulle et kapløb startes. For værdien som en transaktion opnår, der bestemmer, hvor "godkendt" den er (ift. andre modstridende transaktioner) vil nemlig være være bestemt af, hvor meget "lineært arbejde," hvis vi kan kalde det det (altså hvor lang man kan bygge en sammensat kæde over logaritme- eller hash-på-hash-udregninger, som indeholder transaktionens hash ved roden), der er bygget oven på arbejdet, og så vægtet med, hvor synkroniseret dette arbejde var med resten af netværkets data på det give tidspunkt. Så selv hvis man har en lynhurtig processor, der med tiden vil kunne overhale netværkets mængde af lineært arbejde, så vil man ikke kunne dette med denne synkroniseringsvægt taget med i udregningerne. For hvis man skal opretholde en høj synkroniseringsvægt, så skal man også offentliggøre sit arbejde mere og mere i netværket, og også inkludere mere og mere af netværkets tidligere arbejde i sit arbejde, hvilket tvinger en til på et tidspunkt at inkludere data, som indeholder et længere kæde af sin modstanders transaktion, men så har man tabt, for i det øjeblik man indkludere arbejdet for dem originale transaktion, så vil man fra det punkt af lægge ligeså meget lineært arbejde oven i denne transaktion, som den man gerne ville overhale den med. Så ja, vi skal altså lige finde en protokol, hvor opretholdelse af "synkronisering" medfører, at man bliver nødt til at underskrive mere og mere eksternt data fra resten af server-netværket, hvilket så betyder at historikken om alle tidligere transaktioner bliver nødt til at snige sig ind i arbejdet, medmindre man vil droppe til en synkroniseringsgrad på 0. Og hvis man får inkluderet tidligere arbejde med en modstridende transaktion (og husk på at det vil være time-locked, så man opdager altid først dette efter noget tid), så kan man ligeså godt begynde forfra med alt det arbejde, for så arbejder man ligeså meget for den modstridende transaktion. Og hvis man så ikke på det tidspunkt har opnået en højere "godkendelsesværdi" end for den konkurrerende transaktion, så må man altså give op, for det nemlig kun blive sværere og sværere at undgå den tidligere transaktion i arbejdet (og opretholde synkonoseringsgraden). Så efter et vist punkt kan man altså stole på, at ens transaktion ikke vil blive fjernet fra historikken, for så vil servernetværkets eget arbejde, som bygger videre på transaktionen, og som er det arbejde, de alligevel hele tiden skal udføre for at synkronisere sig selv, være nok til at sikre, at ingen, selv ikke med en hurtigere processor(-kerne), kan overhale transaktionen i godkendelsesværdi. Og bum, så har vi en decentral database, hvor alle uploads i sidste ende kan gives en rækkefølge. Og hvem skal så betale for servernes arbejde? Jeg har jo ikke inkluderet nogen mining-løn i systemet, så hvem skal betale? Jamen det skal interessehaverne så bare, simpelthen. Man kunne dog godt inkludere en mining-løn i systemet, om ikke andet så til at starte med. Dette kunne også være en god måde at opstarte mønter på systemet på en fri og åben måde, hvor alle kan komme og mine. Men dette kommer også an på, om man overhovedet vil have en ibunden valuta på kæden, eller om man f.eks. bare vil bruge systemet til NFTs. Og i princippet kan interessehaverne nemlig godt bare sørge for at donere penge i servernes retning. Og hermed bliver der heller ikke nogen idé for serverne rigtigt at spille luskede spil, som sikre dem mere af mining-lønnen, for der er ingen. Så den type (miner-orienterede) angreb skal vi altså heller ikke bekymre os om. En server skal bare sørge for at virke (og være) pålidelig, så den kan tiltrække donationer fra interessehaverne. Og det gode er så også, at serveren protokoller nu ikke kræver særligt meget arbejde. Den største udgift ligger nok bare i at uploade, downloade og gemme data for serveren, for selve arbejdsprotokollerne kan i høj grad bare køres på én kerne (altså dem hvor man opretholder sin grad af synkronisering som server). 
%Cool! Nå, men det kan nu godt være, at jeg må vente med at arbejde videre på denne idé til efter udgivelsen, for jeg vil gerne have den færdigt snart, og jeg synes ikke jeg har tid til, at prøve at inkludere denne (for der kommer nok til at gå meget arbejde til den). Men det er også fint nok. :)^^ (22.12.21)  

%(23.12.24) Jeg bør lige prøve at dykke lidt ind i, hvordan server-netværk skal "synkronisere" efter et DoS. Hm.. ..Det kan være, at man bare skal definere en tærskel for, hvornår to grupper er desynkroniseret, og så kan det bare handle om at ét netværk ligesom skal ansøge om at blive genoptaget i det andet.. Ja, det må blive noget med det. Og så bliver "synkroniseringsgraden" jo bare lav for dem, der vælger at ansøge. Hvis nu et (D)DoS-angreb sker på hoveddelen af netværket, og på en eller anden mærkelig måde sker, så at små grupper stadig kan snakke sammen.. og hvor de også vælger at fortsætte hver især.. Det er usandsynligt, men protokollen skal kunne håndtere alle eventualiteter.. Så er det rigtigt gode ved det hele, at serverne alt andet end lige ikke har noget egentligt incitament for at prøve at kæmpe for at blive dem, de andre ansøger til. For det eneste serverne skal love deres kunder, er bare, at transaktioner sendt til serveren med tiden vil gå igennem. Det er kun enten malicious users eller folk, der potentielt kan komme til at blive ofre for disse, der kan have en interesse i, at transaktionerne sker før snarere end siden. Og hvis brugere derfor bare sørger for ikke at anse transaktioner som clearet, hvis der tydeligvis er DoS i server-netværket, så er der altså ingen fare på færde. Og det gælder jo for alle blockchain-systemer, at der er en vis ventetid, som kan afhænge af, om der sker DoS eller ej. Men for god ordens skyld vil det selvfølgelig være smart, hvis serverne har nogle faste protokoller for, hvem der ansøger til hvem i forbindelse med desynkronisering, bare så tingene kan gå lidt hurtigt i forbindelse med DoS-angreb, hvor problemerne umiddelbart alligevel ser ud til at komme udefra. Og hvordan skal "synkroniseringsgraden" så sættes under mere normale omstændigheder, når der ikke er "desynkronisering?" Det skal jo gerne bare være et mål for, hvor meget en servers tilstand er at spore på andre servere, når man får afkodet deres time-locked data. Og når der nemlig ikke er "desynkronisering," så vil serveren selv (pr. min definition af begrebet) optage og underskrive time-locked data fra andre serveres tilstande, så i denne forbindelse vil en server altså med tiden selv kunne se, hvor synkroniseret den var til et givent tidspunkt, fordi den kan aflæse de andre serveres tilstande i sin egen data og dermed se, hvor meget "kendskab til"/"spor af" dens egen tilstand var udbredt hos disse andre servere. Så serveren kan altså med tiden selv regne sin synkroniseringsgrad ud, såvel som for de andre servere, den har synkroniseret med, og pointen er så, at jo længere tid serverne vil være en del af det samme netværk (og synkronisere i dette), jo mere skal de forskellige serveres resultater, når det kommer til alle de forskellige synkroniseringsgrader, altså matche hinanden (så resultaterne altså dermed konvergerer mod det samme). Nå ja, og forresten er graden ikke kun et må for, hvor udbredt serverens egen tilstand er, men også hvor meget "lineært arbejde," der så er lagt ovenpå den data, der bevidner om serverens tidligere tilstand(e). Cool. :) Jeg tror især det med, at det ligesom skal være en fast tærskel for "desynkronisering" (og så hvordan man skal gøre i tilfælde heraf), lige hjalp til at lappe et hul, jeg ikke havde tænkt så meget over. Jeg skal nok finde flere huller, når jeg engang arbejder videre på idéen, men jeg er tilfreds med disse noter for nu. ^^

%*(28.12.21) Lad mig lige præcisere, at selvom jeg har foreslået logaritme-udregner som en mulighed for det "lineære arbejde," så.. Ja, eller faktisk er dette nok nærmere en rettelse. Det egner sig nemlig ikke så godt, hvis man bagefter.. Hm, men det kan man måske heller ikke.. Nå, det kan jeg ikke lige finde ud af, men uanset hvad, så kan jeg sige, at det lineære arbejde ikke må indeholde nogen (kendte) genveje, hvis man nu kender en kode. Og det må jo lidt ellers være tilfældet, når det kommer til det arbejde, der gøres i forbindelse med time-locked data, for der kan man jo hurtigt låse dataen, hvis man kender koden. Så det kan godt være, at det omtalte linæere arbejde hellere skal være hash-på-hash-arbejde (altså successive hashes), så der ikke kendes nogen genveje, man kan tage (hvis man lærer en kode at kende). For det er nemlig bedst, at knuderne ikke bare til hver en tid kan snakke sammen i det skjulte og afsløre hinandens koder, hvorpå de så lynhurtigt kan producere en masse arbejde, som tilsyneladnede ellers må have taget lang tid for dem at lave. Så ja, successive hashes bliver nok det, man skal bruge til det "lineære arbejde." 





%(23.12.21) Nå, nu hvor dagen i dag lidt går på at samle op på ting og sager, så kan jeg måske også lige forklare, hvad jeg mener med "Open source servers.. Open source businesses to implement a framework for web 2.1 sites.. Hm.." og "Efficient ways to ensure anonymity of users..".. Det første går bare på lige at skitsere en nogenlunde (eventuel) plan for, hvordan man kunne starte en decentral hjemmeside (med tilhørende netværk af servere), som kan begynde at implementere først Web 2.1, hvilket så automatisk, netop pga. det decentrale aspekt, vil føre til Web 3.0. Jeg forestiller mig, at man først sørger for at lave et open source programmeringsprojekt, hvor man for det første bygger en simpel version af en Web 2.1-hjemmeside, og hvor man også begynder at bygge back-end'en, og sidst men ikke mindst begynder man så også at definere protokoller til, hvordan brugerne kan efterspørge ny funktionalitet.. Hm.. ..Ja, og man skal også på et tidspunkt begynde at lave protokoller for, hvordan serverne kan dele arbejdsopgaver op i netværket.. Tja, lad mig lige glemme rækkefølgen et øjeblik, og bare fokusere på, hvad man skal bygge. Man skal bygge en simpel Web 2.0-agtig hjemmeside for det første men tilhørende back-end (open source det hele). Hertil skal man så gerne implementere en organisering af alle ressourcer (og andre "termer") på hjemmesiden, hvor brugerne selv kan deltage og kategorisere ting og sager ved at votere dem ind under forksllige emner/kategorier. Når dette er gjort, kan man nemlig prøve at bygge en protokol imellem serverne, hvordan de hver især kan melde sig til at gemme termer af forskellige emner/kategorier, og også hvordan de så sender klienterne videre til hinanden, hvis de efterspørger kategorier, som varetages af en anden server. Og til en anden side bør man også gå i gang med at definere det programmeringssprog og de funktionaliteter, der gør at brugere kan begynde at uploade ændringer i hjemmeside-designet, som jo er hvad der begynder at gøre det hele til Web 2.1 i stedet for Web 2.0. Og dette leder så videre til arbejdet med at bygge en protokol for, hvordan brugere kan stemme.. Hm, nå nej, det kan jo faktisk godt være et ret sent skridt eventuelt, hvor man ligefrem begynder at åbne op for, at brugerne kan bygge og fremstemme server-algoritmer, der kræver mere og mere grundlæggende magt over back-end-funktioner.. Hm.. Ja, så hvad er det egentligt, jeg overhovedet vil tilføje her? Er det bare det, at man så, når man færdiggør et stabilt open source projekt, kan begynde at oprette server-foreninger og brugerforeninger, hvori man så kan forhandle om, foreningerne imellem, hvilke nogle services serverne skal udbyde..? Ja, måske er det egentligt bare dette, jeg lige manglede at skrive et sted her i noterne. Så Web 2.1-server-netværket skal altså i sidste ende bare forhandle med diverse foreninger af brugere om, hvilke nogle algoritmer, der skal udbydes. Og det gode er, at hvis en forening har en given efterspørgsel, fordi vi snakker Web 2.1., som jo i sin kerne bør være ret fleksibel og brugerdrevet, jamen hvis der så bare er nok, der efterspørger en vis server-algoritme, og at de efterspørgende også kan betale for arbejdet (og fordi det er Web 2.1, behøver de ikke at betale for andet end arbejdet, for brugerne har selv designet algoritmerne), jamen så vil det jo kunne betale sig for servere at udbyde server-algoritmen. Det eneste er måske bare lige, at der også kan gå noget arbejde i at verificere, at algoritmen er sikker (hvis nu den er skrevet i et lidt lav-niveau og ikke helt sikkert sprog), men dette vil jo i så fald så bare være en engangsbetaling. Nå og pointen, jeg så også gerne vil komme med, er lidt bare, at nu med min donationskæde-idé, hvis denne altså bliver til noget, så kan der muligvis meget nemmere komme godt gang i alt denne indledende arbejde, for så kan både alle de indledende open source-programmører og alle de første, der opsætter servere (inden der er kommet gang i alt det med server- og bruger-foreninger osv.) om ikke andet muligvis få deres fair betaling for arbejdet via donationskæden. Og ellers så kan det jo også være, hvis vi er heldige, at folk bare vil være begejstrede for projektet, og simpelthen vil donere direkte til programmørerne.. Hm, selvom i princippet er det næsten mere værd at håbe på, at donationskæden kommer godt i gang, og at folk så faktisk benytter lejligheden og donerer via at denne (og altså ved at købe token-rettegheder fra programmørerne (og server-iværksætterne)).. Ja, det synes jeg, kunne være det bedste, for så ryger hele spørgsmålet om "profit" nemlig automatisk til side: Så vil det være helt klart, at der doneres penge med det formål at skabe mest mulig gavn for fremtiden (og som fremtidige mennesker så dermed sikkert også gerne vil belønne). Okay, dette blev endnu en lidt rodet sektion/paragraf, men jeg tror jeg fik fanget den pointe, jeg syntes, jeg manglede at nedfælde.. Pointen er bare, hvordan jeg altså tror, at der sagtens kan komme rigeligt skub i Web 2.1-idéen, også i form af en nystartet og open source (og decentraliseret) hjemmeside. Og dette er rigtigt dejligt, for hvis Web 2.1-idéen bare startes som forlængelse af en gængs (privatejet) Web 2.0-side, så når vi ikke på samme måde automatisk videre til Web 3.0. ..Nå ja, og desuden kan jeg også lige nævne, at jeg jo også synes, det ville være dejligt med sådan et oplagt projekt til også at sparke gang i donationskæden.. :) 
%Nå, og hvad skulle jeg så sige om "Efficient ways to ensure anonymity of users.."..? Har jeg ikke været inde på det fint nok i mine kommentar-noter under "(old)Web ideas2"..? (Som i øvrigt er lige her nedenfor, og så er det sikkert under "Organized comments," man skal lede..) Jo, det handler bare om, hvordan hver bruger bør opdele sin brugerprofil i flere separate profiler, og bl.a. en profil til standard brug, hvor brugeren sørger for ikke at rate noget, der bevidner om, hvem profilen tilhører, og i øvrigt heller ikke noget, der kan være pinligt, hvis nu andre kigger brugeren over skulderen. Og hvis man så har alle serverne over et anonymt VPN, så kan brugerne altså hermed frit åbne sig op om, hvad de synes om alle mulige ikke-pinlige (og dog også ikke-person-afslørende (dvs. ikke alt for lokale og/eller tæt knyttede til brugeren)) ting i verden. Så ja, det er altså bare de noter, jeg ville kunne gentage her. Brugeren skal selvfølgelig så til hver en tid kunne downloade \emph{alt} (og jeg mener alt!!) deres data, og gerne i et format, så de i princippet let kunne overføres til andre hjemmesider (hvis nu f.eks. en bedre "konkurrent" skulle melde sig på banen; så skal brugerne let kunne overføre alt (alt!!) deres data til denne), og de skal også gerne til hver en tid kunne bede server-netværket om at slette deres data. Og medmindre brugerne har uploadet data, som de har underskrevet, at de ikke længere har rettigheder til at kræve tilbage (hvilket nemlig kan være smart, hvis det er data, som andre brugeres arbejde kan bygge videre på), så skal netværket altså gerne i så fald også slette denne data. (Så med andre ord betyder "decentraliseret netværk" altså slet ikke "lovløst" eller "gør som det passer dem." De må tvært imod meget gerne overholde alle sådan nogle fornuftige regler, som f.eks. at brugere har visse rettigheder over deres eget data --- og eksempelvis retten til at blive glemt.) Cool. :) Har jeg så egentligt flere hængepartier udover fysikken, og udover at jeg selvfølgelig rent faktisk skal have skrevet de ting, jeg skal have skrevet om (altså i udgivbar form)..? ..Ah ja, der er "algoritme-pointene," og så kom jeg i øvrigt også lige til at tænke på, at jeg måske magler at skrive om ressoruce- og liste-HTML-typer under min "Helt ny tilgang"-sektion ovenfor.. Lad mig lige se lidt på sidstenævnte ting først.. 
%Nå, jeg lader lige HTML-emnet stå lidt åbent for nu. Så lad mig lige tænke lidt over de der "algoritme-point," inden jeg holder for i dag (og holder juleferie).. ...Tja, men har jeg ikke været over dem under min nye sektion om p-modeller..? ..Hov, jeg mangler da at tilføje noget der..!.. 
%...Okay, nu har jeg skrevet en kort lille tilføjelse til den sektion, og jeg føler hermed egentligt også, at jeg ikke behøver at tænke mere over de automatiske point (som altså så nemlig kan implementere de endelige sandsynligheder, når brugeren har valgt inputparametre til p-teorien og herefter søger på udsagns-ressourcer) lige foreløbigt. Når jeg har lavet udgivelsen kan jeg måske lige tænke lidt over, om man kan gøre dette på en eller anden særlig elegant måde, men nok først der. Cool. Så har jeg ikke flere hængepartier \emph{på nær} fysik-hængepartiet/erne. Fedt. (23.12.21) 



%Nogle flere web 2.1-noter:
%(29.12.21) Som jeg tænkte på i går aftes, så kunne en simpel version af web 2.1 vel bare være et åbent fællesskab af hjemmesider og skabere (og interesserede, I guess..), hvor hjemmesider og skabere har en aftale om, at hjemmesiderne må og skal dele al data imellem hinanden pr.\ request. Fællesskabet skal som nævnt være åbent, så hvis en ny hjemmeside kan sige, at den vil følge samme principper, så skal den altså optages i fællesskabet alt andet end lige. Hjemmesiderne skal også bare vare open source, og det forventes derfor ikke, at de skal hente deres profit ved at designe hjemmesiderne godt. Det skal brugerne alligevel gerne ende med i høj grad at stå for, hvis målet er web 2.1, hvad det gerne skal være. De skal i stedet bare tjene på fees fra at stille deres servere til rådighed, og så ellers bare donationer og donationstoken-handler ved både at opstarte hjemmesiden og også ved at stille servere til rådighed (hvis nu fees'ne ikke dækker omkostninerne helt). Og skaberne skal også bare forvente at få løn via donationer og donations-token-handler (eller -salg rettere). Og hvad med brugernes data? Nå jo, det skal forresten også være en del af konceptet bag fællesskabet, nemlig at brugernes data f.eks.\ også må og skal deles pr.\ request (med visse undtagelser, som jeg kan vende tilbage til). Og her er det så bare, at man skal bruge mit anonymitets-system, hvor man kører det hele over et anonymt VPN, og hvor hver bruger sørger for at have (typisk mindst tre --- hhv. til person-afslørende, til potentielt pinligt for f.eks. folk, der måtte kigge brugeren over skulderen, og så til normalt, ikke-pinligt og ikke-personafslørende brug) forskellige konti/profiler, så dataen er adskildt, således at ingen kan opsnuse de data-forbindelser, de ikke må se. (Og hvis man så f.eks. gerne vil gøre brug af ML fra den normale profil/konto til søgning på person-relevante og/eller "pinlige" ting, så kan man også gøre nogle ting, for at folk skal kunne dette. Man kan nemlig lave nogle protokoller, hvor folk kan knytte de ikke-"normale" (hvis vi kalder dem det) profiler/konti til en hel gruppe af "normale" konti, hvori de selv indgår.) Så brugerne kan nyde fuld anonymitet samtidigt med at de kan få ML-funktionaliter stort set ad libitum, og skaberne og serverne kan alle sammen få løn og betaling, endda uden at skulle genere brugerne med clickbait og/eller reklamer.(!)   
%Nå ja, og eventuelle web-udviklere, der ikke uploader arbejde via web 2.1-vejen, men som måske er med i en mere tidlig fase, som bl.a. bygger systemet, der gør web 2.1-funktionaliterne mulige (eller som bygger første udgave af hjemmesiden i det hele taget), de kan så også få betaling via donationer og donations-token-salg. 







\section[(old)Web ideas2]{(halvgamle)Web ideas2}

%(19.11.21) Nu vil jeg lige overveje den nye disposition over idéerne.
%Jeg tænker, at jeg at jeg stadig bare starter med en kort sektion om tag ratings. Så kan jeg forklare om organized comments, lidt som jeg havde tænkt det, hvor jeg altså bare snakker om vilkårlige hjemmesider. Men her kan jeg altså både forklare om faner, tag- og link-menuer, unikke faner, relevancy og sti-vurderinger, at have både tommelfinger op/ned og kontinuer rating samtidigt, .. Hm, skal jeg mon også nævne regex her..? Tja nej, der er ingen grund til at gå så meget i små detaljer her (jeg skal jo bare sælge idéen først).. Ok.. Og så tænker jeg nu at fortsætte med en sektion om også at organisere selve ressourcerne på samme måde (via rekursive faner). Her kan jeg så nævne, at dette kan ses som en udgave af et semantisk web. Jeg kan også lige nævne brugerprofiler som objekter kort, og at folk kan uddele point til andre brugere, hvilket bl.a. kan bruges i FOAF-agtige netværker.. ..Hm, eller også kan jeg vente med sem-web- og FOAF-kommentarerne..(?) ..Ja, jeg tror lidt mit mantra bliver at opsplitte idéerne i høj grad og forklare tingene én ad gangen.. Og så kan det godt være, at tempoet i teksten på en måde bliver lidt "langsom," og/eller at man i høj grad skal "læse teksten før man forstår den" (og hvor jeg altså ikke prøver særligt meget at bygge op, hvad læseren kan forvente af idéerne) men sådan må det så nok bare være.. (..Og det gør heller ikke så meget, at sektionerne ikke bliver særligt "ligeværdige" nødvendigvis..) Okay, så i første omgang bliver tredje sektion bare om, hvordan man også kan ordne selv ressourcerne i faner.. 
%Og hvad skal jeg så skrive om derefter..? ... Hm, dette skal jeg faktisk nok lige summe lidt over.. Jeg kunne godt gå direkte til at skrive om brugergrupper, enigheds-bedømmelser, og b.d. ML, men...
%(20.11.21) Ah, i forbindelse med organiserede ressourcer kan jeg godt nævne, for det første hvor bredt systemet gerne må være, og at det desuden kan komme til at kunne bruges som et semantiske web. Derefter kan jeg skrive om FOAF-agtige brugergrupper, hvor man altså uddeler point imellem sig som brugere, og hvor service-yderen så kan oprette "brugergrupper" ved at tage en række sources for en eller flere pointtyper samt en aftagelsesfunktion. Dette er bare et eksempel, jeg vil give; man kan godt finde på andre løsninger. Jeg vil bare lige forklare, at det sagtens kan lade sig gøre at få sådanne "brugergrupper," som f.eks. kan bruges til at administrere no-brainer-kategoriseringer på siden. Herefter kan jeg fortsætte med at se på, "hvad med ting, hvor folk kan have forskellige meninger, kan man også finde brugergrupper, man kan stole på her?" Ja, man kan for det første bruge ML-teknikker, og her vil jeg foreslå, at brugerne faktisk selv får adgang til den fornødne data, så brugerne selv får adgang til korrelationerne. Herved kan engagerede brugere nemlig prøve at finde fortolkninger af korrelationsvektorerne og prøve at sætte ord på dem og forklare dem. Til en vis grad vil man således kunne oversætte visse vektorer til personlighedstræk, og brugere vil så kunne bruge disse til at forbedre deres egne feeds, også selv hvis de endnu ikke har givet så meget data selv. Selvfølgelig vil vurderingsaktivitet altid også være en god måde at finde sine egne korrelationer på, men det er smart at man selv kan gå ind og justere, hvis nu den automatiske ML-algoritme rammer lidt ved siden af for en, og det er også sundt, at man kan se og prøve andre muligheder, og hvad der rør sig for andre brugere. Dette vil faktisk, tror jeg, give et stor positiv bidrag til manges brugeroplevelse, at de ikke er begrænsede til deres egen boble, men selv har kontrol over "boble-algoritmen," og desuden også kan følge med i, hvordan andre brugere har det. Det er også sundt, at brugerne faktisk kan lære en del om sin egen og hinandens psykologi i sidste ende ved at bruge sådan en side, især hvis siden er så omfattende, som den gerne skal kunne være. (Hm, bare jeg ikke springer for hurtigt over nogle ting her; jeg mangler jo stadig at nævne enighedspoint, personparametre (tja, eller det er jo netop, hvad de "engagerede brugere" skulle arbejde på..).. hm, nej mon ikke dette faktisk er en god rækkefølge, for det er kun godt, at visionen kommer før implementationsdetaljer og detaljer generelt for at forbedre systemet..) Det kan forresten godt være, at jeg allerede her skal have nævnt de "personlige parametre," for det kommer meget naturligt sammen med, hvad de "engagerede brugere" skulle arbejde på.. Det næste jeg så kan forklare om (stadig omkring b.d. ML), er at dette kan gøres på flere måder. For det første kan service-yderen bare sørge for kun at udgive korrelationsvektorer baseret på mange nok brugere til, at de bevarer brugernes anonymitet (altså hvad jeg tænkte i høj grad før min tænkepause nu her). Men jeg har også en mere decentral idé. Man kan nemlig også i stedet bare lægge op til, at brugere i høj grad skal bruge anonyme brugerprofiler, når de bruger siden. Jeg har så en hel inddeling, jeg vil forklare om, og jeg har endda en protokol, hvor brugere kan forbinde profiler med visse brugergrupper på en decentral måde. Så det vil jeg forklare om her.. eller hvad..? Er der andre ting, jeg skal sige først (dette hører jo til detaljerne mere end visionerne..)..? ..Hm, jeg kunne godt bare nævne her, at jeg også har en mere decentral løsning, men at jeg bare vil forklare om denne senere. Ja, det tror jeg, jeg vil. ..Så ja, jeg forklarer bare lige om den centraliserede løsning først her (så man lige kan se allerede, at det kan lade sig gøre, uden at man deler sine holdninger til gud og hver mand). Nu kan jeg så forklare om.. hm, men har jeg ikke allerede nævnt enighedspoint i forbindelse med de mere simple "brugergrupper"..? ..Jo, det bør jeg have.. ..Men i den forbindelse kan jeg så bare nævne det i forhold til ting, man let kan overvåge og tjekke (i.e. når det kommer til no-brainer-ting i høj grad), og her kan jeg så nævne, at man også kan bruge enighedspoint til.. ..Hm, men enighedspoint i denne forbindelse er vel især brugbare (hvis de er brugbare..) på et tidligt tidspunkt, så skulle jeg også bare nævne dem i forlængelse af de første enighedspoint..?.. ... Ja, jeg kan godt nævne det her, og ja, jeg bør bestemt nævne enighedspoint i denne forbindelse. Jeg vil således nævne, at dette muligvis kan være en god måde at komme ekstra godt i gang med ML'en, måske især endda inden man når godt i gang med personparametrene. Pointen er lidt, at folk jo ofte vil have \emph{lyst} til at give hinanden point og vurderinger (og "følge" og "like" hinanden), og at dette så altså kan bruges til at booste ML'en. Ok. ..Hm, jeg kunne så snakke om specifikke vurderinger og personparameter-forklaringer af samlede vurderinger, men bør jeg ikke næsten allerede have nævnt specifikke vurderinger..? Kunne (og burde) jeg mon nævne dette under tag ratings..? ..Og/eller under org. kommentar-sektionen..? ..Hm nå ja, det hænger jo nu bare sammen med org.-komm.-sektionen, for det er jo bare at vurdere kommentarer under kritik- og kompliment-fanen/erne.. Ja, så den del giver lidt sig selv, og jeg kunne så her altså tilføje, at man også kunne krydre sine vurderinger ved at forklare, hvilke personparametre, der ligger til grund for dem (og altså forklare fordelingen af, hvilke paramatre, der var betydende for en vurdering). Og dette kan så nemlig også booste ML'en. Hm, men bør jeg nævne noget om, at man også kan give "x ud af y mulige"-vurderinger..? ..Ja, og det kan jeg så lige netop nævne her efter personparameter-vurderingsforklaringerne. Igen er der her en dobbelt værdi i det, fordi brugeren for det første kan udtrykke sig bedre, hvilket både giver en bedre brugeroplevelse og hjælper ressourcens skabere/producenter, og samtidigt kan det så også bruges til at booste ML'en. ..Okay, nu er der vel så bare tilbage at forklare om b.d. design ("web 2.1" som jeg har kaldt det nogen steder her..) og så bagefter endeligt forklare nogle (flere) detaljer om, hvordan jeg ville implementere systemet (bl.a. om anonymitetsprotokollerne og self. også om detaljer omkring fane-systemet inkl. regex-faner, sorteringsindstillingsfaner, at relevans-scoren kan ses som en abstraktion over sti-kommentar-vurderinger.. Og også om protokoller omkring en slags decentraliseret database (hvor servere bare melder sig på nogle data-domæner, og hvor man ikke behøver at gå op i, at query-svarene skal være nøjagtigt det samme overalt og for alle servere og service-udbydere..))..? Er der ellers andet jeg mangler omkring disse ting..? ..Nej, måske var det det. ..Ellers må jeg jo tage det, som det kommer. :) Fedt!
%Så nu vil jeg altså skrive dette. Jeg er forresten kommet endnu mere frem til, at jeg sagtens kan skrive på en ret løs facon og bare tage det sådan lidt hen ad landevejen. Det skal bare være et notesæt, som kan forstås, og hvor mine vigtigste idéer (at få ud i første omgang) er med. Jeg (og andre) kan altså bare betragte det som første version af et notesæt, og hvis der så ikke bliver super meget bid for denne første version, så må jeg jo bare prøve at skrive en mere prængende version og prøve at sælge idéerne lidt mere. Selvom jeg så altså vil skrive det meget ud af landevejen, så kan jeg dog godt lige give plads til nogen subsub-sektioner, hvor jeg kommer med eksempler (for eksempler er jo trods alt altid vigtige(!)).. Fedt. :)



\subsection{Tags with ratings}

%Tags / folksonomies can be found on various websites... (e.g. web 2.0 websites such as Youtube, Facebook and Twitter, streaming platforms such as Neflix etc., and also online shops..)
%With the conventional system, there might, however, be a lot of instances where the tag system does not work so well.. Tags might be shown, that are not very applicable.. Furthermore one is not able to see different degrees of applicaibily.. But if there was a way to add a degree of applicaibily to all tags, the users will not only be able to get a better idea of \emph{when} a tag is applicable for a resource, but they will also be able to get an idea of \emph{how} applicable it is. This open the door up for tags such as scary, difficulty, durability.. I will here present an solution for getting all this..
%This idea here is first of all to add ratings to tags and second of all to make these rating continuous (say going from 0 to 100). When a user sees a tag for ressource on the site, the median of the given ratings should be immediately visible to the user to give a good immediate idea of how applicable the tag is for the resource. And with only two clicks --- one to expand the tag with a rating menu ando one to give the rating --- the user should be able to supply his or her own answer to the tag's rating. *(No, one click. The rating axis should just be shown (maybe in a very bare version) at least when hovering over the tag.) *(The total amount of votes should also be visible.)
%When the rating menu is expanded or dropped down from the tag, the user might be given the possibility to see (def. rephrase) more details about the distribution than just the median. One could even show a whole histogram of the distrubution..
%..Above the rating axis.. one could show for instance stars.. Hm.. 
%The reason why we would want to show the median first above anything else, is that this way users will not get tempted to go to the extremes with their ratings, as they otherwise might with the mean, as a way of influencing the shown value the most.. When showing the median instead each can be seen as a conditional up- or down-vote; they all have the effect of moving the estimator the same amount, just like with up and down votes, but wether they count as a vote up or down now just depends on whether the median is (otherwise) below or above the user's own answer..
%Examples:
%Scary, funny, durable, spoilers, age sensitive content (e.g.\ NSFW),.. and more perhaps...








%Husk:
% - Hm, måske skal jeg bare nævne kontinuere ratings mere til sidst i denne sektion..
% - Husk at nævne min (nye) helt simple udgave af et system omkring "brugergrupper." :) 



\subsection{Organized comment sections}

%A lot of websites (similar to the ones mentioned above) also have comment sections where user can read and leave their own comments about the resource.. The idea I will introduce now is about a way of organizing such comment sections..
%..Instead of giving a motivation for this idea first, let me just go right ahead and explain it. Hopefully I will be able to make some of its usefulness clear as we go along..
%The idea is to let the users be able to add tabs to the comment section themselves and then to vote the various comments into different tabs depending on where they think they belong.. *(No, actually users can first of all just choose which tab to upload their comment to. And later on I will also talk about how users work together and help each others sorting comments into the appropriate tabs, even if the comments was not originally posted in that tab (and perhaps if say the tab didn't exist at the time of upload..)..)
%Examples of tabs could be.. Reactions, critiques, positives, links to source material, link to original sources (if the content is (partially) comming from or based on other resources..), discussions based on elements of the content (for instance claims or opinoins voiced in the resource (which could for instance be a video, image or text (or other things; game, movie..))), discussions about the resource as a whole (or the product or the real-world entity it represents).. 
%The users should be free to add whatever tabs they like (as long as the do not break any policy rules of the site of course), and the tabs themselves should also have a rating, such that the most popular tabs will be shown first in each tab bar.. This way the users can by the way also use tabs as a way to express a want for certain information regarding the tab --- or a want to see / have a discussion of a certain subject related to the resource. 
%I will also suggest that tabs can be layered.. I have mentioned a couple of tabs above regarding links to other resources (internal or external to the site btw..); but we could then also have a general tab called 'links' first of all and then have this tab diveded into several subtabs.. I can also mention some other examples.. I will then suggest an expandable/collapsible tab menu consisting of tab bars and where more and more bars can be added.. (def. rephrase..) ...Uh, og man kan forresten godt have en scroll bar for tab-menuen.. One could also use this for layered discussions.. *(The main tab of a resource should by the way just be interpreted as "relevant comment for the resource"..)
%Regardles of whether a tab has subtabs or not, when viewing a tab there should always be an associated list of comments.. If a tab is also further divided into subtabs, the comments shown will just be a union of all the subtabs.. The comments can then be shown in order of their rating.. (One could also make it so that the algorithm will try to mix comments from the various (top rated above a certain threshold) tabs.. (rephrase all this to be more simple..)
%..So if a user uploads a comment to a subtab, it will automatically be uploaded to the main tab(s) as well (all the way up).
%Let us now address the matter that we of course cannot be completely sure that users will always make use of the tabs. In fact when the system is forst implemented, it will probably take some time before users gets used to posting their comments to the right tabs.. And of course users cannot post to tabs before they have been created.. So what can the users do if they see a commet, which they find valuable, but which they think belong more to another tab?.. Well, one solution could be to simply let users vote comments into other tabs; if enough users votes a tab, one could then make it so that the comment will be added to this tab as well. But in a way, the semantics of a comment might in some cases be dependent on the tab it was posted under. If we for instance have a tab with source material links.. Hm, this will get a bit long.. I could also just say: I have another solution: Have a special kind of comment that users can post, where the cannot say.. Hm, lad mig nu lige tænke lidt mere over dette (for det kan ikke være rigtigt, at jeg skal bruge så meget krudt på det; det må være noget mere simpelt, man kan sige..)..  


%(26.11.21) Jeg er gået hen og blevet slugt lidt af QED-tanker, men lad mig lige skynde mig at nævne i det mindste, at jeg indtil videre er kommet lidt frem til, at kommentarer simpelthen bare forbliver i det faner, de blev uploaded til, og så må man som forfatter eller anden bruger bare kopiere kommentarerne til andre faner, hvis man vil det, og ellers skal der så dog bare stadig gælde, at en upvote i en sub-fane også medfører upvotes i alle forælderfanerne.. Men ja, jeg skal dog lige tænke noget mere over det.. 
%(04.12.21) Nej, man skal ikke "kopiere." Folk har også lyst til ikke at plagiere, mange har i hvert fald. Nej, der skal helt klart være en reference-kommentar --- ah, måske hvor man nærmest copy-paster, for nu tænker jeg nemlig, at der alligevel skal være sådan en slags udklipsholder..! --- hvor man refererer til en kommentar, og hvor denne så kan foldes ud. 


%(04.12.21) Jeg er lige kommet på nogle nye ting (i går aftes/nat og nu her til formiddag i dag)! Man \emph{skal} kunne give relative rating, og det kan så specifikt ske ved, at man flytter (eller rettere copy-paster) kommentarer op og ned i en tab som bruger. Når man flytter en kommentar til et nyt sted, så skal det medføre en relativ rating/vote ift. de 3-5 kommentarer under eller.. ah, måske skal man bare have nogle ranges/selektioner over og under kommentaren lige efter, den er flyttet, hvor der så som standard kan være selekteret 3 ressourcer/kommentarer i modsatte retning, end den man flyttede (og måske én i den samme retning), men hvor man så kan selektere og deselektere ress./komm.'er ad libitum, inden man konfirmerer, hvor alle selekterede elementer under den nye plads så skal give en relativ votering, hvor flyttede element vægtes højere end de nedenstående, og modsat i den anden retning. Virkeligt fedt, det her! For det næste er nemlig så (som jeg lige kom på nu her), at man jo også har sorteringer, og dermed kan man altså sagtens få vist en undermængde af en tab, men hvor man stadig rater/voterer ift.\ den underlæggende (ikke-sorterede) tab, når man enten rater eller laver relative voteringer/ratings. Så alt det jeg har tænkt med at have en akse, hvor man kan vise de mest relevante ressourcer(/kommentarer) for brugeren (hvilket vil sige generelt populære ress'er, samt ress'er som brugeren selv har angivet kendskab til (ved at have ratet dem)), så denne bedre kan bestemme sig for en rating-værdi, det kan man nu bare implementere med en tab og en sortering, og hvor brugeren jo så kan rate ved at flytte ress'en! :D Og noget, som jeg også kom til at tænke på i går nat, er så, at tags jo måske i virkeligheden kunne implementeres som tabs i stedet for som kommentarer..!(?) For det passer jo så ret godt med, at fanerne (de ikke-sorterede) kan betragtes som prædikater/relationer.!.. ..Og så kan man (som jeg også tænkte på, inden jeg faldt i søvn) jo evt. bare også have faner, som er en salgs omvendt-faner, hvor.. ja, hvor elementerne i fanen så i virkeligheden er faner, og hvor ratingen så er identisk med ressourcens rating under den fane, kan man ikke gøre det sådan? Og så kan denne måske over- eller under-inddeles med over-faner..? ..He ja, det virker da, gør det ikke? Kan man ikke have en fane over alle relevante faner --- startende fra en vis overfane --- i bund og grund, som så kan underinddeles helt ligesom at denne overfane underinddeles..?.. ..Uh, og det gode er så, at så kan man også se faner, der måske slet ikke er en del af ens fanetræ, men hvor ressourcen alligevel hører til. Man kan altså.. Ja, det svarer lidt til at have en fane, hvor man kan se alle de områder, som ressourcen er relevant for. Yes!.. ..Yes! :D^^ ..Får vi så to fane-faner, eller kan man blande dem lidt sammen..? ..Ah nej, det bliver vel bare en underfane-fane og en overfane-fane..? Hm, men lad mig lige overveje, hvor godt bliver det at rate ting i overfane-fanen, når man så ikke kan se, hvad der er relativt til? Men det kan jo så være, at man bare som udgangspunkt ikke skal kunne rate direkte i overfane-fanen, men at man skal besøge overfanen (og gerne hvor man bare kan klikke sig direkte over til denne fane og med scrollet ned til den pågældende ressource). Ja, det må da bare næsten blive sådan. :) Ej, jeg føler altså virkeligt, at dette er nogle vigtige tilføjelser..! :D (04.12.21)
%..Og det med at hive ting så som navigationslinks og tags, det kommer nu til at blive mere simpelt, for det kan bare ske med en HTML-definition og så ellers bare en fane- (og sorterings-)selektion.. hm, men kræver det så ikke at fanerne går igen, hvordan var det nu lige, jeg sikrede mig.. ah.. Ja, man kan jo netop bare sørge for at copy-paste faner til underfane-sektionen.. Ja, så siden skal altså lige implementere også, at når faner copy-pastes til en underfane-sektion, så.. Ah, men faner skulle jo også være en speciel type på siden (og som på en måde repræsenterer en prædikat-/relationstype). Så alle faner bliver altså også en selvstændig enhed, som kan slås op under et overordnet fanetype-træ (på lige fod med...).. Hm, nå nej, det bliver så ikke to adskilte træer, for ressourcerne bliver jo så bare de sidste blade på træet.. ..Hm, ah, bortset fra at faner netop gerne skal kunne gå lidt igen i ressource-træet..(?).. ..Ah, skal man ikke bare have en søskendefane-relation på siden, således at faner kan erklæres som søskende til andre faner og dermed blive en del af en søskende-gruppe? Og når man copy-paster en fane, så skal denne så bare automatisk sættes som søskende til den originale..? :D ..Yes! ..Hm, og søskende-fanegrupper bør så næsten faktisk være en separat type på siden, som kan få sit helt eget træ, hvor man så altså kan slå forskellige (søskende-)fanegrupper op.. ..Yes.. ..Fedt! :D 
%... Uh, og man kunne så også gemme specielle sorteringer netop for (søskende-)fanegrupperne!.. Uh, indskudt: Faner skal så også have de to specielle faner af "underfaner" (hvilket særligt er vigtigt) og "overfaner." ..Ja, det er klart.:) Og tilbage: Ja, og man kan gemme sorteringer for faner også, så når man klikker på en fane, hvor man har valgt en speciel sortering til, så vil denne sortering træde i kræft, når man browser elementerne i denne fane. Og ja, når man så gemmer for en hel søskende-fanegruppe, så gælder det samme selvfølgelig, bare for alle medlemmer af gruppen. Hm, måske kunne man også åbne op for andre fanegrupper end bare søskende-fanegrupper..? ..Nå nej, det hedder jo bare en overfane, så never mind. ..Tja, og alligevel; man kunne måske godt give mulighed for, at en poster af en fane kan erklære fanen som medlem af en fanegruppe (og så kan brugere jo bare lade være med at upvote fanen, hvis den er dårligt erklæret..).. hm, måske bliver det lidt for kompliceret ift. brugbarheden..?.. ..Ja, men nu kom jeg også til at tænke: hov, hvis faner ses som prædikater, kan de så ikke bare (måske i nogen tilfælde) være prædikater i konjunktion med overfanen/erne, i stedet for at præsentere fulde prædikater? Ah, der har vi den..!.. Underfaner skal jo tit ses som prædikater i konjunktion med overstående, så underfanerne kan naturligvis godt være selvstændige entiteter også, som så bare kan bruges til at lave absolutte prædikater, som danner absolutte (under-)faner. Ja, eller rettere: underfanerne \emph{er} så de absolutte faner, men det får så også automatisk en (søskende-) relation til, i første omgang det sidste prædikat i konjunktionen, som specificerer fanen i forhold til.. hov ja, det bliver forvirrende, hvis jeg kalder det 'søskende,' for faner der står side om side i en bar kan jo også meget vel kaldes søskende.. men ja, som specificerer fanen ift. dens omgivende faner, og dermed altså indirekte også til alle andre faner, der ender på samme (del-)prædikat. .. ..Hm, og man kunne så godt have et "fane-træ" og altså indføre prædikat-prædikater for.. Ah, nu kan jeg se det: De normale træ kommer til at bestå.. ah, af først et træ af prædikater, hvor relationerne er triviel, for et absolut prædikat vil kun have de forældre, som består af prædikater med et reduceret antal del-prædiakter i deres konjunktion. Så $p_1 \land \ldots \land p_x \land \ldots \land p_n$ vil altså være barn af $p_1 \land \ldots \land p_n$, uagtet af hvor $p_x$ stod i rækken. (Når rækken af fane-barer vises, så viser man dog selvfølgelig bare de forældre, som man navigerede fra (så det altså fungere ligesom et normalt (Windows-agtigt) GUI med faner --- bare hvor man altså godt kan navigere til samme underfane ad forskellige veje). Og ja, hvis man så gerne vil betragte andre mulige veje til at nå til samme underfane, så er det jo netop det, man har den specielle overfane-fane til.) :) Og ja, man kan så netop også overveje et alternativt træ, hvor prædikaterne ikke bare har trivielle forældre, men hvor forældrene er faktiske prædikat-prædikater (og hvor brugerne altså rater disse).. Hm.. ..Ja. For dette kan nemlig så lige netop særligt bruges til at lave smarte ting, når man gerne vil indføre sorteringer, så de bliver ret automatiske, og hvor brugeren ikke så meget selv skal gå ind og vælge for hver nyt prædikat/fane, denne møder. :) ..Sorteringer kan så altså enten tilknyttes (del-)prædikater, hvor man så kan vælge imellem, om det kun skal gælde, når prædikatet er valgt sidst i rækken, eller bare uanset hvornår det er valgt i rækken, eller de kan tilknyttes prædikat-prædikater, hvor man så vælger en vis rating-tærskel, og alle prædikater, som er ratet i høj nok grad under dette prædikat-prædikat, vil så få sorteringsindstillingerne (som så kan blive ligesom i det første (eller rettere de første to) tilfælde).. Cool! Og hvad så med HTML, mon..?.. ..Ah vent, der er noget andet, jeg skal tænke over først. (laver lige et linjeskift..)
%Jeg skal tænke over, hvad en rating så betyder for et absolut prædikat, og om der mon er en måde at.. Ah, nu tror jeg næsten allerede jeg har det, her med det samme.. Når en bruger gør i gang med at rate ting for en fane, så kan man jo lige give et automatisk spørgsmål --- måske bare som et meget synligt spørgsmål, men ikke et der afbryder, hvad brugeren har gang i, så brugeren kan sagtens bare ignorere det helt --- der spørger om brugeren mon har lyst til at fjerne (eller tilføje, måske..) nogle af del-prædikaterne, før der rates, så at ratingen på den måde gøres over et mere generelt domæne (og dermed vil gælde for flere underfaner på én gang end ellers).. Hm, man kunne i øvrigt også have negative underfaner, hvis jeg lige skal indskyde en lille tanke.. ..Tja, eller også kan man gøre det sådan, at man kan toggle de allerede valgte overfaner, så de bliver ved med at være selekteret (og så fane-menuen ikke ændrer sig ellers), men hvor man så bare fjerner det pågældende del-prædikat fra den liste, man iagtager. Ja, det var faktisk en ret god idé (tak, tak).. ..Hm, så kan man måske i øvrigt også gøre det sådan, at man også kan selektere flere underfaner på en gang (så man ligesom kan shift-klikke på dem og tilføje dem til de valgte faner (også selvom de står til siden for de allerede selekterede faner, når det kommer til den nuværende placering/konfiguration/navigation i fane-menuen)).. For når faner bare repræsenterer del-prædikater, så kan man jo sætte dem sammen. Ja, fed mulighed. Noget andet er så også, at man også kunne oprette sammensatte prædikater, som så kan optræde som én fane i rækken af faner (så brugeren kan nøjes med at klikke på én fane, selvom fanen er identisk med to adskilte prædikater sat sammen, og hvor man altså normalt ville skulle vælge den ene og så den anden), og så kan back-end-systemet dog bare behandle det, som havde man navigeret til først det ene del-prædikat og så det andet (i en underfane), eller hvor mange del-prædikater nu er sat sammen. Også en nice mulighed. :) ..Og ja, for at vende tilbage, så kan brugeren altså lige gøres opmærksom på med et spørgsmål, inden denne begynder at rate elementerne i listen, at brugeren jo kan overveje, om denne hellere vil vælge et mere generelt samlet prædikat (i.e. med færre del-prædikater i) at vurdere elementer ud fra, således at de udførte ratinger kommer til at gælde for et større domæne. Fint! 
%Og hvad så med HTML-ændringer; hvad kan man / skal man gøre der?.. (Er der noget smart, jeg kan sige, eller må jeg bare nøjes med en lidt flydende idé omkring det..?) ..Hm, jeg har brug for en gåtur; så kan jeg lige tænke mere over det der...  
%Ha okay, jeg skulle ikke gå særligt langt, før jeg fandt ud af, hvad man kan gøre. Når ressourcer vises i lister (under fanerne), så vises deres thumbnails. Men når man så går ind på selve ressourcen, så navigerer man til en HTML-side, og det er så her(!), at brugere så skal kunne få lov til at justere denne HTML for forskellige typer af ressourcer. Og måden man så vælger HTML-indstillinger, er så bare helt på samme måde, som man vælger sorteringsindstillinger. Bum! Så brugerne får altså en delmængde af HTML, som de så kan benytte til at lave ressource-side-opsætninger (og med tiden kan man åbne op for større og større undermængder, bare hvor man så måske skal ansøge og få clearet koden for nogle (usikre) delmængder, inden de kan tages i brug). Disse html-opsætninger skal så kunne råde over faner/lister relateret til ressourcen, og særligt altså over en overordnet kommentar-liste samt dets tilføjede underfaner. (laver lige linjeskift..)
%Så ja, og her er det så, at jeg har tænkt videre over, hvordan man lige gør med kommentarerne (og f.eks. hvad forskellen på kommentar og ressource er). I den forbindelse er jeg kommet på / har genindset, at der jo kan være relationer, som altså kan tage flere ressourcer (m.m.) som input, og hvor man så kan danne prædikater ved indsættelse i sådanne relationer. Men nu ser jeg det faktisk lidt anderledes, hvor man i stedet bare definerer et større domæne af liste-element-typer, således at relationer på en måde kan behandles helt ligesom prædikaterne, bortset fra at deres lister så bare.. hm, kommer til at bestå af flere elementer sat sammen i tupler/mængder.. hm, men hvordan navigerer man så hen til sådanne lister; så skal man jo vælge dette liste-element-domæne som det første..?.. ..Hm, medmindre man navigerer til dem i prædikat-prædikat-træet.. ..Nå, lad mig lige tænke lidt mere over det... 
%Hm, hvad i stedet for bare at åbne op for triplet-/to-inputs-relationer, som så altså skal gives en ressource (måske m.m.) for at blive til et prædikat? ..Prædikater formet fram samme relation kan så autmoatisk være i en speciel prædikat-gruppe sammen.. Hm, og så kan man eventuelt også åbne op for specielle relationstyper, nemlig symmetriske relationer, således at et element-par uploadet eller ratet under en sådan relation også autmatisk vil blive henholdsvis uploadet eller rates hos den tilsvarende fane, hvor ressource-inputtet til fane-prædikatet/-relationen og den pågældende ressource er byttet rundt. Og når man skriver kommentarer til en ressource, så gør man det i princippet (selvom i praksis kommer brugeren nok ikke til at se dette direkte, men kommer sikkert til at interagere med et interface, der skjuler dette, i hvert fald på sigt..) ved at uploade sin kommentar til en 'comment<x> := "y er en kommentar til ressourcen, x"'-relation, hvor OP-ressourcen så indsættes på x's plads, og hvor kommentaren så indsættes på y's plads.. Og ja, så er det altså bare underforstået ved dette prædikat, at brugeren i virkeligheden også skal betragtes som et tredje input til relationen, således at den helt korrekte fortolkning er: "brugeren, u, har givet kommentaren, y, til ressourcen, x," men hvor u dog altså ikke bliver noget eksplicit input i relationen. Men da alle uploads får en poster-bruger tilknyttet sig, så vil man altid kunne finde u i princippet (medmindre det er gjort skjult af en eller anden årsag), og derved opnå den helt korrekte (underforståede) fortolkning af relationen/tripletten (som så egentligt i virkeligheden må være en.. quadrolet..). Hm, og det gør faktisk ikke noget, at man bare formulerer dette eksplicit i relationen, og så måske endda bare har en særlig variabel til at repræsentere poseter-brugeren. Og for så at vende tilbage til HTML'en, så skal denne jo helt klart kunne bruge alle faner/underfaner (og altså del-prædikater, rettere), der består af en relation med ressourcen som det første, konstante input. Og der skal så meget gerne være en indbygget function, så HTML-programmøren kan indsætte en fane-menu (helt som den optræder (/kan optræde), når man søger på ressourcer) med udspring i visse faner, og her kan man jo så passende designe det, så der bliver en menu (med tilhørende liste under, som ændrer sig, når man selekterer forskellige faner; det hører alt sammen med) under ressourcen, som som tager udsping i comment<current_resource>, hvor current_resource så er den ressource, man iagtager. ..Og dette skal så selvfølgelig også gælde for den standard ressource-side-HTML-kode, som vises for ressource-siderne, inden at brugerne selv begynder at modificere det. ..Nice! Jeg kan virkeligt godt lide det her; rart at det nu ikke afhænger af prædikat-prædikaterne overhovedet (andet end når man vil vil gøre det særligt nice og sørge for at HTML- og sorterings-indstillinger i høj grad kan sættes automatisk, når brugeren navigerer til forskellige steder), så disse bare kan være noget, men implementerer, når man får tid til det. 
%Bemærk at kommentarer nu så faktisk bliver first-class citizens i systemet.
%Hm, hvis prædikat-prædikaterne bare bruges til at hægte sortrings- og html-indstillinger på faner.. kunne man så ikke bare.. Hm, ville det i det hele taget ikke give mere mening, hvis man kaldte det prædikat-grupper i stedet, og så kunne man have et træ over sådanne grupper..?.. ..Jo, måske.. 

%(05.12.21) Jeg kom på i går nat, at hvis man bare tager prædikater og relationer --- og hele triplet-udtryk, hvorfor ikke? --- som first-class citizens, og altså som værende lig ressourcer i systemet, i hvert fald når det kommer til, hvad kan inputtes i relationer, og hvad der kan (inputtes i prædikater og) vises i listerne, så bøhøver man jo ikke noget eksternt træ, når det kommer til prædikater og relationer --- og prædikat-prædikater! Så kan det hele bare være i selv samme fane-træ som ressourcerne! Og alt det med et prædikat-prædikat-træ for at kunne lave særligt smart funktionalitet, npr det kommer til html- og sorteringsindstillinger, det behøver man nu intet ekstra træ for! Her vil det så i stedet bare være, at åbne op for at nævnte indstillinger kan sættes automatisk ud fra, om valgte faner har prædikater under vise andre faner, som så altså bare er i samme fane-træ i princippet, men som nu altså kan implementere vores prædikat-prædikat-faner..! :) 
%Og angående rating-semantikken, så har jeg tænkt over, at man måske kunne give den denne fortolkningskonvention som standard, nemlig at en rating på x \% betyder, at ud af hundrede tilfældigt uploadede ting til fanen, hvis folk bare uploader som det passer dem, så vil x \% af disse sandsynligvis være bedre eller lig i kvalitet som pågældende element, og 1 - x \% vil så være af dårligere eller lig kvalitet end det pågældnde element. Måske ikke en helt dårlig tommelfingerregel-konvention at bruge..
%Jeg kom også lige i tanke om mine "indeholder det samme"-faner, som bl.a. kan bruges, var tanken, til at eliminere gengangere.. ..Ah, dette kan man jo så bare implementere ved at lave en indbygget "har samme indhold"- (og underforstået: "vil nærmest altid ses som en gengangere af hinanden, hvis begge elementer står i samme liste"-) relation, og denne bør jo så kunne findes (med tiden, om ikke andet) som fane under hver ressource, hvor brugere så kan uploade nye forslag til. ..Ah, og.. Hm, måske skulle faktisk ikke gøre denne relation symmetrisk. ..Ah, og man kunne faktisk endda undlade, at den er refleksiv!.. Så i stedet ændrer man altså relationen en anelse, så den siger næsten det samme som før, men faktisk mere siger: "Element x's indhold er indeholdt i element y, som i øvrigt er forståelig og tydelig i sit indhold." På den måde kan man.. Hm, skal systemet så dog se det som transitivt? Det må da næsten du.. ..Ja, så faktisk næsten kunne samle dette selv til grupper, hvor.. Hm, og man kan ikke bare bruge selve ratingen under pågældende fane til at ordne..? ..Hm, nu kom jeg lige til at tænke på, at hvis folk i høj grad kan rate ved at flytte rundt, skal de så ikke bare kunne flytte kommentarer ind under andre, som er gentagede..? Jo, det ville da være skønt, og ja, så kan dette vel bare resultere i en relation som den, jeg startede med at nævne.. ..Hm, jo og skal folk så ikke bare kunne kunne sætte en tærskel for (som så på sigt kan komme til at afhænge af brugergrupper ligesom alt andet..), hvornår et element så skal flyttes under et andet automatisk, som følge af andre brugeres ratings. Og ja, så kan man så også bare gøre dette transitivt, så et element tager sine underelementer med sig, hvis dette selv flyttes ind under et andet element. Bum. ..Ja ej, det er altså virkeligt godt det her..! Jeg elsker virkeligt mit nye design her, hvor folk bare kan rate og gøre ved ved at flytte rundt..!! :D (..Og hvor er det bare blevet lækkert og simpelt i sin opbygning..!!) ..Nå ja, relationen skal så ikke være som mit førstnævnte udsagn, og heller ikke som de andre nævnte, men skal lyde: "Element x's indhold er indeholdt i element y, og behøves ikke gentaget, hvis allerede er i listen." Hm, kan man godt klare sig med, at relationen ikke også tager fane-prdikatet som input? ..Ja, det må man da kunne.. :) 
%Jeg kom også på i går, at man skal kunne selektere med rød.. ah, eller bedre: Som udgangspunkt farves de elementer, der selekteres under og over det element, man har flyttet i forskellige farver, man man bør så også kunne flippe farven, således at man også i samme omgang kan sige, at den flyttede ressource f.eks. er bedre end én, der er placeret over den (måske hvis den ligger imellem en masse, der \emph{er} bedre end den flyttede).. Hm, men kan man så ikke i stedet bare flytte ad flere omgange? Jo, det er nok nemmest i virkeligheden. Ja, så i første omgang: never mind. 
%Og så tænkte jeg lige på, at man på sigt måske kunne indføre 2d-lister også (eller multi-d måske), som så kan arrangeres efter flere end én fane.. Men ja, det kan man jo tænke over på sigt. Fint, det var de ting, jeg lige havde at nævne.:) 

%(06.12.21) Uh, hører ikke nødvendigvis så meget til her, men angående det med, at det er smart at have en "hovedåre" for diskussionstræer, jamen dette kan jo egentligt bare gøres ved, at man altid sørger i sit diskussionstræ (hvis vi implementere det i fane-systemet), at have en 'kronologisk oversigt'-fane, som så altså skal indeholde en kronologisk oversigt (og der er flere måder, hvorpå man kan sørge for, at elementerne, i hvert fald de øverste, er ordnet en vis (f.eks. kronologisk) rækkefølge, bl.a. ved at bruge særlige brugergrupper (i sorteringsindstillingerne), der sørger for dette) over pågældende diskussion/underdiskussion. Så hvis man sammenlinger med mere gængse argumentationstræer, så svarer det lidt bare til for hver diskussion af have mulighed for at opdele denne i to tråde, nemlig en tråd med den faktiske diskussion, og så en tråd med en løbende, kronologisk oversigt over de overordnede resultater (og andre ting relevante for en statusrapport), man kommer frem til undervejs. ..Hm, man kunne egentligt også have to typer af oplags i denne oversigtstråd: 1: Løbende opsummeringer af begivendheder, der sker i den faktiske diskussionstråd (som i virkeligheden kan være et helt træ i sig selv, og altså slet ikke behøver at være en lineær tråd), samt med jævne mellemrum også mere absolutte rapporter, hvor alle konklussioner hidtil og alle åbne spørgsmål opsummeres. Og disse kan så bare komme i kronologisk rækkefølge (så man f.eks. bare kan skimme de absolutte rapporter hver gang, hvis man alligevel har læst alle begivenhedsopsummerings-oplagene siden den forrige samlede rapport.) 


%(10.12.21) Hov, jeg ved ikke, om jeg fik skrevet om det her i denne omgang, men jeg skal også huske mine regex-faner. Med andre ord, skal der altså også i systemet (på et tidspunkt) kunne være syntaks-prædikater, som kan tjekkes automatisk af systemet. Der kunne så inkluderes to former for sådanne syntaks-prædikater, nemlig: "sørg for at fjerne / aldrig at medtage posts, som ikke opfylder pågældnede syntaks" og: "Sørg for løbende automatisk at hente alle nye posts, der opfylder denne syntaks, således at en fane, medmindre at ting efterfølgende er sorteret fra igen via førstnævnte syntaks-prædikat, kommer til at indeholde alle posts med pågældende syntaks." Og man kan jo så selvfølgelig altid også blande disse prædikater. Jeg har nævnt et andet sted (hvor jeg sikkert kaldte det noget a la "regex-faner" i stedet for "syntaks-prædiakter"), hvad mulighederne kan være ved dette, så det vil jeg ikke lige gøre igen her. Good. 

%(13.12.21) Jeg tænkte på i går nat, om ikke man skal kunne have ressourcer/elementer som faner, når man allerede har valgt en relation tidligere, således at ressourcen/elementet bare kommer ind i relationen, og ja, dette er nok en god idé. Hm, og hvad hvis der er flere steder at inputte ressourcen?.. Hm, lad mig lige nævne, at der jo også gerne bare skal være en oversigt, hvor man kan se par (og måske også mere end det) af ressourcer, der opfylder relationen, og her skal man jo også bare kunne klikke på den ene og så ændre den anden kolonne, så kun alle ressourcer, som indgår i et par med den valgte, vises. Hm, og er dette ikke nok langt hen ad vejen, og så kan man altid se på, om det giver mening at kunne have underfaner i form af ressource-input?.. Ja, lad os sige det sådan. Men noget andet er så, at jeg i det hele taget lige skal tænke lidt over, hvordan fanerne helt præcist bliver knyttet til overfaner, hvad var det, man skulle gøre her? Lad mig lige vende tilbage til dette og tænke over det senere.
%Okay, nu er det senere, og jeg kom lige på, at dette vel bare skal fungere ligesom ressourcerne, måske, hvor man for hvert prædikat (også sammensatte) skal kunne få en del-prædikat-oversigt, hvor man kan vurdere enkelte prædikater. Og så skal der her også bare gælde, at man så skal kunne vælge, eventuelt at lade sine vurderinger blive mere generelle ved at fjerne del-prædikater fra det samlede prædikat, som giver den oversigt, man kigger på.. ... Hov, der er faktisk noget endnu mere vigtigt, jeg bliver nødt til at tænke over.. ..Ah, måske har jeg allerede et muligt svar (inden jeg har formuleret problemet her endnu): Måske skal man have en slags.. rating-prædikater, som er adskilt fra resten.. Hm, eller går det?.. 












%Hm, skal systemet egentligt ikke bare være, at der er én rating for kommentarerne, men hvor man så vælger konteksten for den rating i form af den sti, man så rater. For kommentarer kan man så dog ikke bryde ud af ressourcens kontekst, men brugere bør dog kunne referere til andre kommentarer på kryds og tværs, og meget gerne hvor den refererede kommentar udfoldes automatisk (men hvor man jo så altid kan se, at den kommer et andet sted fra).. ..I øvrigt tror jeg altså næsten nu jeg dropper det med, at man enten kan give en akse-rating \emph{eller} en tommelfinger op/ned, for hvis man alligevel kan give en rating bare med et enkelt klik (og måske lige efter, man har bevæget sig over et felt med musen først), hvorfor så ikke bare holde sig til dette?. .. (Hm, og vil alle brugere synes om dette, eller \emph{skal} man mon også gøre noget for, at brugere bare kan vælge et mere simpelt standard-svar..?) ..Hm, og hvad gør jeg forresten ift. at vise antallet af stemmer?.. ..Hm, ved at vise dette, eller måske ved at vise antallet af positive stemmer og antallet af.. Hm, eller faktisk måske bare vise.. fem tal.. Hm, eller måske.. Ah, hvis man skulle vise fem tal, så kunne man vise antallet af næsten-neutrale stemmer, antallet af positive og negative stemmer samt antallet af meget positive og meget negative stemmer --- måske vist i parentes --- hvor sidstnævnte altså så også indgår (og altid er mindre end) de to førstnævnte tal.. ..Ah nej, man kan i første omgang bare vise stemme-antallet, for dette kan jo bare vises for sig ved siden af rating-feltet. Og hvilke tal man så viser i første omgang, når man dropper rating-menuen ned første gang (og hvad der vises, hvis man udvider denne menu endnu mere), det kan så være en længere snak.. 
%Hm, og hvad gør man så for poster-prioritet? Er det nok bare at give et offset? BLiver det ikke et problem, at andre brugere kan komme og rate ens kommentar ind i et andet tab; hvad hvis semantikken så kommer til at ændre sig, og dette så ikke passer med ens intentioner..??.. Hm... ..Hm, skal man så virkeligt bare sige, at kommentarerne bare bliver inden for deres eget tab, medmindre self. de refereres til andetsteds (hvor de så kan foldes ud der); bliver det så ikke et problem, at posterne selv har ansvaret for at kategorisere kommentaren..? Selvfølgelig kunne et forslag være så at gå tilbage til idéen om, at postere kan få notifikationer, hvis mange nok har stemt kommentaren ind et andet sted.. Hm.. ..Hm, det kunne jo være som jeg havde tænkt mig, med at kommentarer nærmest bare får et flag, hvis ikke posteren selv har valgt/godkendt den pågældende tab.. Ja.. ..Så man kunne altså bruge en speciel slags kommentar-monader til at overføre andres posteres kommentarer til en ny fane, hvor denne monade så kan skjules, hvis (og så længe) at OP godkender, at kommentaren passer til denne fane.. ..Ja, det lyder da lidt som løsningen, hva.. ..Hm, og så skal der i øvrigt også vises forskel på, om OP bare ikke har reageret endnu (og/eller altså ikke taget stilling til det), eller at denne ligefrem har afvist, at kommentaren passer til fanen.. 



\subsection{Extending this organization to all other resources}



\subsection{...}









%Generel huskeliste:
% - Husk at nævne donationsforeninger, måske i en af de allersidste sektioner..
% - Brugsbaserede ratings (conditional on a certain type of use)..


















%Hm, jeg tror, jeg vil prøve i stedet at gå over til at bruge mere skabelonsagtig og simpel struktur, og så ahve det fint med gentagne undertitler så som "motivation" og "background knowlegde" og sådan.. Og ja, jeg tror ikke, det gør noget, hvis nogle af disse undersektioner så bliver lige lovligt korte; hvad gør det? Det er kun rart for læseren, at der er en tydlig struktur, også selvom den måske kan virke lidt søgt.. 

\section[(old)Web ideas]{(old)Ideas for websites}%(Short summaries of my?) Ideas for websites}
%%Hm, skal jeg transformere \section-sektionerne om til \chapters, og så transformere \subsections og \subsubsections ét niveau op alle sammen..? ..Ja, lad mig bare regne med det.  
%
%In this chapter I will summarize a variety of ideas regarding websites. The ideas can either be implemented on existing websites or be used to build a new website (or more) with %these specific
%the new 
%functionalities. 
%
%Sections \ref{rating_tags} and \ref{organized_comments} are both about some ideas for organizing user feedback more according to its semantics on %websites such as web 2.0 sites\footnote{The term 'web 2.0,' for those who do not know, refers to websites whose content to a large degree is created by the users and not just the web developers (e.g.\ YouTube, Facebook, Twitter etc.).}, product websites or streaming platforms where users can browse a large number of \emph{resources}, i.e.\ the objects that make up the site's content, and leave feedback to these in the form of ratings or comments. 
%various websites... 
%
%In the first section I will summarize an idea for extending folksonomy tags (i.e.\ user-made tags) with ratings, 
%%to make it easier for users to subsequently judge the same tags and thus make it easier for users to categorize and jug
%not just for the user adding the tag, but where every user will subsequently be able to rate the given tag. 
%%With such a system, any widely viewed resource will, I believe, quickly  
%%With such a system, I imagine that that any widely viewed resource quickly will 
%As different tags will represent different predicates about the resource, the idea is thus that this will add a wide variety parameters with which to categorize and judge resources on the site.
%%Åh shit, det er forkert at kalde det web platforms..(!).. Det må jeg så holde mig fra.. Hm, lad mig også lige tilføje en note i 'Opfølgende'-sektionen..
%...
%%Jeg skal vist finde på noget andet her..







%Kopieret fra 'Opfølgende noter omkring rating-folksonomy-system'-sektion ovenfor:
%(31.10.21) Okay, så en opsummering (/brainstorm) her ude i kommentarerne over, hvad min opsummeringstekst skal indeholde:
%Jeg skal altså forklare om tags med ratings. Hvorfor er dette smart.. Forklare om tag-dokumentationer.. Nævne helt kort at der er måder, hvorpå man kan vise relevante tags givet tidligere tags, og at jeg har en løsning (men behøver ikke at forklare den).. Jeg skal nævne det, at få en oversigt, så man kan se ting relativt til hinanden.. Nævn at ting kan kategoriseres med tags, og at jeg tror sådan kategorisering vil... nå nej, det giver jo lidt sig selv..
%Så kan jeg gå videre til at forklare, hvorfor ordnede kommentarer er smart, og hvorfor kommentarer også skal have mere en én rating. Måske kan jeg også nævne her, at der er måder, hvorpå man kan få relevanta kommentar-kategorier til at blive vist.. nej, måske ikke, for relevante kommentar-kategorier vil jo blive vist, når de får opmærksomhed.. ..Hm, skal jeg egentligt også nævne navigations-relationer/-links..? Hm, måske.. Jeg kunne også nævne $\subset$-relationen kort (altså som en måde at lave kategorier på), men måske ikke.. ..Hm, men ellers kan det være, at det bliver mere relevant at nævne (helt kort), nedenfor ved b.d. ML (og dets "domæner").. ...Lad mig nævne det her, og så kan det være, at jeg laver et afsnit nedenunder omkring at starte sem-web via en sådan platform med en masse brugere (og altså starte med en brugervenlig udgave; "det handler om at opnå data fra folk, og så kan man altid oversætte til et andet formelt sprog bagefter" (hvis det skal passe til andre systemr)).
%Jeg bør også nævne ønske-vurderinger her.. *(Hm, måske kommer det op under rating-folksonomies..)
%
%Jeg kan forklare om brugerdrevet ML; hvordan det kan virke, og bl.a.\ også hvorfor det vil øge vurderingsaktiviteten gevaldigt på platformen (fordi brugere hermed selv faktisk får noget ud af at vurdere ting).. (Husk at folk også selv skal kunne kontruere og offentliggøre deres egne vektorer..) Og ja, det er så her, at jeg også lige kan nævne det forskelle muligheder, som jeg ser det, for "brugergrupper," bl.a.\ især min idé om, at brugergrupper med "bløde algoritmer" kunne være smarte.
%
%Jeg kan så nævne konceptet om, at åbne brugerfladen op for brugerlavede udvidelser, og om så at gøre det til en del af platformen, at der kan være, hvad der svarer til YT-kanaler og/eller som SoMe-profiler, bare hvor skaberen ikke skaber indhold men skaber brugerfladeudvidelser (og hvor folk kan følge, like, og donere). 
%Og så kan jeg nævne brugerdrevede feed-sorterings-algoritmer som en særlig ting, hvor brugerne kan få kæmpe stor gavn, og hvor der (7, 9, 13) automatisk må komme en kæmpe aktivitet --- ikke kun hos særligt engagerede og teknisk formående brugere. Hvis platformen allerede har en god brugerbase, vil denne idé, mener jeg, kunne få siden til at eksplodere; folk vil migrere til platformen fra andre ("feed-platforme") i store mængder. 
%..Uh, og måske jeg lige kunne tilføje nogle ord også om, at platformen jo kan give et sprog til at snakke med dets servere, så brugere hermed har frit spil til at programmere applikationer til platformen. Det er så ikke sikkert, at dette vil føre alverdens ting med sig, men måske, så hvorfor ikke tage chancen og prøve?. 
%
%*(skal nok rykkes ned) Hm, måske skulle jeg indsætte en lille note her om at nå det semantiske web via min platform idé. Måske passer det bedre før bruger-programmerings-noterne, men.. ..Hm ja, på en eller anden måde synes jeg, det passer bedre her. ..Nej, måske kan jeg faktisk gøre dette til sidst, efter debat, og så kan jeg følge op med ontologi-eksempler.
%
%..Hm, kunne jeg også lige sige nogle få ord, om at have mere alsidige wiki'er (basalt set) med prædikater på artikeltitlerne, og om så også at strukturere dem bare formelt og lagdelt?..
%Jeg skal nævne min idé omkring kurser.. Hm, nå ja, denne idé er jo lidt bare en videreudvikling af "wiki-side-idéen," så lad mig lige på et tidspunkt kigge på, hvad jeg helt præcist skal nævne (men dette kan godt vente til efter, jeg har skrevet opsummering over ovenstående punkter)
%..Uh, og jeg kunne måske lige skyde nogle hurtige ord om ratings til kollaborativ programmering ind her..(?) ..Ja, for jeg skal vil alligevel forklare om det, hvis jeg vil forklare kort om kollaborativ redigering..(?)
%
%Jeg skal nævne min debat-platform-idé. ...Nævn at det kan føre til rapporter/kurser og også til "p-modeller" på sigt.
%
%Hvis jeg vil nævne nogle ontologier, kan jeg jo snakke om vejen til det semantiske web her..(?)
%
%Nævn måske nogle ontologier..
%
%Jeg vil ikke forklare om mine civilforeninger her i denne omgang (selvom det er en vildt god idé, tror jeg).





\subsection[Rating tags]{Folksonomies where all tags carry user ratings}


\subsubsection{Introduction}
Conventional folksonomies are when users can write their own tags and attach them to the objects that make up the website's content, which I will refer to as the site's \emph{resources} in the following text. Other users will then be able to see the same tags when they view the same resource and will be able to search for for resources based on these tags. If a lot of users attach tags to a resource, the website might make make only the most popular tags visible to the user. Sites that might use folsonomies are include any site where the users can browse a large number of resources, such as web 2.0 sites (i.e.\ YouTube, Twitter, Facebook etc.), online shops or streaming platforms.

The expressive power of regular folksonomies are, however, limited since users are only able to express for each tag whether that tag is applicable or not. %Furthermore, when only tags above a certain popularity threshold is shown, it will always take ...
If for instance you want to tell other users that a certain movie (on for instance a streaming platform) is scary, regular folksonomies will only really allow you to add a tag `scary' to the movie and not to tell exactly \emph{how} scary, you think the movie is. In theory you could then add additional tags such as `very scary' but this would result in a lot of redundancies in the tag list. 

A better solution would be to add continuous ratings to each tag so that each user who want to communicate how scary the movie is can just focus on the same tag, `scary,' and rate the predicate on a scale. Other users will then be able to see the mean and perhaps distribution of how the predicate has been rated. It will then be possible to compare different resources (e.g.\ movies with our example here) on the site, not just on their overall popularity and on their list of most popular tags, but on any relevant predicate for the resources. %Jeg kunne måske vende denne sætning om og bare skrive den nye mulighed først..




%Er der mere, jeg skal sige i introen?.. 


%Hm, jeg kunne måske et sted nævne, at det kan blive nemmere for brugere hurtigt at give deres vurderinger (så det ikke længere bare er for de engagerede brugere). Og jeg kunne måske overveje at nævne noget meget hurtigt omkring, at systemet håndterer kontroverser nemmere.. ..Hm, måske kunne man så bare bruge et eksempel med `scary' og `not scary' (selvom det så bare er et legetøjseksempel).. 


%Adding ratings to tags also makes it very easy for users to contribute with their opinions, since it only requires a few clicks with the mouse to add an answer to a rating. There is no need for users.. %Hm, jeg tror da faktisk, jeg vil lave en 'advantages and disadvantages'-undersektion..
%... Hm, nej den skal jeg nu nok lægge til sidst.. Hm, kunne jeg lige sørge for at nævne, at man kan rate tags direkte.. ja..

Having ratings for each tag can also make it easier for users to leave their opinions if they can then do it with just a few clicks. I thus imagine a design where users can click on any tag they see for a resource and rate it immediately there and then.


%With such a system, ... %Hm.. ..Noget med at understrege (igen?..), at brugere nu kan finde på alle mulige prædikater at bedømme ressourcer på, og at lave ratings nu kan betyde ligeså meget som høje ratings i princippet.. ..Og at det nu altså er mere en bare formel kategoriesering, men at det løftes til en måde for brugere at udtrykke deres \emph{holdninger} i højere grad.. ..Hm, jeg overvejer så også lidt at indsætte en 'relevans vs. rating'-undersektion her efter denne, og måske kunne man så komme med nogle af disse pointer her..(?) 

Even though it might seem like a small change from the current systems, I believe that tag ratings can make a big difference in what role the folksonomy system plays on a site. Instead of primarily being a way for particularly engaged users to help categorize resources on the site for others, this system will also afford all users with a way to give be able to ...






%\subsubsection{Advantages and disadvantages}
%Even though it might seem like a small change from the current systems, I believe that tag ratings can make a big difference in how users... %Ah, dette bør måske komme til sidst i stedet..
%
%
%%Adding ratings to tags also makes it very easy for users to contribute with their opinions, since it only requires a few clicks with the mouse to add an answer to a rating. There is no need for users..









%Conventional folksonomies are when users can write their own tags and attach them to the objects that make up the website's content, which I will refer to as the site's \emph{resources} in the following text. Other users will then be able to see the same tags when they view the same resource and will be able to search for for resources based on these tags. If a lot of users attach tags to a resource, the website might make make only the most popular tags visible to the user.
%
%This idea expands on such a system by adding ratings to tags. ... %Skal jeg skrive mere..? Skal jeg i øvrigt nævne nogle eksempler på sider, vi kunne snakke om?.. ..Hm, ja og ja, bliver svarene nok..
%This will mean that users can easily add their opinions... %..Easy to contribute to the folksonomy.. Getting an UX of being able to judge a ressource on several parameters.. A big part of the idea is to have continuous ratings.. 
%%*Ja, mon ikke det er en god idé med en kort forklaring her, for så kan jeg nemlig gå viedere til motivation nedenfor og så komme tilbage til løsningen, og hvad man opnår med den..
%
%Before we continue on, let me just underline what kind of websites the idea is relevant for. It can be used for any website that can implement a regular folksonomy system as well, which is any website where users can browse a large number of resources. This could be any web 2.0 site, i.e.\ any site such as YouTube, Facebook, Twitter etc.\ where the resources (to a large extend) is created by its users. It could also be online shops or streaming platforms. Note also that a lot of online services can be accessed via mobile or desktop applications, so when I refer to `websites,' just keep in mind that the same will then also apply for such applications. 




%..Tags will no longer just be for very engaged users; it will be for everyone..


%Summerize what systems we are talking about here.. *(..Or in the outer intro..)
	
	
	%\subsubsection{The limited expression/ive..? power of regular folksonomy systems} %...
	%The limitation af the regular type of folksonomy system comes from the fact that each tag can only convey the answer to a binary question, namely whether the tag is applicable or not. If we for example look at a movie on a streaming platform and want to judge whether it is scary or not, a conventional folksonomy system will only really allow us to 
	%\ldots\footnote{(03.11.21) Let me remind myself to NOT yet make a big deal out of writing correct and well-structured sentences or writing paragraphs too long or too short and so on --- and also to not care too much about missing some small points along the way. This is only the first draft! Better to keep some of my flow from writing brainstorms than to get stuck for too long at a time here and there. Just write some words and don't look back. (Right now I only need to get a good overall idea about what the paragraphs should include.)}\ 
	%judge whether the movie \emph{is} scary or not (by choosing the tag `scary') and will not allow us to judge \emph{how} scary the movie is. 
	%Another example could be judging the difficulty of a game (on Steam for instance). With regular folksonomy systems, we are not really able to judge and convey to other users how difficult the game is, but only whether it is difficult or not. 
	%%Dette skal nok ikke med, men nu skriver jeg det lige alligevel:
	%In principle you could then add additional tags for each degree and for instance add a tag saying `very difficult,' but this option would make the tag menu very clustered and also with a lot of unnecessary repetitions (such as showing `difficult' as well as `very difficult'). It is then much better to simply add (continuous) ratings to the tags.
	%
	%%Måske:
	%In fact when we think about it, most tags will always have various degrees in how much they can be applied to a resource. Even for tags that signify whether a resource contains something or not, typically it will also be relevant then to know (as a user searching for something and/or wanting to avoid something) to what degree that resource contains the thing in question.
	%
	%%	By adding ...



%\subsubsection{Adding ratings to the tags}
%...
%%Hm, hvad er der overhovedet af ting, jeg skal uddybe, inden vi når til dokumentationerne?.. 
%%Hm, jeg udkommenterer for nu..



%By adding continuous ratings to tags, the users will be able to be much more precise when they judge resources as they will not be limited to adding overall predicates but can judge the degree of those predicates as well.
%
%At the same time, it will often be easier to leave this feedback, as the process now only involves clicking on some of a resource's tags and then clicking on a point on a rating bar/slider for the user to add their answers to the ratings as well. 
%
%If a certain tag is not found already attached to a resource, the user will still have to do more manual work to add the tag (just like for regular folksonomy systems) before they can rate it, but here only the first users will have to do this, and a new resource will most likely get all the most relevant tags fairly quickly.
%
%I personally like the idea of using a rating slider (with a scale running from 0 to 100 or from $-100$ to 100) rather than using more discrete ratings such as choosing a number of stars. The latter can work however, as long as there are enough values to choose from then at least. 
%(In my own opinion having just five values to choose from is to little; I think that a smaller gab between the options makes it easier to make a choice if ones opinion is midway between two options.)
%%Jeg beholder ikke dette, men følte lige for at skrive det:
%%Being able to choose from only five stars for a rating is a terrible design decision in my opinion. The fact that two users might give the same rating if one thinks that something is slightly better (or more true) than medium while the other just do not think it deserves a perfect rating is horrible first of all. But at least as important is the fact that it is harder to make a decision for 5-star ratings, since one so often has an opinion midway between two options (who are then far away from each other semantically), at least in my experience... %Tja.. Nå, men jeg tager alligevel ikke dette med.




%..Mention rating relevancy here..? ..Yes..? 
%Maybe mention the badness of 5 star ratings when it comes to user ratings.
%..Maybe recap the advantages.. ..Yes.


%%Hm, jeg tror egentligt hellere, jeg vil lade dette blive en note nede i et appendix i stedet. Og så vil jeg i stedet hellere bare komme med nogle eksempler (der så viser intentionen med skalaerne). Måske laver jeg så bare en sektion, der hedder 'Comparing ressources (based on tag ratings)' eller noget i den stil.. 
%\subsubsection[Semantics of the scale]{Defining the semantics of the rating scale}
%Note that with a system implementing tag ratings, there are now several ways to interpret some tags. If we for instance take the same example as before with a tag representing the predicate `scary,' we can interpret the rating scale either as signifying how sure the users are that the agree with the predicate or we can interpret to signify the degree to which the predicate applies, i.e.\ how scary the users think the movie is. In the first case, we interpret `scary' to represent a constant predicate and the users then rate how confident they are of the applicability of that predicate whereas we in the second case treat the predicate as variable one, whose meaning depends on the values on the rating scale. In practice the difference of these interpretations is that users with the first interpretation might evaluate a tag such as `scary' as near 100 \% (if 100 \% is the maximum) if they are sure that the movie is scary, even though they do not consider it one of the scariest movies. %Det kan jeg godt omformulere..
%But with the second interpretation, the highest scores are only given to the most scary movies. 
%
%Using the second kind of interpretation is what will benefit the user community the most, as more information can be extracted from a rating in this case. I think people will tend to use this interpretation automatically, even with no explicit agreement on the semantics, but still it might be a good idea to make such a convention explicit. The site that adds ratings to its folksonomy system should thus write this convention in its guidelines. %Omformuler måske..
%It would then also be beneficial to add some specific guidelines for how different values of a rating scale should correspond with a degree of the tag's predicate. Such guidelines could try to give examples for as many different kinds of tag predicates as possible, aiming to cover as representative a set of predicates as possible, and try to describe for all cases how different values on the rating scale should be interpreted. 
%
%%If finding very clear-cut guidelines and agreeing upon them turns out to be a hard task, it might also be a good enough solution to simply give some overall guidelines at first and then to just monitor the tendencies of the users over time in order to adjust the guidelines to fit the conventions that have been formed automatically. %Omformuler måske..
%%
%%For some tags the meaning can not be made clear from just the title. For a tag such as `scary,' people can deduce a lot from just the title but there are also more abstract predicates where this is not the case. We could for instance have predicates such as `slow burn' or `slick' (which are both tags that can be found on Netflix), where many users would benefit from at least an overall explanation of what those tags mean. I will therefore suggest that each tag has a link to a documentation, which the creators of the tag author. 
%%
%%%Hm.. Det er da lige før, at det bare er smartere i virkeligheden at glemme dokumentationer og så i stedet bare i høj grad lade semantikken indfinde sig automatisk.. 
%%%... Ah, men:
%%Whether...
%
%But even if you make perfect guidelines with precise definitions, you cannot be sure that the users will all read and follow them. ...


%\subsubsection{Examples where this system could be useful}
\subsubsection{Use cases}
...

%Example with game difficulty (instead of scariness).. (maybe)

%Example with an online shop where users rate the durability (and maybe other stats). *(Nævn prædikat-intervaller for product page-sliders her.)

%Example of a user wanting to allow some but not too NSFW content.. (maybe)

%Example with a AoE on the 'world builder' tag (if I can't find a more normal example)..

%Eksempel a la "Wes Anderson vibe," der forklarer, hvordan der nok bliver mere plads til små og nyopfundne tags..
%Ukendte musik-genrer kunne være et andet eksempel, hvor ukonvensionelle tags kunne få mere at sige..

%Måske også bare simpelt eksempel, hvor en bruger har lyst til at give vurderings-feedback til noget, og nu kan gøre det nemt (og ikke behøver at skrive en kommentar), og hvor andre med det samme kan få glæde af bidraget (ikke kun dem, der læser kommentaren)..



%\subsubsection{Comparing resources based on tag ratings}
\subsubsection{The semantics of tags} %..Defining/finding..

%Hvis dette bliver et af de eneste detalje-afsnit (foruden wish ratings), så kan jeg jo bare skrive, at dette lige er det mest presserende ikke-besvarede spørgsmål, og at resten så bare nævnes i appendix..

How to agree on the semantics of a tag is not a trivial matter. A result of for instance 87 (if we say that the rating scale runs from 0 or $-100$ to 100) for the rating of the tag `scary' might mean something different to two different users. And for more abstract tags such as NSFW the difference in interpretation...

I will suggest that every tag has a link to a documentation of its semantics, which the creators of the tag author. This documentation might also be made to contain guidelines for what different values of the rating scale should represent. But in the end, it would not be a very efficient system, if every user has to read a whole documentation of each new tag they want to rate. So while tag documentation are still a good idea, it is not sufficient to expect that every user reads them and follows them in order for the system to work.

Instead I will suggest that there should be a link for all tags to a comparison page %... %Hm, eller endnu bedre...
where the user can view the whole rating axis and see some other examples of resources (preferable ones known to the user, e.g.\ some that the user has rated before with respect to other tags) and how these has scored so far on the axis. In this way the semantics of a tag does not need to be formally defined for it to be effective, as users can just compare to previously scored resources on the axis. The semantics will thus just depend on the development of how users score more and more resources. 
%
The first users to rate resources with respect to a new tag will then of course have to just use an overall intuition for the scale. For this it might be a good idea for the site to show some constant qualifiers (..?) such as `very,' `not very,'  `extremely' and `not at all' etc. Whether to then keep these qualifiers or remove them when enough scores have been uploaded is a question for the site to find out. In most cases it would probably be best to keep them but there might also be cases where it might be a good idea to loose these qualifiers after a certain point. %I don't know if I want to keep this.. %..Og jeg skal lige finde ud af i øvrigt, om "qualifiers" er det rigtige term at bruge..
%For users who were early, it.. might be a good idea to go back and adjust the rating.. Hm, dette gider jeg næsten ikke at nævne alligevel..


%Måske kunne jeg skrive en note i appendix omkring, at man måske kan omtransformere rating-skalaer --- især hvis semantikken er rent relativ..(?) ..Jo.

...\footnote{Remember appendices..}

Hov, husk at nævne, at folk altså primært kommer til at bruge eksisterende tags, når de rater.. (Hvis det ikke giver sig selv..)


%Having a comparison page.. also... 

%Hm ja, skal jeg så nævne fordelen ved, at ressourcerne sammenlignes her..? ..Ja.. ..Hm, eller..?.. 
%Hm okay, hvor meget kød er der på pointen?.. ..Hm, og pointen er vel, at hvis brugerne kan se en akse, hvor en ressource sammenlignes med andre ressourcer, der er kendt for brugeren, så vil denne have større lyst til at rate ressourcen (som måske er noget, brugeren lige har set/prøvet).. Hm tja, det behøver godt nok ikke at være noget, brugeren lige har prøvet.. ... Hm, uanset hvad, så må dette vel bare høre til en kommentar i appendix, ikke? (Og så nævner jeg den altså heller ikke i listen over advantages..) Jo, det bliver nok sådan. (Og hvis der så bliver mere kød på den, så kan jeg jo altid prøve at fiske den op her igen..)

%As a final...







%The site might also have overall guidelines, but still... (nu har jeg nævt "qualifiers" i stedet, og det er måske bedre i virkeligheden.. Så tjek for nu)

%The comparison page is also useful since...

%For a better user experience, a drop-down menu with a compact version of the comparison page might also be a good idea... ...Dette kunne også ryge i appendix..


%Viewing other (known) resources on an axis while rating.

%Viewing ressources on axes in genral to compare and to search --- which will btw increase the will to rate them..





%\subsubsection{Sorting and comparing resources based on tags}
%
%Having a comparison page like I mentioned is not only useful for deciding what values to give to tag ratings. It can also be useful for users to get an overview of the different kinds of resources on the site. 
%%
%The tag ratings can indeed be seen as forming the axes of a multidimensional space within which all the resources lie at different positions. The positions might not be precisely defined on all axes as some resources will have few or no rating answers for certain tags, if for instance the tag is not very relevant for that resource. %Hurtigt formuleret (men det er alt, hvad jeg skriver nu..) 
%In the appendices \ref{...} I will write how such undefined positions can be handled. 
%
%If we can find a good way for users to explore this multidimensional space of resources, it will allow them to be able to better find relevant resources to a given one% and also to explore... %Hm ej, det er faktisk svært at argumentere for..
%. 
%Showing a list of possibly relevant resources to a given one is a typical feature found on a lot of websites. %Gentagelse
%
%%But here.. 
%%(08.11.21) Hm.. ..Hm, er der nu så meget at komme efter her..? Kan jeg ikke bare nævne.. at man ligesom nu kan behandle andre prædikater lidt som prisen i en online shop..?.. Og så bare tale om det at sammenligne ressourcer langs én akse, og at dette altså kan forøge lysten til at rate..? ..Jo.. For udover at mindre tags (mindre kendte/populære) nok får mere at sige her, så kan man sagtens få et godt overblik bare via tags'ne i sig selv.. Hm ja, og ellers så er der andre måder at give en oversigt over en sides indhold. Så lad mig droppe idéen om et koordinatsystem.. Og så vil jeg rigtigt nok bare nævne, at nu kan man også justere intervaller for andre prædikater end bare for prisen. Og ja, så skal jeg bare sørge for at nævne, at det er smart hvis man kan se en akse.. hm, hvor man zoomer ind i stedet for at få en liste fra ende til anden.. Hm, måske ligger der alligevel noget i idéen..? Hm, ja måske er det en brugbar idé, det med at have en akse, hvor man kan zoome ind, i stedet for at se en liste fra ende til anden, hvor man kan scrolle og scrolle uden at komme videre langs aksen. Cool, men det hører nu dog stadig bare til en appendix-idé. (..Eller til en idé, jeg bare forklarer i dette notesæt..) Hm, lad mig bare lige forklare det her for en god ordens skyld: Hvis man er interesseret i at se.. Hm, men i princippet så kan man jo også bare gøre dette ved at justere en tag-slider ude i siden (jeg tænker ligesom på mange product pages, hvor man kan vælge tags og indstille bl.a. prisinterval ude i venstre side), som man så kan sætte til forskellige værdier (og især hvis det så altså bare er underforstået, at produkterne/ressourcerne vises omkring det valgte punkt på slideren og ikke hverken med præcis den værdi, eller hvor man kun viser f.eks. højere værdier).. Okay, så never mind. Denne løsning er nok i virkeligheden bedre (og nemlig også mere simpel) end min koordinatsystem-idé. 





%\subsubsection{Sorting resources based on tags}
%
%%Hm, nævne en product page.. Hm, på den anden side, kan jeg ikke bare nævne dette i durability-eksemplet? Og hvad er der ellers omkring sorting, som giver mening at nævne her, og som ikke bare kan vente til efter b.d. ML..? ..Nej sgu. Jamen så er det jo bare den ovenstående sektion, der nok er meget god at have med, omkring semantikspørgsmålet, og så er det ellers bare ønske-rating, foruden use cases (hvor jeg bl.a. kan nævne tag-/prædikat-intervaller for "product pages") og advantages/disadvantages-sektion. Nice nok. :)





%Maybe mention 'suggesting relevant tags' here..

%Hm, skal jeg nævne noget om muligheden for at definere sammensatte tags, som automatisk får score ud fra, hvad visse andre tags scorer er..? 

%\subsubsection{An overview of resources relative to each other with respect to their tag ratings}
%...



%Ah, husk også eksempler, der kan vises direkte når brugeren har klikket for at rate..! (tjek)

%Husk også at nævne linked semantics imellem tags..

%\lipsum[1]


%%Dette bliver også i appendix, og så må jeg bare referere til dette, hvis jeg får brug for at nævne det senere:
%\subsubsection{Finding relevancy of tags}
%...


\subsubsection[Wish ratings]{Using ratings to express wishes for future resources}
Maybe (something like): I have a few more ideas about this subject and I will mention the most important ones in the appendices, but the following idea is one that I find worth mentioning here.

I would like to also mention an idea for how users could be able to use tag ratings to express wishes for other resources in the future. The idea is to let the users upload certain objects, which hold sets of fictive ratings. I will call these object `wish ratings' in the following text. The fictive ratings then embodies a wish for what a future resource might fulfill.

There is a question of whether these wish ratings should be attached to a certain existing resource or not. If so, then we can have a system where users can just add any fictive rating for a resource and a tag one at a time. %Omformuler. 
Whenever a user views a resource and think of a wish in regards to a certain parameter (embodied by a tag), %Omformuler (jeg skriver jo bare brainstormy (og disse kommentarer om at huske at omformulere er egentligt bare en måde ligesom at få mig til mig selv mere lov til at skrive tingene hurtigt..)).
the user can just add a wish rating (which again is a fictive rating set apart from the actual rating of that tag (which the user might have rated or not)) for that tag. If he or she comes back later on with another wish, the user can just add this as well. 

With this way of doing it, the system is very easy to make and to use. And it would go far as a beginning.. %(Brainstormy.)
But something is missing from the semantics in this case. First of all it might be hard to know how.. %Hm, lad mig lige skrive det på dansk engang (hvis jeg alligevel skriver så klumpet --- og når der altså alligevel er nogle ting omkring det, jeg skal tænke over)..
%Det kan være svært at vide, hvor grænsen går ift., for hvilke ressourcer brugerens ønsker gælder. Hm, og samtidigt er der også forholdet, om brugeren vil splitte ønskerne op, hvis der ønskes på flere tags, eller om ønskerne helst \emph{skal} være i konjunktion med hinanden.. Men ja, for dette sidste spørgsmål, der kan man vel gøre det lige godt, hvad en ønskerne tager afsæt i en specifik ressource eller ej.. ..Ja, så at brugere kan fremhæve, når ønsker er korrelerede, det virker i begge tilfælde.. ..Ja okay, kan jeg så ikke bare forklare idéen med, når ønske-ratings'ne tager afsæt i en ressource, og så til sidst forklare, at hvis man gerne vil ændre systemet og bruge kategori-tags i stedet, så vil der sikkert blive god mulighed for dette (fordi kategori-tags'ne nok skal blive rigtigt veludførte)?.. Jo.. 
%Okay, men dette giver så nok lidt sig selv. Så jeg tror bare jeg hopper videre til at skrive en kommentar-disposition over næste undersektion (som jeg så kan rette, hvis jeg kommer i tanke om noget), og så vil jeg gå videre til næste sektion.



%Hov, måske skulle jeg alligevel lave en lille undersektion med relevancy..? Hm.. ..Jo, det kan jeg faktisk nok godt..


%\subsubsection{Deciding which tags are the most relevant ones for a resource}
%With regular folksonomy systems, the relevancy of a tag is what... %Hm, eller hvad; hvor vigtigt er det overhovedet, det at de mest "relevante" tags vises for en ressource..? ..Hm, det er vel egentligt netop ikke særligt vigtigt alligevel..(?) ..Hm nej, det er da lige før, at ingen gang jeg ville indføre en relevancy-rating oveni..!..? ..Hm, og hvad så med comments, kan man sige det samme der..? ..Ja, man kan jo vel også bare bruge rating-aktivitet der til at måle relevansen.. ..Okay, lad mig lige tænke lidt mere over det en gang, og så se om det virkeligt giver mere mening bare at bruge aktiviteten som mål (selvom det er lidt ad bagveje, så..).. ...Ah, jeg ville nok hellere implementere det ved at bruge, at kommentar-faner kan implementeres som relation, og at man så endda kan åbne op for relationer med flere variable (og måske med typer endda..).. Og ja, det ville så lidt være den længere måde at gøre det på, men man kunne så også lave en smutvej, og bare indføre nogle få bestemte kommentar-/relations-typer, som eksempelvis "kommentar (eller tag) x er relevant for ressourcen this.parentResource" eller noget iden stil.. Hm.. ..Ja, det kan jeg lige tænke lidt mere over, men jo, det ville nok bare være sådan, jeg ville gøre det (\emph{hvis} (og når) jeg skulle gøre det).. ..Hm, men lad mig dog lige tænke lidt mere over relationer som kommentarer (igen), inden jeg går videre.. ..Hm, links og rettelser osv. kan godt bare implementeres som normale kommentarer (de behøver ikke at tage reference-input; referencer kan bare gives som URL'er).. ..Ja, og så har vi vel nærmest bare relevancy tilbage, og her kan man jo som sagt også bare bruge aktivitet som et godt mål.. ..Hm, og/eller kunne man ikke måske også bare bruge en konvention om, at "tommelfinger op" betyder "er relevant" (og så ændre logoet)..(?) ..Ah, men jeg har jo netop også haft tænkt (og det var jo et tiltænkt punkt i dispositionen endda), at kommentarer kunne have alle salgs ratings.. ..Og så kan relevans-ratingen måske bare gøres til en særlig rating for kommentarerne (som de så pr. default er sorteret efter).. Ja.. Brugbarheden af dette bliver så begrænset; det vil særligt være relevans, enighed, og så diverse advarselslamper såsom spoilers og NSFW, som brugerne vil være interesserede i at benytte, men disse er også vigtige nok ting i sig selv. Cool.:) 




\subsubsection{More on using tags for searches and for feeds..}
I have already mentioned that tags can be used for searches on for instance product pages...

%Det er vel især det med at søge ved at vælge midtpunkt og kurvebredde for et sæt af rating-akser, der skal nævnes her, ikke?.. 






%"Noget andet jeg lige kom til at tænke på, er at sorterings-spørgsmålene måske bare kan behandles i rating-tag-sektionen (under feed-/søgningssektionen). For både forfatter-vægt, aktivitet og tid kan jo også være relevant her.. Og så kan jeg måske i denne sektion bare nævne, at der skal være den omtalte relevans-linje for hver fane, og at man selvfølgelig ellers kan bruge de samme søgeindstillings-principper, og at man altså f.eks. kan sortere med større forfatter-/poster-vægt samt efter tid og/eller aktivitet. Og i tag-rating-sektionen kan jeg så bare nævne der, at brugere jo gerne selv må kunne indstille graden af disse ting; hvorfor ikke?" 




\subsubsection{Advantages and disadvantages}
%Even though it might seem like a small change from the current systems, I believe that tag ratings can make a big difference in how users... 

%Disp:

%A: Kan gøre det nemmere hurtigt at tilføje sit bidrag til tags, når det kommer i form af en rating.

%A: ...











%Adding ratings to tags also makes it very easy for users to contribute with their opinions, since it only requires a few clicks with the mouse to add an answer to a rating. There is no need for users..



%%Og dette kommer helt sikkert i appendix, hvis der bliver noget, der skal med:
%\subsubsection{Additional comments on how to implement a system such as this} %(Maybe.)
%...

%Maybe mention 'suggesting relevant tags' here (if not before), and maybe mention organizing tags..(?)
%Ah, hvad med at jeg bare laver en ny lille sektion efter disse to første, hvor jeg giver "additional comments about implementation"..?(!) Eller jeg kunne slå 'organized comments' sammen med denne..? ..Hm, hælder umiddelbart til den første løsning.. ..Hm, enten det, eller også skulle jeg bare gøre, som jeg har lagt op til nu, og så bare nævne så meget af det som muligt her for tags, og så nævne resten ned til sidst under 'organizing comments'..? ..Ah, jeg mangler da at snakke om relevancy her for tags! Jeg må lige tilføje en sektion..! ..Nå nej, det var jo her jeg tænkte, at man jo også bare kan klare sig med mere trivielle måder at finde frem relevans (hvor brugerne ikke er så direkte involveret i processen).. ..Okay, men så kunne jeg jo netop nævne dette under 'additional comments'.. Ah, det var jo også lige præcis det, jeg lagde op til med "Maybe mention 'suggesting relevant tags' ..." He.. Ja, så jeg tror jeg gør det, som jeg har lagt op til her, hvor begge disse to sektioner, denne og den næste ('organized comments') får en sektion til sidst, og hvor jeg samlet set lige får introduceret pointen om at bruge formelle, triplet-agtige systemer (men altså med visse højere-ordens-relationer også) til at bygge det på. 

\lipsum[1]





\subsection[Organized comments]{Organizing comments by tags}


\subsubsection{Introduction}

%Comment sections can be found on various sites, and here they usually only have a few different sorting options..
%But imagine that we can organize comments by their types / subjects, e.g. by bla bla or bla bla bla..
%The solution I will propose to do this, is to make comments first-class citizens and thus to treat them much like the other (main) resources on the site. In particular comments should be able to get all the same tags and ratings as other ressources. I also want comments to have their own page. When viewing comments under a resource, there might not be as much room for all kinds of tags, as for the resources themselves. But mostly we will probably also only need 'agreement,' 'relevancy' (for its parent resource) as well as some categori tags and some warning tags. And otherwise you can go to the comments own page, if more space is needed to show a list of different tags.. (Dette bliver nok ikke dispositionen, men lad mig lige færdiggøre..) Another reason for comments to have their own page, is so that that page can form the starting point for a subthread for that comment.. 
%Hm, kan jeg mon skille det lidt ad igen; måske på samme måde som før..? ..Eller hvad..? ..Eller skal jeg forklare begge ting her i introen, og så jo bare self. vente med at forklare om fane-algoritmerne/-indstillingerne.. (Hm, kunne man forresten ikke også give mulighed for faner med variable i indstillingerne?.. ..Jo, det kunne jeg vel også godt nævne..) ..Hm, men jeg kunne nu også godt vente med pointen om, at de skal have deres egne sider. For hvem siger, at man ikke også kan få en god drop-ud-menu til kommentarerne, når man ser dem i kommentarsektionen, hvor der også kan være plads til mange nok tags (og hvor man jo f.eks. kan udvide og få en hel liste)? Hm, men skal jeg så nævne denne drop-ud-menu? Ja, hvorfor ikke?. .. ..Ok. 


%"%Comment sections can be found on various sites, and here they usually only have a few different sorting options.."
%"But imagine that we can organize comments by their types / subjects, e.g. by bla bla or bla bla bla.."
%Then it would be easier to find bla bla or bla bla bla. 
%*(Og jeg føler nemlig at det er okay at udskyde use cases, hvis jeg bare lige finder på nogle gode eksempler her..) ..Hm, lad mig hellere lige overveje allerede, hvad disse eksempler kunne være, for det er så ret vigtigt.. ..Jamen det kunne jo være f.eks. rettelser til videoer og links.. Det kunne være diskussionstråde (men skal jeg så nævne det med, at de kan få deres egen side her bare..?).. Ja, jeg kan vel godt nævne her, at kommentarenes egne pages så eksempelvis kan bruges som starten til en tråd. (Og man herved kan lave tråde rekursivt.) Dette kan så bare være sidste eksempel nævnt her.. ..Jeg kan selvfølgelig også nævne 'reaktioner til resourcen' som et slags udgangspunkt-eksempel.. Og, hvad mere?.. ..Oversigt/metadata..? ..Ja.. Så: Reaktioner (og måske positive, negative og mixed..), kilde-links, opsummeringer og(/med) timestamps (hvad jeg mente med "metadata"), rettelser og diskussioner til indhold. Det må da være godt, ikke..? ..Jo. ...Hov, men nu har jeg jo ikke nævnt f.c.citizens allerede ifølge denne disposition.. Hm.. ..Hm ja, og det er ikke en skide god idé heller at komme med flere eksempler, inden løsningen er bare nogenlunde forklaret.. ..Hm, men så må jeg vel bare lægge mere ud med løsningen, eller hvad..? (Går lige i gang så med en ny disposition nedenfor..)
%
%"The solution I will propose to do this, is to make comments first-class citizens and thus to treat them much like the other (main) resources on the site. In particular comments should be able to get all the same tags and ratings as other ressources."
%..Comments should then be in the form of a text (payload) as well as meta-data about what its parent resource is.
%With this in place, we can then start using category tags (i.e. tags with categorizing predicates..) to sort the comments into different comment tabs.
%It might be a good idea to let the poster's rating weigh more in.. nej, det kan bare vente til næste sektion.. ..måske..
%We cannot necessarily expect there to be as much room for tags as for the resource themselves. But we can implement a drop-down-menu and/or we could have a link to the comments own page (as it is a first-class citizen now).. (..Hm, ikke sikkert at dette skal med her, så..)
%Apart from category tags.. Hm, hvordan får jeg lige nævnt relevancy-tagget (og agreement-tagget), uden at det bliver alt for meget en afvej..? ..Ah, kan jeg ikke bare udskyde dette til næste undersektion; det handler jo om sorting?. Jo.:) 


%..Hm, men måske skal jeg bare lige vente med, at nævne diskussioner, og så kunne man godt nævne eksemplerne inden man forklarer løsningen på at få det til at ske..(?) ..Jo. Jo, og så kan jeg jo nemlig bare nævne dette som en smart ting ved at gøre dem f.c., som jeg jo også havde tænkt mig. Så disp: ..Bliver bare ret meget som i ovenstående.. ..Og så skal jeg bare have nævnt (måske til sidst), at comment pages så altså kan bruges som start i diskussionstråde. ..Ok!


%... Okay, nu har jeg faktisk tænkt på et ret alternativt system..!.. Kommentarer kan stadig være f.c., men dette bliver så primært bare til warning tags og måske også lidt til 'agreement' og sådan. Men for relevancy tænker jeg nu, om ikke man i stedet simpelthen kunne have up- og down-votes (som jeg ellers før ikke har brudt mig så meget om), hvor et up-/down-vote så altid tæller i hovedfanen og derudover også tæller i den fane, hvori det blev givet?. Og så tænker jeg også nu, at man så kunne have et flag til specielle faner, som siger: "kommentarer i denne fane har (/ bør have) samme semantik." Så på den måde kan man altså både have brede faner med forskelligartede kommentarer, og så kan man også have snævre faner, hvilke så bl.a. kan bruges til at eliminere gengangere..! Og desuden tænker jeg så også lige en ekstrating, nemlig at siden kan vise både den faktiske vote samt en samlet vote for alle kommentarer, der er ens med den pågældende (for siden kan jo godt se, hvem brugerne er, der har givet kommentarerne (så hvis de har upvoted flere, så tæller stemmen bare kun med som én stadigvæk)). Umiddelbart synes jeg altså rigtig godt om disse idéer, også selvom det på nogen måde virker som et mere ærbitrært, avanceret system, og på en måde altså lidt som til skridt tilbage. ..Men umiddelbart synes jeg altså stadig rigtigt godt om det..! ..Nå ja, og fanerne oprettes altså så bare frit (ligeså frit som at skrive en kommentar). Og så må kommentarer bare få en 'add to'.. eller måske endnu bedre en 'view in'-knap, hvor man så kan vælge en anden fane at se kommentaren i, og hvis kommentaren så ikke er der, skal den så bare stå med lidt grå/gennemsigtige toner, og fane-voten skal så tydeligvis (måske med fed) være lig 0, og hvis brugeren så upvoter kommer kommentaren nu ind i denne fane (men brugere behøver ikke at få vist de lavest votede kommentarer for en fane..).. Eller de lavest votede kommentarer kan om ikke andet stå under en linje, så brugeren er opmærksom på, at her er der altså forøget fare for spam, når man når ned under denne linje.. ..Okay, men lad mig så altså lige summe lidt over denne nye (spændende!) idé. :) 
%(11.11.21) Ja, jeg tror altså, det holder.. Og man kan i øvrigt også vægte up-/down-votes på samme måde efterfølgende, som man kan vægte ratings. Og når det kommer til sorteringsalgoritmer, så kan man netop nok bare have nogle overordnede algoritmer, som så må/kan gælde for alle tabs på én gang.. Ja.. I øvrigt skal folk kunne up-/down-vote tabs, så i det hele taget skal tabs altså fungere meget ligesom kommentarer. Man kunne endda have en speciel tab-tab (fane-fane), hvor alle tabs (self. på nær denne specielle tab) simpelthen optræder som kommentarer. Fedt, det vil jeg foreslå! 

%Så den dispositionen skal altså nu ændres en del.. Jeg må jo tage udgangspunkt i idéen så, om at man kan have ordnede kommentarer i forskellige faner, hvor faner altså kan tilføjes som var de kommentarer. Og min idé er altså så, at et up-vote (eller down-) tæller for den pågældende fane såvel som hovedfanen, hvilket altså skal være en ting, medmindre man self. er på hovedfanen allerede; så tæller den bare her. Faner kan som sagt selv upvotes som var de kommentarer, og faktisk så kan fanerne også vises (lodret) som kommentarer under en fane-fane. 
%Dette bør altså så være omdrejningspunktet, og så kan jeg nævne kommentarer som FCCs i et følgende punkt. Om dette kan jeg nævne.. ja, nu behøver jeg jo så ikke at snakke om.. nå jo: Jeg kan nævne, at votes'ne ikke altid er tilstrækkelige, for man kan godt synes en pointe er vigtig at diskutere, også selvom man ikke er enig. ..Hm, men så bliver det bare ikke helt så relevant det med at samle votes sammen fra tilsvarende kommentarer.. ..Hm, og man kunne ikke (nok lidt kompliceret men) have en speciel up-vote flavor, hvor personen siger, at kommentaren er relevant, men man er ikke nødvendigvis enig med den..(?..) ..Hm, og på den anden side kan man også vise efterspørgsel efter noget, der er relevant.. ved at upvote fanerne.. hm, men det ville nu være rart, hvis upvotes bare betød relevans og ikke andet.. Hm, men hvad var problemet i at lade det være det og så bare have et agreement-tag.. hm, og jeg kan jo forresten bare droppe det med at samle votes'ne; det var alligevel også en lidt avanceret (og lidt kompliceret) ting.. ..Hm, og man kan ikke samle agreement-ratings..? ..Hm, okay der kunne være noget om dette, men så kunne jeg også i stedet overveje, om 'agreement' så ikke bare \emph{skal} være en speciel ting, der også vurderes i form af votes og ikke af ratings (og som så altså kan samles, ligesom jeg foreslog det)?.. ..For måske er det endda en god idé, hvis agreement også gives som votes, for ellers kan der være ret stort encitament til at snyde og bare bruge ekstremerne og rating-skalaen det meste af tiden.. Ja.. Hm.. (Er der egentligt andre steder, hvor dette kan blive mere problematisk, end jeg har tænkt..?) ..Hm nej, måske er det faktisk fint nok, især nu når man så får begge ting: både mulighed for at give votes og får at rate diverse tags. Noget andet jeg lige kom til at tænke på, er at sorterings-spørgsmålene måske bare kan behandles i rating-tag-sektionen (under feed-/søgningssektionen). For både forfatter-vægt, aktivitet og tid kan jo også være relevant her.. Og så kan jeg måske i denne sektion bare nævne, at der skal være den omtalte relevans-linje for hver fane, og at man selvfølgelig ellers kan bruge de samme søgeindstillings-principper, og at man altså f.eks. kan sortere med større forfatter-/poster-vægt samt efter tid og/eller aktivitet. Og i tag-rating-sektionen kan jeg så bare nævne der, at brugere jo gerne selv må kunne indstille graden af disse ting; hvorfor ikke? 
%Ok, men det lyder da umiddelbart ret godt, det her..!.. ..Hm jo, der kunne godt være folk, der ville prøve at booste ratings (eller omvendt) ved at vælge ekstremerne i høj grad (fordi man gerne vil ændre den endelige score).. :\ Hm.. (Og altså f.eks. også bare angående tags såsom 'scary' osv.:\..) ..Hm, og jeg skulle lige til at spørge, hvorfor dælen det lige er/var, at jeg har tænkt, at dette ikke bliver noget problem, men nu kom jeg så lige på en anden tanke også: Kunne det lade sig gøre at modvirke denne effekt ved at køre statistik over folks vurderinger og så simpelthen omvægte/omtransformere deres vurderinger..?! Hm..(!) ..Ej, det er da faktisk en vildt god idé..!.. ..Det kunne godt virke!.. ..Ha! :) ..Hm, og man kan vel så netop stadig ligeså godt bruge det til at finde sine korrelationsvetorer (langt bedre end hvis det var binære ratings)?. .. Hm, lad mig lige tænke/summe over dette. Jeg håber at systemet stadig vil være ligeså brugbart ift. bl.a. korrelationsvektorer, og de andre ting, jeg har tænkt.. 
%... Okay, jeg kom hurtigt frem til, at nej, det vil være for nemt at cheese et sådant system, for så kan man bare give en masse normale vurderinger til ting, man er mere ligeglad med. Hm, og nu hvor jeg skriver dette, kan jeg også med det samme se, at det samme problem.. Hm.. Eller hvad.. Hm, jeg har nemlig imellemtiden fundet en anden mulig løsning, nemlig at bringe b.d. ML hurtigere ind i billedet, men nu hvor jeg skrev dette, kom jeg så til at tænke på, at man måske nok kan cheese('e) dette på samme måde. Medmindre, kom jeg så yderligere til at tænke på, at man måske bare overså alle de gængse svar mere, og så bekymrede sig mere om afstikkere.. hm, men så er problemet jo bare, at folk så ikke vil have lyst til at rate for meget.. hm, medmindre det jo bliver en meget bevidst handling, når man vurderer over eller under, hvad ens korrelationsvektor-sammensætning forudser.. Okay, så idéen er i hvert fald ikke død endnu; jeg skal lige tænke noget mere. 
%... Nej, brugere skal selvfølgelig ikke få noget ud af at rate en masse ressourcer lige i gennemsnitspunktet frem for at rate eksempelvis ud fra en gauss-kurve og altså med tilfældige udsving. ..Hm, men.. dette er jo lettere sagt end gjort; det er jo netop problemet, hvordan man kan gøre, så dette ikke er tilfældet.. Hvordan kan man.. Hm.. ...Og især uden offentlige vurderingsprofiler.. (Eller uden at disse i hvert fald er normale at have.. (hvad man så måske altså kunne overveje at ændre på..))
%... Ah, man kan jo altid bare bruge medianen mere! Ja..! Ok, men skal man så også gå tilbage til ratings for relevancy og agreement..? ..Hm, hvorfor ikke..? (Altså det skulle vel være pga. det potentielle besvær med de ekstra klik..?) 
%... Okay, virkeligt fedt, at medianen kan redde det! Men jeg er kommet lidt frem til, at jeg altså hælder til, at både relevancy og agreement skal være votes. Relevancy fordi man jo har bedømmer noget relativt, og så giver det altså mere mening bare at give voteringen efter, hvad der er over og under, i stedet for simpelthen at skulle forudsige (!), hvor postet kommer til at ligge henne.. Og for agreement, jamen her er det også bare langt nemmere at skulle forholde sig til, er man enig, eller er man ikke enig. Og for begge ting for man altså bare ikke nok ud af at bruge ratings, til at det er det værd. For det koster nemlig først og fremmest mere tænketid (og generelt kræver binær rating mindre tid end kontinuer; det er diskret rating (og særligt når der kun er fem stjerner), hvor jeg mener, at kontinuer rating må slå det i tid), og designmæssigt er det også mere kompliceret, både at få plads til og at bruge. Dog mener jeg stadigvæk, at kommentarerne også ideelt set bør kunne få tags! Jeg er i øvrigt også kommet på et andet godt eksempel udover warnings, og det er, at mange posts på internettet er smarte replikker og modreplikker (altså comebacks), og disse kan nemt få super mange likes, også selvom at de enten ikke holder vand logisk set, og/eller at de er hykleriske, når man analyserer dem til bunds. Man støder på rigtig mange "smarte" bemærkninger, som, når man ser på dem ift., hvad de egentligt bidrager med til diskussionen, bare ikke bringer diskussionen nogen vegne. Vi snakker altså om kommentarer, som i virkeligt ikke gør andet end at håne og/eller lave grin med modparten, men som på ingen måde vil kunne overbevise modparten til at skifte mening (i og med at de rammer forbi og ikke adresserer modpartens egentlige holdninger). Her vil det så være smart med et tag, som siger: "smart line on the surface but with no convincing power to it when analyzed" (eller hvis man kan finde på en noget bedre eller kortere at opsummere tagget med).. "smart, unless analyzed".. "smart but not convincing".. I øvrigt vil "potential hypocrisy" også være et godt tag; et man næsten altid vil kunne bruge, når nogen siger noget a la "For noget tid siden, da snakken gik på det og det, der var jeres holdning sådan og sådan, men nu hvor det handler om det og det, så siger i pludelsig sådan og sådan, ha!" Disse slags bemærkninger er altid rigtigt farlige, for hvis man istemmer sig dem, samtidigt med at man dog selv også havde én holdning dengang, og har den modsatte holdning nu (hvad de fleste nemlig vil have), så er man altså selv hyklerisk. Hvis man med andre ord istemmer sig en bemærkning om hykleriet ved et "holdningskift," når snakken bare er udskiftet med noget, hvor rollerne på en måde er byttet om, men at man også selv er skyld i samme "holdningsskift," jamen så er man jo også hyklerisk selv (i hvert fald alt andet end lige). Så ja, her er nogle eksempler --- som man dog også kan se som en slags "faresignaler," men hvor de altså dog har en lidt anden karakter end f.eks. 'spoiler' og 'nsfw' osv., især fordi de også kan bruges til at irettesætte posteren lidt, og ikke bare er en advarsel til andre brugere.. 
%Nå. Det næste, jeg så skal finde ud af, er noget jeg også har overvejet, nemlig om man alligevel skal have underfaner (og måske rekursivt), og/eller om man virkeligt skal bruge kommentarerne som startpunkt i deltråde.. ..Hm, i princippet ville der vel ikke være noget i vejen med at have underfaner og under-underfaner osv., på nær at det jo så måske kan blive uoverskueligt. Og angående deltråde så er dette jo også kun i princippet godt, hvis man i hvert fald ser bort fra samme argument.. Om deltrådene så har det bedst som en expand-collapse-struktur, eller om man skal gå til kommentarens egen side, det kan man jo så diskutere.. ..Hm, men man kan jo bare vise hovedfanen, hvis man gerne vil kunne expande en deltråd, og så kan man gå til selve kommentarens side, hvis man gerne vil have denne sorteret i faner også! :) Og så er det bare: Underfaner? (Og jeg skal også lige overveje, om ikke b.d. ML nu faktisk skal knyttes til rating tag-sektionen i stedet..) ..Hm, måske er det lige overdrevet nok i hvert fald med rekursive underfaner, især fordi jeg jo allerede tænker at dele votes'ne op i relevancy og agreement, hvilket nemlig allerede vil komme problemet lidt til livs, med at folk bare har det med at upvote ting, de er enige med, og ikke så meget de interessante diskussioner.. ..Hm, men hvad med at brugerne bare aktivt skal folde flere fane-barer ud, hvis de kan se, at der findes underfaner til en fane i den nederste bar, og gerne vil se disse? For hoved-underfanen bliver jo så bare identisk med overfanen, og så er det altså bare denne man ligesom er på pr. default, når man ikke har foldet flere underfane-barer ud. ..Hm, muligvis værd at foreslå.. ..Og hvis underfaner er gentaget i flere blade af det overordnede træ, så kan brugerne (eller rettere brugeren, der opretter nr. 2 af to ens faner) jo bare sørge for at kopiere, så siden ved, de to underfaner er identiske, i stedet for at oprette en ny en. ..Ja, umiddelbart værd at foreslå.. Ok.. 
%Hm, b.d. ML-sektionen kan måske nok godt stadig være sin egen sektion, men den kunne i så fald måske godt lige rykkes op, så den kommer efter tag rating-sektionen.. 
%Hov, jeg glemte da en ting! Jeg fik også en idé angående relevancy og agremment votes'ne: Jeg forestiller mig, at de.. Hm, eller.. Hm.. *(Jeg tror muligvis bare de to votes skal klistre til hinanden for det første klik..)

%(13.11.21) Okay, der er sket en del i går og i dag. I forgårs aften kom jeg lidt frem til en følelse af, at der manglede noget, når det kom til vurderingerne, og hvordan de ligesom kan bruges som alsidige bedømmelser; jeg følte, at det muligvis ikke helt det rakte, som det var.. I går tog jeg så en hel tænkedag og kom frem til nogle ting om at bruge personlige tags til at beskrive, hvad man kan kalde psykologiske parametre, og om at give vurderinger, der så kan referere til disse person-tags. Jeg fik også tænkt over nogle andre småting. I dag har jeg så tænkt videre over person-tags/følelsesparameter-tags og har videreudvilklet idéen. Desuden er jeg lige kommet frem til, at ønske-tags faktisk bare skal være en slags "specifikke tags," som jeg har kaldt dem, hvilket desuden altså inkluderer kommentarer, som kan være skrevet specifikt til den enkelte ressource, og som man så kan vurdere agreement med (med en kontinuer rating). Sådanne specifikke tags skal faktisk ikke blandes sammen med kommentarerne, og de skal altså ikke optræde som kommentarer i kommentarfeltet. Hernede skal gælde andre ting, og særligt mener jeg altså nu, at kommentarer skal vurderes agreement for med voteringer (ikke kontinuere ratings). Og specifikke tags skal gerne overholde nogle formelle krav, så man kan fortolke rating-aksen på en standard måde.. Og nu tror jeg altså, at jeg kommer til at hive ML-sektionen op og gøre det til en del af rating-tag-sektionen, hvori jeg så ovenikøbet vil introducere først specifikke tags sammen med ønske-ratings'ne samt også introducere person-følelses-parameter-tags'ne sammen med den måde, jeg nu har tænkt, at man skal kunne parre dem med de specifikke ratings for i sidste ende at forklare sin overordnede vurdering. Lad mig lige kort forklare sidstnævnte nogenlunde (som altså var den videreudvikling, jeg lige talte om). For hver specifik rating/tag, man ønsker at tage med, kan man vurdere en værdi for, hvordan forskellen på den faktiske rating og så ønske.. Hm.. (Skal specifikke tags så formuleres som predikater, eller skal de formuleres som ønsker direkte, som jeg ellers lidt gik og tænkte nu her..?) (Hm, lige lidt indskudt, så jeg ikke glemmer det, kan jeg lige nævne, at eksempler / use cases jo bare må nævnes undervejs så, måske bare i subsubsections (for de nuværende bliver jo til subsections).) ..Hm, jeg tror umiddelbart, at der gerne må være tale om ønsker til at starte med, når de kommer til specifikke ratings, og så kan man altså også bare danne ønsker via kategori-tags m.m. i form af mine "ønske-ratings." Min pointe var så, at man skal kunne parre disse med person-tags og give hver parring en værdi, og at alle disse værdier så i sidste ende skal forklare ens endelige score-vurdering for ressourcen --- hvor man dog implicit har et 'other'-bidrag, så hvis pointene ikke matcher den endelige score givet, så bør dette altså være fordi (og det bør også forstås (og vises!) sådan), man har nogle andre grunde, som man bare ikke har gidet og/eller villet liste. Så vi har altså prædikat-tags ad libitum, ønske-vurderinger ad libitum, person-vurderinger til bl.a. at forklare, hvilken type man selv er, og sidst men ikke mindst har man altså også score-bidrags-forklaringer --- nå ja, samt også den endelige score i sig selv --- der forklarer, hvad der har bidraget (positivt og negativt) til ens endelige score af en ressource, hvilket altså så kan udpensles helt ned i detaljer omkring, hvilke kritikpunkter/udmærkninger er tale om, samt hvilke personlige præferencer de strider imod / er på linje med, og hvor hver af sådanne præference-øsnke-par kan tildeles et score-bidrag, som viser hvordan de tæller i den samlede score.. Hm, men hvad så, hvis man gerne vil vise, hvad ressourcen kunne opnå af yderligere point, hvis et ønske havde været bedre opfyldt..?.. ..Ah, man kunne eventuelt give en "ud af y point"-tilføjelse til en "bidrager med x point"-erklæring, hvis man vil uddybe dette.. ..Hm, skal man ikke også kunne slå præferencer sammen..? ..Sikkert ikke helt dumt.. 
%... Nå ja, og jeg kom også på en anden ting i går, som virkeligt er vigtig --- den cementerer værdien i rating-tag-idéen --- og det er at systemet også skal kunne inkludere alverdens ting fra det virkelige liv såsom hobbyer, aktiviteter, livsstilsråd, attraktioner, ferieforslag, finansielle råd, råd til sociale forhold, råd til socialt samvær, helbredsråd, fritidsinteresser, bøger, brætspil, politiske meninger/ståsteder, positiv psykologi-råd/-indsigter, indsigter generelt, livsmål, personlighedsmål, advarsler, meninger om god opførsel, meninger om gode persontræk, spændende videnskabelige emner, spændende journalistiske emner, spændende nørdede emner, osv. I øvrigt skal jeg også lige nævne, at person-type-parametre ikke kun behøver at være grundlæggende persontræk ift. ens meninger, men også kan være holdninger og moral, og også vidensniveua samt hvilke ting man allerede har læst/set/prøvet.. 
%Nå ja, og jeg skal også lige sørge for at give ekstra luft til idéen omkring kvitteringer; jeg tror, det kan blive en rigtig vigtig ting på et tidspunkt.. 
%Men ja, lige for at vende tilbage til alle "real-world-ressourcerne," så er det altså dette, der gør at min idé bestemt holder vand i sidste ende, og altså at folk i sidste ende helt sikkert vil kunne få kæmpe gavn af bl.a. at kunne gå så meget op i at finde ind til sine egne præference-parametre (og at der samtidigt kendes statistiske korrelationer, der kan bruges til så at give gode nye forslag, der passer til disse); fordi det kan bruges til så vildt mange ting. 
%Hm, nu kom jeg også lige til at tænke på: Mine "civilforeninger" kunne jo så passende tage udgangspunkt i persontype-grupper (der jo ikke bare kan handle om interesser, men også om brugerens livssituation (m.m.) generelt), når vi når til disse. Og hvis disse så virkeligt bliver en stor ting i fremtiden (forhåbentligt en ret nær en), så kunne alt sådan noget som identifikation af personer sikkert også i sidste ende varetages af disse foreninger. Og hvis man også i høj grad så køber sine varer osv. igennem sin forening, så kan denne jo også varetage \emph{kvitteringer} m.m. (som jo altså er meningen, man skal kunne bruges som en slags kupon til at give ens vurderinger af et produkt (m.m.) ekstra vægt).. 
%Så ja, jeg regnede ikke med, at der ville ske så store ændringer med mine idéer, inden jeg fik lavet udkastet, men sådan noget ved man jo aldrig. Virkeligt fedt, at jeg har indset disse ting. Håber det holder, og desuden at det nu ikke kommer til så at tage vildt lang tid ekstra at skrive om --- det må vi jo se. :) (13.11.21) 

%(15.11.21) Okay, i går blev en summe-dag, og jeg kom faktisk frem til nogle ændringer i går aftes. Og her til morgen er jeg så kommet frem til, at disse ændringer ikke gør det meget mere besværligt at lave ML over det. Jeg har også tænkt over anonymitet og har nogle ting, jeg skal skrive. *(Og jeg skal også lige huske en tanke om, at.. pointere det smarte i, at man kan se sine tidligere vurderinger på aksen, når man rater.. hm..) Nu (sen formiddag) fik jeg så lige en ny indskydelse, som jeg vil skynde mig at skrive ind her! Måske skal "kategorier" faktisk heller ikke bedømmes via tags (i.e. "kategori-tags")..!..(?) (Så måske man skulle udvide fane-systemet til også at kunne indeholde hoved-ressourcerne..) Det vil jeg lige tænke noget mere over!


%(18.11.21) Okay, det blev til en masse tænkedage! Nu er klokken forresten også 15, så i dag har altså også været en. Lad mig prøve at opsummere de tanker, jeg har tænkt siden sidst, og de ting, jeg er kommet frem til. Jeg kan ikke helt huske præcist på stående fod, hvad jeg kom frem til d. 14/11 og om morgnen d. 15/11, men det.. Ah jo, det handlede vist om nogle ændringer til mine idéer omkring personlige paramtetre og ønske-vurderinger / specifikke vurderinger til at forklare sin vureringsscore i sidste ende. Jeg kan vende tilbage til dette emne senere; nu er der lige en hel masse, der lige skal skrives først. Mine tanker omkring anonymitet, som jeg nævnte i ovenstående paragraf, er mere vigtige at få skrevet om. De handler om, at alle frit skal kunne oprette flere profiler, heriblandt 'personlige profiler,' ikke-personlige profiler til alment brug, profiler til brug omkring sensitivt indhold, hvis man ikke vil have denne data direkte forbundet med den almene profil (som altså til en vis grad \emph{er} anonym, men ikke 100 \%; ikke for de mennesker, der gerne vil kigge over skulderen, når man logger ind på den (hvilket man jo tit gør, for denne er altså den "almene profil" til normalt dagligdags brug)), og muligvis også profiler bregnet til semi-personligt indhold, men hvor man ikke har lyst til at dele denne data med gud og hver mand (dette kunne f.eks. være bedømmelser af ting i ens nærområde). Pointen er så, at de forskellige profiler kan "linkes blødt" med hinanden, hvilket vil sige at man linker diverse andre profiler, ikke direkte til sin almene profil, men til en større gruppe af profiler, der nogenlunde tilsvarer sin egen. Og når nu vi snakker om anonymitet, så kom jeg d. 16/11 faktisk endelig frem til en protokol, hvorpå brugere kan oprette disse bløde links på en decentral måde! Så nu behøver man altså slet ikke centrale servere i systemet, andet end til at gemme store mængder data og/eller at administrere "brugergrupperne" (de FOAF-agtige) og evt. andre algoritmer. Men intet af dette indebærer noget, der skal holdes hemmeligt (i hvert fald hvis brugerne så nu bare underskriver deres posts i stedet for at logge ind et sted), så systemet kan altså laves helt åbent i princippet (så alle i princippet fint kan få adgang til alt data)..! Nå, men tilbage til d. 15/11: Her kom jeg så på selv tags og (navigations)link-menu osv. alt sammen bare kan implementeres via kommentar-/ressource-fane-systemet..! Så nu er jeg altså all-in på de faner der..! :D Samtidigt kom jeg også frem til, at alle ressourcer/kommentarer/faner (eller 'objekter' tror jeg, jeg vil begynde at kalde dem for nu..) nu bare kan have de samme ting, nemlig relevancy-votering samt agreement-vurdering, og at sidstnævnte (i.e. agreement, ikke relevancy..) nu bare kan splittes op i to, så man \emph{altid} bare har valget imellem enten at give en tommelfinger op eller ned \emph{eller} at give en vurdering på en akse. ..Hm, jeg kan lige overveje, om det mon kan give mening at åbne op for akse-vurderinger, når det kommer til relevancy.. Jeg kom også lige til at tænke på, om nu min anonymitetsprotokol nu også har en vis egenskab, eller om jeg bare kom frem til, at denne egenskab nu ikke længere var behøvet, og selvom svaret nok var det sidste, så er jeg altså lige kommet frem til, at man faktisk godt \emph{kan} få protokollen til at opfylde denne egenskab (som sikrer en én-til-én uddeling af nøgler, og hvor protokollen ikke er sårbar overfor spam/trolling/DoS). Nice nok.. Og så kom jeg også d. 15/11 (eller 15./11.?.. hm nej, nok ikke..) frem til, at brugere faktisk i høj grad skal vurdere hinanden med en slags point, som repræsenterer, hvor sandsynligt det ifølge pointgiveren er, at pointmodtagerens vurderinger (muligvis inden for et område) vil stemme overens med giverens egne meninger. Så altså et slags FOAF-system / tillidsuddelingssystem, men bare hvor der ikke er fokus på at bedømme, hvor sandsynligt det er, at brugeren taler sandt om et emne, men i stedet hvor sandsynligt det er, at brugerens svar vil matche, hvad man \emph{selv} ville have givet. Med andre ord handler pointene altså om vurderinger, man selv som pointgiver kan tjekke! Og i hele idéen ligger så en forventning om, at brugere, der vurderer, at visse andre brugere nok vil være enige med dem selv i en vis grad, så også angiver, hvor 'vakst' de vil holde øje med disse brugere for at se, om deres vurderingssvar nu også stemmer overens med den bedømte 'enighedsgrad,' eller om man skal revurdere denne. Lidt af en mundfuld, og jeg skal nok komme ind på, hvorfor jeg altså tror, at dette vil være smart, når jeg skriver de faktiske noter over mine idéer (dette er nemlig bare en lidt hurtig kronologisk gennemgang af dem). (Men det smarte ligger jo altså i, at man herved kan overvåge ~tillids-vurderingerne løbende (hvor vi nu altså snakker om en speciel form for "tillid").. samt ikke mindst at denne form for tillid nu sagtens udelukkende kan være baseret på den online aktivitet, og ikke på nogen personlig relation..)
%Den 16/11 kom jeg så altså bl.a. frem til en anonymitets-protokol (eller hvad man lige skal kalde det), og jeg kom også frem til, at jeg nu faktisk nok bare skal lave min "udgivelse" meget mindre formel; mere bare som nogle noter, der bare lige er en anelse finpudsede, men så heller ikke mere end det. Dette er ret nice, for det gør, at jeg kan blive færdigere hurtigere :) (Og der er lidt en grund til, at jeg nu føler, at sådant et format vil være passende --- nemlig pga. nogen af de seneste ændringer --- men det vil jeg nu ikke prøve at uddybe nærmere (det er jo bare pga. min nuværende mavefornemmelse med det hele, så det kan ikke nødvendigvis forklares, selv hvis jeg prøvede).)
%Den 17/11 (i går) kom jeg også frem til en del ting. ...
%(19.11.21) Ja, en del forskellige ting. Jeg kom for det første på, at man også kan uddele bruger-til-bruger-point via dette kommentarsystem ved at hver brugerprofil jo bare får sin egen side, og så kan man give forskellige point ved at up-rate forskellige kommentarer under en "pointfane".. Brugere skal så selv kunne kontrollere, hvad må vises på deres profilside (i hvert fald for en vis standardvisning).. Hm, og ellers gjorde jeg mig vist mest bare en masse tanker, som i sidste ende er ledt til hvor jeg er nu. I går, d. 18/11, fandt jeg nemlig på, at systemet godt kan være typet, men bare på et lavt niveau, hvor man (i.e. brugerne) faktisk selv definerer typerne ud fra regex-udtryk. I øvrigt fandt jeg ud af, at relevancy-upvotes skal ses som en upvote af en pågældende "sti" (altså fane-sti) hen til objektet. Hermed kan man endda have en fane med stier, og så kan relevancy-voteringsknappen faktisk også bare ses som en abstraktion, der gør det nemmere, i stedet for at man skal klikke sig ind på objektets egen side og så upvote den relevante sti (i.e. den man betragtede objektet under) der. Og med disse ting i baglommen, så kan jeg nu se, hvor simpelt et system, jeg kan bygge det på. Og det gode ved at have et simpelt system som grundlag er, at så får alting en letforståelig semantik, og det bliver nemt at udvikle videre på. Lad mig lige forklare en anelse mere om regex-typerne, inden jeg forklarer de simple system.. Tanken (som jeg fik og udviklede i går) er, at man for en ny fane skal have mulighed for at erklære et regex-udtryk for, hvordan kommentarerne skal være opbygget. Kommentarer der ikke opfylder dette, kommer så ikke ind under denne fane. Desuden kan man gerne kunne benytte samme regex, når man skal skrive en kommentar til pågældende fane, så man f.eks. bare kan fylde ind i en skabelon og ikke behøver at taste hele den ydre struktur ind selv. Desuden har jeg også tænkt, at der godt må være faner, hvor alle objekter af en vis struktur bare automatisk kommer ind. Dette er lidt bare en sidenote her, men idéen kan gå hen og blive ret vigtig. ..Hm, man kunne muligvis endda udvide dette system, så regex-tjekkene udvides med mulighed for certifikat-/underskrift-tjek.. Hm, men hvordan skal man så vedligeholde dette?.. ..Ah, ved at man kan bygge nye faner oven på de gamle (og så kan faner bare få en TTL, og når denne udløber kan fanen erstattes med den mest populære, der bygger videre på denne..). *(Ah, og den opgraderede fane kan udgives inden den gamle udløber, så den nye version kan nå at upvotes inden da..) Ja ok, og heldigvis gør dette ikke systemet mere kompliceret rigtigt, og det kan indføres når som helst, for det handler jo altså bare om at udvide regex-mulighederne (så der er ikke noget, der vil tage skade ved sådan en opgradering). Fedt nok. Så nu kan faner altså diktere, hvad indholdet skal opfylde, hvilket er smart i sig selv, og.. ..og det kan så også bruges til at implementere alverdens nye smarte ting med, hvor man kan trække data.. Hm, jeg tænker bl.a. på at kunne trække navigations-links op i sin egen menu (i.e. fra en fane og op ved siden af selve ressourcen), hvilket siden jo gerne skal implementere ret tidligt, men skulle man mon ligefrem designe et sprog så folk selv kan designe sådan nogle ting som disse?.. ..Hm, det er ikke sikkert, at man skal åbne op for dette nødvendigvis med det samme (men self. på sigt..), men lad mig lige tænke lidt over, hvordan det kunne implementeres.. ..Hm, det kunne godt være, man skulle gøre noget med en syntaktisk opdeling af faner også, så..?.. ..Eller hvad..?.. ..Hm, man behøver jo nok ikke dette i hvert fald til at starte med. Hvis der på et tidspunkt bliver et behov, kan man jo altid bare omdanne systemet en anelse i så fald.. Hm, jeg overvejer så, om faner bare skal gives navne --- som serverne og resten af fællesskabet så har ret til at omtransformere, i hvert fald i et vist tidsrum inden konventioner ligesom begynder at indfinde sig rigtigt (for der er selvfølgelig ingen grund til at ændre på navnene, når først de er blevet en konvention..) --- og at man så bare kan lave et query-sprog til at tage faner ved deres navn og så udtrække data fra deres indholdte kommentarer (som har point og sortering ifølge indstillinger, som kan vælges og justeres af brugeren uafhængigt af, hvordan denne brugerfladeudvidelse virker).. Er det noget a la dette, man bare skal gøre (på sigt)..? ..Ja, det lyder da umiddelbart som en helt udmærket mulighed.. Og hvis man på sigt får brug for det, kan man jo bare udvide dette ved at gøre, så at fane-navne ikke kun kan være konstante men også kan være skabeloner (så man altså kan have en fane a la: "myTabTemplate<x><y>").. ..Ja, alt sådan noget kan man på et tidspunkt udvide systemet med (hvis og når der bliver behov).. Nå ja, og grunden til at jeg snakker om "typer" (og jeg har også skrevet "regex-typer i mine papirnoter), er at jeg så også tænker, at brugere skal kunne definere regex-udtryk som en form for type (i.e. med tilhørende navn), som så kan genbruges af andre brugere i deres faner.. Hm, eller man kunne også se det som regex-variable i stedet.. Tja, det kommer an på, hvad man hælder mest til.. ..Det kan godt være, at det er nemmest, hvis man bare erklærer kommentar-regex'en for en fane i bunden af den, og så hertil altså har mulighed for at erklære den via regex-variable, som andre.. Tja, eller måske skal alle regex-udtryk bare erklæres som selvstændige objekter, og så kan regex-fanerne så refere til disse, når de skal defineres.. (Og regex-objekter skal så kunne defineres ved brug af tidligere.) Hm, får vi så brug for navne her..? Hm nej, hvad med at man i stedet bare bruger lokale navne, det er nok bedst. Så kan brugere bare selv finde konventioner for, hvad de kalder regex-objekterne, når de importerer dem som nye objekter. ..Og på sigt kan man jo altid gøre, så at siden selv foreslår navne til de importerede objekter (evt. bare taget fra de mest populære fra en navne-fane for objektet..).. ..Hm, på den anden side, hvorfor ikke bare i udgangspunktet sige, at regex-objekterne skal erklæres selvstændigt, og så kan brugere bare kopiere? For serveren har jo alligevel brug for, at hele regex'en er gemt samme sted, når den skal bruge den. Så jo, det tror jeg bare, jeg vil regne med.. ..Hm, hvad forresten med at faner bare kan erklære sin egen syntaks, hvis det er nyttigt for dem..? Lad mig lige tænke over, hvornår dette egentligt kunne være nyttigt (for jeg har faktisk ikke tænkt noget eksempel endnu; har bare en vag idé om, at det kunne være nyttigt på en eller anden måde..).. ..Hm, måske kunne et eksempel være, hvis man gerne vil inddele sine menuer i undermenuer, hvor underoverskrifterne så kan tages fra fanerne selv.. Ja, det er da et meget godt eksempel.. Og hvis man så bare giver fanerne mulighed for at erklære deres egen syntaks (først), så kan disse (opdelte) menuer jo bare lede efter de mest populære af disse faner, hente underoverskrifterne fra dem og hente data fra de indholdte objekter (/"kommentarer"). (Og jeg fik vist ikke nævnt det, men pointen er så, at man jo kan lave et sprog, så dataen kan hentes fra de pågældende regex-udtryk (hvorved man altså henter noget af den "udfyldte" data i skebelonen, som regex'en definerer).) 
%Jeg fik også et par flere idéer i går, men dem kommer jeg bare til i teksten her. Det er dog lidt et åbent spørgsmål omkring det med at vise, når en kommentar befinder sig i en anden fane, end den OP postede den til, men jeg tror lidt bare svaret bliver altid at holde styr på, hvad der var OP's valgte sti, og så vise et flag, når objektet vises i en anden fane, end den der er til slut i OP-stien. OP kan så evt. få lov til at tilføje flere OP-stier til et objekt, så dette flag ikke vises.. (Altså at OP med andre ord selv kan fjerne omtalte flag for andre faner, hvis denne vil.) 
%Bemærk, at jeg lidt er tilbage nu til et system, som ligesom kan gå hen og nærmest \emph{blive} det semantiske web, men hvor jeg før fokuserede min idé meget omkring prædikat-overskifter, hvor sektionen automatisk kunne udfyldes passende til brugeres indstillinger, så tager mit nuværende system udgangspunkt i noget mere simpelt: Et system med ressourcer og med ordnede kommentarer (ordnet med faner, som bare er en slags kommentarer selv!), og hvor selve ressourcerne så faktisk også kan endda kan ordnes på samme måde (nemlig via (rekursive!) faner!). Og så kan alt semantisk data uploadet, ikke som tripletter (hvor man uploader ressourcer og 3-ære links imellem dem), men ved at uploade udsagnene som kommentarer til de ressourcer, der indgår som subjekt i udsagnene, i en to-delt (mindst) struktur, hvor man først uploader relationen i form af en fane (hvis den ikke eksisterer i forvejen) og bagefter uploader objektet / nr. 2 subjekt til relationen i form af en kommentar til denne fane. Bum, så har vi, hvad der svarer til tripletter! Og nu kan alle være med, for alle kan forstå faner og kommentarer. Og fanerne behøver ingen gang at være formelt formuleret som en relation; relationen som fanen repræsenterer kan sagtens være underforstået (men man vil altid kunne fortolke den som en relation imellem ressourcen og fanens indeholdte kommentarer (og hvor brugerne altså så kan vurdere med deres voteringer/rating-svar, hvor godt udsagnet/"tripletten" passer for hver kommentar)). Nemt! :) Og med dette system får man så et interface til at navigere rundt i sem-web-grafen med gratis, fordi vi jo allerede har et interface bestående af faner (og hvor man så kan blive ved med at folde flere og flere fane-barer ud, jo længere man bevæger sig i dybden af grafen). (Man kan så altid tilføje andre interfaces senere, hvor grafen repræsenteres på en anden måde (men jeg er nu faktisk virkeligt glad for denne mulighed; jeg har aldrig tænkt på et sådant graf-browsing-interace før nu (her på det sidste), og nu er jeg altså virkeligt blevet glad for idéen).) 
%Angående hvad jeg så skal skrive nu: Jeg har overvejet, at lave en tekst under General Notes->Opfølgende-sektionen her i dette dokument, men jeg synes egentligt så godt om, hvordan min kommentar-disposition-proces gik inden denne tænkeperiode, så jeg tror egentligt bare jeg vil gå direkte tilbage til (i kommentarerne) at brainstorme over, hvordan min disposition her skal være. Hvis jeg så lige kommer i tanke om idéer, som jeg ikke ved, om skal med, men hvor jeg finder det det værd lige at nedfælde dem lidt mere detaljeret, så kan jeg jo bare lige enten skrive dem i de kommentarer, eller tilføje en paragraf (under en ny sektion så) i f.eks. General Notes->Opfølgende-sektionen.. Så det er altså planen nu.. Angående om teksten skal være formel eller ikke formel, så tænker jeg stadig, at jeg bare kun skal gøre det akkurat så velstruktureret, som jeg har lyst til. Jeg tænker nemlig at udgive noterne for præcis, hvad de er, nemlig en samling fortløbende noter, som jeg gerne vil skrive mere prængende, hvis og når jeg får tid, men hvor jeg altså tænker, at det ikke skader bare at få dem ud med det samme uanset hvad. Hvis noterne så rammer en stor interesse, så må jeg så bare se på, om det eventuelt giver mere mening bare at gå videre med denne interesse med det samme, eller om jeg skal bruge tid på at udforme bedre versioner af noterne. (Og dette kan jeg bare skrive i selve notesættet (msåke i slutningen om ikke andet..).) 
 





%Husk lige at kig ??-13/11-papirnoter igennem igen.. 






...




%Adding a type to your comments to make them visible under the right.. (fane på engelsk??) *(tabs.)
%Other users can suggest a change, if you upload it to the wrong or to the genrel comment section.
%Makes it easy to find answers you are looking for (e.q. explainations, links, corrections (if any), ...), and it gives users more control to filter away comments, that they are not interested in --- no unnecessary repetitions as well..
%*Ligesom for tags skal folk også gerne bare genbruge faner meget..


%Hm, ville det give mening med subtabs..? ..Så skulle man i høj grad bruge det her med, at folk efterfølgende kan tilknytte en kommentar til tabs.. ..Hm, og/eller *(eller (selvom idéen om over-tabs og sub-tabs dog måske kunne indføres i fremtiden, hvis det kan give mening, det kan selvfølgelig sagtens være)) hvad med, om man bare altid har en hovedfane, hvor alt vises (ud fra en algoritme, der kan bruge ratings, aktivitet, tid, og fane-tilknytning (hvor man bl.a. kan sørge for at tage lidt fra hver fane)), og så skal brugere bare kunne klikke på en kommentar og gå direkte til den pågældende fane (og den fane skal så sættes tidligt i listen --- og muligvis i en særlig "recent"-liste).. Hm, og nu kommer jeg så til at tænke på, hvis jeg alligevel er ret opsat på at foreslå at kommentarer bliver førsteklasses, hvorfor så ikke bare sortere den ud fra tags..? Faneblade kan så repræsentere kategori-tags.. ..Hm, var det ikke bl.a. for at sørge for, at forfatterens mening skar meget igennem..? ..Hm, og man kan ikke bare gøre noget med, at forfatterens(/posterens) mening vægter højt, når det kommer til netop fane-blade? ..Eller at faneblade simpelthen bare får indstillinger, hvor man kan vælge, hvor høj vægt forfatterens vurdering skal gælde? Lyder umiddelbart ret godt.. ..Ej ja, det bliver det, jeg vil foreslå i stedet: Kommentarer sorteres (i faner og i rækkefølge) også ud fra tags (med ratings), og så kan fanebladene bare bruge posterens mening i høj grad (og altså vægte dennes højt). Nice. 





%..Og så kunne jeg nemlig have dette med før use cases..: 
\subsubsection{Filtering and sorting algorithms for comment tabs}
...

%Disp:
%The question now is: which comments should be shown in which tabs? A simple soluton this question would be to have a certain constant threshold for the rating, and comments with a rating above that threshold then automatically belongs to the relevant tab.
%But we can do more than this. I will propose a solution where users upload custom tabs just like they can upload their own tags.. The tab then have.. Hm, men der er faktisk også noget smart ved ofte kun at have én fane pr. kategori-tag.. (For så kan man få et link fra kommentarens tag..) Hm.. ...Hm, og kunne man så ikke bare introducere compound (automatic) tags, hvis det så endeligt er..? ..Hm, men hvad så med tid, forfatter-rating og aktivitet..? ..Hm, dette kunne jo så evt. bare ske hos overordnede sorteringsindstillinger, som så kan gælde for alle tags samtidigt.. (Er dette ok; at det gælder for alle samtidigt?..) ..Hm ja, umiddelbart faktisk ja.. ..Men er det så det, og skal man så foreslå compound (automatic) tags..? 
%... Okay, nu har jeg faktisk tænkt på et ret alternativt system..!.. Hm, jeg bør nok skrive det ovenfor i stedet (selvom det jo også har med dette emne at gøre), så lad mig lige rykke derop..




%Husk så netop at tale om at kunne sortere gengangere fra.



%Hm, eller måske skulle jeg faktisk vente lidt med use cases for en gang skyld her..?..
\subsubsection{Use cases} %Hvis jeg kan finde på nok eksempler (..hvad jeg jo nok kan (tre er fint))..
...


%Example ...





%\subsubsection{Giving comments their own pages}
%...



%\subsubsection{Comments as first-class citizens}
%...


%Så her kan jeg altså f.eks. starte med at skrive om at kommentarer kan være f.c., således at man også kan give disse tag ad libitum, og kan følge en sti til deres egen side. Kommentarernes egen side kan så bruges som udgangspunktet for en deltråd. Forskellen er så bare, at nu kan denne deltråd selv deles op i flere kommentar-faneblade, og dette kan man så blive ved med. Dette kan f.eks.\ bruges til diskussioner, hvor man, lidt som man kender det fra "argumentationsgrafer," altid hurtigt kan finde frem til f.eks., hvad er de mest populære argumenter for, og hvad er de mest populære modargumenter..
%Hm, sidenote: Kunne man egentligt gøre noget med at gruppere ens kommentarer, også \emph{inden for} det samme faneblad..? ..Så man f.eks. kan undgå gentagelser af (populære) argumenter.. ..Hm, og kan man ikke bare skrive argumentations-emnerne som faneblad-titler..? ..Bum.. ..Jo.
%Så det vil jeg altså også nævne, at man kan bruge fanebladene til i disse deltråde. Og så kan jeg slutte af med at forklare om, hvad der er de særlige tags, der vil være brugbare for kommentarer (som jeg kan tænke mig til). Og her kan jeg altså nævne, at 'relevans' gerne må have lidt specialbehandling af siden, så kommentarerne i høj grad er ordnet efter denne pr. default. 



%Kommentarer har så self. bare et link til forælder-resourcen, samt den fane, de blev uploadet til, og foruden dette er relevancy-ratingen så altså også lidt speciel for kommentarer (den er i hvert fald specielt tiltænkt kommentarer..). 



%"..In particular "agreement" and "relvancy," but one could perhaps just extend rating-folksonomies to the comments as well.. *(also warning tags..)"

%"Hvor jeg altså vil nævne, hvis det ikke er nævnt (og ellers bare understrege), at comments også kan rates med alle mulige tags. Og så vil jeg ikke mindst i denne sektion altså komme ind på det her med, hvordan kommentarer kan udgøre eller pege på diskussions-starter-kommentarer, under hvilke folk så kan diskutere diverse del-diskussioner."

%Og til links kan men bare bruge URL'er, ikke? ..Jo.


\subsubsection{Raising warnings, clarifications and links to separate menus}
...

%Hm, og hvordan skal man implementere dette, skal det være noget med at sige, at brugeren kan vælge visse (konstante) fanebladstitler, hvorfra de mest populære kommentarer kan hives op? Ja, hvad med at have nogle "fremhævede (sammensatte) faneblade," eller hvad dælen man nu lige skal kalde dem, hvor brugere altså selv vælger sine fremhævede faneblade i princippet, samt hvilke kommentar-faneblade de skal indeholde. Og i starten kan man så bare bruge en fast algoritme, som forsøger at tage de mest populære kommentarer fra fællesmængden samt også prøver at vise lidt fra hver kommentar-faneblad, hvis der altså er nogen med høj nok relevansscore at tage af. Cool. :) ..Fedt, ligefrem..!



\subsubsection{Advantages and disadvantages}
...




	
%	\subsubsection{Organizing comments by types and subjects}
	
	
	%Hm, ny idé: Hvad med at kunne inddele kommentar-faneblade i undersektioner, og så i princippet kunne blive ved med at gøre dette?..!.. ...Hm, eller hvad med at man kan "hejse" faneblade ud som.. hm, dynamiske ressourcer..? ..eller bare som ressourcer i form af et diskussionsemner..? ..Hm, eller skal jeg sige "never mind, for man kan jo bare hejse individuelle kommentarer ud som ressourcer"..? ... Ja, kommentarer skal bare selv være ressourcer, og på den måde kan man f.eks. lave et kommentar-faneblad over diskussionsemner, hvori kommentarer så skal opsummere diskussionsemner omkring ressourcen --- og/eller linke til mere selvstændige ressourcer, der repræsenterer diskussionsemner (hvis diskussionen berører et bredere emne end bare den originale ressource, og hvis man derfor vil igangsætte diskussionen mere uafhængigt af denne (og altså mere selvstændigt)) --- og når man så navigerer hen til disse kommentar-ressourcers sider, så kan disse så videre have forkellige kommentar-faneblade med undermener til diskussionen (inkl. underdiskussioner). Nice! (01.11.21) Og ja, alt dette bør jeg da helt klart nævne i denne udgivelse. 
	
	
%	\subsubsection{Using multiple ratings for comments as well}
%	...
	%..In particular "agreement" and "relvancy," but one could perhaps just extend rating-folksonomies to the comments as well.. *(also warning tags..)
	
	
%	\subsubsection{Comments as first-class citizens}
%	...
	
	%Hvor jeg altså vil nævne, hvis det ikke er nævnt (og ellers bare understrege), at comments også kan rates med alle mulige tags. Og så vil jeg ikke mindst i denne sektion altså komme ind på det her med, hvordan kommentarer kan udgøre eller pege på diskussions-starter-kommentarer, under hvilke folk så kan diskutere diverse del-diskussioner. 
	
	
%	\subsubsection{Raising warnings, clarifications and links to separate menus}
%	...
	
	
%	\subsubsection{Additional comments on different ways to implement organized comment sections} %(Maybe.)
%	...
	
	\lipsum[1]
	
	
%\subsection[Wish ratings]{Using ratings to express wishes for future resources}
%...




%\subsection{User-driven machine learning}
%...
%
%	\subsubsection{...}
%	...
	%..Hm, jeg må faktisk lige læse lidt op på ML igen, inden jeg skriver denne sektion, og særligt bør jeg læse på eller udregne, hvad man gør, når man har partielle data-vektorer (og altså når man har null-værdier i dataen).. (01.11.21) 
	%(02.11.21) Uden at have læst op på det endnu, så tror jeg dog, at dette faktisk må være ret simpelt at gøre. 
	%Hm, jeg bør vel egentligt have feed-algoritmer direkte efter b.d. ML, så måske jeg skulle rykke lidt om på det?.. 




%Husk at udtrykke, hvorfor min idé nemmere og hurtigere vil kunne få luft under vingerne end gængs FOAF-teknologi. (vist ikke længere så relevant, men skal skal dog huske at forklare om sådanne "bløde algoritmer"..)









%\subsection[User participation in development]{User participation in development of algorithms and user interfaces \label{web 2.1}}
\subsection{Weighing ratings according to publicized data correlation vectors} %..? (titel?)
%..Eller: Partitioning users using public data correlations.. 
%..Eller: Adjusting feeds using publicized data correlation vectors.. 
%..Eller: Using public data correlations to adjust search results and feeds / and other feeds / and more.. 
...

%\subsubsection{Preface}?
%Some of this wil get a little technical..

\subsubsection{Introduction}
...

%A commen part of machine learning is to find correlations in datasets. If we for instance look at a system with tag ratings, we could have two tag, bla bla and bla bla bla, which might be correlated; If one thinks bla bla then one might also be likely to think bla bla bla. Finding such correlations can be used for instance to suggest relevant ressources to the user. In particular it can be used to make a feed more relevant to the user..
%This is typically done behind closed doors (strictly server-side).. And there can be good reasons for having at least part of it closed to the public; users might not want their data to be public.. 
%I will, however, suggest a compromise here where users can ask the site to publish a large set of relevant correlation vectors (wich is how correlations are represented; as vectors) were care is just given as to not publish any correlations that is based on too few datapoints (and thus on too few users).. 
%The idea is that users can select a relevant domain of resources and tags (if we keep looking at such systems) and then request correlations from the site. Other users should then be able to sign on to requests.. ...It should be possible to define domains by tag predicates.. 
	%Ikke alligevel: The users can then investigate these correlation vectors and see come up with explainations for the.. Hm.. ... Nej, jeg skal måske faktisk ikke snakke om omparametrisering her.. Måske bliver det for kompliceret..? ..Og det giver nemlig sig selv, når brugerne kan lave linearkombinationer i sidste ende i deres mellemregning-vægtninger. ..Nej, lad mig bare lade være med at nævne det.. 
%Examples: This could be specific domains such as movies but if we expect correlations to extend to other areas (series, vedeos, books..), it is probably a good idea to try a union set of these things. It could also be political veiws and/or social views, if the site includes comments and discussion on these areas (such as e.g. Twitter). The correlation vektors will then show.. The can be represented (use another term than 'represented') by a histrogram of different resource and tag combinations.. 
%And what can these correlation vectors be used for except for users to learn more about their own tendensies and how they compare to other users?. .. Hm, men jeg vil jo nok også gerne fremhæve netop denne fordel lidt.. Okay så: What can the be used for?. First of all they can be used for --"--. 
	%Ikke alligevel: I think that this might actually even be a good motivation for users to be more active in rating things, since now they get something out of it in return (exept the experience of giving their opinion).
%In the appendices I also have a small note about how users might be able to refine the correlation vectors even further..
%But the big point I want to make about these correlation vectors is that they can be used in feed algorithms --- just like how they are normally used, except that now every other can try every other setting.. Hm.. Kunne man også påpege: It also means that users can create anonymous account but still get the same feed settings.? .. ..Det går jo måske godt nok lidt imod pointen omkring at folk vil være mere motiverede til at give vurderinger, vil det ikke..? ...Hm, jo måske.. ..Hm, hvad i øvrigt med, om man brugte (modererede) brugergrupper til dette i stedet..? ..Så bliver anonymitet også endnu mere simpelt, fordi brugere så selv melder sig til grupperne.. Hm..(!..?) ..Det vil jeg lige tænke lidt over...
%Okay, jeg skal nok egentligt bare undlade at argumentere for, at brugerne får større motivation til at give vurderinger med mine idéer.. ..Og jeg gider heller ikke nævne den pointe, jeg skrev lige her ovenfor..
%
%Så hvis vi fortsætter fra: Hvad er pointen med disse korrelationsvektorer?: Jamen det er, at så kan brugerne vælge at se ratingerne igennem flere forskellige briller ud fra, hvilke nogle brugertyper, der har vurderet det. Og alle vurderinger skal bare være anonyme, så man kommer ikke til at kunne reversere dataen og finde frem til ting, der bør være anonyme. Ok.. Og så har jeg jo i hvert fald to (rimeligt generelle) eksempler, hvor man kunne bruge dette. Og det bliver nok afslutningen på denne undersektion..
%..Uh, eller kunne man ikke næsten bare nævne feed-sortering hurtigt her i slutningen af dette , som det er nu..?(..!..?) Lad mig lige se på det.. ..Ah! Kunne man ikke bare nævne i rating-tag-sektionen, at man jo ikke bare kan bruge tag ratings til at lede efter noget specifkt (og til at browse forskellige muligheder, men stadig relativt målrettet), men at man også kan bruge det til feeds?.! Og så kan jeg også bare lige sørge for at nævne, at til begge ting vil det ikke være en dum idé, hvis brugeren også kan stille på kurvebredden (på en klokkekurve) over søgningen/feedet! Og så er alt det allerede ude af verden, og med ret få ord! Nice. Og ja, så kan jeg bare slutte af her med altså at nævne, at dette så kan indgå i søgninger (inkl. product page-søgninger) og i feeds. ..:) 



%Hm giver det mening at kalde det u.d. ML?.. Bliver brugernes deltagelse nu ikke ret simpel..?.. 
%Tja, det gør det vel nok ikke alligevel.. Hvad skal jeg så kalde det? ... ..Public data correlations..? ..Det kan jeg lige kalde afsnittet for nu..






%Nævn også anekdote om YT; at kontinuere rating nok er bedre til at forudsige præferencer ud fra ift. binære ratings..



\subsubsection{Use cases}
...









%\subsubsection{User-driven development of feed algorithms}
%...

%..Er det så her, jeg skal introducere vægtede brugergrupper (og også lige nævne "bløde algoritmer" til dette formål)..?
%Ja. Og nævn også lige noget med at inddele efter f.eks. alder og bosted osv. ..Hm, men måske det kommer til at fortjene sin egen sektion, det tror jeg nok..


%\subsubsection{User groups and user divisions}
\subsubsection{Additional ways of dividing users into groups}
...



%Jeg skal vist også tage mine (FOAF-agtige) brugergrupper mere seriøst igen.. Og (ikke-)tillids erklæringerne kan godt sagtens lade sig gøre, selvom vurderingerne holdes anonyme; websiden administrerer jo bare tillidsvægtene og kan bruge dem i mellemregningerne.. ... Ah, og brugere skal måske så kunne skilte med deres samlede tillid ift. en brugergruppe på deres profil, men viser så ikke det eksakte tal; kun hvilket niveau man er inden for.. (Så folk ikke kan se den eksakte vægt.) Hm, og hvad så, når overgange sker? Kunne man mon bare have et vist overlap, og så gøre det tilfældigt, hvornår det skifter..? Eller skal brugeren bare selv vælge, hvornår.. Ah, eller man kunne også bare kræve, at brugeren skal have været på niveauet i.. Hm.. Ah nej, man kan bare gøre det sådan, at man kun skifter niveau til en fast tid én gang om ugen. Ja, det må være fint.  

%(13.11.21) Jeg skal også tage offentlige profiler mere seriøst igen.. (Og man kan nemlig godt stadig bevare anonymitet og muligheden for at slette ting fra offentlige profiler osv.)



\subsubsection{Advantages and disadvantages}
...

\lipsum[1]










%%\subsection{Ideas for wiki sites and for collaborative writing and editing in general \label{wiki_ideas}}
%\subsection{Ideas regarding collaborative writing and editing}
%...
%
%\subsubsection{A broader variety of article types for wikis} %(using tags and with ratings)
%...
%
%\subsubsection{A graph of knowledge levels for various subjects connected with course articles to go from one level to another} 
%...
%
%\subsubsection{Modular editing with formal dispositions and ratings for section modules}
%...
%
%
%
%\subsubsection{The possibility of applying the same methods for collaborative programming}
%...

%\subsection{Ideas for wiki sites and for collaborative writing and editing in general \label{wiki_ideas}}


\subsection{Using wish ratings for modules in collaborative writing/editing} %..
...

\subsubsection{Introduction}
...

%The concept of the wish rating dispositions..
%What having tags for articles means (i.e. that we can get a more varied type of wiki..).. 

\subsubsection{Use cases}
...


%\subsubsection{A broader variety of article types for wikis} %(using tags and with ratings)
%...

\subsubsection{A graph of knowledge levels for various subjects connected with course articles to go from one level to another} 
...

%\subsubsection{Modular editing with formal dispositions and ratings for section modules}
%...



\subsubsection{The possibility of applying the same methods for collaborative programming}
...



\lipsum[1]







%Skal jeg stoppe med at kalde det 'platform' her, eller beholde begrebet/navnet..?
\subsection{A debate website with symbolic bets} %(Using the concepts of view point groups and knowlegde level graph from previous section.)
...

\subsubsection{Introduction}
...

%At bruge brugergrupperne (eller lave nye).. Til at diskutere ting i formelle diskussioner, hvilket er diskutioner, der følger en bestemt ontologi, ligesom.. 
%Jeg vil foreslå at man så tager runder..
%Man skal så vælge en tredjepartsgruppe som dommer (eller man kan vælge flere..), hvis opgave bl.a. er at få parterne til at holde sig til et sæt retningslinjer.. Og ellers er den store opgave altså også, at dømme de væddemål, der opstår (og gerne skal opstå) undervejs.. 




\subsubsection{Use cases}
...


\subsubsection{Continuously summarizing the ongoing arguments in the depth of the graph in an upper branch...}
...

\subsubsection{The reports..} %Hm, men så kan jeg dog ikke nævne p-modeller her, hvis jeg nu gemmer web x.x-sektionen til senere.. 
...


\subsubsection{Ideas for useful web 3.0 ontologies...} %Bare så jeg lige husker emnet..
...



















%\subsection{Possible ways to implement the above ideas} 
%\subsection{Follow-up remarks on the above ideas} 
\subsection{Some remarks on the future of the web} 
...

\subsubsection{A broad and open source implementation} %Måske skal dette under næste undersektion..?
...


\subsubsection{The technology as a possible front runner for the semantic web}
...


%Uh, det var altså en ret god idé, jeg fik (i går, var det vidst), med at pointere, at man med fordel kunne bruge web 2.0-konceptet mere til fremførsel af det semantiske web..! 

%Mention predictive semantic web! (or web 3.1 or what to call it (perhaps neither of those)..)



\subsubsection{Extending web 2.0 to include user interfaces and applications}
...

%Måske kunne jeg her nævne fordelene ved derfor at have implementeret første idéer med et formelt triplet-agtigt system, så hele systemet herved er mere åbent over for, at brugere kan lave algoritmer..? (..Eller også kunne man nævne det her i en efterfølgende (ny) undersektion..)

%Jeg tror bare jeg vil nævne "... to include more backend functions and perhaps functions to install applications client-side" i denne undersektion i stedet (for at gøre det på to undersektioner). 


%	\subsubsection{The possibility of extending the public API to include more backend functions and perhaps functions to install applications client-side} %Kan overveje, hvad en kortere titel kunne være..
%	...




\subsubsection{Predictive ontologies...} %..?
...


\lipsum[1]











%\subsection{Ideas for useful web 3.0 ontologies \label{useful_ontologies}} %..Måske.. *Jo! (Især handlings-/udfordrings-ontologi.) Men måske skal det bare op under debat-platform. Og så kan det nemlig introduceres som en mere kreativ version af formelle diskussioner.. *..Ja.
%...













%\section[QED]{Short summery of my theory of quantum electrodynamics (QED)}
%...


%Jeg kunne måske putte nogle detaljer omkring 'hvorfor' og 'hvordan' i et appendix..










%\section[Existence]{Some thoughts on existence and the multiverse}
%...
%Jeg streger denne sektion. :) Se under 'Andre opfølgende noter' for en opsummering af mine seneste tanker. 



%Symmetry of the multiverse..
%Consciousness laws.. (Why not have them at the center?. ..)
%Time --- subjective and global.. (?..)
%Infinite hospital problems..
%Fundamental laws of the multiverse. (Nice to find some meaningful ones, and they also matter to the prior..) *(mention decision graphs instead of sentences of a language(!)..) (..mention why 'multiverse' works here but is not completely conventional..)
%100 \% chaos with a simple example of a hypothesis (with lexical ordering).. 
%Quickly why I think these problems disappear.. 
%Short analysis of some hypotheses.. *(..Or maybe not really..)
%
%"Life after death?"..
%Is existence good?. ..






%Husk: subjektiv tid fjerner tids-"paradokset" (eller "det undelige omkring tid" kunne vi sige..)
%Husk at diskutere omvendt Bayes.. Her bør jeg også lige overveje nærmere, om ikke man kan argumentere for min yndlingshypotese frem for andre på baggrund af dette.
%Måske bør jeg sørge for at opridse, hvad "målene" er for disse tanker. 
%Nævn Hilberts hospital -metafor, når jeg skal forklare omvendt Bayes.. 
%..Prior ud fra multiversets grundsætningers konstruktion..
%Subjekttiv tid kan vel nævnes, når jeg kommer ind på at prøve at køre mine hypoteser igennem analysen..(?)..




%(04.11.21) Okay, i går og i dag har jeg gået og overvejet eksistens lidt mere. Jeg har fået nogle nye tanker, og der var på et tidspunkt, hvor jeg også troede, at jeg ville skifte mening om nogle ting, men på en måde er jeg endt lidt det samme sted.. Jeg synes stadig, at det er en ret meningsfuld hypotese, den om, at hver indivuduelle realitet beskriver en oplevelse. Hvis nu en realitet på en måde kan bekrive et univers, der bare indeholder bevidstheder \emph{inde i} sig (i stedet for at bevidsthederne udgør universet), så bliver det hele lidt mere ussikert, som jeg ser det, når vi kommer til at skulle tale om prior-sandsynligheds-analyse. En realitet bestående af bevidsthed, men ikke bare af én, nej, mange adskildte bevidstheder, kan godt give mening for mig til en vis grad, og dette vil, som jeg ser det, ikke føre til samme prior-sandsynlighedsproblemer, hvis så bare antallet af bevidstheder ligesom ligger tidligere i.. spørgsmålet/realitetssætningerne.. åh, det er sgu svært at forklare. Men det korte af det lange er, at prior-sandsynligheder kan konvergere forskelligt (eller divergere) alt efter, hvordan man summer dem sammen, og hvordan man så mest naturligt må skulle summe dem sammen, afhænger måske så (som jeg ser det) lidt af, hvordan grundsætningerne til realiteterne på en måde selv er opbygget.. ..Altså hvilke nogle spørgsmål, man må besvare først, når man "kaster med terningerne" og "trækker en tilfældig realitet" (hvad man jo netop ikke gør; hvis der fandtes sådan en underliggende tilfældighedsproces, så var prior-sandsynligheder lige pludselig en nem ting at tænke på og analysere..).. Åh (det er sgu lidt svært at beskrive), men anyway, hvis vi holder os til en samlet eksistens, hvor individuelle realiteter alle beskriver en oplevelser, og at alle realiteter demred tæller lige meget i prior-sandsynligheds-summationen, så bliver det hele mere behageligt. At finde et gennemsnit over en uendelig mængde er stadig ikke en \emph{logisk} ting at regne på, men i det mindste vel der alligevel komme et ret naturligt svar på, hvad prior-sandsynligheden må blive (altså ikke at vi kan regne den ud selv, men man ville kunne det i princippet, og desuden er den rimeligt nem at lave overslag på --- i hvert fald hvis vi ser på oplevelsesformer, der minder om vores egne, og altså ser bort fra muligheden for "eksotiske bevidshedsformer"..).. Nå, men jeg tror altså stadig bare, jeg vil fremføre min yndlingshypotese her som et eksempel, og så bare snakke om, hvorhenne man kan støde på problemer, og altså hvordan nogle hypoteser kan føre til meget uklare svar på prior-sandsynlighederne og/eller føre til 100 \% kaos.. Og selvom det er vildt kompliceret, så tror jeg faktisk godt, jeg kan komme med et legetøjseksempel, der viser problemet --- og så kan jeg jo nemlig bare sige, at jeg har haft problemer, når jeg har kigget på nogle alternative hypoteser.. Og så kan jeg bare skippe videre derfra, og så snakke om "liv efter død?"-spørgsmålet, og så kan jeg runde det af sådan.. 
%...Hm, men jeg kom jo faktisk lige frem til noget interessant.. Jeg kom lidt frem til, at der måske giver mere mening at tænke på en ja/nej-graf frem for en liste af sætninger (f.eks. i et sprog a la formel logik).. Ja, og så er det da lige før, at alle hypoteserne så igen vil give det samme..(!!..) ..interessant!..(!!).. (04.11.21)  
%(05.11.21) Ah, nu har jeg (i går aftes) faktisk forstærket min tro igen i hypotesen om, at eksistens allerede handler om oplevelser til at starte med..! Pointen er, at hvordan \emph{kan} man beskrive bevidsthed ud fra logik? Det kan man jo ikke gøre med f.eks.\ FOL, og jeg tror altså heller ikke, at SOL er stærk nok.. (Altså spørgsmålet er, om man med helt vildt komplicerede SOL-sætninger kan formulere begrebet om bevidsthed..) Hm, nej. Man kan ikke beskrive bevidsthed ud fra.. Ja, ud fra noget sprog, hvor man ikke allerede har begreb om konceptet til at beynde med, vel..? For man vil jo altid kunne "spille dum" over for en forklaring og kræve noget målbart i beskrivelsen.. ..Hm, tjo, det er ikke nødvendigvis et holdbart argument, for en korrekt beskrivelse af bevidsthed behøver ikke at resultere i muligheder for at måle det, men jeg tror nu alligevel selv på, at man ikke kan beskrive bevidsthed, uden at modtageren allerede forstår det. Vi kan så ikke a priori sige, at den fundamentale logik / det fundamentale sprog om alt ikke "er stærkt nok" til at kunne beskrive hele realiteter fra grunden op, hvori konceptet om bevidsthed så også beskrives.. Men for mig virker det altså mere "fornuftigt"/"meningsfuldt"/.. hvis spørsmålet om eksistens bare handler om følelser/oplevelser til at begynde med. Og det dejlige ved denne hypotese er nemlig også, at den ikke fører til nær de samme prior-sandsynligheds-paradokser, som hvis man bare sagde: "alle mulige realiteter findes, hvor hver realitet selv kan beskrive, hvordan bevidsthed \emph{opstår} i dem".. ...Ah..! Nu kom jeg lige på noget andet..(!..) ..Hm nej, måske ikke.. ...Jo!.. Eller dvs., nu har jeg noget lidt andet: En realitet hvor bevidstheder kan opstå og forsvinde løbende må vel nok kræve en masse ekstra forarbejde, for hvad dælen vil det sige at være en bevidsthed, der ikke er født endnu, eller at være en bevidsthed, der har sluttet?. Så selvom sådanne relaliteter må kunne klare sig med nogle ret simple fysiske love og så bruge resten af tiden i lov-sætningerne på at beskrive et stort nok tal, så man derved bare kan forcere, at der bliver et enormt antal lommer i realiteten/universet, hvor ting tilfældigvis opfører sig som om, de fulgte mere komplicerede fysiske love, (Selvom at de kan det,) så vil disse realiteter nok skulle bruge "tid" på at beskrive, hvad det overhovedet vil sige, når bevidstheder starter og slutter.. Hm, og jeg kom også lige til at tænke på: måske skulle jeg også lige prøve at se på, om man ikke kunne lave et stærkere argument for brugen af "besjælingsprincippet" også.. (Altså at det er nemmere at beskrive en bevidst oplevelse ud, hvis man tager udgangspunkt i de "kræfter" som de fysiske love beskriver, i stedet for at man bare beskriver oplevelser ud fra hjerne-konfigurationer og(/eller) -bevægelser.) Hm, nej det kan jeg nok ikke argumentere nærmere for, men det gør vist heller ikke noget, for jeg føler faktisk, at argumentet om at skulle "bruge tid" på at beskrive opståen og afslutning af bevidstheder holder ret godt, nu hvor jeg tænker lidt mere over det.. ..For godt nok kan man sige, at sådant et spørgsmål muligvis "ligger længere ude" i realitetsbeskrivelsen, hvilket så muligvis kan føre til, at man bør.. hm.. Nu kan jeg dårligt lige huske argumentet, men modargumentet er altså uanset hvad, at tal-beskrivelser også godt kan "ligge længere ude".. Hm ej, jeg bør lige tænke over disse argument/modargumenter lidt mere.. ... Hm, men pointen bliver nok den samme.. Hvis man på en eller anden måde kunne dele det op i ydre og indre realitetssandheder (hvor man så siger at alle skal gælde), hvis så på en eller anden måde prior-udregningen kan afhænge af, hvilken ydre sandhed man befinder sig i, jamen så er det nemt, for så må vi bare befinde os i en, hvor der ikke er 100 \% kaos (med al sandsynlighed).. (Hvis dette overhovedet giver mening (hvad det jo nok ikke gør), men nu har vi bare antaget, at det gør (og så konkluderet, at lt er fint i så fald)). Hvis ikke man kan dette, så skal vi jo summe op over ydre realiteter på en eller anden måde, når vi skal vurdere prior-sandsynligheder (på et overordnet plan). Og pointen er så her, at ydre realiteter, hvor store tal indgår som noget fundamentalt så må tælle så meget desto mere.. Hm, og hvis nu man kan have uendeligt mange bevidste oplevelser i en realitet..? ..Så må disse betyde alt, men dette billede kommer så faktisk til at svare til min yndlingshypotese, gør det ikke, for så kommer man med det samme bare til et spørgmål om at finde love for denne uendelige mængde oplevelser, hvilket svarer til at finde.. Ja, det må give ca. det samme. Og hvis ikke der findes sådanne universer med uendelige mange bevidste oplevelser på én gang, så er vi altså ude for, at de realiteter, hvor store tal er en fundamental byggesten, de vil få overskyggende betydning, når man summerer op.. ..Ja, for selv hvis man skal dele summationen op i en ydre og en indre del, og selv hvis den indre del kan være uendeligt mange ting, så må man på en eller anden måde summe dem op samtidigt; man kan ikke bare.. jo vent, man kunne måske argumentere for, at man finder gennemsnittet over alle indre dele for hver ydre del, og så lader disse gennemsnitte tælle lige meget, når man summerer den ydre del op, men ja, så når vi også til en pæn, ikke-100-\%-kaos prior-sandsynlighed. Og ellers skal man altså summere dem op samtidigt på en måde, hvilket så vil svare til en summation, hvor der kun er én del. Så jeg kan altså lægge tanken om, at der kan være en ydre del og en indre del, når en realtitet "dannes," og så.. Hm.. Hov nej, problemet kan netop være (hvis vi vil bruge argumentet om, at det vil kræve mere "tid" at beskrive en realitet, hvor bevidstheder kan dannes og afsluttes automatisk, som i stedet kunne bruges på at konstruere et stort tal, som realiteten kan opnå et større antal oplevelser), hvis en sådan beskrivelser sker "yderligere" i realitetens lovsætninger, og hvis dette så på en eller anden måde medfører, at dette ikke dermed på en måde medfører, at denne realitetstype kommer til at tælle med med mindre frekvens i summationen.. ..Ah, men igen: Den eneste måde dette kan forekomme på, er hvis man så i praksis skal tage regne ved at finde et gennemsnit for hver "ydre realitet," og herved opnår man så stadig i sidste ende et resultat uden 100 \% kaos (men netop med stor frekvens af (tilsyneladnede) "lav-informations universer" som vores). Og hvis man ikke skal gøre dette, så må man finde en orden at summere alle de individuelle, "fulde" ("indre" plus "ydre") realiteter op, og så er det, at det vil være naturligt, at.. Hov, vent.. Men hvis en "ydre" realitet netop kun indeholder endeligt mange bevidste oplevelser, hvad så..?.. Ah, men så vil "ydre realiteter," der "bruger tid" på at beskrive store tal frem for komplicerede fler-oplevelses-realitetsgrundlag (hvor f.eks.\ bevidsthe oplevelser kan opstå og afsluttes løbende på en univers-global tid) vinde kapløbet. Jep. Og i sidste ende: Hvis realiteter ikke kan deles op i en ydre og en indre del (i.e.\ et realitetsgrundlag og en faktisk udgave af en realitet, der følger dette grundlag) --- og at vi altså antager at vi skal summere op over realiteterne ved at ordne dem efter deresl lovsætninger, som var de formuleret i et sprog og ordnet derefter, hvilket nemlig er den antagelse, der er kilden til at man kan nå 100 \% kaos-konklusioner, som jeg ser det (jeg tror nemlig ikke, at en ordning, hvor man i stedet følger en decision-graf over, hvilke sandheder skal gælde, fører til samme problem..) --- jamen så er billedet det samme idet beskrivelser af komplicerede fler-oplevelses-forhold igen vil "tage tid væk," som man kunne bruge på at forøge sit store tal i lovsætningen. 
%Så med dette nye skyts i baghånden, så tror jeg altså ikke, at man vil få problemer med 100 \% kaos (og vil blive nødt til at hive fat i f.eks. "besjælingsprincippet" for at løse det), selv ikke hvis man altså antager, at prior-udregningerne skal gøres ved at man ordner realitets-lovsætningerne ud fra, hvordan de formuleres i et (eller andet grundlæggende) sprog, når man skal summe over dem. Og med en alternativ ordning, hvor man i stedet stiller en række spørgsmål a la for en automaton (og altså hvad jeg har refereret til som en "decision graph," hvad der sikkert også er rigtigt nok), hvilket jeg på en måde faktisk selv synes er mere meningsfuldt at gøre, jamen så tror jeg altså heller ikke, man kommer i problemer (for så skal "tiden," det tager at formulere et stort tal, og "tiden," det tager at formulere rige univers-love, ikke ligesom "kæmpe" med hinanden her; de to længder vil altså ikke blive (omvendt) korreleret med hinanden pga. den orden, man summer efter). Så vi undgår altså tilsyneladende 100 \% kaos uanset hvad.. ..!! (05.11.21)
%Okay, og hvor meget i al verden af dette, skal jeg så prøve at formulere i teksten..?? Nå ja, jeg kan jo bare forklare problemet med at ordne summationen leksikalt ud fra et sprog, og så forklare, at jeg selv føler, at jeg har gode argumenter for, hvorfor selv dette alligevel ikke vil blive et problem i sidste ende (og altså ikke vil medføre 100 \% kaos). Ok.. :) 






\section{(old)Appendices}


\subsubsection[Semantics of the scale]{Defining the semantics of the rating scale}
Note that with a system implementing tag ratings, there are now several ways to interpret some tags. If we for instance take the same example as before with a tag representing the predicate `scary,' we can interpret the rating scale either as signifying how sure the users are that the agree with the predicate or we can interpret to signify the degree to which the predicate applies, i.e.\ how scary the users think the movie is. In the first case, we interpret `scary' to represent a constant predicate and the users then rate how confident they are of the applicability of that predicate whereas we in the second case treat the predicate as variable one, whose meaning depends on the values on the rating scale. In practice the difference of these interpretations is that users with the first interpretation might evaluate a tag such as `scary' as near 100 \% (if 100 \% is the maximum) if they are sure that the movie is scary, even though they do not consider it one of the scariest movies. %Det kan jeg godt omformulere..
But with the second interpretation, the highest scores are only given to the most scary movies. 

Using the second kind of interpretation is what will benefit the user community the most, as more information can be extracted from a rating in this case. I think people will tend to use this interpretation automatically, even with no explicit agreement on the semantics, but still it might be a good idea to make such a convention explicit. The site that adds ratings to its folksonomy system should thus write this convention in its guidelines. %Omformuler måske..
It would then also be beneficial to add some specific guidelines for how different values of a rating scale should correspond with a degree of the tag's predicate. Such guidelines could try to give examples for as many different kinds of tag predicates as possible, aiming to cover as representative a set of predicates as possible, and try to describe for all cases how different values on the rating scale should be interpreted. 

%If finding very clear-cut guidelines and agreeing upon them turns out to be a hard task, it might also be a good enough solution to simply give some overall guidelines at first and then to just monitor the tendencies of the users over time in order to adjust the guidelines to fit the conventions that have been formed automatically. %Omformuler måske..
%
%For some tags the meaning can not be made clear from just the title. For a tag such as `scary,' people can deduce a lot from just the title but there are also more abstract predicates where this is not the case. We could for instance have predicates such as `slow burn' or `slick' (which are both tags that can be found on Netflix), where many users would benefit from at least an overall explanation of what those tags mean. I will therefore suggest that each tag has a link to a documentation, which the creators of the tag author. 
%
%%Hm.. Det er da lige før, at det bare er smartere i virkeligheden at glemme dokumentationer og så i stedet bare i høj grad lade semantikken indfinde sig automatisk.. 
%%... Ah, men:
%Whether...

%But even if you make perfect guidelines with precise definitions, you cannot be sure that the users will all read and follow them. ...






















\newpage
\part{Some QED notes}

\chapter{Some QED Notes}

{\slshape
	Indtil videre står mine interessante QED-noter ikke her, men kan i høj grad findes under ``Mere om min QED-teori''-undersektionen ovenfor (under ``Andre opfølgende noter''-sektionen).
}



\section{Lidt fodarbejde omkring feltintegralerne}
(28.11.21) Jeg har på fornemmelsen, at jeg har lavet en fejl mht.\ de gaussiske integraler, der gør at udtrykket bliver afhængigt af $M$ (altså den tidslige lattice spacing). Lad mig lige prøve at gennemgå udregningerne (hvor jeg bare bliver i det diskrete hele vejen).




..Så hvis vi har
\begin{align}
\begin{aligned}
\braket{\boldsymbol q_2 |e^{-i \hat H \delta t}|\boldsymbol q_1}
&= \frac{1}{(2 \pi)^{N}} \int d^N \boldsymbol p_1 
\, e^{i \boldsymbol p_1 \cdot \dot{\boldsymbol q}_1 \delta t - i H(\boldsymbol p_1, \boldsymbol q_1) \delta t} + \int d^N \boldsymbol p_1 \, O(\delta t^2),
\label{path2}
\end{aligned} 
\end{align}
%	
og
%
\begin{equation}
\begin{aligned}
\int d^N \boldsymbol x \, e^{-\frac{i}{2} \boldsymbol x^T A \boldsymbol x + i \boldsymbol b^T \boldsymbol x}
= e^{-\frac{i N}{4}} \bigg(\frac{(2 \pi)^N}{\det A}\bigg)^\frac{1}{2} e^{\frac{i}{2} \boldsymbol b^T A^{-1} \boldsymbol b}.
\label{Gaussian}
\end{aligned} 
\end{equation}
%
og
%
\begin{align}
\begin{aligned}
\int d^N \boldsymbol p_1 \, e^{i \boldsymbol p_1 \cdot \dot{\boldsymbol q}_1 \delta t - i H(\boldsymbol p_1, \boldsymbol q_1) \delta t}
= e^{-\frac{i N}{4}} \Big(\frac{2 \pi m}{\delta t}\Big)^\frac{N}{2} e^{\frac{i}{2} m (\dot{\boldsymbol q}_1 - \boldsymbol C(\boldsymbol q_1))^2 \delta t - i V(\boldsymbol q_1) \delta t},
\label{Gaussian2}
\end{aligned} 
\end{align}
%
og så ser på

\begin{align}
\begin{aligned}
\braket{\boldsymbol q'' |e^{-i H t}|\boldsymbol q'}
&= \int \prod_{i=1}^{M} (d^N \boldsymbol q_i) \, \bra{\boldsymbol q''} e^{-i H \delta t} \ket{\boldsymbol q_M}\bra{\boldsymbol q_M}  e^{-i H \delta t} \ket{\boldsymbol q_{M-1}}\dots \bra{\boldsymbol q_1} e^{-i H \delta t} \ket{\boldsymbol q'}, %\\
%&= \Big(\frac{1}{2 \pi}\Big)^{N (M + 1)}  \int \prod_{i=1}^{M} (d^N \boldsymbol q_i)\,
%e^{i \sum_{i=0}^{M}  \big(\boldsymbol p_i \cdot \dot{\boldsymbol q}_i -H(\boldsymbol p_i, \boldsymbol q_i)\big) \delta t} \big(1 + O(\delta t^2)\big)^{M+1}.
\end{aligned} 
\end{align}
%
hvad får vi så\ldots? 

\ldots Ah, jeg skal for det første lige se på, hvilket integrale measure (i.e.\ hvilken metrik), der får et generelt integrale såsom $\int \mathcal{D} \boldsymbol q \,
e^{i \int_{t'}^{t''} d t \, L(\dot{\boldsymbol q}, \boldsymbol q)}$ til at blive uafhængigt af $M$. 
\ldots Hm, det er da lige før, at man bare skulle Fourier-transformere, så man bedre kan argumentere\ldots\ \ldots Tja nej, man kan nu godt se sig ud af det uden\ldots\ \ldots Hm, eller kan man\ldots? 

\ldots Åh, jeg har det som om, at eq.\ (\ref{Gaussian2}) ikke rigtigt kan passe\ldots\ Eller der mangler i hvert fald så et ekstra integrale til sidst til lige at håndtere matrix elementet (og det at $\dot{\boldsymbol q}^2$ får frekvensen til at stige, når man integrerer). Hm, skulle man så virkelig prøve med approksimerede $\ket{\boldsymbol q}$ og $\ket{\boldsymbol p}$ i stedet, og så se på deres matrixelementer?\ldots\ (Eller altså rettere på $\ket{\boldsymbol q_1}$ og $\ket{\boldsymbol q_2}$'s?) \ldots Hm, og det vil så lige være en gaussisk integration oveni, og så må jeg hellere bare lige sørge for at normalisere det hele manuelt (og undervejs) for en sikkerheds skyld\ldots\ %..Det er nu ikke sikkert, at jeg orker og/eller når så meget mere i dag (og der kommer også CS nu her), men nu må jeg jo se.. Det kan godt være, at jeg får klaret noget, hvis ikke jeg bliver for opslugt af kampene.. ..Hm, eller også får jeg bare set på mine web-idé-noter i stedet.. %He, jeg fik alligevel lavet/opnået en del i går. ^^

\ldots Ah, jeg tror faktisk, at det er det helt fornuftige at gøre, og fordi $\dot{\boldsymbol q}^2 \delta t$ må gå som $\delta t^{-1}$, så tror jeg, det kommer til at blive et fornuftigt resultat\ldots\ 


(29.11.21) Ah, never mind denne normaliseringsfaktor; den skal nok blive korrekt, og den betyder nemlig ikke noget alligevel (som jeg umiddelbart kan se).




\section{Coulomb-energien i min løsning / min teori}
(29.11.21) Hm, bliver der ikke en afhængighed af $L$ (eller $L^3$, om man vil), eller er det bare mig, der tænker noget forkert\ldots? \ldots Eller er der mon en form for amplitude-afhængighed af $L$\ldots? 
Nå ja, der er der jo netop. \ldots\ Ah, $\hat \Pi_{A_\parallel}$ får nok en afhængighed af $N$ ved at være proportionelt med $N^{-1/2}$, men det er ok ift.\ $V$-elektron-energien, for her vil bidragene så også være proportionel med $N^{-1/2}$. Og når det så kommer til $\hat \Pi_{A_\parallel}^2$'s dobbelte produkter, så disse gå som $N^{-1}$, hvilket det netop bør, hvis summen ikke skal afhænge af $L$. ..Uh, uh, uh! kunne der så ikke på en måde komme en faktor af $(2\pi)^{-3}$ ved at gå fra $\boldsymbol x$- til $\boldsymbol k$-rummet!?\ldots\ \ldots Hm, man skal vel faktisk på en måde dividere med deres spacing i $k$-rummet.. Åh, det tegner godt, det her.. Okay, og deres spacing er?.. Ja.. Jeg behøver ingen gang at slå det op, gør jeg? Hm, jeg kan også bare lige hurtigt udregne det, for en sikkerheds skyld: Vi har $k=2\pi / \lambda$, så $k$ må gå fra omkring $k=2\pi / L$ til omkring $k=2\pi / a$, hvor $a$ altså er spacingen i $\boldsymbol x$-rummet. Spacingen er således omkring $2\pi / L$. Jeg skal dividere med denne spacing i tredje, og ja, så får jeg netop min faktor af $(2\pi)^{-3}$.! Ja, og faktoren af $L^{+3}$ går jo ud med $N^{-1}$ for integralet, som det skal.. Okay, det betyder, hvis jeg har regnet rigtigt ellers, at jeg nu bare skal se, om diverse faktorer af potenser (inklusiv kvadratrødder) af 2 nu også går op og bliver det rigtige! :D
%Swee-eet!

Hm, for det første, så \emph{skal} vi ramme en energi på $e^2 / 4 \pi r$ for to elektroner, og ikke $e^2 / 2 \pi r$, som man måske ellers godt hurtigt kunne tænke. (Og vi kan bare ignorere $e^2$ for nu, for jeg er rimeligt sikker på, at dette nok skal komme med som så.) Vi bør stadig have et 2-tal fra ``det dobbelte produkt.'' Hm, mon ikke dette så bare vil gå ud med 2-tallet, der (nok) må komme fra $1/(\sqrt{2 N})^2$? Og tilbage får jeg så alt andet end lige: $4\pi/r \times \pi/2 \times 1/(2\pi)^3 = 1/4\pi r$\ldots\ hvilket jo umiddelbart ser rigtigt lovende ud\ldots!! 





%\section{Lidt spændende udregninger (11.12.21)} %Og dette betyder jo bare "startet d. 11/12," bare lige for en påmindelses skyld. (Det gør det også for i starten af paragrafer i princippet, men på det seneste har jeg nu vist nærmest altid skiftet paragraf, hvis jeg har fortsat en ny dag..)
%
%Partikler:
%\begin{align}
%\begin{aligned}
%	\begin{pmatrix}
%	\sqrt{E + p_z}\\
%	\sqrt{p_x + i p_y}\\
%	\sqrt{E - p_z}\\
%	\sqrt{- p_x - i p_y}
%	\end{pmatrix}\frac{1}{2\sqrt{E + |p_x + i p_y|}},
%\quad
%	\begin{pmatrix}
%	\sqrt{p_x - i p_y}\\
%	\sqrt{E - p_z}\\
%	\sqrt{- p_x + i p_y}\\
%	\sqrt{E + p_z}
%	\end{pmatrix}\frac{1}{2\sqrt{E + |p_x + i p_y|}}.
%\end{aligned}
%\end{align}
%
%
%
%
%%Lad os Taylor-ekspandere dette. Vi har
%%\begin{align}
%%\begin{aligned}
%%	&\frac{\partial}{\partial p_x}\frac{\sqrt{E + a p_x + b p_y + c p_z}}{2\sqrt{E}}
%%	=\\%&  
%%	&\frac{1}{2}\frac{a + \frac{2 p_x}{2E}}{2\sqrt{E} \sqrt{E + a p_x + b p_y + c p_z}}
%%	- \frac{\sqrt{E + a p_x + b p_y + c p_z} \frac{p_x}{E}}{-4E^{\frac{3}{2}}}
%%	=\\
%%	%=& 
%%	&\frac{a E^2 + p_x E}{4 E^{\frac{5}{2}} \sqrt{E + a p_x + b p_y + c p_z}}
%%	- \frac{p_x \sqrt{E + a p_x + b p_y + c p_z}}{-4E^{\frac{5}{2}}}
%%	=\\
%%	%=& 
%%	&\frac{a E^2 - a p_x - b p_y - c p_z}{4 E^{\frac{5}{2}} \sqrt{E + a p_x + b p_y + c p_z}},
%%\end{aligned}
%%\end{align}
%%og
%%\begin{align}
%%\begin{aligned}
%%	&\frac{\partial^2}{\partial p_x^2}\frac{\sqrt{E + a p_x + b p_y + c p_z}}{2\sqrt{E}}
%%	=\\
%%	&\frac{2 a p_x - a}{4 E^{\frac{5}{2}} \sqrt{E + a p_x + b p_y + c p_z}}
%%	-\frac{a E^2 - a p_x - b p_y - c p_z + \ldots + 4 E^{\frac{5}{2}} \ldots}{4 E^5 (E + a p_x + b p_y + c p_z)},
%%\end{aligned}
%%\end{align}
%%%Åh, jeg orker ikke mere for i dag/aften.. ..Burde næsten bare give det til en lommeregner i stedet..
%







\newpage
\part{(old)My ideas for blockchain, a new open source business model, interactive theorem proving (ITP) and formalized mathematics, formalized programming and a more open and user-driven web}
%Since there are ways that all my ideas for all these different subjects intersect, I will just lump them together and not spend too much energy into separating them out neatly into their own sections.
\chapter{ITP, F-IDEs/PIDEs and a more semantic and more open source web}




%\section{Draft for paper on mathematical programming and how bring it about (27.05.21--???)}
%%And by `paper,' I don't necessarily  mean an article.. I should look the term up..
%
%
%%Tanker (brainstorm) om motivationen til lige mit ITP-system:
%%(28.05.21) Jeg tror godt man kan komme uden om mine idéer, ved at bruge programmer, hvis semantik bare selv er uformelt defineret.. Men igen er spørgsmålet så, hvorfor ikke forvente det samme af selve ITP-programmet, som man vil forvente af programmer i et matematisk programmeringsparadigme, nemlig at semantikken er defineret matematisk ud fra nogle sætninger? Og jeg har så en måde, hvor man faktisk kan opnå dette for selve ITP-systemet. Ja, er dette ikke motivation nok næsten? Jo. Og så kan man også ret hurtigt forklare om universaliteten i sådan et system, og hvorfor dette er ret lækkert.
%
%
%{\small
%	Abstract: %\lipsum[1-2]
%	%In this paper, I will introduce an idea a type of interactive theorem proving (ITP) system well-suited for program verification, as well as an idea for an online platform where users can uplaod their own and validate each other's programming solutions. ...
%	...
%}
\section{Sketch of my idea for a semantic web of programming solutions...}%more semantic and formal stackoverflow-like internet platform (???--???)}

%Brain: Angående relationen til det semantiske web, så vil det jo stadig blive nemmere at søge på løsninger, når de er matematisk definerede. Det besværlige ved matematisk programmering, hvis ikke det var for brugen af ikke-semantisk-formaliserede bevisskridt, ville jo være beviserne, og ikke at man skal opskrive dokumentationen matematisk (som jeg forudser det). Så en ekstra gevinst ved denne nye teknologi er altså stadigvæk, at det bliver nemmere at søge efter løsninger. Derfor giver det så også faktsik god mening at snakke om et "semantisk web af programmeringsløsninger" (og altså stadig at bringe 'semantisk web' på banen).
%Hm, kunne en disposition så være noget i retning af: Fremtidsmuligheder omkring matematisk/semantisk programmering (meget a la noget, jeg har skrevet før), hagen ved, at det så er ektra arbejde at bevise ting formelt, idéen om at tillade uformelle skridt i beviser, når bare man så stadig opsætter formelle systemer omkring at tage sådanne uformelle skridt (og særligt ved at at bruge hinanden i et fællesskab til at rate bevisskridt (og argumenter i det hele taget)), idéén om at inkludere brugerne mere i styringen af af webbet, som en måde at sætte skub forretningen (Hm, jeg skal faktisk lige genoverveje, hvor meget man kan sige her, for hvis man ikke siger, at kunder kommer til at overtage mere og mere, vil man så ikke miste... nej vent.. hm... *(Okay, min første version af denne idé-skitse bliver faktisk uden noget af det, det nok ville kræve, at man startede et helt nyt firma.)), og så ellers slutte af med at tilføje nogle tekniske muligheder, bl.a. mine idéer omkring en refleksiv ITP, der lægger op til brugen af certifikater m.m.?
%Okay, hvis jeg så starter med ikke at lægge op til, at der nødvendigvis må startes et helt nyt firma (for at kunne lave de særlige aktier og diverse kontrakter omkring dem), kunne jeg så alligevel nævne det med, at danne en server-virksomhed, hvor brugerne kommer til at bestemme algoritmerne alt efter, hvor meget de har betalt?.. Nå jo, og man kunne da så også godt nævne (som jeg også har tænkt før), at man bare så kan udlove belønninger til programmører (lidt ligesom at YT-skabere før belønninger), hvor man så også på samme måde kan lade brugerne få indflydelse her.. Det tror jeg næsten jeg vil, og så kan jeg bare nævne det til sidst som nogle ekstra-tanker (i første version af idé-skitsen). Hm, men så begynder det så at blive oplagt, at blive en applikationsvirksomhed også, der sætter betalingsmure op for IT-løsninger... Hm, og da dette jo vil være den helt store forretningsidé, er den ikke rigtig til at komme udenom.. Okay, men hvor går grænsen så for, hvad jeg bør nævne i denne version (så idéen ikke kommer til at virke for naiv-visionær i for mange folks ører)?.. ...Hm, hvis jeg nævner det med at bygge en IT-virksomhed, der belønner bagud til bidragsydere, så er jeg jo allerede inde i et lidt indviklet territorium.. Men på den anden side... Og hvis jeg putter det hele til sidst i dokumentet og bare siger, at jo bare er nogle lidt løse idéer til noget muligvis meget interessant.. ... Ah, kan jeg ikke bare cutte den der, hvor man allerede fra start lover, at de fremtidige brugerne får mere og mere del i aktierne osv., og så kun fremføre idéen om, at man jo eventuelt kan betale både arbejdere og brugere ved et system, hvor pågældende bliver belønnet med aktier, men hvor man altså umiddelbart stadig beholder al ret til at ændre på dette løbende (sådan at man som investor dermed stadig kommer til at sidde med så mange kort som muligt (hvilket dermed slet ikke bryder med den kapitalistiske tanke)). Det tror jeg, jeg vil. (02.06.21)















\section{Interactive theorem proving and formal IDEs: some summarizing notes (03.02.21--11.03.21)\label{ITP_feb-maj}}

I have worked for quite a while on an idea for a interactive theorem prover (ITP) and have at this point gathered a lot of ideas and visions for how such an ITP could be designed. 
The point was not as much to have a program to do the proving for the user, as is the case for some ITPs, but more to have a nice digital environment where the user can expand a library of propositions, from all kinds of simple equations to complicated and important theorems, by taking formal steps in a work session and saving the result. My thought was (and I actually started working on the idea even before I knew of other ITPs) that for any mathematical step that a mathematician can take on paper, it must be possible to formalize the rationale behind that step into a manipulation of a proposition in the digital environment and to prove that said manipulation is correct for all relevant input. I understand now that this might have been a naive assumption, and that there are a lot of abstract steps mathematicians use that are hard to even formalize, let alone to prove.

\subsection{My overall vision with the ITP}
But maybe it is enough to be able to just formalize the steps and leave out the proof for later, or for the reader to simply accept the method without a formal proof. The point with a digital environment for working with mathematics is not to put extra requirements on the user's proofs, but simple to give more functionalities to the user than a paper or a normal editing application offers. At a minimum, these functionalities should include the possibility to prove and utilize all kinds of simple formula manipulations as well as arithmetic and logical computations. 
And when it comes to more complicated parts of a proof, where a typical mathematician in practice would make some assumptions about the correctness of certain method in order to step over that part of the proof more quickly, a user of this kind of ITP should then be able to do exactly the same. 
The most simple way to do this is to add an assumption to the proof about the specific step and then add some text explaining what is happening. So far, I have described a text editor with added ITP functionality for the simple parts of a proof. But I think we can do even better than that. Instead of just adding some text to describe the method used for the step in a natural language, I imagine an ITP where users are also able to formalize those methods themselves. 
For this, a good reflection principle is needed. I have a specific way I want to introduce this reflection principle, which is has become a very central part of my ITP ideas, and that is to extend the mathematical foundation at an early level by introducing reflection axioms. These axioms are intended as a way to add automation to the ITP *(well, actually not quite: Reflection \emph{rules} are used for this) as well as to add new, practical avenues of proving. The reflection axioms will make it possible to prove a property of a mathematical object that reflects the mathematical foundation itself (minus said reflection axioms (or perhaps just with less powerful reflection axioms (RAs) that then speaks about a reflection of the foundation without any RAs at all)) and then use this to add corresponding propositions to the outside theory, i.e.\ which works on ordinary propositions as opposed to their representations in the reflected model. 
This has several benefits but on the topic of formalizing methods for the more complicated proof steps, it means that we now get a mathematical language to describe methods where we can then actually use those methods directly in the proof, either by assuming that they are correct or by proving it (perhaps subsequently). If a user for instance want to use some algorithm on a formula to deduce the next one, but the correctness of that algorithm has yet to be proven, then the user can describe this algorithm in the reflected model, assume it to be correct, use it to prove that the step is valid in the reflected model given that assumption and then, via the axioms of reflection, ``lift'' the resulting proposition object out of the reflected model and transform it to a standard proposition of the outer theory. To take this little ``detour,'' if one will, instead of carrying out the procedure in the outer theory might be practical for a few reasons. First, it might be easier and/or more intuitive to prove the correctness of the method in the reflected model, where one can work purely with mathematical objects (even though the method might manipulate the fundamental propositional formulas themselves). Furthermore, once the specific method is assumed, or indeed proven, if the method is general enough to work for a whole set of cases, the user might get the opportunity to apply it to other parts of the proof, or to subsequent proofs. And last but not least, it might be possible to search for the relevant method online. This search might happen after the user has developed (and formalized) the method, in which case the user either finds said method and sees if it is proven already or does not find it but might then want to upload the method for other users to investigate and perhaps prove (or disprove). The search might also happen as soon as the user suspects that a certain step must be possible and want to see if such a method is recorded and investigated, and perhaps proven, by the community. 

This is what I mean when I say that I think we can do better than just write explanations in a natural language for steps that we do not want to proof in detail but want to move quickly over. Instead the user might be able to formalize the method of the step, which increases the benefit of having a proof for the method, as it can be generalized to other cases, and it also increases the likelihood that other users will have investigated the method already, subsequently saving other users from having to do the same work. And with the reflection axioms as mentioned, the proven methods can be imported directly to a given proof that relies on them, without having to restructure the proof at all. This is all very fine, but we can do even better than this by sort of mixing the two approaches, i.e.\ of modeling the method mathematically and of just explaining it in a natural language. I have already alluded to the fact that a user might benefit from others having investigated a method, even if they have not succeeding in proving it. This is possible if the overall community have a system (or several for that matter) where users, or user groups, can be attributed certain credentials to them. Now, since these can be digital and use cryptography, such that one can be certain that some user (group) with certain credentials has signed some evaluation of a method (or any general proposition), this actually means that these signatures can be used as part of proofs. Parts of the community can thus work together to create whatever system of user credentials they desire and use that system to give corresponding credentials to specific proof steps or to generalized methods depending on what user (group) has signed onto it and in which way. In the end every user can then in principle design a way to transform these proof credentials into levels of acceptance, or even probabilities if one wants to be so bold. 
%\footnote{Of course then again, why not? I would actually suggest probability curves as a way to get the best of both worlds: It removes the discreteness of using acceptance levels but it means that the user does not have to think about each probability that much. The point is to more answer the question of, ``what do you think the probability curve is curve would you ....}
The user can thus create their own landscape of acceptance levels for different proofs depending on other users signatures as well as individual assumptions (e.g.\ how much does this individual user trust a certain method or reflection rule (which we will get more into), or a certain type of hardware etc.). In practice, however, this is not something that each user is required to do. Instead the (perhaps advanced) users can just work together to develop such a landscape which can then be adopted by anyone who wants to. While it should be encouraged that each user has the ability to customize their own acceptance level / probability landscape, it is certainly nice to have some agreed upon standards as a larger group working to create and enhance proofs in order to build and fortify libraries of propositions and methods.


And thus I actually imagine that this technology can help to formalize the field of mathematics more. But compared to other such attempts, like Hilbert's program, In this approach nothing needs to necessarily change in what mathematicians require of proofs; the proofs can be constructed exactly the same way as before with the same level of detail. The difference is that in my vision, how details are skipped and how rigorously a proof needs to be investigated by the community is then formalized itself. If for instance a mathematical community uses a system of picking out a number experts and have them go over parts of a proof a certain amount of times and spending a certain amount of time on it, then they just make this process formal by constructing the relevant formal statements (in a digital format) that the relevant parties need to sign (which actually might, for all I know, be exactly how things are done already). The fact that the documents and signatures are digital, at least at the end of the process (i.e. someone can in principle just gather it all at last and finally sign a digital document with the results, if that someone is a trusted enough party), then the document can be uploaded digitally in the community and used by individual users to develop their acceptance levels.

By having such an online community that works to build different mathematical libraries, and with a good interface to search through and take part of these libraries (such as a good website or a module of the ITP application (or both) to engage with all designated databases across the web that the community has access to), this will also increase the accessibility of mathematical results (as the ITP application should also for sure be open source itself), including all of applied mathematics and more. So far, a have focused on the field of mathematics itself, but one can certainly also make the case that this technology (and I take organizing a community to be part of the ``technology'') can benefit other fields as well, especially fields apply mathematics a lot. 
%So far, a have focused on the field of mathematics itself as I wanted to make the argument, that this technology (and I take organizing a community to be part of the ``technology'') can benefit this field. But it is even easier in my opinion to make the case that this technology will benefit other fields
By having immediate access to the latest development in mathematical techniques, as well as having open access to all the  more standard techniques as well, other professionals and hobbyists will thus have mathematics much more available to them. This is another reason why I think the field of mathematics should be formalized and moved more online this way. The way I see it, any scientific field such has mathematics should not only consider its success by how well-developed their theories and practices are, i.e.\ the ones shared among the experts of the field, but also how accessible that information is to other parties that can benefit from this knowledge. So while experts of a field can have a good system of sharing knowledge between themselves, they should not necessarily be contempt with that if there an alternative is discovered where the same is true but where the same knowledge is also more accessible to the public.\footnote{This is certainly true for any field of research that is funded mainly by the public, such as with mathematics, but having knowledge more accessible (and technologies more open source) can of course also be beneficial to other sectors as well (which is why there is a lot of energy behind open source already today, and that energy is not coming especially from the public sector as far as I know).} 

Speaking of other fields of science and technology, I imagine that these will be a big part of the community as well. It is natural for me to focus especially on mathematics since the topic is theorem proving but all the same arguments can be repeated for other fields. In some ways there might even be a stronger case when it comes to these fields. I think in particular about the fact that such fields does not tend to apply the same rigor as mathematics, when deducing the correctness of their methods. And at the same time, the mathematical deductions are typically on a lower level to my understanding of things,\footnote{I have a background in physics (mainly) and computer science so I cannot claim to know a lot about how mathematicians typically work.} because the problems tend to be more concrete in comparison. And by doing these calculations in an ITP instead of in a calculator application, I thus think they can achieve a higher level of rigor, since the ITP can help make sure that said calculations are used correctly to reach the correct conclusions.

%Jeg kan godt begynde at speede mere op med skriveriet; jeg skal alligevel genskrive det.
%
And as I will get into shortly, %later, 
I think that this kind of ITP will beat any regular calculator program that does not use theorem proving and open source. So far I have only talked about having reflection axioms and formalizing methods as well as the whole process of using otherwise not-so-formal steps. But I also have an idea for how til introduce certain reflection rules that will give users the ability to prove the correctness of binary DLLs to have a certain effect on binary objects (in the memory) and with the use of the reflection axioms, these computations can be used for anything up to carrying out whole proof procedures. With a good system of assigning trust (i.e.\ acceptance levels) to any not-yet-proven proposition, e.g.\ to propositions about the action of DLLs of a certain format and on a certain OS and with certain hardware (and here the user can just chose his or her own OS and own hardware to get the appropriate acceptance levels of the actions carried out by the reflection rules), users can then get to use these DLLs. This is a big problem by the way with current ITPs, namely that mathematicians have not yet (as I understand it) developed a good system of assigning trust to machinery and thus to ITPs that use computations via the principle of reflection. But once the whole process is dissected and formalized, which means that each assumption necessary for using said reflection is written up as a mathematical proposition, then it is just up to the involved parties to go through the list of assumptions and check that they all are clear-cut enough to be able to assign some probabilities to them, and from that point on each user can in principle make their own quick calculation by assigning individual probabilities to the assumptions and get their own resulting acceptance level. And once this is done by enough trusted parties (with expertise in the matter) in the community, the rest of the users can just adopt their general probability / acceptance level landscape as an easy way to get going with the relevant DLL. Oh, and an important point to mention, I just realize, is that such assumptions should be as effective as a natural language and thus be able to talk about real-world things and in an abstract way even, such as ``can this binary DLL instruction be trusted (and how much so?) to have this effect in the overall program, if said program halts without error?'' and ``how does the probability of an certain output to be the correct output of a program, if said program is run on several pieces of hardware (and perhaps several times) and produces the same output in all cases?'' But how does one convert such statements into mathematical propositions? The answer is of course by the use of signatures of certain trusted parties, just like as when i described the formal proof investigating process before. So you basically transform any NL proposition $T$ into ``does there (or \emph{will} there, depending on where you are in time) exist a signature of party $x$, that undersigns the text $T$ to be true with probability $p$?'' Here, $x$ is whatever (to some extend trusted) party you like to inquire about their answer, and $p$ is of course a numerical probability. (One could also use probability curves or levels (as in acceptance levels).) The user then extracts $p$ from this and use it to assign his or her own probability $Q(p)$ by continuing the proposition as ``\ldots, then proposition $r$ is true with probability $Q(p)$,'' where $r$ is a proposition that speaks about the correctness of the more concrete propositions, possibly about the correctness of some methods, DLL effects and/or not-yet-proven theorems\footnote{Or how a certain forcing of the given theory into some special model, perhaps the standard model of the theory, will mean that a certain proposition that cannot be proved otherwise is true. An example of this could be if one wants to assign a probability of Goldbach's conjecture being semantically true in the standard model of ZFC, despite not necessarily being provable in the theory.}, that you intend to use from all this. One can also use several $p$s this way, coming from different sources, and thus use a $Q(\boldsymbol{p})$ over a whole vector, $\boldsymbol{p}$, of probability variables.

So to get back to why such ITP application will likely beat regular calculator programs, one obvious reason is then, that the userbase will be able to continuously update the application with new algorithms and by making existing ones more effective. And since the reflection rules should have the power to prove correctness of DLLs (as long as they only use some subset of the relevant instruction set), there is no limit of how effective these procedures can be. But the same can be said about a (low-level) programming language; here the userbase can also work together to built and maintain effective libraries of algorithms. There are, however, a few important differences that set the ITP application (plus community) apart from just another programming language (plus IDE, plus user community. First of all, the fact that the correctness of the algorithm are proven both means that they can be more easily trusted and, more importantly, it can makes it easier not to accidentally draw incorrect conclusions from those results, since you get correctness propositions out of it that can be used to draw the desired conclusion if the math holds up. With the way that I envision the ITP, even updating the application itself, including its whole user interface, will be a very efficient process. I even think that this will be able to give momentum to a whole new (mathematical, formal) programming paradigm, since the methods/algorithms in question might as well be modeling any program/application of real world. I will get back to this topic and speak about F-IDEs (an existing concept which is, as I understand, in line with what I imagine for this programming paradigm).

Additionally, on the topic of comparing to regular calculator programs and programming languages, I also imagine that it will be a lot easier to search for the desired algorithms and techniques when the users can search for propositions. This means that the users will be able to make exact semantic searches for what they are looking for.\footnote{I am a big believer that the \emph{semantic web}, as it is called, will be a very important technology in the future, and that we thus will be able to add much more precise semantics to our internet searches at some point. I imagine that this will end up being the case for all kinds of subject and specific question, even very abstract ones. But for the more technical subjects, the case is even more clear-cut. And it is exactly by constructing such a technical language based on mathematical propositions (in such an active technical community) that I think we will be able to get going with this semantic technology.} 
If a user want to find a method for making a certain calculation/operation on some mathematical object (including one that represents a list of propositions), this user can thus search for methods on the specific domain and then narrow the search by giving properties about the method. The point is then that this narrowing will be semantically precise and the relevant server can then look up a method that has exactly those properties. The matter is then of course that the server will probably not have all possible variations of propositions recorded about a certain method and thus might not have exactly the queried proposition on hand. But if the server has power to rewrite propositions and possibly make logical deductions, and if the user community is foreseeing when it comes to for what properties their uploaded methods might be sought after, the server can still have a good chance of being able to provide the correct methods when they exist.

Since the methods can range from being exact computational operations to being large processes involving human actors that sign onto various abstract statements along the way, this technology even have the potential to go a lot further than just getting the right help for making certain computations (and in a safe environment that helps ensuring that the computation results are evaluated correctly). There is in theory no formal process, in the science community as a whole or indeed in any professional community, that cannot be formalized this way. And once it is formalized, that also means that all people, if it is also made available to the public, can search for this technical/scientific process. This will mean, I imagine, that e.g.\ upstarting companies, learning scientist, learning professionals in general, upstarting communities/organizations\ldots\  basically any party that is setting up something new or learning something new, including experts branching out/over to new things, that these will be able to quickly learn about and adopt the whole process, since this is described in detail. And, perhaps even more importantly, other parties (including consumers/users/audiences/readers) will then quickly be able to see whether the process meets the required standards, since this will simply be a matter of whether the relevant certificate is able to be produced. Of course, all this might be a bit further away in the future, i.e.\ before this technology (this meta-method) can branch out into all these other areas. But the possibility is there. When I get back to discussing F-IDEs and the new programming paradigm, I will get into how I imagine that the development process itself will be formalized. I imagine that this will be one of the first important ways where this meta-process will be developed and used. I then see this as the potential first important step that then opens up for other abstract areas of work to adopt and advance this meta-process afterwards. 

As a last point, when comparing to a regular programming language, I imagine there will be quite an easy learning curve for this application and its child applications. I have mentioned that the application will be efficient to upgrade by its userbase itself (i.e.\ in an open source way), and that this includes anything up to the interface itself. This is because all you need to do is to prove that the theorems produced are still as correct as before, and when it comes to the user interface, one only need to prove that the GUI is a separate module that have no direct control over the underlying data structures but can only send messages to it to guide the proof operations. With such updates the application can not only increasingly powerful and efficient but can also branch out to serve different kinds of users. One particularly important kind of branching of the application, as I foresee it, will then be to create more simple and user-friendly mathematical environment especially designed for learners and novices of mathematics. I imagine environments where standardized problems are set up very nicely, much along the lines of what GeoGebra is aiming for, where the user have just enough freedom (or perhaps more, depending on the level of the user) do the relevant proving tasks and solve the problems. I imagine that such child application (let us call it that for now) might actually work much in the same way as the parent application where the user has a library of proof steps to work on a domain of objects (including geometrical objects for that matter), but where this library is then very limited to only very specialized (and easy-to-follow for novices) steps. The point is then that as the users learn more and more, they can then peel these limiting layers back more and more, so to say, and add more and more of the more low-level capabilities to the environment (in the form of more proof steps and more domains). We can think of such child applications much like embedded (and specialized) languages of a programming languages. Here, one can also get an effective language that is easier to learn to use than the encapsulating language (or whatever it is called) but where it is relative easy to then learn to use the encapsulating language more and more for more complicated tasks. I think that there might be a difference in how gradual one can make this transition with proof steps and domain expansions rather than escaping the syntax of an embedded language more and more, but this is not something I can really predict a lot about. I will, however, claim that this sort of application might be a lot easier to learn compared to any embedded language, simply because the latter still requires the user to learn some programming language, which will always constitute a learning obstacle for some beginners. When on the other hand we deal with an environment where the user makes progress by carrying out actions on a set of objects, which indeed describes only the basis of any application and thus includes even very simple and easy-to-use applications, we do not (necessarily) have this obstacle. A very real possibility is that such child applications can be even transported to tablets or smart phones (by focusing most of the resulting interface around point-and-click actions). Users who learn math by using such applications will then not only get what these applications afford in terms of practicing mathematics, but it will then also be easier for said users to advance to the more complicated applications (by ``peeling of the layers'' like I mentioned), since they are now already trained in using a very similar environment (and I do not see it as mattering much if the users have to migrate to a desktop/laptop along the way; that just means user will (perhaps) have to use a mouse for the point-and-click actions instead). I think that this possibility for gradual change in the applications when the users advance will be very important, especially because of the possibilities the main application will end of offering (as I predict). 
As I will get more into later, I think such an ITP will be the basis of a new important programming paradigm and thus that the application, or rather some designated ``child applications'' of it, will be used as a programming IDE. The point I am trying to make with this is that this gradual advancement, i.e.\ by ``peeling of layers'' and introducing more and more low-level functionality, the novice user might not only end up being capable of making advanced and rigorous proofs but might also end up being able to program. And needless to say, I think a good step-by-step road all the way from primary school mathematics (hopefully, at some point) and into programming at even an expert level will be a good thing to have in the future, both near and far. 
%But even this paradigm will not be very popular, the main application (and lets just include the ``children'' in that term) will still be able to handle...



So to summarize things so far, I have an idea for an ITP whose mathematical foundation first of all have reflection axioms, which ensures that you one can prove propositions indirectly by proving them for some inner model that reflects the foundation itself (without the same reflection axioms to avoid circularity) and then lifting them out an extracting them as normal propositions of the foundation (which is by the way probably done by having an IsTrue$()$ relation and a rule to go from IsTrue$(x)$ to $p$, where $x$ is an object that represents proposition $p$ in the reflected model). These axioms then basically mean that you get to use the logic of the foundation (which should of course be a sound theory such as ZF or ZFC, which is in fact what I would use) on the foundation itself (or at least a version of itself with restricted reflection axioms). The foundation then also have additional reflection rules to then add potentially unlimited automation to the ITP. These two things put together then means that users can prove properties of mathematical algorithms and then use them even to make new (arbitrarily complicated) rules for the ITP. I would for sure use ZF or ZFC for the ITP and then actually make the fundamental reflection rules (which can be upgraded over time) act on executable binary objects rather than functions. So we are more in the camp here of Turing machines rather than lambda calculus. How these rules should work in more detail, I will get into shortly below. Specifically, the reflection rules should even allow for building new ITP modules/programs and/or new user interfaces such that the community can develop the ITP itself (by proving the correctness of said modules/programs). 
On top of this, I the imagine how the use of signatures and certificates, and resulting probability/acceptance levels will mean that we can formalize everything else. other than just the fundamental mathematics, including the whole process of how to investigate proofs that skip certain rigorous proof steps as a community, as well as what not-yet-completely-proven methods (i.e.\ rules and algorithms and such) can be taken with some reliability, and also what reflections rules can be trusted on different pieces of hardware and with different runtime environments. The latter is also especially important since it means that the reflection rules can be updated continuously by the community.
With these fundamentals, I have then tried to describe some of the visions I see for the ITP and its community. These visions included being able to formalize the whole process around mathematics, to advance a new programming paradigm (and I will talk more about this) where algorithms are proven and were applications can be quickly updated, that searches on algorithms and other methods/processes will be very efficient, and even hopefully be a gateway into developing a more semantic web (and I will talk more about this also), and hopefully the meta-process developed around this ITP\footnote{
	Note by the way that I often say e.g.\ ``this ITP'' when I actually mean ``this \emph{kind} of ITP,'' and where I also take the community around it to pe part of the concept itself. So here for instance, ``this ITP'' refers to any ITP system with a user community connected to it that uses the reflection axioms and rules as described as well as the signatures/certificates for including even abstract processes in the ``proofs'' (which do then not necessarily yield 100 \% true-or-false answers).} 
can even be exported to the rest of professional society. The visions also include that other technical professionals that uses mathematical and/or computational methods will benefit greatly from such an ITP and that learners of mathematics, and/or other technical fields, will get a good educational environment to do so, and where there is an easy transition into more professional use of the application (by ``peeling of layers'' as mentioned). 





\subsection{Design details}
\subsubsection{Reflection rules and the programs they run}
I will now explain more about the reflection rules. As mentioned, I think that some of the very fundamental reflection rules should be about making executable files, and in particular DLLs if this is possible, so that the whole ITP program does not need to be recompiled every time a new rule is added to the environment (but only require a restart of the relevant module). Well, come to think of it, it might also be beneficial to also be able to include libraries in the more ordinary way and not just as DLLs so that the efficiency can be increased for rules that are used very routinely. The important point is just that the library files are separated out so that the user only has to proof the correctness of these individually and not have to consider how these are compiled and so on. How the libraries are compiled and included, in the ordinary way or as DLLs, should be taken care of by the main application itself. Whether or not to trust this compilation can then initially be chunked together with the whole question of trusting the main application itself. At some point, all these assumptions about the main application itself should of course be investigated by the community (and bugs can be corrected), but only once this starts to be feasible to do. Of course, an ITP can in theory not be the basis of a proof of its own correctness, but in practice, this will indeed be a very sensible thing to do. But initially, the compilation process should just be hidden part of the application itself, and thus part of how the fundamental reflection rules act, when seen from the user's perspective. The users should not even have to consider anything about the hardware or OS themselves in the beginning. I would suggest instead that an intermediate language in the form of a simple instruction set, which the application then can compile and run on all the common architectures and operating systems. Since there are enough open source compilers (and since the ITP should for sure be open source itself), this should not be a very difficult thing to implement. (I could of course be mistaken in this, but I hope I am not.) This instruction set could very well just be a subset of an existing intermediate language (like CIL perhaps (I would not know)). The subset should then initially be very basic and not include anything on the level of the OS, i.e.\ no faults, no messages, no forking, no concurrency and such. *(This is not completely true. The programs should have library functions included to allocate and free memory and to read and write to files. I will propose making use of a specialized API for this so that the intermediate language can remain OS independent, and also simple in terms of defining its semantics. I will return to this subject below.) 
Well, come to think of it, there should of course be instructions to exit the execution, and maybe it would be okay to include some other potential faults, as long as\ldots\  No, never mind. The users can implement their own way of letting algorithms ``fail'' without the actual processes failing, and this will only end up being more efficient than using the OS for the same. So and instruction to exit the program is sufficient here at the early stage. 

On top of reducing the instruction set, the programs also have to meet certain prerequisites before even being considered a valid program by the fundamental theory of the ITP. Its has to work the same regardless of where the program and lies in the address space as well as where the ``input/output objects'' lie, which means that it returns the same output upon successful execution. For the basic programs, it should be assumed that each I/O object is stored on an interval of the memory with consecutive addressed and with the appropriate read/write permissions granted. So even though pointers should be a basic part of the programs, no actual pointer values should ever be leaked to the user upon successful execution (i.e.\ unless the process is somehow stopped by the OS before reaching the exit instruction). Apart from the I/O objects, the program might also use extra memory, where the resulting contents are never disclosed to the user afterwards (which then means that pointer values do not have to be erased before end of execution), and where non of addresses are read by a program before having already being overwritten by the same program. I will just include such objects in the term `I/O objects' in the following and then just conclude that the different kinds of programs should have such prerequisites in order to make sure that each successful execution has the same result as seen from the user before we can even begin to consider their I/O action mathematically. There is thus no need to assume that invalid programs even have outputs and there is certainly no need to run them. If a valid program fails successful execution, the ITP should of course just go back to the previous stage and inform the user of the failure. Once a program of some kind, e.i.\ with some set of I/O objects with certain permissions/restrictions, is proven valid, the reflection rules should then make it possible to run the programs on some specific \emph{meta-classes}, let us call it that, of binary objects. I call it a mata-class since it is not only defined by its mathematical properties from the perspective of the reflection rules but also how it is stored. The rules will for instance not work on a variable that represents a binary object (and 'binary object' should also be an actual class (and yes, the reflection rules and axioms will require certain classes to be predefined, before leaving subsequent definitions to the users)) but the object has to first be a constant that is completely written out, at least when seen in memory (the GUI might still collapse the binary object into something smaller as it appears in the formula, but somewhere all the 0s and 1s have to be located in subsequent memory addresses). The actual class of binary objects should of course just be some class of sets that can be represented by a list of 1s and 0s, and there should then be fundamental rules as well to transform such binary objects back and forth between the reflection-prepared states, where all 0s and 1s are in the memory and where the formula node just has a single pointer to the first address (and should probably also hold the size), and other states where for instance the same objects is written as a concatenation of two such reflection-prepared states, or where it is a list construction (cons) of a single (or perhaps several) binary number(s) and then a reflection-prepared formula node with the rest of the binary object. One can note that already with these two possible transformations, as well as their reverse, we have the basis to transform back and forth between the reflection-prepared states, which I should find a batter name for (\ldots\  what about just memory objects?), and more conventional formula trees.  Of course, cons and concat are not so conventional formula nodes but these should for sure be fundamental to the ITP. I have not yet talked about it in these notes, but the reflection rules and axioms are thus not at all the only extensions of the base foundation, which I think should be ZFC in FOL. The FOL that I am thinking about does not even have function variables in it (and I do not think it should), so having for instance cons and concat formula nodes, as well as memory object nodes (let us indeed call the reflection-prepared nodes memory objects for now), so these are obviously already part of an extension on top of the underlying foundation. I will get back to what I think this initial extension should include, before the users take over and get to design their own further extension (via the reflection rules and axioms).

Oh, and a quick but important note that I just thought about: I do not know whether it makes completely sense to call either the reflection rules or the reflection axioms that. As far as I have gathered, the reflection principle is what allows for automation in existing ITPs (that use this principle and this term for it). And to me, ``reflection'' seems like a good word to describe building an inner model in a theory to mirror the theory itself. When I read about it, it also seemed to me that the word came from this line of thinking, but, as I am just realizing now, I am not completely sure, that this is true. Well, regardless of this fact, I have to call it something, so I will just stick to using these terms throughout these notes. I hope that what I mean is still clear by the end of them.

Now, to get back to the binary objects/memory objects as I/O objects of the programs, there can, and should, be several kinds of these depending on how they are stored. I got a little bit ahead of myself when writing that the should be stored in consecutive memory. Even though this would certainly be best in most cases, where the objects have moderate size, there might be exceptions to this. And in particular, there \emph{should} in fact be at least one exception to this, which I have yet to talk about so far, namely file objects. Oh, and we can probably clear up the nomenclature so far by saying that memory objects denote objects that are strictly in (virtual) memory at all times until they are discarded, file objects are objects that are basically implemented like file pointers of C, and will with all likelihood be implemented in the application by the use of file pointers or something similar (e.g.\ the lower-level file descriptors of C or whatever file APIs that C++ has to offer (and I would choose C++ by the way if I were to begin building the ITP)). How and when data is loaded into memory and written to the hard disk should thus just be underneath the API of whatever C/C++ libraries is used, and should therefore not be a concern neither of the programmers, and certainly not of the users. All the users should need to know is that, if a proposition uses a file object rather than a memory object, then this proposition can be stored directly in the proposition library (which has the form of a file; one could also use (a relation of) a local data base, but saving proposition libraries as files is probably a more realistic choice in my prediction). Propositions that use memory objects instead thus have to be converted before they can be stored. Let us then for now call the relevant ZFC class `binary objects' as say that such objects has to appear as an ``atomic'' binary object node in a formula, before a program can run on it. It should just in theory just be up to the user not to open too many files and not to use to much memory, but the ITP application could of course help by giving warnings. 


The most common programs used in custom rules will probably be the ones that just takes a single objects plus some additional memory and then rewrites that binary object. But there might be a great number of possible types of programs with the different types of binary objects and with different permissions/restrictions. Each of these programs will then, as mentioned, have to be proven valid before they can be run. There will probably not be an real difference this proof whether some binary object is a memory object or a file object\ldots\  well, except that file object is not limited in size the way the standard memory objects are. 
%I do not think that dynamic memory allocation is necessary for the first version of the ITP. Instead the object sized should probably just be parameters of the custom rules using the programs and then the ITP can just try high enough sizes that the algorithm succeeds, i.e.\ that it yields a valuable output when the execution (successfully) halts rather than outputting an error message when the execution (also successfully) halts. ITP should thus be able to repeat the procedure with different result until it accepts the result (probably by parsing the result according to some user-defined regex) and then find the minimal, or just some somewhat minimal, size parameters. Sure, this process will be a bit slower the first time around, but for all subsequent verification by other users running the same proof script (and I will talk more about this since we have not touched upon it yet), the process will then be more effective than if ones just used dynamic allocation. This is not to say that dynamic allocation should never be part of the ITP programs, as this could indeed be used to make finding said size parameters a bit more effective for the one constructing the proof. 
%In fact, there is no limit for how complicated the programs can be in the future... (Not dynamic allocation to begin with...?? But what about child applications..??...)
%%In the beginning, I would even just let the users have to find the sizes themselves...
I have thought about whether it is necessary for the programs should be able to do dynamic allocation in the beginning, and I have concluded that I think it is a good idea since it will probably be needed especially for the ``child applications'' (which is where an alternative version of the main application is constructed as one of these programs within the theory and then run by the use of the reflection rules). When I speak about using file pointers (or such) for the file objects, this will also require more than just the arithmetic and logical instructions of the intermediate language. So I suggest that there is a library included with a few simple functions to allocate and free new memory objects and to create, read and write to the file objects, which should of course have an actual file underneath them that gets the same changes. The semantics of these library functions on top of the simple instruction set should then also have as simple and straight-forward semantics as possible. In terms of the additional memory blocks used by a program, I still think that this should be part of the rule that should be handled by the ITP before execution. The ITP should thus make sure that a (no-reading-before-writing-to-addresses) of the relevant size (which might depend on a parameter of the rule) binary object is prepared beforehand. As opposed to the actual I/O objects, the users should thus never need to deal with these preallocated memory space objects themselves, apart from when they need to choose and adjust their size before calling a rule. 


Using library function like this, on top of the otherwise reduced intermediate language, we also get an easy solution to another problem. When it comes to getting the memory address of the memory objects, as well as their sizes, this can also just be handled by certain library functions. One could for instance imagine a function that just writes all this information to a memory location (such as a preallocated memory block so that the data is local to the process and does not have to be manually erased by it), which can then be called at the start of a program. 
%When it comes to how many binary objects a program uses and what formats these have, there is a lot of possibilities. (Hm, hvor vigtigt er dette egentligt nu, er det ikke meget trivielt?)
% More on formats!... 
With such functions we could also choose to handle the permissions/restrictions within the programs, letting them fail if the right permissions are not there, but I think it is a better idea to let that be handled by the ITP application. I thus think that what types of objects a program require is part of its specifications, and that these requirements are checked by the ITP whenever a user wants to call a rule that uses the program. 
%When calling a program to rewrite some binary object, the ITP application of course then has to first make sure that all the relevant binary objects have the right permissions/restrictions before running the program. 


%Each I/O object is then assumed 

%How these pointers and sizes are input to the library function might vary from the different types of programs, and there might even be a way to connect several objects to one, even though they lie at different places in memory... and there should also (maybe?) be programs that use files...


As I keep writing, these programs are used by ``rules.'' I thus take the term `rule' to include all steps that a user can take as part of constructing a proof that should then be repeated by a verification program. Like with any other ITP, the point of this kind of ITP is of course to build a proof script that can be run by a verification module/program. A user might want to edit a proof after completion, but I imagine that users will mainly build these proof scripts indirectly by taking actions in an environment of formulas, where the user can rewrite terms and use logic rules to deduce new propositions from the others and so on. I have some ideas for how this working environment could be designed, for the main application or a child application, but there are many good ways to do this, so this is not that important to the whole idea. I will probably write about it a bit in these notes, and otherwise one can find a lot of the most important of my ideas in the earlier set of notes on this subject found in section \ref{main_12} in the appendices of this document (unless this text is copied over). Whenever a user takes such an action and using some rule, this rule is then added to the proof script automatically by the ITP.

So when I speak about `reflection rules' these are also actions that the user can take as part of a proof, specifically proofs that proves the correctness of one or more programs to rewrite binary objects. I will now get into what it should take to prove the correctness of programs and then also how to use these binary objects for any kind of proof using the reflection axioms. 
Let us first of all note that the main application should already implement a nice extension of the theory, so that the users already from the beginning can take high-level logic actions and use automatic arithmetic operations and list/tuple operations such as cons and concat (and I will probably get more into my ideas for this). This also means that even the reflection rules used in the main application can be an extension of a underlying deductive system. Therefore, I will just explain the overall semantics of the reflection rules here. As mentioned, the reflection rules and axioms already require some predefined classes and predicates (e.g.\ the IsTrue$()$ predicate as mentioned above). Notes that we are in an extension of FOL so ``predicates'' hare actually means variable propositional formulas. Before the reflection axioms can be introduced we thus for instance at least need a way to define variable formulas that can then be collapsed and expanded in the formula tree (which is also just a necessity, since otherwise the users would have to write up all formulas in bare FOL, without any function symbols and without any other predicates than IsMemberOf$()$). This means that the users will be able to define their own ``predicates'' in a sense, which can be used in atomic formula nodes in the formula tree. Here, `variable formula' (variable in the input terms that is) will be a better terms for this predicates, at least in an early extension of the underlying FOL deductive system. That is if we do not just use SOL with Henkin semantics as the underlying deductive system, which would be just as good. 
%
%A quick point about using FOL or SOL with Henkin semantics as the underlying foundation of the ITP before going back to the reflection rules: As is generally understood, the expressiveness of normal SOL or HOL comes with a cost. I think that one can maybe think of it this way: It is intuitive to want to express theorems as general as possible, because that is a fundamental part of math; we want to aim at proving things as generally as possible. But once one starts using the expressiveness of e.g.\ SOL to generalize propositions to more profound levels than can be achieved with FOL, then you actually, un-intuitively, loose generality. The point is that a mathematical language such as FOL is naturally intended to be applied to any real-world (or thought-up) system by formally reinterpreting the involved predicates, such as IsMemberOf$()$, as concrete predicate of whatever system we are looking at (and then of course check that all axioms and axiom schemata are fulfilled). So even though most people never think about reinterpreting IsMemberOf$()$ formally when using theorems of set theory on real-world systems, this would be the formal way to do it, if one had to dissect and specify this process formally. But this process means that the fever assumptions we have about the predicates, 
%
%
%Maybe SOL with Henkin semantics is better..!
% (...In this way, FOL is already as expressive as SOL, not in terms of its models, but what it can be applied to...)
%
%
%(Come to think of it, it might even be a more natural choice than FOL. With this you also get the possibility for axiom schemata for free\ldots\ well, at least\ldots\ Okay, you might have to then introduce a way to actually set explicit restrictions on the set of predicates for quantifiers, which would mean that you do not ``get the possibility for axiom schemata for free.'')
The reflection rules should then generally give the functionality to record a variable formula $P$ where there is a proven proposition on the form $\forall x,y \in B: y = F(x) \to P(x) \to P(y)$ where $B$ is the set of binary objects and where $F(x)$ represents the output of a program taken on $x$. There might also be additional antecedents to this proposition to define $P$. The reflection rules should then make it so that for any proposition with the same antecedents, if there is a $P(x)$ somewhere in the formula with an even number of negations in front of it, where going into an antecedent of an implication node count as a negation, and where there is no biconditionals on the path to $P(x)$, then a valid proof step is to simply rewrite $x$ with the output of $F(x)$. Another rule should by the way state that a similar proposition, but with $ P(x) \leftrightarrow P(y)$ instead of $P(x) \to P(y)$, should mean that $P(x)$ can be rewritten anywhere it appears (as long as the required antecedents are still there). Note that these antecedents are free to be whatever and dos not have to be true at all. Therefore $P(x) \to P(y)$ might not be true without the antecedent, even if the proposition was already meaningful without them. There is no problem with this; it just means that the user is making computations in vain. As long we do not assume something similar for $F$, we are fine. And this is an important point. When a program, $F$, is defined and subsequently proven valid, it is all done without any additional assumptions (antecedents) by the user. This way, the user will never get to run invalid/incorrect programs (as long as no there is no contradictions in the core theory of the ITP).

For programs that run on several memory blocks and/or uses files as well, there should simply be similar rules but for another class (if not several other classes) that represents a collection of memory objects and file objects, and where memory objects can be created and deleted (i.e.\ allocated and freed) and file objects can be created, opened and closed. 
In terms of the file objects, there should be input files that are read-only. This means that the users get to basically prove something about the content of files.  When proof scripts are shared between users, these files should thus be shared along with them. The other files is used to store intermediate data and to store final results. The users can also share these output files and verify each others result. I think it will be a good idea if the verification module can recognize output files and compare them with the result of the proof script, but where the data is only read from the result file and where the comparing thus just happens in computer memory. The ITP application should thus keep track of verified and not-verified files. The ITP should of course also keep track of input files with recorded proves about them, such that if the user accidentally changes something in an input file, the application should notice and warn the user, perhaps just by looking at the metadata --- one could also use checksums or hashes, but I do not think that the ITP needs to guard against malicious actors; if they have access rights to change your files, you are already owned by them. When it comes to sharing files, however, it might be beneficial to be able to use certificates in special cases as a way to be able to trust propositions about some files without having to go through the verification. A particular important use of this could be when installing the application and wanting all the user-made update / child application (as we will get more into below). It would be nice to just be able to verify the that the hashes of these files matches some certificate and then you are good to go.\footnote{
	Certificates might also even be used to add instructions to the intermediate language, but this is more advanced territory. The idea is basically that users might adjust their own trust levels of certificates and then make the ITP calculate what forks of itself the users will trust. I think that this might be useful at some point, but again, this is only for when we get to a very advanced level.
}

An additional type of object that I have not yet mentioned, due to the fact that did not think of it then, is a user input object. Its purpose is similar to the stdin file of POSIX systems; it should be a way to make the programs interactive. The standard behavior for the ITP should then be to record this user input and save it as an input file, which the verification module can then use when verifying the proof script. The proof script generated from running an interactive program can thus just pretend that the input file was there all along and use this as the user input when running the (modified) program again. This means that all programs that uses user input objects should have a direct translation into programs that use input file objects instead. For interactive programs where the execution does not wait forever on new input, but take inaction as a valid action, and generally for programs that see static actions as part of the set of user actions, there should just be a way to record prolonged actions such as no\_input(50) or key\_combination\_xyz(121) to say that certain static actions is maintained for x (e.g.\ 50 or 121) amount of input cycles. These user input streams does of course not have to be limited to keyboard and mouse inputs in the long run, but this will be a good enough start.

Now that we have mentioned that programs can be interactive, we should also look at how we get a GUI for such programs. The GUIs should follow a very simple principle: They should only be able to read the binary objects, not write to them. This means that they do not need to be, and should not be, part of the deductive system itself, as opposed to the underlying programs. If users want to prove something about the properties of GUIs, it should be with their own assumptions about how the source code for the GUIs results in objects being rendered certain ways. Such proofs should thus be like any other proof of something in the external world to the ITP. I guess the way to implement, might be to have a module that can (only) read the binary objects and send the data on to other processes for rendering. I remember reading somewhere, that when sending data over the local host port of the network card, the data actually stays in memory and is shared in an efficient way between processes. I guess that something similar can probably be said about POSIX pipes and so on, so I am sure there are several good possibilities when using separate GUI processes. Alternatively, of course, the rendering module can just use some open source GUI library. Oh, and I guess that if these libraries are included as DLLs, it means that they can still be updated separately, and the main part of the application would not be ``tainted'' so much by them\ldots\ Yes, and especially since the (constant) API of the rendering DLL (which should be independent of the underlying GUI library used) can just be limited to a single render() function (or perhaps several (render01(), render02(), \ldots) to give the user several rendering options to choose from). 
%in terms of implementing GUIs...?

*I have come to realize (while writing the rest of these notes) that the programs should also be able to use directory objects. For some time, I just planned on mentioning them when I got to the certificates, since they are relevant for these (hence I realized we needed them), but it fits better in this section. Note therefore that I will ``forget'' about them again until the certificate section below. Okay, the point is to introduce directory objects (or `folder objects' if one will) on a pretty much equal footing as the binary objects, but where the directory objects\ldots I will actually call them `folder objects' instead\ldots but where the folder objects are of course represented by sets of binary objects or other folder objects. %Well, I guess because we use ZFC, we should be careful here\ldots Yes, it would not do to use this straight-forward class as it cannot be represented by a set, which will likely be problematic\ldots Or maybe there is if we just assume the folder sets to always be finite\ldots Ah yes, of course. And this is perfectly fine; I thought about just assuming them to be infinite as a way of being able to create new files in them, but we can just as well use an abstraction were we just never know the number of files for any (sub-)folder that is open for having new files created in them. Oh, and we could also just choose a big enough number, since it would not be a bad thing if there were some limit on the number of files one can create (even though it is probably easier to handle this in a different way and just assume the file number to be unspecified). Okay, so the user should be able to extracts files and possibly write new files to the folder objects, given the right permissions are set, and this should of course results in files being created on the hard drive. Just as with the binary objects, all the desired folder object actions should both be available via designated rules in the main proof script and should also be available to the IL programs. Programs should thus also be able to handle the folder objects. Oh wait, never mind about the infinite objects and so on! I will actually keep these thought in here so I can explain the following. The reason why I thought about this, is that I had a certain kind of solution in mind of how to deal with mutable objects such as files and folders in a mathematical context, where (Hm... I will probably remove / out comment this anyway...) everything is either constant or just a placeholder for a constant (i.e.\ a variable). An interesting solution is then to just keep some variables in the objects that you work with and only specify them when necessary. And this is in fact the solution I have used for the user input objects; their lenght are indeed kept unspecified until the end of the program execution. But a much better solution for the general case is just to treat them like constant and thus make transformations to different constants along the way, jep, I let me just out-comment this; it is not interesting enough to keep in.
Since we can assume all folder objects to be finite in the number of files and sub-folders they contain, the class of folder objects still form a set in ZFC even though they can sort of ``contain themselves'' in this way, which is a good thing (the class forming a set that is). The users should then be able to extract files and possibly write new files to the folder objects, given the right permissions are set. (And this should of course results in files being created on the hard drive.) Just as with the binary objects, all the desired folder object actions should both be available via designated rules in the main proof script and should also be available to the IL programs. Programs should thus also be able to handle the folder objects. Just as with the file objects themselves, it would not make sense for the user to overwrite or to change the input folders of the proof in anyway, since it would mean that the proof cannot be verified. It would by the way be a good idea if the ITP hindered users from granting write/create permissions to input files/folders. The user should not be prevented from this completely, as it can make sense if the user intends to use the so-called ``rewrite procedures'' (at least that is what I am calling it for now\ldots Hm, why not \emph{rule substitution procedures} instead?\ldots), that I will get to in section next after the next one. If the user thus intends to use a rewrite / rule substitution procedure right away, and is confident that valuable data will not be lost this way, that user should then be able to do so, I think. As a last point, which is probably worth mentioning here, the user should be able to specify a restriction on what format the included file/subfolder names should have when the folder object is being initialized (from an actual folder on the hard drive), and also restrictions on the tree height of the directory tree. For instance, a user might choose to only include files with a certain ending, and also, say, to discount all hidden files as well. I will get more into how to initialize objects in the certificate section (three sections (or subsections if one will) down).


\subsubsection{*(More reflection rules and) Reflection axioms *(which a will actually end up proposing as \emph{rules} as well)}
This means that all the program runs on binary objects. We can thus implement (and re-implement) all the arithmetic operations of the ITP with this. But what about the ``operations'', i.e.\ the rules, to manipulate, rewrite and deduce new propositions? With the \emph{reflection axioms}, the users also get the opportunity to re-implement and add to these if they want. 
Even if the reflected representation of the deductive system is not modeled by binary objects, which it probably should, one can still make the transformation and then use the reflection axioms to finally extract the propositions. Since we in principle need to actual propositions for the reflection rules, at least with the way I have stated these above, it would be a good idea to have this process automated, so that new programs can be easily initialized for use once they are proven, even when they are proven inside a binary object that reflects the deductive system (or of course an extension of it) and with the use of other programs. So let us assume that the reflection axioms are defined via the same binary objects, so that programs can be quickly initialized\ldots\ Hm, maybe we can do even better than that. Would it be a good idea to be able to prove programs to process binary objects and extract programs from them? The way I see it, it would not be too much extra trouble (and in fact it also some things easier)\ldots\ Okay, I think that part of the collected reflection rules should also be to take propositions on the form $\forall x,y,z \in B: Q(x) \land 
y = F(x) \land z = \mathrm{ApplyProgram}(y, x) \to a_1 \land \dots \land  a_n \to P(x) \to P(z)$, 
*(Actually, this rule should probably be generalized to have $P(x) \to R(z)$ as the consequent)  
where $F$ is the program we want to test, which extracts (and possibly compiles) a program as a binary object from another binary object. Note that since $z = \mathrm{ApplyProgram}(y, x)$ has to be evaluated in the core theory of the ITP, we must have all potential antecedents defining $P$, $a_1, \ldots, a_n$, on the RHS of this antecedent. From such a proposition (proven), the additional reflection rules should then be able add a rule-creating rule, which can take a proposition on the form $Q(x)$ (where $Q$ is the same variable formula that appears in the previous proposition) and apply $F$ to $x$ %any binary object $x$ found in an atomic formula $Q(x)$ (again when an equal number of negation and no biconditionals are found as its parent nodes, and where there also is a similar rule without these restrictions for propositions with $\leftrightarrow$ instead of $\to$) 
and use the resulting binary to initialize a new program rule that applies on the ``$a_1\land \dots \land a_n \to \mathrm{LogicalConnective}(P(x))$'' formulas. The nice thing is then that when $Q = P$, and when all the antecedents, $a_1, \ldots, a_n$, are consumed in $P$, this means that for propositions on the form $P(x)$, users can use all existing program rules on $x$ continuously and even initialize new rewrite rules, without having to go back and consider the outside formula. Again, there should also be generalizations of these rules to include programs that uses several memory/file objects and can create and delete (and open and close etc.) these. The users will thus be able to dive completely into these objects, so to speak, and do computations there, only using the main application to save intermediate results and to (finally) store the obtained propositions (whereby the binary object collections are converted to corresponding file object collections and thereby saved on the hard drive). %*(The main application of course also has to be used to initialize any new rules, but this process can be hidden to the user..) 
And if $P$/$Q$ furthermore defines a class of valid states of a deductive system that reflects the outer theory, possibly with additional neat extensions, then the users can thus ``dive into'' programs that are updates, or ``child applications,'' of the main ITP.

Now, to get back to the reflection \emph{axioms}, these then ensure that there is a way to convert propositions proved within $x$, i.e.\ some binary object (or collection thereof), back out to the original propositions of the language of the ITP, which in turn means that users do not have to necessarily agree on their preferences for these child ITPs. If two groups of users for instance insist on using different child ITPs, then they can still communicate and accept the propositions of one another, since the reflection axioms ensures that there is a transformation from each child ITP to another, at least for all propositions that can be expressed in the language of main ITP as well (and where the relevant transformation has been defined by the users). 

So how should these axioms be defined?\ \ldots\ Well, now that I have thought about these additional reflection rules, %I actually think it might be 
% Hm, hvordan tænkte jeg det nu i går? Skal det mon ikke i stedet være, at man bliver ved med at vise transformations-/ekstraktions-funktioner/programmer for hver child application, og så \emph{får} man faktisk de lag, som jeg har snakket om?... Ah, ja, hvor man så faktisk starter fra aksiomernes model og så har regler til at initialisere transformations-/ekstraktions-programfunktioner til at danne nye, potentielt udvidede, reflekterede modeller..!..
I actually think they might be implemented as rules instead of axioms. (This, by the way, also means that the core theory / deductive system of the ITP will actually, as far as I can see, be a conservative extension of the underlying theory (which probably will be ZFC).) 
Therefore, I will stop calling them reflection axioms in the following, and instead just include them under the term reflection rules. 
These rules should then define a translation between a binary object and a list of propositions, as well as a certain predicate / variable formula, let us call it AreProvable$()$, that describes a class of binary objects that represents lists of exclusively provable propositions. The way to define this is naturally to start from the list of axioms and then define rules to obtain new valid proposition lists (modeled as binary objects). Note that for FOL, this would actually mean that we would need to have proposition schemata as standard part of the core theory, and probably actually in a way where users can even prove proposition schemata.\footnote{
	I have actually came to like SOL more, when writing these notes. With Henkin semantics, it has the same power as FOL with the possibility for axiom schemata and it is also more intuitive. It also fits the fact that the core deductive system of the ITP should be able to use variable formulas anyway. And additionally, when users want to express something more profound, all they need to do is to reinterpret the language back to standard SOL. So it almost seems that you get the best of both worlds this way; the ability to express profound propositions as well as having a pragmatic framework where theorems are formally applicable to real-world systems, without having, in theory, to first assume something profound. I can see a potential problem that users might be inclined to chose axioms that are more intuitive of what they want to express but less pragmatic for a mathematical framework, but this should not be a problem when we have an ITP where the axioms are already chosen beforehand. We thus do not need to consider what users might choose to work with since they already have plenty of incentive to work with the axioms already chosen for the core of the ITP (by experts). I have read something about FOL having some nice qualities, but it should not\ldots\ Well, it does not matter. There are formal ways to reinterpret these languages, given by their semantics (witch can also be reinterpreted formally), and thus transform propositions between languages, so the underlying foundation does not matter much in practice. Even theories such as ZFC can be reinterpreted into anything else, really; it is part of the formal semantics of it (since IsMemberOf$()$ (i.e.\ $\in$) can be reinterpreted wildly). So the choice really does not matter much in the end on a philosophical level; we should just choose whatever is the most practical foundation for this particular ITP. 
} 
*(No, this is not necessary after all,
%, as we do not need to define one starting proposition list for the reflected model. We can instead define a whole set of valid propositions lists to begin with and then add a rule to put to valid proposition list together as a conjunction. And since the underlying theory is constant, we can also just in practice add a rule to add propositions ...) 
as we can just make sure that proposition schemata in the reflected model cannot be transported out to the core language of the ITP. We could even make it so that proving proposition schemata is possible in the reflected model but not in the core theory, and it would be completely fine. So there are several possibilities here.)
Note also that these rules does not need to be those of the underlying theory exclusively, but can themselves be part of an extension. They can even include their own, reflected, reflection rules if this turns out to be beneficial *(it is probably not), but we will get back to that. 
The reflection rules should then apply to theorems on the form AreProvable$(x)$ where $x$ can then be translated to the corresponding list of propositions (and proposition schemata, potentially).

In addition to this, there should then be a rule that basically adds new reflected models as well as rules to ultimately pull out proven propositions from them. This can be done, I think, by being able to prove the correctness of programs to translate propositions between a reflection model and another. 
This rule should the probably require a proposition on the form $\forall x,y \in B: P(x) \land y = F(x) \to \mathrm{AreProvable}(y)$, where $P$ then becomes the predicate / variable formula that is analogous to AreProvable$()$ but for the new reflection class and where the program of $F$ is also recorded. The rule should then add a new rule to act on propositions on the form $P(x)$, where $x$ is picked out and first rewritten via $F$ before being immediately given as input to the built-in program, that extracts the actual proposition list from the standard AreProvable class. 
It seems like a good idea to also request or require a proposition on the form  $\forall x,y,z \in B: \mathrm{AreProvable}(x) \land y = G(x) \land z = F(y) \to \mathrm{AreProvable}(z)$, where $F$ is the same program function from the first proposition, to then define a reverse translation program $G$ of $F$. This could add a rule to go from a proven proposition $\mathrm{AreProvable}(x)$ to $P(y)$, where $y$ is acquired from $G(x)$. 

Well, now that I think about it, why do we not just add a rule that takes a proposition on the form $\forall x,y \in B: P(x) \land y = F(x) \to Q(y)$ for arbitrary $P$, $Q$ and $F$ and add a new rule to deduce $Q(y)$ from propositions on the form $P(x)$ where $y=F(x)$. Then we do not need the last rule I wrote about. *(Hm, do we actually need the one before that as well? No. We only need the rule to go from the AreProvable model to the core theory.) The reason why I thought of this, is that I would also like for users to be able to translate propositions directly between different reflected models, without having to use the AreProvable class as an intermediate step. So by adding this rule, we also get this nice feature. And now my idea where users can ``peel away layers'' of the child applications (i.e.\ the reflected (and extended) models (on which interactive programs are then run to rewrite them)) will thus also make more sense again, since new users can then translate their simplified models into more advanced, underlying model, which we can call the parent model or parent application, by translating the propositions over to that parent model. This translation could in some cases just be temporary, such that the user goes to the parent model in order to prove something more advanced and then return the acquired propositions back to the more simple child model afterwards. Note, by the way, that a program to translate to the AreProvable class is still required to be run whenever one wants to use the (higher-order) rules to add new rules a child model (since we need the proposition expressed in all the way back to the core language in order to use these rules).

With these rules, we already get the ability to create any (efficient) child application, as long as it can run in isolation --- in what is pretty much a sand box. But this should also be all we need for the proof construction and verification modules. Any communication with the rest of the system, the LAN and/or the internet (for instance if a user wants to search for a proposition or method (i.e.\ a program function that they need)) can be handled by other modules of the main application. One thing that we might still want, is the ability for our intermediate language (IL) to use several processor cores for when this can increase efficiency and when the ultimate output does not depend on how many cores were used in the process (and any of the randomness in general that must be introduced into the execution semantics because of this). But I would exclude this from the first versions of the ITP (to take one step at a time). Even without this, we have a pretty efficient ITP already. Of course it is advised that higher-level languages is built one top of the core IL and that the users (and I implicitly also include the programmers of the main application whenever I say ``users'' in these cases) build compiler programs for these (using the ITP). By proving the properties of these compilers, one can thus prove the semantics of the higher-level language and (potentially) all its properties. And in the same way that higher-level languages makes programming intuitively easier, this will make both building and proving the correctness of programs in the ITP much easier. 


I have wondered previously about whether the reflected model should only reflect the underlying theory itself or a, perhaps non-conservative, extension that includes reflection axioms (with less power).\footnote{
	If the think about extensions that all have the underlying theory assumed plus axioms (or rules) to extract propositions from a reflected model, we can then take the underlying theory to be the 0-order extension, and take each subsequent, $n$-order, extension to be the underlying theory (the 0-order extension) plus a reflection model that models the $(n-1)$-order extension. What I have wondered about is then whether one might use a higher-order extension instead of just the first-order one. (And my conclusion is that, no, the first-order extension is sufficient with an underlying theory as powerful as ZFC.)
}
But with an underlying theory as powerful as ZFC, there should be no reason to use such an extension for the reflection model. In fact, I do not think there is reason to even extend the reflection model at all, at least not beyond how the core deductive system of the main ITP is already extended besides the reflection rules. We might thus for simplicity choose to still use the core theory /deductive system of the main ITP instead of the underlying theory (which could very well be ZFC in either FOL of SOL) so that new users does not have to read up on the underlying theory necessarily to follow along, but only has to study the core deductive system of the ITP. But this is of course not for me to decide (and I do not think I can even predict what is best). 
%*(Well...)\\ ***

So to summarize the reflection rules (which now also includes the what I called the ``reflection axioms'' before), we have rules to rewrite binary objects within a predicate (or variable formula, but let us just call it a predicate), which can be any predicate defined by the user (assuming that the main ITP does not use typing\footnote{
	While typing should definitely be part a good ITP, it is probably a good idea to leave this for a child ITP. I thus imagine that the main ITP should actually be kept pretty simple in terms of the formula language. The only thing I can think of, that would make users want to have typing in the main ITP, is if it would make identifying trivial propositions easier, but on the other hand, it would be easy to implement a filter just by converting to a typed child ITP and then back.
}), and thus to go from $P(x)$ to $P(y)$ to put it simply, where $x,y\in B$ and where I use a capital letter $P$ to denote that the predicate is a specific, constant one. We then also have rules to go from $P(x)$ to $Q(y)$, i.e.\ to transform binary object between two constant predicates. Both of kinds of these rewrite rules uses specific program to transform the binary object, which is identified when the rules are being initiated. Note that the rewrite rules are not the reflection rules themselves; the reflection rules are the higher-order rules that is used to initiate (or initialize --- whichever is the more appropriate term) the rewrite rules. We also have rules, initiated by a higher-order rule once again, to compile a program $x$ to another one $y$, $x,y \in B$, whenever $x$ fulfills some predicate $Q$, and then add a new rewrite rule to rewrite formulas on the form $P(a)$ to $P(b)$, $a,b\in B$. Well, come think of it, we should probably generalize this rule so we can rewrite terms on the form $P(a)$ to terms on the form $R(b)$, i.e.\ we should not require that the predicate is necessarily the same for the resulting rewrite rules. Note that this is thus a higher-order rule to generate higher-order rules to generate rewrite rules (from compiled programs). In the first step, we extract and compile the relevant program and then, in the second step, we use it to make a new rewrite rule. Then we also have what I initially, when writing these notes, imagined as axioms, namely the rules to extract/export propositions from/to a special AreProvable class. The rule for extraction should thus work on propositions on the form AreProvable$(x)$ and extract a list of propositions to add to the current workspace of proposition in the core ITP. And the reverse rule, which I might actually have forgot to mention in the above text, come to think of it, should give the ability to select a list of propositions in the workspace of the core ITP, i.e.\ all the propositions proven so far (and not since discarded from said workspace by the user / the proof script) in the proof script, as well as ones pulled forward from previous proposition libraries, in order to have a binary object generated from the AreProvable class that represents this proposition list. The AreProvable class should reflect the core theory and possibly extend it a little. The only kind of extension so far that I have thought about which might be beneficial for the AreProvable (reflected) model to have, is if proposition schemata is not included in the core theory itself, except for the axioms. In this case, I would argue that the reflected model should implement proposition schemata, such that the users can even prove new proposition schemata (as well as generate individual propositions from them). On the other hand, if SOL is used, which I have actually come to think might be a good idea, then this might not even be an issue in the first place. This sums up my thoughts on the reflection rules. In conclusion, these are all the reflection rules I can come up with that I imagine an ITP like this should have. 


\subsubsection{Substitution of rules in the proof script and recovery of failed programs}
I have not really talked about these topics yet (since most of what I am about to write, I have come up with while writing these notes (around this point in the text)). There are a couple of problems that we have yet to deal with, which are solved the same solution, namely to be able to substitute program rules in the proof scripts\footnote{
	Note, by the way, that I just use the term `proof script' because I do not know what else I should call it. There might thus be a better term for it that I just have not thought of yet.
} for other program rules, which should be equivalent in output but which can use additional binary objects as input. Specifically these program rules can use the output of the former rules, and even if these programs failed originally, which is what is used for recovery. It can also be used to transform parts of a proof that can be optimized after construction, either because computation power went into other things, such as preparing objects for rendering for any interactive program, or because the proof procedure uses an algorithm (for instance one that involves a search of some kind) that can be replaced by a quicker algorithm on a non-deterministic machine, or, equivalently, if there is an oracle present that can tell us whatever search result we need in constant time. In practice, this ``oracle'' will then be the output objects left over from the initial rule/algorithm used, which is then being replaced. This is by the way why I think an ITP such as this one can work fine, where the proof construction and the proof verification in principle uses the same script, and thus where there are no mandatory transformation from the list of actions taken by the user and the resulting proof script that can be send for verification; they can in fact be the same. By making sure that the user still have the possibility to edit and optimize the proof afterwards, I see no real reason why we should make such a transformation mandatory. Well, of course, one argument would be that it would mean that the ITP could have less restrictions on its proof construction module; if the proof is verified anyway, why put any sharp restrictions on the construction process? My answer to this is then that it actually does not cost us anything to do this. If users want to, they can still just use a class of programs with less restrictions on them to generate their own analogue to a proof script and then create a verification program to verify these custom proof scripts and get the actual propositions as outputs of this custom verification program. And the whole point of doing it this way is that the verification process can then also be optimized by the user-made programs, which is a very important part of it all. One could then argue, that it is on the other hand desirable to have as simple a verification procedure as possible in order to be able to trust it more, which is normally the case, but the whole point of this ITP is to allow for varying degrees of trust/certainty when it comes to different proof procedures. The point is to then allow for the potential for more optimized proof processes, both when it comes to the construction process but also especially when it comes to the actual verification process, and then instead give the users the ability to keep track of trust levels (/ acceptance levels / probability levels etc.) for all propositions, depending on which procedures were used for the proof (which I will get back to in the next section). Okay, nice to have this whole point underscored. Let us get back to the rule substitutions.

When it comes to rewriting proofs, one can also build programs to rewrite and optimize whole proofs. And of course, these programs can be built within the ITP itself. Users can thus make their own rewrite procedures before sharing proofs with one another. But this is definitely a more complicated process than if users can just rewrite a string of rules with equivalent ones and if the equivalence is just checked immediately by the ITP. And since I imagine that most users will work in a child ITP, i.e.\ an interactive program that reflects the core deductive system, this both means that being able to rewrite such program rules will cover a lot of the user's needs, and it also means that such rewriting will be needed very frequently. So even though the users will be able to rewrite proofs on their own as separate processes, I can imagine that this feature to do it immediately while constructing and editing a proof (and not afterwards) will be heavily used. 

For rules in the middle of a proof script being built, the ITP should as mentioned make sure that the output is the same for the substituted rules. The input objects does not need to be the same, except of course for input that is also part of the output of the rules. By `output,' I mean any binary object that is referred to by the formula being added or rewritten by the rule. For rules at the end of the proof script, we technically do not need this restriction. Therefore, it could be left as an optional choice whether the users wants to check for equivalency or not for a certain rewrite procedures when it comes to these end-of-script rewrite procedures. For recovery programs in particular, it does not make sense to require equivalency, as the output of a failed program can be seen as random for most intends and purposes.

If we expected the users to mainly use relatively short programs to run algorithms on different data structures and make computations on these, one might not see the feature of being able to create recovery programs as a very important feature. But since I imagine that most work will actually be done inside the ``child ITPs,'' which are interactive programs that implements a different ITP environment, I therefore find this feature very important. It means that the child ITP can implement there own way of continuously saving the progress (both automatically and manually) to memory, should the interactive program itself fail, as well as to the hard drive, should the main application fail as well. The child ITPs can thus implement their own analogue to the proof script of the main application (which should maybe also be saved to a recovery file automatically, at least if the user can then turn this functionality on and off). 

I should mention, that in the way I imagine the rules of the main ITP to work, these should refer to the formula nodes they act on, as well as any additional input of course, when they are written as instructions in the proof script. If I were to implement the ITP, I would even make sure that all relevant data is kept in memory for at long as possible, so that the user can undo rule actions and redo them again a long way back without computations needing to be redone. This way, the users should thus be able to undo and redo while editing in much the same way as with conventional text editors. But with this, we might run into a problem for rules that rewrite big binary objects; if the user uses a lot of these in a row, it might very well strain the memory if the data needs to be recorded at every proof step. But with the possibility to make recovery programs of different sorts, this might still actually be an okay way to implement the main ITP (I of course cannot say for sure, but I think so). The reason is that if an interactive program can save data while running this way, it can also save however many intermediate states one want it to. And if we thus expect users to mainly work within child ITPs anyway, they will have to design their own recovery (and undo and redo) procedures anyway. I thus think we should be able to expect the users who design the child ITPs to handle how intermediate states of large binary objects are saved themselves by handling. And since a whole interactive program run in a single `rule' (i.e.\ a single proof script instruction of the main proof script), this means the user (or rather the designer of said program) gets to choose however many intermediate states are included in the output (both upon successful of failed execution). This is why it might be okay to simply just have the main ITP save all intermediate binary objects between rules, even large ones, as we should be able to expect the userbase to design rules to apply on large such objects in a way where only one rule is used per object (such that we only \emph{have} to save the initial and final state).\footnote{
	Sorry, if this explanation got a bit convoluted (even more so than usual). I hope a potential reader (other than me), will get the point. (\ldots as can be said with all my convoluted explanations.)
} If this will be the case, i.e.\ that all rules save and refer to their input formulas as well as their output formulas, where these then furthermore hold references to any input or output objects respectively, then there is a pretty straight-forward way to implement the rewrite procedures, including the special recovery procedures. For failed executions of program, we can just define a convention for how the ITP outputs the ``random ouput objects'' from such a failure where these objects are included in a non-restrictive sub-formula, e.g.\ a formula that basically just concludes that ``this object is of the binary object class.'' Then when users design the recovery and rewrite procedures, they should simply be given the ability to first design how and what output objects are extracted from the output formula (e.g.\ by defining a parsing of that formula, perhaps simply be writing a REGEX-like expression), and then design a procedure for choosing what program should run on the extracted binary objects. So the main ITP should give the users the ability to design these rewrite/recovery procedures, and the way this is done can be completely different from how other rules are designed. In fact, there does not need to any real restrictions on these procedures, and they can be seen as on a layer above the deductive system. One could thus even imagine several different version of the ``main ITP,'' which are completely compatible but nonetheless implement different ways to design rewrite\footnote{
	By the way, maybe ``rewrite procedures'' is a bad choice to call them. I see now how it might be confusing, since we then have ``rewrite rules'' to rewrite objects, ``rewrite procedures'' to \emph{rewrite rules} (as a verb, not a noun). This might indeed be confusing. *(Maybe `rule substitution procedures' would be a decent choice.)
}/recovery procedures. This is possible if the check for correctness is simply done by the main application itself each time. And just requiring the main application to do so each time seems to me like a reasonable way to do it since it has to do the same computation anyway (so why not just add a quick check for equivalency on top, and then the users do not have to prove correctness of rewrite procedures at all). *(Oh, regarding whether the main ITP should save each intermediate binary object between rules, maybe a better solution would be to make it optional whether an intermediate state is saved between two rules. If a user chooses not to do so, then he or she cannot undo back to that particular intermediate state, but can go further back. One can thus see it as a way of lumping rules together. In order to separate them, the proof construction module will just have to redo the computation to retain the intermediate state. It also just occurred to me, that trivial binary objects, e.g.\ if the user initializes a zero-filled data structure for subsequent computation, should probably not be saved to memory. Expanding this idea, the main ITP might even have a standard way to decide which intermediate states are kept, and for how long (!), which the user can then manually correct at any time. The ITP might even time the execution of programs in order to lump rules together in a reasonable way. Also, rules that operate on the same objects might be prioritized for lumping together as well. Yes. So while it maybe does not need to be the aim of the first version of the ITP, I definitely think that all this would be a good idea to have at some stage.)

So by having recovery procedures, the users will get to implement their own way of generating a proof script of any desired format for their child ITPs and will be able to control how frequently intermediate states are saved and when the are discarded again and so on. This also means that the proof script does not need to be especially optimized, just like with most other things for the core deductive system; it does not matter if it is not very optimized since these optimized updates are supposed to happen with the child ITPs anyway. In fact, it is probably better to aim for simplicity in general rather than optimization with the main ITP. I mentioned above that interactive programs should use an abstraction of user input as memory (or file) objects. But for safety reasons, as well as to reduce the memory usage, the programs should not output these input objects as a rule of thumb. So these should not generally be made recoverable, neither as input objects nor as output objects. Instead it is probably better to advise users to always convert the user input to a (strongly reduced) script, namely a child ITP / child application script if one will. The input should thus be translated into the (correct) actions it resulted in before the object should be used for recovery. 

I have a few things yet to mention in this section. One quick thing, that I have just realized, is that when the main ITP fails for some reason or is interrupted while a program rule is running, it should be able to recover all file objects of the program and then add the conclusion that I mentioned above, namely that ``following objects are of the binary object class.'' 
Another important thing I will mention, that is very easy to explain now that I have talked about lumping rules together just above, is that users should of course also be able to rewrite a whole string of rules with another one (equivalent, of course). And of course the number of rules should not have to match the substituted ones, so for instance, several rules could be replaced by even just one. 
%The last thing I would like to mention is that I think the ITP, in some later version of it, should make it possible to turn on automatic guesses of the next rule, such that users can make it so that certain rules, if the return a certain class of output, will be immediately and automatically followed by the main ITP trying one or (perhaps) several other rules. The rules tried can be depending on the output of the former rule, and whether to keep the tied rule, as well as whether to halt the process or go on and try a different rule, can depend on the output of the tried rule. This would be beneficial if a user want automated assistance with a proof that, and if that for some reason is easier... No this does not make sense; the user should just go to a reflected model and program the same (reflected) procedure within that.

%(Shouldn't I also mention rewriting libraries in this section?) \\ ***  *(Nope.)

This pretty much concludes this section on why such restricted procedures to rewrite rules under the construction of a proof is almost a necessary feature of the ITP application. I last point I want to mention, which relates to rewriting rules, ... *[(11.03.21) Here, I wanted to write about how it would be a good idea for an ITP to make it easy to simply remove part of a proof, such that the proof could be replayed by others, but where they then have to complete the same construction process as the original author did for parts of the proof. This could be a good way to teach math and at the same time get the students trained in using the ITP to construct proofs. And it would be nice, if the teacher only have to make simple changes to the proof, removing certain inputs from the proof script, such that the teacher also does not have to learn a lot of extra stuff in order to be able to create these interactive guided-proof programs.]

% recovery: using predicates to handle the "random output." (tjek)
% Husk: How recovery works. (tjek)
% Nævn igen at (hovedsagen er at), brugere så kan implementere deres egne PS'er for child ITP'erne. (tjek)
% Nævn det med at en streng af rules også bør kunne blive omskrevet. *(Husk: grundene til at bruge flere regler på samme object (fordi de er tilgængelige separat og fordi man gemmer og afbryder som eksempler), og evt. om at lave super-programmer, der altid kan samle sådanne program-regler til en regel efterfølgende.) (Meget af dette behøves ikke længere, nu hvor jeg allerede har talt om at "lump'e regler sammen," så tjek)
% Uh, og hvad med det med kørsel af folks beviser med fjernet input? Det skal vel også med her.

% ITP using output of programs to choose and try some next rule(s) automatically. (no, not needed)
% Rewriting at the end of proof does not necessarily need restrictions. (tjek)
% Recocvery is important mostly due to child applications. (tjek)
% Åh, jeg skal også have en refleksionsregel til at hive fil-/memory-objekt-pointers ind i reflektsionsmodeller..!! Ja, det har jeg sjovt nok ikke tænkt på -- kun at få dem med i programmer. ... Hov, nej, det giver ikke rigtigt mening alligevel.. Nej, never mind. (nvm: tjek)
% Det med omskrivning af input-filer. (tjek)



%Note that the child applications so far just records all input of the user and saves them to a file. But this is both wasteful and there is also a risk that a user might accidentally type sensitive information into proofs and not realize this. It would be a good thing if the main application does not record all input at all times and that the input stream halts if the child application GUI is minimized or is otherwise not in focus. But even so, there is still plenty of reason why we would want the user input streams to be rewritten (as compact and sanitized versions) after the execution and before being saved to a permanent file. We thus need a way to transform the proof scripts. 



%
%It would even be a good idea if this process is not just automatic, but where the users also get the ability to take part of the process of restructuring and reducing proofs. If this process can even happen inside a child application, then that is all the more better. And since we are already in the habit of solving such problems by introducing a way where the users can build and prove their own procedures for such rigorous transformations, 

%why not add a (higher-order) rule that can be used to add pseudo-rules to transform the proof script directly, along with their input files? I call them pseudo-rules because these will never be visible in the final proof script (except potentially in comment lines) and is not recognized by the verification module for instance. (This is unlike the higher-order rule itself, which does appear in the proof script before it is used.) 





\subsubsection{Certificates, meta-antecedents and proposition libraries}
I have talked about above how it was important to be able to have the ITP recognize certain certificates agrees upon by the user community (or any sub-community for that matter) in order to accept propositions, or, as I called them, `methods,' even when these are not yet proven (or if the user does not want to (download and) repeat the check of the proof). These methods are then of course our program rules, which, because of our reflection rules, we can even use for proof steps. So for both propositions and `methods,' the relevant certificates of course relates to propositions; for `methods,' that proposition just speaks about a quality of a program/algorithm, which is then meant to be used for a rewrite rule. In this section, I will describe how these certificates are used, as well as how the ITP checks various conditions before running a proof script, or before initializing one for construction. This will also relate to the acceptance level / probability distribution (or what we should call it) that I mentioned above as well, so we will also get into that. We will also get more into how propositions libraries are loaded on so on, since it also relates to the certificates. 

The main solution I will present for all this is to use what I will call \emph{meta-propositions}. ... *[(11.03.21) Here, I wanted to write about what I had in mind at this point about what will now be the meta-assumptions (though we can still use the term meta-propositions as well). This was a bit more tricky at that point, before I got the idea to a bit different approach to the main-ITP, where this is more powerful --- powerful enough to describe itself fully --- and where the ``main ITP'' that I am describing here will actually be a kind of ``child application'' itself instead, which is by the way why I am ending these particular ITP notes before they are done completely. So the points that I intended to write for this section will be both a bit more central to my new version of the idea, and will be more trivial explain (and to get right).]
% ...MP'er taler så om refleksions-programmer, og om beviste... Hm... Hvordan gør man, skal brugere bare enes om biblioteker, eller...? ..For brugerne kommer jo (vel) til at bruge fil-biblioteker meget... Ja, men de leveres jo med som input-filer. Så hvis der bare var en måde, at... Jeg skulle til at sige sammensætte biblioteker, men hvad med bare at kunne udtrække props fra filer i stedet?..! ..Det har jeg jo også lige tænkt på angående certifikaterne... Ja, så det virker helt klart som om, at vejen frem findes her et sted.. Ja, det skal vel nærmest være standard måden, at loade propositioner på, og så skal der bare være et standard program til at loade dem fra simple prop-filer. Det ville så være smart, hvis man bare kunne sige, at hvert program brugte et vist filnavns-format (og også en meta-antecedent), og holdte styr på (ITP'en gjorde), hvilke propositioner, der allerede er vist (hvilket ikke nødvendigvis er så smart, men det ville være fedt, hvis det var).. For det ville være smart, hvis brugerne så bare kunne hente nye filer af det format, og ITP'en så bare kunne give et checkmark på den fil... Uh-ha, der er lige nogle ting, jeg skal tænke over, for søgning på propositioner bliver jo et problem så, som main-programmørerne ville skulle tackle, og man vil jo i øvrigt gerne have, at man kan lede efter (query'e) propositioner direkte fra en sky eller fra internetlokationer. Så hvad gør man lige her?... Er svaret, at ITP'en skal have en abstrakt database, eller..? ..En todelt database... Så er problemet bare lidt... Hm, eller hvad med bare at holde styr på alle biblioteksfiler med diverse formater og tilhørende udtrækningsprogrammer, og så bare implementere noget søgning i et lag over... Hm, men så bliver det nok igen et rod med alle de filer.. Hm, hvad med at man tilføjer et sammensætningsprogram til et biblioteksformat, så man også får en måde at fusionere biblioteker? Der lyder da som en meget god idé... Uh, man kunne også samtidigt så have biblioteksklasser, således at biblioteker ikke er konstante, men kan udvides af brugernes egne patch-filer, men hvor hver klasse så har en bestemt karakter, ift. hvad deres propositioner kan bruges til. ... Uh, i øvrigt: Det med at bruge programmer til udtrækning gør jo også, at der er mulighed for at udbygge algoritmer for, hvordan korrolarer kan vises under søgningen for dem, i stedet for at man skal gemme alt direkte, eller at man skal opdatere det i main ITP'en (hvilket er mere besværligt, især når brugere har forskellige præferencer, de gerne vil have tingene på forskellige måder). Men ja, og så er det også vigtigt, at brugere kan dele propositioner på en smart måde. Og jeg vil altså så gerne have, at der skal være propositions query-regler, hvor det så er tænkt, at disse skal behandles forkelligt af forskellige brugeres ITP-applikationer (alt efter hvad de har af biblioteker og udtrækningsprogrammer). Og hvis en proposition så ikke findes, så kunne der bare komme en tilsvarende antecedent (-header) på de beviste propositioner (og man kunne så endda have en måde at fjerne antecedenter direkte i prop-filer, så man kan gøre dette uden at gentage beviser, især fordi disse godt kan være gamle og/eller eksterne, således at det er særligt besværligt at skulle gentage beviset). Så vi skal have programmer til at udtrække propositioner og programmer til at sammensætte og til at deducere (og udføre) ændringer i dem (som f.eks. elimination af antecedenter). Og så er det jo klart, at kompilering og beviser omkring refleksion så kræver, at alle ikke-godkendte (meta-)antecedenter er elimineret. Når et PS har kørt færdigt, hvor der så er nogle burde-have-været-der antecedenter sat på, eller hvis PS'et fejler pga. at det prøver at vise noget om korrektheden af refleksionsprogrammer ved brug af en ikke-fundet proposition, så kunne ITP'en jo så kunne lægge op til, at den kunne søge på nettet (på webbet eller i en sky f.eks.). Hm, og skal internet-søge-regler under konstruktion så substitueres... Hm.. Tja, eller det burde vel mere være en lidt anden form for PS-ændring, hvor man så først downloader det relevante bibliotek/den relevante patch, hvorefter ITP'en så sørger for at ændre PS-headeren, hvorved den/de relevante proposition(er) bliver tilføjet til workspace-konjunktionen, som er en abstrakt liste over propositioner, som PS'et kan bruge af med det samme, uden at lave yderligere søge-regler. De normale søgeregler handler så om at tilføje til denne workspace-liste (som ikke er en faktisk liste i form af et matematisk objekt, men en liste hvormed brugeren kan eleminere antecedenter m.m. og/eller udvide arbejdsformulare-konjunktionen med, og hvor de også kan hive propositionerne ind i en reflekteret... Nå jo, det er faktisk helt fint bare at sige, at workspace-konjunktionen er et matematisk objekt, for i praksis vil man alligevel for det meste gerne have propositioner fra den hevet ind i reflekterede modeller, så det gør ikke noget, at der også bare kan være en automatisk ordning af arbejdspropositionerne, således at man kan generere et binært objekt fra en AreProvable-agtig klasse direkte..)..  Ja, så den resulterende header skal vel egentligt bare initialisere start formularen, hvor der så kan være referencer til memory-objekter, der indeholder propositioner. Jeg skal så lige bl.a. tænke over, om det går, hvis man så kan ændre start-formularen inde midt i PS'et (med søge-regler) (hm, det skal det gerne), og jeg skal i øvrigt også lige tænke lidt mere over, hvordan ITP'en holder styr på alle definitionerne...
% Jeg skal huske at nævne, at renderingsmodulet gerne må være simpelt, for så kan brugere også vurdere hvor fornuftige diverse grafiske representationer er, og man kan i det hele taget gøre renderingen mere modulær. Og endnu mere centralt for disse tanker, så skal jeg lige sørge for at tænke over (som allerede antydet), hvordan definitionerne bliver i god del af main-systemet. Desuden kunne man også overveje at udvide refleksionsreglerne lidt, så de kunne have to inputs, f.eks. en definitionsfil og en arbejdsformular-fil... Tja.. Det kan jeg lige overveje senere. Men så main-ITP'en skal have en måde at definere prædikater på, inklusiv selvfølgelig de prædikater, som den så også gemmer refleksionsregler omkring. Hvordan skal brugere dele disse imellem sig? Hm, og den reflekterede model skal også kunne erklære definitioner, så jeg tænker at definitioner bare skal være på lige fod med propositioner, og hvor der såldes er regler, både i main og i den reflekterede model, for at erklære definitionerne samt for at bruge dem til at omskrive andre formularer med dem (frem og tilbage). Og så må brugerne vel så bare selv sørge for opdele biblioteksklasser, så definitioner hører til de passende biblioteker, og hvor man også eventuelt deler dem op, så der er rene definitions-biblioteker, hvis man f.eks. vil det. Cool, så langt så godt. Tjo, men hvordan gemmes de så af main-ITP'en? Skulle man ikke gøre det sådan... Jo, det skulle man: Man skal gøre det sådan, at der bliver forskel på at verificere propositioner og på at verificere definitioner og, mindst ligeså vigtigt, reflektions-prædikater og -programmer, som man gerne vil have at sine main-ITP gemmer i sit eget interne lager og gør sådan, at de kan bruges til næste bevis. Hm, uh, jeg har jo snakket om at bruge DLLs; det kan jo så passende være en del af program-defineringsprocessen, at man først beviser korrektheden, og så kører en initialisationsproces efterfølgende af programmet, hvorved ITP'en så passende lige kan genstarte, så man derfra kan bruge sit nye DLL. Og (grunden til mit "uh") så kunne man jo i øvrigt måske implementere, at man kan få lov at vise programmer og bruge dem dynamisk (og hvor ITP'en så bare kan "glemme" dem igen bagefter (så at de altså bare hører til beviset)), ved at bruge sit OS's exec-funktion til at forke et barn, der så udfører programmet (altså efter kompilering). Det kunne da være lidt smart. Det kræver nok så, i hvert fald hvis man vil bruge den løsning, at man definere en funktion i IL'et til at forke et barn med en skrive-pipe til forælder-processen, hvor forælderen (som altså er det oprindelige program) så venter på at pipen sender EOF, hvorved forælderen så har resultatet... Tja, eller man kunne bare sørge for på en anden måde, at forælderen holder til barnet sender signal om, at det er færdigt, hvorved forælderen kan forsætte med antagelse om, at nu er pågældende memory-/fil-objekt omskrevet færdigt. Kunne som sagt være meget nice, men tilbage til det forrige om, at selvfølgelig skal man køre DLL-initialiserings- og/eller prædikatdefinitions-scripts for sig. Så kan man nemlig også altid køre andre brugeres beviser kvit og frit, hvis disse ikke er disignede til at lave permanente definitioner; for selv hvis det ikke bringer noget, så glemmer ITP'en det bare igen (på nær det nye lille verifikationsflag på pågældende PS med tilhørende biblioteksfiler, som verifikationsprocessen forventes at sætte). Hm, men det bringer mig så tilbage til: Hvordan skal ITP'en så gemme de viste propositioner?... Hm, skal alle PSs have et udtræknings-program tilknyttet sig? Tja, eller også kunne de have et slut-format... eller rettere slut-formater tilknyttet sig, som siger hvilke formater, de endelige output-filer er i... Tja, men det afhænger vel bare af den endelige proposition. Pointen er så bare, om ikke ITP'en så kunne sørge for, at visse output-filer automatisk bliver transformeret til et andet format, hvis brugeren ønsker dette. Jo, men det kan vel også gøres meget nemt; det må kunne gøres med en slags post-processer, der analyserer outputtet og så konverterer, eventuelt ved at skrive ekstra linjer til pågældende PS samtidigt med... Tja, eller ikke. Så i bund og grund bør man så bare have mulighed for at fortælle sin ITP, at den automatisk skal omdanne visse $P(x)$-formularer, ved at køre en passende $P(x) \to Q(y)$-regel. Okay, men skal ITP'en så ikke bare gemme pågældende output-proposition i sit interne lager? Jo, det skal den vel... Ja.. Og i starten kan der så gemmes lange konjunktioner med vigtige sætninger for at komme i gang med refleksionsmodellerne, hvorefter biblioteker kan gemmes mere som filer. Hm, men så får jeg stadig lidt problemer med, hvor godt der så (ikke) kan holdes styr på alle filerne.. Kunne man gøre noget med, at ITP'en godt kan transformere sine egne gemte sætninger? Så man eksempelvis kan flytte, omskrive, sammensætte og patche biblioteker?... Uh, og fordi man jo endda kan ændre på hele mapper ad gangen, så kunne man... hvis man kan lave beviser, hvor inputtet ikke gemmes... så kunne man omstrukturere hele mapper. Uh, skulle man så have endnu et særegent modul til at lave beviser, der kan ændre de gemte ITP-sætninger og samtidigt have lov at omstrukturere mappen (selvfølgelig med passende advarsler om at der bør tages back-up, samt en indstilling til automatisk at lave en backup før sådan en proces)? Jo, simpelthen. Mange gode idéer, jeg har fået i dag (d. (04.03.21)). Holder lige en pause, inden jeg muligvis fortsætter.
% Kom på nogle flere tanker i går. Jeg tænker lidt, om man måske skulle inkludere PS-filerne, og gøre lidt som for substitutionsreglerne for dette biblioteksomskrivningsmodul.. Og så tænke jeg også i de baner, om ikke man så skulle involvere automatiske opdateringer i det også.. Og også i øvrigt, at man måske kunne lave gendannelsesprocedurer også.. Hm, skal lige tænke lidt mere... Tja, det virker egentligt ret fornuftigt, men jeg skal lige overveje, hvordan man bruger internettet så (et ekstra modul? IL-instruktioner til at generere "random" input filer, der så i virkeligheden kommer fra nettet?...), og hvor meget brugerne egentligt kommer til at bruge det (f.eks. hvis nu de primært programmerer, hvad mange sikkert vil gøre)... Tja, men uanset hvad, så er det handler det jo bare om... og man kan endda sige, at brugerne helt selv sætter restriktionerne for biblioteks -omskrivnings og -gendannelses procedurerne... og så handler det bare om at sige, at brugere kan bruge ITP'en til at verificere egenskaber ved de opdaterings- og patching-procedurer, som brugeren benytter. Ja, så der er faktisk ikke noget at rafle om der. :) Uh, ville det egentligt ikke være smart at gøre det samme for substitueringsreglerne; nemlig at brugerne selv kan sætte restriktionerne? Jo, hvorfor ikke; for jeg mener at jeg alligevel kom frem til, at de med fordel kunne lave programmerne i selve ITP'en... ja, det kan de jo.. Ja, og så behøver man ikke nødvendigvis at køre programmerne igen, hvis man nu ved at man altid kan stole på en given omskrivningsprocedure. Nice nok altså..! Og også god idé, den med at generere "tilfældige" input-filer fra nettet. Så kan det rigtigt nok ske via et internet-query-modul, og så kan API'en til det modul bare holdes helt frit *(dvs. man sender bare nogle bytes og så dannes der en "tilfældig" fil som resultat). I starten kan modulet så bare bruge www-URL'er osv. (dog med sikkerhed i tankerne, så man ikke ufrivildigt broadcaster lige præcis, hvilke sætninger man leder efter, til omverdnen, men det giver lidt sig selv). Og ITP'en kan så altid få opdateret sit internet-modul løbende, altefter hvordan community'et udvikler sig, og hvilke nogle services som det, eller sub-communities, tilbyder (men fint stadig at have mulighed for at hente instanser fra en hvilken-som-helst URL, hvis brugeren virkeligt vil). Cool.. Og så tænkte jeg, at hentning af nye propositioner så kunne ske ved... Tja, der er flere måder at gøre det på. Man kunne nemlig, når vi når langt med teknologien, lave bib-omskrivningsprogrammer, der holder øje med, hvad der er blevet benyttet og efterspurgt i PSs, og så bruger denne information til at opdatere brugerens bibliotek (ved at hive patches ned fra nettet) på en måde, så det passer til brugerens behov. Men ellers kan man bare hente det bibliotek, man skal bruge, når man skal bruge det, og så efterfølgende køre et program, der ser på hvilke propositioner, man brugte fra det downloadede bibliotek (som godt f.eks. kan bestå af bare én proposition (nemlig den man søgte efter) osv.), og så sørger for at sammensætte dette bibliotek med brugerens eksisterende (men ikke nødvendigvis sammensmelte; det kan godt bare tilføjes som det er), på en måde, der ikke nødvendigvis bevarer de propositioner fra biblioteket, som brugeren ikke benyttede. Når jeg siger at API'en for internet-modulet skal holdes frit, så er det self. kun set med kerne-sprogets øjne. På sammen måde dom for render-modulet, og det rigtigt smart stadig at definere en API selv (ved antagelser), således at programmer kan vises at overholde properties, givet brugerenes antagelser om API'en. .. Okay, så et grundlæggende princip i alt det her bliver lidt, at alle hjælpe-modulerne omkring konstruktions-verifikations-(kerne-)modulerne, selv i sidste ende skal være programmer, hvor brugerne bestemmer deres restriktioner, og det man får ud af dette er så, at de nemt kan opdateres; opdateringer (der bruger) certifikater kan så hurtigere rulles ud, og med stor mulighed for at preferencer har lov til at variere, hvilket nemlig er fordelinde ved at opdatere via matematik og ITP'en selv, frem for bare at bruge f.eks. Git eller noget i den stil.
%Okay, men jeg bliver stadig nødt til at tænke over, hvordan man sørger for at brugere ikke ender med at få svært ved at dele deres proof scripts med hinanden, hvis de bliver for specialicerede til brugergruppers behov uden at bevare kompatibilitet... Tjo, men ved brug af certifikater, så kan en bruger jo bare sørge for at levere udtrækningsprogrammet med, hvis der er behov, for certifikat-antagelsen behøver jo ikke at sige, at propositionslisten certifikatet refererer til, behøver at være bevaret i samme format i mappen; man kan godt tillade transformationer først. Hm, så skal man bare passe lidt mere på med formuleringen, for man kan så ikke bare sige, at alle transformationer er gyldige. Men man kan så bare sige, "alle de tranformationer der kan gøres som en del af de proof scripts, der kommer til at findes i mappen." Ja, og så bør brugere egentligt kunne dele propositioner fint. Og programmer kan jo bare deles i et IL (eller højere-niveau-sprog for den sags skylds). Hm, men hvad så med query-regler i PS'et..? Hm, kunne man gøre API'en fri på samme måde her, og så sige, at vis opslags-modulet ikke returnere sandt, så kan propositionen føres på arbejds-propositionslistens antecedent-header, hvorved man så bare ikke kan bruge viste propositioner til at initialisere refleksions programmer herefter. Så kunne man så bare fastholde strenge restriktioner for opslagsmodulet... tja... Og min tanke er så også, at programmet i opslagsmodulet så gerne må prioritere at kigge i en tilhørende fil til PS'et for at lede efter en opskrift for, hvordan man finder eller beviser propositionen. Så kan brugere dermed generere og dele hjælpe-filer, som kan sendes med PS'erne, og som kan bruges til at hjælpe med opslagene (bl.a. give hint til, hvordan man skal søge, og til hvis der nu skal gøres noget logisk deducering før man opnår præcis den proposition). Disse hjælpe-filer kan så også give en liste over de nødvendige (gerne nogenlunde konventionelle) biblioteker, hvormed PS'et burde lykkes, og for hvilke hints'ene så kan rettes imod. Hm, jeg tænker lidt, om man ikke kan lave noget tilsvarende til alt dette inde i en reflekteret model bare, men... Men nej... Hm, og med certifikater og meta-antecedenter kan man vel godt putte strenge restriktioner på opslagsmodulet, for... Hm.. Jeg føler, at jeg bare lige mangler at kunne slå helt fast, at brugere altid nemt kan lave verifikationer på den mere normale vis, så... Hm, tja, det jeg prøver at sige er vel, at man altid bør kunne gå tilbage til et mere simpelt opslagsmodul, eller... Ah, nu tror jeg, jeg fandt tråden igen. Det er vigtigt, at brugere altid bevarer beviset, der viser korrektheden af opslagsmodulet ud fra de første certifikat-meta-antagelser, som brugeren har startet med at gøre (og dette bevis skal så selvsagt bruge et opslagsmodul, der kun kan bruge de certificerede antagelser fra start, samt de propositioner, beviset selv udleder, men ikke andet). Og herfra kan omskrivningsmodulet så lave alle de omskrivninger af mappen, som brugeren tillader, men må ikke ændre i de beviser, som opslagsmodelets korrekthedsbevis brugte (så alle disse indledende beviser skal bevares). For ellers ender brugerne i en mærkelig situation, hvor de ikke længere rigtigt kan konkludere noget ud fra deres beviser, hvor hvem siger, at der ikke fandet en fejl, da opslagsprogrammet vistes; "hvor er beviset henne?" Det giver selvfølgelig i det hele taget mening at gemme beviser, men hvis man også gerne vil omskrive beviser løbende, så er det jo vigtigt, at man kan stole på den omskrivning, og at man ikke mister beviset for selve gyldigheden af omskrivningerne... Hm, ja og måske kunne også kræve at beviset for omskrivningsprogrammerne ikke selv omskrives..? Hvad er nødvendigt her, og hvad er muligt (og hvad er bedst)? Tja, man kan jo godt i princippet omskrive beviset for omskrivningsmodulet, for hvis man omskriver korrekt er det jo ligesom om, at man bare lavede det sådan i første omgang. Ja, og det er så forskellen ift. opslagsmodulet: Hvis man ændrer reglerne for, hvordan tingene slås op, så ændrer det jo også hvad man kan med et bevis, og så kan man ende ud i en situation, hvor folk kan hævde, at dine beviser for opslagsmodulet kun virker, fordi du bruger det fejlagtige opslagsmodul til at vise sig selv. Ja, det er det! Det er det, man gerne vil undgå. Dejligt at jeg lige fandt frem til, hvad min bekymring her skyldtes. Ja, så hvor det godt kunne ske at det ville være fornuftigt nok at tillade at oskrivningsprogrammer omskriver beviset for dem selv, så må man ikke bruge opslagsprogrammer såvel som reflesionsprogrammer før man har vist dem, og må så ikke begynde at bruge dem til beviser af sig selv bagefter. 
% Jeg summede lidt videre på min fridag i går (i dag er det d. (07.03.21) btw), og der er altså lige nogle ting jeg skal overveje. Der er nogle ting, der ikke er helt skarpe for mig i mit hovedet, så der skal altså lige lidt tænkning til. Jeg har nogle stikordsnoter, som jeg lige kan komme ind på senere, men nu vil jeg gerne lige overveje, hvordan omskrivning-opslag-og meta-antecedent-sammenspillet skal være. Jeg tænke jo "omskrivning" som omskrivning af biblioteker, men nu bliver det ligeså meget af proof scripts. Men når opslags-modulet ændres, så kan det vel også hænge sammen med en ændring bibliotekerne? Ja, det bliver altså en smule cirkulært, for man vil jo gerne omskrive PSs og sætningslister, når man omskriver filer (...men skal alle filtyper omskrives eller kan man dele det op?), og det kan man vel først rigtigt, idet man ændrer opslagsalgoritmen... eller en sådan ændring må i hvert fald gerne kunne medføre en ændring af filerne... Hm, men går det ikke at holde sig til, at omskrivning foregår som en slags regelsubstitution af hele (eller dele af) PS-og-bib-filerne, så man altid gerne skal kunne køre hele verifikationen forefra og nå de... nu muligvis omskrevne ITP-sætninger (altså sætningerne som ITP'en har gemt i en særlig liste (på en særlig måde og muligvis i sin egen program-mappe, således at den er lidt skjult for brugeren))? Og når man så har vist et gyldigt opslagsprogram, så kan man så begynde at ændre biblioteksformat derfra; altså kun i efterfølgende sætninger. Jo, det var jo også lidt det, jeg kom frem til i går... Okay, men der er stadig flere ting at overveje.. Hm, i starten gemmer man jo bare sætningsbiblioteker via AreProvable formatet. Man kunne endda, hvis man ville, først gemme sætninger (på samme måde), der kan bruges til at definere et omskrivningsprædikat, og så kan man allerede bruge sit eget format derfra. Hm, og det får mig til at genoverveje, hvad et oplsagsmodul, overhovedet vil sige... Er det bare et slags program over ITP-sætningslisten, som skal skal have mulighed for at åbne de filer, listen refererer til? ..Og så skal man på en eller anden måde have, at resultetet så kan bruges i et PS, hvor (meta-)antecedenter så sættes på bib-headeren... Hm... Er bib-headere så bare en del af den reflekterede model? Men så kan man kun bruge meta-antecedenter for AreProvable-sætninger... *(Nej, headeren kan også bare være en antecedent til en ITP-sætning.) (Okay, her er vist et hul, jeg skal tænke nærmere over..) Hvad så med de meta-antecedenter, der kommer fra at hive gamle-sætninger ind, og hvad når man ikke kan finde dem..? Går det at sige, at ITP-sætningslisten bare er et åbent objekt i ethvert proof script, og så kan der så være meta-antecedenter/-propositioner for, hvordan man må slå propositioner op i dem, og hvilke antecedenter, der skal hægtes på ved diverse opslag..?... Og det er så for de normale (sætnings-)beviser og så skal programmbeviserne, inklusiv opslagsprogrammer, være særskilte... og findes på en særskilt sætningsliste...? Hm, det virker egentligt nogenlunde fornuftift, at ITP-sætningslisten bliver inkluderet som et objekt. Og så kunne det rigtigt nok være meta-antagelser, der så lægger grundlag for, hvordan man må udtrække sætninger fra denne liste (så at man ikke nødvendigvis ser den som 100 % sand). Sætningslisten kunne endda, måske med fordel, være et mappe-objekt i stedet.. Nå jo, men objektet bør stadig være skærmet af, så man kun kan bruge opslagsprogrammet på det, sådan at PSs kan være mere opslagsprogram- og særligt biblioteks-(altså sætningsliste-)uafhængigt. Okay, men hvad så, når man gemmer sætninger og gerne vil flette dem ind i tidligere biblioteker, og dvs. i tidligere sætninger. Ah, men man kunne jo så bare køre et omskrivningsprogram, efter at man har gemt den seneste sætning. Så standard vil så bare være, at udvælge sætninger, der skal tilføjes ITP-sætningslisten (for normale sætningsbeviser), og så køre et omskrivningsprogram lige efter. Og man kan så eventuelt også have flere lag af sine sætningsobjekter, udover at sætninger om omskrivning og opslag selvfølgelig skal være i en anden liste. Definitioner og meta-antagelser må så gøres i underliggende (eller overliggende om man vil), med definitioner nederst i øvrigt (på nær meta-prop-atomerne, som selvfølgelig er helt grundlæggende). Okay, men hvad så med meta-antagelser om at de underlæggende biblioteker, som ikke kan nås i opslagende og omskrivningerne, fordi de ligger under disse; hvad skal man sige om deres sandsynlighed for at være sande (og ikke korrumperet)? Tja, omskrivninger kan jo ses som en seperat ting (bare hvor ITP'en stadig lægger op til, at brugerne bygger og vedligeholder sådanne omskrivningsprogrammer på matematisk vis (altså ligesom med resten af programmere)) og behøver altså i princippet ikke nogen restriktioner (selvom dette er anbefalet). For det er vel ikke meningen at alle definitioner og opslagsprogrammer skal slås op i starten af hvert bevis for sig (men hvordan siger man ellers noget formelt om deres korrekthed?)?...
% På en måde ja, jeg tror muligvis det er svaret.. Jeg tænker at man måske i det mindste skal lave en abstraktion så at de mere grundlæggende sætningslister også kan ses som binære objekter (hvad de jo også godt sagtens kan være; de kan endda være filer (én-til-én), hvis det skulle være), således at det altid kan noteres for en meta-antagelse fra start, at disse sætninger er korrekte... Hm, og jeg har så tænkt noget med, at der så kunne blive mulighed for at eftervise korrektheden ved at tjekke checksums og/eller hashes, men så skal denne metode jo så selv komme af en endnu tidligere meta-antagelse... Hm... Ja, og det gør det jo lidt besværligt, og det er i forvejen ikke rigtigt noget, man kommer til at bruge --- eller har lyst til at bekymre sig om --- så hvorfor ikke bare sørge for, at sætningslisterne er skarpt laginddelt eller noget i den stil, så der ikke er mulighed for at komme til at omskrive de grundlæggende meta-propositioner...? Jo, det bør vist bare være noget i den stil.. Ja, simpelthen, for det er på en måde også lidt det bedste af begge verdner.. Ja, det passer egentligt helt med det, jeg lige var i gang med at foreslå, så det er vel egentligt bare parfekt, er det ikke?.. Uh, og meta-antecedenter kan faktisk godt referere til sig selv, hvis det er. Jeg skal lige se, alle regler og programmer bliver vel så bare laoded med alle definitions-/meta-propositions-/program-/sætnings-listerne, der starter med at være til stede ved indgangen af beviset, og hver af disse har så meta-antecedenter tilknyttet sig... Hm, eller skal man lave en indledende gruppe, der alle bare har en mere eller mindre implicit m.ant. på sig, der på en måde siger "medmindre noget er gået helt galt," og så gemme de faktiske meta-antecedenter til senere lister? Hm, eller kunne man ikke benytte at antecedenter kommuterer til bare at gemme denne mest grundlæggende (samlede) m.ant..? Hm... Ah, jo, det kan man i princippet faktisk godt. ...Eller er det lige en tand mere kompliceret end som så? Jeg fisker lidt efter, om man kunne tilføje meta-antecedenter, som man på en måde allerede har antaget er sande ved at antage grundsætningerne, men som man alligevel skal bevise, før man kan bruge PS'et til noget særligt, så som at tilføje andre fundamentale ting (eller måske før man uploader ting, eller før man omskriver tidlige lister osv.). (Uh, forresten kan jeg ikke også lade selve aksiomerne være op til brugerne?..! Altså netop ved at lave et opslagsmodul..! Så er det bare standard-semantikken, brugerne skal være enige om, og ikke alle aksiomerne. Lyder altså ret nice..!!) ..Og uden faste aksiomer er der måske netop også plads til sådan nogle grund-meta-antagelser.. Tja, eller man skal vel egentligt bare have, at visse propositioner (meta-) ikke kan gemme og/eller trækkes frem via opslags-modulet. Og det skal så nemlig særligt være propositioner, der giver lov til at initialisere visse programmer og/eller gemme ting i de tidlige lister --- og/eller at uploade ting eller søge på ting på internettet osv. Hm, eller det er jo ikke nok; det er ikke den rigtige måde at gøre det på... Nej, det skal faktisk være en slags eksterne programmer, der giver et output og resulterer i en sætning, som man ikke kan vise ellers, ingengang hvis man viser en modstrid... Tja, og det er vel egentligt den pointe, der er vigtig, uanset om det skal være centreret om programmer eller ej?... Hm, man burde vel så holde en adskilt liste over viste-metapropositioner i proof environmentet.. Ja.. Jeps, det bør der virkeligt være, og brugeren bør så sætte meget strengere regler for, hvilke regler og sætninger, der kan bruges her. Ja, måske bør man netop centrere alle "regler" her om kørsel af godkendte programmer, og ikke så meget på logiske regler osv. Ja, hvorfor ikke, der lyder fornuftigt nok. Men det skal brugeren nok stadig bare kunne vælge selv. Nå, hvordan skal de viste meta-propositioner så bruges? Skal de ikke bare bruges i slutningen af en verifikationskørslen af PS'et, kørt af brugere specifikt med henblik på at gøre noget avanceret? Og så eventuelt af regler i den normale del af beviset, der kræver ekstra tilladelser.. ja, sikkert nok.. Ok.
% I går aftes (som var d. 7.) kom jeg på, om det ville give mening faktisk bare at se arbejdsformularen som et slags binært objekt selv og så have at selv regler så antages fra starten i et PS. Nu hvor aksiomer jo er ret frie, hvorfor så ikke se, om regler ikke også kan være det..? Jeg kom lidt frem til, at det næsten måtte være ligeså godt, som det jeg ellers har tænkt, men er ikke så sikker nu, så jeg skal lige tænke lidt (hvad jeg godt nok skulle alligevel). Jeg skal bl.a. tænke over, om PS og arbejdsformular så skal være adskilte eller ej. Jeg skal også tænke over, hvordan man sikrer sig, at PSs ikke bliver for brugerspecifikke og sværre at få til at passe sammen. Men det er i øvrigt også noget, jeg skal tænke over alligevel, for som det var før, skulle man jo også lave en masse regler selv, hvor man først skal have programmerne.. Nå ja, det ændrer rigtigtnok ikke så meget på den måde, for jeg havde jo alligevel tænkt mig, at folk skulle lave og bruge deres egne PS formater til child apllications, og så bare blive enige om at bruge de nogle af de samme. ..Men det bliver lidt sært med alle fil- og memory-objekt-regler... Tja, eller gør det? For programmer har jo allerede evnen til at åbne filer og mapper og allokere plads osv.. Ja, og de kan jo som sagt også have muligheden for at exec'e visse andre programmer.. Ja, virker stadig oplagt nu at prøve at give mulighed for at opløfte child apllicationer, så de kan gøres direkte efter de relevante mate-antecedenter og blevet tjekket, i stedet for at man skal køre dem inde i et konstant PS-sprog. ..Hm, hvad specifikt med bruger-input-objekter, hvordan komme de i brug? Jamen svaret er vel igen, at det har programmerne jo (eller det skal de have: jeg vil under alle omstændigheder sørge for at (forslå at) de ikke behøver at gå igennem main for noget, men kan køre helt selvstændigt) evne til selv at styre. Hele pointen er jo, at child-applikationer skal have mulighed for at erstatte main ITP'en helt, så jeg har på en måde allerede lagt op til det her. Hm, og med mine seneste tanker, så skulle ethvert PS alligevel starte med at konfirmere (ikke vise, bare bekræfte) korrektheden af child-applikationsprogrammerne, så det er faktisk ikke rigtigt nogen stor forskel; det er vel nærmest bare, om processen starter med at køre et nyt program direkte, eller om processen bare er pågældende program i stedet. Hm okay, hvilke meta-propositioner skal så fastslåes inden kørsel af child-programmet? For det første at programmet er gyldigt.. Og så kunne man også sige, at programmet til så at udtrække den resulterende sætning også skal erklæres gyldigt inden kørsel, så det ligesom er på plads og så man bare kan køre det direkte. Og hvad så med recovery og hvad med substitution? Recovery kommer jo så bare til at køre på samme måde, hvis man tænker på child-applikationer, for det handler jo bare om at køre et nyt program over det gamle, som er beregnet på at bruge de filer, der er efterladt af det gamle. Hvad så med recorvery, hvis del-procedurer fejler, men uden at starte det helt forfra? Kunne man så ikke bare give exec-programmer semantisk mulighed for at fejle..? Matematik kan jo godt beskrive tilfældige programmer, ikke, så mon ikke man sagtens kan det?... Jo, det kan man, og det ville i øvrigt være en god måde helt at fjerne kravet om en abtrakt input-fil (som vi skal forestille os er der, men som ikke rigtigt er der). Nu kan man så bare lade input være noget der på "tilfældigvis" styrer eksekveringen, hvilket nok.. måske.. er lidt nemmere at tænke på det som.. Ja, det tror jeg, det er, for så er der et klarer skel mellem normale input-filer og så brugerinputtet, når man udformer programmet. Cool nok. Og så kan exec-underrutiner godt få error handling implementeret for sig, hvad man så kan bruge til gendannelse fra bløde crashes. Hvad så med løbende substitution? Kan man så bare gøre dette til en del af programmet, og altså gøre det til regler bare? Nå ja, det samme ville i øvrigt gælde for child-applikationer i det, jag havde før. Ja, det må man jo kunne. ... Virker altså meget godt, alt det her. Så må man sørge for, at IL'et/IL'erne har en sikker måde at se, hvilke filer der er output fra exec-kald, og så er vi vist ret godt kørende. Og så kan (child-)ITP-applikationen bare køres via et exec-kald fra udtrækningsprogrammet, således at dette til sidst læser ouput fra pågældende exec-kald, og så sørge for at skrive til/omskrive biblioteket, sådan at de nye sætninger tilføjes.
% Meta-propositioner bør vist også så kunne holde lidt styr på, hvilke nogle filer var output fra sidst og sådan, bare for at gøre f.eks. recovery-procedurerne lidt nemmere at programmere og sådan.. 
% Uh, ved at se input som noget der kan genereres under eksekveringen, helt eller delvist tilfældigt, så kan man pludselig også nemt nok give instruktioner lov til at spørge om f.eks. permissions. Og så kan man sådan set bare have det som en del af programsemantikken, om den overholder alle permissions, evt. ved at programmet først query'er dem. Nice nok. Hm, på en måde bliver meta-proposition-delen af et bevis så bare lidt det der før var main-PS'et... Ja, fint; men det er så bare fokuseret på at tjekke diverse ting omkring diverse moduler, inklusiv pågældende ITP-program. Det giver vist god nok mening.. Ja, og ved at holde det sådan adskilt, så kunne man sagtens ændre på API'en for denne MP-del; det er ikke ligesom det gamle main-PS-sprog, der gerne skulle være nogenlunde konstant. 
% Okay, så man har altså et meta-script til at sætte ITP-programmet i gang, eller hvad man nu ellers genre vil: Man kan også sætte en verifikationsproces i gang, eller en omskrivningsproces, eller en regel-inisialiseringsproces eller så videre... Hm, disse regel-init-processer handler vel så enten om at ændre ITP'en aller også bare tilføje en implementation til en funktion, måske en placeholder-funktion, via en DLL.. Ja, så ville IL-semantikken faktisk skulle indeholde det at man kan sammensætte programmer ved brug af DLLs.. Bemærk forresten, at man stadig bør starte med et deduktivt system, som jeg har tænkt mig, med refleksions-model og -regler og det hele, men bare hvor dette system så ikke nødvendigvis bør ses som det grundlæggende forældersystem. Pointen er nemlig, at der kan være mange forældersystemer, der når til de samme børne-ITP'er. Og så bliver det, som jeg før tænkte på som core-ITP-systemet, så bare til et godt udgangspunkt. Og den virkelige forælder er på en måde så den applikation, der kører meta-scriptet osv., men her er en pointe jo så som sagt, at denne faktisk kan ændre sig meget. For det eneste man kræver af meta-scriptet er bare, at den kan verificere diverse permissions og køre de relevante programmer, der skal til i den samlede (meta-)applikation. ...Hm, men meta-programmet skal vel på en måde tænkes med, når man laver en ny børne-ITP, eller hvad. Skal man bare kunne regne med nogle faste ting fra meta-applikationen; det der så nu bliver "main-applikationen" (skelettet til ITP'erne nærmest)? Eller skal man ligefrem behandle main-meta-applikationen som et program selv, som så også kan skrives om.... (Håber ummidelbart lidt på det første, men lad mig nu lige se...) Ja, det må da næsten blive det første; at main-ITP'en bevares mellem transformationerne til børne-ITP'er, men at den bare er ret åben overfor, at det kan gøres på mange måder.. Det tænker jeg nemlig lidt, den kan være, da den jo bare skal tjekke permissions og sætte programmer i gang... Hm, man meta-script reglerne \emph{skal} jo være normale programmer, så mon ikke egentligt det kommer til at give sig selv meget godt... Tjo, men så skal meta-script-programmet vel også bare være et program (altså et der kan kodes med dets eget IL-sprog). Dette program har så bare maksimale permissions.. ...Hm, kunne man måske se det lidt som, at man har nogle programmer i et vist hierarki, hvor der genre skal gælde visse forhold for de filer, der findes i deres tilhørende mapper.. Hm, men skal der så ikke være filer i main-program-mappen?.. Jo, det kunne man da godt have. Man kunne godt sige at programmer både har en "skjult" mappe, som lidt er beregnet på at gemme sætningslisterne, og så en mere åben fil-mappe, der gerne skal være lidt mere resistent over for, hvis brugeren kommer til at ændre lidt på noget, eller, mest vigtigt, kommer til at smide nogle filer ind det "forkerte" sted osv. Denne mappe skal altså være en, hvor brugeren har frihed til selv at tilføje filer og mapper, og også gerne være en, hvor det aldrig er helt fatalt, hvis filer ændres eller slettes; så er det nemlig bedre at f.eks. DLLs osv. kommer i den "skjulte" mappe. Okay, men hvad så, ville det så være en idé, at main-applikationen bare er en samling programmer med deres tilhørende mapper, man man så starter.... Åh, det er sgu en smule kompliceret, det her (men sådan går tingene jo gerne..)... men ja, man måtte vel så starte med at have et IL defineret semantisk i forhold til matematiske egenskaber ved dets virkning (og ift. en ekstern compiler (altså en som ikke selv er skrevet som i IL-sproget --- i hvert fald ikke nødvendigvis)). Hm.... Tjo, kunne man så også bare beskrive semantikken om meta-applikationen generelt, sådan at man i udgangspunkt-applikationen så kan beskrive semantikken er tilsvarende applikationer, hvor diverse moduler er ændret?... Hm, er det sådan, den skal skæres?.. Og så kunne man måske endda have det åbent for, at man kan ændre på meta-applikationen, så man kan køre alle mulige programmer i stedet også.. Hm, jah, det lyder egentligt meget fornuftigt.. Så altså main-applikationen har bare et udgangspunkt implementeret (som i princippet kan ændres fra version til version (der kan nemlig være flere forældrer til samme barn)), og ellers handler main-applikationen så bare om at initialisere børne-applikationer, og skabe en oversigt over dem, og disse applikationer er så alle sammen stedfortædere (medmindre de har en halt anden funktion) for det, der før man "main-applikationen," hvilket altså vil sige at de også selv implementerer deres eget meta-script osv. osv. Hm, jeg skal selvfølgelig tænke noget mere over det, men det lyder da egentligt OK..
% (09.03.21) Okay, så jeg laver nok det skift her, hvor main-applikationen så bare bliver en der administrerer mapper og sætter programmer i gang, men hvor der så som sagt er implementeret et udgangspunkt, som er ret meget ligesom det, jeg ellers har haft i tankerne, og så inklusiv det seneste omkring det særskilte meta-script. Og så bliver en af de store ændringer, ift. hvordan jeg tænkte at ITP'en blev at arbejde i, at refleksionsreglerne... Hm, man kunne jo godt sagtens --- og bør faktisk --- beholde de gamle refleksionregler. Men nu får man så også mulighed for at omskrive selve ITP'en. Hm, jeg overvejer lidt, om man måske så skulle vente med, at udvide ITP'en så den går fra at være main-applikationen til at være en udgangsimplementation i en senere version... Tjo, men uanset hvad, så bør man næsten alligevel indføre en regel, der gør at man kan gå direkte til en child-ITP, i stedet for at bruge P(x)-udtryk.. Hm, men er det så ligetil?... (Jeg bør i øvrigt sikre mig, at det fremgår, at brugere let skal kunne arbejde med binary object collections, så man sagtens kan have programmer, der bruger en masse forskellige binære objekter, og som kan allokere nye og forlænge dem. I øvrigt: selvom programmerne nu godt kan være "tilfældige," så er det vist stadig en god idé at kræve, at de aldrig læser fra ny-allokerede bytes, men altid overskriver dem helt, før der må læses; informationen skal ligesom gå tabt.) ... Hm, men er det nemmere at programmere start-ITP'en i sit eget sprog..? For nu skal grund-IL'et jo faktisk udvides, så det virker på normale fil-systemer (dog begrænset til nogle bestemte mapper) og ikke bare abstraktioner over dem, så nu burde det derfor have alt, hvad der skal til. Ja, så man bør sagtens kunne sikre sig, at start-ITP'en kan kompileres til grund-/start-IL'et. 
% (10.03.21) Det var, som ses, ikke så meget arbejde, jeg fik lavet i går (da jeg havde det lidt skidt indimellem), men jeg fik da tænkt videre. Og det bliver altså en rigtig god måde at gøre det hele på. Så skal der altså bare være en ITP fordelt på et antal mapper, som kan være udformet på mange måder (for den kan nemlig også altid udskiftes), men den skal altså så have et deduktivt system, hvor den er i stand til at beskrive og analysere et IL, som faktisk ser mapper og filer, som de er (og laver først abstraktioner over dem (som i binære og mappe-objekter), så navne og meta-data bliver skjult osv.), og som beskriver og er i stand til at analysere selv ITP'en selv. Dertil skal den så også have en måde at erklære meta-antagelserne om disse mapper, altså selv den mappe, den selv ligger i. Disse meta-antagelser skal så resultere i, at man ITP'en, der selvfølgelig også skal have refleksionsregler til at køre IL-programmerne, kan køre programmer efter opstart for at udtrække og/eller verificere propositioner. Meta-antagelserne bør altså beskrive ét eller en mængde af programmer, som så er ITP'ens opslagsprogrammer ligesom. Det essentielle er så, at ITP'en så kan initialisere programmer, som lægges i mappen --- og hvor deres IL-kildekode råder over hele eller en delmængde af selve ITP'en mappe-mængde --- og som både ITP'en og brugeren (altså også imens at omtalte ITP-program er lukket) så kan give sig selv lov til at køre, ved først at vise, at dette program overholder meta-antagelserne om mapperne, både idet det oprettes i en mappe, og idet det efterfølgende køres et vilkårligt antal gange (med vilkårligt input (som nu btw ikke modeleres via input-objekter, men hvor programmer bare fra start ikke er deterministiske, men hvor programmerne hele tiden kan få input fra, hvad man kan se som orakler i modellen)). Og eftersom IL'et er stærkt nok til at beskrive ITP'en selv, så man sågar initialisere søskende-ITP-programmer på denne måde, i sådan en grad at man potentielt kan have at ITP'erne helt har "glemt," hvem der initilaiserede hvem; og man kunne måske endda have startet med to. Søskende-ITP'en skal så (per definition) og varetage en meta-antecedent-liste (osv.), og kan så selv til hver en tid verificere, at denne er i overensstemmelse med den anden (måske originale) søskende-ITP's meta-antagelser. Meta-antagelser kan endda være af højere orden, således at man godt kan tillade, at der sker ændringer i mappen, som så kommer til at definere nye meta-antagelser. Særligt kan en ITP således give staffetten videre til en søskende-ITP, eller en barne ITP, som jeg kommer til straks, således at brugeren kan foretage ændringer i meta-antagelserne her, og hvor disse ændringer automatisk propagere videre til de andre ITP'er. Så de nye ITP'er kan altså sagtens være på lige fod med de gamle, men man kan så selvfølgelig også oprette en barne-ITP, som så altså kun råder over en skarp delmængde af den forrige mappe-mængde. Og så kommer endnu en vigtig pointe, for det betyder så, at enhver af disse ITP'er, som kan bruges som et udgangspunkt for sådan en ITP-familie, kan så altid i princippet selv være et barn til en yderligere ITP-forælder, uden at "vide det" selv. I praksis betyder det så, at brugere altid i princippet kan eksportere hele deres ITP-mappe-mængde til f.eks. en ny, opdateret ITP (med et opdateret IL-sprog) eller til andre brugere, hvor disse mapper indsættes som undermapper i en eksisterende ITP-mappe (hvor meta-antagelserne så selvfølgelig skal tillade, at man gør dette). Og på denne måde, at har vi ikke behov for noget grundlæggende deduktivt system, for alle deduktive systemer (med disse kvaliteter) "ved ikke" i princippet, om de bare selv er en barne-ITP til en anden ITP, som er baseret på et andet deduktivt system. Barne-ITP'er behøver så som sagt ikke, at kunne det samme som forældrene, hvilket så også giver fin anledning til (som antydet), at ITP'er kan opdateres ved at smækkes ind som børn til nye, mere kraftfulde ITP'er. Samtidigt behøver børn heller ikke have de samme kvaliteter, som grund-ITP'en har, men kan sagtens være meget forsimplede versioner af ITP'en. Ja, faktisk vil det være en god idé generelt at arbejde i en mindre kraftfuld barne ITP end grund-ITP'en, fordi man så med større sikkerhed kan lave antagelser om, ITP'en fungere korrekt, simpelthen fordi der så vil være færre sikkerhedsbrister (hvad der kan være, fordi antagelserne ikke behøver at være 100 % korrekte, især ikke til at starte med (hvilket jeg vil komme mere ind på lige om lidt)). Ja, og de programmer, der initialiseres af de kraftfulde ITP'er i familien (eksempelvis af det der var den første ITP), behøver slet ikke at overholde nogen af de samme restriktioner; det er helt op til brugeren. ITP'erne kan således initialisere alle mulige programmer. Det eneste der kræves af programmerne er som sagt, at de ikke bryder meta-antagelserne. Fordi IL'et så skal være så kraftfuldt, at ITP'en kan beskrive selv sig selv, så åbner det bare op for, at man sågar kan lave tilsvarende søskende-ITP'er (og barne-ITP'er selvfølgelig), men brugeren har altså fri mulighed for også at initialisere alle mulige andre programmer. Lige for at slå meta-antagelserne mere fast, så er de propositioner om, hvad der kan forventes at forekomme i mappen, både umiddelbart efter opstart og i løbet af kørslen af ITP-programmet. De beskriver altså en invarians, som både ITP'en selv skal overholde, og som brugeren skal overholde ved ikke at lave forbudte ændringer. Man kan derfor se det som en slags (ikke-bindende, men vigtig) kontrakt mellem bruger og ITP. Brugeren kan så forstærke sikkerheden i ITP-mapperne på forskellige måder, bl.a. ved at have "skjulte" mapper, som er placeret et sted, hvor brugeren ikke regner med, at noget eller nogen (inklusiv brugeren selv selvfølgelig) ændrer tingene i den mappe. En anden vigtig ting, som brugeren kan gøre, er at bruge certifikater. Med disse kan brugeren have en hel mappe-mængde, hvor brugeren frit kan smide filer ind, også ukendte filer (med de sædvanlige sikkerhedsforanstaltninger om ikke at sikre sig mod at downloade malware til sin computer), hvor ITP'en så kan tjekke eventuelle certifikater (også efter eventuelle data-tranformationer; filerne kan således f.eks. godt være komprimerede på en arbitrær måde, uden at meta-antagelserne skal kende komprimeringsmulighederne på forhånd) og så hive propositioner direkte ind via deres opslagsprogrammer (selvfølgelig med en passende konvention om at vedlægge passende antecedenter, så der holdes styr på probability/security/certainty/acceptance levels på en passende måde). Til sidst skal jeg lige nævne et par ting, bl.a. at der selvfølgelig så også gerne må ligge en et compiler-program i en af ITP-mappen. Og det bør så selvfølgelig være en (ret essentiel) del af meta-antagelserne, at dette program skal køres på IL-programmerne, hvorved en executable bliver dannet (med et passende navn), som brugeren og/eller ITP'en så kan give sig selv lov at bruge. Jeg tænker at det er en fin idé at sørge for, at ITP'en holder styr på de kompilerede programmer og så selvfølgelig også har mulighed for at køre dem. (Man kunne så i øvrigt både give sin ITP en kør-og-vent-funktionalitet samt en kør-og-luk-selv-funktionalitet, således at man også f.eks. kan bruge sin hoved-ITP til at launche andre programmer, hvor ITP'en så lukker selv med det samme.) Og ift. "eksterne programmer," så kan det så selvfølgelig også sagtens være en del af meta-antagelserne, at man også kan køre visse executables, også selvom man ikke kompilerede dem selv på standard vis. Og ved brug af certifikater kan man så endda få frihed til at hive certifikerede programmer ned fra internettet og køre dem direkte. Dette kan også være smart, hvis man f.eks. gerne vil køre closed source-programmer. I forhold til at hente ting fra internettet, skal det selvfølgelig også stadig gerne være en del af det omtalte, meget "kraftfulde" (eller hvad man skal sige) IL, at man også kan query'e internettet med det (ligesom jeg nævnte det ovenfor). (Og da programmerne nu kan være ikke-deterministiske, i hvert fald set fra meta-antagelsernes synspunkt, så er det jo ingen vildt kompliceret sag at give semantik til sådanne instruktioner (da man bare kan antage at den hentede data er tilfældig).) (Åh, forresten: Man skal så lige sørge for at formulere certifikat-antagelserne, så de ikke strider imod, at man kan have tilfældige binære objekter i sine mapper. Man skal altså lige sørge for at flette en antagelse om, at der ikke kommer til at forekomme fejlagtige certifikater, med en antagelse ved at der kan forekomme "vilkårlige" objekter i form af de tilfældige objekter, så det ikke kommer til at give en modstrid. Dette kan løses eksempelvis ved at man altid sørge for at huske et "medmindre uheldet med tilfældighederne virkligt skulle være ude," hvor man så på en måde beholder denne antecedent, men bare overser den i alle praktiske sammenhænge (medmindre man går videre og regner på risikoen, hvad man også kunne gøre).) Nå ja, og slutteligt er pointen så, at den pågældende ITP så sagtens kan være udformet, så den passer med alt det, jeg ellers har beskrevet i denne notetekst --- ja, man kunne endda have et mindre kraftfuldt IL, som denne ITP kan bruge til at lave alle de mere interne refleksionregler, som jeg har beskrevet dem. Den store forskel bliver så bare, at man så skal sørge for at beskrive alle de antagelser, der skal til for at redegøre for semantikken og korrektheden af selv ITP'en (og altså alle de forhold, jeg var i gang med at beskrive), matematisk. Men heldigvis behøver man dog ikke at vise dem alle sammen, før man lancerer første version af ITP'en. Her kan man sagtens give sig selv den samme frihed, som man ville gøre ellers, til at konkludere de ting omkring kildekoden til ITP'en, man skal bruge. Man skal så bare lige sørge for at formulere disse antagelser matematisk, og helst på en måde, at de kan bevises på et lavt niveau senere (og med "lavt niveau" mener jeg, at de kan verificeres med en computer senere, uden at bero alt for meget på abstrakte NL-sætninger, der skal verificeres af community'et i stedet for). Okay, så beskriver faktisk allerede den nye form for meta-ITP, som er min idés nuværende udviklingsstadie. Og det virker altså bare super lovende at gøre det på den måde; det er jo ret fantastisk, at man bare helt kan undlade stort set alle bekymringer om det grundlæggende deduktive system, sådan bare lige. (Ja, det er altså på nær at meta-ITP's deduktive system skal indeholde refleksionsprincipper, der tillader den at køre programmer, og man skal også gå med på tanken om meta-antagelser, men det ville jeg jo kræve alligevel.) Jeg tror faktisk ikke, jeg får arbejdet meget mere i dag, fordi jeg har nogle ting, jeg skal nå, men nu er der også virkligt bare god opklaring på tingene. Det næste jeg skal er lige at gennemgå de ting, jeg manglede at skrive, og sørge for at de bliver nævnt, og så kan det være at jeg egentligt bare afbryder noterne og smækker denne tekst (bare på dansk som den er) bagpå. I forhold til de ting, jeg måske ikke havde tænkt så meget på at skrive om, før jeg noget hertil, så kan det godt være, at jeg lige skal skrive lidt mere om... Nå ja, der er faktisk flere ting, man godt lige kunne skrive om, som gør at min ITP-løsning, som jeg var i gang med at skrive om, kommer til at passe bedre til denne nye teknologi. Jeg skal så bl.a. skrive om de her meta-scripts, og jeg skal også give nogle eksempler på acceptance-/...-antecedenter. Sidst men ikke mindst, bør jeg faktisk også slå et slag for, at bruge NL fra start --- en ting som jeg faktisk ikke har tænkt så meget på at nævne, fordi jeg vist nok tænkte det som enten ret indlysende eller som ikke så vitgit, men, nej, der er faktisk en vigtig pointe at undertrege med det, som ikke er helt indlysende. *(Men nu vil jeg også sætte noterne om mine ITP-idéer sammen med noterne om semantisk web osv., så det kommer faktisk lidt af sig selv --- bare jeg lige husker pointen.) Men ja, virkeligt fedt, alt sammen..!!


% Okay, så lad mig lige samle en liste over de ting, der er særligt vigtige at få med, også medregnet de ting, jeg ikke har fået skrevet om endnu (i brødteksten). Helt generelt kan man jo sige, at jeg med denne nye version af min idé ikke behøver at gå nær så meget i implementationsdetaljer (dvs. mindre end jeg allerede behøvede), for der vil længere være noget pres på at den første version skal være forudsigende ift., hvad man kommer til at ønske i fremtiden. Det skulle jeg jo alligevel med min gamle version af idéen, fordi den krævede et grundlæggende deduktivt system, og derfor ville det være rigtigt godt at sørge for, at dette system blev både simpelt og med alle de ønskede muligheder i det allerede fra start af. Men nu kan jeg i stedet bare forklare en rimelig simpel kerne-meta-ITP, og så tage de vigtigste pointer med mine tanker om den specifikke type ITP-miljø/-applikation, jeg har skrevet om ellers. Så følgende er altså stikordsnoter over hvad de vigtigste pointer er:
% - Jeg skal selvfølgelig forklare om, hvordan meta-antagelser kan lægge op til, at man udleder specifikke mere meta-antagelser fra dem, og særligt ved at bruge certifikater. (Tja, denne pointe er også beskrevet i ovenstående tekst, og jeg vil egentligt prøve at skippe de pointer her, så husk at læs ovenstående også for en opsamling, Mads.)
% - Jeg bør forklare om, hvordan jeg tænker at "fodnote-antecedenter" kan bruges, og om forskellige forhold, hvor man giver en vis tillid til en metode (inklusiv opslags-programmer), men sørger for at denne tillid afhænger af, hvor meget og hvordan man bruger metoden (så man usikkerhederne hober sig op, og særligt hvis man bruger metoden på en meget afsøgende måde, og dermed nærmest leder efter fejlene). Nævn også, hvordan det nok er klogt ofte at skjule disse lidt ved at have dem mere som en slags fodnoter eller meta-data, man skal klikke (højreklikke måske) for at se, så man bedre kan fokusere på formularerne og matematikken der.
% - Nå ja, jeg kan jo så også ligeså godt nævne lidt om typetræet og sådan, nu hvor mange af disse ting alligevel på en måde hører til, hvad jeg før anså som child-ITP'erne.
% - Apropos sidste punkt, så kan jeg lidt inddele det i kerne-meta-ITP'en først, som man kan forklare om og begrunde med, at man alligevel bør have både refleksionsregler og meta-antagelser, så kunne man forklare nogle detajler om, hvilke moduler og sikkerhedsforanstaltninger *(ved f.eks. at bruge en undermængde af IL'et, der så måske har tilføjede funktioner og bruge mere "objekter," hvor visse ting er skjult.. Og ved at kræve at der vises ting programmerne, før de overhovedet anses for programmer) en generel ITP kunne have, så den kan bruges effektivt og så brugerne på en nem måde kan dele deres propositioner med hinanden, også selvom de har forskellige præferencer, og til sidst kunne jeg give et bud på, hvordan en fin arbejds-ITP-applikation kunne se ud; med typetræ og operationer osv. 
% - Angående det med at brugere skal kunne dele beviser osv., så skal jeg huske at beskrive, hvordan beviser skal ses som sekventer, således at brugere godt kan udføre beviset (i visse tilfælde: Når vitale ting så som dynamisk definerede (og derpå brugte) refleksionsmetoder ikke afhænger af dem) fint, og så bagefter sørge for, at vise de antecedenter, der f.eks. ikke blev fundet af opslagsprogrammet.
% - Jeg skal også forklare, og dette mangler jeg faktisk lidt at overveje pt., hvordan meta-antecedenterne udformes. Det handler jo selvfølgelig bare om at definere en model, der svarer til ITP-mappe-mængden, og så sørge for at sin grund-ITP, hvis vi skal kalde den det, har en måde at query'e dette objekt og køre programmer på det osv. Denne ITP-skal altså ligesom bare have magt over mapperne, som var det er internt matematisk objekt, og så følger resten meget ligetil; så er det bare at erklære antagelser om dette objekt (altså om hvad objektet kan være ved opstart inklusiv, hvad de tilgængelige programmer gør, så ITP'en også kan sikre sig ikke at bryde invarianten i henhold til meta-antagelserne).
% - At compilere også selv kan blive en del af meta-antagelserne, således at kompileringen kan blive mere og mere open source for ITP'ens synsvinkel, og så ITP'en selv kan begynde at håndtere mere og mere, hvad computerens hardware- og OS-specifikationer er, og herved sørge for at kompilere programmer både korrekt men også muligvis mere optimalt. Og at dette også åbner op for at brugere også closed source programmer, hvor at certifikaterne kan sættes med hjælp fra ekspert-brugernes/programørernes egen ITP, da disse nemlig kan hjælpe med at verificere, at programmet er kombatibelt med brugeren system. 
% - Jeg skal have færdiggjort og/eller nævnt det med lære-elev-fjernet-input.
% - "Omkring intuitiv, action-/metode-baseret bevisførsel, måske med grafisk hjælp (hjælpetegninger). Og: Selv hvis det kun virker for diskret matematik (f.eks.), så er det stadig en succes. *("Uh, og for applied matematik som et andet godt eksempel.")"








%The main solution I will present for all this is to use what I will call \emph{meta-antecedents} and \emph{meta-assumptions} here. Meta-antecedents are antecedents that are not necessarily expressed in FOL/SOL alone, but might include atomic formulas that is meant for the ITP to check before the propositions can be loaded into the environment of a proof script (both for construction as well as for verification). Among the things the ITP might check are hardware/OS specifications, the ITP version,  compiler module specifications/permissions (which define the IL, or the set of ILs for that matter, that can be used by the reflection rules) as well as any other kind of permissions/restrictions or preferences that the individual user might choose (i.e.\ all the settings\footnote{
%	I feel like mentioning the following examples: These settings might for instance include how much memory is allowed to be used and how much hardware space. It could also include things such as how many processor cores can be used when the IL at some point gets a version that is able use concurrency in its programs. 
%} and/or extension options the ITP might have). 
%Oh, and the ITP might even include a module to run external programs, in which case the meta-antecedents should be able to speak about the correctness (and its properties in general) of these. 
%And additionally, I would also propose that the meta-antecedents are used to specify the different protocols of when to accept unproven methods and propositions that each user can choose according to their own preferences (but where most users will probably just follow popular conventions). 
%Lastly, the meta-antecedents should, I imagine, also be the way to load the (previously proven) proposition libraries into the environment of a given proof script. I thus think that each proof script should include a header (much like how conventional programming languages opens libraries), where these header meta-antecedents are then added to all propositions proven from the script. A way to implement this would then, by the way, be to add meta-antecedents headers as well for the library files. I even imagine that the library files could have a tree like structure with local contexts, much like for instance the namespaces of C++, such that a single proposition file can have propositions with different meta-antecedents but still in a neat way, such that the antecedents does not necessarily need to be listed for each individual proposition. Okay, back to the proof scripts. Apart from the header antecedents, which should be repeated for every single proposition proven (but should be kept as implicit), each proposition might also have its own set of meta-antecedents, just like it can have its own set of regular antecedents. The meta-antecedents should thus not really differ from regular antecedents; the only real difference is that meta-antecedents include what we can call meta-predicates, which speak about the users system basically, as well as the outside world, but we will see in a minute how assumptions about the outside world (and in particular the userbase of the ITP (in its different versions)) can be formulated via statements about the users own system (namely by assumptions about what is contained in specific folders).\footnote{
%	Note that I am allowing myself to speed up the writing even more now, since I feel like it has taken me long enough already and I know that I will rewrite (and edit) it soon (hopefully) anyway. I will for instance allow this paragraph to be a bit messy, along with the rest of the text. This is not to say that the previous text is very neat at all; only to state (to myself more than anything else) that I will try to allow myself to be write even more messy text from now on.
%}
%%The only other difference that the meta-antecedents might have as compared to the regular ones is that the ITP might provide more ways for the users to collapse the mata-antecedents, perhaps even in the core language of the ITP. But then again, collapsing meta-antecedent into, say, expandable footnotes at the beginning (or end) of a proposition, could also just be something that is supposed to be programmed for the child ITPs. I guess the reason why one might want to do it for the core/main language as well is to make the propositions in the library files more directly readable by humans. But then again, since meta-antecedent should also be able to be translated into the reflected model (where the mata-predicate formulas can just be treated like any other atomic formula), users can just define their own format for the proposition library files anyway. (Note that they do so, by translating a would-be proposition library into a single proposition about a file instead, which then can have use format that they define. This makes me realize, by the way, that it probably would not be a bad idea, if the files a proposition library refers to can be recorded as part of the same library file instead of as a separate file. This is possible since the ITP should be able to initialize file object from file segments and not just from whole files, which I intend to mention later on i this section (since it relates to the proof script headers)).\footnote{
%%	Note that I am allowing myself to speed up the writing even more now, since I feel like it has taken me long enough already and I know that I will rewrite (and edit) it soon (hopefully) anyway.
%%}
%I think the core language of the ITP should first of all provide an infinite scheme of possible meta-predicates, which does not all (of course) how any semantics defined for them to begin with, but  which instead serve as placeholders for meta-predicates that future versions of the ITP can use. On top of this, the user should also be able to declare custom predicates defined via compound formulas which use said meta-predicates. All formulas that include such meta-predicates should then be treated on an exact equal footing as all other formulas. In particular, they should also be part of the reflected model. This will then mean that meta-predicates can also be pulled into a reflected model. The only way that the atomic formulas formed from meta-predicates can be declared to be true, however, is then to transport them back to the core language and ask the main ITP to confirm or deny them; the reflection programs can not decide this. Well, that is of course unless an IL is at some point given functions to query the main ITP for meta-predicate formulas, where the resulting output, with perhaps a \texttt{bool option}-like type, is then taken to be equivalent of the answer of the same query done by the equivalent query in the proof script (by the designated rule), if the output is not \texttt{None} that is. One could indeed imagine something like this for an advanced version of the ITP. The reason why I think it good to include the whole infinite schema of meta-predicates at once, is that then we do not need to keep changing the language of the main ITP, as well as of the reflected model, every time there is an update that allows for more options. Oh, wait, I actually did intend to suggest that the reflection programs could query the meta-predicates (going back to the previous point). The reason was that this would mean that users would be able to make more machine/system/version independent programs. \ldots Sure, but this would require very very advanced IL semantics, so this would definitely still be a thing, I would suggest be left out of the first versions of the ITP. 
%
%So when should the meta-antecedents be checked, and how? ... 
%% Hm, is it too advanced to have meta-antecedents for settings and so on..? Should I keep it more simple and then mention additional possibilities after..? And what folder meta-predicates should be there? Der skal være dem, der spørger til mappe-objekter, og så kan bruges til at konkludere meta-propositioner... (Hm, jeg skal nok omskrive en del af denne sektion, kan jeg så småt mærke...) Hm.... (Ja, jeg tror, jeg starter forfra.) Hm, giver det mening at have en et meta-prædikat (/ relation) til at sige, at mappe-onjektet, x, passer på en mappe fra brugerens filsystem?... Eller hvad..? Det er jo lidt sådan jeg har tænkt den egentlige semantik for mappe-meta-prædiketerne, som jeg så har tænkt skulle bruges til certifiketer... Eller dvs. at semantikken skulle snakke om en faktisk mappe... Hm, nej det er vel lige gyldigt. For brugere skal jo bare sige, at der gælder noget om deres mappe, og så lægges der bare op til, at andre brugere ændrer den antagelse til en egen mappe, der indeholder deres antagelser. Hm, man burde næsten så kunne tilknytte en mappe et certifikat i stedet for hele tiden at gøre det omvendte på en måde... Hm... Ja. Ja, så brugerne bør kunne tilføje mapper til vise certifikat-antagelses-prædikater, sådan at ITP'en selv sørger for at søge efter og matche mapper til de certifikat-antagelser, der er i PS-headeren.. Disse mapper bør vel så kun indeholde propositioner (evt. med flere individuelle meta-antecedenter selv), eller hvad..? Og i øvrigt bør certifikat-mappe-antagelserne vel så også kunne indeholde en filtrering. Hm, angående første spørgsmål (om hvorvidt det bare skal være prop-lister), så nej. Termet "certifikater" refererer jo netop til, at der skal være en kryptering på, der gør, at prop-bibliotekerne ikke kan ændres på, uden at det kræver et nyt certifikat. Men kunne man / har man så behov for, at ITP'en automatisk kan låse bibliotekerne op og udtrække propositionerne?.. Tja, det ville måske også være fint, hvis man bare kunne sørge for at ændre stien... Åh vent, jeg forvirrer vist lidt meta-antecedent-headerne med meta-antagelses-listen igen... Jo, men hvordan skal headeren så ændres --- og kan der gøres noget automatisk? Hm, hvis nu man sørger for at certifikater altid er tilknyttet et vist ID... Hvor det så må være brugernes ansvar helst at undgå kollisioner... Så kunne man bare angive det id i sin header, i stedet for en given mappe, og så sørger ITP'en, via meta-antagelserne om samme, at matche det id med en mappe (plus et filter-format) i stedet. Hm, og så bør brugere vel sørge for kun at bruge generaliserede programmer til at udtrække propositioner fra denne mappe? Hm, eller brugerne kunne definere et program til hvert certifikat-format, der handler om at tjekke certifikat og udtrække propositionerne, som ITP'en så slev kan gemme i et bibliotek, som så har den relavante meta-antecedent i sin header... Hm, skulle man mon så definere en bestemt slag meta-antecedenter, der så skal igennem denne proces, før de kan bruges; ville det give mening, eller...? (Jeg har det lidt som om, der er en mere simpel løsning, og måske endda en jeg lige har glemt lidt..) Hm, det giver da meget god mening. Så handler certifikat-antagelserne/-antecedenterne lidt om at sige, "følgende program resulterer i en korrekt propositionsliste (med eventuelle yderligere meta-antecedenter inkluderet), for alle filer der findes på denne maskine, og som vil findes i en nær fremtid" (og så eventuelt med en faktisk time-to-live-dato tilknyttet). Hm, dette er så en antagelse mere end en antecedent, for antecedenterne skal jo ikke bruges til dette... Men der kunne så være tilhørende meta-antecedenter, der... Nå, nej det er jo faktisk meget godt ikke at sætte unødvendige antecedenter på, for det kan jo sagtens være, at samme propositioner og/eller metoder bliver vist på et senere tidspunkt... Hm, giver antecedenter så overhovedet mening i PS'et..?? ...Til andet end at loade biblioteker og fremhive individuelle propositioner fra dem (ved en automatisk søgning)..? Tja, og til at sikre sig reflesion-regel-specifikkationer/-tilladelser.. Ja, det virker til at give mening nu, sådan her.. 




% Assumptions, then. Certificates and folder props.. 
% Hm, man skal vel have et helt skema af placeholders for metaantecedenter (jeg jeg lige indset), så kunne man ikke godt lave en måde altid at kunne kollapse dem selv i main-sproget?.. Ah, jo hvis brugeren altid kan definere sine egne sammensatte meta-prædikater (ligesom man kan for normale prædikater).

% Also assumptions about external programs, compiler acceptance, methods and proposition acceptance. 
% "... meta-antecedent should also be able to be translated into the reflected model (where the mata-predicate formulas can just be treated like any other atomic formula)..."
% There should be different levels to the antecedents... (Compiler assumptions should not rely on questionable methods.)


%
%The checks can thus both be about what is available in terms of the application and the system it runs on and also be about what the user has allowed for the ITP, which can then include any kind regular settings the ITP might provide (such as how much space on the hard drive and how much memory is allowed to be used etc.), as well as something a bit more interesting, namely assumptions about certificates. The point is then that if the assumptions, which we can call `meta-assumptions,' are formulated mathematically, that means the users can actually deduce specific things about the ITP in terms of its reflection rule capabilities...
%

%The circumstances that the ITP needs to check is not necessarily something that is 100 \% known to be a fact, and I will therefore refer to them as `assumptions', i.e.\ as `meta-assumptions.' So by this term, I basically refer to what the ITP has been allowed to assume about itself and its own capabilities, so to say, as well as what the user has assumed to be true without having first proved it (or, as we will see, having proved it in a previous session and relying on the files having not been changed since then). ...



%\footnote{I will use `antecedents' here, but one could also use assumptions instead (as in `meta-assumptions' etc.).}

%PS-headere, hvor ITP'en selv tjekker forholdende (og noget af denne header må vist gerne forekomme i bib-filerne).
% Headeren skal aldrig \emph{sætte} nogle antagelser; den skal altid kun spørge ITP'en om tilstande (eller bruges til at sikre, at fremtide tjek bliver gjort (for fodnote-antecedenter især)). At sætte tilstande skal ske for sig i ITP'en, og under en slags advenacerede indstillinger for at beskytte brugere. Herved sikrer man sig også, at man altid bare kan trykke play på ukendte PS'er, givet at man stoler på sine egne indstillinger. Nævn i øvrigt også, at man (mon ikke?) også kunne lave hukkommelses og lagerpladsrestriktioner via meta-antecedenter.

%Certificates can also be used for closed source programs, or programs of a incompatible instruction set. This can both be used for rewrite rules and for compilers...
% We do not need only one fundamental IL anymore.. (Men det er nok klogt stadig at starte med et. Og nævn, at man også kunne starte med et højere-niveau-sprog, men jeg tror mest på at starte med et meget lav-niveau-sprog (fordi man allegevel plejer (mener jeg) at definere semantik for et højere-niveau-sprog via dets kompilering (men jeg ved jo ikke, om det altid er sådan, og/eller om det altid er bedst..)))
% And programs can also maybe ask for compiler/hardware/OS/ITP-version/ITP-settings specifications.
% At programmer gerne må fejle, og at ITP'en gerne må kunne try'e nye regler (i en senere-end-første version), hvis programmer fejler. (Og nævn, at jeg har overvejet mere komplicerede ITP-regel-prøvnings-procedurer, men at det ikke giver så meget mening alligevel.)
% Husk at certificat-antecedenter godt kan være gjort brede, så man kan bevise mere specielle certifikater fra en mere generel antagelse.
% Headere skal også navn-transformere fil-objekter.
% ITP'ens / brugerens egne certifikater (måske asymetriske).
	% A thing related to initializing files: one should also be able to initialize directories with all its file of a certain format (e.g. with certian file endings). These should also be objects sort of like the binary objects, and both the PS and the IL should have rules/methods to initialize new binary objects from a directory object. (Nu har jeg nævnt folder objects i program-regel-sektionen)
% Apropos, nævn også, hvordan fil-objekter godt kan referere til fil-intevaller.
% "Jeg kan i øvrigt også nævne noget omkring at slå propositioner op og om mapping af p-biblioteksmoduler." Hm, kunne man nævne omskrivning a biblioteker i denne forbindelse, eller skal det være i forrige sektion? Hm, hvad med bare at nævne det kort her? Ja.
%""Metode-antagelser kan med fordel formuleres i en reflekteret model" (fra papir-noter). (Her refererer "metode" altså til de "metoder" jeg jo nævnte i vision-sektionen.)"
% Husk også det med en mulig præprocessor.
% Og husk at nævne flere ting om prop-biblioteker, bl.a. at de skal kunne indeholde inputfilerne, hvis jeg sletter det, jeg allerede har skrevet. ..Hvilket jeg gør / har gjort.
% "... meta-antecedent should also be able to be translated into the reflected model (where the mata-predicate formulas can just be treated like any other atomic formula)..." (tjek)

% Vi kan forresten kalde det assumptions, når det er dem brugeren sætter, og antecedent, når det er dem som ITP'en skal OK'e. Hm, er der mon et bedre navn end antecedenter, så?.. Nej, antecedenter virker fornuftigt nok.

\subsection{Other points overall about this kind of ITP and its uses}
... *[(11.03.21) Here, I just wanted to write a few additional points about the ITP overall and to perhaps mention a few things that I had not included in the above text --- and perhaps also to reiterate a few point that I had.]

% Omfortolkning inde i child modeller.
% Transformering til og mellem eksisterende ITP'er.
% Gentag (hvis jeg har nævnt det før?): Når jeg har sagt "brugere," så er det tit eksempelvis main-programmører, der laver de første child ITP'er: Der er ingen grund til at gøre unødigt arbejde i main, når man alligevel vil sigte mod at have det vigtigste interface som en child ITP.
% "Perhaps: On how these design details does not offer anything that cannot be achieved with existing ITPs in theory, save for efficiency perhaps, but even so, might still make a very good fundament, mainly due to: Efficiency, the ability to make use of other technologies and to communicate between them (which, by the way, can maybe also be said for programming languages) and having certificates and trust levels as an integral part of the ITP."
% Det med at mere intuitive beviser, og at det jo vil være en succes, også selv hvis det kun f.eks. virker for vist diskret matematik. Uh, og for applied matematik som et andet godt eksempel. 
% "Jeg skal huske at nævne, at renderingsmodulet gerne må være simpelt, for så kan brugere også vurdere hvor fornuftige diverse grafiske representationer er, og man kan i det hele taget gøre renderingen mere modulær."

%Slut af med (ny subsub-sektion) F-IDEs og semantisk web (og nævn noget om åben brugerdrevet web), og læg gerne bare op til, at det skal komme i nogen af de senere sektioner. (Husk at jeg måske har lovet at komme ind på bl.a. sem-web.)



\subsection{Why I finished these notes a bit early}
(11.03.21) I got an idea, that in a way is not a big change from what I ended up having in mind, just before I realized this new approach, but the change it quite fundamental, so I have to rewrite these notes to make my new idea clear. Otherwise, the notes would be to confusing. I of course would already have to rewrite them in a better structured and better formulated version anyway, but the point of these notes were to explain my ideas in a coherent (albeit not necessarily super neat and concise) way (as it is a very good exercise when having developed an idea to some stage of maturity at least), but this is no longer possible unless I rewrite some of the previous sections of these notes.

I have just written some notes as LaTeX comments in Danish, so I think I will just include these now:
%\footnote{
%	It looks a bit ugly (I just used \texttt{\textbackslash texttt\{\}}), but it is at least kept within the page, except one word, which is `funktionalitet.'
%}

\begin{sloppypar}
{\ttfamily
\% (10.03.21) [...] Og det bliver altså en rigtig god måde at gøre det hele på. Så skal der altså bare være en ITP fordelt på et antal mapper, som kan være udformet på mange måder (for den kan nemlig også altid udskiftes), men den skal altså så have et deduktivt system, hvor den er i stand til at beskrive og analysere et IL, som faktisk ser mapper og filer, som de er (og laver først abstraktioner over dem (som i binære og mappe-objekter), så navne og meta-data bliver skjult osv.), og som beskriver og er i stand til at analysere selv ITP'en selv. Dertil skal den så også have en måde at erklære meta-antagelserne om disse mapper, altså selv den mappe, den selv ligger i. Disse meta-antagelser skal så resultere i, at man ITP'en, der selvfølgelig også skal have refleksionsregler til at køre IL-programmerne, kan køre programmer efter opstart for at udtrække og/eller verificere propositioner. Meta-antagelserne bør altså beskrive ét eller en mængde af programmer, som så er ITP'ens opslagsprogrammer ligesom. Det essentielle er så, at ITP'en så kan initialisere programmer, som lægges i mappen --- og hvor deres IL-kildekode råder over hele eller en delmængde af selve ITP'en mappe-mængde --- og som både ITP'en og brugeren (altså også imens at omtalte ITP-program er lukket) så kan give sig selv lov til at køre, ved først at vise, at dette program overholder meta-antagelserne om mapperne, både idet det oprettes i en mappe, og idet det efterfølgende køres et vilkårligt antal gange (med vilkårligt input (som nu btw ikke modeleres via input-objekter, men hvor programmer bare fra start ikke er deterministiske, men hvor programmerne hele tiden kan få input fra, hvad man kan se som orakler i modellen)). Og eftersom IL'et er stærkt nok til at beskrive ITP'en selv, så man sågar initialisere søskende-ITP-programmer på denne måde, i sådan en grad at man potentielt kan have at ITP'erne helt har "glemt," hvem der initilaiserede hvem; og man kunne måske endda have startet med to. Søskende-ITP'en skal så (per definition) og varetage en meta-antecedent-liste (osv.), og kan så selv til hver en tid verificere, at denne er i overensstemmelse med den anden (måske originale) søskende-ITP's meta-antagelser. Meta-antagelser kan endda være af højere orden, således at man godt kan tillade, at der sker ændringer i mappen, som så kommer til at definere nye meta-antagelser. Særligt kan en ITP således give staffetten videre til en søskende-ITP, eller en barne ITP, som jeg kommer til straks, således at brugeren kan foretage ændringer i meta-antagelserne her, og hvor disse ændringer automatisk propagere videre til de andre ITP'er. Så de nye ITP'er kan altså sagtens være på lige fod med de gamle, men man kan så selvfølgelig også oprette en barne-ITP, som så altså kun råder over en skarp delmængde af den forrige mappe-mængde. Og så kommer endnu en vigtig pointe, for det betyder så, at enhver af disse ITP'er, som kan bruges som et udgangspunkt for sådan en ITP-familie, kan så altid i princippet selv være et barn til en yderligere ITP-forælder, uden at "vide det" selv. I praksis betyder det så, at brugere altid i princippet kan eksportere hele deres ITP-mappe-mængde til f.eks. en ny, opdateret ITP (med et opdateret IL-sprog) eller til andre brugere, hvor disse mapper indsættes som undermapper i en eksisterende ITP-mappe (hvor meta-antagelserne så selvfølgelig skal tillade, at man gør dette). Og på denne måde, at har vi ikke behov for noget grundlæggende deduktivt system, for alle deduktive systemer (med disse kvaliteter) "ved ikke" i princippet, om de bare selv er en barne-ITP til en anden ITP, som er baseret på et andet deduktivt system. Barne-ITP'er behøver så som sagt ikke, at kunne det samme som forældrene, hvilket så også giver fin anledning til (som antydet), at ITP'er kan opdateres ved at smækkes ind som børn til nye, mere kraftfulde ITP'er. Samtidigt behøver børn heller ikke have de samme kvaliteter, som grund-ITP'en har, men kan sagtens være meget forsimplede versioner af ITP'en. Ja, faktisk vil det være en god idé generelt at arbejde i en mindre kraftfuld barne ITP end grund-ITP'en, fordi man så med større sikkerhed kan lave antagelser om, ITP'en fungere korrekt, simpelthen fordi der så vil være færre sikkerhedsbrister (hvad der kan være, fordi antagelserne ikke behøver at være 100 \% korrekte, især ikke til at starte med (hvilket jeg vil komme mere ind på lige om lidt)). Ja, og de programmer, der initialiseres af de kraftfulde ITP'er i familien (eksempelvis af det der var den første ITP), behøver slet ikke at overholde nogen af de samme restriktioner; det er helt op til brugeren. ITP'erne kan således initialisere alle mulige programmer. Det eneste der kræves af programmerne er som sagt, at de ikke bryder meta-antagelserne. Fordi IL'et så skal være så kraftfuldt, at ITP'en kan beskrive selv sig selv, så åbner det bare op for, at man sågar kan lave tilsvarende søskende-ITP'er (og barne-ITP'er selvfølgelig), men brugeren har altså fri mulighed for også at initialisere alle mulige andre programmer. Lige for at slå meta-antagelserne mere fast, så er de propositioner om, hvad der kan forventes at forekomme i mappen, både umiddelbart efter opstart og i løbet af kørslen af ITP-programmet. De beskriver altså en invarians, som både ITP'en selv skal overholde, og som brugeren skal overholde ved ikke at lave forbudte ændringer. Man kan derfor se det som en slags (ikke-bindende, men vigtig) kontrakt mellem bruger og ITP. Brugeren kan så forstærke sikkerheden i ITP-mapperne på forskellige måder, bl.a. ved at have "skjulte" mapper, som er placeret et sted, hvor brugeren ikke regner med, at noget eller nogen (inklusiv brugeren selv selvfølgelig) ændrer tingene i den mappe. En anden vigtig ting, som brugeren kan gøre, er at bruge certifikater. Med disse kan brugeren have en hel mappe-mængde, hvor brugeren frit kan smide filer ind, også ukendte filer (med de sædvanlige sikkerhedsforanstaltninger om ikke at sikre sig mod at downloade malware til sin computer), hvor ITP'en så kan tjekke eventuelle certifikater (også efter eventuelle data-tranformationer; filerne kan således f.eks. godt være komprimerede på en arbitrær måde, uden at meta-antagelserne skal kende komprimeringsmulighederne på forhånd) og så hive propositioner direkte ind via deres opslagsprogrammer (selvfølgelig med en passende konvention om at vedlægge passende antecedenter, så der holdes styr på probability/security/certainty/acceptance levels på en passende måde). Til sidst skal jeg lige nævne et par ting, bl.a. at der selvfølgelig så også gerne må ligge en et compiler-program i en af ITP-mappen. Og det bør så selvfølgelig være en (ret essentiel) del af meta-antagelserne, at dette program skal køres på IL-programmerne, hvorved en executable bliver dannet (med et passende navn), som brugeren og/eller ITP'en så kan give sig selv lov at bruge. Jeg tænker at det er en fin idé at sørge for, at ITP'en holder styr på de kompilerede programmer og så selvfølgelig også har mulighed for at køre dem. (Man kunne så i øvrigt både give sin ITP en kør-og-vent-funktionalitet samt en kør-og-luk-selv-funktionalitet, således at man også f.eks. kan bruge sin hoved-ITP til at launche andre programmer, hvor ITP'en så lukker selv med det samme.) Og ift. "eksterne programmer," så kan det så selvfølgelig også sagtens være en del af meta-antagelserne, at man også kan køre visse executables, også selvom man ikke kompilerede dem selv på standard vis. Og ved brug af certifikater kan man så endda få frihed til at hive certifikerede programmer ned fra internettet og køre dem direkte. Dette kan også være smart, hvis man f.eks. gerne vil køre closed source-programmer. I forhold til at hente ting fra internettet, skal det selvfølgelig også stadig gerne være en del af det omtalte, meget "kraftfulde" (eller hvad man skal sige) IL, at man også kan query'e internettet med det (ligesom jeg nævnte det ovenfor). (Og da programmerne nu kan være ikke-deterministiske, i hvert fald set fra meta-antagelsernes synspunkt, så er det jo ingen vildt kompliceret sag at give semantik til sådanne instruktioner (da man bare kan antage at den hentede data er tilfældig).) (Åh, forresten: Man skal så lige sørge for at formulere certifikat-antagelserne, så de ikke strider imod, at man kan have tilfældige binære objekter i sine mapper. Man skal altså lige sørge for at flette en antagelse om, at der ikke kommer til at forekomme fejlagtige certifikater, med en antagelse ved at der kan forekomme "vilkårlige" objekter i form af de tilfældige objekter, så det ikke kommer til at give en modstrid. Dette kan løses eksempelvis ved at man altid sørge for at huske et "medmindre uheldet med tilfældighederne virkligt skulle være ude," hvor man så på en måde beholder denne antecedent, men bare overser den i alle praktiske sammenhænge (medmindre man går videre og regner på risikoen, hvad man også kunne gøre).) Nå ja, og slutteligt er pointen så, at den pågældende ITP så sagtens kan være udformet, så den passer med alt det, jeg ellers har beskrevet i denne notetekst --- ja, man kunne endda have et mindre kraftfuldt IL, som denne ITP kan bruge til at lave alle de mere interne refleksionregler, som jeg har beskrevet dem. Den store forskel bliver så bare, at man så skal sørge for at beskrive alle de antagelser, der skal til for at redegøre for semantikken og korrektheden af selv ITP'en (og altså alle de forhold, jeg var i gang med at beskrive), matematisk. Men heldigvis behøver man dog ikke at vise dem alle sammen, før man lancerer første version af ITP'en. Her kan man sagtens give sig selv den samme frihed, som man ville gøre ellers, til at konkludere de ting omkring kildekoden til ITP'en, man skal bruge. Man skal så bare lige sørge for at formulere disse antagelser matematisk, og helst på en måde, at de kan bevises på et lavt niveau senere (og med "lavt niveau" mener jeg, at de kan verificeres med en computer senere, uden at bero alt for meget på abstrakte NL-sætninger, der skal verificeres af community'et i stedet for). Okay, så beskriver faktisk allerede den nye form for meta-ITP, som er min idés nuværende udviklingsstadie. Og det virker altså bare super lovende at gøre det på den måde; det er jo ret fantastisk, at man bare helt kan undlade stort set alle bekymringer om det grundlæggende deduktive system, sådan bare lige. (Ja, det er altså på nær at meta-ITP's deduktive system skal indeholde refleksionsprincipper, der tillader den at køre programmer, og man skal også gå med på tanken om meta-antagelser, men det ville jeg jo kræve alligevel.) [...] Men ja, virkeligt fedt, alt sammen..!!
}
\end{sloppypar}

So there we are. I will get to rewriting these notes as a new separate set of notes, maybe as the next thing, or maybe I will take a break and write more about F-IDEs and a semantic (and more open source) web first (I think so). There are other comment notes that I have not included here, but I keep backups, so these comments will not disappear. I think I have a good enough overview of my ITP idea in its current (and hopefully somewhat final (in terms of the core idea)) state, so I should be able to take a short break and write a bit about other things before getting back to it again.  





% Man man ikke også have præprocessorer? Det tænker jeg.. For så kan højere-niveau-sprog compiles til forskellige ILs, og compiles videre med forskellige compilers derfra også, for den sags skyld (men det har vist ikke så meget med en præprocessor at gøre). Ja, det må helt klart kunne lade sig gøre, og sikkert være en meget god idé.
% Jeg kan i øvrigt også nævne noget omkring at slå propositioner op og om mapping af p-biblioteksmoduler.



% Jeg skal også i det hele taget tilbage og snakke om proof script og certifiketer og accaptance landskaber. *Uh, og compiler-certifikater gør også, at vi ikke behøver ét fundamentalt IL..
% Jeg skal også snakke om at uddele proof scripts, hvor der mangler visse inputs, men hvor eleverne (eller læserne m.m.) skal køre hele scriptet og fylde ud, hvad der mangler.






		% Metode til at omskrive regel-række af samme klasse, hvor de subsituerede regler kan bruge output fra de foregående regler som input (hvilket er smart til recovery, til interactive programmer -- både til at gøre input-filer kompakte og saniterede og til at skære renderings arbejde fra -- og self. også til algoritmer, hvor ikke-deterministiske algoritmer slår de deterministiske). (tjek)
% Det med at kunne køre beviser, hvor input manger.

	% Man kunne så også lige nævne personlige certifikater, for at obfuskere bevis.. (måske). *(Giver vist ikke rigtigt mening...)
		% Om at man selvfølgelig skal bygge higher-level languages og compilere til disses (til IL'et). (tjek)
% Meta-antecedenter og fodnote-antecedenter, særligt om mappe- og fil-antecedenter, om compiler- og hardware-antecedenter og om certifikat-mappe-antecedenter. (Og ift. fodnote-antecedenter, så at det kan bruges i andre brugeres meta-tjek, og særligt også til at danne tillids-/acceptance-distributionen.) Også om muligheden for en præprocessor og for IL-funktioner til at spørge main ITP'en om compiler-/hardware-settings m.m. *(Husk her også at beskrive regler til at hente sætnigner fra tidligere biblioteker.)
% "Certificates can also be used for child applications! (and other programs)" med fokus på "other programs." *Og mere om certifikater generelt, og særligt om compiler modul.
% "Pointe om at lave de formelle omfortolkninger til andre teorier inde i en child application ..."

%Tilbage til hvad kerne ITP'en kan have af metoder, og at man evt... Nej, men måske det, at programmørerne kan "tyvstarte" på (en eller flere) reflekterede modeller. 
% Perhaps: On how these design details does not offer anything that cannot be achieved with existing ITPs in theory, save for efficiency perhaps, but even so, might still make a very good fundament, mainly due to: Efficiency, the ability to make use of other technologies and to communicate between them (which, by the way, can maybe also be said for programming languages) and having certificates and trust levels as an integral part of the ITP.

% Og så om F-IDE og programmerings paradigme og måske om semantisk web (eller dvs., det skal nok være i en anden sektion, men så må jeg lige rette det i resten af teksten, hvis jeg har lovet det).






% Husk:
		% - Hvornår automation rules kan adopteres (og altid uden ikke-viste antecedenter). (...tror der er tjek på)
		% - Hvordan proof scripts virker (har allerede nævnt at det er actions mere end kodning for visse brugere). (har nævnt, så tjek)
		% -- Reflection rules in general. (F.eks hvordan ting skal representeres som binære objekter som noget grundlæggende, og hvordan konceptet om memory-addresser skal være med, men at faktiske addresser alrdig må indgå; aldrig må blive leaket til brugeren (mest fordi beviset ikke skal afhænge af disse).) (Også hvordan man måske kan have flere typer af programmer, men at det så alligevel kun handler om: program der transformerer hukommelse, program der læser hukommelse og skriver til en anden hukommelse, og så eventuelt blandinger af disse.. *(plus filer)) (tjek)
	% - Library files and so on... (mangler scripts..) (har vist nævnt)
		% - Other kinds of reflection rules, e.g.\ methods from the first application (with its extensions of the fundamental theory). *(ikke rigtigt nødvendigt, for det hele bliver binært og dermed fra grunden af ret hurtigt. Men derfor kan ITP programmører jo godt begynde at bygge et miljø i en reflekteret model.)
% - "Men derfor kan ITP programmører jo godt begynde at bygge et miljø i en reflekteret model."
		% - How to prove the programs and make the rules. (tjek)
% - Why one should not begin with a higher-level language such as a subset of C.
% - Certificates can also be used for child applications! (and other programs).
% - Pointe om at lave de formelle omfortolkninger til andre teorier inde i en child application (så altså inde i en model af selve teorien).
		% - Metode til at compilere input fil til proof script! (behøver jeg ikke at nævne længere)
		% - Giver det alligevel mening at fejle? For andre ting end for PS-omskrivninger? *(Ja, og det skal jeg lige have skrevet om, men det står også i et punkt nedenfor nu.) (gentaget nedenfor)
% - Bør brugere ikke bruge asymmetriske nøgler til at give certifikater til deres egne applikationer..? (joh..)
		% - I det hele taget, skal jeg lige snakke mere om certifikater, for det er jo ret vigtigt. (jaja)
% - Færdiggør reflection-sektion. (tjek) *(Men færdiggør lære-elev-fjernet-input-paragraf.)
		% - Ret det med propositionsskemaer i AreProvable-klassen; vi behøver allerhøjest skemaer for grundaksiomerne, for man kan bruge AreProvable(x)-sætninger til alt andet. (tjek)
% - Måske bør jeg skrive lidt om at slette og sammensætte gamle input- og propositions-filer.. Ja, det bør jeg nok. (Og her kan man bl.a. bruge regelomskrivninger.) (mangler om prop-filer)
% - Man skal i øvrigt også kunne gå ind og ud af mapper i PS'et for at klargøre fil-objekter, og det er så her, altså ved at vælge den rigtige mappe, at PS'erne gerne skal ændres fra bruger til bruger, når de deles, for at det passer.
% - Husk at tjek at "et fil-objekt kan godt referere til et interval i en fil" er nævnt.
% - "Jeg kan i øvrigt også nævne noget omkring at slå propositioner op og om mapping af p-biblioteksmoduler."
		% - Det med redo og undo for child-applikationer. Og også for main PS'et. (tjek)
% - Det med at det ville være godt, hvis lærere bare kan fjerne input fra regler, som eleverne selv skal udfylde.
% - Omkring intuitiv, action-/metode-baseret bevisførsel, måske med grafisk hjælp (hjælpetegninger). Og: Selv hvis det kun virker for diskret matematik (f.eks.), så er det stadig en succes. *("Uh, og for applied matematik som et andet godt eksempel.")
% - Forkellige måder (mere uafhængige) programmer kan spørge om ITP og hardware/OS tilstande.
% - "Metode-antagelser kan med fordel formuleres i en reflekteret model" (fra papir-noter). (Her refererer "metode" altså til de "metoder" jeg jo nævnte i vision-sektionen.)
% - "Husk også det med en mulig præprocessor."
% - "Desuden kunne man også overveje at udvide refleksionsreglerne lidt, så de kunne have to inputs, f.eks. en definitionsfil og en arbejdsformular-fil."
% Programmer skal kunne åbne filer selv osv. Programmer skal også kunne have fil-skrivnings/-oprettelses og pladsforbrugs-permissions som en grundlæggende ting.
% - Sikkert en gentagelse, men husk at advokér for, at man faktisk bruger NL fra start. Pointen er nelig, at man så kan udtrykke sætninger med kompliceret semantisk betydning, selv før at de bliver defineret i sprog-ontologien, men hvor man så alligevel kan benytte sig af semantikken, fordi man så bare kan regne med, at denne bliver defineret på et senere tidspunkt.
%
% Vigtige ting:
% -- Reflection rules. (som lige nævnt) (tjek)
% -- Semantisk web (har lidt lovet det ovenfor, men måske det bliver i en anden sektion..).
% -- Husk det med child applications og særligt de ting, jeg mangler at sige omkring interfaces.. (tjek)
% -- F-IDEs and the new paradigm.








%\subsubsection{More on certificates and individual trust levels}
%Since I have claimed that using certificates and trust/probability/acceptance levels (/distributions), I should also be sure to try to describe a good way that this could be implemented as part of these overall ideas about design details. Once again, I find it important to make the main ITP very open for specifications on how to do this (so that even though the programmers of the main ITP might also implement a specific way to do this, the users should have the freedom and ability to overwrite these specifications with their own procedures). 
%
%The solution I would suggest is therefore to rely on users own programs to generate the `trust qualifier distribution', let us call it that for now. I thus imagine that the ITP will be able to assign `trust qualifiers' to each proposition as well as each proposition scheme. The users should then be able to define their own set of trust qualifiers, which each can be defined with a specific set of numeric parameters, such as integers, floats or binary values. For proposition schemata their might even be a formula for how to distribute the trust qualifiers such that the propositions in the scheme might each get different parameters. I will get back to this, as I have not yet figured out how this could work... %Hm, hvad egentligt med at gøre dette til en del af udtrækningen fra AreProvable-klassen?... Hm, eller måske skulle man bare angive dem med et program, hvor proceduren så er at oversætte til AreProvable-klassen først og så trække de individuelle propositioner ud... Hm, men skal dette mon så ikke bare være sådan, at man sætter TQs generelt? For man skal jo alligevel hive individuelle propositioner ud fra filer... Hm, kunne man så gøre det så at individuelle propositioner får faktiske TQs, og AreProvable(x)-propositioner kan så tilmed få tilknyttet et program, som kan trække endnu flere TQs ud, når propositionerne trækkes ud. Hm, men bør man så egentligt ændre det mere grundlæggende for så at sige, at der ikke er en grundlæggende AreProvable-klasse, men at brugerne altid selv skal definere programmet til at trække propositioner ud..? Jo, men hvordan gør man lige det? Der skal jo altid være en grundrepresentation, som main-ITP'en altid kan forstå og aflæse. Så i stedet må det jo bare handle om, at man tilføjer TQs til AreProvable-klassen, er det ikke sådan? Jo, bum. Og hvad så med propositionsskemaer? Behøver man overhovedet propositionsskemaer rigtigt, og særligt, behøver man TQs til propositionsskemer (når nu man kan bruge AreProvable(x)-propositioner i stedet)?... Svaret er vel nej, for vi behøver jo ingen gang aksiomsskemaer i AreProvable-klassen (må jeg lige rette i ovenstående tekst også). Man kan nemlig bare definere skemaet som en procedure til at opnå en mængde af korrekte (start-)AreProvable-onjekter i stedet for at prøve at starte med kun ét AreProvable-objekt. Så vi behøver i det hele taget allerhøjest (hvis vi bruger FOL) skemaer til grundaksiomerne i main-ITP'en (hvilket så faktisk bør implementeres som en slags regler, så man ikke blander skemaer og normale propositioner sammen). Bum. Nice. ... Okay, lige efter, at jeg skrev dette, hvilket faktisk var i forgårs (jeg holdte helt fri i går (i dag er det d. 22/2)), indså jeg, at der nok var mere at tænke over. Spørgsmålet er, om ikke det kommer til at ændre (for) meget ved programmerne på tværs af brugere. Så nu tænker jeg lidt over det. Jeg er lige kommet på, at mon ikke man skal gøre metode-permission-tjek til en del af IL-programmerne..? Nej, måske var min løsning egentlig god nok: Gør TQs (eller hvad vi skal kalde dem) til en fundamental ting, som så også skal med i de reflekterede modeller. Hm, så handler det vel om, at man... udformer prædikater, som programmerne skal overholde.. Hm... Tja, man kunne jo også lave det i et lag over.. Hm, det ville så kræve, at man kan finde en god måde at tranformere TQs i form af en slags antecedenter.. Øhm jah, nu hvor jeg tænker over det, så er det vel det mest pragmatiske at gøre. Det er naturligt, at antagelser tages ved at formulere dem som antecedenter.. ...Ja, og man kan transformere antecedenter til ækvivalente eller til løsere krav, f.eks. kan man således tranformere en specifik antecedent over til en mere bred kategori om, at beviset overholder restriktioner indenfor en mere bred kategori. Særligt kan man så f.eks. udlede at beviset for en vis proposition overholder en vis (mere bred) regelsætning (fra først at have vist en mere specifik regelsætning), der gør at propositionen godt må indgå i korrekthedsbeviser for programmer (som et eksempel, hvor der så må være en ret stor tillid til bevismetoden). *(Tja, eller måske skal det behandles lidt for sig, det med hvilke programmer ITP'en må køre. Ja, det er nok bedst.. Hm, nå ja, men man kan vel sige lidt det samme alligevel..)
%% Så skal jeg lige tænke over, hvordan det vil fungere, for kan alle TQs så være en slags metode-antagelser/-antecedenter? Hvad så med f.eks. compiler antagelser (og er der flere ting?)? Nå ja, og skal skal også stadig finde ud af, hvordan man sørger for, at PS'erne ikke ændres alt for meget af det fra bruger til bruger, hvis det kan lade sig gøre (hvad det helst skal).
%%Okay, jeg tænker, at compiler-antagelser osv. (f.eks. hardware-antagelser m.m.) skal have en særegen måde at skrives op på, så de kan skrives som en slags specielle antecedenter. Proposition-biblioteksfiler skal have en eller flere headere med antecedenter, og jeg tænker at brugeren så godt må løsne disse undervejs, hvis denne gerne vil tilføje flere sætninger til, der dog kræver lidt bredere antecendenter (i.e. der tillader mere ift. konsekventen (og altså dermed holder flere antagelser i sig)). Jeg tænker så, at disse antecedenter så også skal holde udsagn om, at andre filer/biblioteker er viste. Og så må det bare være standard, at verifikationsmodulet så også sættes til faktisk at vise disse automatisk (hvis det ikke er gjort). Ja, så det kan godt være, at disse antecedenter også skal gøres lidt specielle. Hm, hvilket måske også alligevel er nødvendigt, fordi de jo sikkert skal bruge de faktiske fil-navne..! Ja, okay. Jamen, er det så ikke nærmest bare det? Skal man så sige, at AreProvable-klassen også skal holde disse headere; er vi mon så ved at være fremme? Ja, det kan man ligeså godt sige, at de skal, og så kan man endda undlade dem i de ydre biblioteksfunktioner, for man får dem alligevel altid tilbage, når man transformere ud igen. Og da man skal transformere ud, før man kan compile programmer osv., så er det jo også helt ok. Hvis man så får flere antagelser, inden man transformerer tilbage, så må man jo bare få en konjunktion af dem, som man så selv kan forkorte på forskellige måder. Hm, men hvad så med fil-antecendanterne, hvor filnavnene indgår..? Tja, dem kunne man jo nok bare undlade at trække ind..? Ah, og for certifikater bør man næsten kunne sige noget om en hel mappe ad gangen, så man kan sige, "alle filer i denne mappe opfylder følgende: Hvis de har sådan og sådan certifikater, så gælder sådan og sådan"...!! Og det er sådan, at vi endelig får de certifikater i spil (på en måde, der endeligt giver mening)! Og så, for at vende tilbage, skal mappe- og fil-antagelser ikke at kunne hives ind i reflekterede modeller, og måske er det heller ikke nødvendigt for visse andre specielle antagelser, men det kan jeg jo lige tænke lidt over. *(Nej, compiler-antagelser (m.m.) må faktisk ikke trækkes ind, da man skal sikre sig, at det altid er gjort af brugeren selv. De skal derfor være i et yderliggende lag.) *(Jo, det kan man godt: Ja brugeren skal selv antage dem, og så skal det bare være en del af de antagelser, hvilke nogle TQ-antecedenter, der tillades for en given compiler-konklusion, før den kan benyttes af main ITP'en.)
%% Mappe-antagelser kan så selvfølgelig også bruges til at antage korrektheden af tidligere verificerede biblioteker osv. (og her kan man jo også bruge en slags certifikater, så det er jo så fint). Ah, hvor dejligt at jeg endelig har fattet, og har sådan en fin løsning på, hvordan certifiketerne skal implementeres. 
%% Compiler-antagelser m.m. må gerne adskilles meget fra resten (og skjules lidt for brugeren) og bør altså gøres i main ITP'en alene (og kun af brugeren selv!). Man bør dog til gengæld sagtens kunne bruge certifikat-antagelser i compiler-antagelserne, sådan at man løbende kan vise ting omkring, hvad ens antagelser medfører, og så man også kan opdatere compiler-antagelserne bare ved at dowloade filer med de pågældende certifikater (som så vil være designede af ekspert-brugere, som brugeren stoler på).
%% Mappe- og fil-header-antagelser skal bare indebære en transformation fra fil-/mappenavn til fil-/mappe-objekt (med et anderledes ID (som godt kunne være et navn)). Så hvis brugeren skal redigere headeren på et bibliotek, så er får det nok mest at gøre med at ændre fil-/mappenavne, samt fil placeringer, for et fil-objekt kan godt referere til et interval i en fil, så de transformerer til de rette interne objekt-ID'er. 













%As a solution to this matter, I will therefore propose 
%I will therefore propose that the core reflection model, i.e.\ the fundamental AreProvable class from before, is extended to also require that the binary object representing a proposition list now also has to include its own proof script as well. ... Is this necessary..? We need to tranform the user input files/objects... Nå ja, så man kan godt faktisk bare gøre det lidt som med de andre regler, hvor man viser korrektheden af en tranformation. Nice. ...Hm, men måske behøver man ikke nødvendigvis at kræve et bevis... Hvordan kommer det i øvrigt til at blive imellem refleksionsmodeller?.. Nå nej, vi får ikke bare én input fil; vi får en række programregler med hver deres potentielle input filer/objekter til.. Og det giver ikke rigtigt mening at have disse regler med i modellen selv, så... Tja, eller vi kan godt have én input fil og én regel, der skal omskrives, men vi vil vel også gerne kunne omskrive (meget) mere end det... Hm, nå ja og input objekterne vil bare beholdes i hukommelsen indtil man gemmer, så der er masser af tid til at omskrive inden de ryger ned på harddisken, også selvom det bliver håndteret af main applikationen... Så bliver svaret ikke, at man får mulighed for at tage proof script udsnit med tilhørende bruger-input objekter (og generalt al input, der ikke forekommer i nogen af konklusionerne) og tranformere dem? Hvornår skal disse tranformationer så køres? Tja, men kunne jo have programmer, der skal køre inden brugeren får lov at læse sin egen input-fil, der så reducerer og saniterer program-input for hver kørsel (altså lige efter at programmet er kørt færdigt), og så også have programmer, der kan køre med større tidsintervaller imellem dem, og som kan reducerer flere regler (med flere bruger-input-filer) ad gangen, og som så særligt bør køres, når brugeren gemmer. Hm, og måske gør det ikke noget, hvis man siger, at de simpelthen også bare skal bevises ligesom de andre programmer. Men hvad så, hvis diss programmer nu fejler; hvordan gør man det lige sikkert, så brugere ikke kommer til at skulle gentage beviser..? Hm, problemet er vel her, at reglerne jo godt kan komme til at implementere meget langvarige programmer med tilsvarende meget input, så der er sikkert for farligt bare at stole på, at programmerne altid bliver udført korrekt (de kan jo løbe tør for hukommelse, eller applikationen kan afbrydes). I det hele taget er det et problem vis man bare hakker derudaf i hukommelsen og ikke kan rede, hvad man allerede har opnået, hvis man løber tør og/eller at applikationen (også på anden vis) chrasher eller bliver lukket. Hm... Ah, det skal bare være sådan, at programmer kan give tilladelse til, at memory- og fil-objekter skal kunne være tilgængelige for brugeren, basalt set, efter at programmer fejler eller at applikationen chrasher eller lukkes pludseligt. Disse fil- og hukommelses-objekter vil så ikke have nogen sætninger vist om dem, man kan så bruges som input i nye programmer. Det er så meningen at man skal bruge dem i særlige recovery-programmer, som så kører det oprindelige program delvist ud fra, hvad der er gemt i recovery-filerne, hvilket gerne bare må være et proof script, og så når den ønskede tilstand er nået (som gerne skal være cirka, hvor man slap), så skal det så skifter over, så det igen kører ud fra en bruger-input-strøm. Hm, kunne man gøre dette en anelse smartere ved at lægge mere op til alt dette, og give et godt format for, hvordan man genererer disse recovery-programmer ud fra de originale?.. Nå ja, inden jeg lige prøver at finde et svar på det, så skal jeg også lige nævne, at det så selvfølgelig er meningen, at recovery-programmet så også skal transformeres bagefter, så bevis-scriptet ikke nødvendigvis kommer til at bære præg af komplikationen, og i hvert fald så at al råt bruger-input når igennem en saniteringsproces så hurtigt som muligt og kan smides væk igen efter recovery. Hm, men det er vel så også i det hele taget oplagt, at programmer bare holder fast på bruger-input og så bare har en sideløbende process i gang, der hele tiden komprimerer dette og skærer unødige ting fra. Det svarer jo så til, hvordan main ITP'en også virker, hvor hver gyldige action bliver optaget som en regel-instruktion i proof scriptet. Ja, det må være rigeligt, både ift. komprimering og ift. sikkerhed. Så programmerne bør bare følge et godt princip om, at inputstrømmen altid omskrives over til et anden fil/hukommelsesblok, således at alle gyldige actions bliver skrevet ind og alt andet bliver sorteret fra, princis ligesom at proof scriptet fungerer i main ITP'en. Og så skal det også være en del af princippet, at brugere skal kunne undo'e (og selvfølgelig gerne også kunne redo'e) i disse børne-ITP-programmer. Dette princip bør så faktisk formuleres matematisk, for så kan main ITP'en selv stå for hele recovery-proceduren, uden at brugere behøver at tænke mere over dette. ITP'en kan så bare gøre dette ved at holde styr på proof script-objekterne fra fejlslagne eksekveringer og så køre recovery-programmerne (på brugerens kommando), ved at minimalistisk input bliver genereret ud fra proof-scriptet (det reflekterede, ikke main-proof-scriptet, og at samme program så bare kører på dette til at starte med, inden det så derpå begynder at læse brugerens input som normalt. Så ITP-programmerne skal altså kunne opfylde disse restriktioner, hvorved ITP'en så kan forstærke dem med gendannelsesmuligheder. Og hvis vi går tilbage til det med at omskrive proof-scripts, så kommer det så bare til at handle om, at input-filerne som første skridt bare erstattes med dem, der genereres ud fra proof-scriptet. Uh, og idet denne transformation bør defineres som et program, så bør man så også lave en regel, der tager et proof-script til et program og genererer input-objektet (i hukommelsen) ud fra dette og så kører det pågældende program, det hele på én gang. Så kan man nemlig lige spare lidt plads ved at gemme proof-scriptet for barne-ITP'en i stedet for at gemme en input-fil (da der vil være langt færre gyldige actions end der er tastaturkombinationer) *(dette er ikke nødvendigvis rigtigt, hvis der er mange program stadier, hvor samme input fører til forskellige actions). Hvad angår omskrivninger af main proof scriptet på et makro plan, det kan jeg lige tænke lidt nærmere over.. Hm, det er i øvrigt nok ikke så praktisk, hvis brugere skal vise korrekthed omkring at tillade undo og redo. Det er dog et anbefalelsesværdigt princip. Jeg overvejer derfor lidt, om man kunne implementere dette via main applikationen, men det er måske også lidt svært...?
%Hm, ift. at omkrive hele main-proof-scripts, så bør man vel bare kunne give disse til programmer, som så...
% (19.02.21) Okay, mon ikke bare recovery skal fungere ved at have programmer, der læser de efterladte objekter fra fejlslagne eksekveringer, og så genererer en bruger-input strøm. Så kræver det vel bare to ting, nemlig at de efterladte objekter bliver tilgængelige (efter at bruger har meldt dette i deres programmer (via permissions)), og at programmer nemt kan omskrives så de f.eks. går fra at tage én bruger-input-strøm som input til at tage først en regulær input-fil/-blok sammensat med en bruger-input-strøm, der så følger lige efter. *[Tja, måske skal jeg lige genoverveje dette...]
% Og hvad så med omskrivning af input-filer direkte, og hvad med omskrivninger af hele proof scripts, løbende og efterfølgende?
% I forhold til det førstnævnte så er min umiddelbare indskydelse at løse det ved at... Hm, men det er nok ikke så nemt... Jo... Jo, for i modsætning til IL-programmer, så behøver vi ikke vise noget om proof scripts korrekthed for at benytte dem... Nå nej, men det kan jo nemt alligevel kræve, at man kender programstadiet, før man kan oversætte input actions... Hm, kan vi mon nøjes med kun at oversætte fra... Nå ja, og hvad så, om man skal kende programstadiet; man kan jo køre programmer igen, hvis man kender input. Og man får nok mere brug for transfomationen fra script til input. Hm, men her kan det dog godt være, at man gerne vil køre programmet samtidigt med..
%Hm, nu har jeg vist en forestilling om, hvordan man kan gøre meget af det her. Man bør kunne omskrive enkelte program-/regel-kald ved at erstatte dem med andre programmer, der fører til samme input. Pointen er så, at de andre programmer godt må tilføje deres egne input-filer. På den måde kan man også helt generelt køre programmer til at finde et svar, og så erstatte reglen, der fandt dette svar, med en regel, der kendte svaret på forhånd (på "magisk vis"). Bemærk at dette også betyder, at brugere er helt frie til at gå over til et princip om, at lægge al bevis- og korrekthedsbyrden over på en bevistjekker (et verifikationsmodul) (og nemlig at de så ingen gang behøver at offentliggøre noget af søgningsprocessen efterfølgende). Hvis det er det, de har lyst til. 
% Uh, og det passer rigtigt godt, hvis proof scripts også kan forvente input, når de køres (af verifikationsmodulet vel..?), så man kan lave delvise beviser, hvor den pågældende bruger, der kører det, (eleven eksempelvis) så selv skal udfylde det manglende. Og her kan det så faktisk være smart ikke at reducere input-filen alt for grundigt, så f.eks. at markørpositioner, klikketider, og tastetider (bare for relevante keyboard-input) kan gemmes, og således kan det verificeres, at løsningen er unik og ikke kopieret fra andre (elever f.eks.). Lærer kan så også lade programmet afsløre end nøgle på et tidspunkt i programmet, som skal indsendes til en server, for at få en nøgle, som så skal bruges for at komme videre i programmet. På den måde kan man også verificere at programmet kun bliver kørt af eleven én gang og/eller at tiderne passer. 
% Så lad os gå tilbage til recovery-(gendannelses)programmerne først.. Spørgsmålet er, om man skal lave noget indbygget til at danne disse programmer..? Det tror jeg ikke.. Når brugere laver ITP-applikationer, eller diverse andre applikationer hvor data skal gemmes, så er det naturligt at arbejde ud fra et underlæggende script, ligesom proof scriptet i main applikationen. Hvis man f.eks. lavede et tekstredigeringsprogram, så ville "scriptet" bare være en strøm af den endelige data (uden backspaces osv.), bare for at give et andet eksempel. Så det handler bare om løbende at gemme dette script i binære objekter, som kan læses efter en eksekveringsfejl. (Og her vil det så være naturligt ofte at gemme til et memory-objekt og så med lidt større mellemrum gemme dette videre over i et fil-objekt.) Og så skal man så bare også lave et program til at køre disse scripts, hvilket ville være naturligt at gøre alligevel. Dette program kan nemlig så både bruges til at opstarte et program for at prøve at gendanne den sidste tilstand (fra en fejlet eksekvering), og også til at erstatte reglen, inden beviset deles og/eller sendes over til verifikationsmodulet. Når programmet skal variaficeres skal det altså ikke køre et program, der skal aflæse bruger-input og behandle de objekter, der skal renderes i GUI'en. I stedet kan verifikationsmodulet bare køre et forsimplet program, og hvor input så er erstattet med et mere direkte (og reduceret) script. Så det er altså ret naturligt, det hele. Og i den forbindelse kan vi også se, at det giver god mening, at verifikationsmodulet også bare kører de interaktive programmer som normalt (hvilket som lige nævnt kan bruges i uddannelsessammenhænge), men hvor det så bare er standard at omdanne alle de regler, der bruger bruger-input (til tilsvarnede, men mere kompakte, regler, der kører på scriptet i stedet).
% Og så er spørgmålet bare, hvordan man omdanner ydre proof scripts, hvis man altså overhovedet skal det (jeg hælder nemlig lidt til nu, at dette bare er brugernes eget ansvar at finde ud af).. Nå ja, og så skal jeg også lige tænke lidt mere over certifikater og sandsynligheds-/acceptance-landskaber i det grundlæggende proof script, når jeg når til dette. Hm, ja jeg tænker ikke, at der skal være indbyggede måder automatisk at omskrive proof scriptet i main, udover den nævnte mulighed for at erstatte programregler med andre resultet-ækvivalente programregler. Tja, og måske er det dog en god idé lige at udvide dette, så en række af programregler på samme prædiket-/variabel-formular-klasse godt kan erstattes samlet set, for det er jo meningen, at brugere gerne skal kunne bruge flere forskellige af disse (og det hele er derfor ikke nødvendigvis bare et langt interaktivt program), og i disse tilfælde er der ingen grund til, at man ikke skal have mulighed for at erstatte flere regelkald på én gang, evt. med et enkelt regelkald (eller bare færre). Fint. Men derfra må det være op til brugere selv at bygge programmer til at tranformere ydre proof scripts. Dette må selvfølgelig meget gerne foregå med ITP'en selv, men ITP'en behøver ikke at byde på indbyggede metoder til så at erstatte disse proof scripts for biblioteker; som udgangspunkt (altså om ikke andet så i tidlige versioner) kan brugere selv få lov at gøre dette manuelt. Cool-cool-cool.

%Så kan man vel implementere muligheden for at erstatte programregler og samtidigt køre gendannelsesprogrammer, ved at bruge specielle input-objekter, som så skal læses fra den tidligere, evt. fejlslagne, regel. Ja.






%\newpage
%\section{Opsummering (og udvikling) af mine tanker om formel programmering (og F-IDEs) og et mere semantisk, og mere open source, web (11.03.21--???)}
%Okay, så jeg tror altså, at jeg bare vil lave et sammenhængende notesæt, der forklarer mine vigtigste tanker om alle disse emner. Med denne tekst er det meningen, at jeg lige skal danne mig et godt overblik over, hvad alle disse tanker så gik ud på, og eventuelt (sandsynligvis) udvikle lidt mere på dem undervejs. 

%Jeg skrev meget godt i følgende sektion, taget fra mine appendiks-noter nedenfor (som jeg altså lige kopiere ind igen her):\\
%
%By having the built-in (lower-level) program(s) powerful enough to be able to run interactive programs with graphical interfaces and all, there is then no limit for how the users can basically extend the application. 
%This is of course true for any programming language but in this instance, it might be even more useful, since there is a difference between updating and making extensions for the application itself, which requires testing and peer-reviews, and then just proving some theorems about a certain program. This is one of the big possible benefits from a new mathematical (proof-based) programming language paradigm if it can work, namely the fact that one does not need to test and review updates. If we look at object-oriented programming, we already have a good way of implementing a program in a top-down way and divide it up into modules. These modules can then be documented for their intended function and consequently implemented and tested. But what if the documentation is instead written mathematically, so that the implementation can be proved to have the correct function? At a first glance, all this seem to do is to shift the work of programming the module to the documentation. But perhaps writing a documentation mathematically and writing a rigorous documentation normally might not be too different once the paradigm has developed enough. And when I think about it, I think it would often be easier to describe a module rigorously instead of programming it in the imperative. This might differ from task to task as well as from person to person, but this is why it is a good idea to have several programming paradigms in the first place. Furthermore, \iffalse This is a good way of putting it: Good with several paradigms, and furthermore...\fi it would be possible to compile some of these mathematical documentations to imperative programs, and once the paradigm gets developed, even quite abstract documentation might be compiled this way, at least partially. The IDE could then give suggestions for different design patterns (prepared by other users by having things proved about them) to pursue, where potential gaps to solve, i.e. when a part of the solution is dependent on outside factors that therefore needs a specially tailored proof by the user dependent on these outside factors, are left open. This is why even the implementation itself might get to be a more top-down process with such a mathematical programming paradigm. Imagine writing a rather abstract documentation of what a certain module needs to do and then just choosing between a couple of implementation patterns and then to have the IDE write out the whole template for the implementation as well as the new mathematical documentation for each submodule that needs further implementation if there are any (and where you might even be able to use the same procedure for some of these submodules). I think this can save a lot of work. And since this is all just mathematics, it is very easy to just import any new and useful implementation patterns as these are all just mathematical propositions (so no need for any rigorous peer-reviewing). And then once you are done with the implementation you are done! No need for further testing or for using external theorem provers. This fact alone could save a lot of work even at early stages of the paradigm where it might still be more work to make mathematical implementations rather than imperative ones. Furthermore, when you are done, the product also gets more value since other parties that depend on your program does not need to trust the correctness of your tests or your documentation as the can just run the proofs themselves. As an additional point, a mathematical paradigm might be easier to work with in big projects with a lot of different programmers where a lot of rigorous intercommunication is needed. Such project obviously need rigorous planning and documentation anyway, so why not make this mathematical in order to get rid of any ambiguity. Also, while it is possible to make mistakes in the documentation and for instance forget certain eventualities, a lot of the documentations of a big project will be documentation of submodules and the correctness of the more overall documentation of the program will thus be dependent on these. If it is made sure, that it is first proven that the overall (mathematical) documentation holds if the documentation of the submodules hold (in a top-down fashion), then there is never the risk that individual programmers work on a perceived task that is different from what was intended (or at least that risk is reduced I should say). 
%Oh, and there is another point as well. In an open source community, having mathematical documentations might also greatly increase the value of such programs in that community, since it might be possible to then make it easier to search for a certain solution to a problem at hand. It might de easier for servers to serve the right solutions to the clients once everything is mathematically described, as the servers then might just need to do a little bit of math on their own order to find a good solution. A bit like Wolfram Alpha, I guess, but more rigorous and for programs (submodules and subroutines) as well. I will expand more on the possibilities I see for an open source community when programs have mathematical documentations in the next section/chapter. Oh, and yet another point is of course how it makes updating applications much more easy. With normal paradigms each update to a unit of a program has to be rigorously tested both individually, and should also preferably be tested in the whole context. But once a program has a mathematical backbone all the way through, each unit can then be updated as many times as desired and all that is needed to do before exporting is just to prove that it still fulfills the already defined documentation. So I would imagine that updating could be a much simpler task overall this way. 
%
%So to sum some things up about the paradigm, it means that programmers are intended to use theorem proving rather than tests (but does not necessarily force them to do so, as documentations can be relaxed to conclude a probability of working as intended instead of a yes/no, and then one can make use of assumptions about have testing yields probabilities, which would also help make testing more rigorous in a way, but this is still not really what is intended for the paradigm) and to make these proofs in a top-down way. I know that with only these describing attributes, one could just choose to use this same approach with existing paradigm (for instance with the functional paradigm), but this would still not be exactly the same. The point is that in this mathematical paradigm, the theorem proving becomes part of the regular programming and people can then share lemmas which proves that a certain implementation design pattern yields a certain functionality given some restrictions of some variable submodules with each other and then these sort of lemmas can furthermore be used by IDEs to come up with possible solution routes for implementing a module and then print out any such chosen template so that it is ready to be completed. To make theorem proving the main part of the actual programming itself thus should open up to new way of programming (collaboratively) in my view and might thus constitute a valuable new programming paradigm.\\
%
%Jeg skal lige se, om jeg har noget at tilføje her, men det lød egentligt meget fornuftigt overordnet set, da jeg lige læste det igennem igen nu her. %Hm, det kan faktisk godt være, at jeg lige skal repetere mine blockchain-idéer og også lidt om min forretningsidé, inden jeg går videre, for ellers er jeg bange for, at jeg glemmer noget vigtigt omkring det (dvs. mere bange..). Det kommer forhåbentligt ikke til at tage særligt lang tid; mon ikke en halv til en hel arbejdsdag? *Tjo, det kunne det så, men for at være fair, så samler jeg lige op på hele mullevitten, og gentænker det lidt undervejs, inden jeg så begynder at skrive diverse opsummeringer ind her igen (tager dog nok bare lige et par dage mere). *Ja, det tog lige et par dage mere. Nu er det d. 17., og jeg er lige gået i gang med et notesæt under 'General notes', som også bare "opsamler hele mulevitten," så denne sektion bliver muligvis bare udkommenteret/slettet og så hævet op i denne sektion i stedet.
%
%
%
%% Husk:
%% - At webbet kommer til i høj grad også at handle om, at klassificere brugere, så man kan bruge råd og anbefalinger fra grupper, der matcher én selv.
%% - "Modelérvoks-metafor." 
%% - "Og uanset bevægelse eller ej, så er det godt med et godt donations-/tipping-system."
%% - "NL i sem-web."
%% - Det med at ITP er vigtigt for servere i det mere åbne web.
%% - "Bayes-ontologier er vigtige for brugerdrevent web."
%% - Princippet om en slags omni-side (skal jeg lige uddybe langt mere).
%% - "Og husk: Big data på en ny (anonym) måde."
%% - "Sem-dokumenter."
%% - Det med at få det sådan, at brugere ejer (..selvfølgelig) deres bidrag, så de også nemt kan trække det væk, og dermed har en handlekraft ift. donationer osv. (og "fagforeninger").
%% - "Domsmænd og trust-grupper på sem-web." ..Giver næsten lidt sig selv..
%% - I forlængelse af forrige punkt: "Det behøver ingen gang at være så avanceret, for når "kontrakterne" er for grupper, så kan grupperne bare samle på (og kæmpe om) de mest troværdige medlemmer."
%% - Fra mine papir-noter er også et punkt, der lidt kortere kan skrives: "Semantiske (og dynamisk opdatérbare) moduler og interfaces til konventionelle applikationer også."
%% - (Også lidt omskrevet:) "Big data i forbindelse med science."
%
%
%
%
































\newpage
\section{Hurtig opsummering af mine tanker om ITP (02.02.21--03.02.21)}
... Hvilket altså omfatter mine egne idéer til, hvordan en ITP kunne udformes, og mine idéer og tanker omkring, hvordan ITP kunne gøres mere anvendeligt med et online fællesskab, og hvor man så kan formalisere den process at springe skridt over.

\subsection{Tidligere noter omkring ITP}
(02.02.21) Jeg var i gang med at lave de første spadestik til at udforme en tekst omkring mine ITP-idéer. Man skal lede lidt efter dette arbejde, hvis man ikke ved hvor det er, så jeg vil derfor lige nævne, at det altså kan findes i filen main\_12.tex *(Jeg har nu taget den relevante tekst fra dette dokument og sat det ind nedenfor i Appendix, se sektion \ref{main_12}) i min backup-mappe fra før, jeg gik over til at arbejde på endnu et stort dokument, hvor jeg beholder alle fortløbende noter (altså dette dokument). Her kan man finde den første uformelle brødtekst jeg skrev, inden jeg ville gå videre med at lave en mere formel og bedre struktureret tekst (deraf den nye disposition i toppen af dokumentet). Jeg fik dog nogle vigtige indsigter kort efter, som gjorde denne tekst ret ubrugelig. Jeg fandt ud af, at refleksionsprincippet, som det åbenbart hedder, allerede er et udbredt princip for ITP, og jeg fandt også ud af, at der allerede er arbejde på området omkring at udvikle, hvad der så hedder en Formal IDE (F-IDE), og jeg er i det hele taget bestemt ikke den eneste, der har tænkt på programmering og udvikling ved at formulere requirements matematisk og dermed have en mere top-down tilgang til design og udvikling (og vedligeholdelse) af et system. Så selvom jeg måske kan trække nogle tanker og idéer ud, som kunne være interessante, omkring hvordan jeg har tænkt min ITP skulle designes, så tror jeg overordnet set ikke jeg har vildt meget at tilføje på det plan. Selvfølgelig kan det være værd at gentage tanker og visioner, også selvom andre allerede har offentliggjort tilsvarende, men så er der bare ikke rigtigt grund til at gøre det andet end meget kortfattet. Men derfor kan jeg ligeså godt bare henvise til nogle af de gamle noter, inden jeg så prøver at omformulere og opsamle dem mere kortfattet. Her er mine main\_12.tex-noter fra d.\ 15.\ december 2020 nok de mest relevante at nævne. Derudover er der selvfølgelig også alle mine arbejdsnoter, som vist hedder matematik-logik (.txt) 1, 2, 3 og 4 samt ``huskenoter,'' men disse omfatter meget meget mere end den endelige idé, så de er nok ikke værd at dykke ned i overhovedet. Jeg vil derfor bare nævne dem her, men slet ikke anbefale at læse i dem. Jeg kan knapt nok anbefale at læse main\_12-teksten til nogen overhovedet, selvom denne virkeligt er kort relativt til resten: Jeg må hellere prøve at genskrive de vigtige pointer fra omtalte noter her i dette dokument i stedet. Men nu har jeg i hvert fald nævnt at disse noter eksisterer.


\subsection{En ITP baseret på ZFC med et centralt refleksionsprincip, der giver mulighed for at kompilere regler til lavniveau kode (en delmængde af assembly-sprog), og hvor man også på tilsvarende vis kan programmere nye brugerflader til selve ITP-IDE'en}
Her vil jeg med andre ord prøve at beskrive, hvad jeg ville gå efter, hvis jeg skulle bygge en ny ITP.

*Nej, nu har jeg skrevet nogle punkter og så smidt mine tidligere noter ind i dette dokument, og så begynder jeg faktisk bare at arbejde på en engelsk text med det samme (ovenfor).

\subsubsection{Punkter jeg bør komme ind på (02.02.21--03.02.21)}
\begin{itemize}
\item Kan måske hurtigt nævne det at FOL-fundamenter samt SOL med Henkin-fortolkning altid bare kan omfortolkes til hinanden så sætninger kan importeres direkte. Angående dette så vil ZFC være bedst til min løsning, fordi jeg gerne vil bruge refleksionsprincippet på et imperativt sprog (a la Turing-maskiner) på et grundlæggende plan, og så fordi ZFC er super udbredt og velkendt, og fordi det er super simpelt og effektivt (selvom man skal omfortolke sætninger, som man tit tænker på i SOL i hovedet, til en FOL-struktur, så er dette vildt nemt at gøre (hvilket kommer af at mængder er meget lig prædikater, så derfor kan man nemt bare omformulerer sig i i form af mængder, hvor man intuitivt tænkte på prædikater)).

\item At jeg vil bruge refleksionsprincip til at bevise noget om instruktionssæt, der så kan køres på et lavt niveau. *(Og hvordan refleksionen ikke bare skal være til at bruge funktioner, men skal være aksiomer! Det skal således ikke bare åbne op for automatik, men gør også, at det matematisk er gratis at gå ind i højere-niveau-modeller (udgaver af HOL f.eks.) og arbejde derinde i stedet for ude i grundteorien.)

\item Hvordan man skal kunne have mulighed for at ``vise'' nye brugerflader, og hvor nye brugere så kan starte i det meget brugervenlige (brugerskabte) brugerflader og så derefter gå bagom til de underlæggende lag for på den måde at nå flere, mere avancerede, muligheder.

\item At man kan vise sætninger om filer, som så via refleksionsprincippet skal kunne eksporteres ud til de faktiske sætninger, de repræsenterer. Også hvordan disse filer så kan deles som en god måde at dele sætninger på. 

\item ``By having the built-in (lower-level) program(s) powerful enough to be able to run interactive programs with graphical interfaces and all, there is then no limit for how the users can basically extend the application.''

\item At det ofte kan være nemmere at beskrive en funktion matematisk end at programmere den imperativt (eller funktionelt), især når det bliver veludviklet nok og man begynder at kunne bruge mere abstrakte beskrivelser. (Hører godt nok mere til F-IDE/PIDE-feltet generelt, men det er fint nok: Nu bliver det bare lidt blandet.)

\item Det med at kunne søge efter skabeloner, der løser ens dokumentation, og at kunne skifte imellem flere løsninger.

\item Hvor vigtigt det er, at få et godt matematisk miljø, så folk kan blive vildt nemt, og folk kan blive vildt gode, at at bevise ting formalt og matematisk!

\item Det giver formentligt mening at teste applikationen selv med den selv og vise korrektheden af den (også selvom det ikke teoretisk set er en vandtæt procedure).

\item `` ... And since the task of extending the main application is one that requires rigorous mathematical deductions anyway, it would probably by a task that is very well-suited for a mathematical programming paradigm. This means that the main application does not necessarily need to be super advanced to begin with. At the same time, however, extending the mathematical deductive system/environment of the main application will not be wasted work either, especially if it is programmed in a language that is easy to reason about in a rigorous mathematical way, since then these implementations can just be transpiled and used again in the extension applications.''

\item ``Since the embedded applications also can have such read and write permissions there is no need for the main application to be active while an embedded application runs. This cements the fact that the main application probably does not need to fulfill many requirements in the long run.''

\item Min efterfølgende pointe om, at man nok bare bør prøve hurtigt at nå et stadie, hvor man kan vise at ``workspace-transformationer'' er korrekte, og herfra er det så bare at tage et brugbart sprog til at programmere brugergrænseflader og definere en delmængde af dette matematisk (inklusiv hvordan den kan læse fra og skrive til filer). Her er en meget simpel løsning så for resten at bruge et sprog, der bruger XML, samt et meget simpelt I/O-modul (altså hvad angår filer). Så er man hurtigt i gang. Nå ja, og i princippet behøver man så kun at bevise ting om I/O-delen, så det grafiske kan man lade være frit.

\item  `` ... A learner of math should certainly be able to only use rules with a limited amount of automation in accordance with what is expected of a proof at the relevant level.'' ...

\item Generelt er det rigtigt vigtigt, at alle brugere kan få et miljø, der er nemt at arbejde i for dem, og især nye brugere skal kunne komme let i gang med et simpelt matematisk miljø (f.eks.\ skoleelever, gymnasieelever osv.). 

\item I hvert fald nævne type-træet... Og brugerdefinerede ``operationer.'' ... 

\item Og regler definerede med matematiske sætninger (og muligvis farve-regler)..

\item Og i det hele taget hvordan jeg tænker, at man kunne skabe en rar arbejdsplads for matematikere, nye som gamle. 

\item Selvom man grundlægger det i FOL og ZFC, så kan man hurtigt indføre en teori, der svarer til HOL, så man kan arbejde med virkeligt intuitivt skrevne sætninger (hvor man kan tale i et væk om ``(for) alle prædikater'' og behandle dem som matematiske objekter osv.), og hvor alle de sætninger man viser så automatisk kan oversættes tilbage til ZFC.

\item 

\end{itemize}



\chapter{Blockchain}

\newpage
\section{Some blockchain brainstorming}
\subsection{Den seneste update på mine blockchain idéer (12.01.21)}
(12.01.21) Så nu er jeg igen ved at have en samlet vision for, hvordan jeg mener, blockchain nok bør udvikle sig i den nære fremtid. Jeg har nemlig gjort nogle nye tanker på det seneste. Jeg startede med at interessere mig i at se på, hvordan blockchain måske kunne bruges i samspil med mine andre idéer, her i efteråret. Jeg er så kommet på en idé til, hvordan man kunne have en blok-kæde, der kan opdateres løbende ud fra demokratiske beslutninger \emph{på} kæden. Og disse opdateringer vil så have mulighed for at ændre kæden nærmest fuldstændigt, i sådan en grad at nye konkurrenter til kæden altid bare kan efterlignes, hvis de står til at blive mere populære. Det vil jeg skrive mere om, men jeg har så lige brugt noget tid på at få bedre styr på en masse ting, så som hvor vidt man ikke kan gøre noget tilsvarende på state channels til de eksisterende kæder, og jeg har også fået bedre styr på, hvor værdien i coins'ne på kæden kommer fra.

Angående det sidstnævnte så fik jeg endda heldigvis en ekstra indsigt her i sengen i går nat (efter egentligt at have følt, i går aftes, at jeg havde fået godt styr på det hele). Jeg var begyndt at synes at blockchain-``investeringer'' var lidt malplacerede, fordi man bare investerer i en valuta-form, og ikke investerer i selve udviklingen, men det passer ikke helt. Man investerer i faktisk i udviklingen set på den måde, at man investerer i minerne, og sørger for at de får løn for at mine blokke. Hvis man så betragter et køb af bitcoin, som eksempel, som et køb af en form for en aktie med et negativt afkast (uagtet om værdien er stigende eller aftagende) i form at der hele tiden bliver minet nye mønter, så kan man så se denne aktie som ikke andet end et bevis, der siger, ``Denne aktie er et bevis på et bidrag til pågældende blockchain i form af al betaling, der har fundet vej til miners'ne igennem kædens levetid, delt med samlede antal af sådanne mønte-aktier, der ellers er på kæden.'' Pointen er så at man sælge rettigheden til al gunst og alle priser, der må gives for dette bidrag i fremtiden. Siden der ikke på nuværende tidspunkt er lovet nogen priser for at have givet løn til miners som sådan, så er værdien af f.eks.\ bitcoins nærmest nul i princippet, hvis alle gik ud fra denne fortolkning af værdien. Det gør alle selvfølgelig ikke. Det gør sandsynligvis ingen. Det der giver f.eks.\ bitcoins værdi nu er nok mere disse tre faktorer, nemlig at man forventer at andre ``investorer'' vil hoppe med på bølgen, også selvom alle ved at værdien i princippet er nul, at man mener at blockchain er en ny smart teknologi, og at bitcoin, f.eks.\ derfor vil bestå, også selvom værdien er volatil, og at man måske regner med at usikkerheder i det resterende samfund vil få folk til at søge væk fra FIAT-penge, og dermed regner man med at folk så sandsynligvis vil søge mod f.eks.\ bitcoin. Hvis man så skulle begynde at gå over til at den fjerde investeringsgrund, nemlig det med at betragte det som at man køber et investeringsbevis, blev mere udbredt, så ville det jo kræve en eller anden måde, hvorpå bidrag, der ikke på forhånd har været udlovet belønninger for, kunne få udlovet belønninger med tilbagevirkende kraft. Dette er jo netop noget, jeg har en idé til. Jeg har nemlig en idé til et lidt andet forretningssystem, hvor investerings- og arbejdsbidrag bliver belønnet mere bagud, end som tingene er nu, hvor langt de fleste belønninger udloves på forhånd. Man kunne derfor håbe på, at nuværende blockchain-investorer derfor kunne være med til at bringe energi og midler til den bølge, der så kommer til at underbygge dette fjerde grundlag for mønt-værdien. Min idé omhandler en måde at belønne folk bagud, der har bidraget til at fremhæve selve det system, hvor folk kan belønnes bagud (hvilket jeg selvfølgelig vil komme mere ind på). Derfor bliver det på en måde lidt så selv-refererende, fordi spørgsmålet om, hvorvidt f.eks.\ bitcoin-værdien, og den betaling til miners, den repræsenterer, kan regnes for et bidrag til dette system, så lige netop kommer til at afhænge af, hvorvidt bitcoin-ejerne, som eksempel, kan kanaliserer arbejdet omkring bitcoin over, så det kommer til at bane vej for det nye system. Det gør ikke dog ikke noget, at det er selv-refererende, snarer tværtimod da det så måske kommer til at skabe et pres på eksisterende bitcoin-ejere for at deltage i denne overgang. For hvis de ikke gør det, så kan det så tænkes at deres bidrag ikke bliver vurderet som et bidrag til det nye system (eller det nye forretningsparadigme rettere) i nær samme grad som dem, der investerede i overgangen.

Så denne lille ekstra-idé kan potentielt set gå hen og vise sig meget vigtig ift.\ den kommende udvikling, hvem ved? Men ellers er jeg kommet frem til, ift.\ nogle af mine tidligere tanker, at den kæde, jeg forstiller mig, hverken behøver en central mønt i sig, og behøver i det hele taget ikke så meget central planlægning. Jeg er kommet frem til, at det hele godt kan ske mere liberalt. Man kan nemlig, som en vigtig pointe, sagtens danne foreninger ret hurtigt, hvor man kan have tillid til at de giver et ærligt og unbiased svar på kontrakter formuleret i naturligt sprog. Og fundingen på kæden kan sagtens ske ret liberalt også. Jeg tænker, at investeringsfirmaer og kontrakt-jura-vurderings-foreningerne, som jeg tror jeg vil kalde dommerforeninger fra nu af, kan gå sammen på kæden. Disse foreningssamlinger skal så selv, som jeg tænker det, stå for at danne en kæde (altså en blockchain) inde på en overordnet kæde, en del-kæde om man vil, hvor foreningerne så selv står for at udlove betaling til miners. Disse foreninger kommer så til at udgøre et netværk i form af den samlede kæde, men der er så ikke nogen speciel mønt/valuta forbundet med den samlede kæde. Det skal i stedet foreningerne selv oprette. At foreninger samarbejder på en fælles kæde (som mere er et slags krypto-internet end en kæde) er bare til fordel for deres klienter, idet de så kan ske handler og aftaler på tværs af kæder, og brugere kan måske også melde sig på som klienter hos flere forskellige foreninger. Jeg tænker så, at al investering i kæden faktisk kan formaliseres, så man ikke længere er i tvivl, hvad man investerer i, når man køber aktie-tokens eller coins på delkæderne, men at alt dette er direkte. En pointe, som jeg vil prøve at forklare, når jeg når dertil (i en senere sektion), er så, at det at have gode dommerforeninger på en delkæde er så vigtigt for at kæden er brugbar at dommerforeningerne får et kæmpe stake i kæden, og dermed også kan påtage sig rollen som administratorer for hele handlen på delkæden, og for at styre de demokratiske valg om opdateringer. En vigtig ting er også, at disse foreninger bør bestå af mange uafhængige foreninger, og at man dermed, samt med nogle flere commitments fra starten af fra disse foreninger, som jeg vil komme mere ind på senere, og at man herved hurtigt kan opnå meget stor sikkerhed for, at de i fællesskab vil fungere korrekt (også med den stake, der bliver i det taget i betragtning). Bemærk, at disse idéer stadig er i udviklingsfasen, og at det at jeg skriver nu er en del af idé-udviklingsprocessen. Så meget af denne tekst vil ikke være så gavnlig at læse igen, men den er gavnlig for mig at skrive. De næste sektioner jeg laver om emnet vil blive mere belysende, men også disse vil formentligt blive rodede. Når jeg endelig når til noget, jeg mener kan få en høj genlæsningsværdi, så skriver jeg det formentligt på engelsk. Det er i hvert fald planen.

Er der mere, jeg skal nævne angående mine seneste idéer omkring blockchain og/eller om mine idéer før det? Nå jo, at et vigtigt punkt i mine idéer har i lang tid været at få en matematiske teori ind som en kerne af blok-kæden, eller blok-nettet, om man vil, hvor denne teori specielt skal indeholde formatet og protokollen for kæden selv. Jeg tæmker nemlig for det første, at det vil blive smart at kunne udforme kontrakter i et matematisk formelt sprog. Tanken er så nu, og det har den også været i noget tid, at teorien, eller modellen om man vil, også fra start af skal indeholde naturlige sprog. Tanken er så at modellen skal udbygges undervejs af brugerne, og specielt skal de naturlige sprog formaliseres mere og mere, så man kan regne på, om udsagn er sande i højere og højere grad. Men dette vil jeg også komme meget mere ind på. Jeg tror jeg stopper for nu. Nå nej, vent. Der var faktisk også lige en ekstra tillægsidé, som jeg fik i går nat (eller i nat? Hvad siger man? (Gid man kunne sige i nats, og at det ikke lød dumt...)) lidt før den nævnte, og det er at kæderne også kommer til at kunne agere banker med reserve... Hvad er det nu man kalder det? Reserve banking? Reserve spending? Det at man kan udlåne mere, end man ejer.. Det kommer blok-nettet også til at kunne gøre, men nu på en åben (open source) måde. *(`fractional reserve banking' hedder det vist.) Det tror jeg i hvert fald.   



\subsection{Mere opsummering-og-udvikling af mine tanker om blockchain}
(13.01.21) Så jeg tænker altså, at man må kunne danne et internet/web af krypto, hvor man afmystiserer meget omkring blockchain, som det er nu, og får det til at fungere meget mere ligesom et normalt liberalt (hvor jeg associerer ordet liberalt med fri handel, frie kontrakter, usynlig hånd osv. osv.), hvor det ikke er en distribueret sammenslutning omkring en vis konsensus, der driver selve værdien (e.g.\ bitcoin-kursen osv.) på kæderne, men hvor værdien er bundet i mere reelle ting. Der er så to punkter, hvor man stadig skal bruge en konsensus-sammenslutning, og det ene er om at bekæmpe dommerforeninger, der har bedraget i fortiden. Dette er noget af det mest naturlige for os mennesker, og bliver derfor ikke et problem. Man vil således nemt kunne opnå et bredt system, hvor det på ingen måde er i sådanne foreningers egeninteresse at dømme ulødigt på baggrund af en korrupt sammensværgelse. Det næste punkt kræver meget mere samarbejde, men dette er en udvikling af teknologien, jeg mener skal ske i et lag over det basale kæde-netværk. Det handler om, hvordan jeg tænker, at man må kunne lave en forretningsbevægelse, hvis mål er at skabe mest værdi for folk som muligt, og som belønner bidrag til denne forening bagud. Det er forresten vigtigt, at jeg husker at prøve at definere, hvad `værdi' her bør betegne mere præcist. Jeg tror bestemt at denne idé kan være med til at gavne udviklingen af blok-krypto-internettet, men den bygger altså bare ovenpå en allerede selvstændig løsning og er altså ikke essentiel for blok-krypto-idéen, hvis ikke jeg tager fejl.

Så jeg vil derfor prøve her at forklare denne ``selvstændige idé'' om et blokchain-internet/-web i første omgang. ...

(15.01.21) Nå, jeg har ikke fået skrevet så meget de seneste dage, primært pga. træthed grundet søvnunderskud. Kvaliteten af de her første sektioner af dette dokument bliver tilsyneladende ret lav, men sådan er det bare. Til gengæld har jeg fået et par nye idéer og indsigter (som regel midt om natten, når jeg burde sove :/, men jeg føler næsten, at det har været det værd, for det har ledt til noget godt, mener jeg). Siden jeg skrev sidst, har jeg indset at det muligvis faktisk er mere vigtigt at blande forretningsidé og blockchain idé sammmen, end jeg har skrevet ovenfor. Jeg er også kommet frem til, at måden man skal bestemme værdien af bidrag på, selvfølgelig skal være en demokratisk proces, hvor man går sammen i foreninger om at udbygge en model for værdibestemmelse af bidrag, ud fra principper som man også inkluderer i modellen og som også kan opdateres i sig selv (altså man kan både opdatere selve modellen, der giver løfter om, hvordan man vil belønne bidrag i fremtiden (bagud), og man kan opdatere hjemlen for, på hvilket grundlag man bør sigte efter at opdatere modellen løbende, som altså så er løfter for, hvordan man vil varetage (maintain'e) og opdatere sin model (bl.a. kunne man vedtage, at foreningen bør sigte efter løbende at underskrive kontrakter, der binder dem selv yderligere til sine eksisterende løfter: hvis foreningen har i sinde at overholde sine løfter, hvorfor så ikke sætte noget mere på spil for således at binde sig selv til de løfter yderligere)). Der skal så kunne være flere af disse foreninger, og de skal kunne sameksistere fredeligt, fordi man så bare skal sørge for, at der er relativ fri mobilitet mellem foreningerne for de almindelige medlemmer, sådan at disse ``selv er uden om det, hvis de vælger en forening der har en knap så god strategi ift. belønningsløfter.'' Det er på ingen måde en ny idé, at der skal være flere foreninger med forskellige tilgange til samme grundprincip om at belønne bidrag bagud, men det er vist først nu, at jeg er kommet i tanke om, hvordan det løser problemet om at skulle bestemme bidragbelønningshjemlen ift.\ denne blockchain-idé.. Hm, jeg kan mærke, at mine tanker stadig er lidt rodede. Jeg skal lige op i gear. Det skal også siges, at lige nu er det ikke sikkert, at jeg skelner så godt i teksten her, hvad der er idéer, og hvad der bare er indsigter og formodninger og tanker. Jeg har nogle faktiske idéer til blockchain, men der er ikke så mange. Det er mest bare forestillinger om, hvilken rolle jeg mener, at blockchain kommer til at spille i fremtiden. Men det kan altså godt være, at jeg kommer til at kalde det `idéer' i teksten her.

Jeg skal også lige tænke mere over, om ikke man kan bruge crypto coin-princippet på en måde. Jeg føler, at jeg har en god idé (selvom den endnu er lidt (ret) løs) til en måde, hvorpå man kan lave en kæde med selv-opdaterende konsensus (på demokratisk vis), og jeg tror samtidigt også virkelig at et demokratisk internet/web med krypto-protokoller lidt tilsvarende blockchain kan blive kæmpe stort, så det er derfor virkeligt ærgerligt, hvis jeg ikke kan finde en måde at få det til at spille sammen med, hvad der på nuværende tidspunkt skaber interesse omkring bloskchain. Problemet er bare, at meget af det der skaber interesse omkring blockchain er lidt nogle dårlige ting. Selve teknologien omkring at have distribuerede systemer med digitale penge/likvider (kan man sige det? Likvider?), hvor man kan opsætte smart kontrakter osv., det er en super god idé, men der er bare ikke rigtigt nogen idé i, at have en volatil penge-enhed, der ikke rigtig baserer sig i noget, andet end en udbredt fascination. Og jeg er lidt bange for at bitcoin-``investorer'' mest er interesserede i den del, at der er en penge-enhed, der ikke er baseret på andet end en fascination og derfor skaber et vildt spil (som nærmest er et pyramidespil), hvor der potentielt er mange penge at vinde, hvis man lige for lavet de rettidige handler. Hvis det fremtidige blocknet kommer til at bruge sådanne valutaer, så bør de som minimum være bakket op af en masse løfter om, at vedholde og understøtte sådanne valutaer. Ja, og det er så her, at jeg skal overveje, om ikke pointen bare er ``fractional reserve banking;'' om ikke foreninger kan gå sammen om at danne en slags bank baseret på en sådan krypto-valuta? Og det må man næsten kunne, hvis man putter en velfungerende forening, der også har mange andre formål ind som en slags stake, således at et kollaps af valutaen er forbundet med et kollaps af foreningen. Men når man ser det på den måde, så er det vel i princippet bare en slags aktier, man så kan udstede på bagrund af et firma eller en forretningsforening, så er det ikke bare bedre, at gøre det sådan?... Hvad skal disse ``aktier'' så bestå i? Jamen, ikke andet end at firmaet committer til at udbetale visse ydelser eller likvider, eller produkter for den sags skyld, for aktierne, så længe at de ikke bliver overrumplet med indløsninger på én gang, så de ikke kan følge med, og muligvis med den hage, at hvis ``aktien,'' som nu er et slags lån, kollapser, så er det ikke sikkert at investor/lånafgiver får sine penge tilbage. Hm, ja og det er vel mere eller mindre præcis en bank, jeg har beskrevet her... Men banker virker jo, så det er jo fint: Hvorfor ikke have mere open source banker? Ok, men så er vi dog stadig tilbage til at konkludere, at de ikke-understøttede krypto-mønter ikke rigtigt fungerer, som de er nu.. Altså at der ikke er en måde at rede f.eks.\ bitcoins på (hvis vi altså har lyst til det)? Tjo, jamen så skal skulle det vel være noget med at værdsætte, at eksisterende blochchain communities... Jo, at de har investeret en masse i noget, der har skabt en stor udvikling og har samlet en masse resurser, både penge og regnekraft, omkring en teknologisk udvikling. Jo, og så har jeg jo tænkt, at hvis denne energi kan kanaliseres, således at resurserne ikke bliver hængende i forældede blockchains men kan føres over i nye teknologier, så er det jo et kæmpe ``bidrag'' til udviklingen og fortjener vel belønning, vil mange sikkert mene. Det er så i forbindelse med disse tanker, at jeg har genopdaget tankerne med at have flere forskellige foreninger, der går sammen om at understøtte en belønningsmodel, således at der netop kan være plads til delte meninger, og så at Bitcoin-entusiaster f.eks.\ kan sørge for at joine en forening, der værdsætter den udvikling og de bidrag, der har været omkring Bitcoin osv. Og så gælder det bare om for den forening at finde en fortolkning af, hvad det vil sige at købe BTC, således at man med tilbagevirkende kraft kan vedlægge en værdifortolkning til BTC, og gerne en fortolkning, der passer godt sammen med foreningens vedtægter ellers af, hvordan medlemmerne får lov at handle med bidragsaktier (hvor man altså køber det hele eller en del af retten til den udbetalte løn for pågældende bidrag, når denne udbetales engang i fremtiden). I øvrigt er det i sig selv et bidrag, hvis man som en underforening gik sammen om en bevægelse, der lykkedes at kanalisere energien fra en tidligere teknologi ind i den nye forretningsteknologi på en god måde, og selv hvis en sådan forening startede som selvstændig, så må dette bidrag tælle med og blive belønnet korrekt, skulle denne forening senere lade sig optage i en større forening. Nå, men spørgsmålet er så bare, for at vende tilbage til det med, om man kan bjerge eksempelvis BTC, om jeg tror dette kan holde, eller om ikke bare folk vil sige, at meget af energien kom fra folk, der var ude på at vinde en hurtig personlig gevinst, og kan overordnet set ikke rigtigt siges at være et bidrag til den nye teknologi. Det vil nogen i hvert fald nok mene... Nå, jo, men pointen er jo så netop, at man kan betinge belønningen, så den kun netop retter sig mod de nuværende blockchain-investorer, der formår at kanalisere energien over på en god måde. Problemet er så bare, om dette overhovedet kan lade sig gøre, for en del af værdien på f.eks.\ BTC ligger vel i, at folk satser på, at der ikke kommer en anden opdatering af teknologien lige foreløbig, som slår Bitcoin af banen...? Hm, men... Er det ikke bare at lade de eksisterende blockchains være...? Og altså bare lade valutaen der leve sit eget liv?... Hvis jeg ikke kan finde nogen måde at bruge det paradigme på, så rører den udvikling, jeg har i tankerne, vel heller ikke ved eksisterende kæder? Så er vi tilbage til, at de andre kæder bare skal kunne logge sig på kæde-internettet ved at låse BTC og ETH osv. til tokens på kæde-nettet? Ja, det er vel så simpelt som det.. Jeg har naturligvis bare været tiltrukket af at kunne finde en løsning, hvor at de eksisterende kæder kan opdatere sig selv (f.eks.\ til mere PoS-agtige kæder), men... Ja, selv hvis jeg kan finde på sådan en løsning, så må det være en idé for sig. Jeg tror nemlig selv meget mere på kæder, hvor værdier enten er bundet op på faktiske ting, eller som minimum er bundet op på et fractional reserve banking-system. Så jeg har derfor ikke tænkt mig overhovedet, at prøve at gøre en magisk valuta som en central del af et krypto-kæde-netværk/-internet/-web.  Fint. Og der \emph{er} nemlig måder, hvorpå man kan låse selv BTC *(eller hvad? Hvordan er det lige, man laver state channels med Bitcoin?..) til det nye kæde-netværk uden at bekymre sig for, den oprindelige kæde... Tja, eller der er et par issues, men det er nogen, jeg mener, man bør kunne overkomme. Problemet er... Hm, vent.. Hvis man ikke ændrer på, hvordan coins bliver minet, så er det jo totalt ligegyldigt, om man låser mønter til state channels eller ej. Det er kun fordi, jeg har tænkt på, om ikke man også kunne bremse mining'en, men hvis ikke man gør det, så er der heller ikke fare for, at man skaber rystelser i kæden og i mønt-kursen. Min tanke var nemlig, at hvis man alligevel ikke benytter miner'ne, hvorfor så ikke gå sammen om at udfase dem, men dette kan nemlig ved nærmere eftertanke skabe rystelser, og derfor er det meget bedre bare at sige, at dette er et separat problem, som kædemedlemmerne selv må håndtere. Ja, faktisk så tror jeg at den kontante mønt-skabelse på en måde kan være med til faktisk at få kursen til at stige, selvom man skulle tænke det omvendte. Så selvom det ikke er sikkert, at dette er rigtigt, så er der i hvert fald ingen grund for mig i at spekulere særligt meget over, om det ville være smart at prøve at udfase sine miners eller ej på de eksisterende kæder. Og... Så kunne man jo faktisk også foreslå en nye kæde, som kan opdatere sig selv på demokratisk vis, hvis man bare foreslog det som en adskilt idé... Hm, spændende... Så selvom jeg ikke tror vildt meget på nuværende kæde-mønter, så vil jeg godt indrømme, at de måske godt kan vise sig at blive langtidsholdbare, og derfor gør det jo heller ikke noget at komme med en idé til en udvikling af sådanne kæder... Hm, problemet er så bare, om det overhovedet er en forbedring, at man kan opdatere sin kæde... Hm... Okay, så det her, hvor jeg lige skal tænke nogen ting igennem...

Ah, jeg tror næsten, jeg ved det nu. Jeg tror vist godt, at man kan danne en brugbar forrestningsforening-forretningsmodel, hvor man starter med at have en krypto-valuta a la BTC eller ETH, men hvor man så fra start giver en hjemmel (i sin model) for, hvordan denne valuta skal overføres til at være enten en aktie i selve foreningen eller den grundlæggende mønt som skal bruges internt i foreningen, eller en blanding... Så kan man måske også mildne den negative effekt, der opstår, hvis der opstår en meget skæv fordeling af mønten, således at man ikke kan undgå at gøre noget alt for rige, hvis den samlede pengemængde skal nå op og dække det behov, der er. Man kan således på forhånd give en hjemmel for, hvornår man bør øge eller bremse penge-skabelsen, hvordan man har lov til at uddele nye penge, og evt. hvordan man har lov til at lave progressiv beskatning af medlemmerne. En hoarder har selvfølgelig ikke lyst til at blive beskattet på sådan en måde, men hvis ikke det sker, så er der fare for kollaps (hvis folk ikke synes det kan betale sig at holde fast i den skæve økonomiske fordeling, og at der derfor er risiko for, at de trækker stikket), og hvis det også fra start har været en del af forenings hjemmel, jamen så vil sådanne hoardere jo blive nødt til at gå med til at blive beskattet. Men ja, pointen er altså, at hvis forretningen bliver stor nok, så vil der jo være en masse intern handel, og derfor giver det mening at udstede en pengemængde. Det svarer altså således ikke helt så meget til fractional reserve banking, men mere til hvordan en nationalbank virker: En stor gruppe mennesker går sammen om at beslutte, at der skal være en samlet pengemængde, der ikke bunder i andet end troen på denne sammenslutning. Virker som en vigtig ting at indse, glad for at jeg har gjort det. (15.01.21) (stadig (jeg kan godt finde på at opgive datoer midt i mine notetekster som sådan her, hvor datoren så peger på den tekst, der kommer lige inden, hvor den står)). Hm, på den måde, så virker princippet vel også selvom det med at have matematiske modeller som en del af kæden ikke kommer op at køre...? Hvis nok mennesker drager fordel af, at systemet ikke kollapser, så kan systemet opretholde sig selv... Uh, jeg håber virkeligt, at disse tanker holder..! Hm, og kan prisen ikke presses op af selve efterspørgslen af valuta-formen, for hvis nu pengemængden bliver for lav, men at folk stadig har behov for den slags valuta, så vil de vel give lidt mere for pågældende penge, og det vil vel vare ved, indtil at penge-mængden så stiger så meget, at der ikke længere er pengemangel... Interessant. Rigtigt interessant.. (15.01.21) ..Og hvis man anerkender dette, så kan man også analysere, hvor værdien burde ligge, hvilket måske kan hjælpe til at stabilisere værdien yderligere.. Hm, og hvis foreningen så har mulighed for at justere pengemængden efter en sådan analyse og på en hensigtsmæssig måde.... Jep. Jeg tror sgu, jeg har fanget det nu. Og tror faktisk nu (igen), at en krypto-valuta godt kan blive en vigtig del af et blok-internet (om end stadig ikke en kritisk bærende del)..! (15.01.21)

%(16.01)
Det gode er jo så også, at det nu bliver meget nemmere at starte en demokratisk kæde, fordi man så bare kan lade genisis-protokollen være PoS-agtig, således at folk simpelthen får antal stemmer tilsvarende deres møntbeholdning. Dette er selvfølgelig ikke langtidsholdbart og skal derfor helt sikkert tages til genvalg (hvor foreninger så fra start af bør formulere en hjemmel for, hvad man vil sigte efter, når valgprocessen skal opdateres i første omgang(e)), men det er en rigtig god start. 

(18.01.21) Så nu kan jeg gå tilbage til rigtig mange af de ting, jeg har tænkt omkring en demokratisk opdatérbar kæde. Grundlæggende for idéen er, at der skal være en matematisk teori/model, der starter som en grundlæggende matematisk teori, så som mængdelære (som jeg vil skrive om et andet sted, så vil jeg forslå en udvidelse (extension) af ZFC i FOL, særligt med et refleksionsprincip, så man kan opbygge en indre model, der reflekterer (størstedelen af) den ydre teori, og hvor man kan definere algoritmer, og hvor man så kan ``køre en algoritme og notere outputtet'' som et gyldigt deduktionsskridt (en gyldig regel)), hvor man så tilføjer nogle antagelser, der definerer hele kædeprotokollen i sig selv, inklusiv hvad start-teorien/modellen indeholder. Så denne model refererer altså til sig selv, men det kan jo godt lade sig gøre. Særligt skal der så defineres en funktion, som en del af protokollen, som givet en mængde data kan fortælle, hvordan den samlede pengemængde (af de grundlæggende kæde-coins) er fordelt på forskellige kontoer, defineret ved et ID. Der behøver ikke at være opgivet nogen restriktioner på disse ID'er, da ugyldige ID'er bare kan få funktionen til at returnere 0, og ID formatet kan således også være åbent for opdateringer. Funktionen kan godt være partiel, hvis det gør det nemmere, så længe man stoler på, at brugermængden altid samlet set vil holde data, som kan fodres til funktionen og give et svar. Pointen er så simpelthen bare den, at hvis datamængde A får funktionen til at returnere en pengefordeling S, og at datamængde B giver en pengefordeling T, hvis der så eksisterer et C således at B = A $\cup$ C, så ``slår'' svaret T svaret S. I teorien vil den rigtige pengefordeling være funktion taget på den samlede data til et givet tidspunkt, men funktionen skal selvfølgelig være defineret på en måde, således at man ved at se på den offentligt tilgængelige data kan få det svar man leder efter. Det er således bare en generalisering af, hvordan man f.eks.\ bestemmer pengefordelingen af bitcoin. Her sammenligner funktionen, hvis man skulle definere den, så bare kædelængder returnerer den pengefordeling beskrevet ved den længste kæde. Hm, og den vil faktisk være partielt defineret i princippet, når to forskellige kæder er lige lange. Man kunne dog også bare antage, at den bare udvalgte én af de lige lange kæder, for der er ingen der siger, at dette valg skal være betydende på nogen måde for, hvilken kæde der ender med at blive valgt i praksis. Men det er nu nok pænest at antage, at funktionen godt må være partiel. Nå men ligesom at man ikke for Bitcoin behøver at bekymre sig om, hvorvidt der eksisterer en lille datamængde, der giver en anden pengefordeling, end den man er kommet frem til, fordi man er sikker på, at det ikke ændrer billede gevaldigt, og at denne data nok enten skal komme frem i lyset efter lidt tid, eller nå at blive irrelevant i denne tid, så er det altså meningen at den generaliserede funktion skal fungere på en tilsvarende måde og give tilsvarende sikkerhed. Jeg tænker at det måske ville være en god idé at definere en grundlæggende protokol for, hvordan man bestemmer pengefordelingen... Tja, måske. Pointen er så, at denne protokol mest bare skal fungere som en teoretisk måde at afgøre en sag på, men at denne stadig kan erstattes løbende af andre protokoller, der er mere effektive. Men det kan jeg lige vende tilbage til.  

Modellen, og særligt den del af modellen, der beskriver kæde-protokollen i form af den nævnte funktion, skal så kunne opdateres løbende. I starten tænker jeg som sagt, at dette så bare skal foregå ud fra en PoS-agtig protokol, således at folk med stake i kæden (i form af de grundlæggende mønter) kan udnævne forslag til ændringer i modellen (i form af at man tilføjer eller fjerne antagelser), og hvis disse forslag så efterfølgende for opbakning af et flertal, bliver de formelt set tilføjet til modellen. Der skal selvfølgelig være en kerne af modellen, som ikke kan røres ved, særligt den del der beskriver hele denne proces. Brugere, der ikke har tid til aktivt at deltage, skal kunne give deres stemme midlertidigt til andre kontoer (som eksempelvis kan være eget af en forening af brugere), men skal selvfølgelig altid være fri til at tage denne stemme tilbage, også bare for at stemme for individuelle forslag for så at give stemmen tilbage igen. En (nok) vigtig ting, som jeg tænker, er så, at der skal være en udløbsdato for, hvornår denne protokol ophører. Det er så meningen at brugerne inden denne tid (og gerne i god tid, for man kan så altid bare ændre beslutningen inden udløbsdatoen) skal udforme den næste protokol for, hvordan modellen opdateres. Jeg har nævnt, at man på et tidspunkt bør gå væk fra PoS-stemmeret (hvilket er fordi det så næsten vil være uundgåeligt at folk med mange penge så på et tidspunkt i fremtiden vil bruge denne magt til at forøge deres formue, og så har man en uundgåelig cirkel med positivt feedback af magt, og denne korruption vil så bare vokse og vokse), men det er ikke dette man skal til at ændre i første omgang. I første omgang er det nemlig bare vigtigt at opdatere kæden til den næste fase, hvor der kan være flere modeller i gang på en gang. Efter denne fase bliver ændringer af modellen således ikke noget, der nødvendigvis skal stemmes om globalt, men i stedet kan man her have mange forskellige foreninger af brugere, der har koblet deres kontoer til en specifik udgave af modellen, som de finder mest effektiv, og særligt som de tror, vil blive den mest populære i fremtiden. Så fra at ændringerne til modellen danner en kæde, kommer de så i den nye fase til at danne et træ med mange mulige forgreninger, hvor hvert blad så har en hvis brugermængde tilknyttet. Dette træ vil så have kæden fra første fase ved roden. Men i overgangen til fase 2 er det så meningen at man omstrukturerer denne kæde, således at alle antagelser, der ligger til grund for fase 2 så kommer ned i bunden, således at brugere også kan få lov til at forgrene sig væk fra selv de tidlige beslutninger fra fase 1, som ikke har med selve protokollen for denne anden fase at gøre. Den vigtigste pointe i denne fase er, at hvor at de ikke skal koste noget at udbygge modellen for en pågældende gren, så skal det koste noget at trække modelantagelser tilbage. Jeg tænker, at denne pris skal være i form af, at mønter der har været tilknyttet en gren, hvor antagelser er blevet trukket tilbage (svarende til at man formindsker grenen, evt.\ så meget at den trækker sig helt tilbage og bliver et blad midt på modergrenen), får en skat trukket fra sig. Den effektive pengefodelingsfunktion skal så normalisere pengemængden efter at have vægtet hver mønt med denne skat. Ved indgangen til fase 3 skal denne skat så endelig trækkes fra, således at mønter, der ender op på samme gren, bliver vekslet i overensstemmelse med deres skatter, således at de efterfølgende alle bliver identiske igen. Fase 3 handler så på en måde om at starte forfra. Nu har brugerne af den oprindelige kæde fået mulighed for at gå sammen om at forke væk, hvis de mener, at den mest populære gren ikke dur... Hm... Ah, og hver gren, der gerne vil prøve at fortsætte for sig selv, skal så bare aftale, hvordan de eventuelt skal merge med en anden gren, hvis nu denne gren viser sig at vinde som den mest populære gren på den længere bane. Ja, så bør så primært komme til at handle om, at danne fundamentet for, hvordan modellen, og særligt pengefordelingsfunktionen, opdateres derefter. Hvis brugergrupper så er uenige herom, kan de danne hver deres gren. Op til udløbsdatoen af fase 2 bør disse forskellige brugergrupper så hver især aftale en protokol for, hvordan de vil optage brugere fra andre grupper, skulle de få lyst til at skifte over til en anden fork/gren efter denne udløbsdato. I fase 2 bør der altså stadig være en underlæggende protokol, der gør at en gren kan referere til andre grene i deres model. Således kan brugergrupperne indgå aftaler ved at antage protokolregler, der siger, at hvis en pågældende rivaliserende brugergruppe har de og de antagelser i deres egen model i udgangen af fase 2, så gælder så og så antagelser i grenens egen model også efter udgangen af fase 2. Rivaliserende brugergrupper kan således på denne måde antage betingede protokoller for, hvordan de andre brugere kan optages (og nogle af deres mønter beholdes efter en vis transitionsskat er påført dem), som så er betinget på at denne brugergruppe selv har defineret vis (formentligt tilsvarende) protokol i deres model. Grunden til at dette skal udskydes til fase 3, er, at sådanne protokoller meget vel vil komme til at indeholde en form for tidsbestemmelser, da de nævnte skatter formentligt vil vælges så de kommer til at stige over tid (så man efterhånden kan glemme de resterende anden-fase-forks med tiden). Og da det er lidt kompliceret at blive enige om en sikker måde at måle tiden på tværs af forks, som skal kunne holde for en del tid, så bør dette altså vente til man når fase 3. Fase 3 er så en fase, hvor man har gjort mere sofistikerede bestemmelser end fase 2 for, hvordan brugere har lov til at gå sammen om at forke hoved-modellen, og hvad de satser ved at gøre det, samt hvad de kan vinde. Hm, jeg kom til at tænke på, det bliver nok lidt underligt, hvis der skal være en skat for hver tilbagetrukne antagelse, for man kan jo lave konjunktioner osv., så måske er det bedre bare at have nul skat for at migrere mellem grene (og altså for at trække antagelser tilbage) som bruger indtil man når fase 3. Så fase 1 er meget simpel og handler bare om at bestemme protokollen for fase 2, hvis formål igen bare er at bestemme protokollen i fase 3, og det er så først i fase 3, at det bliver spændende og at folk kan satse på forskellige grene af modellen. I forhold til pengefordelingsfunktionen og modellen generelt, så kommer en faseovergang til at bestå i, at modellen siger ``antagelse 1 til og med n fra fase x er sande, samt alle efterfølgende antagelser der gøres i fase x+1 efter en ny protokol, der er defineret så og så...'' Ja, så her er det faktisk også vigtigt med et godt reflektionsprincip i det grundlæggende fundament, så man således er fri til at lave antagelser defineret ud fra en protokol, velvidne at de resulterende propositioner som nås i den indre, reflekterede model, altid kan løftes ud som ægte propositioner i den ydre model/teori. Okay, jeg holder lige for nu. Når jeg fortsætter, skal jeg så skrive mere om, hvordan jeg tænker, at fase tre skal være, og om hvad man i det hele taget opnår med denne kæde derfra. Jeg skal også have defineret de grundlæggende protokoller nærmere, samt hvad den grundlæggende konsensus, som man skal starte med at antage, skal være. 


(19.01.21) Nå nej, alle de har ting med faser osv., dem er der jo nok ikke vildt meget behov for alligevel. Der er nok ikke så meget behov for at prøve at straffe brugere, der eksperimenterer med modelantagelser, der så viser sig ikke at være holdbare. Det eneste er, hvilket netop er grunden til, at jeg har tænkt i de baner, at der skal være en god måde for brugere (og særligt brugerforeninger) at give løfter på kæden, som kan formuleres som en slags antagelser i modellen. Det skulle være den eneste grund til at sætte restriktioner, der tvinger brugerne til at gøre sig umage med kan at vedtage sande antagelser, i stedet for bare at lade det være frit at eksperimentere med modeller. Min tanke har så været (og jeg har ikke afvist den endnu), at man skal lægge ud med en stor konsensus om kun at bruge den tilegnede model i kontrakter... Hm, jeg kan lige forklare nærmere om dette senere, men hvordan opnår man lige, at man får en konsensus om at udvikle en lødig model, som kan bruges til kontrakter, der så kan dømmes lødigt på sigt...? Det er jo noget af det, jeg har tænkt meget over, og jeg føler lidt, at jeg tidligere er kommet på et godt muligt svar; det skal jeg så bare lige huske nu... Hm, en af pointerne var altid at bevare en lødig fortolkning af kontrakter udformet i naturlige sprog (hvor det så kan være implicit, hvilket sprog og i hvilket år, der er tale om), men det ændrer jo ikke på, at man så gerne skal kunne blive enige om en fælles konsensus, omkring hvilken model skal bruges... Tja, men min tanke var vel så bare, at foreninger kunne melde sig på banen på kæden, som organisationer der er til at stole på, om som har kapacitet til at dømme kontrakter. Og så var en pointe bare, at man kun som bruger bør stole på foreninger, der har indgået en aftale med mange (alle der også tager sig selv seriøst) andre foreninger om en måde, hvorpå de kan udluge brodne kar fra den overordnede sammenslutning, og også som samtidigt lover løbende at lægge flere og flere juridiske bånd på sig selv, så deres eget incitament for at dømme på lødig vis hele tiden forøges. Men disse tanker spiller så bare ikke super godt sammen med den nye struktur, hvor man skal have en global mønt. Mine idéer for, hvad disse mønter skal bruges til, og hvordan man kommer til at fastholde en værdi for dem i fremtiden, afhænger nemlig lidt af disse dommerforeninger, som vi kan kalde dem. Hm... Ja, og i sidste ende er man vel så afhængig af... Hm, eller man kan jo altid bare vælge en model samt en opdateringsprotokol for denne, når man udformer kontrakter, der skal bedømmes i fremtiden...

(20.01.21) Okay, så pointen er, at mønternes værdi kommer til at afhænge af, hvor effektive og lødige de tilgængelige dommerforeninger er, så set fra disse foreningers synspunkt, er der ingen grund til at belønne brugere unødvendigt, som ikke har bakket op om disse foreninger. Hele idéen med kæden beror på at få sådanne lødige foreninger op at køre, så de fleste brugere vil klart hoppe med på vognen, men derfor giver det stadig mening med en udvælgelsesproces som beskrevet ovenfor, hvor modeller med uholdbare antagelser (og løfter) bliver skilt fra. Jeg tænker dog, at skatten/bøden til brugere, der har eksperimenteret med andre modeller, og/eller som har bakket op om, hvad der har vist sig at være ikke-lødige foreninger, skal være meget minimal i starten. Det er først, efter min mening, når dommerforeningerne virkeligt er kommet godt op at køre, sådan at mange vil være enige om, at de er lødige og at de ikke vil korrumpere, men forbedre sig selv, i fremtiden, at disse dommerforeninger bør begynde at initiere nogle restriktioner for, hvor let man kan hoppe over til andre modeller. For en ting er, at man altid bare kan vælge model og opdateringsprotokol for kontrakter, men hvis mønterne skal have værdi i form af, at de kan bruges i kontrakter, så skal de også gerne.. støtte de dommerforeninger... Ja, og her er der jo tale om en afgift til dommerforeningerne, der enten kan ske for hver kontrakt, hver handel eller bare for hver gang, at deres services skal bruges, eller en blanding af disse. Hvis der er en lille afgift for hver kontrakt bare, så kan det jo faktisk også være en måde, hvorpå man kan justere mønt-værdien, hvis man samler en større og større formue bag dem... Tja... Mh, hvad med at der bare er en standard variabel afgift med hver kontrakt, som så skal stemme overens med, hvad den dommerforening, man har valgt i kontrakten, forventer af betaling..? 

Okay, så det er måske faktisk alligevel ikke en super velfungerende struktur, det med at se det som en slags FIAT-valuta, der har værdi efter, hvor meget efterspørgsel der er efter den ydelse, der følger med, for det kommer nok til at blive meget usikkert, hvis folk så bruger denne valuta til opsparing. Og det er jo en stor del af energien bag f.eks.\ Bitcoin; at folk gerne vil skabe/vedligeholde en opsparing.. Men det gør nok ikke så meget, for man kan nok i stedet godt lave en kæde, der næsten bruger princippet om at betale bagud ud fra bidrag, men ikke helt. Man kan således måske lave en god mellemting, som så kan være en god overgang til en bagud-belønningsforretningsforening. Man kan således starte med at udlove belønninger forud, men hvor de så stadig bliver betalt i fremtiden. Hvis kæden nemlig alligevel afhænger af dommerforeningers succes, samt en eller anden forventning om at denne forening vil opretholde mønternes værdi og/eller belønne ``investorer,'' jamen så er det da meget nemmere bare give konkrete løfter fra starten om belønninger. Ligesom jeg har tænkt for min forrestningsidé, så skal disse belønninger representeres via tokens, der så kan handles (helt eller delvist) som penge, indtil de bliver betalt. Forskellen er bare her, at løfterne så måske bare skal starte som værende mere konkrete om en vis betaling, muligvis svarende til x antal nutidspenge, ved f.eks.\ at formulere det som, ``hvad man kunne få på det tidspunkt for x antal dollars.'' Ja, det er nok faktisk rigtigt smart at starte med sådanne konkrete løfter, og så kan pågældende forening på et tidspunkt udvikle sig til, at lave belønninger og fra en mere abstrakt belønningshjemmel. (20.01.21) Det gør også processen meget lettere med at finde den korrekte værdi, fordi denne så bare kan bestemmes ud fra pågældende handel --- og en mere abstrakt hjemmel, som jeg tænker den, skulle netop sigte efter at belønne efter, hvad der ville være en god handel, hvis man spolede tiden tilbage, hvor man dog havde kendskab til visse parametre for, hvordan fremtiden vil forløbe. Så selvom det smarte ved sådan en orden er, at man ikke behøver at formulere end masse avancerede kontrakter i hvert tilfælde, og at der derfor kan være meget større dynamik i foreningen, så slår det stadig ikke, hvad man kan opnå, hvis man rent faktisk laver de pågældende kontrakter. Og især for simple handler, som f.eks.\ handler om at give betaling for mining osv., hvor ydelsen er meget konkret, jamen så kan det derfor helt sikkert være en fordel bare at forhandle belønningen med det samme. Og bare fordi man har indgået sådan en forhandling, så betyder det ikke, at man ikke kan blive belønnet ekstra senere, hvis nu det forløber sig sådan, at vedkommende ender med at gå et ekstra skridt for foreningen (enten af lyst eller pga.\ en uforudset forhindring). Og kontrakter kan stadig gøres mere og mere avancerede for at tage højde for alle mulige fremtidige udviklinger, indtil man på et tidspunkt kan gå over til bare at følge en bred belønningshjemmel i stedet (således at alle tilsvarende kontrakter derfra med et bliver vildt avancerede i praksis; skulle man prøve at eftergøre dem med en forudsigende kontrakt i stedet). (20.01.21)

(21.01.21) Okay, så jeg tror nogenlunde, jeg har et slags overblik nu. Problemet med en normal mønt-blockchain er, at det ikke er sikkert, at det giver så god mening, når folk vil spare op af mønterne (og ``investere'' i dem). Og i sidste ende så skal der jo nok være god mulighed for at digitalisere kontrakter og værdipapirer, og benytte et distribueret netværk til at udføre protokollerne omkring dem; lave udregninger, bestemme sandheden af udsagn og udføre overførsler. Og således får en digital mønt nok i sidste ende bare sin værdi fra den forening, der har lovet at lægge arbejde bag de kontraktformer, som mønten er tilknyttet, samt af eventuelle løfter som denne forening har gjort omkring mønten. Så vi ryger tilbage i et meget simpelt, liberalt system, hvor der nok ikke rigtigt vil være nogen konsensus om at opretholde en eller anden magisk værdi, som alrdig rigtigt bundede i nogen løfter til at begynde med. Men: Hvis man nu tror på at PoS-kæder vil få en succesfuld fremtid, jamen så har jeg da en løsning (eller rettere, jeg kan hurtigt udarbejde en), der slår den nuværende teknologi. Og så gør det jo ikke noget, at jeg også har en idé bagved, som jeg selv tror mere på, hvor man starter med at binde pengene til løfter. For så er den overordnede idé jo interessant uanset hvad; uanset om man kan starte sådan en temporarily closed source-forening med stor succes, sådan at man kan udstede løfter fra starten, eller ej. Men det kan man nu. Der er også bare simpelthen for mange argumenter for min idé der. Selv uanset hvor godt mine ``prediktive modeller'' kommer til at virke i praksis, så er der ingen tvivl om, at en bred temporarily closed source-forening med løfter om belønninger bagud, vil blive en kæmpe succes, hvis den først vinder frem. Og hvis man ser det fra perspektivet om hele den idé, så er det bare rigtigt smart, at man kan bruge krypto som fundament til netværk, hvor betalinger og kontraktbedømmelser osv., kan ske på et distribueret netværk. Så selvom idéen godt kunne klare sig med det normale internet, så hører den rigtigt godt hjemme på sådan et højteknologisk distribueret og selvstyrende netværk. Og et sådant netværk er også bare en udvidelse af internettet, og altså meget værd i sig selv.

Så men nu behøver man nok ikke helt at tænke så meget i faser osv., gør man? Tjo, det kunne være en god måde at opstarte det, fordi så kan folk og foreninger logge sig på efterhånden i stedet for at alle disse aftaler skal laves privat, osv. Ja, jeg tror det er en god idé bare at starte en opgradérbar kæde, som så har en grundhjemmel om, hvordan den bør udvikle sig, og så kan foreninger melde sig på for at varetage nogle af de løfter, der er lagt op til. Et af de første løfter skal så være at give god betaling for de første miners, selv inden at foreningen meldte sig på kæden. Dette er også en god måde for foreningerne at vise velvilje overfor hele tanken/bevægelsen. Jeg tænker at miners skal få lov at udstede mønter til sig selv, som lover denne betaling til den endelige ejer. Bemærk at der således ikke er nogen standhaftig valuta på kæden, der er kun aktier som neutraliseres igen på et tidspunkt, idet de kun har et enkeltstående afkast. Foreningerne er selvfølgelig også helt frie til at opfinde deres egne former for ``aktier.'' Jeg tænker så nu, at kernen af kæden skal bestå af en række udsagn/erklæringer skrevet i et formelt men naturligt sprog (engelsk), som så skal bekendtgøre rammerne for, hvordan kæden, eller rettere kæde-nettet, skal udvikle sig. Disse udsagn kan så særligt referere til matematiske sætninger for, hvordan kæden skal udvikle sig. Her er der altså tale om udsagn, der sætter hele den formelle protokol på plads, og som definerer en indre model, hvordan denne skal opdateres i praksis, samt sætter nogle rammer, der siger, at modellen skal udvikle sig, så at modellen forsøger som minimum at indeholde svar på gængs accepteret viden, og at den skal sigte efter at formalisere naturlige sprog i sådan en grad, at kæde-aksiom-erklæringerne fra og med et tidspunkt kan vurderes som sande eller falske ud fra denne model, hvor svaret altså så kommer til at tilsvare den gængse, sandfærdige holdning iblandt kyndige folk. En hypotetisk kæde, der går igennem en udvikling, hvor erklæringerne ikke bliver holdt, men hvor brugerne bare konspirerer om så at få modellen til at skjule dette faktum, må så betegne en korrupt kæde. Det store mål er at nå full circle, sådan at når man begynder at have kræfterne til at vurdere, om erklæringerne var overholdt inden for kædens egne juridiske evner, så vil disse på sandfærdig vis bekræfte at kæden ikke er korrupt. Og fra det tidspunkt af, vil det så kræve en større energi at korrumpere kæden, fordi den i første omgang så selv vil opdage det på en måde. Selvfølgelig vil den stadig være korrumpérbar, men når man når hertil, vil det så være et alarmsystem sat op i det mindste. Det er så selvfølgelig vigtigt at gøre erklæringerne løse nok til, at de kan nå at rettes, hvis man finder ud af et bedre aksiom-sæt, men disse rettelser skal så bare igen ske ud fra en erklæret hjemmel. Særligt skal erklæringerne være stærke nok til, foruden at definere protokollen, at fastslå, hvordan miners kan forvente betaling, og med hvilken overordnet forretningshjemmel, som de foreningerne, der melder sig på og tager stafetten videre, skal sigte efter. Tjo, men selv her kan man også nøjes med at gøre erklæringerne ret løse, sådan at de kun optegner intentionen med kæden, men så kan det også være en del af den senere udvikling at få dette aksiomsæt mere nøjagtigt defineret.


(22.01.21) Aktier kan jo godt være ligesom mange normale aktier, hvor man handler med dem, og hvor firmaet så kan få lov at bruge nogle af de penge, da dette holder aktieværdien oppe. I kæde-netværket kan man så bare erstatte `firma' med `bevægelse.' Forskellen er så, at investorende løbende kan justere på, hvem aktien ligger hos, på en måde. Aktierne ligger nemlig i en hjemmel om en bevægelse, og hvis f.eks.\ dommerforening ikke viser sig at være værd at investere i alligevel, så vil brugerne bare lukke for hanen der og åbne den andetsteds, men deres aktier forbliver nogenlunde de samme, for hvert bidrag i god ånd tæller med, også selvom de endte med at gå til de forkert. (Dermed ikke sagt, at det ikke skal belønnes, hvis man investerer de rigtige steder, og straffes, hvis man investerer skødesløst, men det kan nu alligevel være smart for bevægelsen at forsikre tidlige investorer om, at deres bidrag nok skal blive værdsat, hvis de er gjort med omtanke og på et ansvarligt grundlag.) Så nu har vi faktisk en slags mønter på kæden! (Jeg fik i øvrigt disse tanker i sengen her i går nat (i nat, ville nogen sige), og fik forstærket dem lidt efter at jeg vågnede i morges.) Disse aktier kan nemlig handles som normale penge, og kan stige og aftage i værdi på samme måde. Og så kan jeg endelig tappe into den samme hype, som der er omkring eksisterende blockchain. Ja, nu er det sågar bare en forbedring af den version af idéen, som bare handler om en løbende opdatérbar kæde med normale kæde-mønter. I forhold til denne idé er det nemlig \emph{altid} en forbedring, hvis mønterne bliver fortolket og behandlet mere som aktier i stedet. For al ræsonnement omkring at ``investere i Bitcoin' f.eks.\ handler nemlig lige netop om det: Folk forklarer det med, at de investerer i noget. Men der ligger ikke nogen hjemmel bag, hvad en bitcoin repræsenterer. Så kan man jo sige, at det måske er en del af charmen, at folk ikke helt fortår, eller ikke helt er enige om det, men hvis man ser på det rationelt, og antager, at der altså eksisterer et grundlag, der gør Bitcoin en fornuftig investering, selv hvis alle aktionærende handlede som en person, og kun tænkte på at vinde afkast for hele gruppen samlet set (og lad os nøjes med at antage at denne gruppe bare er alle nuværende investorer, så at disse godt kan være interesserede i som gruppe at vinde andre menneskers penge i fremtiden), jamen så må dette grundlag kunne formuleres. Og hvis det kan formuleres, jamen så må det jo værd at gøre, i det mindste som et godt forslag til en alternativ kæde. For hvis man får formuleret mønterne som de aktier, de bør repræsentere, så vil det jo kun være med til at skabe stabilitet på kæden, hvilket mange ville mene vil være fordelagtigt. Og det er ikke så svært at regne ud, hvad den rationelle begrundelse bør være for f.eks.\ at investere i Bitcoin: Man investerer i teknologien, særligt fordi man forestiller sig, at disse distribuerede digitale netværker kan blive meget vigtige i fremtiden. Jamen hvorfor så ikke erstatte mønterne med faktiske aktier? Jo, der er selvfølgelig det, at hele pointen med systemet er, at det er distribueret og ikke afhængigt af gængs jura. Så derfor er det lidt svært at investere i. På den anden side er hele pointen med systemet, at det er distribueret og ikke er afhængigt af gængs jura, så hvis man tror at dette system kan holde, hvad man jo må gøre, hvis man er investor, jamen så må man vil derfor godt kunne formulere aktierne på selve kæden. Så selvom der ikke til at starte med og nogen lovmæssig binding af disse aktier, så vil der stadig være penge bag, og der vil samtidigt være meget stake i, at nå hen til et juridisk system \emph{på} kæden, der kan håndhæve disse aktier i fremtiden. Ergo må man jo regne med, at investorerne vil guide kæden hen til sådan et system, hvilket de kan gøre, fordi en del af det er, at investorerne styrer sluserne for, hvor de negative afkast (ved ikke, hvad det hedder, men jeg taler om den faktiske investering, der når frem til det firma, aktien ligger i). Og når denne kæde, eller dette kæde-netværk, så vokser sig stor og stærk, således at data-overførsler, storage-tilbud, mønt- og aktie-udstedelser og -overførsler og kontraktbedømmelser kan ske effektivt, jamen har man et veludviklet netværk med meget stake i fra mange forskellige sidder, og det vil altså overhovedet ikke være bare sådan lige at kopiere, slet ikke. Og nu sidder man så som investor på aktierne til dette netværk, hvis virken er afhængigt af at kontrakter bliver dømt lødigt, inklusiv aktier, så der er simpelthen ikke andet for, end at netværket så må betale de afkast, som var lovet via aktien. Samtidigt er der ingenting til hinder for, at man løbende sørger for at udbygge alle vigtige kontrakter og aftaler med kontrakter off-chain. Dette vil tværtimod være stærkt anbefalelsesværdigt. Og bum, her har vi vores idé til forbedret blockchain-teknologi! Og så kan man oven i hatten udbygge denne idé ekstremt meget ved at starte med en hjemmel om, at sigte mod at alle bidrag bliver betalt tilbage ud fra en belønningshjemmel, som jeg lige vil se på at definere senere, men som svarer til at man kan udskyde handlen til lang tid efter, at bidraget er blevet gjort, men alligevel sørger for som minimum at være så fair over for bidragsyderen som muligt, og gerne mere end fair, og om at alle virksomheder højest er temporarily closed source, men på et tidspunkt, løbende, vil overgå til hele foreningen (hvor der så i øvrigt også bør være en hjemmel om at være så inklusiv som muligt i foreningen og aldrig holde befolkningsgrupper udenfor, heller ikke den ``befolkningsgruppe,'' der ankom sent til festen (sent besluttede sig for at ville være medlemmer)). At man vælger at deltage i denne bevægelse, gør for det første, at der er ekstremt meget mere at vinde for sine aktier, og giver samtidigt også en væsentligt større sikkerhed for investor, for selv hvis hele kæden kollapser, så kan du stadig notere dit bidrag, og da det er en virkeligt central (og nemlig super vigtig, for uden den lægger man kraftigt op til selv at blive taget ved næsen, når den fremtidige generation pludselig står til at skulle belønne en selv; gavmildheden betaler sig, og det er rigtigt vigtigt med en ``ubrudt kæde'' ift.\ at alle, der ærligt har forsøgt at give reelle bidrag i bevægelsens navn, bliver belønnet derefter, også selvom bevægelsen når at skifte karakter en smule!) del af kæden, at alle bidrag skal blive belønnet, selv for hvad der skulle vise sig at være falske (men ærlige!) starter (som i ``false starts''), så sikrer man sig jo herved belønning, hvis man ærligt bakkede op om denne bevægelse, hvad man jo gør, når man investerer i den. Fedt fedt fedt! Og jeg kan så komme mere ind på, hvorfor der ellers er meget at vinde ved denne bevægelse (inklusiv en bedre verden), når jeg forklarer om den idé generelt (for denne idé er heller ikke afhængig af blockchain, ligesom at jeg nu endelig, langt om længe, er lykkedes, forhåbentligt (7, 9, 13!), at gøre blockchain idéen holdbar, uden at den afhænger af min forretningsbevægelse-idé). Der er i øvrigt flere grunde til, at det er godt, hvis idéerne fungerer uafhængigt af hinanden, også selvom de er bedre sammen. En af de ting, der lige netop gør det fedt, at blockchain-idéen kan fungere for sig selv, også selvom jeg helst så at det bliver hybrid idéen, der forfølges allerede fra start af, er simpelthen den, at det gør det meget, meget nemmere at forklare om. Det gør det så meget nemmere for mig, hvis jeg bare kan starte med at sælge idéen omkring blockchain for sig, og så bare udbygge idéen efterfølgende, når først jeg har vækket nogens interesse (og dermed er klar til at høre videre). *(Nå ja, og jeg havde også i sinde at nævne: Det koster jo heller ikke noget, at melde sig på bevægelsen, også selvom man ikke gør sig afhængig af den. Derfor er der selvsagt kun noget at vinde ved det.)

Okay, så lige lidt nærmere om, hvordan aktierne skal defineres, for brugerne skal jo ikke bare kunne udstede deres egne aktier. Så pointen er altså i stedet, at brugere der ejer aktie-mønterne kan tilskrive disse mønter visse foreninger/firmaer, der varetager funktioner ift.\ udviklingen og driften af kæde-netværket. Herved gives så en ret til (formentligt efter anmodning først) at udstede nye mønter af samme slags, hvilket jo så kan ses som en konkret investering fra alle andre mønt-ejeres side. Det bliver så investorer og firmaers opgave efterfølgende, altså når kæden når til et vist udviklingspunkt, at forhandle en hjemmel for, hvordan sådanne (løbende) investeringer skal belønnes, også alt efter hvilke firmaer, de gik til undervejs (og brugerne kan selvfølgelig, som nævnt, ændre på dette undervejs og altså tilskrive mønt-aktierne til nogle andre). Det er så smart fra starten af at give rammer for, hvordan denne handel skal forløbe. Man kunne f.eks.\ passende fra start give en hjemmel, der siger, at belønninger skal være en så og så stor faktor alt efter, hvor meget denne investering (efter en ærlig analyse) viser sig at yngle, samt hvor tidligt den blev gjort i udviklingen, således at man også belønner investeringer mere, hvis de blev gjort dengang, der var en større risiko involveret med investeringen. Man kunne også bare sige, at den forhandlede risiko-udfald-vurdering skal sættes konstant for hele netværket, således at alle belønninger, til bidrag i fortiden såvel som i fremtiden, kommer til at regnes ud fra den samme generaliserede formel-hjemmel (som så dog skal kunne justeres lidt efterhånden med tiden alt efter, hvordan holdningen og behovet skifter). Og når folk så gerne vil sikre sig, således at ærlige bidrag bliver belønnet uanset, om de ender med ikke at yngle som forventet, jamen så kunne de jo bare være op til investorende om at forene sig, således at de kan forsikre hinanden. Hm, men (hov) er jeg så ikke ovre i at inddrage forretningsidéen igen? Hm, vel ikke helt: forskellen er jo, at magten til forhandlingerne bliver på investorendes hænder. Så investorer behøver ikke at stole blindt på, at der vil være en samlet bevægelse, der ender med at betale dem tilbage. Hvis bare er nok, der tror på idéen, så vil der også være efterspørgsel efter mønt-aktierne, og da disse aktier er baseret på netop dette, vil aktiehaverne også sørge for at finde villige firmaer/brugerforeninger, og også bare individuelle brugere, til at udvikle og drive kæde-netværket på en sådan måde (og det kan de jo styre, da det er dem, der betaler), forhandlingen, når vi når det punkt, kommer til at foregå efter den oprindelige tanke, og at der således fastsættes en hjemmel for rimelig ret fast risiko-udfald-belønningsformel, og at netværket så kommer til at arbejde ud fra den hjemmel, inklusivt for hvordan de belønner aktionærerne. Hm, det er da i virkeligheden den bedste version af idéen den her. Tja, eller det kan jeg ikke sige. Det kan også være, at folk vil hælde lidt mere til bare at lave investeringerne direkte i stedet, imod selvfølgelig at de får et token for det pågældende bidrag, uden at man behøver at gøre det samme indirekte med mønter. Det kommer kun an på, hvad folk vil hælde mest til. Det er jo ret meget det samme billede kan man sige. Den egentlige forskel er vel i bund og grund, om man vil lade investeringer være mere åbne, eller om man effektivt set vil kræve, at firmaerne/foreningerne vil holde sig til de samme investorer, repræsenteret mønterne, men hvor mønterne, og dermed investeringsprivilegierne, godt kan skifte hænder. Ja, ``investeringsprivilegier'' er vel nøgleordet ift.\ at slå ned på, hvad forskellen er. Hm, og det gælder vel egentligt også for, hvordan et investeringssytem baseret på aktiehandel fungerer, ligesom det vi har, i stedet for et system, hvor firmaet bare søger om specifikke investeringer ud i den brede befolkning / det brede marked, hver gang det føler, at det er nyttigt med yderligere investering. Ja, måske er det sådan, man kan se på det. Hm, ret interessant tanke i hvert fald. Hm, og det system giver vel egentligt meget god mening, i hvert fald hvis man overgiver sig selv til liberalisme: hvorfor ikke handle med det, hvis der er efterspørgsel om det, også selvom ``det'' er noget så abstrakt som at få adgang til en vis form for handel, nemlig en investering? Så ja, det giver vel faktisk bedst mening at gøre sådan. Ja. Grunden til at jeg lige studser lidt, det er, at man i en økonomi, hvor folk bliver belønnet bagud ikke har så meget behov for formuer skal kunne yngle af sig selv helt vildt. Men dette er jo kun på sigt; formuer må bestemt gerne kunne yngle, sikkert i lang tid, men især under opstartsfasen. Og hvis man bare sørger for at kæde-netværks-hjemmelen ikke lægger op til, at disse privilegier er til tid og evighed, men at de vil udløbe på et tidspunkt (muligvis med et ekstra afkast oveni som skal modsvare det faktum at privilegiet ophører), så får man ikke skudt selv selv i foden på den måde. (Det ville være sindssygt ærgerligt at starte et kæde-netværk på baggrund af en hjemmel, der så ikke kan overholdes. Derfor er det super vigtigt altid at prøve at undgå vedtagelser, der ikke kan ændres på efter hånden, og aldrig lave aftaler for tid og evighed.)

(24.01.21) Okay, så jeg har faktisk et lidt nyt syn på det nu. For det første mener jeg ikke rigtigt, det giver mening at prøve at investere i en valuta forud for den underlæggende økonomi, som valutaen skal baseres i. Det svarer lidt til at købe et lands valuta, inden landet er oprettet. Valuta giver mening, når man har et område af verdensøkonomien, så som et land, hvor der er meget indbyrdes handel, og hvor det så er fordelagtigt for de involverede at udstede en mønt, så man bedre kan handle med hinanden, idet man så også kan opspare disse mønter (og bruge dem på et senere tidspunkt). Hvis man så investerer i en (del-)økonomis valuta, før denne er kommet op og køre, jamen hvem siger, at denne økonomi vil bruge den planlagte valuta? Hvorfor ikke bare udstede nogle nye penge. I et idealt system, hvor alle havde perfekt overblik over, hvor meget de skyldte og dermed, hvad de har kapacitet til at låne ud, så ville den bedste løsning bare være, at folk kunne udstede en slags checks som betaling i form af kontrakter om tilbagebetaling (i form af produkter eller ydelser). I et andet fint system har man bare en central enhed (en nationalbank f.eks.), man kan gå til for at låne mønter til at lave en handel, og hvis disse mønter ikke ændrer kurs, kan man så begynde at handle frit med dem derefter, uden at trække et spor af kontrakter efter sig. Når der så er nok penge i omløb kan den centrale enhed stoppe med at give/låne mønter ud, for så kan folk klare sig med dem de har. Her er det så bare vigtigt, at mønterne i starten fordeles på en fair måde, for når man ikke forventer at de skal leveres tilbage til den centrale enhed inden for en tidsramme (og der ikke er nogen renter på ``lånet''), så er det unfair, hvis man starter med kun at give mønterne til nogen. Nå men hvis vi så forestiller os, at der allerede er en enhed, der siger, ``hey, vi har allerede mønter klar til jer, men hvis i bruger vores mønter, så følger der lige den hage med, at vi allerede i vores gruppe (f.eks.\ Bitcoin-fælleskabet) har hoarded en masse af disse mønter selv, og forventer altså selv at kunne bruge disse mønter løbende til at kunne købe jeres produkter og services på lige fod med jer selv.'' Så Bitcoin-idéen giver kun mening, hvis enten at Bitcoin-fælleskabet selv skal udgøre den økonomi, der er tale om (så de kan skal sælge denne aftale til sig selv), eller hvis man satser på, at det fælleskab, der i sidste ende skal levere de produkter og services, der skal handles i Bitcoin, enten har meget positive følelser over for Bitcoin-fælleskabet (og altså bitcoin-ejerne særligt), eller at de på en eller anden måde ikke selv vil være i stand til at oprette en valuta som beskrevet. Nogle punkter, man kunne nævne i flæng som modargumenter til disse antagelser, at bitcoin-ejerne nok ikke har så bidraget så meget til udviklingen andet end (på indirekte vis) at have givet løn til minerne, og at systemet samtidigt er så ineffektivt, og svært at opdatere, at det kan være svært at argumenterer for, hvad al den mining skulle gøre godt for, når alle og enhver kan se, at Bitcoin ikke dur som valuta-system i sidste ende (for det koster al al for meget energi... Hm, nå nej, det kan være at den energi (strøm og computerkraft) aftager, når mining-lønnen bliver lille nok....). Hm.. Okay, så har Bitcoin en lille chance. Men igen kræver det enten stærke positive følelser over for Bitcoin-fælleskabet eller en manglende evne til at opstarte en ny valuta.

Hvis man nu erstattede denne premature investering i valuta med faktiske investeringer i web 3.0-teknologier osv., jamen så er der for det første en større sandsynlighed at web 3.0-teknologier vil føle positive følelser, eller i det mindste føle sig i gæld, over for investorene. Og endnu bedre, dele af dette fælleskab \emph{vil} så faktisk være i gæld til investorene (da det \emph{er} faktiske investorer), og dermed kan man eliminere chancen for, at dette forretningsfælleskab vil vende ryggen til deres investorer. 

Og dette er i tråd med det nye jeg også tænker, nemlig at lægge op til, faktisk bare at starte alle kontrakterne off-chain, eller i hvert fald så mange som muligt. Det er i hvert fald en god idé at gøre langt hen ad vejen. Der kan også være en idé i at binde sig til kontrakter på det nye netværk: Det kan f.eks.\ være en rigtig god idé for firmaer, der følger min forretningsidé, at købe deres egne bidragstokens, hvis de også har i sinde at være med at opstarte hele dette belønningssystem, for det vil være en klar måde at signalere over for mulige klienter, at man tror på idéen og at man har stake i idéen/bevægelsen. Plus at det muligvis kan være lidt nemmere at gøre, end at formulere nogle eksakte juridiske kontrakter med samme effekt off-chain (altså i den ``virkelige verden'' så at sige).

Lad mig lige nævne, at Bitcoin sagtens kan fungere som en selvstændig ting, der bare passer sig selv, og ikke prøver at være yderligere med i web 3.0-bevægelsen, og det er også helt fint (især hvis energiforbuget på mining på verdensplan aftager, så er det jo intet problem). Ja, mon egentligt ikke Bitcoin kommer til at overleve, om ikke af andre årsager, så alene af den grund, at der er så høj anonymitet i sådan en type kæde..

Så jeg er altså tilbage til at fokusere på en idé, hvor det ikke handler om at handle med mystiske coins på en kæde, men hvor der så vildt muligt handles med aktier og kontrakter osv. Det er stadig super oplagt for idéen, at benytte web 3.0-teknologier, og idéen er også stadig interessant ift.\ gængse blockchains, for jeg mener, at man bør kunne starte med en idéen om en kæde a la Bitcoin og så ``opdatere'' den i skridt, indtil man når en idé, der er en delmængde af min fulde web 3.0-idé. 

Man bør nok stadig kunne danne nogle aktier, som er samlede aktier i en masse netværks-firmaer/-foreninger, og hvor investorene så har ret til at lave udskiftninger af firmaerne på den samlede aktie, og i det hele taget har ret til at åbne og lukke investeringssluser, når det kommer til at gøre køb og lave udvidelser for de individuelle firmaer/foreninger. 

Jeg har haft meget på sinde, at en prædiktiv ontologi bliver en vigtig ting at samle i web 3.0-databasen, men noget andet, der selvfølgelig er vigtigt er at samle på folks bidrag, inklusiv uploadede idéer og programmeringskode og opdateringsforslag. Man bør også holde styr på folks stemmer, så folk der har indstemt en god idé også kan belønnes efterfølgende af den samlede bevægelse. 

Jeg skal også huske, at det ikke kun handler om risiko og udfald. Der bør også være en faktor, der tilgodeser god eksponering af bevægelsen. Hermed bør især de første bidragsydere til bevægelsen virkeligt blive belønnet meget. Og jeg vil gerne understrege i den forbindelse: Der er ingen grund til at være nærig med disse belønninger! Folk vil hellere end gerne hylde og værdsætte dem, der har gjort meget for økonomien. De fleste mennesker har endda tilbøjelighed til at være betaget af sådanne stjerners rigdom. Og det er jo virkeligt dejligt. Denne indsigt kan forstærke den indledende opbakning til bevægelsen, hvilket er super vigtigt, og \emph{er} super vigtigt at belønne! (24.01.21)

(26.01.21) Jeg har lige ville huske at nævne, at det stadig er værd at foreslå idéen med mønter, der giver en slags investeringsprivilegier, men måske er det faktisk mere end det. Ja, det er smart, hvis man kan lave rigtigt mange off-chain kontrakter til at understøtte det hele, men det ville også være super dejligt med en løsning, hvor man hurtigere kan komme i gang (ved at have ret sikre on-chain kontrakter fra starten af). Og måske kan man så faktisk netop bruge sådanne mønter i opstartsfasen. Pointen er, at hvis mønterne har et afkast, som jeg tænker det, så er der ingen problem i, at de falder i værdi med tiden: Ikke hvis de falder netop fordi man forbruger af det samlede afkast, der er lovet løbende til møntholderne (hvor mønterne så kan skifte hænder; der er altså tale om (digitale) aktier med afkast for en begrænset periode (længden hvilken godt kan afhænge af mange faktorer, ligesom at selve afkastet også kan)). Så disse aktie-mønter kan altså være en ret til at finansiere f.eks.\ miners imod senere afkast for disse bidrag. Og kan så bindes på selve mønterne og ikke på brugerne, sådan at man også bare sælger al tilbageværende afkastet, når man sælger ``mønten'' (for så skal man ikke til at identificere brugere osv.\ fra starten). I øvrigt kan mønterne også give stemmeret på kæden, hvorved man så også får ret som møntholder til privilegiet for at være med til at afgive de første mange vigtige stemmer omkring kædenetværket, hvilket også bør give ret til at afkast, hvis altså mønten har været med til at stemme for tilsyneladende gode forslag. Det er så også vigtigt, at man hurtigt får gjort det muligt at logge alle mulige andre aktier på kæden, så kædenetværket også kan indgå aftaler om belønninger (og altså afkast) til folk, der bidrager ved at udlåne (imod nævnte afkast) eksterne værdier på kæden. Nå men den store pointe her i alt dette er, at man måske godt kan lave en kæde, der ikke nødvendigvis afhænger af en stor økonomisk bevægelse (forretningsbevægelse), og hvor der er mønter på kæden fra start af, men som faktisk bunder i noget. Hm, spørgsmålet er jo så, hvor godt man kan få disse mønter bundet op på noget, så de er reelle aktier, man kan stole på, samt hvor meget denne sikkerhed beror på, at man forventer, at der kommer en bevægelse med tiden, der vil sikre, at tidlige bidragsydere vil blive belønnet..  Hm, det ville være dejligt, hvis minere også indgik en form for kontrakt... Tja, men det gør de vel på en eller anden led, hvis de bliver betalt af selve mønterne... Hm, og hvis man måske kunne lave en skat af al senere handel... og måske bruge juraenhederne til at bedømme senere værdi, der er logget på kæden..?.. Tja, min anke (som er ret ny) imod gængse blockchain-mønter er jo lidt, at det ikke baserer sig på en underlæggende økonomi, men det gør det jo lidt alligevel, idet at selve teknologien og netværket kan danne ramme om en økonomi med meget ``intern handel.'' Hm... Men det kræver så igen, at nye firmaer og aktører, der logger sig på kæden, skal være i gæld til møntholderne på en måde; så måske hvis de er finansieret af samme mønter?... Hm, det korte af det lange er vel, at det ikke kræver en ny forretningsbevægelse for, at kædenetværket vil have en stor afhængighed af, at on-chain-kontrakter bliver dømt sandfærdigt på kæden, så dermed kan man vel nok skrive kontrakter på kæden fra starten af uden at behøve en masse juraarbejde fra den ``virkelige verden.'' ... Og at aktie-stemmeret-mønter måske gør det hele lidt lettere fra starten af... Hm.. Ja, virker det egentligt ikke? Virker det ikke med aktie-stemmeret-mønter? Tjo, jo, og forskellen fra hvad jeg før længe har tænkt er så, at nu kan jeg se, hvordan værdien af disse mønter bør grunde i et fremtidigt afkast. Og her handler det så om, at kunne vurdere ekstern værdi over kæden, så at kontrakterne omkring dette afkast kan bedømmes korrekt. Jeg tænker så lidt, at møntejerne starter med at formulere en lidt løs hjemmel for, hvad mønternes afkast skal være på sigt, alt efter hvordan netværket udvikler sig, og skal særligt love en udløbsdato på stemmeretten (svarende lidt til at stemmeretten kommer i form af et ``afkast''), og at det så skal være fællesskabet af møntejeres ansvar, at denne hjemmel bliver overholdt, men at møntejerne stadig bevarer magten til at formulere de endelige, mere konkrete mønt-aktier, når teknologien og netværket er modens dertil. Det bliver altså en slags handel, hvor møntejerne må forholde sig til, at hvis de bryder hjemmelen, så kan det vel være, at den bredere brugerskare vender dem ryggen, men hvor de stadig vil prøve at fastsætte kontrakten så fordelagtigt som muligt for dem selv uden at være i fare for at overskride denne tærskel. Hm, jeg tror altså næsten godt, det her kunne virke... Der bør så lige være lidt opdeling af netværket, så der er større sikkerhed for den bredere brugerskare, at de kan efterlade møntejer-fællesskaber, hvis de bryder deres hjemmel (eller at man finder ud af, at hjemmelen var utilstrækkelig og/eller havde huller, og hvis disse huller så bliver udnyttet af møntejerne). Tja, men der må man jo lave et system, hvor møntejere kan bryde ud af mængden for så at danne ramme om en ny tilsvarende kæde, men hvor kontrakterne så kan fastsættes anderledes.. Hm, eller måske skal man gøre mere end det..? Hm, eller også skal man faktisk tvært imod helt lade være.. og så sige, at en del af satsningen ved at bakke op om en kæde, der er at man satser på, at møntejerne, når den tid oprinder, ikke vil bedrage den bredere brugerskare og fastsætte kontrakterne uretfærdigt og anderledes, end hvad der rimeligt var forventet. Og hvis ikke de formår det, så må den brede brugerskare simpelthen bare kollapse kæden og prøve at starte forfra (nu med mere viden og erfaring, men selvfølgelig dog med et mere splittet samlet fællesskab). Bemærk at denne version af idéen kun giver mening som en måde at fremme udviklingen af kæden, fordi man så kan undlade en masse juridisk arbejde fra start af, og også fordi at aktierne så kan starte med en rigtig lav værdi, og at de dermed vil få en høj relativ stigning, hvilket kan skabe værdi, fordi mange mennesker så vil få dollartegn i øjnene, og den underlæggende idé behøver altså ikke at bruge disse indledende aktie-stemmeret-mønter, især ikke hvis man tror på den underlæggende forretningsbevægelse, men derfor kan det stadig være en rigtig rigtig nyttig idé. Det kan sagtens være, at det bliver dette, der skal til for at sætte gang i hele kædenetværket. Og det er jo kun godt at have flere forskellige versioner af en idé, hvis de alle virker potentielt gangbare. (26.01.21)


(30.01.21) Brainstorm over nuværende blockchain-idé: Kæde der matematisk definerer token/mønt-fordeling ud fra kæde og data, som kan opdateres... Blok-vægtning?... Semantisk model, der kan tale og måle på resten, som også kan kan opdateres (meget), som bl.a.\ bruges til at indfortolke en værdi bag de tokens/mønter, man starter med at få dannet på kæden... Hvad med protokol-opdatering?... Mere end bare blok-vægtning?... Jeg tænker også faser igen... Men det kan vist komme ovenpå... Man skal kunne rette modellen, men man bør altid holde løfter, for hvad gør man, når først man har brændt brugere af (og vist at dette er en mulighed)? Så indledende præcision af modellen er mest for de første brugeres (med kontrakter) skyld... Ja, opdatérbar blok-vægtning er faktisk et tilstrækkeligt princip til protokol-opdatering; og helt på sin plads, at der er en hovedkæde i denne forstand... Blok-vægtning gør også, at man kan lade vigtige blokke komme foran i køen og udskyde på mere mondæne *(kedelige; mondæn betyder vist noget andet på dansk) opdateringer... Skal jeg lige tænke lidt mere over, hvordan man sikrer sig, at de første kontrakthavere ikke bliver efterladt?... Tja, nej, for man kan vel mindst det samme som på Bitcoin... Men så om at token/mønt-vedtægter bliver fortolket lødigt og håndhævet ærligt?... Eller skal man bare se det som aftaler mellem mennesker, hvad det hele jo alligevel er?... Ja, det hele er jo bare aftaler, så hele kædens integritet afhænger af, at kernebrugerne/kerneenhederne kan overholde aftaler... Ja... Og så handler det også bare om at nå et punkt, hvor kontrakter kan ankes så de kan genvurderes i fremtiden, eventuelt med ``fremtidsception-princip'' (som bare er at man kan blive ved med at anke beslutninger til næste generation af brugere, og at uærlige dommere også kan få en vis straf i sidste ende), og hvor fællesskabet så eventuelt kan give erstatninger... Ja, så jeg har vist rimeligt godt styr på det grundlæggende lag nu. Nej, vent... Udstedelsen af mønter/tokens... Skal bare være frit for enhver bruger eller gruppe. *Blok-rotation?... Kræver bare en PoS-protokol, som man så også kan opdatere. Det hører jo med, at fordelingsfunktionen ikke skal behøve al data for at fungere. Det kan derfor sagtens lade sig gøre at danne en protokol, hvor man kan omskrive og erstatte tidligere blokke i kæden.

Og i det næste lag skal man så tænke over, hvordan man starter et fælleskab på en god måde. Det er så her jeg tænker, at der skal dannes en variabel mønt-mængde, som kan mines PoW til at starte med, og hvor man så fra start af får indskrevet en hjemmel (aftale) for, hvad disse mønter skal indbringe og hvornår, hvis fællesskabet ender med at denne ramme om et godt web 3.0-netværk... Hm, men det betyder at mainchain skal være inklusiv over for alle side chains... Hm, nej mainchain skal ikke gå efter at være en hub for al web 3.0-trafik, ikke nødvendigvis... Tja, mainchain kan godt lægge op til, at den vil prøve at host'e så mange løsningsforslag til en web 3.0-start som muligt, ved at den så nærmest kan formere sig og danne nye tilsvarende børnekæder, der så får mulighed for at kommunikere med hinanden, og i det hele taget bør den prøve at holde styr på alle andre web 3.0-strukturer, særligt alle andre blockchains, men derfor bør den alligevel selv prøve at sigte imod at blive en web 3.0-startløsning i sig selv... Og hvis den fejler kan man så bare lade et barn tage over og blive ``mainchain'' i stedet... Hm, kunne man gøre det sådan, at børnekæderne faktisk skal sørge for at bygge nye blokke til hovedkæden indimellem?... (Uh, og jeg har endnu ikke fået nævnt blok-rotation; det skal jeg lige huske... Nu har jeg.) Hm, for det ville være meget rart, hvis man bare havde nogle mønter, og at hjemlerne for, hvordan man fortolker disse mønter kunne være modeller, der blev opbygget og udbygget/ombygget med tiden, og hvor hjemlerne så også kan forke, og hvor man kan logge sine penge på forskellige hjemler... Ja, så det er sådan et system, vi er ude efter; hvor hjemmel-modellerne udbygges og forker ligesom jeg har tænkt det meget med de prædiktive (og har også tænkt det med hjemlerne, men nu har jeg lige husket det lidt igen)... Kæder kan så også bruge hinandens prædiktive modeller til at hjælpe med at bedømme (mere upartisk endda), hvorvidt (under-)kædens egen hjemmel er blevet overholdt.

(31.) Hm, noget med at minere selv vælger modellerne?... Tja, måske (faktisk)... Måske med en valgfri header, som f.eks.\ startblokken kan få, der henviser til modellen... Tja, fin måde at gøre det på, men det er ikke vildt vigtigt... Hm, en god mainchain skal jo bare sørge for, at alle reelle grupper (der bare kan udføre selv en lille smule work, og dermed vise, at der er en interesse i at skabe en ny underkæde), kan få lov at udstede deres egne mønter med tilhørende protokoller og model og hjemmel... Hm, men hvad er idéen så i at spawne disse underkæder fra en main-kæde?... Er det bare, så kæderne kan indgå aftaler indbyrdes (og f.eks.\ forsikre hinanden)?... Og så der er et fælleskab, hvor alle kæder ligesom kan anerkende hinanden formelt?... Og en af de aftaler, som underkæderne kan indgå med hinanden kan være, at de alle skal bidrage lidt til hovedkæden... eventuelt... Hm, kæder bør i øvrigt sørge for at få så mange personligt underskrevne statements med på kæderne som muligt, så man sikrer sig en større sandsynlighed for, at partier vil blive straffet i fremtiden, hvis de bedrager og ikke overholder deres løfter... Ja, dette web 3.0, jeg beskriver her, handler i stor grad om, at danne et netværk, der gør det nemt at lave en masse aftaler og kontrakter med en bred brugerskare... Og som jeg har nævnt på papir, så skal det være muligt for brugere af andre kæder (der ejer mønter fra disse), at logge deres mønter på en fortolkning af værdien for således at prøve at skabe en konsensus om, hvordan de skal veksles, hvis nok andre brugere går med på idéen, og så skal det bare være sådan, at disse brugere altid kan afmelde sig denne fortolkning igen, inden at tilslutningen er nået over en vis tærskel, og kan derfor sagtens logge sig over på en ny fortolkning inden da, hvis der kommer en, der er bedre for dem. Hm, hvad skal jeg mere brainstorme om?... Tja, nu var det lidt tilbage til første lag... Og der er flere måder, man kan starte dette på... Og derefter handler det jo bare om, hvordan kæder kan forsikre hinanden, hvordan de opdaterbare protokoller kan se ud, hvordan brugere kan ``overvåge'' hinanden uden at lægge private oplysninger til omverden, men hvor oplysningerne bliver i en lille tillidsgruppe, som hver bruger kan være en del af... hvordan gamle kæder kan omfortolke sine egne mønter... hvordan kæder kan referere til deres egen udvikling for at give brugere mulighed for at lave endnu mere komplicerede kontrakter, end det er muligt at dømme ved tidspunktet af underskrivelsen, og hvordan man f.eks.\ kan bruge ``fremtidsception.'' ... Hvordan kæder kan danne et netværk, hvor alle kæder primært passer sig selv, men hvor de også kan snakke sammen med hinanden og lave overførsler og indgå kontrakter på tværs... Og hvordan kæder bør benytte og vedligeholde prædiktive modeller... Jeg synes således, jeg har ok styr på det nu. Kan være at jeg bare vil fortsætte denne ``brainstorm'' her (lige neden for), hvis jeg får flere små tilføjelser, som jeg vil skrive ind her i dokumentet (så tingene stadig står lidt samlet (og kompakt), også inden jeg rent faktisk begynder, at skrive en længere sammenhængende tekst).

(02.02.21) Okay, så lige for at runde helt af, så vil jeg lige nævne, at man bare bør udvikle et format for, hvordan man initierer en kæde med en vis grundmodel. Det bliver så det allerførste lag. Og så er det så smart at starte en kæde med en model, hvor denne kan opdateres over tid (med PoS-protokol), så man ikke behøver at bekymre sig om, hvorvidt man finder en bedre start-model en kædens egen (for så kan man bare selv adoptere denne). Meget af en kæde handler også alligevel af de aftaler, der bliver underskrevet og disse vil så typisk kun rette sig til en specifik kæde. Bemærk at man kan omfortolke eksempelvis Bitcoin ved at oprette en ny kæde, der bruger Bitcoin-protokollen, men hvor der så er et ekstra lag af aftaler til at omfortolke bitcoin-værdien, hvis nok brugere går med på det samtidigt, og således altså migrerer til den nye version af kæden med det ekstra lag. Jeg har tænkt mig at gå tilbage nu og skrive om ITP, prædiktive modeller og alt det omkring en ny open source bølge for endeligt at returnere til blockchain. (Og så vil jeg gå videre til eksistens og bevidsthed og slutte af med fysik (men måske først efter lige at gå igennem de andre ting endnu en gang), som jeg nu faktisk tænker (det kom jeg frem til for et par dage siden), skal være det første, jeg satser på at komme ud med for alvor.) I mellemtiden vil jeg så bare lige holde en åben sektion lige nedenfor, hvor jeg kan skrive nogle punkter ind af vigtige ting, når det kommer til de øvre lag af blokkæderne, og altså hvad man bør og hvad man kunne sigte efter, med sin protokol og hjemmel-model. *(Det har jeg droppet igen.)



%\subsection*{Korte huskenoter omkring idéer til, hvordan de øvre lag af mine selvbeskrivende blokkæder kunne/burde være (02.02.21--???)}
%
%\begin{itemize}
%\item 
%
%
%
%\end{itemize}



\chapter{VR}



\phantom{P\\}
\newpage
\section{Possibilities with VR}

(22.01.21) Jeg har nogle idéer omkring VR. Jeg vil bare lige nævne her, at jeg lige kom til at tænke på, hvordan bedre VR-teknologi kan gøre det muligt at være freelance underviser. Jeg tænkte så videre på, hvordan VR-teknologi også kan komme til at forene folk meget bedre. På det nuværende internet gemmer folk sig ofte bag anonyme navne og avatars, og er meget offensive og krigeriske over for hinanden. Men hvis man fik et godt VR-system, hvor folk kan leve sig ind i det, og hvor folk kan bruge ikke-anonyme brugere, hvor selve samtalerne selvfølgelig er anonyme, men hvor den individuelles adfærd kan rygtes, så ville det åbne op for, at folk meget nemmere kunne mådes og tale med folk, som de ikke ville ses med ellers. Dette kunne åbne op for meget mere forståelse og forsoning mellem forskellige befolkningsgrupper (inkl. klasser osv., og også på tværs af lande). Det relaterer sig så lidt til det med virtuel undervisning, for det er også en slags undervisning at høre om andres liv, erfaringer og den viden, de bærer på; vi bærer jo alle sammen på en stor viden, og på mange indsigter. 




\chapter{Objects and machines}

%(21.10.21) Jeg gennemgik lige mine papir-noter. Der var lige en lille idé omkring, at et simpelt (igloagtigt) teltdesign med en solid og isolerende ydermur måske kunne være en billig måde at skabe en masse nød-bolliger for en stor mængde mennesker. Og hvis man så bare kan koble flere telte sammen til et samlet alarmsystem (som måske kan være totalt modulært opbygget uden noget behov for en central enhed), så kunne man måske nemt impelementere et overvågnings- og alarmsystem, så folk kan føle sig sikre i teltene og ikke mindt i at forlade teltene med ejendele i sig uden frygt for tyveri. Bare en lille tanke.
%En anden lille idé er bare en gængs lille produkt-idé til en fysepose, hvor man kan hive i en snor efter at have puttet f.eks. kød eller en gryderet, hvor posen så strammer sig sammen til, hvad der svare til en isterningpose, og hvor man altså herved (måske efter lige først at have fladgjort og udjævnet posen og dens indhold en smule) kan få posen opdelt i lommer. Man kunne så brække lommerne af, men jeg tænker, at man bare skal kunne åbne posen forfra og tage en "terning" (som i "isterning" (men terningerne skal selvfølgelig gerne kunne være meget større)) af indhold ud ad gangen fra posen. Også bare en (meget) lille idé.





\phantom{P\\}
\newpage
\part{(old)Existence and consciousness}
\chapter{Existence and consciousness}

\subsection*{Note omkring hvorvidt ``sjælen'' hurtigt finder videre eller forbliver i et limbo i lang tid, antaget at den ikke dør og at den ikke er spredt ud på flere individer}
(16.01.21) Har taget det virkeligt roligt i dag og bare tænkt lidt filosofi indtil videre. Jeg tænkte på at nævne, at jeg jo er nået frem til, at vi meget vel kan have en sjæl, der binder sig til én hjerne ad gangen, men som overlever vores egen død, så at ``reincarnation'' derfor kan være en mulighed i en meget direkte og bogstavelig forstand. Så har jeg så tænkt lidt i dag, om ikke en sådan sjæl (og jeg er stadig tilbøjelig til at tro at det samlede multivers indeholder mange sjæletyper, som så selv har en frekvens, der afhænger af, hvor kort de kan beskrives i et gennemsnitligt sprog) vil blive hængende i længere tid, og prøve at koble til døde ting, inden den potentielt kan finde vej til en ny hjerne. Men jeg er så lige noget lidt frem til, at det i forvejen må være et ret skarpt potentiale, der fastholder sjælen til en meningsfuld hjerne med meningsfulde oplevelser. Hm, det var faktisk et meget cool billede på det, det med at tænke det som et potential-felt. Nå, men således tænker jeg altså, at hvis sjælen lever videre, så må det faktisk ret hurtigt finde vej hen til et nyt potentiale-lavpunkt, og dermed hen til en ny hjerne. Bemærk at med antagelsen om, at der kun er én sjæl i gang ad gangen i et univers, hvilket giver meget god mening, for som jeg ser det er alternativet en sammensat sjæl, hvor den samlede bevidste oplevelse af universet består af... Okay, det er faktisk også meget interessant. Men lige først, hvis der kun er én sjæl i gang ad gangen i et univers (som så er centreret alene om denne sjæl (og navnligt om dennes bevidste oplevelser) og ikke andet), så kan en sjæl altså sagtens finde vej til en allerede levende hjerne; det behøver ikke at være en nyfødt hjerne eller en hjerne der er ved at fødes eller, da der jo så ikke er nogen sjæl, der allerede optager den. Dermed er det så ikke sagt, vil jeg lige nævne, at man på noget tidspunkt kan betragte andre mennesker som ikke-levende, for det vil de bare være på et andet tid/sted i det samlede multivers. Nå men det spændende alternativ, som jeg lige kom til at tænke på, som også virker som en meget fornuftig mulighed, ville så være, at der er en form for udbredt sjæl, som en slags udspredt bølge i universet (he, et ``sjæle-felt'', om man vil, hihi), hvor at der så dannes bevidstheder i de steder i rum og tid, hvor der er sammenhængende hjerne-aktivitet, og hvor man så derfor kan forestille sig en fordybning i et potential. Jeg ville så være tilbøjelig til at tænke, da jeg ikke rigtigt kan tænke bevidstheder som andet end noget diskret, at der så vil være en tærskel for, hvornår en hjerne-aktivitet er stærk nok til, at der dannes en bevidsthed. Denne tærskel vil så formentligt afhænge af det individuelle univers, og selvfølgelig også af, hvad der grundlæggende set skal til for at udgøre en bevidst oplevelse; hvad der som minimum er ingredienserne i den samlede eksistens (det samlede `multivers', som jeg bruger ordet). He, det er altså en meget cool metafor for det, med et sjæle-felt og et hjerne-potential, også selvom det selvfølgelig dækker over noget mere indviklet. Lidt cool at metaforen også minder så meget om kvantemekanik, når man så også endda antager den nævnte diskrethed for, hvornår sjælen fastholdes i et potential. Cool-cool.



%
%
%\section*{Thought and predictions about the future of programming and of the web}
%%\chapter*{Semantic F-IDE and ITP community, predictive ontologies and a more user-driven web (with a mention of my blockchain ideas)}
%
%Some of the areas that I am especially excited about, where I think a lot of good will from future development of the technologies, has to do with software development and the development of the internet and web. ...
%
%\section*{Thoughts on the future of interactive theorem proving/provers (ITP)}
%I have actually spent a lot of time thinking about ideas for an interactive theorem prover. The reason...
%
%But I have realized that it does not come down to what logical foundation... 
%You can do a lot of things... And there is already thought about reflection.
%
%\subsection*{}
%So the overall things to take away from this whole project of mine, is just a few ideas about some perhaps different approaches to the subject than the ones that already exist.
%
%The ideas might still be good in terms of the development of ITP, and they also tie in nicely with the subsequent ideas that I will write about here. 
%Since I think it will be very important to be able to formulate ontologies mathematically... ...improved ITP environments will be important no matter what. But with my recent revisions of the whole idea, from where I have started to focus more about filling gaps in rigid proofs, with non-rigid, but still formal, deductions that uses human certificates to add certainty to non-rigid parts of a proof, the idea is now also closely related to my blocknet idea [I will actually rewrite and restructure this section! ***********].
%
%\subsection*{}
%I have a few ideas worth mentioning about how the ITP itself could be constructed and then I have an idea about how it might be beneficial to take a different approach to theorem proving itself and allow for parts of proofs to be written informally and to then use human-certificates to add certainty to those proof part. 
%
%\subsection*{}
%Let us discuss the first/latter...?
%
%
%
%
%
%%\subsection*{(About how and when I have thought about this subject as an introduction)}
%% subsection?: Even less about getting the machine to find a proof...
%\subsection*{Embedding the PIDE and using nested embedded environments for teaching and for start-off points}
%
%\subsection*{How a community can formalize how to skip steps (for later)}
%%%Certificates and certainty parameters, protocols, 
%	%(protocols for how to give a certainty parameter to a statement, and then a pridiction about said parameter on top.) 
%	%(example: explode concepts that are part of a protocol.) %(Whatever real protocol you use, formalize it right away and rest easy that it can be exploded futher later on.)
%	%Security is important (actually, very)!
%	%And the fact that folksonomy-(or semantic comment-, rather)people have an interrest in being understood (a big interrest in fact)!
%	%And technical ontologies are of course important as well..
%	%Mention other applications where people want to give attributes as trust points to experts..
%%%F-IDEs mention. Technical ontologies mention.
%%%More on protocols probably.. 
%%%semantic ontologies with concepts, not words, as the fundamental terms.
%%%Predictive user ontologies?... Yes, and technical ontologies.
%%%F-IDEs finish.
%%%Web and Desktop applications.
%
%%"...Hm, og så har jeg (måske) noget at tilføje omkring, at mere formelle specifikationer kan åbne op for mere bruger-( og freelance-)deltagelse og for bedre bruger-feedback.. Og en del af dette er så i øvrigt, at det kan være smart, hvis brugere bare kan uploade ræsonomenter (a la tactics), som andre brugere så kan vurdere som sande eller falske, og self. evt. bevise. Tror dette er en god og vigtig lille pointe; så selv det at "springe formelle skridt over" kommer til at gøres på mere formel vis."
%%Og læs også nedefter fra den sektion (og det første af sektionen) i 'huskenoter.'
%%Har muligvis indflydelse på dispositionen: "- Og man behøver vel ikke 100 % bevise semantik for sprog og compiler; her gælder samme princip om, at man godt bare kan nøjes med sandsynlighedsparametre i stedet og så sige: så længe disse muligheder for ukorrekthed bare er formaliserede i det mindste.."
%
%\section*{Thoughts on the future of F-IDEs (or formal programming and/or other relevant terms) (with an explanation of predictive ontologies)}
%
%\section*{Thoughts and predictions about the future of the web (and Desktop applications) ...}
%
%\section*{Ideas for blockchain technology in the future}
%
%
%
%	%BC mathematical contracts does not need to follow the most popular ontology.
%
%
%
%\section*{A business community for a more open, dynamic and sustainable businesses world}
%
%%mention parties..
%
%
%
%
%
%
%
%
%
%%Not part of these notes: \chapter*{QED theory} (but I should just mention it, I guess)
%%\chapter*{Energy and agriculture ideas}
%%\section*{My idea for nautical agriculture...}
%\section*{Other thoughts}
%\section*{The not-too-far-away future}
%\subsection*{Tourism and eventful local communities}
%\subsection*{Futocracies (futurocracies?) in general} %Maybe mention that policing will be trivial, once the system is transparent and very trustworthy for each individual.
%\subsection*{My idea for immersive VR}
%\subsection*{My energy and agriculture ideas}
%\subsection*{How we will even map out our psychology, which will prevent future failed systems (and no failed/misguided revolutions) from that point on}
%%Maybe also mention that there most be a way to put the brain in stasis, and that the future people will only benefit from being able to meet and talk to people of past.
%\section*{Evolution and psychology (maybe; and only short notes if so..)}
%\section*{Happiness in general}
%\section*{Plugging my QED idea, which I will write about in another document (maybe by mentioning it in relation to the end of the galaxy..)}
%
%
%
%
%
%
%
%
%
%
%
%
%
%
%%Intended to be the first chapter when I wrote it:
%\section*{CUH and consciousness}
%The theme of these notes is going to be a positive outlook on the future and the world overall so it makes sense to begin with this very positive subject on how the multiverse\footnote{I take the `multiverse' to mean the entirety of existence. This is by way even if one assumes this multiverse to consist of only one universe.}, I believe, is infinite and with an infinite number of repetitions as well as approximate repetitions. In other words, we will live again an infinite number of times and our lives will have an infinite number of variations. 
%
%Some of these life path variations will not be as likely as others and thus will not have the same frequency as others in the total multiverse, but there will certainly exist variations from each imaginable life path, i.e.\ each continuous collection of conscious experiences, to another one. 
%This means that in a very real way, it makes sense to think of all beings, humans in particular, as being reincarnations of one another (the order does not matter; everything repeats itself anyway). To me this is a very nice thought and it is perhaps one that might instill a more caring attitude in some people if they believe it to be true.
%
%There are a lot of different ways to reach the conclusion that everything repeats itself this way. In fact, I am convinced that most attempts at a theory of existence would reach a similar conclusion and I can think of only a few type of theories that will avoid this conclusion. If we look at theories that take a mathematical approach to the multiverse and assumes that each universe follows the laws of some mathematical (possibly very broad) theory and where the actual multiverse then either consists of all possible universes of this kind,\footnote{It should then of course also be part of the theory what exactly ``consists of all possible universes'' means, and in general, one have to deal with the problem of assigning probabilities to an infinite set, whenever a theory entails that we live in an infinite multiverse. I will get back to this matter.} or that the set of universes in existence is picked randomly by some mechanism and by some probabilistic theory on top of the multiverse theory. Any such theory that does not entail an infinitely repeating multiverse will then either have to assume a very narrow set of possible universes to a point where it just does not make any sense for a theory of the entirety of existence to be that narrow, or it will have to assume that only a finite number of universes are picked by a random process. In either case it just does not make sense really to limit the entirety of existence this way. In my view, once you assume that there is a mathematical theory behind the laws of all universes in existence (ever!), you almost have to agree that theory should both be very broad on a fundamental level (it should in fact fulfill some profound symmetry of whatever fundamental logic is behind the concept of existence itself, if we are really honest) and it certainly should not be limited by some finite number. 
%As an example, one could theorize that the multiverse consist of only universes with the same kind of physical laws as ours. The current best guess is that the physics of our universe comes from a number of so-called quantum fields with a certain set of symmetries to them, at least if we disregard the physics of space and gravity. But even in a theory that describes a set of possible physical laws for all universes only restricted to such quantum field laws (i.e.\ laws of a \emph{quantum field theory}), unless one chooses some special limit on this set, the set would include an infinite number of quantum field theories that reflects any individual theory. If we for instance look at our own universe, where current wisdom has it that a lot of the physics is based on a SU(3)$\times\,$SU(2)$\times$U(2) symmetry ...
%
%Let me restart this last paragraph. Even though I have my own beliefs on how the multiverse is structured, and how conscious thought and experience is part of it, I believe that you do not have to agree with me all the way to still reach the same conclusion about the infinity of existence (and why things will then repeat themselves). Let me thus try to analyze the set of different approaches one can reasonably take when theorizing about existence. First of all, we have to agree that there must be some fundamental logic to it all, even if that logic includes more than we humans can understand or not. There must be some fundamental workings to it all, even if those workings are embodied as a conscious being, i.e.\ a god of all existence, or not. Let us think about these fundamental workings as a language, and in case that our mortal reasoning would not be able to understand this language, we can just say that it is a language of god(s). Now, the set of all possible multiverses (which, again, I use as a synonym for an ontology of all existence) 
%is then expressible in this fundamental language. Note that we do not need to assume this language to be unique but just expressive enough to encompass the fundamental workings of all existence. This set of all possible multiverses then form a theory. We can think of this as a mathematical theory but again, we do not so far have to assume anything about whether this theory can expressed in any earthly language. By definition, this theory will then state all there can be, including of course what can be in terms of conscious thought and so on. The questions we have to analyze when theorizing about the multiverse is then what kind of multiverses, and in particular what kind of universes, this theory holds, and also what kind of mechanism, if any, there is to select the actual multiverse. Note that we are speaking about a hypothetical selection process which is not part of this theory of all possible multiverses. So either the actual (``selected'') multiverse is one with perfect symmetry in relation to have the set of all possible universes is expressed in a/the fundamental language of all existence, or we need to introduce an asymmetry in the form of a asymmetric selector to choose the multiverse. This selector will then either select the total set of universes at once or selects one after each other, at random or by some order, but whatever the case, the total set of universes brought into existence will be asymmetric in relation to the fundamental language (by definition of an asymmetric selector). 
%
%I think that there are effectively two options for such a selector. Either it can be a god which is powerful enough to act as a choice function on the set of all possible multiverses, or equivalently\footnote{Equivalently at least in terms of how we can think about it.} as a series of choice functions on the set of all possible universes, or we can have just some fundamental mechanism which has the power to act as such a choice function. One could then argue that this process is just random for all intends and purposes or one could argue that it is guided by some higher principle. Especially in the case of a god, one could make the case that god selects the multiverse along some principle such as `goodness' or some other profound quality. One could also argue this for the mathematical mechanism but I think very few people would do so. I imagine that people arguing for a mathematical (asymmetric) choice function would most likely entertain the idea of a random process rather than a function guided by some profound principle. But in any case, note that this `principle' will then still be expressible in the fundamental language of everything. This means that a god adhering to some principle when selecting the multiverse can still understand his own workings despite holding the power of making asymmetric choices (in relation the/a fundamental language of everything). Oh, and there is by the way another possibility, which is almost similar to the hypothesis of an asymmetrical god, and that is where there is not just a single being\footnote{Oh, and I am also disregarding the possibility for having several gods since this is essentially the same as having one (but perhaps more conflicted) god to the extend that affect us mortals.} that sort of looks down on the multiverse, but where the conscious beings inhabiting the multiverse are the selectors themselves such that their souls unconsciously were the ones to select the universe they inhabit, and where they possibly also help to selects the events of that universe unconsciously. In this way of thinking, one can see our souls as godly entities or one can see our souls as being part of god himself. I should mention that one can also see our souls as one entity all together, without making them part of a higher consciousness, i.e.\ one could for instance say that god simply consists of all our souls put together. However, while I think that these possibilities are worth mentioning, they still give the same picture in terms of theorizing about what the multiverse holds. *Hm, okay it can actually make a difference; one can take a more individualistic (solipsistic) view on existence. 
%
%We have now discussed what types of reasonable theories one can give for a asymmetrical multiverse,\footnote{Hm, is `omniverse' a word? Would this not be more on the point?} at least the types that I can come up with. The reason why I want to focus on these kinds of multiverses is that I believe that the multiverse as some perfect symmetry according to a fundamental language of existence, which for sure, in my opinion, means that there are an infinite number of universes and where everything therefore also repeats itself like a have deed. But even if we look at the other possible contenders where the multiverse is assumed to selected by an asymmetric process, there are only a few reasonable examples where this is not true, i.e.\ where the set of all conscious experiences in the multiverse (for all time) will not be infinite. 
%
%The most reasonable of these are probably one where a godly entity or force is selecting the universes of the multiverse by some profound principle and where this principle somehow includes a preference for a finite number of collections of continuous conscious experiences, i.e.\ lives. It could for instance be argued that this god only might be interested in some sample of all lives possible and not be interested in initiating almost identical lives (i.e.\ making it so that the lives are experienced by some conscious entity, namely a soul, whether or not this souls is a separate being or somehow part of god himself). This theory will certainly be in conflict with the hypothesis of an infinite set of lives being lived in the total existence. It will, however, not be a very discomforting thought generally for the people choosing to believe this. When assuming that there is a god of all existence who is only interested in a sample of all possible lives, most people would be inclined to then believe that this god is good and knowing what he is doing; that he will not throw souls away needlessly and will not condemn them to a unhappy and/or boring afterlife, but will be forgiving and also would let people relive the happy parts of their mortal lives and/or have them otherwise entertained and/or give them bliss. I know that the idea of hell and eternal torment is a popular belief presently but no one really believe such a fate for themselves. And if they do, they only have to think a little bit about what an eternity actually is, on how absurd the thought of eternal torment really is. And if they still believe such a thing even after that, then they are probably just trying to torment themselves (and maybe fairly so, so why even give to much thought about this possibility?).
%
%Another reasonable eventuality is where our souls are unconsciously choosing our reality and where there is no higher being involved and thus where you cannot trust that there is a good plan with everything. People subscribing to this point of view would probably not have any reason to believe that these souls would not go on to live other lives after each live lived; since boredom is a mortal quality, there is no reason to believe that an immortal soul would get tired of living nw lives. And for the few people that would believe this, for them the thought of one's soul retiring from living new lives would then not be a negative thought anyway. And in the case where we assume souls to only live one life, I do not think that anyone would really believe the number of souls to be finite then...
%
%Well, maybe some people would not find the thought of an infinite number of souls choosing to live various lives very comforting at all, since the thought of a soul is normally associated with strict borders between consciousnesses. I do not tend to believe this myself, I am pretty convinced that these borders between conscious thought is only something we perceive ourselves, and if an infinite number of lives is lived, then it does not matter whether it is your current soul that lives them or other ones, but then again, I do not really believe in souls in the first place... Anyway, I am sure that most people subscribing to (or open to) the idea of souls would then see them as having borders to each other. Hm, what I am trying to say is that for those people, statements such as ``I hope to be a rich person in my next life'' makes sense, which is not the case for someone with my way of thinking, where I see all conscious thought in the entire existence as effectively one thing. When I just wrote about there being several souls in existence, I was still then just thinking about souls as being what can be thought of as record player needles for our common consciousness. Sp while I am open to the notion of an order to the lives lived, and so would not necessarily object to the notion of having a next life, I see it as irrelevant what that next life is, since you would effectively (by being part of the common consciousness in existence) be living both. Maybe a better example is that it does not really make sense for me, if someone hopes wishes to only live very happy lives from this point on and that someone else, i.e.\ another soul, would have to live the rest for you. For all intends and purposed, if there are several lives lived after yours (where `after' does not have to mean in in-universe chronological order), the you can see it as if you live all these lives.
%
%Anyway, back to my last thought. Even when assuming that there are individual souls that are not part of a greater whole that only lives one life, people who would believe this would most likely still believe souls to be persistent enough to live on after life and having some sort of afterlife, either one where memories are relived or one where there is a more blissful place that you go to, or a combination of these. An interesting thought is about the possibility then that the soul chooses the afterlife itself (since we are already assuming that it has the power to chose the mortal life), which could then mean that the soul is sort of set free after the mortal life and can then choose to its own afterlife that fits that souls needs and wants. Even though this might sound a bit far-fetched and made-up at first, I think that is actually a quite natural extension of this set of assumptions when we think about it.
%
%Now, there is one problem with this kind of theory of existence. So far I have mostly just discussed the theories in terms of how reasonable they are at a first glance and how comforting they are to people fearing a non-existent afterlife. The is, however, also the matter of what set of ethics, they point toward. I like that my understanding of the multiverse means that everything, in a very literal sense, comes back around (as in the saying ``what goes around comes around), both good and bad. And I actually imagine that some people will have their views of existence changed by hearing the arguments for such a hypothesis. And furthermore, I believe that this belief that what goes around in some sense literally comes around will help some people approach their lives in a less individualistic way. I thus think that some people might start to value the thought of other people's happiness a bit more (after having convinced themselves rationally that said happiness will come around to them in a sense).
%
%For theories that assume that souls are choosing their own lives to live, one could argue that egoistic actions are okay even though they are very damaging to other if one is then convinced that souls will just choose the better lives more often. However, this can only really be believable for people who have only had good fortune in their lives, because souls by these assumptions would not choose anything less. And since well-functioning societies will not rely on fortunate individuals to not act individualistic but will have law structures in place to prevent such acts, the fact that a few individuals are persuaded by an individualistic philosophy should not really be a problem. And these individuals would also have to close their eyes to other possibilities in order for them to be able to excuse egoistic behavior. One thing is whether you tend to believe such a hypothesis, which is somewhat reasonable, and another thing completely is be 100 \% convicted of this hypothesis, which is not. It is also a bit circular, which makes it a weird thing to actively believe in, since any self-sacrificing action, as well as any unfortunate event for the individual, would then point against this hypothesis, and a fortunate person who believe the hypothesis to be true would then effectively subscribe to a fatalistic point of view where they cannot be self-sacrificing and where the cannot turn their own fortune, even if they actively tried. Very few people would believe such a thing if the really think about it.
%
%There are also another type of existence theory which eliminates the need for good (altruistic) ethics and that is where a profound principle is assumed to be guiding/deciding what lives come into existence and where this principle is one that effectively states that there is a balance of good and evil, of happiness and unhappiness,
%in all of existence. This would effectively mean that each good action is balances by a bad action and vice versa. Even though the good action does not cause the bad action, it would still effectively do so in the grand scheme of things. More precisely, any set of people who has it in them to be good towards each other would be balanced by another set of people who counterbalances this, including sets of people who believe this hypothesis and where it then effects their set of ethics and/or their morality. So for these people, they would then have no rational reason to decide to be good or bad from a true altruistic point of view. But even so, they will still have self-serving reasons to try to make efforts to be the good society and not the bad counterbalance to it. And once again, it would be unreasonable to believe this hypothesis with 100 \% conviction, which would then mean that even societies who tends to believe the hypothesis in question would still have plenty of reason to try to forward good actions and to make it a happy society. Apart from these argument, I also find it a weird thing to believe. I cannot imagine that we would actually find any evidence that our own universe is balanced this way, especially not in the future where we will have had plenty of time to analyze the good-versus-harm balance and where living standards will probably be even better, hopefully even a lot better, than they are now. If this is true, then one would indeed have to believe in shadow universes where all the goodness in our universe is undone. I find this a very weird thing to be convicted about with no evidence. I find it first of all unreasonable and I also find it unlikely that people living in an overall good universe would have the need to believe in such shadow universes, at least unless we find good rational arguments for it, which I also do not imagine will happen. So all in all there is no need to worry about this hypothesis either in terms of ethics. 
%
%So as far as I can see, there is no need to worry about either the ethics or the comfort of people when assuming a multiverse where godly entities are deciding what comes into existence. With all that out of the way, we can go on to discuss a much more important subject of what a more mathematical existence theory will yield in terms of the ability to believe in effective afterlives or reincarnations. The reason that I see it as a more important question is that I am sure most people being discomforted by the fear for the afterlife are ones that fear for a non-existent afterlife and for whom this possibility is correlated with there being no godly entity. *(Sure there might still be some people concerned about going to some hell, but as I mentioned, I would imagine that these are very few and I do therefore not really consider this an important topic.) %
%
%But if we do not have some profound immortal soul, and if consciousness is related directly to the material movements of matter in our brains, then if someone in the future would have almost an identical live as ours, or would have an identical life up to some point and then will start to diverge, then those persons can be seen as us for all intends and purposes. If consciousness is just caused by movements in a brain, then if there will be almost exactly the same movements in the future, the resulting consciousness will be almost exactly the same as ours. In a way this consciousness will in fact be more us, than our own future selves since we obviously change as persons through our lives. 
%%*(I guess one could be worried about us having a mortal soul without there being any meaning...)
%
%So far so good, but why would this conclusion only apply in this forward temporal direction. Why should we only be comforted if things repeat themselves in the future of this universe? What about the future of other universes. If there are an infinite number of universes for instance, there are sure to be someone exactly (or almost exactly) like us at some later time than the current one. Now, if there are nothing profound about our ``soul,'' i.e.\ our conscious being, then borders between universes should surely not matter. And if even if we think of all universes as happening at the same time with some global concept of time that applies to all of existence, then their will still be someone in the global future like you. Even if we assume all universes to have the same kind of sudden birth and assume all universes to be somewhat like ours, if they either are infinitely many or have infinite space to them, there will always be somewhere where a populated world is either surviving by incredible luck or is birthed into existence by incredible rare circumstances and quantum fluctuations. In fact there will be an infinite number of such places at all times. This is true even for ever-expanding universes that gets colder and colder. So even with a global time to the multiverse, where universes are all birthed at the same time and where all universes are somewhat like ours, if there are in total an infinite amount of space for the set of universes where life like us can exist, there is an infinite number of yous in the global future of this multiverse. 
%
%There is an interesting problem with this view, however, which is worth mentioning now. So far we have only really discussed different theories in an a priori way, but we can also say something about existence theories based on empirical observations. If we thus have a set of theories which makes sense to us a priori but only a few of these fits our experiences, then we have to only accept the ones that fit our observations. For familiar, we have to apply Bayes' theorem essentially. If a theory gives an infinitesimal likelihood for us to have a life that fulfills some characteristics that we observe, i.e.\ where there should be an infinite larger proportion of lives with conflicting characteristics, then we have to discard that theory in favor of theories that fit the observations. In this instance, where we have assumed a view where there is a global time to all universes and where we are less than 14 billion years into the global time (as we are also assuming the all parts of the universes to die a heat death / cold death\footnote{Here, I am taking `heat' to denote the quantity and `cold' to denote an adjective about temperature. A ``cold death'' is thus equivalent to a ``low-pressure death.''} at some point), this would be infinitely unlikely by all accounts unless we assume our life to be ``selected'' and turned into existence at the same ``time'' as every other life. But if this is the case, one can no longer argue that at any point, what happens in the `future' is more important than what happens in the `past.' I know that this is a complicated argument, but I am sure I can describe it more concisely so that it makes sense. In short it does not make sense to fret about being part of a multiverse where all part of the universes die at some point, not just because of the previously mentioned fact that no true death of all of the multiverse will happen despite this as long as it is infinite in space, but also because we in this case would be able to discard the temporal-centric view where time is more important than space in how the ``selection process'' is happening. 
%Hm, I guess I should have talked more about why we need a ``selection process'' for infinite universes. It comes from the fact that there has to be a prior underlying likelihood for what we experience next, since otherwise we can never argue anything about probabilities which contradicts what we know to be true: we can rely on probabilities in our own life. We can for instance rely on the fact that jumping down from a high enough building would break our legs more likely than cure an already broken leg. (We could find a million examples but this will probably do.) We know this to be true so it does not do to have a theory where there is no meaningful prior to our collection of experiences making up our lives. So a theory has to describe in what order our lives are selected in a sense, even for infinite universes, or has to have some other way of assigning probabilities to lives.
%
%Well, in terms of the picture with an infinite number of young-but-at-some-point-dying universes, I guess a view where conscious experiences are non-discrete and thus more fluid could explain such a multiverse, I guess. I have not mentioned such a possible view yet, but it is where the set of all lives in the multiverse is collected into groups where conscious experiences can then differ in some sort of ``strength'' such that conscious experiences that are more frequent in the multiverse will not be lived several times but will instead be lived once with a ``strength'' according to how frequent those kind of experiences are.
%
%Well, never mind this hypothesis, actually... (No, mind it a bit, actually, but not much.) It is worth discussing when trying to come up with ways how conscious can couple to a system of moving objects and to in-universe quantum mechanical many-world worlds, but it is not really that interesting to discuss in length otherwise. If we assume such a hypothesis, then everything can either happen at once or can happen according to a global time such that only lives happening at the same time are grouped together this way. In the first picture, one then just has to make sense on what ``everything happens at once'' entails and how, if in any way, it is different from everything happening in some order. In the second picture, there will indeed be a problem with even an infinite number of dying universes, since the `strength' of conscious experiences will die out somehow. But who can makes sense of such a picture anyway, and why would anyone prefer it? And since arguments for this picture and arguments for a very restricted set of universes in all of existence, super restricted in fact, are uncorrelated, one has to be really pessimistic to assume them both.
%
%I had a Christmas break before resuming this text, just between the previous paragraph and the one prior to that, and I came up with something that I should mention here just before leaving. In the picture where there are individual, distinguishable souls in all of existence, why would these souls only stick around for one life necessarily? Why could there not be some kind of souls that stick around for several lives? If a soul couples to something material and can come back even after the brain has essentially been dead (or even disassembled and put back!), why should it not stick around after death searching for something to couple to again, probably something that ``feels'' like the old brain is much as possible? Well what should be the expiration date on these souls? If there is none, which would almost be the only reasonable answer, but where they might still have some inner clock before choosing the next target, well then we just have the reincarnation picture, which most people would be perfectly comfortable with (I am talking about comfortable knowing, not necessarily comfortable accepting (for people with strong beliefs already)). I find it weird to believe in a godly kind of soul where each person has a unique one of that and then it knows when to die at the right time but does not have an afterlife but will simply fade away from existence, while still even retaining its uniqueness for all time and all of existence. If we think of all the potential ways to revive, duplicate and combine brains (perhaps in the future), then this picture even makes less sense (whereas a picture where a disjoint set of souls continuously ``feeling'' their way across the material universe and coupling to brains (mechanical or biological) in order to get their experiences can still make sense with these potential eventualities in mind).
%
%Additionally, if you assume that we have such individual, distinguishable souls where each soul die at some point, bit still believe there can be several kinds of laws regarding such souls, if you then subscribe to a view where there is a symmetry of all things such that all types of souls will then exist, well then there will still be a virtual 100 \% chance that you are one of the undying, or virtually never dying, souls. 
%
%So in short, If we assume that there is a godly being, well then we are in his hands and most people will trust such a being to be good to them. If there is not such a being, but we have immortal souls, well then it is only extremely pessimistic theories, where either the multiverse is assumed to be finite in time and space or where the souls are assumed to strangely be very mortal, where your soul will not live forever. If we assume a materialistic view on consciousness where there are no intermediate supernatural soul between your brain and your conscious thoughts and experiences, then only the same very restricted multiverses will result in a conclusion where you cannot for all intends and purposes take a correct view where you are essentially reincarnated forever. 
%
%By the way, since ...
%
%
%%%Points about mathematical symmetry. Points about simple universes having far greater frequency than complicated ones. Points about MUH vs CUH. Points about brains of evolution vs. synthetic (mathemtically) brains, and how such brains would likely just be doing math if they are pure, which means that we can be thought up by a sort of god. A view of god thinking about everything in general. Points about how I think conscious experiences to not happen in point, but more between points (a frosen person does not have the same point of experience all the while he is frosen). Points on the universe having a center, such that the concept of SR is used to rearrange time, which means what we see at the edge of the universe would indeed be the edge if we happened to be at the center and means that brand new galaxies are continuously born. Points about a many-world universe; how there has to be a selection process always, and how I think that consciousness-branching is probably not very likely for us, even though there is not anything wrong with that picture (it is just overall more complicated than to just select multiple full paths, as far as I can see). Hm, is things calculated or is calculation-time irrelevant..? It is not always irrelevant, not for all universes... Hm, hard to say... Well, moving on. Points about `besjæling.' Points brian-to-experience-coupling: How feeling could be experienced differently in different universes, and e.g.\ colors could be experienced differently. Points about reviving, recombining, copying and seperating brains. Points about how symmetric universe groups are more likely than single asymmetric universes. Points about how spirit is shared, and how the most happy moments are the most generally shared experiences. And how we start out as more simple and more alike (and often end so as well). Points that in a quantum mechanical universe, somewhere there will always be a version of you that is branched away from a previous you. Oh and of course points about having consciousness as the base entity of existance instead of having physical objects. Points about the problems with the other view, for instance sand dune brains and so on. How the first view does not shance anything otherwise; it does not mean that we can say that things are just thought up by us, just because there existence is dependent on our consciousness. Thought and objects alike are still equaly deterministic, so object can therefore not be said to be affected by our thoughts in any way. 
%
%
%
%
%
%%This actually also shows that there is issue for any reasoning that results in effective reincarnations but says that only future versions of yourself will count in this incarnation view. It is not necessarily a wrong view, we cannot conclude that at this point.
%
%
%% The standard god might not be a god of all existence. For instance we could easily imagine god to go to some god hell, if he really is a great tormenter, needlessly tormenting souls for an eternity, virtual or not. 
%
%
%
%
%
%
%
%
%
%
%
%
%
%
%
%%\chapter*{User-driven and variable web (and Desktop applications)}
%%\chapter*{User-driven and variable predictive ontology}
%%\chapter*{A more open, dynamical and sustainable economy}
%%\chapter*{Blockchain ideas}
%%\chapter*{QED theory}
%%\chapter*{Arguments for the computable universe hypothesis (CUH) and how to solve the question of consciousness}
%%\chapter*{Energy and agriculture ideas}
%%\chapter*{Other thoughts}
%
%
%\end{document}
%
%
%
%
%
%
%
%
%
%%
%%[...]
%%\chapter*{A more open, dynamical and sustainable economy}
%%
%%\section*{Goals}
%%% Motivation:
%%% Åbenhed: Forestil dig en økonomi, hvor alle har adgang til et ret gemmengående indblik i firmaers arbejde og udviklingsprojekter, og hvor folk digitalt kan komme med forslag til ændringer og justeringer og bidrage med at løse arbajdsopgaver fra fjernt. 
%%% Dynamisk: Forestil dig en effektiv måde, hvorpå disse bidrag kan blive belønnet på en fair måde, så bidragsyderne ikke behøver at gøre sig nogen særlige overvejelser om, hvordan de skal sørge for at markedsføre deres bidrag, men hele i stedet bare kan gå åbent og ærligt til værks og stole på, at deres bidrag nok skal blive vurderet rimeligt. 
%%% \sout{Bæredygtig: Forestil dig endvidere en økonomi, hvor lønninger til arbejdere, kreative bidragsydere, initiativtagere, risikoholdere osv. ikke er forskudt så profit for risikoholdernes synspunkt er en særligt afgørende faktor i lønsatsforhandlingerne, og hvor der er et stort positivt feedback ift. at tilegne sig assets, fordi disse assets i sig selv kan være et trinbræt til at tilegne sig flere, og endvidere bare kan betale andre for at lave en stor del af arbejdet med at pleje formuen og foretningerne, og således kan tjene penge, alene på bagrund af at eje, og uden særlig stor risiko... Hm....}
%%% Bæredygtig: Forestil dig endvidere en økonomi, hvor lønninger til arbejdere, kreative bidragsydere, initiativtagere, risikoholdere osv. ikke er forskudt så det favoriserer en funktion frem for en anden, eksempelvis ved at favorisere risikoholdernes profit, fordi de har mere magt i forhandlingerne, men hvor alle er lønnet på en måde som optimerer brugernes/forbrugernes gavn af firmaet, inklusiv de brugere, der gavner ved at opfylde en funktion i det. Forestil dig derfor, at der er en meget lille grad af positivt feedback ift., hvad ens position er, så folk ikke bliver holdt unødigt nede af ikke at eje særligt meget, og at folk ikke bliver holdt unødigt oppe af at eje meget, men hvor der er større migration på tværs af funktioner og stillinger, så disse i højere grad bliver opfyldt af de bedst egnede, dog uden at ødelægge den kapitalistiske og liberalistiske tanke, om at individer kan stræbe efter og opnå løkke ved at være arbejdsomme og snilde, og hvor de så kan nyde frugterne af deres arbejde undervejs og bagefter. 
%%% Forestil dig ydermere, at denne form for foretningsdrivelse, hvad den end måtte indebære mere specifikt, udbreder sig til en meget stor del af samfundet, så den får indflydelse på hele økonomien, og hvordan vi driver virksomheder, således at bruger-/forbruger-skaren bliver så stor at alle forhold i samfundet, hvor disse virksomheder bærer et præg, kommer til at indgå i beslutningerne for virksomhederne, da disse som nævnt (indtil videre pr. antagelse) i optimal grad virker mod at optimere brugernes/forbrugernes gavn af virksomhederne, og fordi de som nævnt giver alle brugerne/forbrugerne et meget åbent indblik i virsomhedens gang og beslutningstagninger. Hermed kommer den nævnte bæredygtighed ikke bare til at handle om, hvordan virksomhedernes lønninger efterlader samfundet i en mere lige eller ulige økonomisk situation, men om bæredygighed på alle punkter, der kan have indflydelse på den omfattende (pr. antagelse) (for)brugerskare. 
%%% Overgang til beskrivelse af vision:
%%% Det var den indledende motivation. Nu kommer en mere dybtegående forklaring på, hvordan jeg forestiller mig sådanne virksomheder kunne være udformet, og hvordan samfundet som hele ville have gavn af dem. Jeg vil dog stadig lave nogle midlertidige antagelser for på den måde at holde detaljerne, ift. hvordan man realistisk kunne opnå sådanne virksomheder, adskilt fra, hvad jeg forestiller mig, man kan opnå med dem. Der kan nemlig godt være flere måder, hvorpå man kan få disse detajler opfyldt, og det er ikke sikkert, at mine løsninger er fyldestgørende i sig selv, og det kan således godt være, at disse må supleres af andre tænkere. 
%%% Vision:
%%%% \sout{En central del i idéen er, at jeg ser et kæmpe foretningspotentiale i, at forbrugere forener sig mere for at få mere indflydelse over firmaer bl.a. til at deltage i beslutninger, få større indsigt (kræve større gennemsigtighed) og til at handlen kommer til at favorisere kunderne mere (så at kunder kan forene sig om at være lidt hårdere alle sammen), når det er muligt.}
%%%% - Brugere får alligevel større og større indflydelse på produkterne.
%%%% - Det at køre firmaer bliver mere og mere formelt struktureret, så der ikke skal så meget arbejde til.
%%%% - Patenter og rettigheder udløber, og der kommer flere og flere open source-løsninger alligevel.
%%%% - Arbejde bliver alligevel mere og mere automatiseret, så at man bare skal eje robotter.
%%%% - Der er en slem og en god singularitet ift. robotter og selvbyggende frabrikker, hvor forskellen er, om det når at blive fælleseje, eller om det bliver en gigant-rigmand, der kommer til at eje det. 
%%%% - Hænger lidt sammen med første punkt her, men der kommer også til at blive en større og større handel med brugeres kreativitet og idéer og viden, især hvis vi får mere åbenhed, og selvfølgelig hvis vi får gode bruger-drevne ontologier og et godt bruger-drevet web i det hele taget (ikke mindst!).
%%%% - Når virksomheder er drevet af kapitalister, der vil forøge/pleje deres formue, så er beslutningerne der taget ikke altid fordelagtig for alle involveret, men kan være stærkt vægtet mod kapitalisternes bedste - og pga. den nemme handel med aktier, kan de kortsigtede beslutninger tit vægtes højere end de bedste langsigtede beslutninger.
%%%% - Uligheden denne form for kapitalisme danner, er ikke smart i sig selv. Det er bedre, at folk tjener til forbrug, og at der er skarpe limiters på, at bruge det man tjener til at tilegne sig ny kapital, som man kan tjene på, i hvert fald efter en vis grænse. Jeg kunne nævne et B&B-eksempel, jeg har tænkt på. Den nye form for kapitalisme vil ikke forsøge at gøre sådanne ting ulovlige, men vil bare sørge for at få dem frem i lyset, så forbrugere har mulighed for at reagere samlet, hvis selvstændige folk tilegner sig mere på denne måde, end hvad der er hensigtsmæssigt fra forbrugernes synspunkt. Fordi en sådan økonomi vil have mere åbenhed og automatisk mere afslappede IP/copyright/osv.-love, kan andre så altid kopiere forretningen, hvis den begynder at blive grådig. Dette drejer sig om selvstændige forretninger, men de fleste forretninger vil nok hellere indgå aftaler med forbrugerne, så de ikke selv skal tænke på, at de ikke bliver for grådige, men at der er formelle foranstaltninger, der sørger for, at den positive kapital-feedback ikke tager overhånd. Og lad mig præcisere, folk må hellere end gerne blive rigeligt belønnet for gode idéer og initiativer, i mange tilfælde måske endda mere, end de bliver i dag i vores nuværende samfund, men der er bare ingen grund til at disse belønninger så skal til at yngle af sig selv, og at folk bare kan samle mere og mere magt sammen på bagrund af de første tilegnelser, hvad end det er pga. held (hvad det også tit er!) eller snilde. Det er bedre, at folk får mulighed at stige, ikke i kapital, men i deres stilling. I den henseende kan man godt tillade et positivt magt-feedback i en forstand, ved at folk så kan stige inden for en branche. Folk skal også bestemt kunne opnå bevillinger til at kunne starte virksomheder op på deres helt egne præmisser, og her kan man også tale om et positivt feedback, fordi folk først skal vinde anseelse for at få disse bevillinger, som så igen kan bruges til at opnå mere anseelse, men dette er altså helt fint. Man kan i øvrigt også sagtens tillade folk at efterlade arvtagere formuer (eller dele af disse), det går også fint. Så skal disse penge igen bare være øremærket til forbrug (inklusiv f.eks. boligkøb/husleje) eller uddannelse og altså ikke til kapital, der kan give videre afkast. Kan også nævne, i forbindelse med at folk gerne må få mere ud af gode idéer i mange tilfælde, at der er et problem med alt-eller-intet situationer, når det kommer til opstart. Enten bliver opfindere underbetalt i forhold til den samlede indtjeneste af deres idé, eller også bliver de kapitalejere og bliver stinkende rige og får en kæmpe formue, der kan yngle. Her er det værd at nævne, at ingen har godt af sådanne "fuck-you money," hverken på et samfundsmæssigt eller et personligt plan; man mister nemlig empati, og man får i øvrigt ikke mere lykke (hvad alle dog er klar over og ignorerer...). *[Vigtigt faktisk også, at folk gerne må kunne opnå et punkt, hvor de kan hvile på laurebærkrænsen; jeg har nemlig måske nok få det til at lyde lidt modsat i denne paragraf. Folk må meget gerne kunne arbejde sig op til en højere levestandard, hvor de så ikke skal løbe som en gal (rød dronning (eller hvid dronning, som jeg husker TTLG)) for at bevare denne tilstand, men hvor de faktisk godt kan tage de med ro og føle sig sikker på at denne levestandard kan opretholdes. Jeg vil faktisk håbe, at fremtidens samfund tillader dette enddnu mere, da jeg generelt tror på, at folk er villige til at arbejde, og at det samlet set er bedre ikke at gøre folk afhængige af at skulle arbejde... Men pointen er stadigvæk, at dette så skal være en del af pakken, en del af belønningen, med det samme, og det skal ikke så være afhængigt af, om man lige for plejet sin formue korrekt (og heldigt!), og der er ingen grund til at folk skal spille hazard, og ingen idé i at nogle folk så vinder enddnu mere på bagrund af de tjente formuer. Der må selvfølgelig gerne stadig handles med aktier og investeres i et grundlæggende kapitalistisk samfund, men brugerforeninger kan bare sammenslutte sig om at gøre det mere favorabelt for aktiehavere, at indgå aftaler, hvor aktierne bliver varetaget af en gruppe, og hvor afkastet så er mere stabilt og moderat, og hvor aktiehandlen således varetages af eksperter og ikke af ejerne. He, man kunne næsten bare komme med idéen, at folk bare sætter pres på banker for at investere bæredygtigt og til fordel for den brede kundeskare, og så sætte pres på firmaer til at have aktierne stående i denne bank, men denne idé alene vil jo nok ikke være nok til at få folk med.] 
%%%% - Der findes mange funktioner i samfundet, hvor der tjenes gode penge på ikke særligt meget arbejde eller risiko. Forbrugere kan spare meget, hvis man kan skære flere mellemmænd fra, og hvis de bl.a. kan eje deres egne husstande.
%%%% - Man kan spare meget ved at have mere gennemsigtig og forbruger-bestilt reklame (sammen med fleksibilitet og samarbejde, så taber-producenterne hurtigt bare kan skifte til det bedre produkt eller den bedre model) i stedet for at bruge en masse energi og penge på reklame, der endda ofte har til formål at narre og manipulere forbrugeren. (Husk dog, at reklame der forsøger at skabe hype og/eller gøre noget fashionable ikke altid er 100 % dårligt; tit vil man gerne have sådanne tinge som forbruger.)
%%%% - Mindre forarbejde med at finde kapital og lave; bare find en skabelon for din opstart, sørg for at finde nogle folk, der kan sige god for dit projekt, hvis du har brug for forsikring og sørg for at overvåge din opstart in real time, så andre kan gribe ind, hvis du gør noget galt. 
%%%% - Grunden til at det er særligt værd at nævne alle disse fordele, også selvom de godt kan være en del ude i fremtiden, er pga. at man faktisk kan lave en bevægelse, der tager disse fremtidsudsigter og laver dem om til nutidspenge! (Se bare bitcoin som et godt eksempel der peger i retning af dette princip. Bitcoin har nemlig en hel del at gøre med dette princip, og det viser bestemt at der kan være mange penge i det.) ...
%%%% - På kort sigt har vi dog også brugerdrevne apps og ontologier og web osv.
%%%% - Man behøver ingen gang at give retningslinjer til foreningsmedlemmer, kun analyser. Og endnu mere vigtigt: man behøver ikke at give belønninger! Alt dette kan ske helt decentralt via bidragstokens. Det er virkeligt vigtigt at pointere, hvor vigtigt det er ikke at blive grådig her. Det gode er dog, at dette også kan medregnes i bidragene og dermed i hele regnskabet. Man kan således sørge for, at alt for frembrusende og grådige beslutninger bliver nedvurderet pga. risiko for at ophidse det omkringliggende samfund og skabe rivaler, også selvom disse rivaler ikke ender med at få overmagt, så kan man stadig nedvurdere folks bidrag ved den risiko, de skabte, ved at skabe rivaler og modstandere. Det gode er så også, at der er en ret effektiv måde at undgå rivaler på, og det er ved at være så inklusiv som muligt og også gerne så forsonende som muligt, for så kan rivaler altid bare joine i stedet for at se sig sure på bevægelsen. Man vil få særligt gavn af at lade foreningen stå åben for alle, for så bliver det nemlig rigtigt svært at komme efter foreningen politisk, i modsætning til, hvis man giver mulighed for på et tidspunkt at kunne lukke foreningen og bringe de resterende ikke-medlemmer i en særligt dårlig situation, for i så fald bliver det nemt at skabe politisk modvilje mod foreningen. Men stadigvæk: Grådighed bør afstraffes tilbagevirkende, når dette vurderes at have skabt uhensigtsmæssig risiko, og dette kan så propagere tilbage i tiden via bidragstokens'ne. Uh, og dette gør det også mere hensigstmæssigt ikke at være grådig overfor ligesindede foreninger/tidligere foreninger. Det gør også at løfter, f.eks. til gamle open sourcerers, kan belønnes med det samme, fordi filk kan forudse den positive effekt i fremtiden.
%%%% - Åbenhed gør release-dage mindre vigtige, og gør at der altid er insentive til at udvikle ting efter release, og gør modding-muligheder til totalt almindelighed.
%%%% - Det er helt klart også værd at nævne globalisering som en mulig negativ side af den nuværende økonomi.
%%
%%% Semantik-kontrakter:
%%% Når folk indsætter penge, skal de helts også give til en pulje, hvor de på en måde investerer i den oprigtigheden i jura-bedømmelse-strukturen. Hm, og det må jeg gerne hænge sammen med, at folk også generelt gerne bare vil betale for mulig jura-oversigt på kontrakter, skulle det blive nødvendigt.. Ja, for så kan man bedre hænge det direkte op på oprigtighedsvurderingen. Så her er det jo altså bare smart med afkast-aktier i betaling til jurister. Yes, men kan man gøre andet? Ja, selvfølgelig vil der så også blive mange kontraktholdere, der har stake i at kontrakten kan efterses oprigtigt, og dette kan man helt sikkert bruge, men måske skal man lige tænke sig en anelse om først.. Det kan nemlig være svært at måle, om en person har stake her eller ej, eller om der ligefrem findes et underliggende modsat stake. På den anden side er det vist ikke nok, hvis man bliver nødt til at betale jurister en hel masse bare for at kunne holde et afkast oppe, som ikke tjener andet formål end stake... Hvis vi tænker i blockchain, og det kan vel ikke skade at gøre, så er det vigtigt at se på, hvad der holder den oppe. Dette er først og fremmest jura-lønninger. Det kan også være det system man har bygget via kontrakterne. Hvis hele foreningen er bygget ovenpå, så er der en hel del stake. Og her skal man så tænke på stake-pyramiden.. 
%%
%%
%%\chapter*{Blockchain ideas}
%%[...]
%%%Okay, jeg har udviklet idéen. Nu har jeg faktisk en idé til en ny form for kæde! Det handler i bund og grund om, at lave en blokkæde, hvori der er en matematisk model, og hvor alle smart-kontrakter så skal udformes som matematiske udsagn, der benytter aksiomerne fra den pågældende model. I denne model skal der så være en representation af selve kæden, som altså kan indgå som variabel i kontrakt-sætningerne, og derudover skal modellen også indeholde nogle af de ting, som jeg har snakket om, nemlig udsagn om hvad brugere er, og hvad deres computationelle kræfter kan regnes for at være (særligt i begyndelsesperioden bare), samt et genereliseret internet og antagelser og værdi (og arbejde self.). Således bør man starte med en model der siger nogle brugbare ting omkring verden, som man så kan bygge videre på, og derudover bør modellen så også indeholde et naturligt sprog, som beskrevet, så man kan sige ting som "denne engelske tekst vil som tolket i år 2020 være sand ifølge et flertal af adspurgte med kendskab til pågældende sprog, og som taler ærligt." Der skal så fra start af være defineret nogle ting omkring kædens udbyggelse. Heriblandt skal der så udformes en protokol for, hvordan der jævnligt kan stemmes sætninger igennem, som man så kan stemme om, skal antages eller ej. Det skal gerne være sådan i starten, at det er brugere med meget stake i kæden, hvis stemme tæller mest. De brugere, der stemte imod kan så hver især vælge, om de vil godkende antagelsen og gå med på hoved-forken, eller om de vil gå sammen med eventuelle andre nej-sigere, der også gerne vil starte en ny fork, og så gøre dette. Brugernes valutta bliver så delt, og så skal det så være sådan, at brugere frit kan gå over til den anden kæde (af de to) og tage sine penge med sig til denne, dog ved først at betale en bøde til destinationskæden som helhed (som kan betales ved automatisk at ændre kursen mellem de to kæder en anelse). Størrelsen af denne bøde skal så starte ret lavt og stige langsomt over tid, sådan at brugere på den mindst populære med tiden vil blive tvunget til at hoppe over på den populære kæde eller deltage i et hard fork, hvor der ikke er nogen garranti for at få sine penge med sig ved kollaps, og hvor penge kursen alt andet end lige vil starte på den originale kurs (minus start-bødeværdien, hvis ikke denne er 0 (hvad den nok helst bør være)) (så bliver altså nemt at finde start-kurserne på de forkede kæder, og der kommer ikke lige pludselig anseelig mere eller anseelig mindre værdi på den samlede kæde, idet et fork sker). Protokollerne for alt dette er også bare beskrevet i modellen, og det bør så ikke være antaget fra start, at denne protokol varer ved for al tid. I stedet bør dette være åbent, sådan at opdateringer af protokollerne kan ske via gennemstemte antagelser i modellen. Modellen er selvfølgelig en "prediktiv ontologi," som jeg har kaldt det og stadig kalder det i mine noter. Herved kan man så danne en blok-kæde, hvor protokollerne kan opdateres undervejs, uden at danne nogen hard forks nødvendigvis, og selv hvis en opdatering resulterer i et fork, så skaber dette fork ikke nogen pludselige værdiændringer. 
%%% Der skal selvfølgelig også være antagelser fra starten om, hvordan kæden må opdateres og genrelt hvikle antagelser må gøres, sådan at kæden overholder nogle genrelle udsagn, så at værdifordelingen mellem brugere er bevaret ift., og at man ikke forhindre nogen brugere i at kunne lave transaktioner osv. under en opdatering.  
%%% Da modellen er prediktiv får mange sætninger, eller rettere propositioner, tilknyttet en sandsynlighed til sig, som kan ændres som tiden går. De basale konrakter skal altså ikke bare kunne eksekveres, når en proposition kendes 100 %, men skal også kunne køres selvom de beror på propositioner, som ikke er afgjort 100 %. Der skal således kunne sættes en sandsynlighedsgrænse, når man danner kontrakten (samt også en udløbstid), så at kontrakten kan afsluttes, hvis denne grænse nås. Det skal så også være muligt at danne kontrakter, hvor resultaterne kan ankes i en vis forstand, således at de skal kunne udstede mønter, hvis værdi afhænger af kontrakt-propositionerne. Hvis en forsmået kontraktholder så tror på, at lykken nok skal vende stadigvæk, og kontrakten er af denne type, så kan denne holde fast på sin propositionsafhængige token-mønt og således håbe på at denne stiger i værdi i fremtiden. Hvis pågældende bruger på den anden side mister håbet på et tidspunkt, kan kontrakten fuldbyrdes helt ved at brugeren sælger (eller giver) sin mønt til indehaveren af den modsatte mønt, således at de kan samles og de originale kæde-mønter kan låses fri i stedet. Der kan også være en udløbsdato på kontrakter, så man kan definere, hvad der skal ske, hvis kontrakten ikke opfyldes efer en vis tid. Selvom diverse specifikkationer for, hvordan disse kontraktgrænser kan sættes godt kan være åbne, så skal det altså være en af de grundlæggende antagelser i modellen, hvordan at sådanne grænser skal eksistere og kunne sættes af brugerne. Det er nemlig vigtigt at smart-kontrakterne er defineret i det helt grundlæggende, selvom at det godt må være en genereliseret version af sådanne kontrakter, der er defineret, således at kæden stadig er åben for opdatering. 
%%%Der skal så også fra start af defineres diverse belønninger for bl.a. at mine blokke og for at vise sætninger til modellen. I sidstnævnte kan man også måske definere noget med at belønningen for at definere sætninger, der bruges meget skal være større end sætninger, der ikke kommer så meget i brug. Man kunne endda undgå at brugere rotter sig sammen om at bruge særlige versioner og samme sætninger, hvis man samler sætninger med alle deres korrolarer. Man kan sikkert også gøre andre ting, så som at sikre at et bevis ikke gentages unødvendigt, når man bare kan bruge en anden brugers eksisterende bevis eller de eksisterende sætninger.. (Ja, jeg ved, hvad jeg mener.) Der må også gerne være belønninger for at komme med udsagn, som stemmes igennem som noget der måske skal gøres som antagelse. Derudover bør blokkæden, da der er tale om en prediktiv ontologi, også give belønninger til folk, der kan finde udsagn, som folk er uenige om, og selvfølgelig i det hele taget også bare for at tilføje data, som kan bruges til at vægte under-modeller, i Bayes-undermodeller.. Man bør faktisk også udstede dusører til at lokke folk til at indgå vædemål om interessante spørgsmål. Ligesom så meget andet, så skal mange af disse ting foregå ved at man jævnligt stemmer nogle forslag igennem, og afsætter nogle penge til at få folk til at forholde sig aktivt til de igennemstemte forslag, og self. til at belønne de brugere, der aktivt kommer med brugbare forslag.. Nu bliver det lidt rodet, så skal lige have redet nogle ting ud her, til når jeg skal skrive noget mere formelt. 
%%
%%
%%%% Det med at undgå at nye brugere tager kæden som gidsel, og om at undgå oportunistisk adfærd generelt, men uden at man behøver at gøre kæden meget mindre fri overordnet set, da man bare kan implementere det som nogle defensive kontrakter, hvor folk på en måde lover ikke at have sådan adfærd (når de subscriber til kæden og til de pågældende stabiliserende antagelser).
%%%% Det med hvordan man kan bruge gennemsnitsarbejdstimer som en god værdimåleenhed på en så smart kæde, sammen med computerarbejde self., og hvordan de første brugere godt kan købe sig selv rigdom ved et pyramide-system, så længe de ikke bliver for grådige.
%%%% Det med at Stake-havende brugere godt kan forhandle sig til at blive optaget som startbrugere i en forening, der følger min "foretningsidé" (som jeg beskriver ovenfor), imod at denne forening kan blive optaget i modellen. Dette kan endda gøres ved at lave fremtidsantagelser, hvis man tør være så fræk. (Men det er heldigvis ikke meget at bede om; sådanne foreninger skal jo gerne være ok gavmilde med bidragstokens til folk, der på en måde har bidraget for stiftelsen af foreningen.)
%%%% Jeg skal have uddybet mere om, hvilke nogle antagelser man kan tage, og særligt hvor bløde de kan være; de kan jo komme med Bayes parametre.
%%%% Nævn belønninger for at udbygge modellen og for at finde udsagn, som skal afgøres osv.. (Delvist tjek.)
%%%% Hvis jeg ikke har nævnt det, kom med en (indsæt pyramide-tegning) forklaring på, hvad værdien i kæden kommer til at ligge i.
%%%% Og jeg skal selvfølgelig forklare, hvorfor at stake i kæden gør, at man vil bevæge sig mod sande modeller.
%%
%%
%%%Okay, nu har jeg lige (det er d. 24/11 2020 i dag btw) gjort nogle nye tanker (og de sidste par dage), der faktisk klarer nogle ting op angående denne blockchain idé. For det første har det været ret vigtigt at åbne op for, at kæden kan benytte sig af mine forretningidé-principper, så der potentielt kan dannes mere hype om den, og så der også er mere at vinde ved at vedligeholde en sandfærdig og lødig model. Pointerne fra disse tanker er... Nej, lad mig bare forklare om kæden ifølge de nye tanker, i stedet for at prøve at finde den præcise difference fra før. Kæden skal starte ret simpelt med et format og en protokol (..He. En proto-protokol.. Thihi.), der forbinder en matematisk model med smart-kontrakter og giver de første restriktioner på, hvordan denne model kan udvikle sig (selvfølgelig med matematiske aksiomer inkluderet) og særligt, hvordan kæder skal forke. Jeg ved ikke, om det er strengt nødvendigt, men det er nok en god idé uanset hvad, hvis modellen også inkluderer udsagn om, hvordan penge skal bevares på kæden og over eventuelle forks. Modellen skal selvfølgelig også indeholde koblinger til selve kæden, så man kan referere til kædehistorien med kontrakterne. Nu handler det så om for engagerede brugere, faktisk at skabe det der svarer lidt til domsmandsforeningerne, ved at gå sammen om at arbejde på en fork af kæden, hvor de spiller en mere central rolle og således påtager sig ansvaret for, nok via en veto-ret, tænker jeg, at styre modellen i den rigtige retning, så der ikke sker ulødige antagelser, og så den udbygges mere og mere, og så smart-kontrakterne kan blive mere og mere automatiserede. Disse brugere kan så tage sig betalt for deres besvær, ikke fordi andre ikke kan kopiere deres modeller, men de kan tage sig betalt for at agere domsmænd plus de kan anmode om, at de mere generelle brugere (og jeg skal lige nævne, at forskellen bør være lidt et spørgsmål om stake: altså strukturen bør være relativt decentral, idet der bør være en høj mobilitet (for sådan tiltrækker man bedst andre brugere)) giver deres input til kæden krypteret, således at høj-stake-brugerne kan gå sammen om at varetage denne data. Denne data, hvis brugerne har samtykket til, at det kan offentliggøres, bør så selvfølgelig kun være temporarily closed source; altså kan lukket som en måde at hjælpe centerbrugerne med at forhindre andre kæder i at kopiere kæden, hvilket alle brugere på sådan en kæde under udvikling er interesserede i. Det langt mest sikre for en kæde er så at bevæge sig mod rigtige og lødige modeller, der ikke forsøger at snyde tidligere kontrakter (og da høj-stake brugere kan bevare en del magt i lang tid, vil det være meget usandsynligt, at sådanne schemes bliver stemt igennem), og der er meget at vinde, hvis man bliver den kæde, som folk søger mod (for det første pga. smarte smart-kontrakter og for det andet pga., at kæden sandsynligvis, hvis altså den udkonkurrerer de andre, så bliver udgangspunktet for en decentraliseret forretningsidé-forretning). Denne version af idéen har altså en mere liberal tilgang til, hvordan kæder med gode modeller i kan dannes. Kæde-ontologierne behøver ingen gang at blive ret avancerede, før at de bliver brugbare. Man kan nemlig hurtigt definere Work osv. (og data-lagringsservices) og ved at danne en slags "banker" på kæden, bliver det nemt at holde styr på kurser. Man kan i øvrigt også gauge folks formuer, både på kæden og deres samlede likvide formue generelt. Værdien på kæden kommer så først og fremmest ved de services, der kommer på kæden, og de værdipapirer, der bliver løst til tokens på kæden, og derudover, vil der så blive en fælles stake i, at kæden ikke går under, og dette kan høj-stake-brugerne så tage sig betalt for, ved at have sat nogle penge til side til sig selv på kæden, som så får lov at vokse i værdi, tilsvarende med at kæden vokser, og der kommer mere og mere værdi på kæden. Samtidigt bliver der stake i, at modellen fungerer, og der bliver så en masse værdi i den brugerdata, der hjælper denne (prædiktive) model med at fungere. Når høj-stake-brugerne så ejer en god del af denne data (dog med løfter/kontrakter om at udgive det med tiden), så bringer dette altså mere stabilitet til kæden (fordi den ikke så let kan kopieres og genskabes), og dataen bliver således også selv en form for kapital. Brugere, der bedrager til at holde denne data, og holde den hemmelig og/eller holde rettighederne til den, og sørger for at den alligevel kan bruges i queries, kan dermed også tage sig betalt i form af penge, der kan gro i takt med kæden. Så allerede her har vi en vildt god blockchain idé, også selv hvis kæden bare mest kommer til at handle om udveksling af kryptovalutta og digitale services. Men der kan faktisk blive endnu mere stake på kæden, hvis man også fra start formulerer, hvordan man gerne vil starte en forrestningsidé-forretning over kæden, således at kæde-starterne kan ses som start-bidragere til den decentraliserede forretning. Dette handler så bare om at man definerer, hvad det vil sige at bidrage til forretningen, og definerer tokens som så skal kunne indløses i fremtiden (med en eller anden frist eller så videre) med en værdi ud fra, hvor godt bidraget regnes for at være (gerne bare målt i kroner og øre; hvor meget ville forretningen betale for et sådant bidrag, hvis det kunne gå tilbage i tiden, dog med en limiter på, der forhindrer tårnhøje belønninger). Grupper kan så gå sammen om at forsikre brugere, der gør, hvad man tænker er gode bidrag pt., men som kan vise sig ikke at bringe noget. Som individuel bruger vil det være smart at melde sig på sådanne forsikringsgrupper, fordi brugere der er gode til at tage risiko og hæfte får skaberne helt klart vil udgøre et stort bidrag i forretningen, og hvis de sørger for at melde sig til store foreninger med nogle smarte hoveder, så er der helt sikkert mere at vinde end at tabe ved at melde sig som forsikrer. At betale løn til bidragere imod at påtage sig en del af risikoen deres bidragstokens (ved altså at købe andele af disse) vil selvfølgelig også udgøre vigtige bidrag i forretningen, det er klart. Det er også klart, at det er super vigtigt for sådan en forretning med en ren og lødig model til at afgøre bidragene, så derfor bliver der meget værdi i som høj-stake-brugere af kæden, at kunne danne en meget tillidsværdig og ren model i deres kæde. 
%%%Og det der gør, at man godt kan starte denne forretning på en god kæde, er, at det er en del af hele idéen, at folk kan stole på belønninger, så det duer bare ikke at snyde de første bidragere. Selvfølgelig kan man forhandle sig til, at start-bidragere bliver mindre grådige, men dette kan også sagtens bare gøres ved at brugere strækker fra forretningen, og at grådigheden så dermed kommer til at medføre et negativt bidrag, som så kan trækkes fra i den lange bane. Men alle mennesker, der bidrager imod denne idé, skal også kunne belønnes, og skal ikke snydes af, at der ikke lige var den tilstrækkelige lovsikkerhed til stede på deres tidspunkt. For der vil også altid eksistere huller i lovsikkerheden, så det duer bare ikke, at forretningen først bliver uren, og det vil brugere også være enige i: Der vil bare ikke blive den samme tilslutning til en uren forretning, slet ikke. I øvrigt værdsætter folk generelt sikkerhed og stabilitet langt højere end kortsigtede udsigter til mere velstand, så folk vil hellere end gerne slutte sig til et system, der fungerer, endda selv hvis dette system ender ud til den lidt grådige side (hvis der altså ikke er så mange konkurrerende kæder fra starten af), især da dette med tiden vil hele, hvad der jo er en vigtig del af idéen.
%%%Jeg er ikke kommet så meget ind på endnu, hvordan overkæden skal håndtere forks, men det kan jeg tids nok gøre. Skal vist også også sige mere om betting osv.. (24.11.20)
%%
%%
%%
%%
%%%% Husk blok-rotation.
%%%% Værdi relateret til Work kan bestemmes med tidslåste bruger-nøgler. Værdi relateret til værdipapirer kan også hurtigt bestemmes, når det system kommer på benene... Ja, men sidstnævnte er jo ikke nødvendigvis så langtidsholdbart... Men det gør vel ikke noget, når man kan bestemme værdi via Work..
%%
%%
%%
%%\chapter*{QED theory}
%%
%%
%%
%%\chapter*{Arguments for the computable universe hypothesis (CUH) and how to solve the question of consciousness}
%%
%%
%%
%%\chapter*{Energy and agriculture ideas}
%%
%%
%%\chapter*{Other thoughts}
%%
%%

\part{(old)Appendices}
\chapter{Appendices}
\section{The notes from main\_12.txt that I have mentioned\label{main_12}}
%\section*{Disposition of draft:}
%
%
%\section*{-- Introduction}
%%\subsection*{-- Overview of the idea}
%%%The deductive environment (quick overview)
%%%And mathematical programming paradigm (quick overview)
%%	\subsection*{-- A bit about the evolution of the idea}
%\subsection*{-- Overview of the idea:}
%A self-updatable (or easy-updatable) F-IDE/P-IDE that formalizes the kernel assumptions themselves, where the kernel is then a set of assumptions about assumptions about subsets of a language and an intermediary language and about relevant compilers, but where all assumptions potentially can be temporary and are not assumed 100 \% true. Using Gödel number lifting rules together with these language and hardware assumptions it can rely on a low-order deductive system to prove the correctness, or rather the value of a correctness parameter, of an embedded deductive system of any order. Furthermore, it is possible to extend the application thus such that it becomes easy to use for even learners of math at o low level. This will possibly attract a large userbase and the beauty is then that these users can peel away the layers in there own tempo in order to get to more advanced P-IDEs (or `deductive environments') that embeds the more simple ones. Whatever the size, the userbase should be connected as a community (or as several communities) and make use of each other to verify propositions that cannot yet or is yet to costly to verify by a formal (automatically verifiable) proof. This then leads to a F-IDE where both humans and machines takes part in verifications but where each unproved assumption is still noted formally and has the possibility for being proved even more rigorously in the future. %Hm, man kunne endda bruge noget tilsvarende det, jeg har tænkt for BC, hvor NL-sætninger kan tillægges sandhedsværdier, således at matematiske beviser også kan bestå (helt eller delvist) af NL! Jeg har forinden dette tænkt, at NL kunne være en del af kontrakter til tekniske metoder, som kan beskrives i F-IDEen, men det at man ligefrem kan bruge det i beviser, er en ny tanke.
%
%\section*{-- Points about ZFC and FOL}
%%\subsection*{-- My initial thoughts about making it intuitive}
%\subsection*{-- Why set theory is somewhat un-intuitive}
%\subsection*{-- Why this does not matter (with FOL)}
%%% And why FOL is sufficient.
%%\subsection*{-- Why C (choice) is fine as well}
%%%%It is actually fine to make the axioms more flexible as well.
%
%\section*{-- Using Gödel number lifting rules}
%\subsection*{-- The rules themselves}
%\subsection*{-- How to use it to freely make inner models (and how to translate certain operation scripts with certain operations into permissible actions resulting in computer operations (by proving an invariant))}
%%\subsection*{-- How to extend this further with assumptions of program execution on workspace formulas}
%
%
%\newpage
%\section*{-- Introduction to the nice deductive environment overall and about embedded extensions}
%% Or \subsection! (or in another place)
%\subsection*{-- Not just about writing lemmas/conjectures and expecting an automatic proof}
%\subsection*{-- A workspace of propositions}
%\subsection*{-- Nice rules that follows intuition and are not limiting compared to paper proof manipulations/actions}
%\subsection*{-- Proof programs}
%\subsection*{-- How extensions can actually be proved at a certain point and be made as DLL extensions (such that the environment can update itself)}
%%Files should be discussed here.
%
%%\newpage
%\section*{-- A mathematical programming paradigm}
%% I should mention that I do not know how powerful and easy to work with theorem provers are it this point, but I am sure that there is really no upper limit to how effective it can be, if enough people start programming mathematically.
%%Uh, har vist (måske) endnu ikke skrevet det punkt med at folk kan generalisere løsninger for hinanden..
%
%\newpage
%\section*{-- How I imagine the deductive environment}
%\subsection*{-- Few-sentence recap of workspace, theorem storage and proof program}
%
%%Skal jeg nævne noget om ordningen af theorem databasen og "arbejdssætninger," og at man kan loade dem i packages? Ja. Måske her, måske i introductionen.
%\subsection*{-- Custom rules introduction (what are they and what should they be able to do overall)}
%\subsubsection*{-- Some good examples of rules}
%\subsubsection*{-- Some of the important logical rules that should be added at an early point}
%
%\subsection*{-- The type tree, operations and formula structure}
%\subsubsection*{-- Constants and operations and their types}
%\subsubsection*{-- Tree structure of formulas with (with operations, lambda terms and tuples)}
%%Mention rule to extract lambda term and that lamdas could have several parameters (that discussion..).
%\subsubsection*{-- Some of the possible sorts of types}
%\subsubsection*{-- Higher-order operations}
%\subsubsection*{-- Template operations and template types}
%\subsubsection*{-- Insertion using the type tree}
%\subsubsection*{-- How rules can belong to types and to specific operations}
%\subsubsection*{-- How the type tree can be translated and thus how it can be proved rather easily}
%
%\subsection*{-- More on rules}
%\subsubsection*{-- Rules as propositions and with automation}
%
%\subsubsection*{-- How high-light cursors can be used for rules}
%
%\subsubsection*{-- Colors (in more advanced rules)}
%
%\subsubsection*{-- HasColor() and other meta-predicates (HasType(), ...)}
%
%\section*{-- Teaching apps, technical ontologies and mathematical IDEs}
%
%%
%%\section*{-- Introduction}
%%	\subsection*{-- Overview of the idea}
%%	%%The deductive environment (quick overview)
%%	%%And mathematical programming paradigm (quick overview)
%%%	\subsection*{-- A bit about the evolution of the idea}
%%
%%
%%
%%\section*{-- Points about ZFC and FOL}
%%	\subsection*{-- My initial thoughts about making it intuitive}
%%	\subsection*{-- Why set theory is somewhat un-intuitive}
%%	\subsection*{-- Why this does not matter (with FOL)}
%%	%% And why FOL is sufficient.
%%	\subsection*{-- Why C (choice) is fine as well}
%%
%%\section*{-- Using the self-referential axiom (if this is not a proposition of ZFC)}
%%	\subsection*{-- The axiom itself}
%%	\subsection*{-- How to use it to freely make inner models}
%%	\subsection*{-- How to extend this further with assumptions of program execution on workspace formulas}
%%
%%
%%\newpage
%%\section*{-- Introduction to the nice deductive environment overall and about embedded extensions}
%%% Or \subsection! (or in another place)
%%	\subsection*{-- Not just about writing lemmas/conjectures and expecting an automatic proof}
%%	\subsection*{-- A workspace of propositions}
%%	\subsection*{-- Nice rules that follows intuition and are not limiting compared to paper proof manipulations/actions}
%%	\subsection*{-- Proof programs}
%%	\subsection*{-- How extensions can actually be proved at a certain point and be made as DLL extensions (such that the environment can update itself)}
%%	%Files should be discussed here.
%%
%%%\newpage
%%\section*{-- A mathematical programming paradigm}
%%% I should mention that I do not know how powerful and easy to work with theorem provers are it this point, but I am sure that there is really no upper limit to how effective it can be, if enough people start programming mathematically.
%%%Uh, har vist (måske) endnu ikke skrevet det punkt med at folk kan generalisere løsninger for hinanden..
%%
%%\newpage
%%\section*{-- How I imagine the deductive environment}
%%	\subsection*{-- Few-sentence recap of workspace, theorem storage and proof program}
%%
%%%Skal jeg nævne noget om ordningen af theorem databasen og "arbejdssætninger," og at man kan loade dem i packages? Ja. Måske her, måske i introductionen.
%%	\subsection*{-- Custom rules introduction (what are they and what should they be able to do overall)}
%%	\subsubsection*{-- Some good examples of rules}
%%	\subsubsection*{-- Some of the important logical rules that should be added at an early point}
%%
%%	\subsection*{-- The type tree, operations and formula structure}
%%		\subsubsection*{-- Constants and operations and their types}
%%		\subsubsection*{-- Tree structure of formulas with (with operations, lambda terms and tuples)}
%%		%Mention rule to extract lambda term and that lamdas could have several parameters (that discussion..).
%%		\subsubsection*{-- Some of the possible sorts of types}
%%		\subsubsection*{-- Higher-order operations}
%%		\subsubsection*{-- Template operations and template types}
%%		\subsubsection*{-- Insertion using the type tree}
%%		\subsubsection*{-- How rules can belong to types and to specific operations}
%%		\subsubsection*{-- How the type tree can be translated and thus how it can be proved rather easily}
%%
%%	\subsection*{-- More on rules}
%%		\subsubsection*{-- Rules as propositions and with automation}
%%		
%%		\subsubsection*{-- How high-light cursors can be used for rules}
%%	
%%		\subsubsection*{-- Colors (in more advanced rules)}
%%		
%%		\subsubsection*{-- HasColor() and other meta-predicates (HasType(), ...)}
%%
%%\section*{-- Teaching apps, technical ontologies and mathematical IDEs}
%
%
%
%
%
%
%
%
%%\newpage
%%\section*{Introduction}
%%The first idea I will introduce in these notes is an idea of mine for a type of application with which to create and edit mathematical proofs and to build a library theorems that can be used for subsequent proofs. My idea is thus essentially an idea for a proof assistant, but one of the key parts that, to my knowledge\footnote{And I must admit ...}
%%
%%%I will also present some thoughts on how 
%%%Part of the idea deals with how 
%%%There are two parts to the idea. One is how to make a nice environment for % %mathematicians, as well as learners all other 
%%%people working with mathematics in which to prove propositions. The other part of the idea is about how to use this environment for programming as well.
%%
%%In terms of how to create a nice environment for working with mathematics, the underlying thought can be motivated as follows. 
%%
%%
%%%It is almost like a conventional proof assistant\footnote{To the extend that I am aware...} in some regards but more focused  
%%%the focus is not on automatic theorem proving as much as it is about
%%%in what I will refer to as a `deductive environment' in these notes. 
%%
%%
%
%
%
%
%
%
%
%
%
%
%
%%
%%\newpage
%%\section*{Second pre-draft notes}
%%I have just recently been researching the current technologies for formal specification and verification a bit. It has turned out that most of my ideas and many of my visions are already held by (many) others. So this is therefore a second body of pre-draft notes meant for correcting the first one. 
%%
%%Hm, but it seems that it is still somewhat of a new idea to basically prove via Gödel numbers, or Gödel objects rather. So with this idea together with some of the other few ideas and visions, this idea is still worth describing.
%%
%%
%%
%%
%%
%%
%%
%%
%%
%%
%
%
%
%
%
%
%
%\section*{}
%\newpage
%\section*{First pre-draft notes}
\vspace{.5em}
\Large
\textbf{First pre-draft notes}\\
\normalsize
\vspace{-.5em}

\noindent
%Mindre personlig motivation men:
For a long time, I have worked with the idea of designing an application where users who are working with mathematics, both as learners and as professionals, could work with proving mathematical theorems in a nice and efficient environment.

Actually, %idea was initially just about having 
the concept I first worked on was basically to have an open database of theorems (or propositions) where users could upload new propositions if they could be proven. The idea was then that would not be much reviewing necessary in that database as there should be space enough to just accept whatever can be proven. The point was that propositions are then just queried for by users, which is why there is no need for reviewing and accepting propositions (beyond having copyright bot-crawlers (or whatever they are called) I guess). Unused propositions can then be marked for removal when space needs to be cleared. This idea is thus basically equivalent, I know now, to having theorem provers which can access an open database of propositions. 

Anyway, the idea then became focused on designing a nice and efficient environment as mentioned. For a long time I wanted to get the deductive system (the underlying logic) just right so that users could choose their own axioms but still get high-level functionality to define operators and to define new rules, even with automated subprocedures. The solution I landed on, before I realized why it was better just use ZFC, was to have some requirements, some meta-propositions basically, that operators had to fulfill in order to get the high-level functionality. This was basically just to have the axioms as a layer under the deductive environment that you really wanted to work with, and then prove that you are allowed to have all the functionality that you want given those axioms as the first thing. If a user for instance wanted to work with ZFC axioms and wanted to initialize these to get high-level functionality, he/she would have to define predicates such as IsNumber(x) (which in ZFC means to define what representation of the numbers should be the fundamental one), IsGreaterThan(x, y) and so on, as well as AreEqual(x, y) (used for all objects) and IsTheSum(x, y, z) and so on. By doing this, and by proving that these fulfill the second-layer meta-propositions, the user can thus unlock high-level functionality to prove things automatically about numbers and also for instance to substitute terms directly when they have been proven to be equal. 

To have such an interface layer with meta-propositions is not needed, however, when you realize how axiom sets such as ZFC can be interpreted (or reinterpreted) basically as anything you want. It does not make sense to make an interface layer since ZFC can already be your interface layer. Even though set theory is very un-intuitive at its core, as you for instance have something as simple and intuitive as numbers represented as weird containers with no structure or internal order, which only contains other containers of... nothing! Just abstract containers of basically nothing, how is that all of a sudden e.g.\ 4?! But the point is, as I have now finally realized, that you can still reinterpret these containers and, more specifically, reinterpret what the predicate Contains(x, y) mean, even so that it gets a completely different interpretation than what it means in the standard model (or other close-to-standard models) of ZFC. You can thus take any intuitive theory that you have constructed from a set of atomic predicates (using first-order logic), which might have the most generalizable and agreeable standard interpretation, and then you can still just find some structure within that theory that fulfills ZFC, where Contains() are just reformulated in terms of the other set of predicates, and then you can export all propositions of ZFC into that other theory. If that theory is efficient enough, which does not take a lot, then all the propositions about the structure can easily be transformed into the desired propositions that fits the standard model (the standard interpretation) of said theory. We do not even need to use second-order logic for this as the process is in theory just to insert another variable formula instead of the Contain() predicates, which is to simple a process for it to be part of the deductive system itself, especially since will not actually have to do all this in practice. The only real reason to work with other axiom sets than ZFC, is if one is concerned about the purity of the theory, but as we have just argued, ZFC is as pure as any theory which has the same power, since there is nothing in first-order logic that prevents us from just reinterpreting the predicates.

Some people might argue that the C part of ZFC is a potential source of impurity of the theory, which might make ZFC less acceptable by all users than say ZF (minus C) with perhaps the potential to add C (maybe as a footnote to the theorems that use it). This is certainly true in some way, but only until we get all mathematicians who has not done so yet to consider Gödels constructible universe (and other constructible models such as this). We only need to realize that we can take non-constructable models such as the standard model of ZFC and divide all objects up into to classes, constructible objects and non-constructible objects, and see that we can never form any proposition that speaks about the non-constructable objects in particular (by definition), which is as if they do not exist in the model. If we then take any non-constructable model and just remove non-constructable obejects from the base class, this will not change a single proposition. Furthermore, given that mathematics is about constructing formel systems for arriving at propositions which can then be applied to the real world, and since removing all non-constructible objects from a model can never reduce its applicability to describe anything in the real world, even rather abstract things such as manifolds of GR and so on, which is true since we can never observe anything non-constructable, or even observe any indication that there exists non-constructable objects in a theory applicable to our real-world universe (or any consistant universe, we can think about), we then see that such constructible model must be exactly as ``good'' as the original non-constructible one. We can certainly think about universes with `oracles' that can observe a difference, but any observer that is itself constructible (or at least is not able to tell whether he or she is constructible, I guess...) is not able to observe any difference. It is by the way also easy to see how the ``paradoxes'' that might be commonly attributed to the axiom of choice such as the Banach-Tarski paradox is really trivial and un-paradoxical in a constructible model. The Banach-Tarski paradox thus simply becomes a ``paradox'' that is almost equivalent to the rather trivial Hilbert's hotel paradox (I believe it is called so, but it is the one with the infinite number of rooms). %*So by including C... Hm. Does adding C to a constructible universe not only make ``perfect sense,'' is we do not use the choice function for further constructions --- well we obviously cannot, but what then. It seems there would be a tier list of choice functions.. But that should be simple enough, right?... Yes. There is a trivial order to the members of all sets once these are constructed, and there is then nothing stopping us from using this ordering to construct new sets.


I guess could talk more about why set theory is actually brilliant (despite my naïve opinion for a very long time), and I guess I also could argue against any other objection to first-order logic in general and against not having the axiom of the excluded middle and so on. But it will not really make sense for me to do it here. There is obviously a reason why ZFC is so well-liked (and why conventional first-order logic is so well-liked as well). %But I will not do so here in writing. I am not even sure that it would really be possible for me, as I cannot predict exactly where any disagreement would lie if I were to discuss with someone of another opinion than me about the possible fundamental systems of logic. 

So... Moving on! I wrote this whole introduction mostly for my own sake, and I do not think any of this will really be kept in the actual drafts. Only if I decide that the thing about constructible universes and the C in ZFC is worth discussing. From now on I will probable not speak so much about my whole thinking process as well but mainly about the end-result ideas themselves.

\subsubsection*{(self-referential axiom)}
My ideas for this environment are then pretty simple at this point. I have a few design ideas that are probably worth mentioning and then I have just one idea for the deductive system, which means that even though I have found this big appreciation for conventional ZFC, I still actually want to use an extension of it, which makes it well-suited for automatic proof and automation in proofs in general. The extension is not related to ZFC specifically but is a trick that could be applied to all theories. The idea is to go ahead and model the theory within the theory itself and then assume a new axiom (which can just be implemented as a rule in the deductive system, which would probably be the simplest way to do it *[No! Never mind, it should of course be an axiom, so that it for instance can be used to deduce the same for other representations and maybe more.]) which states that any proposition object of this inner model that can be proved to be ``provable'' in the inner model can be lifted up and translated into an actual proposition with exactly the same form and then be marked as a (true) proposition of the theory. In ZFC, this could be done by first defining a specific representation of the language of well-formed formulas as a set within ZFC and then defining a new set to hold all the provable formulas of ZFC only, only with the original axioms and rules and not any additional axioms or rules from the extension. When this is defined, you can then add the rule or axiom to lift any formula object from the set of ``true'' formula objects up, so that a equivalent proposition is added to your list of true propositions.\footnote{Do I know, whether an axiom like this can be proven in the un-extended ZFC? Probably not, right?} If the formula object is a conjunction of several stand-alone propositions, then there should probably be the option to add each of these propositions separately in one go instead of adding the full conjunction. It is of course important not to use any of the new axioms/rules of the extension in the inner model, as this would lead to a self-reference. 

This is the only extension I would add to the underlying deductive system itself, but then there should also be as many other quality-of-life extensions to the actual environment as one could want, like the ones I have mentioned such as automated addition and so on. In particular, there should then also be an extension to run certain programs which are defined as mathematical objects. This is simply done by defining the relevant representations as well as a ``this program halts with this output'' predicate (or rather a variable formula, since we are using first-order logic). It does not matter how many programming languages are added this way to the environment (or application if one will) but it is certainly beneficial to include a very low-level programming language, probably an assembly language even, to the environment/application so that such programs can be run very efficiently. This is especially important since this would also mean that compilers can be defined as mathematical objects themselves, which means that users get the full freedom to add new programming languages (without the need for further quality-of-life extensions by the programmers of the math-application itself). By having the underlying deductive system extended by that almost-self-referential axiom/rule as described, this then means that these programs also get the power to basically add new propositions to your proposition database / file system by outputting a formula that can be transported to said database / file system immediately (and this transport could even happen automatically itself if an extension is added for this). 

By having the built-in (lower-level) program(s) powerful enough to be able to run interactive programs with graphical interfaces and all, there is then no limit for how the users can basically extend the application.

This is of course true for any programming language but in this instance, it might be even more useful, since there is a difference between updating and making extensions for the application itself, which requires testing and peer-reviews, and then just proving some theorems about a certain program. This is one of the big possible benefits from a new mathematical (proof-based) programming language paradigm if it can work, namely the fact that one does not need to test and review updates. If we look at object-oriented programming, we already have a good way of implementing a program in a top-down way and divide it up into modules. These modules can then be documented for their intended function and consequently implemented and tested. But what if the documentation is instead written mathematically, so that the implementation can be proved to have the correct function? At a first glance, all this seem to do is to shift the work of programming the module to the documentation. But perhaps writing a documentation mathematically and writing a rigorous documentation normally might not be too different once the paradigm has developed enough. And when I think about it, I think it would often be easier to describe a module rigorously instead of programming it in the imperative. This might differ from task to task as well as from person to person, but this is why it is a good idea to have several programming paradigms in the first place. Furthermore, \iffalse This is a good way of putting it: Good with several paradigms, and furthermore...\fi it would be possible to compile some of these mathematical documentations to imperative programs, and once the paradigm gets developed, even quite abstract documentation might be compiled this way, at least partially. The IDE could then give suggestions for different design patterns (prepared by other users by having things proved about them) to pursue, where potential gaps to solve, i.e. when a part of the solution is dependent on outside factors that therefore needs a specially tailored proof by the user dependent on these outside factors, are left open. This is why even the implementation itself might get to be a more top-down process with such a mathematical programming paradigm. Imagine writing a rather abstract documentation of what a certain module needs to do and then just choosing between a couple of implementation patterns and then to have the IDE write out the whole template for the implementation as well as the new mathematical documentation for each submodule that needs further implementation if there are any (and where you might even be able to use the same procedure for some of these submodules). I think this can save a lot of work. And since this is all just mathematics, it is very easy to just import any new and useful implementation patterns as these are all just mathematical propositions (so no need for any rigorous peer-reviewing). And then once you are done with the implementation you are done! No need for further testing or for using external theorem provers. This fact alone could save a lot of work even at early stages of the paradigm where it might still be more work to make mathematical implementations rather than imperative ones. Furthermore, when you are done, the product also gets more value since other parties that depend on your program does not need to trust the correctness of your tests or your documentation as the can just run the proofs themselves. As an additional point, a mathematical paradigm might be easier to work with in big projects with a lot of different programmers where a lot of rigorous intercommunication is needed. Such project obviously need rigorous planning and documentation anyway, so why not make this mathematical in order to get rid of any ambiguity. Also, while it is possible to make mistakes in the documentation and for instance forget certain eventualities, a lot of the documentations of a big project will be documentation of submodules and the correctness of the more overall documentation of the program will thus be dependent on these. If it is made sure, that it is first proven that the overall (mathematical) documentation holds if the documentation of the submodules hold (in a top-down fashion), then there is never the risk that individual programmers work on a perceived task that is different from what was intended (or at least that risk is reduced I should say). 
Oh, and there is another point as well. In an open source community, having mathematical documentations might also greatly increase the value of such programs in that community, since it might be possible to then make it easier to search for a certain solution to a problem at hand. It might de easier for servers to serve the right solutions to the clients once everything is mathematically described, as the servers then might just need to do a little bit of math on their own order to find a good solution. A bit like Wolfram Alpha, I guess, but more rigorous and for programs (submodules and subroutines) as well. I will expand more on the possibilities I see for an open source community when programs have mathematical documentations in the next section/chapter. Oh, and yet another point is of course how it makes updating applications much more easy. With normal paradigms each update to a unit of a program has to be rigorously tested both individually, and should also preferably be tested in the whole context. But once a program has a mathematical backbone all the way through, each unit can then be updated as many times as desired and all that is needed to do before exporting is just to prove that it still fulfills the already defined documentation. So I would imagine that updating could be a much simpler task overall this way. 

So to sum some things up about the paradigm, it means that programmers are intended to use theorem proving rather than tests (but does not necessarily force them to do so, as documentations can be relaxed to conclude a probability of working as intended instead of a yes/no, and then one can make use of assumptions about have testing yields probabilities, which would also help make testing more rigorous in a way, but this is still not really what is intended for the paradigm) and to make these proofs in a top-down way. I know that with only these describing attributes, one could just choose to use this same approach with existing paradigm (for instance with the functional paradigm), but this would still not be exactly the same. The point is that in this mathematical paradigm, the theorem proving becomes part of the regular programming and people can then share lemmas which proves that a certain implementation design pattern yields a certain functionality given some restrictions of some variable submodules with each other and then these sort of lemmas can furthermore be used by IDEs to come up with possible solution routes for implementing a module and then print out any such chosen template so that it is ready to be completed. To make theorem proving the main part of the actual programming itself thus should open up to new way of programming (collaboratively) in my view and might thus constitute a valuable new programming paradigm. 


But back to the graphical interfaces and so on. This means that users can now expand the application itself using a mathematical programming paradigm. This would make sense to do, as it would also be nice to make a proof of the correctness of the main application, even if that proof checker is the main application itself; it might not be theoretically viable to test the correctness of a program with itself, but in practice it is okay since it would be very weird if the very source of incorrectness is what makes this particular proof flawed while still remaining undiscovered otherwise. Plus the proofs made can also be written out and checked by humans once they are made so any flaw will eventually be caught this way anyway. And since the task of extending the main application is one that requires rigorous mathematical deductions anyway, it would probably by a task that is very well-suited for a mathematical programming paradigm. This means that the main application does not necessarily need to be super advanced to begin with. At the same time, however, extending the mathematical deductive system/environment of the main application will not be wasted work either, especially if it is programmed in a language that is easy to reason about in a rigorous mathematical way, since then these implementations can just be transpiled and used again in the extension applications. I have yet to mention some of the features that I think is worth mentioning here about what could be useful in such a deductive environment, and have yet to explain how I imagine the main parts of such environment, but the point of this paragraph is then just that, while these features probably are important to develop quite soon in order to get a good environment, they are free to be developed as part of the main application or as a feature of an embedded extension given what suits the application programmers and the users best. Oh, and I should mention already that I imagine the main application to have read and write permission to dedicated directories, so that users can append files to their proofs when sharing these and thus be able to prove something about these files, which can be any data files including for instance proposition storage files or program files. Since the embedded applications also can have such read and write permissions there is no need for the main application to be active while an embedded application runs. This cement the fact that the main application probably does not need to fulfill many requirements in the long run. In fact, it might be beneficial to reduce it in complexity and remove functionality from it in the long run, if it is mostly used to start up more specialized environments anyway (in order to reduce start-up time). But again, all this should not deter programmers from making a high-level main application as this work will not, as mentioned, be wasted. 

Well, now that I have thought a bit more about it, I think it might be a smart thing to do to only implement a few quality-of-life extensions at the beginning and then from there already start to define the semantics of a subset of a language, probably an assembly language, in the mathematical theory and then make it so that programs of this language (subset) can be run on the workspace propositions (which can be just a conjunction tree of propositions) to transform them into new workspace propositions. These programs should of course also have the ability to read and write to files and thus to store their own propositions. The assumption that connects this extension to the underlying theory could just be an assumption that every proposition that is written to a certain file (with a certain format) can be imported as propositions to the workspace or the storage file of the underlying pre-programming extension. Then all new extensions from there can simply be proven mathematically in terms of how workspace terms can be transformed into other workspace terms an how files can be read/written to. This is except extensions that adds more semantic definitions to the language or add new languages to the set of languages that can be run (as well as update how they are compiled and run). One way to implement this programming extension could be via DLLs for the application that the application can change itself and then restart. Another option is to have a compiler connected to the application and then recompile the whole application (and restart), when such a user-update is done. I like the first option best, but I do not know if that is as machine independent as the second option. When making programs to extend the application, there can then be a set of actions the user can take on the workspace term and how these actions are made does not need to be defined further. When making the graphical interface for the embedded workspace application, all that is required in terms of input is that the actions can be mapped to I/O-inputs. There does not need to be any assumptions about the graphics either. Here you just give permission to the graphical renderer to read the current workspace and then there is no other restrictions on how this renderer can draw the workspace term. 

\subsubsection*{(How I imagine the environment)}
Okay, now on to how I imagine the deductive environment overall. %So we obviously have the deductive system itself which defines the underlying language of the propositions, at least in their most low-level form, as well as the basic deductive rules. As I have mentioned this should just be ZFC, just with an extra axiom which allow us to basically extend the deductive system within the system itself and thus to for instance run programs that simulate the original deductive system and then still be able to lift the propositions proved out and turn them directly into propositions of the main theory without having to repeat the proof (and not be able to cut the corners that we might potentially be able to cut in the inner extension). ZFC is cast in first-order logic, and this should also be true for our deductive system. At an early point it is probably a good idea to extend the basic logic quite a bit. First of all, it would be a good idea to be able to
First of all there should be a workspace of propositions and a database of propositions that can be loaded into this worksspace and where new propositions can be added when they are proved in the proposition workspace. So far so good. The way I imagine it, working with the propositions is not intended to be like working with a normal theorem prover, where you write some propositions and lemmas and expect the computer to do the rest. While it is certainly to write up beforehand what you intend to prove (which can be done by writing up tautologies with these propositions), I imagining the proving itself consists of taking sequential actions on the workspace propositions in order to transform them or to add new propositions to that workspace. Theorems can then be loaded from the proposition storage to make these mathematical actions. I imagine then that the user can arrange these theorems so that they can be loaded into the environment (and into the memory of the application) to have them directly at hand, but where they are not immediately loaded into the mix of propositions related to your proof (what I have called workspace propositions so far). Instead they should be loaded into a the environment as a ``rule'', i.e.\ a new mathematical action that can be taken on the propositions. Of course there should also be a more overall rule to make it possible to import an imported ``rule'' into the proposition workspace as well (so as to not having to search the theorem database again for the same theorem). Such rules could come in a large variety but to give some good examples, one rule could be to be able to change the order of terms in a sum and thus move terms around, and another rule could be to move a term to the other side of the equation, changing its sign. Last mentioned rule is a common mathematical action one can take when solving an equation, which would be acceptable in a hand-written proof for instance. The point is that all such rules that one can quickly take on paper should also be rules that can be included in this digital deductive environment. The rules should by no means be devoid of automation. For instance, another useful rule would be to cancel terms in a sum or to cancel factors in fractions, which should include automation to assert that the relevant terms are indeed equal (and with opposite signs if required). A learner of math should certainly be able to only use rules with a limited amount of automation in accordance with what is expected of a proof at the relevant level. Often the requirements of a proof is set by what is generally considered trivial among peers of the prover and the environment should certainly be able to demand the same requirements of the prover in order for him to progress (so that this user does not need to fill in gaps when he or she is done but can hand the proof in directly). But on the other hand there should be no restrictions either on the possible levels of automation so that experts can get exactly the amount of automation they want. This means that one should also be able to have the computer do even large computations in order to complete a mathematical action, i.e.\ a ``rule.'' 

I also imagine that a type tree would be a good idea. I call it a type tree because it in practice will include representations for predicates, propositions, quantifiers and also non-set classes but in theory, I imagine these all to just be sets that represent these types in an inner model of ZFC which models ZFC itself, or rather an extension of it. With the almost-self-referential axiom (or proposition if it can actually be proven from ZFC already, which I would not think is possible, but which I do not know for sure) such inner representations of extensions of the theory are just as valid to work with in practice as when working in the outer theory, since all the propositions proven in the inner model can be lifted out directly to the outer theory, without any additional meta-logic needed. I imagine then that the user should be able to define constants and operators (or `operations' if one will) that is added to the relevant type of the type tree. Operations (I actually prefer to use `operations' rather than `operators' in this context, but maybe that is just me) should be added to their output type. With the constants and operations one should also be able to add (user-defined) specifications for how these should be rendered. For both constants and operations, this includes what symbols should be used, and for operations in particular this will also include the typesetting specifications of how the operands should be placed when rendering the operation. Variables should also be able to be added to the tree so that there can be a predefined priority for which variables gets which symbols, so that the user does not have to decide on what symbol should be used each time a variable is added to a proposition in the workspace. I thus imagine the propositions to ultimately  be tree structures with operations at each non-leaf vertex. Well, except that I also then imagine list terms and lambda to be a parts of these tree structures. I guess it can be done several ways but the way I would construct the propositions of this environment, I would then also have tuple terms and lambda terms. Operations should then always have tuple terms as a singular child, which means the the operations are always unary in principle (similar to how functions are represented normally in set theory). But the render should then be free to ``explode'' the tuples and not render them how stand-alone tuples are otherwise rendered but should be free to separate each tuple element and place them wherever is desired. This ability should even be recursive, so that the operation rendering also gets control over element of any nested tuples these elements are also desired to be rendered in a specific way in the operation and not as a normal tuple. As mentioned, there should also be lambda-terms as a different kind of vertex, which probably specify the id of the lambda parameter, and probably also its type, (or perhaps several parameters at once, but maybe one should just use a tuple type then) and then has a term as a child, where the lambda parameter(s) is/are able to be used. This is important for the quantifiers to work and is also generally a nice thing to be able to work with. I thus imagine that the users might want to for instance be able to extract lambda terms from other terms, where some inner child terms are replaced for parameters, in order to insert them directly into new terms. Quantifiers should then be able to introduce variables of certain types (and should also always just be able to introduce a whole list of variables by basically just introducing a tuple (that can just be rendered as a comma-seperated list if wanted in the quantifier (but where the whole tuple itself is also still available in the child proposition as well, just as if the tuple where introduced as a variable, but where the tuple is then always rendered explicitly written out, unless the quantifier also explicitly associates a variable with the tuple))). When the user wants to substitute a variable with a composite term, the type tree can then also be used to insert that term sequentially by replacing it with either other constants or variables or with operations with new variables in them that are ready to be substituted further. Substituting this way should then not require any logical proof that the term is of the correct type as this is given from the procedure. This is perhaps unless the type includes a complicated restrictive predicate, which is meant to be checked automatically after insertion. In this case the application can then automatically check this predicate and either O.K.\ the insertion or cancel it. A user should also be able to extract terms (including lambda terms) from other terms in the workspace and insert these, and in this case, the type should also be checked automatically. This could be done many ways; it could be done by recording the types at the vertices or by asserting that the relevant term can be generated by insertion according to the type tree or perhaps by a mix of these approaches. 
I final point I want to mention for the type tree is that the rules added to the environment will have a type associated with them and some of them will have an operation associated with them. This means that when a vertex is selected a list could be printed with all the relevant rules (maybe structured in different sections given the type of rule) that can be taken on that vertex in the current environment with its current set of rules. So in a way, the rules can also be seen as belonging to the type tree if one will. 
Oh, and I should also mention that a type tree can basically be seen as a bunch of proposition in conjunction with the workspace propositions and, whenever constant, variable or operation is used, as antecedents to these propositions. It would make sense to then actually define a translation that pull the type tree into the proposition themselves (as conjugated propositions and antecedents). This would help translating back into previous (lower-level) extensions and would also be the natural way to prove the correctness of the type tree extension. 

Lastly, I should mention some other particular kind of rules that I imagine would be beneficial to the environment, before moving on in this pre-draft chapter (i.e.\ not an actual first draft but more of a loosely structured explanation that should be ready to be restructured into an actual first draft) to some more of the prospects for such a deductive environment application. Now that I have talked about how for instance the type tree actually hides some antecedents and turn them into more external definitions, it almost goes without saying, that there should be no need to insert into and restructure your theorems in order to tailor them to work on a specific proposition every time you need to make a prove a new proposition that uses a specific case of the theorem. I think that this should already be part of an early extension, that there is a rule to apply a theorem to a term in a proposition where all the additional antecedents and conjugated propositions, and all other irrelevant terms around the relevant term for that matter, are disregarded automatically, and where any variables in the theorem are also substituted automatically (and implicitly) in the theorem in order to make it fit. 
A similar thing that I also think is probably an important thing to be part of an early extension is a rule to substitute variables of inner quantifiers both with terms from the type tree and, especially before a type tree extension, with variables from quantifiers that further out in the formula and thus are parent vertices of the relevant inner quantifier. Dependent on how many negations and how many implies antecedents are gone into in order to get to the relevant quantifier, the quantifier should then be a universal or an existential quantifier accordingly (and if it is nested within some bicondition, it should of course not be possible to just substitute). 
Apart from this, there should also be all the nice logical rules that you can want for logical manipulations of the propositions including rules that automatically substitute variables to match other terms. There should thus for sure be a rule to automatically match the antecedent of a proved implies proposition with another proposition and insert in both to match each other automatically in order to prove the consequent, now possibly with substitutions in it. 

Let me by the way mention that it is probably a good rule of thumb that the default behavior should be to create new propositions whenever there are any such substitutions happening, and thus to keep the original ones in the workspace, but then on the other hand to transform a proposition in-place whenever it is just a reduction or a rewritten form of the previous proposition. A ``reduction'' could for instance be if a subproposition of a composite proposition (such as an implication, conjunction or disjunction) is proven to always be, say, true (or false) according to other proven propositions. In general the rule of thumb is to always create new propositions and add them to the workspace as the default behavior, whenever it would otherwise result lost information, that the user would have to prove anew in theory upon removal.\footnote{This can be explained in a shorter way.} 

\subsubsection*{(Automated rules as propositions)}
Another kind of rules that I have already talked about, but which needs a bit more explanation, are user-defined rules with automation in them. How exactly should the users program these rules when using the high-level deductive environment? Well, of course we could just have the programs, that are written in conventional programming languages and then compiled, possibly to DLLs, and then just have rules that does nothing but to execute such a program on a term. But I think we can do more than this. I think the users should be able to define advanced rules just by writing it as a theorem of a certain form. This form should just be either an equation or a biconditional, when the rule is reversible, or a conditional if the rule is not (and there should probably be a new proposition added to the workspace) wrapped in a number of antecedents (i.e.\ such that it is a nested consequent). When applying such a rule, the first thing to be checked is which side of connective or equality sign in question should be matched with the current term (and which one is then the resulting term). For conditionals, this is known beforehand, since it will depend in the number of outer negations and implication antecedents that the term is nested in. If a match is successful, is recorded which variables had to be substituted in order for the match to work as well as what the should be substituted for. This is analogous to binding variables to values in a program. Note that there might be variables that are free to be bound still after this match. Then each antecedents are ``executed'' in order. For this execution, the should be a list of automatic reductions which are what the rule will go through and try in order to reduce the relevant type of term. Note that when a subproposition of a connective is successfully reduced, the whole connective can be reduced as well. This leads to a means of branching the rule program, since it means that one for instance can make if statements with conditional antecedents. When an antecedent includes an existential quantifier around a defining equation for its bound variable, then this can be used as a variable definition (which binds a value to a new local variable). Similar quantifiers without the equation can be used as variable declarations and equations on this form but without any quantifier can be used to assign a value to a variable that is still free, potentially one coming from the consequent equation/(bi)conditional. Some of the automatic reductions apart from this could be things like doing basic arithmetic on numbers or to match to sides of an equation to see if it is true (which can then also be done term-wise with further potentially happening before each match is completed). If all the antecedents are successfully reduced to `true' (yes we can have a `true' constant in our inner model), and thus reduced away, the resulting rule of the consequent can be applied where all variables that were bound to a value is substituted accordingly and where all variables that are still free afterwards are kept so as well, i.e.\ kept as free variables. If any of the antecedent does not reduce automatically, the user should get the option to just get the proposition at its current form in order perhaps to try to see if the user can prove the antecedents true manually or to just keep the proposition if it turns out that there exists exceptions to the antecedent, in which case the resulting proposition might then still be a useful proposition. 

\subsubsection*{(Colors)}
Having such automated rules that can be defined as mathematical propositions will make it easy for users that are not programmers but still want to work with math to tailor their deductive environment themselves and get the automation they want still. It might also serve as a good avenue for such users to get more and more into programming in general. 
The rules could thus be programmed as a sort of high-level programs. I have also thought of a way to add more power to these programmed rules, which might be a good way to take this sort of rule-programming further to make it easier to make even more advanced rules without requiring the user to write conventional programs to do so. I thus imagine that the rules could use ``colors'' (or ``marks'') which are values that the formula vertices can hold which states something about the formula tree. In the implementation of the formula tree, one could for instance make space for a variable value to hold a ``color.'' An example of how a color could be used is if we want to reduce $\boldsymbol{true} \land p$ to $p$ inside a larger formula. This is only possible if there is and equal amount of negations, if we include implication antecedent as `negations' that is, when go from the root and in to the relevant formula vertex. As mentioned, I think this automatic rule should be part of an early extension and not something the users have to do themselves, but it still serves as a good example of what could be done with colors. The point is then to have a HasColor() meta-predicate as part of the inner model and in this case then use this to ask whether the relevant $\land$-vertex has a certain ``parent color,'' which we could call \texttt{outer\_negation\_parity}, equal to \texttt{even} (if we use enums \texttt{even} and \texttt{odd}). I call it a `parent color' since it is set by recursively asking the color of the next parent vertex until the formula root is reached. I think it would be a good idea to divide the colors into such categories of how they are set, i.e. what recursive scheme they follow. We should then also have `child colors,' `left-sibling colors,' `right-sibling colors' (recall that operations always take a tuple as input and can thus be viewed as a vertex with several children). These colors are in principle set recursively, but they do not need to be set by recursive function calls in the implementation, as one could just count the number of vertices before reaching an already-set color or a stop, and then count backwards to color all the vertices in between including the relevant one. When defining the HasColor()-property for a certain new color, one can then use additional HasColor() meta-predicates and thus the color procedures can be as advanced as is needed. We could thus even color our way through large mathematical operations including for instance sums, matrix multiplications, graph coloring and so on. When the definition of HasColor() for a certain new color does not have any circular dependencies and when all the operations for obtaining a color at a given vertex are either formulated as predicates (with automatic reductions attached to them) or other such non-circular HasColor meta-predicates, the correctness of the color procedure becomes easy to prove since this correctness can then easily be turned into a normal proposition which the user can then prove. 

Expanding on this way of programming via propositions, we could also have other meta-predicates like HasColor(), which outputs a true or false but which has a certain automatic procedure attached to them. We could thus also have meta-predicates to query the type tree (and perhaps even make changes to it as a sort of side-effect, if that makes sense) and, more importantly, to deal with files. Files should probably be included into the formulas by having certain special file vertices. A meta-predicate such as IsAFileTerm() would thus probably be good to have as well as predicates to get byte arrays from the file such as FileBytesAreEqualTo(number, file, new\_var) (where new\_var is supposed to be a fresh variables so that the predicate always successfully gets the byte array, at least if you return a structure where End-Of-File is handled as part of it). HasFileSize(file, new\_var) would probably also be nice to have and so on. One could even append to files by always keeping track of a files current size and then only ever basically ask for a ``subset'' of that file so that all parts of the rule procedure would be repeatable even if someone appended more to the file afterwards. (Hm, so HasFileSize() is actually redundant, at least for these types of files that are open for appending to them, if one wants several types of files this way, since a file is in principle seen as infinite in the model.) This means that a rule procedure can write to a file and then does not have to overwrite the same thing again but can instead just match the thing that is already written with what should have by written to the file. This then also means that users can hand each other the resulting files when handing the proof, since the successful execution of the proof will not be dependent on whether these files are included or not, but it might make the verification faster.
A HasType() is another useful meta-predicate, which uses the type tree in order to automatically decide whether a term has a certain type. Note that all type definitions of the type tree can also be written as predicates, so this is of course again how you prove the correctness of a given rule, i.e.\ by proving the proposition where the meta-predicates are transformed to regular predicates of the same extension (If it becomes required, one could also take a step back to an outer model, where the meta-predicates have their automation defined and then go from there in the proof, but the point is that this advanced stuff should not be necessary expect perhaps in special cases when proving the correctness of a very complicated rule (maybe when there is the potential for un-halting procedures)). 
There could also be other meta-predicates to identify vertex categories such as IsLambdaTerm() and IsTuple() etc.\ if needed.

Oh, and I have not yet talked about the proof programs themselves! Oops, I guess. It is of course a very essential thing to the deductive environment that the proofs made are stored and that users can then hand these proofs to other users, which can then verify the proofs themselves. This means that every mathematical action the user does should be recorded as a file. This file might be compiled a bit afterwards to reduce redundant actions, but this can also just be left as editing for the user itself. These proof programs should thus include all the rule calls and all the theorem and rule getting from storage and all rule definition and type tree definitions/manipulations and file manipulations and so on. It should even potentially include type setting specifications if the user wants to share his or her custom defined operation type-settings and symbol choices. 
Given that a proof program is build up as the proof is being constructed and edited by the user, the proof program can also serve as a way to help store previous states, since it as of course important that the user can undo (and redo) when constructing/editing a proof. Dependent on how it is implemented, the user can then get back to earlier stages either by having the rules reversed or by starting at a saved checkpoint state and redoing the actions from there, or by a mix of these options. The proof language is not constant but will of course change from extension to extension, as the set of mathematical action that I talked about in regards to proving the extension grows and/or changes. 

A couple of last things I should mention are as follows. 
I have not mentioned cursors yet. When applying a rule, I imagine that the user first moves some high-lighting cursors to relevant points in the formula tree in order to select the specific vertices that the rule needs to have selected in order to work. Given that rules should be able to be programmed as propositions, there should probably be a way single out some specific free variables which are meant to be the selected ones. One way to do this could be to have a HasCursor(new\_var, number) meta-predicate that can be included in the proposition to mark the relevant variable as a user-selected one. On of the important things the users should be able to do this way, is to mark with the cursors which elements of a tuple, including a operation child tuple, should be selected for a certain rule on that tuple or operation. 
%
I should also mention that the lambda terms can be used as a way to give functions to higher-order functions and then have the resulting operation rendered using the type-setting of the body of the lambda term. One could for instance give a lambda relation, such as $-10\leq i < 10$ for instance, to a summation function and have it rendered as $\sum_{-10\leq i < 10}\mathrm{lambda\_body}(i)$.
%
I should also mention more about the possible structure of the type trees. In this inner model we basically get the freedom of higher-order logic where we are free to always build new types on top of other types including higher-order predicates (which are not true predicates but are modeled by relations). We can also build quantifiers up to any order (not that it is necessarily meaningful to do so). So starting from the basic root terms `term' and `proposition' and being able to derive tuple types, function types and types restricted by a predicate, and also set types well as (i.e.\ a type that is basically supersets of the original type, when interpreted as a set (which it also actually is of course)), we can get any sort of type we like. The types that are restricted by a predicate will be subtypes of the original type. The HasType() predicate should of course be able to recognize a subtype as part of a supertype automatically. One could also add the possibility to assert types to be disjoint and/or for subtypes to have several (non-equal) parent types, but all this will be obvious things that the community can figure out along the way easily. I will mention here, however, the possibility to have template types, which are types that include parameters. These are important in order to be able to define template operations in a neat way, so that the type tree does not explode with type leafs and can remain constant in the environment, only changing when the user adds novel features to it. A template operation is different from a normal operation in that both its output type and its input type (or of course both) can be dependent on a parameter. This for instance means that we only have to define one type for matrices and for vector tuples for instance, if we want to use such, but where we still can use the type tree to make sure that the types are compatible for an operation (if we for instance want to type check an expression such as $\bar{\bar M} \times \bar{v}$ or, say, $d/dt\,(\bar{\bar M} \times \bar{v}(t))$ and so on). 
%
Speaking of the type tree, I could also mention as a last point, that some useful types might very well include standard computer types such as char, short, int, long, unsigned and signed, and perhaps single and double precision floats as well, all in order to use them in rules so that these can run faster than what would otherwise be possible. 


%% Teaching, though!



%% Top-down rather than bottom-up --- hope it works! (tjek)
%% Jeg kan allerede nævne design-principper som matematisk defineret her, og så kunne næste emne være mere semantisk programmering og open source applikationer. ...This is helped by people defining..., which should then also aid design a great deal. (rimeligt tjek)
%% Og nævn self. også lærings-apps som embedded apps. 
%% Jeg skal også lige tænke over, om regler også skal kunne tilføjes direkte via programmering... Det skal selvfølgelig (helt klart; det kan ikke være anderledes) sådan, at man kan lave regler i form af sætninger med antecedenter, der bliver afviklet automatisk, men er det ikke også en god idé, hvis brugerne kan udvide disse direkte (altså uden først at lave en ny embedded applikation)? Det er det vel, medmindre man nemt kan lave embedded applikationer, der får meget af den samme funktionalitet, hvilket virker lidt kludret, men... Hm, tror faktisk det giver mening uden at automatisere og generalisere, hvordan man kan tilføje regler... Hm.. Pointen er i hvert fald, at man sagtens kan lave embeddede applikationer, der kopierer den samme kildekode som main-applikationen, så selvom det bestemt ikke er spildt arbejde at lave extensions til denne, så kommer brugere nok i sidste ende til at arbejde mere i udvidende embedded applikationer. Ja, men så er pointen jo også bare lidt den samme her: Det ville være smart med en formel måde at tilføje nye regler til sit workspace, hvor der er rig mulighed for at definere selv advancerede regler i den (under-)applikation, man nu ender med at arbejde i. Man kan derfor fint tilføje det til main-applikationen, men hvis man ikke når at gøre det, er det også fint nok. Hm, men hvis man skulle gøre det i main-applikationen, så svarer det vel også til et nyt aksiom..? Nej, det gør det ikke. Man kan antage, at ens workspace er en conjunktion, hvad jeg jo også regner med bliver standardfortolkningen, og så handler det bare om at besive et program, der kan tage viste propositioner som input og bruge dette til at vise nye propositioner, hvorved man jo så altid kan vælge at discarde de gamle. Så jeg mener ikke, at dette bør regnes for et aksiom i sig selv, men kan altså også bare regnes for en quality-of-life (og 2.-lags-) extension. Ja, for på et grundlæggende plan kan man jo bare bruge sit program til, sammen med almost-self-reference-aksiomet, at udlede en proposition som er en implikationsproposition fra den ene workspace-konjunktion til den næste. Dermed indbefatter en sådan extension, der laver nye regler via programmer, altså bare at springe nogle trivielle omskrivninger over, hvilket er helt generelt for alle sådanne 2.-lags-(quality-of-life-)extensions. Ja, så det er virkeligt bare et spørgsmål om, hvad der lige passer, om man vil indføre en sådan extension i main-applikationen, eller man lader brugerne om at implementere det i en (embedded) brugerudvidelse. (tjek, på nær det med custom-reglerne)
%% Det med filer og designerede mapper, både for main-applikation og for embedded applikationer. (nogenlunde tjek)
%% Husk: self-reference axiom because it opens up for the extension programs. (bør nok understrege mere)




%Not using function symbols. Hm, but this must be the convention for ZFC anyway..



\end{document}
%
%

