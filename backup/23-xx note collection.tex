\documentclass{report}
\usepackage[utf8]{inputenc}


%\usepackage{amsmath}

%\usepackage[toc, page]{appendix}
%\usepackage[nottoc, numbib]{tocbibind}
\usepackage[nottoc]{tocbibind}
%\usepackage[bookmarks=true]{hyperref}
\usepackage[numbered]{bookmark}

\hypersetup{
	colorlinks	= true,
	urlcolor	= blue,
	linkcolor	= black,
	citecolor	= black
}

\usepackage{comment}

\usepackage{lipsum}


\title{
	Note collection (2023--20xx)
	\author{Mads J.\ Damgaard%
		%\footnote{
		%	See https://www.github.com/mjdamgaard/notes for potential updates, additional points, and other work.
		%}
		%\footnote{
		%	B.Sc.\ at the Niels Bohr Institute, University of Copenhagen.
		%	B.Sc.\ at the Department of Computer Science, University of Copenhagen.
		%	E-mail: fxn318@alumni.ku.dk.
		%	GitHub folder: https://www.github.com/mjdamgaard/notes.
		%}
	}
}

\usepackage[margin=1.5in]{geometry}

\begin{document}
\maketitle

\section{Foreword}
{\centering\noindent
	\vspace{-\baselineskip}
	\hspace{-0.7em}
	{\hspace{-4.em}$|$\hspace{\linewidth}\hspace{8em}$|$}
}
Note collection from 10.01.23--???. 

%Kopieret nedenfra:
	%"(10.01.23) Okay, jeg har lige påbegyndt dette notesæt. Jeg er pt. i gang med at skrive version 2 (en meget mere simpel udgave) af min SRC-artikel, og så har jeg ikke kunne lade være med at bruge en del af tiden (som jeg "burde" have brugt på at skrive om SRC) på at tænke mere over, hvordan jeg ville starte mine hjemmesider, jeg har i tankerne, hvis jeg selv skulle gå i gang (hvad jeg faktisk stærkt overvejer). Særligt har jeg tænkt nogle tanker om en alternativ opsætning end den med fanerne, når det kommer til hele den applikation der. De tanker vil jeg skrive ned, når jeg vender tilbage her, sikkert senere i dag/aften. (For inden da vil jeg lige fortsætte lidt mere med SRC-arbejdet.) (17:17)"
%

\ 

\lipsum[1]


\chapter{Web ideas}

%(10.01.23) Okay, jeg har lige påbegyndt dette notesæt. Jeg er pt. i gang med at skrive version 2 (en meget mere simpel udgave) af min SRC-artikel, og så har jeg ikke kunne lade være med at bruge en del af tiden (som jeg "burde" have brugt på at skrive om SRC) på at tænke mere over, hvordan jeg ville starte mine hjemmesider, jeg har i tankerne, hvis jeg selv skulle gå i gang (hvad jeg faktisk stærkt overvejer). Særligt har jeg tænkt nogle tanker om en alternativ opsætning end den med fanerne, når det kommer til hele den applikation der. De tanker vil jeg skrive ned, når jeg vender tilbage her, sikkert senere i dag/aften. (For inden da vil jeg lige fortsætte lidt mere med SRC-arbejdet.) (17:17)

\section{Nye tanker om, hvordan Web 2.0--3.0-siden kunne være til at starte med (10.01.23)}

(17:58, 10.01.23) Jeg har nogle nye tanker om, hvad jeg gør mht.\ mine hjemmeside-idéer. Sidst jeg skrev om det var for ret kort tid siden i mine 22--23-huskenoter, og inden da har jeg skrevet nogle tanker ned i det, der endte med at blive udkommenterede noter under SRC-artiklen (version 1). Men jeg har nu skiftet mening siden de noter. Jeg tænker således ikke længere på at starte med det der fane-værk. Nu har jeg en anden indledende opsætning i tankerne.

Men inden jeg går i gang med at beskrive den opsætning, så lad mig også lige sige, at jeg nu faktisk tænker at fokusere mest på web-applikationen, der har med semantiske strukturer, tag-ratings, opdelte kommentarer, ``automatiske point'' (og ``brugergrupper'') osv.\ at gøre, og derudover så også min debatside-applikation. Og jeg tænker så altså lidt at starte med førstnævnte emne, og så også tænke på at tilføje en debatside-applikation til den hjemmeside også inden for en nær fremtid efter, at jeg har fået den første applikation i gang. Desuden tænker jeg nu at gøre det open source. Jeg tænker altså ikke længere på at fokusere på at starte en SRC omkring det. I forhold til det monetære, så synes jeg nemlig, det er bedre at gøre dette i et separat lag oven over det applikationsmæssige lag. Så jeg tænker dermed altså, at den monetære del bare skal implementeres via donationer, og muligvis særligt via en donationsforening, ligesom.\,.

Ok. Nej, vent. Og fordi jeg vil gøre det helt open source, så tænker jeg altså også bare at tage med arme og ben fra andre open source-hjemmesider. Særligt kunne det være smart at tage fra en YouTube-agtig open source applikation og en Reddit/Twitter-agtig (måske Mastodon) applikation.\,. Og så altså derfra påsætte min.\,. hm, lad mig bare kalde det min `semantik-applikation' for nu; det er nemmere end, hvad jeg ellers har haft gang i (så som min ``Web 2.0--3.0-hjemmeside''). Jeg vil altså prøve at påsætte den applikation oven på de andre --- og efterfølgende påsætte en debatside-applikation oveni, så det bliver altså lidt af en Schweizerkniv-hjemmeside, men hvorfor ikke?\,. (sagde han helt naivt.\,. ej, jeg håber, det kommer til at give god mening sådan, 7, 9, 13).\,. 

Nå, og nu til at forklare, hvordan jeg tænker min semantiske applikation nu --- og det er ved at blive småsent, så jeg ser bare, hvor langt jeg når (og hvor sammenhængende det bliver), og så må jeg samle tråden op igen en af de kommende dage.

Som jeg forestiller mig min semantiske applikation nu (altså hvordan den kunne starte med at se ud rettere, for i fremtiden kan den så få alle mulige former, det er jo en del af det (altså at brugerne selv skal have frihed til at ændre udseendet)), så forestiller jeg mig altså bare en enkelt HTML-side, hvor at når man ser på lister over ressourcer, hvilket bliver en rigtig central del af applikationens brug, så er dette altså bare en liste midt på HTML-siden. Dog skal der så være en menu i siden, som gerne skal kunne være fold-ud. Denne menu bliver så til, hvad vi kan betragte som brugerens ``workspace.'' (Lad mig kalde det `arbejdsbord' på dansk.) .\,.\,Hm, lad mig lige holde en pause, og så ser jeg lige på, om jeg vender tilbage, eller om jeg bare holder for i dag.\,. (18:22)

(12.01.23, 16:58) I forgårs aften fik jeg tænkt lidt flere gode idéer omkring opbygningen, særligt af fold-ud-menuen. Men lad mig lige starte med at gie en helt grundlæggende beskrivelse, og så kan jeg føje flere detaljer til bagefter.

Lad os forestille os at en bruger kigger på kategorien `film,' og har valgt, ikke at se på en liste over film, men valgt at se en liste over underkategorier til kategorien `film.' Bemærk altså at `film' er implementeret, ikke som et er\_film-prædikat, men som et term simpelthen. Lad os så sige, at brugeren vælger `komediefilm.' Brugeren føjer hermed dette term til sit arbejdsbord, ved siden af `film.' Dette term ligger altså nu i fold-ud-menuen. Brugeren kan så nu i princippet gå til fold-ud-menuen og vælge en knap, der skifter visningen fra underkategorier\_af(term) til termer\_af\_kategorien(term). Og ved så at vælge `komediefilm' fra menuen (der også indeholder brugeren arbejdsbord nemlig), så får brugeren altså nu en liste over alle de termer, som passer på termer\_af\_kategorien(komediefilm). Dette er altså en mulig vej for brugeren i princippet, men allerede i prototypen af hjemmesiden skal det altså gerne være indbygget som en standard ting, at når brugeren har en liste over kategorier, så er der også en knap der gør disse ting på én gang, nemlig tilføjer kategorien til arbejdsbordet og går direkte hen til visningen af termer\_af\_kategorien(term), hvor inputtet så er den valgte kategori. 

Nu ser brugeren så en liste af ressourcer (komediefilm i dette tilfælde). Nu vil brugeren så måske gerne lægge et specifikt filter over denne liste, så kun visse ting bliver vist og i en vis rækkefølge. Nu kan brugeren så i fold-ud-menuen vælge en knap til at se, hvad vi kunne kalde filter-prædikater. Lad os sige, at brugeren allerede har browset film mange gange før og derfor har alle de almindelige filter-prædikater klar såsom `populær,' `godt bedømt' osv., men af en eller anden grund mangler prædikatet `sjov' i listen. Nu vil brugeren så gerne finde et sådant prædikat frem. Dette kan så gøres af to primære vejen, foruden at brugeren selvfølgelig kan lave en normal, ikke-semantisk søgning på prædikatet, hvilket man i praksis ofte vil gøre med tiden, men jeg har dog ikke i sinde at fokusere særligt på den mulighed med min prototype. I stedet kan brugeren enten gå til en grundlæggende visning over prædikat-kategorier, og så navigere frem til det ønskede prædikat, nemlig på samme måde som da brugeren navigerede til `komediefilm,' bare hvor brugeren altså nu søger i prædikat-kategorier (og til sidst -termer) og ikke i ressource-kategorier (og til sidst -termer). Når brugeren så har fundet prædikatet `sjov,' kan denne så tilføje det til arbejdsbordet, hvorved det så bliver vist i filter-prædikat-delmenuen. Ellers kan brugeren også forsøge at vælge en liste over prædikater\_relaterede\_til(term), hvor brugeren så kan vælge et term fra arbejdsbordet at putte ind her som input. Og her vil det jo så være oplagt, at brugeren inputter `komediefilm.' Den resulterende liste over prædikater kan så forestilles at indeholde det søgte prædikat (rimeligt tidligt i listen), og brugen kan så vælge det fra denne liste direkte. 

Når brugeren nu har tilføjet `sjov' til sine filter-prædikater, så kan brugeren nu gå tilbage og se termer\_af\_kategorien(komediefilm). Her kunne brugeren med fordel have valgt at navigere væk fra denne visning i første omgang med et midterklik, eller med ctrl + klik, således at den midlertidige søgning på prædikatet foregik i en ny fane. Pointen er nemlig, at arbejdsbordet skal ændres på tværs af faner (hvis brugeren altså bliver i det samme workspace, men det kommer vi til). I så fald kan brugeren bare klikke tilbage til browserfanen, hvor han/hun kom fra, og får herved nu vist `sjov' i filter-prædikat-menuen. 

Nu kan brugeren så tilpasse filteret mht.\ dette prædikat, hvilket indebærer at basalt set indstille en kurve over en akse for, hvor `sjov' filmen er. Jeg forstiller mig således en todelt gaussisk kurve, som altså næsten er en gaussfunktion, bortset fra at den kan have to forskellige spredninger til hver side (så altså en asymmetrisk version af en gausskurve). I dette tilfælde vil brugeren så sandsynligvis ønske at sætte toppunktet for kurven i enden, hvor `sjov' er mest gældende. Herefter kan brugeren så bestemme spredningen til den eneste af de to sider, der her er relevante. Og sidst men ikke mindst skal brugeren også kunne indstille en min- og en maks-værdi til filteret, således at for eksempel ingen film bliver vist, der har under en vis `sjov'-score. 

Jeg forestiller mig mere specifikt dermed at brugeren får vist tre barer, når brugeren folder indstillingerne ud for et filter-prædikat: Den første har én knap, der kan bevæges over hele baren, og som bestemmer kurvens toppunkt. Den næste bar kunne så være til at indstille de to spredninger og kunne så være en todelt bar, hvor knappen i hver siden kan sættes tæt på midten eller tæt på den respektive ende af baren, alt efter om spredningen til den side skal være lille eller stor (helt flad kurve, hvis man sætter dem i endepunkterne). Den sidste bar kan så være to knapper, der også kan sættes hvor som helst, og som så angiver min- og maks-værdierne for filteret. Imens brugeres finindstiller filter-prædikatet således, forestiller jeg mig, at brugeren også for vist den pågældende kurve i en lille graf over barerne. 

Nu kunne man så tro, at brugeren bare vil tage en virkelig lille spredning for kun at vise de film, der er har fået de allerhøjeste `sjov'-scorer, men i så fald ville de andre filter-prædikater, såsom f.eks.\ `godt bedømt' *(`god,' rettere) blive ubetydende, og derfor kan det altså ofte give god mening ikke at gøre kurverne alt for smalle. 

Cool, lad mig lige tænke over, om der er mere jeg skal sige til denne første lille del-rundgang, og ellers vil jeg gå over til at opsummere detaljerne, som jeg har tænkt mig nu her omkring denne nye opsætning (efter en lille pause.\,.). (17:55)

(18:09) Jo, jeg skal selvfølgelig også lige omkring det her med at rate ressourcer. Lad os sige, at brugeren undrer sig over, at en vis komediefilm kommer enten tidligt eller sent i listen, og så kigger på `sjov'-ratingen og ser, at den ikke ser ud, som brugeren ville forvente/synes. Så bør brugeren så kunne klikke sig ind på en rating-visning, nu hvor.\,. hm, jeg skulle til at skrive, at det er hvor alle andre spredninger er sat til uendeligt og at alle andre kurver end det relevante prædikat dermed bliver til firkantede trinkurver mellem deres valgte min- og maks-værdier i stedet, men er det nu også virkelig ideelt.\,.\,? 

\ldots Ah, man skal selvfølgelig bare have nogle separate filter-indstillinger for at vise rating-lister, og i starten vil de fleste så sikkert kunne nøjes med at bruge et enkelt prædikat udover det pågældende man rater i forhold til, og det er så popularitetsprædikatet. Senere kan man så prøve at implementere nogle indstillinger, hvor brugerne også får prioriteret at vise termer, der er bekendte for brugeren i listen, men det vil så være mere kompliceret at få dette op at køre. Så ja, når brugeren trykker, at denne vil ind og rate termerne, f.eks.\ ift.\ sjovhed i dette tilfælde, så får brugeren altså bare en ny rating-visning, hvor der også er særlige filter-indstillinger til, og hvor brugeren så kan rate ressourcerne/termerne. Og som jeg har forklaret i tidligere noter, så forestiller jeg mig altså her, at brugeren så særligt skal kunne rate termer/ressourcer (ift.\ pågældende prædikat) ved at trække termen op eller ned i listen --- og hvor brugeren kan zoome ind og ud i, hvor mange termer, der vises i listen, hvilket så altså meget vel ofte kunne indebære at indstille på, hvor mange termer vises i listen efter popularitet (altså mere præcist stille på, hvad min-værdien er for popularitet i listen). (Og jeg forestiller mig altså, at brugeren endda kan ``zoome ind og ud'' imens denne trækker en ressource/et term også.) 

Når brugeren så har rykket lidt rundt på termer i denne liste og derved givet dem rating i henhold til pågældende prædikat, kan brugeren jo så gå tilbage til, hvad brugeren egentligt var interesseret i, nemlig at se på en liste over sjove komediefilm. Slut på denne lille overordnede rundgang. Og så kan jeg skrive nogle flere detaljer om disse seneste idéer en anden dag (måske i morgen). (18:58) 

.\,.\,Hm, lad mig lige nævne, at når brugeren skifter type af visning, altså hvis denne f.eks. skifter fra ressource- til prædikat-visninger.\,. hm.\,. .\,.\,Hm, jeg skulle til at skrive noget med, at det ofte gerne må være standarden, at når brugeren går til en anden visning, at der åbnes en ny fane i browseren, men på den anden side er dette svært at sige præcis, hvad der er smartest her. Og måske er det fint, hvis brugeren bare husker at trykke ctrl (eller midterklik) på de rette tidspunkter og, nå ja, så kan brugeren jo altid gå tilbage i browseren --- især idet listerne som regel er simpel HTML, hvilket sikkert vil sige, at browseren endda tit vil kunne huske, hvor i listen brugeren var og dermed kan vende tilbage til samme sted.\,.\,:) 


(13.01.23, 16:40) Nu vil jeg så prøve at tilføje lidt flere detaljer om den opsætning, jeg nu tænker, at prototypen på den semantiske hjemmeside kunne have. Ja, og jeg kan jo starte med at sige, at jeg jo tænker at hjemmesiden kunne have flere applikationer i sig, bl.a.\ en YouTube-agtig applikation og en Reddit-/Twitter-/9gag-agtig applikation, hvor brugere så kan uploade ressourcer i forbindelse med de applikationer, som så derefter også kan findes i den semantiske applikation.

Og hvis vi så ser på den semantiske applikation, så forestiller jeg mig altså en ret central fold-ud-menu, hvor brugerne kan skifte visningen som beskrevet lige her ovenfor. Og når brugeren skifter visningen skal det så bare i prototypen føre til en ændret URL, hvilket så gør at brugerne kan gå frem og tilbage i deres seneste visninger, og åbne flere faner for visninger, ved simpelthen bare at bruge deres browsers velkendte funktionalitet hertil.

Fold-ud-menuen kan så have en liste af faneblade i toppen, der afgør, hvilket workspace brugeren arbejder i. Som sagt, så skal det være sådan, at hvis brugeren føjer en ting til et givent workspace i én fane, så skal ændringen ske i alle faner (som opdateres, når brugeren klikker ind på den igen (eller når brugeren går frem og tilbage i sin sessions browserhistorik)). Men ændringen skal dog kun ske i pågældende workspace.

Nedenunder denne liste kan så være endnu en liste over overordnede undermenuer i fold-ud-menuen. Den mest centrale undermenu er så den, hvor alle de tilføjede termer i workspace'et findes. Jeg forestiller mig, at en mulig løsning her for prototypen kunne være en træ-struktureret menu a la den jeg har her ude i venstre side af min TeXstudio-editor over sektioner og undersektioner (og hvor man så kan folde oversektionerne ind og ud). Sådanne oversigter har sikkert et teknisk navn, men det kender jeg ikke lige på stående fod. Og en endnu simplere løsning, måske til en tidlig version af prototypen, kunne bare være at have en enkelt liste ordnet kronologisk ift.\ hvornår termerne sidst blev brugt/tilføjet. I øvrigt kunne man også tænke sig en blanding af disse muligheder, sådan at man tilføjer endnu en fane af muligheder, hvor den ene så er omtalte træ-struktur-oversigt, og hvor den/de andre bare er lister over seneste brugte og/eller seneste valgte og/eller seneste tilføjet.\,. Ok.

Men udover at se lister over termer (som altså også inkluderer prædikater og relationer --- og også relationer og input-termer sat sammen til prædikater), som så kan bruges til at vælge nye visninger med, så skal der også være en (overordnet) undermenu med filter-prædikater, som nævnt i overordnede tekst. Og der skal være en menu specifikt beregnet til rating-visningerne (altså når brugeren iagttager en liste med udgangspunkt i et specifikt prædikat, som termerne i listen så kan rates i forhold til (gerne ved at brugeren kan flytte op og ned på termer)). 

En anden undermenu, der faktisk bliver ret væsentlig selv på nogenlunde kort sigt, er en hvor brugerne kan vælge indstillinger for, hvordan de gerne vil have diverse ressourcer og lister vist. Her bør man så faktisk ret hurtigt gøre det til en del af applikationen, at brugere kan aktivere CSS styles for diverse ressource-typer. (Og ja, det er en god idé, at man fra start af indfører diverse typer for de forskellige, ja, typer af ressourcer. Så applikation skal altså have et helt typesystem, hvor brugere kan tilføje nye typer, og hvor man så også har indbyggede list-constructor're til at danne list-typer. Her kan der så endda være forskellige typer alt efter om vi f.eks.\ snakker en standard oversigtsliste, eller om vi snakker om en liste beregnet til en rating-visning. Når en bruger opretter en ny ressource-type, så sker dette ved at definere et HTML-template for typen.\,. Tja nej, faktisk to HTML-templates: Et der udgør selve ressourcen, og et der udgør den data, der skal vises, når ressourcen indgår i en listeoversigt over flere ressourcer (f.eks.\ thumbnail og kort beskrivelse og sådan). Indholds-HTML-skabelonen bør så også have datafelter i sig med f.eks.\ titel, tilhørende tekst, og hvad man ellers kan finde på, og hvis ressourcen indeholder et billede eller en video af en art (måske i et eller andet fast format), så skal skabelonen selvfølgelig også indeholde de datafelter. Når så en type er oprettet, og brugere skal uploade andre instanser af typen, så skal de så bare uploade data af passende formater, der passer til felterne i HTML-skabelonen. Og når brugere så får vist ressourcen, så sættes alle disse datafelter altså ind i HTML-skabelonen. Og når brugerne for vist en liste over ressourcer af den pågældende type, så vil elementerne i listen vises i form af den anden HTML-skabelon, hvor det passende data (så som f.eks. thumbnail og kort beskrivelse) er loadet ind. Og for så at vende tilbage til den undermenu, som vi startede med at snakke om, så skal brugere altså gerne kunne indstille, hvordan diverse typer helt præcist skal vises, bl.a. ved at vælge CSS-indstillinger. Og på sigt (gerne kort sigt) skal brugerne endda og også få mulighed for at ændre i og lave tilføjelser til selve HTML'en (ligesom browserudvidelser fungerer), sådan at de kan få vist det lige som de vil have det. 

I øvrigt, og det gælder både, når brugere opretter HTML-skabelonerne, og når brugere opretter udvidelser til eksisterende HTML-skabeloner, så vil det være godt på sigt at få det sådan, at brugere kan lave (div-)felter i skabelonerne, som faktisk får mulighed for at query'e selve hjemmesidens database, og altså vise ting fra denne database. Dette kommer til at gøre, at man kan få levende ressourcer, hvis udseende og struktur kan afhænge af databasens tilstand. For eksempel kunne man lave HTML-skabeloner, der automatisk henter lister over mest relevante ressourcer til pågældende ressource, eller henter kommentarer, ratings, annotationer osv.\ osv.


Det skal så dog nævnes, at den semantiske applikation allerede indeholder en måde, hvorpå brugere kan kommentere ressourcer, nemlig ved at tage termen, der udgør ressourcens reference, og så vælge en relation i den forbindelse, således at man danner et prædikat: er\_kommentarer\_tilhørende\_ressourcen(term), hvor `term' så her er en placeholder for på-gældende ressource. Herved kan man altså gøre kommentarer til ressourcer til en del af den samlede (semantisk strukturerede).\,. database, eller rettere: selve den underlæggende database bør jo (nok) bare være en relationel database, men udefra set får man altså en semantisk struktureret database, og det er altså det jeg mener, når jeg siger en semantisk struktureret database; det er sådan, den ser ud fr brugerne. 


Nå, tilbage til fold-ud-menuen: Derudover kunne det nok også være smart (på sigt) med menuer, mest for de avancerede brugere, til at vælge opsætningsindstillingerne for selve fold-ud-menuen, sådan at selv denne bliver en del af alt det, som brugerne i sidste ende selv kan bestemme opsætningen for.

Ok, så det var altså den overordnede struktur.\,. Nå ja, på nær at jeg også lige kan nævne, at man på sigt også kan åbne op for, at selve applikationen får sin egen måde at have gang i flere visninger på én gang, sådan at dette ikke bare kun opnås via browserens funktionalitet (og ved hele tiden at skifte URL for hver visning). Så på længere sigt skal man måske også gøre, så at brugere kan implementere sådan navigation, og også gerne mulighed for f.eks.\ split-screen-visning, i applikation selv. 

Okay. Det var den gennemgang. Så havde jeg vist lige en eller flere ekstra punkter, som jeg også gerne ville nævne (men som jeg dog har talt om i tidligere noter), hvis jeg ellers kan huske, hvad de var.\,. (Og hvis det altså ikke bare var dem, jeg allerede har nævnt nu her.\,.) .\,.\,Nå, hvis der ar noget mere, så er det lige glippet, så jeg vil bare gå og summe lidt over det, og så ellers bare vende tilbage, når jeg finder nogen tilføjelser, jeg også bør nævne. (17:44)

(14.01.23, 11:48) Okay, der er nogle tilføjelser, jeg mangler, og jeg har også tænkt på lidt nye ting. Jeg mangler at nævne, at al data gerne skal gemmes via, hvad der svarer lidt til tripletter, men hvor de dog ikke behøver at være begrænset til 2-nære relationer; de kan også være 1-ære (altså prædikater) og 3-nære. Der må så vidt jeg kan se også være relationer med endnu flere input en tre, hvis brugerne får behov for dette. Og foruden relation-ID og ID på alle input objekterne (hvilke også kan være placeholders, for man skal f.eks.\ gerne kunne danne prædikater ud fra (f.eks.) en 2-ær relation og en inputterm, so vi har set ovenfor), så skal der også i alle disse udvidede tripletter være bruger-ID for hvem, der siger/har uploadet udsagnet, samt også et tal, der bestemmer hvor meget brugeren mener at udsagnet er sandt, og som i øvrigt også kan være negativt, således at brugeren kan negere udsagnet uden at skulle skifte relation. Dette med at alle udvidede tripletter, som jeg fra nu af vil kalde udsagn, har sådan en floating point rating med sig, er faktisk en super vigtig ting for hele idéen. Uden dette ville applikationen ikke blive nær den samme. .\,.\,Og endda selv for mange relationer, hvor man umiddelbart tror, at man kun er interesseret i at høre en sandt-eller-falsk vurdering fra brugere, kan det alligevel være gavnligt med et floating point-tal --- ikke for alle tilfælde (og så må man bare omfortolke tallet til en binær størrelse), men for mange. For eksempel hvis vi tænker på et udsagn: ``hører denne term til en vis kategori eller ej?'' (som underkategori eller som genstand, der hører til kategorien). Her skulle man tro, at man bare var interesseret i et ja-nej-svar fra brugerne, men faktisk kan det her være rigtigt gavnligt, hvis udsagnene alligevel kan gradbøjes, for nogle underkategorier (eller ressourcer) bare mere relevante end andre, og hermed kunne man altså få en nem måde at gøre, så at når en bruger ser på underkategorier til en kategori, så er det de mest relevante underkategorier, der popper frem øverst på listen. 

Jeg fik heller ikke nævnt noget, jeg ville sige om mine ``brugergrupper,'' som jeg har skrevet en del om i tidligere noter (se disse). Angående dette emne, så tror jeg bare man skal starte med den type ``brugergrupper,'' hvor en vis gruppe af brugere (muligvis bare én) starter med en lige mængde af nogle delelige tokens, der giver dem stemmevægt ift.\ at bedømme udsagn. .\,.\,Hov vent, jamen \emph{skal} vurderingerne så komme med udsagnene, eller skal udsagnene gemmes for sig, og så kan brugeres vurderinger af dem gemmes for sig også.\,.\,? Hm.\,.\,. (12:12) .\,.\,Hm, og brugeren, der uploadede udsagnet, bør i så fald også bare gemmes som et separat udsagn, oprettet automatisk af serveren.\,. hm, så man også kan slette det igen uden at slette brugeren, men hvordan ser man så.\,. Nå jo, i princippet kan serveren så oprette to udsagn for at gemme, at brugeren var ophavsmanden; udsagnet selv og så et udsagn om, at denne server siger, at det var den bruger (med 100\,\% sikkerhed, hvorfor ikke?\,.), der uploadede det originalt (eller rettere set med pågældendes servers øjne (men den siger så ikke noget om, hvorvidt andre brugere var først på andre servere)). Ja, sådan kunne det sagtens være.\,. Hm.\,. \ldots Ja, så nu går jeg faktisk ind for tripletter igen (har jeg ellers ikke gjort i lang tid, mener jeg).\,. .\,.\,Hm, og man kan så faktisk bare droppe de der server-vurderinger,
hvis man bare i stedet gør sådanne, at visse relationer er off-limits ift. hvad brugerne selv kan uploade, nemlig de relationer som er beregnet til at blive ``uploadet''af serveren(erne). 

Nå, men tilbage til ``brugergrupper.'' Omtalte tokens kan så deles i flere, og de kan så efterfølgende gives (delvist) ud til andre brugere. En giver af tokens i en brugergruppe må i reglen godt altid annullere en overførsel af tokens, medmindre.\,. Ja, lad mig sige det sådan her: Tokens kan lånes ud til andre brugere, som kan låne dem videre ad libitum, og hvis så en udlåner af tokens skifter mening, kan denne hive sine tokens tilbage med det samme (uanset hvor mange gange, de er lånt videre). Brugere kan også \emph{give} tokens til andre brugere, nemlig hvis de gerne vil pensioneres som ansvarshaver i brugergruppen. Hver token har så en et floating point number, der bestemmer stemmevægten, som gives af denne. Den samlede stemmevægt summer så op til 1. Og bum, så har man allerede et ret effektivt system til at danne diverse grupper.

Pointen med ``brugergrupper'' er så, at brugere skal have mulighed for, når de indstiller et filter-prædikat, at vælge hvilken/hvilke brugergrupper, som vurderingen skal beregnes ud fra. Så i pågældende menu skal der altså gerne for hver filter-prædikat være en knap til at folde en liste af brugergrupper ud (inkl.\ den basale, hvor alle brugere bare har én stemme), og hvor man så, muligvis ved at indstille vertikale barer tilhørende hver brugergruppe, kan indstille sin vægt til hver brugergruppe i filteret. 

Ok, det var allerede rimeligt dækkende, men jeg har stadig en del flere ting, der skal nævnes, og faktisk også som skal overvejes. Men lad mig starte med en positiv ting, og det er, at brugerne via de ovenfor omtalte HTML-skabelon-udvidelser kan implementere diverse knapper og fold-ud-menuer, når ressourcer af en vis type vises i en liste. For eksempel skal der som en standard være en ``udvidelse'' (som dog altså er standard) til visningen af kategorier i lister.\,. lad mig sige `visning af kategori-\emph{referencer}' fra nu af, som giver mindst to knapper: én hvor brugeren bliver ledt hen til underkategorier (og får kategorien tilføjet til workspacet, om ikke andet så i menuen af `seneste termer' (af typen `kategori')), og én hvor de brugeren bliver ledt hen til en visning over ressourcer/termer i kategorien. .\,.\,Nå ja, og der skal så også gerne være en knap, der leder brugeren hen til en visning over prædikater, der relaterer sig specifikt til den kategori. Denne ``udvidelse'' skal så gerne oprettes, inden applikationen åbnes op for almindelige brugere. 

Og lad mig lige indskyde, at filter-menuen gerne skal have nyligt tilføjede prædikater vist i toppen, da man må regne med, at brugere ikke nær så ofte vil behøve at stille på de typiske prædikater. Men hvis brugere alligevel har nogle typiske prædikater, de gerne tit vil stille på, så skal de så bare have lov til at ``pinne'' dem til toppen af menuen, som man siger. 

\ldots Hm, nu overvejer jeg, om der overhovedet er så mange flere ting, der skal siges i denne omgang, for jeg synes egentligt, at løsningen med hurtigt at arbejde i HTML-udvidelser til referencevisningerne faktisk løser meget af det problem, jeg havde, med at det virkede for indviklet at bruge applikationen i udgangspunket. Jeg tænker lige lidt mere over det, men nu vil jeg ellers bare lige tilføje den idé, at der også skal være en (overordnet) undermenu i fold-ud-menuen (som mange brugere i øvrigt sikkert vil have konstant foldet ud, hvis de arbejder på en computer (ikke en telefon)), som simpelthen er en konsol, hvor brugere kan skrive de udsagn (med placeholders), de gerne vil søge på, og også dem de gerne vil uploade, og hvor der så bør være automatiske forslag til udfyldning af ordene, hvor applikationen altså så (primært) søger i termerne i brugerens workspace, når den skal give forslag til udfyldning af ordene (altså `word completion'). (13:39)  

\ldots\ Der vil også være behov for tokens, der ikke kan videregives af modtager (og som modtager i øvrigt ikke skal acceptere eller afslå, men bare får tildelt sig af en anden). Dette kan f.eks.\ bruges til at flagge spammere; så kan de styrede brugere uddele tokens, enten til alle spammere, eller alle ikke-spammere (som man ikke har mistanke til). Brugere kan så bruge disse bruger-grupper i et er\_spammer- eller et kommer\_fra\_en\_spammer-filterprædikat. (18:03)

(16.01.23, 16:30) Der skal også være en slags automatiske brugergruppe tokens, men nærmere bestemt kan man også kalde dette for `automatiske point.' Vi snakker altså point, som kan gives til brugere --- eller til alle mulige andre termer i databasen, nemlig (og bruger-ID'er indgår nemlig også i databasen som termer) --- ud fra data i databasen om dem. Dette kan så specifikt bruges til at implementere, hvad jeg også i mine tidligere noter har kaldt `brugerdrevet machine learning (ML).' Hvis vi så tager en vis korrelationsegenvektor, når man har lavet ML-statistik over brugerne af applikationen, så kan man altså nu, via disse `automatiske point,' tildele brugere point alt efter, hvor stor projektionen af deres brugerdata ind på på gældende egenvektor er. Og dette kan man jo så bruge videre i diverse filter-indstillinger. For eksempel kunne man forestille sig, at man kunne sige: `sorter disse film ud fra `sjovhed,' med særlig vægt på folks vurderinger, som følger den og den korrelationsvektor. Og ja, man kan i øvrigt også gøre mere simple ting, så som bare at sige: `sorter disse film ud fra `sjovhed,' med særlig vægt på folks vurderinger, der også synes at det og det var sjovt --- altså en mere simpel form for statistisk brug a brugerdata i brugerlavede filteralgoritmer. Mulighederne er virkeligt åbne.

Og måden man så kan query'e databasen om sådanne point kan så bare være ud fra en syntaks, der følger det logiske programmeringsparadigme, nemlig således at pointen både gemmes som tripletter i databasen for hver bruger, og hvor man så kan query'e disse point via den relation, der nu hører til pointene. (16:45) .\,.\,Og lige for at præcisere, så er det altså brugerne selv, der kan uploade forslag til nye automatiske point, nemlig ved at de så uploader den relevante metadata, samt den formel, som det hele handler om, nemlig den formel ud fra hvilken de automatiske point bliver givet til termerne (som i øvrigt i reglen vil være af en bestemt type, som så også defineres som en del af omtalte metadata (eller `header-data' er nok mere rigtigt at kalde det.\,.)). 

(17:13) Jeg havde også i sinde at skrive lidt om, hvordan jeg så forestiller mig, at man også kunne implementere en ``debatside-applikation'' i dette system, hvor brugerne kan bruge dette brugergruppesystem, og jeg kunne måske også finde på nogle små nye tilføjelser i denne forbindelse, men jeg tror nu, jeg bare vil lade emnet være for nu. Jeg har skrevet fint om det, i mine tidligere noter, og selvom jeg sikkert kunne finde på nogle små ting at tilføje, så tror jeg ikke, det vil ændre så meget på idéen overordnet set. Så lad mig lade det emne være for nu. Det ville alligevel også skulle implementeres \emph{efter}, at man får den semantiske applikation op og køre.

Jeg bør også på et tidspunkt vende tilbage her og overveje nærmere, hvad jeg skal begrænse en prototype til, for lige nu har jeg nævnt væsentligt flere features, end man bør prøve at få med fra start af. Hm, jeg mener dog, det vil være en god idé, hvis man så hurtigt som muligt gør, så at brugerne selv kan lave omtalte HTML-udvidelser. Men ellers er det nu godt bare at starte simpelt, og så bygge på derfra. (17:20)

\ 

(24.01.23, 11:39) Man kan sagtens bruge triplet-konstruktioner, man kan også sagtens tillade +2-ære relationer, hvis man vil, det betyder ikke så meget, for man kan altid omstrukturere, hvis man finder ud af, at der er en bedre standard. Så længe brugerne har frihed til at skabe de konstruktioner, de vil, og at grammatikken i disse konstruktioner er veldefineret, så går det fint. To gode muligheder er derfor, 1, kun at tillade 3-term-konstruktioner (tripletter), eller at gøre det helt frit ligesom i en logisk database.

Nu kommer det mere vigtige, dog: I første omgang handler det hele om at konstruere `udsagn' (som brugerne så efterfølgende kan rate (bedømme)). Men disse udsagn skal så \emph{ikke} tolkes som i standard formel logik, hvor har en binær (boolsk) værdi, så at sige. I stedet skal de ses som termer, der beholder informationen om hele deres indre konstruktion, når de indgår i en kontekst som sammensat formular. Hermed for man nemlig særligt mulighed/lov til at sige ting som: ``Jeg synes udsagnet er\_sjov(film) er sand til en grad af 9/10 rating score.'' Man åbner altså herved op for en meget mere intuitiv måde at kontruere sætninger på, en hvis man skulle beskrive den samme sætninger ud fra mere formel logik, hvor er\_sjov(film) altså bare ville være en binær værdi over alt, hvor den indgår, og hvor denne sætning så ikke ville give nogen mening. Man kan så også konstruere mange volapyksætninger med denne semantisk, og kan altså lave ugyldige og eller paradoksale sætninger, men det er kun et sundt tegn, ift.\ hvad sproget skal bruges til. 

De helt centrale udsang, som nemlig er de eneste udsagn, der repræsenterer direkte \emph{sandheder} i systemet, det er så rating-udsagnene. Disse består af et bruger(-ID)-subjekt, et udsang-term (så som `er\_sjov(film)') og så en rating score. Tja, eller dvs., der mangler én mere information her, og det er informationen om, hvad konteksten er for ratingscorens talværdi. Her kan man så eksempelvis, hvis man tillader +2-ære relationer, som jeg så forresten i virkeligheden er funktioner, fordi alle udsagn er termer (og disse funktioner vil så i øvrigt også som regel være bijektive, i henhold til hvad jeg lige nævnte om, at information ikke bør gå tabt i udsagnene (indmaden forsvinder ikke, med andre ord; hvert udsagn ``kender'' sit eget udsende på papiret)).\,. hvis man tillader 2-ære relationer, så kan man så tilføje en funktion, der sender et udsagn så som `er\_sjov(film)' til et nyt udsagn, der siger ``er\_sjov(film) ud fra ratingscoren $x$,'' hvor $x$ så også er input til omtalte funktion. Nå ja, så lige med dette eksempel er der så ikke behov for +2-ære relationer, eller funktioner rettere, men hvis man skal kunne sådanne ting, så bør man dog kunne have funktioner af alle mulige typer (og ordner). *(Hm, så never mind, at tripletter er fine for systemet. Jeg synes faktisk, at databasen bare mere bør være som en logisk database.)

*Hov, jeg skal forresten lige nævne, at de eneste restriktioner i databasen, udover på hvor meget data og hvor mange termer hver bruger må uploade hver dag/uge/måned, hvilket i øvrigt vil være en god idé at have, nok ellers bare skal være, at det kun er brugeren selv, der må uploade rating-udsagn, hvor brugerens bruger-ID indgår. Så hvis en bruger-udsagn-rating-instans forekommer i databasen, så er det fordi pågældende bruger har givet den rating. Hver bruger skal også have mulighed for at slette sine egne ratings igen fra databasen (hvorfor det faktisk ikke dur med et decentralt system, synes jeg). I den forbindelse kan jeg så også nævne, at man godt kan få det sådan på sigt, at ressourcer faktisk automatisk sender rating-information til databasen på brugerens vegne (hvis brugeren har valgt de ``HTML-udvidelser,'' der gør dette), hvilket f.eks.\ så kan bruges til at sende, om brugeren har set en vis ressource eller ej (hvilket kan bruges til at fjerne gengangere i et feed, f.eks.), men så kan brugeren altid bare selv fjerne sådant data igen efter eget behov. Nå ja, og herved er det jo så også vigtigt lige at nævne, at brugerne sagtens skal kunne uploade data til deres egen private del af databasen, sådan at f.eks.\ ikke alle kan se den data om, hvad brugereb har set og ikke set, eller kan se alt hvad brugeren har ratet for den sags skyld. Brugerne kan sagtens rate rent anonymt, og så bare kun bruge den data til at forbedre egen oplevelse, hvis de vil. Men de kan så selvfølgelig også vælge at offentliggøre noget af denne data (dog med mulighed for at slette det igen fra den offentlige del af databasen), hvis de gerne vil bidrage til, hvad andre brugere ser af brugerratings (hvad de fleste brugere gerne vil; det er typisk en vigtig del af, hvorfor vi normalt afgiver ratings rundtomkring på internettet). (12:42)

Men kommer brugerne ikke bare til at bruge den samme type rating hele tiden? Nej, jeg har nemlig tænkt på, at det vil være rigtigt gavnligt, hvis brugerne selv kan vælge, hver gang de rater, om de bare vil rate ud fra en treværdi-score, nemlig negativ, neutral, positiv, eller om de vil bruge flere muligheder, f.eks. fem stjerner eller ti stjerner, eller hvad det kunne være. Og her snakker vi så ikke den rating, jeg har beskrevet, hvor brugerne rater ud fra en liste med en masse andre relevante ressourcer i. Her snakker jeg, hvis brugerne bare skal rate ressourcen alene i dens egen kontekst, enten på ressourcens egen side, eller måske når den vises i en liste. 
\ldots(12:42) Her er det jo så vigtigt, at brugerne så kan sætte dette i system, så de alle kan kende forskel på, hvad den pågældende ratingscore repræsenterer. 

Jeg har skrevet ovenfor et sted, at man kun skal have termerne ordnet ud fra det (filter-)prædikat, man rater med hensyn til, når man rater ressourcer ud fra en liste, og at alle andre filter-prædikater så kun skal være et spørgsmål om at sortere ressourcer fra i listen. Men dette behøver ikke nødvendigvis at være sandt. Man kunne således godt tænke sig, at.\,. Tja, eller rettere, det skal nok være det samme prædikat, så som eksempelvis `er\_sjov()', men det behøver ikke nødvendigvis kun at være én type ratingscore, man så bruger til ordningen. Men kunne således godt forestille sig en ordning, hvor både folks negativ-neutral-positiv-ratings er med, hvor femstjernede og tistjernede (etc.) ratings tæller med, og hvor selvfølgelig folks rating-ud-fra-en-liste-ratings også tæller med. Så kan de forskellige typer scorer indgå på forskellige måder ud fra den endelige sortering efter brugerens eget behov, men når brugeren så afgiver sine egne svar, jamen så er det så bare rating-ud-fra-en-liste-scoretypen, der bliver valgt for den rating, der uploades til databasen herved. .\,.\,Hvordan man så blander de forskellige ratingtyper sammen til en enkelt sortering på en fornuftig måde, det er så en noget mere kompliceret sag, men det er også lige meget her, for det er nemt at overbevise sig selv om, at det må kunne lade sig gøre på en fornuftig måde (især hvis man finder en god måde at implementere ``usikkerheder (statistiske) / spredninger'' på i forbindelse med diverse ratings). 

.\,.\,Lad mig lige tænke over, hvad jeg ellers gerne vil sige noget om.\,. (13:07) .\,.\,Hm, jeg kunne sige, at hvis man så bruger, hvad der svarer til en logisk database (i hvert fald i interfacet med brugerne), så vil man så skulle lave en query-API, som brugerne kan bruge i ``HTML-udvidelserne,'' der svarer til logiske query-sprog, men det giver jo næsten sig selv.\,. .\,.\,Hm, det var faktisk muligvis alle de tekniske ting, jeg ville nævne for denne omgang. Så har jeg også nogle mere overordnede ting, som jeg tror, jeg vil skrive på engelsk under en ny sektion. Så kan jeg jo derfor også bare vende tilbage hertil, hvis jeg har nogle tilføjelser til nogen af disse tekniske ting. (13:18)

*Jo, der er faktisk lige den tilføjelse, at min SRC-idé jo også kunne virke godt til denne idé, nemlig især hvis man gerne vil have et fastansat hold af programmører til siden --- og i det hele taget når det kommer til, at man jo gerne, efter min mening, skal have nogle centraliserede databaser, som så kan underskrive kontrakter om ikke at offentliggøre eller videresælge privat data fra brugerne, samt at brugerne skal kunne bede om at få deres data slettet.

*(17:05) Dette er ikke en teknisk tilføjelse, men en bare en kort tilføjelse omkring min ``debatside-applikation.'' Jeg nævner det nok igen i Sektion \ref{Some_hopes_in_terms_of_my_ideas}, men angående min debatsideidé, så kan det godt være, at den ikke rigtigt vil.\,. take off.\,. få god ind i sejlene, før at den videnskabelige verden virkeligt kommer med og deltager i hele projektet med at strukturere, i dette tilfælde viden og videnskabelige diskussioner, i semantiske grafer. Så jeg håber altså på, at den videnskabelige verden vil benytte denne semantiske teknologi mere og mere, og herved vil det jo så selvsagt være relevant at afholde sådanne diskussioner, som jeg tænker dem i forbindelse med, hvad jeg kalder min debatside-idé. Et alternativ ville være, hvis mine tanker omkring e-demokrati fik god vind i sejlene forinden, så tror jeg også, at dette kunne blive en vej til, at få godt gang i sådanne graf-diskussioner, som jeg tænker dem.\,.

*(26.01.23) Vil lige for det første kort nævne, at jeg synes, det giver rigtig god mening, hvis man i bruger-udsagn-rating-entiteterne også har et flag / en reference til, hvordan ratingværdien skal tolkes. For det synes jeg nemlig giver bedst mening rent fortolkningsmæssigt: Så kan man nemlig se udsagnet for sig som sin egen ting, og se ratingen samt dennes fortolkning som sin egen ting også. (12:52)

*(12:52) Nå, den næste, mere vigtige ting, jeg vil tilføje, er, at jeg nu er kommet på, at brugerne måske også skal kunne tilføje URL-RegEx'er til at scrape indhold fra andre sider. Og når man så har disse scrape-formler, så kan brugere så tilføje specifikke inputs til disse RegEx'er, der giver en gyldig URL til en eksisterende Web-ressource. Og så kan databasen i princippet bare gemme selve inputtet, og når brugerne så iagttager pågældende ressource, så kan hjemmesiden bare hente de relevante URI'er fra sættet af de fuldendte URL-RegEx'er og indsætte dem på hjemmesiden. Så dette kunne altså være en hurtig måde at få en hel masse indhold på siden.

*(12:59) Samtidigt åbner dette så også op for, at man ret nemt kan lave et ``overlay,'' som jeg før har kaldt det i mine noter, nemlig en slags annotationsudvidelse til browseren, således at en bruger kan folde en menu ud og se relevant data, særligt ``rating-tags'', som jeg har kaldt det, for en ressource de browser på en anden hjemmeside. Så når brugerne browser andre hjemmesider, kan de altså få adgang til den ``rating folksonomy,'' der tilhører semantik-hjemmesiden. De kan også få adgang til anden data, så som relevante links, inklusiv den/de oprindelige kilde(r) til ressourcen, og også diverse advarsler, f.eks.\ om NSFW/NSFL, og ikke mindst også om `misinformation.' Disse muligheder kunne muligvis være med til at gøre semantik-applikationen/hjemmesiden populær lynhurtig i sammenligning med, hvis brugerne kun kunne få alle disse muligheder, når de browser semantik-hjemmesiden selv. Man kunne i øvrigt også sælge hele denne idé denne vej rundt, altså hvor man starter med at sælge idéen om et tværgående system, hvor brugere kan få adgang til en ``rating folksonomy,'' der fungerer på tværs af alle mulige hjemmesider, og hvor brugerne også kan få andre links og advarsler herved, og så derfra pointere, at denne applikation så også bør have sin egen hjemmeside, hvor brugere kan se alle tingene samlet, endda på en semantisk struktureret måde, der giver et meget effektivt overblik over alting. 
*(RegEx'erne skal jo også gerne hente al mulig gavnlig metadata såsom titler osv., hvis de kan, og dermed bliver det også relativt nemt at finde en ny passede URL (og måske med nye RegEx'er) til ressourcen, hvis nu hjemmesiden, hvor den er hentet fra i første omgang, skulle ændre sig.)

*(13:11) Nå, og den tredje/fjerde ting, jeg også lige ville nævne her, er omkring trust-fordeling. Når det kommer til brugere, der tilføjer nye ting i applikationen, jamen så vil det være rimeligt nemt at sortere skidt fra kanel, for så kan brugere bare stige i tillid, jo mere de tilføjer, der bliver godkendt af andre brugere, men ellers vil det altid også bare være sådan, at sensitive og/eller utålmodige brugere bare kan have et ret skrapt filter for nye tilføjelser, der altså kræver mange up-votes, før de selv får tilføjelsen at se, og mere engagerede brugere kan så have mindre skrappe filtre og dermed blive mere aktivt med til at godkende og afvise nye tilføjelser. Og angående tillid ift.\ bruger-ratings, så kan man sikkert komme rigtig langt i starten ved bare at antage, at alle konti repræsenterer en unik bruger, og så snart siden vokser bare lit, så kan man begynde at implementere et friend-of-a-friend-system, således at man kan begynde at skille bots fra. Så jeg tror dermed faktisk ikke, at dette bliver så svært, og jeg tror således heller ikke, at mine simple, token-baserede ``brugergrupper'' beskrevet ovenfor bliver særligt nødvendige.\,. .\,.\,Og det er rart at man applikationen ikke er tvunget til at vante på, at brugerne for oprettede sådanne halvavancerede systemer (som kræver noget brugerengagement) til at uddele trust, før at applikationen kan blive god og gavnlig. 



\section{Overall selling points for my Web 2.0--3.0 ideas}

(13:19, 24.01.23) This section is best understood if one has read my previous notes on the subject, including the ones above in the previous section (in Danish). But the notes might make some sense still, even without having read my previous notes.

%(13:22) Jeg tager lige en hurtig pause og tænker lidt over, hvad jeg vil sige i denne sektion... ...(13:52) Okay, lad mig prøve at fortsætte.

One little idea, which can be implemented on any conventional Web 2.0 site as well, is my idea for ``rating folksonomies.'' The idea is basically to attach a rating bar to all tags. Even if the site does nothing further about this technology, it will still be nice for users to be able to see a rating score along all the tags that they observe on the site. On from there it is a very short step to start using this extra data to make searching better on the site, and to make feed algorithms better. Thus, it is a very simple, and I believe very useful, idea that is easy to get going with.

The next idea is to have user-driven search filter / feed algorithms. This idea is more complicated, but I have described how it could work in the notes of the previous section. The big selling points here, are that users will then be able to not just get a single feed algorithm (as well as some very simple search filters) that depends on the user's data, but can always choose from an array of algorithms. This gives the users much more freedom of customization on the site, and in a way such that they can shift between different customizations depending on what they are interested at in the moment. Now, if we think about YouTube in its current state, this is a very good example of a quite terrible user experience with the feed algorithm (in my opinion as well as some other's). I am actually afraid to click on anything that I don't recognize, since I am afraid that that will then trigger a whole bunch of stupid recommendation in the future, where I have to repeatedly click `not interested' in order to get that ``contamination'' out of my feed again. Such ridiculous situations will all be past with a more user-driven system such as the one that I imagine. Here users will instead be able to control their own algorithms quite effectively, and will be able to use a wide array of different sittings depending on what the are interested in at the moment.

User-driven algorithms also means that the users don't even have to feed the machine a lot of their personal data in order to get the feed/search result that they desire. They can instead stand on the back of other users' commitment and simply use their preferences. A
Or they can keep their own data completely private and still use it to find out about, where they lie in the general space of user preferences. They can also at anytime open up completely anonymous accounts and import preferences from other accounts (without having to reveal this data to the public). The users will thus be able to get much more freedom in how the use their data, and will have much more control and ownership over it.

And continuing on the topic of feed/search algorithms, I certainly believe that these will get a lot better for the individual user when there is a large open source community behind them (and when the user has so much ability to use a wide range of settings). This is finally a point where the open source part of the idea starts to get important. Once suc a site takes off, the sheer number of users who want to contribute to better making better search/feed algorithms will make the possibilities far exceed what any private team of web developers can muster for their users. This is just my belief, but yeah, I believe it quite strongly (perhaps with the only exception if AI is going to come in and help provide super good and varied feed algorithms somehow.\,.). .\,.\,But yeah, most likely, a big open source community around such an open source site as what I have in mind will be able to achieve much better things than what we know currently.

Another idea where the open source part is really important, is about having the users be able to change the appearance of the site themselves. Again, we are then talking about a case where the site has already taken off and has gained a very big user network around it. One it has this, this open source community will be able to achieve much more in terms of the usability and nt least the ability for customization than what a limited team of web developers can achieve. And here I should mention that part of my ideas related to this subject is that users are able to choose different settings for different types of resources. Each users can thus customize the appearance of the various interfaces on the site in a modular way. (See my previous notes for more details.)

Moving on, another big selling point about my ideas has to do with having a semantic (I guess) structure of all the resources contained (and referenced) on the site such that all resources can be found in a tree (or graph, rather) structure of categories and subcategories and so on. Hm, well perhaps `semantic' is actually not the right term to use here.\,. %(14:35)
.\,.\,Hm well, yes, I guess it is, cause that is essentially what I achieve with having users being able to also choose more and more filter predicates at the same time as they browse the category tree, such that the end up choosing both a subcategory for the resource they are interested in as well as a bunch of predicates which can specify the resource they are interested in further. %(14:39) ..Går lige en lille tur, før jeg fortsætter. (14:46)

%(16:06)
If keyword searches was never invented, such semantic catalogs would probably have been how we would have ordered resources on the web. Such a resource graph might then have been centralized to begin with, but at one point, an open and user-driven implementation would have become popular. I am glad we keyword searches was invested, but I really think we have been missing out on something big all this time. I really think that such user-driven resource graphs over the contents of the web would have been such a useful thin to have. Whenever a website has content added to it, the author of that content would have wanted it to add a reference to this content in whatever semantic ressource graph would be popular at the time, such that more people would see that content. And since the Web of Trust ideas are quite similar to my ``(user-driven) user groups'' in a lot of ways, I am pretty sure that we would have also quickly implemented ways for users to choose different ways of applying trust when it comes to rating useful contributions to said semantic resource graph. 

Anyway, I think that having such (user-driven) structured graphs over all the content on the web will be such a useful thing to get in the future, and I see (as I have described in the previous section) how I site such as the one I have in mind could be one way to get there. So to add a selling point to the list: Having such a semantic structure of content like the one I have described in the previous section could become a massive thing in the future. 

And the last point I want to mention, which is related to the last point, is that I think it will also be very useful to have content and data related to any specific resource ordered with a similar semantic structure. Having ordered comment sections would thus potentially be quite a nice thing to have, and on top of this, there is also related data/content such as related links, annotations, source material, etc., which could benefit from a similar user-driven, semantic structure. (See my previous notes for more details.)

So that concludes the list of selling points that I have in mind. The first ones are something that a regular Web 2.0 site might also be able to implement to some degree. This is kinda unfortunate for someone like me who thinks that the open source way is the ultimate way to go, for if these selling points could also only be realized on a much more user-driven site, it would sell the idea of such user-driven sites all the more. However, my hope is that other people will see the big potential in starting up an open source site which sets out to realize more and more of all these mentioned ideas, and that we can thus get a big community going around starting up such a site. This community will then consist of mainly open source programmers initially, and there will be a time where the site has yet to attract users from the general public, but once some of the first points mentioned in this list of ideas starts to become realized, users will slowly migrate from other Web 2.0 sites and start using this open source version more and more. The following network effect will then mean (as I foresee it) that some of the last points on this list of ideas will start to become realized, and from there, the sky is the limit. %(16:32) 

(13:22, 26.01.23) Let me also quickly mention the point that I think my system where users rate things by moving them around in a list will be good at encouraging users to give a lot of (valuable) rating data to the system. 

(13:38) In accordance with what I have just added in the previous section, there is now also the selling point, that the application will be able to be used on all kinds of other websites as well, namely such that users can see semantic data about resources they are viewing while browsing other sites. 

(15:10) And just to make clear, there is also another great point, which might not be so easy to ``sell'' since it is hard to argue that things will go according to how I imagine them, but which is really the big underlying reason why I'm so interested in all this. The point is that I believe that this technology can get us to a point where all of science can also be structured in a great semantically linked graph such that is becomes easy to look at all point and counterpoints to a given question, and to look at all existing solutions to a problem (and see arguments for their benefits and drawbacks). The same can also be said for open source programming: I believe we can get to a point where all programming solutions (modular) can be ordered in a great semantically linked graph. I believe that my ``Web 2.1'' ideas here, as we can call them, potentially might be able to bring about such a future, and I really think that this will mean so much for our scientific (and societal) advancement.\,.\,! %(Let me by the way mention here in the comments that I have thought about this today and reconsidered if I still really believe that my Web 2.1 ideas can lead to this, and luckily I have sort of arrived at the point where I think I will double down on that belief. For the way I see it, having a semantic graph over web content can very well become very popular, and this might very well further lead to the scientific --- and open source programming --- community/ties also making use of this technology to structure all scientific knowledge and discussion (each individual scientist (or programmer or amateur) taking part partly of selfish reasons to make their work reach a larger audience). And once such a well-structured graph becomes a reality, I believe this will... Hm, let me actually write this in the rendered text instead.. )
Let me by the way mention that I have thought about this today and reconsidered if I still really believe that my Web 2.1 ideas can lead to this, and luckily I have sort of arrived at the point where I think I will double down on that belief. For the way I see it, having a semantic graph over web content can very well become very popular, and this might very well further lead to the scientific --- and open source programming --- community/ties also making use of this technology to structure all scientific knowledge and discussion (each individual scientist (or programmer or amateur) taking part partly of selfish reasons to make their work reach a larger audience). And once such a well-structured graph becomes a reality, I believe this will greatly increase people's --- scientists/programmers as well as all other people --- ability to look up specific knowledge and to engage in discussions and innovation/solution-finding processes. I thus see that this technology can maybe sort of create a giant online collective intelligence --- not an artificial intelligence, but metaphorically speaking still a big collective brain. These are large words, but I really do think that such technology will give us intellectual powers as a civilization that is many times greater than what we have now. Anyway, I hope so. Hm, I guess that this paragraph belongs more in section \ref{Some_hopes_in_terms_of_my_ideas} below than here, but let me keep it here and simply copy (not cut) and paste it below there as well.\,.




\section{Flere tanker om min semantik-applikation}

(30.01.23, 9:51) Jeg var lidt begyndt at second guess'e, hvor stor en gennemslagskraft min hjemmeside/applikation vil kunne have, men nu føler jeg faktisk virkeligt, at jeg har fundet den røde tråd igen, som gør, at jeg igen virkeligt tror, at det vil kunne bliv kæmpe stort, og på ikke så lang tid endda. 

En stor pointe er, at nu har jeg godt nok snakket med om et prædikat såsom `er\_sjov,' og sådanne prædikater er også vigtige, nemlig prædikater der bruges mere til bedømmelse af en ressource frem for en kategorisering af den. Et relateret prædikat til er\_sjov, der i stedet bruges til bedømmelse, kunne være er\_komedie, når vi snakker film, og ellers kunne det være noget såsom har\_humor\_som\_et\_vigtigt\_fokus. Der vil så generelt være meget mere tværgående enighed omkring sidstnævnte typer af prædikater, hvor bedømmelserne af førstnævnte vil afhænge meget mere af personlige holdninger --- og dermed altså også hvilken `brugergruppe,' man ``spørger'' i forbindelse med diverse filterindstillinger. 

Nå, den store pointe er så, at kategoriseringsprædikater vil være mindst ligeså vigtige for brugerne, og jeg tror på, at disse i høj grad også vil være villige til at bedømme kategorier, som vi kan kalde det, frem for bare at bedømme, hvor godt ressourcen lever op til sine kategorier, osv. Og på den måde, så bliver applikationens brugbarhed altså set ikke så todelt, som jeg egentligt lidt har gået og tænkt på det sidste, for det vil i stedet være sådan, at hele den del af applikationen, der handler om at kategorisere ressourcer, i høj grad også vil hænge sammen med tag-rating-delen. Og dermed tror jeg altså lynhurtigt, man vil få brugerne godt i gang med at tilføje og bedømme kategorier. 

For når man som bruger af gængse hjemmesider afgiver bedømmelser, så er det ofte i høj grad for at støtte skaberen og ikke mindst hjælpe til at andre med samme interesser kan finde frem til det samme. Og dette vil kategori-rating tags jo lige netop kunne hjælpe gevaldigt med. Jeg er altså ret overbevist om, at brugerne i høj grad vil benytte sig af disse.

Og oveni, hvis vi går tilbage og ser på, hvad der generelt vil få applikationen til at blive en stor succes efter min mening, så vil applikationen jo hurtigt kunne bruges vildt bredt. Vi snakker jo således slet ikke bare film og videoer, men også tekster (bl.a.\ fra Wikipedia, men også fra alle mulige andre steder), varer af alverdens afskygninger, bøger, spil og alverdens andre ting omkring fritidsinteresser. Og ja, jeg tror altså nu, at der ikke vil gå særligt lang tid, før at brugerne får udbygget en omfattende kategoriserings-semantik-graf over alle sådanne ting. .\,.\,Nå ja, og i øvrigt forestiller jeg mig også, at politiske holdninger og andre meninger også kunne blive en vigtig type ressource. Altså tekster, der opsummerer en eller anden form for mening om noget, og som folk så kan bedømme efter enighed. Dette kan blive en rigtig god måde for folk at udtrykke sig på (politisk og i andre sammenhænge), f.eks.\ hvis nu de hører i fjernsynet eller i radioen om, at ``der har været en stor bevægelse/underskriftindsamling/shitstorm / et stort backlash/.\,.\,you name it.\,. hvor de selv føler, at de ikke selv er repræsenteret i disse reaktioner. Så kan det være rart at kunne gå ind at give sin mening til kende ved at stemme på de relaterede `meninger' på hjemmesiden/applikationen, og også så at kunne se et mere klart billede af, hvor mange mener det ene og hvor mange mener det andet. Og ja, det kan det selvfølgelig også i mange, mange andre sammenhænge (og også i tilfælde, hvor `meningen' ikke er vildt aktuel i nyhederne, men det er den måske for brugeren selv). Men ja, og hvis vi går et skridt tilbage til at kategorisere (og bedømme!) varer *(og servicer forresten) af alverdens typer, så vil det også blive en kæmpe stor ting. Hvis man f.eks.\ ser på Trust Pilot, så vil den hjemmeside/applikations muligheder være vand ift., hvad man kan på min semantik-hjemmeside. For det første vil man kunne kategorisere alle varer i en semantisk grafstruktur, så det er let at finde frem til, hvad man leder efter, og let at sammenligne, og for det andet vil man så også have mange flere parametre, man kan bedømme varerne (og servicerne) ift.\ (og se bedømmelserne for). 

Og ja, angående vidensressourcer, så tror jeg nu også, at applikation lynhurtigt vil blive super gavnlig, og at brugere således vil kunne bruge den til meget nemmere at finde frem til tekster omkring ligepræcis det, de er interesserede i. 

En lille teknisk tilføjelse omkring videns-ressourcerne, hvilket jo bare er tekster, eventuelt hypertekster, plus kilde-URL'en, jamen så ville jeg vælge bare at gemme (hyper)teksterne direkte i databasen fra start af (i modsætning til tungere ting såsom videoer og billeder). Hvis så kildesiden ændrer struktur en anelse, eller hvis selve teksten bliver skrevet en anelse om, så må man alligevel kunne finde tilbage til det eksakte sted, hvor teksten stammer fra, nemlig ved bare at lave en automatisk søgning og finde det bedste match til teksten på kildesiden. *(Jeg glemte lige at nævne her, at man dog nok ikke bør lade brugerne se disse tekster direkte, da de jo let kan være copyright'ede. .\,.\,Hm, men vent, for vil det så egentligt være særligt smart sådan? Måske på sigt, men det vil jo kræve ret meget programmering, måske man kan finde på noget smartere.\,.\,? .\,.\,Ah, mon ikke hvis man bare gemmer de første par sætninger i teksten i stedet (foruden overskriften, hvis der er en), så kan man vel nok ikke komme i klemme. .\,.\,Nej, det skulle undre mig meget, hvis man kan det.)

Men ja, så det korte af det lange er altså, at jeg virkeligt tror, at hjemmesiden/applikati-onen ift.\ hver enkelt type ressource vil gå fra 0 til `nu er siden brugbar' til 100 på virkeligt kort tid, for når først den semantiske struktur er brugbar for en del brugere, så tror jeg udviklingen vil accelerere helt vildt derfra, nemlig fordi flere og flere brugere vil komme til, og disse vil føje flere og flere ressourcer og bedømmelser (inklusiv kategori-bedømmelser) til. (10:43)


(10:54) Jeg har også tænkt på noget andet, som jeg passende lige kan nævne her, inden jeg går videre, og det er, at følgende donationsidé nok virkeligt også vil være gavnlig, nemlig at brugerne i fællesskab (og hver for sig og, ikke mindst, i grupper) kan udarbejde donationsfordelingsopskrifter, som så altså er opskrifter/planer, der bestemmer, hvordan penge doneret til opskriften/planen skal fordeles mellem skabere/bidragsydere (og altså ikke bare skabere, der har uploadet ressourcer specifikt til webapplikationen; det kan også være skabere, som ikke selv har tilføjet deres indhold specifikt til webapplikationen (men til en anden hjemmeside/applikation), men som dog har registreret sig selv og tilføjet en konto man kan donere til). Hver doner kan så selv vælge, hvilken plan/opskrift, de vil bruge. I princippet kan de så vælge en opskrift, der bare giver pengene tilbage til den selv (ved også at registrere sig selv som en skaber), men hele pointen er så, at andre brugere så også kan se, at du ikke er donor til en af de mest populære planer/opskrifter, men i stedet er donor til en måske meget obskur opskrift. Dermed vil andre brugere altså ikke have nær så stor tendens til at bruge dig særligt meget i diverse filteralgoritmer, mere end hvordan de bruger ikke-donerende brugere. (Og det er nemlig en rigtig gavnlig ting, at brugere kan vægte filteralgoritmer ud fra, hvor meget brugere har givet til visse populære og fornuftige donationsopskrifter. 

En lille anden tanke omkring bruger--skaber-økonomien, som jeg lige kom på nu her (de fleste andre idéer jeg nedskriver nu her stammer ellers fra weekenden, hvor jeg var på Fyn), er at man i stedet for at tænke på at starte en SRC over webapplikationsvirksomheden kunne starte den over en IP-fællespulje i stedet. Her har jeg jo før tænkt, at en sådan IP-organisation kunne indgå som et modul i den samlede SRC omkring en sådan hjemmeside/webapplikation, men nu tænker jeg altså: Hvorfor ikke bare nøjes med at foreslå, at skaberne kan gå sammen (efterfølgende) og oprette en SRC til at forene sig omkring deres samlede IP-rettigheder? Så det tror jeg faktisk er min plan nu: Jeg vil oprette en virksomhed omkring denne semantik-webapplikation, som altså skal stå for de grundlæggende ting, ikke mindst at passe databasen (og som tiltrækker penge ved at vise reklamer (muligvis hvor brugerne selv kan slå dem fra (altså uden at bruge en add blocker)) og så ikke mindst via donationer, og som i øvrigt ikke prøver på at ekspandere særligt meget), og så er tanken, at diverse skabere (indholdsskabere og ``ramme-skabere,'' som jeg har kaldt det) selv kan oprette en SRC omkring en sammensat IP-pulje, hvis de vil. 

Ok, og har jeg andet, jeg vil nævne, inden jeg går videre til mine nye idéer omkring webapplikationens interface og alt det.\,.\,? (11:22)


(13:25) Okay, jeg tillod lige mig selv at tænke noget mere over det hele. Lad mig starte med det vigtigste først. Nu tænker jeg ikke længere at fokusere på den der fold-ud-menu, jeg har snakket om ovenfor, ikke i starten i hvert fald. %Lad mig prøve at opsummere, hvad jeg tænker, man bør prøve at konstruere i starten i stedet for.
I stedet forestiller jeg mig, at det bare er helt standard, at hver ressource i en liste får en lille liste af muligheder, ja, som essentielt set så vil være list-prædikater, der hver er dannet af en relation (2 inputs), hvor pågældende ressource automatisk er sat ind som det ene input. Denne lille liste af muligheder må så gerne kunne afhænge af ressource-typen, men er ellers rimeligt konstant. Det vil sige, at brugeren derfor i princippet vælger en liste for hver ressourcetype (f.eks.\ `kategori,' ress.\,.).\,. Nå nej, lad mig kalde det et term i stedet for en ressource, og så kan jeg bruge ressource som en over-termtype, der indeholder (ressource-)termer såsom `videoressourcer,' tekstressourcer,' `meninger' (som jeg har skrevet om ovenfor), `varer' osv. Okay, og for at fortsætte, så vælger brugeren altså derfor i princippet en liste for hver termtype, nemlig i form af en lille liste af relationer. Disse relationer kan så f.eks.\ være `er\_overkategori\_til\_x' (hvor x så er den automatisk indsatte term), `er\_underkategori\_til\_x,' `hører\_under\_kategorien\_x,' `er\_et\_relateret\_link\_til\_x,' `er\_en\_relateret\_ressource\_til\_x,' osv.\ (hvor disse nævnte relationer så ikke alle vil være relevante for alle termtyper (de to første vil nemlig nok bare være relevante kategorier, hvorimod de tre sidste vil være relevante for diverse ressource-termtyper)). 

Selvfølgelig vil de fleste brugere i starten bare bruge de lister her, som webudviklerne (som jeg så forestiller mig at være en del af) vælger (og justerer løbende), men med tiden vil brugerne selv tage over og gøre det for hinanden, som de vil have det (men her snakker vi altså på langt længere sigt). 

.\,.\,Nå ja, men brugerne kan dog fra starten altså vælge imellem nogle forskellige muligheder for hver termtype, og selv oprette nye liste for hver af disse, hvor de selv tilføjer eller fjerne relationer herfra (ift.\ til standardmulighederne). *(Og de skal selvfølgelig også gerne selv kunne vælge brugervægtningen ift.\ ratingen af relationerne i listen.)

Når brugerne så iagttager en given liste over termer, der er blevet bedømt positivt i forhold til et givent prædikat, så kan de så ved at klikke på en term folde denne lidt ud for at få nogle valgmuligheder med den, og her skal den nævnte lille liste altså så også findes. Her kan brugerne så skifte imellem de forskellige valgmuligheder i listen, hvilket så bestemmer, hvilken liste af termer, som vises hvis brugeren vælger at folde en ny liste ud for det givne term. Jeg forestiller mig så, at brugeren kan folde en ny liste ud på to forskellige måder, via to forskellige tilhørende knapper: èn knap med en pil nedad og én knap med en pil til højre. Hvis brugeren trykker på knappen med pilen pegende nedad, vil en liste foldes ud under termet selv (muligvis med et lille indryk), imellem dette og så den næste term i listen. I øvrigt foldes listen kun lidt ud med de øverste fem-ti valgmuligheder til at starte med, og ved endnu et klik på en nedadpil i bunden kan brugeren så få endnu flere valgmuligheder i listen. Alternativt kan brugeren klikke på pilen pegende mod højre, hvorved samme liste så bliver udfoldet i større format i en kolonne ved siden af den forrige kolonne, hvor det givne term fandtes i. Brugeren kan så også skifte frit imellem valgmulighederne over relationer i den lille liste, jeg har snakket om her, hvorved indholdet i listen, hvad end den er foldet ud under termen eller i en kolonne til højre for den forrige, skifter til den nye relation.

Når brugeren trykker på et hvilket som helst givent term, skal der også enten foldes en rating bar (eller to.\,.) ud til at starte med, eller også skal der være en knap til at folde ratingen/erne ud. For alle termer i en liste skal der så i første omgang være en rating bar, som lige præcis handler om prædikatet om, hvor godt pågældende term passer til den liste, den vises i. Eller med andre ord, for enhver liste afhænger jo af et specifikt prædikat, så skal man altså kunne rate termen i forhold til lige netop dette prædikat. .\,.\,Hm, for prædikat-termer skal man så kunne gøre noget mere, men lad mig lige tænke lidt mere, inden jeg fortsætter omkring dette.\,. (14:09)

(14:32) Okay, når det kommer til prædikater, så skal der også være endnu en rating bar, nemlig hvor brugeren kan rate andre termer ud fra pågældende prædikatet (altså i stedet for at rate pågældende prædikat-term ud fra det prædikat, der danner den liste, som pågældende prædikat-term vises i). Og her skal brugeren så kunne skifte subjekttermen til denne rating bar. Titlen (og måske et tilhørende billede/ikon på længere sigt) kan så vises over eller ved siden af denne rating bar, og ved at klikke på en knap nær ved denne titel, skal brugen så kunne cykle imellem alle de aktuelle\,.\,. hm, de aktuelle termer, men det vil jo i høj grad være ressource-termer, vi taler om her, ikke.\,.\,? Jo.\,. .\,.\,Ja, ok, så i første omgang (som noget der vises allerførst i listen over termer, som brugeren kan rate på denne måde --- hvis altså ikke det bare \emph{kun} er ressource-termer, som brugeren skal kunne rate på denne måde, det kan faktisk godt være.\,.) skal alle de aktuelle ressource-termer altså stå i listen over, hvad brugeren kan rate her. Og disse ressource-termer er så altså nærmere bestemt alle de ressourcer, som brugeren har klikket på for nylig. 

For hver eneste rating bar, der vises, hvad så vi snakker den ene eller den anden af de to nævne ratingbarer (hvor den anden kun giver mening, når pågældende term selv er et prædikat), så skal der kunne udfoldes en lille beskrivelse af, hvordan det relevante prædikat (som rates ift.) er defineret mere eksakt (end bare ved at læse dens titel), og når brugeren peger forskellige steder på ratingbaren, så skal der også gerne vises en lille tekst til det specifikke interval, som brugeren holder over, hvor der given en forklaring på, hvordan en rating i dette interval generelt bør fortolkes (ifølge forfatteren). Det kan godt være, at disse tekster vises, når brugeren indstiller knappen på ratingbaren, måske særligt hvis vi snakker et mobile device, men der skal så selvfølgelig altid være en bekræftelse af hver afgivne rating (og hver ændring af en tidligere afgiven rating), hvor brugeren skal trykke bekræft eller annuller. (14:47)

Okay, jeg har det som om, jeg mangler at nævne noget, men lad mig bare lige gå lidt videre for nu. Hvis vi tænker på de filter-indstillinger, jeg har snakket om ovenfor, så kan dette fungere ret meget ligesom, jeg har beskrevet før, måske endda med en fold-ud-menu fra venstre. Jeg forestiller mig dog, at brugeren skal gå til hjemmesiden for applikationen selv for at lave nye filterindstillinger, og at brugeren defor bare eventuelt har en række yndlingsfiltre, som denne kan vælge fra, hvis brugeren tilgår applikationen via et overlay på en anden hjemmeside.\,.

.\,.\,For jeg har jo allerede nævnt ovenfor, at jeg nu forestiller mig, at applikationen ikke bare skal fungere på sin egen hjemmeside, men at brugerne også kan tilgå den, når de browser ressourcer på andre hjemmesider. Her forestiller jeg mig så, at dette overlay (eller hvad man kalder det) skal kunne rulles ind fra højre, hvis brugeren trykker på en knap, og i øvrigt at denne knap skifter udseende, hvis den ressource, som brugeren betragter på den anden hjemmeside allerede er tilføjet til databasen (altså den semantiske), og hvis der så findes relevant data (så som ratings og relevante links m.m.) til ressourcen i databasen. Men jeg forestiller mig dog lidt, at overlay-applikationen kan være en tand mere simpel, end hvad man kan tilgå på webapplikationens egen hjemmeside. Brugeren skal dog selvfølgelig stadig være logget ind i overlay-applikationen og hermed have adgang til de relation-liste-indstillinger m.m., som jeg har nævnt nu her i dag, og jeg synes også nok gerne, at brugeren må kunne folde underlister ud til højre for forrige kolonne i en ny kolonne ved at trykke på pilen til højre, som beskrevet her ovenfor, men ud over disse ting, så kan det dog godt være, at overlayet bare skal være en tand mere begrænset. Men til gengæld bliver brugeren så bare simpelthen dirigeret om til applikationens egen hjemmeside, hvis denne trykker på en valgmulighed, som ikke er implementeret i overlayet, og så skal brugerens nuværende position i hele semantik-grafen, samt de ressourcer brugeren har valgt, bare overføres, så brugeren starter med helt det samme sted, som denne kom fra i overlayet (nu bare med nogle flere valgmuligheder). Og i øvrigt skal denne omdirigering altså ske via åbning af et nyt vindue, så brugeren ikke forlader den side, han/hun var på. 

Så lad os forestille os, at brugeren betragter en ressource på en vis hjemmeside, og nu ser ude i siden, at der findes data til denne ressource i semantik-databasen. Brugeren kan så folde overlayet som en menu fra højre. Her skal brugeren så.\,. Nå ja, det har jeg ikke nævnt! Jeg vil gerne have.\,. Hm.\,. .\,.\,Jo okay, jeg vil også gerne have en standardvisning for en given ressource, hvor alle relationsmulighederne vises som overskrifter i en sammensat liste, hvor der så kun lige vises de allermest populære termer i hver liste, men hvor man så også får mulighed for at folde disse lister mere ud, både ved at trykke pil nedad eller ved at trykke pil til højre (som beskrevet ovenfor). Hm, vi kan så se denne mulighed som en standardmulighed for hver relationsliste (vi snakker altså denne ``lille liste af muligheder,'' som jeg startede med at forklare om i dag), hvilket altså for lige at gentage det giver en liste, som er sammensat (i rækkefølge) over alle de egentlige relationer, som er i denne liste. Så brugeren kan altså se denne standardliste, hvor alle relation-mulighederne kommer efter hinanden, eller brugeren kan også trykke på en specifik mulighed (ikke i den nye søjle med omtalte standardliste men i den forrige søjle med objekttermen selv), således at hele den nye liste så bare \emph{kun} kommer til at indeholde en specifik mulighed i form af en af de pågældende relationer i relation-listen. (Håber dette giver mening, hvis man læser det et par gange.)

Ok. Brugeren kan herved så trykke overlayet frem fra højre og med det samme se en lille oversigt over de mest relevante ratings og links m.m.\ til pågældende ressource. Og ved bare ét klik mere, nemlig på en nedad-pil, kan brugeren så folde en af disse dellister længere ud og for så at se nogle flere muligheder (i.e.\ en række af de næstmest relevante prædikater/links m.m.). (15:20) .\,.\,Og herfra kan brugeren så endda navigere endnu videre i grafen, navnligt måske hvis denne hurtigt lige vil finde en passende kategori og/eller et passende prædikat som ikke indgår i listen (.\,.\,eller f.eks.\ hvis brugeren lige hurtigt vil tilføje et relevant link). Og hvis brugeren så gerne vil bruge webapplikationen endnu mere end dette, såsom f.eks.\ at betragte ressourcen og/eller andre ressourcer i en liste sorteret ud fra visse filterindstillinger (og alt det jazz), så kan brugeren altid bare klikke på en knap, så applikationens hjemmeside åbner i en ny fane, og hvor brugeren så starter i helt samme tilstand (stort set), som denne var i i overlayet (nemlig de samme sted i grafen og med de samme midlertidigt valgte termer). (15:26)


(02.02.23, 9:32) Jeg har nogle få ændringer. På en måde tror jeg nu, at interfacet skal være lidt en blanding af, hvad jeg skrev om, dengang jeg introducerede og forklarede om fold-ud-menuen (fra venstre) ovenfor, nemlig hvor man har arbejdsbord, og hvor man så føjer flere og flere termer til et arbejdsbord, når man navigerer rundt i grafen, og ja, også så en blanding af det jeg lige har forklaret om, brugeren vælger en lille liste af muligheder for hver term type over, hvad man kan folde ud fra (ved at klikke nedad-pil eller højre-pil) en given term. 

Lige nogle lidt selvstændige tilføjelser, der er rare at få sagt med det samme, inden jeg fortsætter med det mere generelle: Nu forestiller jeg mig, at navigation i den semantiske graf primært kommer til (i det interface, jeg vil sigte mod at bygge) at foregå ved, at man folder flere og flere lister ud, ikke kun af over- og underkategorier, men sådan set af (kategori-)prædikater generelt. Og her skal brugeren så faktisk kunne aktivere flere kategori-prædikater på én gang, nemlig ved at klikke på dem (under navigationen/browsingen) og føje dem til sine aktive (kategori-)(filter-)prædikater. Sådanne prædikater behøver altså slet ikke at være disjunkte, og brugeren kan derfor altså sagtens få brug for at vælge flere på én gang i sin søgning. Jeg forestiller mig, at når brugeren klikker et kategori-prædikat aktivt i sin søgning, så skal den (skifte farve og) highlightes i den pågældende liste, samtidigt med at den også føjes til en speciel mappe i arbejdsbordet. 

Og den anden selvstændige tilføjelse er, at man også skal kunne have filter-prædikater, der har antecedenter foran sig, f.eks.\ til at gøre filtreringen afhængig af termtypen, men antecedenter bør også kunne bestå af eller indeholde andre filter-/kategori-prædikater. Hver bruger kan så endda have en hel lille preamble af faste filter-prædikater (sammensat med diverse antecedenter foran sig), som altid er aktive, når brugeren har gang i en søgning.

Nå, en anden ting er så, at jeg nu tror, at brugerne rigtigt gerne bare vil kunne søge efter nøgleord, når det kommer til bedømmelses-prædikater, som det eksempelvis gerne vil bedømme for en ressource de iagttager (måske på en anden hjemmeside og altså dermed via overlayet (som jeg stadig bare kalder det indtil videre)). Her kunne man sikkert komme langt med gængse søgemaskine-funktionaliteter, men derfor kan man stadig også godt gøre sådan, at brugerne kan tilføje nøgleord til prædikaterne. Så nu forestiller jeg mig altså, at forfatteren i første omgang får lov at tilføje nogle nøgleord, og at andre brugere så ellers også selv kan tilføje flere, samt rate tilføjede nøgleord op og ned. 

Så angående overlayet, så tror jeg bare man i staten kan nøjes med den standardvisning, jeg snakkede om her lige ovenfor i denne sektion, nemlig hvor lidt af alt bliver vist, og hvor brugerne så kan folde listerne mere ud (måske bare via nedad-pil i starten), og så ellers et søgefelt, så brugerne kan søge på prædikater.\,. eller andre termer såsom ressourcer.\,. de ikke finder i denne standardvisning. Hvis det så findes i databasen, kan de så rate det som relevant for pågældende ressource, og ellers kan de gå til hjemmesiden for at tilføje et nyt prædikat eller en ny ressource(-reference) selv. 

\ldots Nå ja, og jeg forestiller mig også nu, at når brugeren folder nye søjler ud via højre-pil, så ryger søjler, der allerede står til højre, ikke væk, men i stedet bliver den nye søjle bare indsat imellem de eksisterende søjler (lige til højre for dens forældersøjle), således at de gamle søjler til højre for bare rykker en tak længere til højre allesammen. Brugere skal så bare selv manuelt lukke de søjler, de ikke længere har behov for. 

Hvis jeg finder på nogle flere tilføjelser, der er relateret til de andre ting i denne sektion, i løbet af de næste par dage, så vender jeg nok tilbage og tilføjer dem her. Og ellers vil jeg nu gå i gang med at planlægge, hvad jeg vil begynde at prøve at programmere for min første prototype af applikationen.

*Hm, man kunne gøre sådan, at hvis brugeren holder shift inde, så vil en ny søjle erstatte den, der tidligere stod til højre, i stedet for bare at blive sat imellem dem.\,.

*Kopieret nedenfra: ``Der skal også gerne være en liste over ressourcer, der konstant ændrer sig, når man vægler flere og flere kategoriseringer til, hvilket vil sige, at vi måske faktisk nærmest skal tilbage til det der med at have navigationen i en fold-ud-menu, eller noget der svarer lidt til.\,. .\,.Hm, lad os sige, at (graf-)søjle-prædikatsøgningen og ressourceliste visningen kan foregå i to faner.\,. som dog godt også kan vises side om side i en tredje fanemulighed.''

*Kopieret nedenfra: ``Okay, jeg tror faktisk, at jeg vil gøre det sådan, at forfatteren til et prædikat faktisk kan tilføje et vilkårligt antal intervalbeskrivelser, endda med vilkårlige og muligvis overlappende intervaller. Disse skal så vises i en liste under ratingbaren, hvor teksten for kun de intervaller, hvor bar-knappen er indenfor, vises. På sigt vil jeg så faktisk også gerne have, at brugeren bare kan trykke på en tekst og se intervallet, og så endda vælge, om ratingen så skal gives ud fra bar-knappens position, eller i stedet bare ud fra den highlightede teksts interval, hvorved ratingen så bare forståes som, at brugerens rating ligger inden for dette interval! Så dette kan altså således også blive den måde, hvorpå brugerne basalt set giver hinanden mulighed for at lave mere diskrete ratings! Virker fornuftigt nok.\,:) Men ja, i første omgang til prototypen skal teksten bare vises, og ratingen skal bare gives ud fra bar-knappens position altid. :) ''


*(03.02.23, 10:34) Jeg skal også på sigt have implementeret et lille aritmetisk sprog (også med sammenligninger, og gerne inklusiv en if-then-(m.m.-)syntaks) til mine ``automatiske point,'' hvilket så må blive en slags ``automatiske prædikater og ($n$-ære) relationer.'' I dette sprog skal brugerne også kunne oprette deres egne funktioner, og ikke mindst skal de kunne tilgå rating distributioner via funktioner, hvor de altså kalder en funktion, der så har værdier ud fra den nuværende rating-distribution af et prædikat eller en relation.\,. ja, nej, eller af et udsagn rettere. Desuden skal brugerne også kunne kalde diverse almindelige descriptors (er det ikke det, det hedder?) (altså parametre så som gennemsnit, skewness osv. *(samt også hvor mange afgivne stemmer ratingen har)) fra nogle andre faste funktioner (hvor udsagnet er et input). ``Returværdien'' af et automatisk prædikat eller en automatisk relation bliver så typisk findes i form af en automatisk rating af pågældende.\,. ja, vi kan jo så passende kalde det et `automatisk udsagn' (altså et automatisk prædikat eller en automatisk relation taget på nogle inputs).\,. hm, hvis vi snakker $n$-ære relationer, så kan jeg jo bare sige at $n$ godt kan være 1, og dermed undgå at sige `prædikat' hele tiden.\,. Nå, men ``returværdien'' er så typisk en beregnet rating for den automatiske relation taget på givent input. Men på sigt kunne man også tillade, at automatiske relationer ligeledes kan give ``returværdier'' i form af input-placeholders, således at den automatiske relation altså ikke behøver at få visse dele af dets input men i stedet selv genererer, hvad dette input skal have af værdi i det endelige (automatiske) udsagn. (Det svarer altså lidt til at give en pointer med til en funktion i C, hvortil en returværdi så i sidste ende overskriver den oprindelige værdi på adressen, rent konceptuelt.) Og ja, så kan man så overveje, om man ligefrem så vil implementere et logisk sprog på baggrund af dette, men fordi brugerne jo allerede har mulighed for at implementere funktioner, så er dette nok ikke nødvendigt, og brugerne kan altså således bare nøjes med at have de automatiske relationer som noget, hvor det endelige returværdier findes, og altså ikke som noget, der kan indgå i selve sproget, der koder for returværdierne. (10:54)

*(10:59) Noget andet ret vigtigt er, at ressourcer i grunden bare skal være et begræsnet antal felter, der definerer ressourcen, nærmere bestemt ikke mht.\ \emph{hvor} den kan findes, men i forhold til hvad der definerer, den ting ressourc(e-referec)en refererer til.\,. Og hvis så duplianter findes på siden, så skal hjemmesiden og brugernetværket altså i reglen sigte efter at slå duplianterne sammen! Så brugerne kan altså hjælpe med at flagge, når der er duplianter (inklusiv når en ressourcetype er duplikeret), og hjemmeside-crewet skal så tjekke dette, og hvis flaggingen er korrekt, så skal duplianterne slås sammen i databasen. Det vil sige, at man ser på den hemmelige info om, hvilke bruger-ID'er ingår i diverse ratings for udsagn, hvor dublianterne indgår (i hver deres udsagn), og så merger databasen disse udsagn, sådan at alle brugere har stemt én gang. Brugere der har stemt på begge ressourcer til samme udsagn (hvilket nok ikke er så mange), de får så en notifikation om, at to af deres tidligere afstemninger er blevet merget til én (måske med gennemsnittet som det endelige svar, hvis det kan lade sig gøre), og kan så eventuelt rette denne rating, hvis de vil. Sådan noget som URL'er til en ressource, jamen det skal så gerne indgå som et 0-til-mange-felt for ressourcen, sådan at brugere altså selv kan føje flere URL'er til. Dette kan så foregå ved, at databasen opretter en ny relation til den pågældende ressourcetype, som så får et passende navn, der indeholder ressourcetypens navn som et slags efternavn, og hvor forfatteren til ressourcetypen så bestemmer fornavnet. Foreksempel kunne relationen komme til at hedde `Movie.hasLocation(),' og inputsne kan så være en URL og en dato for, hvornår URL'en virkede, samt selvfølgelig den pågældende ``Movie.'' Og ja, brugerne kan så efterfølgende rate diverse URL-forslag. Desuden skal der også være et Obsolete-prædikat, som brugerne bør bruge, hvis f.eks.\ en tidligere URL var gyldig (og derfor har en høj gyldighedsrating i udgangspunktet), men at den pludselig er blevet ugyldig. Således kan brugerne altså hurtigt signalere, at en URL er blevet ugyldig, uden at de skal ``kæmpe mod'' den oprindelige gyldighedsrating (som så i stedet bare bør fortolkes som `var URL'en gyldig ved den pågældende dato'), og samtidigt gør dette så også, at man efterfølgende vil kunne se, om en given URL var gyldig (og populær) dengang den blev oprettet (hvilket man ikke kan, hvis brugernetværket skulle ændre den oprindelige gyldighedsrating, nemlig hvis de ikke havde Obsolete-prædikatet). (11:23) .\,.\,Nå ja, og det at flagge dublianter kan (og bør) selvfølgelig også bare ske via en dertil indrettet relation i databasen. 


\section{My first prototype}

(10:30, 02.02.23) Jeg tror jeg vil kalde projektet (og applikationen/hjemmesiden) for SemDB, indtil videre.\,. Hm, lad mig lige hurtigt se, om det er taget.\,. .\,.\,Hm, kunne ikke finde nogen hits, der var særligt relaterede, så lad mig rigtignok bare bruge dette (SemDB) som det midlertidige navn.

For det første kan jeg nævne, at jeg vente med overlayet, så dette bliver altså ikke en del af den allerførste prototype. 

.\,.\,Lad mig også udskyde RegEx-halløjet og i stedet bare selv prøve at populere databasen med nogle eksempelressourcer.\,.

Jeg skal implementere, at man kan oprette sig og logge ind som bruger.\,. .\,.\,Brugere skal kunne tilføje nye prædikater.\,. .\,.\,Brugere skal kunne samle sig et ikke-struktureret (for jeg vil udskude, at de selv kan oprette mapper osv.) arbejdsbord over.\,. Hm, eller skal jeg prøve at føje struktur til.\,.\,? .\,.\,Nej, ikke med det allerførste.\,. De skal så bare have en enkelt liste over prædikater, og én over ressourcer (never mind relationer for nu), hvor de så bare kan fjerne elementer fra listen som den eneste struktur-ændrende handling her. 

Hm, lad mig nøjes med forfatter-tilføjede nøglefraser til prædikaterne.\,. .\,.\,Lad mig også nøjes med en enkelt type rating, nemlig bare den kontinuere type (fra et negativt tal til et positivt), og lad mig vente med at gøre sådan, at forfatterne kan tilføje intervalfortolkningsbeskrivelser. Så prædikaterne har altså bare en titel, nogle nøgleord/-fraser, en beskrivelse, og det er det for nu. .\,.\,(Så eventuelle intervalfortolkningsbeskrivelser føjes altså bare til beskrivelsen her for prototypen.)

I starten kan brugerne bare vælge imellem et lille antal af prædefinerede filterindstillinger for hvert prædikat (og med en knap til at flippe kurven horisontalt, så man ordner fra negativ til positiv i stedet).\,. 

.\,.\,Hm, lad mig lige tænke over, hvor meget jeg vil gøre ud af flersøjle-prædikatsøgningen i starten, før jeg begynder på oerlayet.\,.

I øvrigt skal filterindstillingerne kun foregå i en menu i højre side i starten, så når brugeren ser et prædikat i en liste, skal de altså kun kunne læse om det, og så vælge og tilføje det til arbejdsbordet --- og sikkert også kunne folde en ny søjle (eller underliste) ud fra den, men det vil jeg lige tænke over.\,. 

.\,.\,Jeg vil forresten bare bruge en PHP-server (og med den type SQL-database, der lige hører til den, jeg finder (.\,.\,hm, som sikkert bliver en Apache-server.\,.)). 

Okay, jeg går en tur i solskinnet og tænker videre over, hvor meget af flersøjle-navigationen, jeg skal tilføje her i starten, samt også hvordan brugeren skal finde frem til ressourcetermer og bedømmelsesprædikater.\,. (11:24)

(12:33) Okay, jeg tror vist bare, at jeg for prædikater skal have to muligheder i starten, når det kommer til at udfolde børnelister/-søjler, nemlig `relevante kategoriseringsprædikater' og `relevante bedømmelsesprædikater.' .\,.\,Hm, eller rettere `kategoriseringsprædikater som er relevante til brug for at lave en underinddeling, når givne prædikat er valgt.' .\,. .\,.\,Kortere sagt kunne man bare sige `relevante underkategoriseringsprædikater,' eller endnu kortere: `underkategoriseringer.' 

Der skal også gerne være en liste over ressourcer, der konstant ændrer sig, når man vægler flere og flere kategoriseringer til, hvilket vil sige, at vi måske faktisk nærmest skal tilbage til det der med at have navigationen i en fold-ud-menu, eller noget der svarer lidt til.\,. .\,.Hm, lad os sige, at (graf-)søjle-prædikatsøgningen og ressourceliste visningen kan foregå i to faner.\,. som dog godt også kan vises side om side i en tredje fanemulighed. 

Hm, udover at brugere skal kunne føje prædikater til listen over aktive prædikater, skal brugere så også kunne rate relevans for den pågældende søjle, direkte når brugeren har klikket på et prædikat i en søjle.

Lad mig forresten bare holde mig til søjler og dermed altså udskyde omtalte nedad-pils funktionalitet til et senere tidspunkt (og altså kun have højre-pilen). 

\ldots Ressourcer skal kunne rates efter hver enkelt af de valgte prædikater (som inkluderer de aktive prædikater), og barnesøjle-/fold-ud-mulighederne for ressourcer kan bare være `relevante bedømmelsesprædikater,' `relevante kategorier'.\,. Nå nej, never mind. Begge disse to ting er ikke nødvndige (af hver dere grund).\,. .\,.\,Men `relevante ressourcer' er selvfølgelig en god ting.\,. .\,.\,Nå jo forresten `relevante bedømmelsesprædikater' skal faktisk med.\,. Hm.\,. \ldots Der skal rigtignok være `relevante bedømmelsesprædikater' som en fast relation til ressourcetermer, men det er så bare vigtigt, at man sørger for at brugerne også kan få bedømmelsesprædikat-forslag fra ressourcens kategorier i stedet (altså fra når et prædikat er et `relevante bedømmelsesprædikat' til et kategoriprædikat, og hvor ressourcen så er dømt som inden for den kategori).\,. 


\ldots\ Okay, jeg tror faktisk, at jeg vil gøre det sådan, at forfatteren til et prædikat faktisk kan tilføje et vilkårligt antal intervalbeskrivelser, endda med vilkårlige og muligvis overlappende intervaller. Disse skal så vises i en liste under ratingbaren, hvor teksten for kun de intervaller, hvor bar-knappen er indenfor, vises. På sigt vil jeg så faktisk også gerne have, at brugeren bare kan trykke på en tekst og se intervallet, og så endda vælge, om ratingen så skal gives ud fra bar-knappens position, eller i stedet bare ud fra den highlightede teksts interval, hvorved ratingen så bare forståes som, at brugerens rating ligger inden for dette interval! Så dette kan altså således også blive den måde, hvorpå brugerne basalt set giver hinanden mulighed for at lave mere diskrete ratings! Virker fornuftigt nok.\,:) Men ja, i første omgang til prototypen skal teksten bare vises, og ratingen skal bare gives ud fra bar-knappens position altid. :) (16:14)

Hm, når brugeren tilføjer et prædikat til arbejdsbordet, kan denne tilføje det som aktivt eller ikke aktivt. Jeg tror ikke jeg vil lave to lister til aktive og ikke-aktive prædikater i prototypen (bare have én liste) i arbejdsbordet. I stedet skal de aktive prædikater bare highlightes, og brugeren kan så slå prædikater til og fra her, samt ændre på filterindstillingskurverne, og selvfølgelig også fjerne prædikater fra listen igen som nævnt. 







\chapter{Hopes for the future}

\section{Some hopes for the future in terms of what my ideas can hopefully help bring about}
\label{Some_hopes_in_terms_of_my_ideas}

(16:35, 24.01.23) In terms of my SRC idea, my other economy-related ideas, and my ideas about happiness and local communities.\,. %Oh wait, I have some other stuff that I want to write about my web ideas.. ..Hm, jeg tager lige en kort pause og for samling på de og disse tanker.. ...Oh no, they are actually related to this section.. But.. Hm, let me just mention them in the Web ideas section first.. ..There..
\ldots I really hope that this can lead to a future where people generally are busy with activities/businesses that are much more efficient in adding to their own and other people's happiness and at the same time much more efficient in advancing our technological level. The way I see it, the two things goes quite a lot of hand in hand, cause in my opinion, the way in which we busy ourselves according to the current societal systems are just very wasteful. Most of these activities deals with bettering the lives of, well, consumers, but I believe that if you look at the calculation in terms of how much the activity of each person actually benefit or total happiness, this activity is generally very inefficient. The big trouble is that our society is geared towards an unspoken, unelected ``philosophy'' that consumption brings happiness, when in reality there are much, much, much, much more important things to consider, especially in generally wealthy societies. I thus believe that if we really think about it and start planning our lives and societies better (in a decentralized way, btw), we can achieve much more happiness as people for a fraction of the effort. And this means that we can generally spend much more effort into activities that advance our technological level as well, so its really a win-win; if we optimize our activities in terms of bringing more happiness to people, we can then also spend more energy on activities that advances us as a civilization.

Okay, so that was some very broad strokes in terms of describing what my ideas related to these topics might be able to help achieve, namely without giving any reason for why I believe they can help achieve this. In terms of my economy-related ideas that aims towards less capitalism, I then believe that these ideas can basically help us get out of said unspoken, unelected ``philosophy;'' help us get away from that direction as a society. 

And then there's the idea, which I haven't written about in a long time, about being able to ``pay workers/contributors backwards'' (what I have often called ``bagudbelønning'' in my Danish notes). I think this could bring about a lot of good in the future, but probably in terms of the web more so than anywhere else. I have just written somewhere above that I hope that the scientific community will generally join and take big part in the semantic web at some point in the future. I then also really hope that this will bring scientist and all other people much closer together, with amateurs also being able to take quite a big part in science and knowledge sharing.\,. which they already do a lot, come to think of it. And I hope that a good community around giving donations to helpful contributors, both amateurs as well as professionals (to thus add to their total income), can do a lot of good for the web. Maybe this ``backwards payment'' could even be a big part of getting scientist to join the semantic web in a big way.\,.

And in terms of my happiness ideas, well that goes pretty much without saying: If these ideas can be efficient in bringing happiness to people, this can then save a lot of wasteful and inefficient work/business/activity at bringing consumers happiness, and this saved activity can then be used on other things instead. 

In terms of the ``planning'' of how to change direction as a society, finding what problems to solve, and in terms of finding out what can more efficiently bring people happiness, I think that the semantic (and user-driven!) web can really help with all these things, potentially. One of the reasons for this is of course that I believe that we will be able to discuss matters much better on the semantic web, both in terms of the quality of discussion (because they can be better structured and because more people can be engaged in a single discussion), and also in terms of the quantity of active discussions that we as a civilization will be able to handle at once. But apart from this more trivial point, I also think the future (user-driven) web really can help people find together with other people with similar interests, and can be used much better to find ideas for activities --- and also ideas to structure one's life. Ugly sentence (and I've had a lot of those today), but I hope it makes sense (at least if one has read my previous notes.\,.). Thus, I think that the ideas regarding ``user groups'' and ``user-driven ML'' will make people able to much better find things that interest them, and to be able to much better find people to be friends with. And as I have written about in my earlier notes, I hope that we will get to a point in the future where it will be normal for people to move together in small or larger communities with others that share similar interests, a similar demeanor, a similar approach to life, and so on, and that people will thus end up living much more in communities with exactly people they want to be around, instead of just living more or less with a random sample of society around them, and then having to look for friends other places. 

Alright, this summarizes the some points of what I hope my ideas can help achieve. I know this section, what I have just written, isn't super well-written and easy to understand (without understanding the ideas mentioned already), but I just had to write these thoughts down, at least for my own sake. So here we are.\,.\,:) (17:58)


(15:38, 26.01.23) Copied from above: ``And just to make clear, there is also another great point, which might not be so easy to ``sell'' since it is hard to argue that things will go according to how I imagine them, but which is really the big underlying reason why I'm so interested in all this. The point is that I believe that this technology can get us to a point where all of science can also be structured in a great semantically linked graph such that is becomes easy to look at all point and counterpoints to a given question, and to look at all existing solutions to a problem (and see arguments for their benefits and drawbacks). The same can also be said for open source programming: I believe we can get to a point where all programming solutions (modular) can be ordered in a great semantically linked graph. I believe that my ``Web 2.1'' ideas here, as we can call them, potentially might be able to bring about such a future, and I really think that this will mean so much for our scientific (and societal) advancement.\,.\,! %(Let me by the way mention here in the comments that I have thought about this today and reconsidered if I still really believe that my Web 2.1 ideas can lead to this, and luckily I have sort of arrived at the point where I think I will double down on that belief. For the way I see it, having a semantic graph over web content can very well become very popular, and this might very well further lead to the scientific --- and open source programming --- community/ties also making use of this technology to structure all scientific knowledge and discussion (each individual scientist (or programmer or amateur) taking part partly of selfish reasons to make their work reach a larger audience). And once such a well-structured graph becomes a reality, I believe this will... Hm, let me actually write this in the rendered text instead.. )
Let me by the way mention that I have thought about this today and reconsidered if I still really believe that my Web 2.1 ideas can lead to this, and luckily I have sort of arrived at the point where I think I will double down on that belief. For the way I see it, having a semantic graph over web content can very well become very popular, and this might very well further lead to the scientific --- and open source programming --- community/ties also making use of this technology to structure all scientific knowledge and discussion (each individual scientist (or programmer or amateur) taking part partly of selfish reasons to make their work reach a larger audience). And once such a well-structured graph becomes a reality, I believe this will greatly increase people's --- scientists/programmers as well as all other people --- ability to look up specific knowledge and to engage in discussions and innovation/solution-finding processes. I thus see that this technology can maybe sort of create a giant online collective intelligence --- not an artificial intelligence, but metaphorically speaking still a big collective brain. These are large words, but I really do think that such technology will give us intellectual powers as a civilization that is many times greater than what we have now. Anyway, I hope so.''


(12:13, 27.01.23) Hm, I think that AI and semantic web technology can potentially both be really instrumental in the development of each other: I believe that the development of A.I.\ can be greatly accelerated by having an open ``predictive model,'' as I have called it (or we could it a predictive knowledge/statement graph.\,.), with a lot of active users, and I also really think that AI could help the the development of such graphs a lot, namely since AIs could help generating a lot of these graphs automatically. I really see the potential of a great ``symbiosis'' in this regard.\,. But let me point out, that AIs will not be able to give us such predictive knowledge/statement graphs on their own, since a big part of these are open and free way for each user to implement various algorithms for distributing trust. Furthermore, while future AIs might get the ability to keep an internal ontology over the (conceptual) world, such an ontology might be very unreadable to humans, unless said internal ontology developed ``in symbiosis'' with a human-readable semantically structured knowledge/statement graph. (12:24)



















\chapter{Existence theory}


\section{Empathy utilitarianism}

(19.01.23) Jeg tænkte i går (omkring kl.\ et, var det) for sjov på, at man kunne kalde min etiske lovsætning, som jeg har beskrevet i de udkommenterede noter i Chap.\ \ref{notes_from_2022}, for `empatilitarisme.' Og så kom jeg så efterfølgende til at overveje seriøse bud på et navn, og så kom jeg jo hurtigt på, at man kunne kalde det `empatisme.' Umiddelbart et ret flot og passende navn. Nå, men lidt efter fandt jeg så også på, at et oplagt navn jo ellers vil være `empatiutilitarisme.' Jeg har i øvrigt ikke søgt på, om det allerede eksisterer, det kan også godt være. Men hvis ikke denne etik er kendt allerede, så håber jeg altså på, at jeg kan (være med til at) udbrede den. Og så kunne `empatiutilitarisme' (`empathy utilitarianism' på engelsk) altså være et ret passende navn. (12:39)



(26.01.23, 17:53) Princippet om, at ``alt hvad der kan eksistere, eksisterer,'' er vistnok (allerede eksisterende og) kendt under navnet `the principle of plentitude'.\,.

Det virker i øvrigt på den PBS space video (kendte ikke kanalen før), som jeg lige faldt over (hvor jeg lige har hørt om the principle of plentitude), at MUH gør nogle flere antagelser i sin konventionelle udgave, end den egentligt behøver. Men det kan selvfølgelig også bare være kritikernes overfortolkning af det (det virker som en standard ting i filosofi: Kritikere kan altid bare overfortolke et udsagn eller en teori, og kan dermed så nemt finde en måde at erklære sig dybt uenig med det/den.\,.), det ved jeg jo ikke. Men hvis ikke det bare er en overfortolkning, så kan min, mere generelle, udgave af hypotesen altså helt sikkert være gavnlig. Og selv hvis det bare er en overfortolkning, så er jeg stadig ret overbevist om, at jeg kan hjælpe diskussionen på vej en hel del. (Umiddelbart tror jeg også, at de fleste mennesker, selv fagfolk, har en ret specifik forestilling om, hvad matematik er, hvilket jo så gør det let at overfortolke hypotesen, når man så navngiver den `MUH'.\,.) (27.01.23, 12:02) .\,.\,Det skal faktisk også nævnes, at selv hvis den konventionelle udgave har færre antagelser, end at nævnte video antyder, så er det dog stadig helt sikkert, at min teori er meget mere generel end den konventionelle MUH/CUH, for det virker helt klart til, at denne om ikke andet antager hypotesen om, at der ikke er nogen global tid, og (samtidigt) at den regnemæssige kompleksitet af et univers ikke har noget som helst at sige. (Dette er en fornuftig nok hypotese, men man behøver den ikke; man kan sagtens arbejde med en mere generel mængde af muligheder.)

%Hm, lad mig lige endeligt søge på, hvad der er af grene inden for utilitarisme..
(12:47) For at vende tilbage til `empatiutilitarisme,' så har jeg lige søgt på utilitarisme, og det virker til at den eksisterende idé om `preference utilitarianism' på en måde er ret tæt på mine idéer. Dog synes jeg min version med `empatiutilitarisme' er meget mere elegant, og man behøver ikke at tilføje alle de caveats, som præferenceutilitarismen gør. Og tilmed følger der også en god forklaring med til ``empatiutilitarismen,'' hvilket der ikke rigtigt gør for præferenceutilitarismen, ser det ud til. Og hertil skal det siges, at jeg ikke engang synes det er nødvendigt at antage, at vi selv for all intends and purposes kommer til at leve alle mulige liv igen og igen, før at empatiutilitarismen er begrundet; jeg synes også en etisk grundsætning om at man bør leve som om, at man skal leve alle mulige andre liv, giver ogd mening i sig selv, nemlig fordi den bare er ækvivalent med at sige: ``Sæt ikke din egen oplevelse af lykke og smerte --- og andre følelser --- foran andres (når du skal beslutte, hvad er etisk godt og etisk dårligt i princippet).'' 

I øvrigt så er min udgave af utilitarismen (``empatiutilitarisme'') også meget præcis, når det kommer til, hvilke levende væsener, man bør (og ikke bør) begrænse det til, (nemlig fordi man bør antage, at man skal leve deres liv også (så vidt man tror på, at væsnerne kan have en bevidst oplevelse, og så vidt man tror på, at de kan føle diverse følelser)), og hvordan man skal forholde sig til spørgsmålet om, hvad det betyder at noget tilfældigvis fik et vist udfald frem for et andet. Og min teori behøver heller ikke at snakke om, at nogen personer ikke forstår, hvad der er godt for dem, osv., for ``empatiutilitarismen'' fodrer ikke, at individer, der overvejer, hvad der er etisk godt og dårligt, kan blive enige om det --- ja, faktisk så antager min teori ingen gang at der findes noget ultimativt svar på det!\,. Den siger bare, at hvad person, der stiller sig selv spørgsmålet om, hvad der er etisk godt eller dårligt, i princippet skal lede efter svaret ved at forestille sig, at vedkomne skal leve alle mulige liv (og særligt livene af de personer, der er berørt af vedkomnes handlinger (og hvor man jo gerne vil maksimere den samlede lykke (forventet af individet og dennes evne til at leve sig ind i disse andre menneskers sted) ud fra et statistisk synspunkt --- medmindre, i princippet, at man selv forestiller sig, at man i andre menneskers sko heller ikke ville have lyst til at maksimere lykken fra et statistisk synspunkt, men så er vi også virkeligt langt ude.\,.)). (13:10)



\chapter{Notes from 2022 (out-commented)} \label{notes_from_2022}




\begin{comment}

Disse noter er bare nogle korte ting, som jeg ikke har lyst til at skrive ind i mit nuværende "main-tex"-dokument (altså mit 2021-22-notesæt), og jeg gider heller ikke starte et nyt (2022-xx-)notesæt lige nu, bare for det.. Så dette dokument bliver altså et slags mellemled. (08.07.22, 12:19)



## Tanker fra i morges (08.07.22) omkring bl.a. børneopdragelse, men også meget mere

Jeg tænkte bare lidt på, at der sådan noget som børneopdragelse, og også sådan noget som hvordan man skruer en hverdag og et (sam-)liv sammen som et andet godt eksempel, at der kan jo være rigtigt mange forkellige parametre og stille på: rigtigt mange forskellige tilgange, man kunne eksperimentere med. Og min pointe, jeg har lyst til lige at notere, er, at med mine web 3.0-idéer så kan folk jo på globalt plan diskutere sådanne tilgange, og ikke mindst arbejde på at sætte omtalte parameterrum op --- og her kan man helt sikkert bruge ML som en stor hjælp. Så vi vil altså i fremtiden kunne få et meget bedre overblik over sådan et helt rum af forskellige tilgange. Dette kan man så diskutere omkring og analysere, og bl.a. prøve at gætte på, hvilke parametre, der kunne spille godt sammen, og hvad der kunne passe bedst til forskellige omstændigheder/forudsætninger. Så man vil altså i dette globale netværk meget bedre kunne opstille en masse forskellige muligheder, og derfra bruge dette til at komme med gæt og forudsigelser, som er værd at slå ned på og "undersøge." Og i de to tilfælde, i.e. børneopdragelse, samliv generelt, og også bare >>liv<< generelt, der betyder at "undersøge" jo så, at nogle mennesker og/eller nogle lokalsamfund prøver at teste nogle af disse hypoteser simpelthen ved at udleve dem (i en længere periode i det mindste). Og med sådan et globalt (videns)netværk, så vil man hurtigt kunne opnå det samme, og meget mere endda, end hvis man havde kreative teoretikere (eller hvad man skal kalde sådan en som mig) til selv at udtænke diverse tilgange, der kunne vise sig at bære frugt under diverse forudsætninger. Og en side-konklusion er så derfor også lidt, at selvom jeg tror, dette emne *(omkring børneopdragelse og sådan noget) lige præcis er et, hvor jeg kunne være god, og hvor jeg kunne lægge rigtigt meget arbejde potentielt set (i fremtiden), så vil dette altså være endnu en ting, hvor mine 3.0-web-idéer også bare (formentligt(7, 9, 13) og fohåbentligt!) vil komme og ændre billedet totalt (og i sådan en grad, at der vil være langt mindre behov for enkeltindivider, eller enkelte små grupper, til at designe sæt af gode tilgange fra bunden og op). (12:38)
%*(18.08.22, 13:39) Denne idé/tanke er jo meget en naturlig fortsættelse af mange af mine andre tanker. Det er jo lidt bare, at man kan skabe gode muligheder for diskussioner, og så vil befolkningen meget hurtigt og effektivt kunne udvikle en masse gode nye idéer, gode nok til at de er værd at afprøve. Denne del af det ligger senere end nogle af de andre forestillinger (f.eks. vil disse muligheder sikkert først komme rigtigt en del tid efter, at man får gode muligheder for debatter.. men ja, jeg tror helt sikkert at dette også vil blive et resultat af hele den udvikling på sigt). Og lad mig så også lige nævne en lille ting, som nok egentligt (også) burde stå i en "sektion" med et andet navn: Jeg vil bare gerne lige understrege, at ja, jeg tror virkeligt på, at vi med denne udvikling, som jeg forudsiger, vil blive gode til at diskutere ting godt. Og ja, faktisk tror jeg på, at vi i en ikke al for fjern (faktisk rimelig nær) fremtid også vil blive i stand til som befolkning(er) (globalt og lokalt), virkeligt at få diskuteret grundigt, hvordan f.eks. vore politik skal være (lokalt, men også mere globalt), og i det hele taget hvilke nogle retninger, vi skal bevæge os som samfund, og hvilke mål vi skal betræbe os --- og hvordan vi skal bære os ad med dette. Dette bliver dog ikke en ting vi opnår lige med det samme: Det ligger altså nok som en af de lidt fjernere muligheder. (Der er nemlig mange muligheder, som denne udvikling vil bringe, som vi hurtigt kan få gavn af, og så er der altså også nogen, som nok vil tage længere tid om at komme ordentligt skub i. Og ja, denne sidstnævnte ting er nok ikke en situation, man skal forvente vil opstå med det samme, men jeg tror altså som sagt, at der ikke vil gå mange mange år, før den beskrevne situation bliver en realitet.) (13:50)


## Fortsat omkring diskussioner og videndeling, som det fremtidige internet vil åbne op for

(05.09.22, 20:08) Jeg har sikkert nævnt dette hurtigt et sted i mine 2021--22-noter, men noget andet som virkeligt bare vil blive godt i fremtiden, er når vi kan få bygget en god ontologi / et godt kort over, hvilke personlige problemer og/eller klager og/eller ønsker folk har i samfundet. Det ville hjælpe samfund (altså vores nuværende store samfund, i.e. lande) gevaldigt, hvis forskellige befolkningsgrupper nemt kunne få langt større indblik i, hvordan de andre befolkningsgrupper har det, og hvilke problemer de slås med. Og så vil man jo i det hele taget også bare kunne overveje politiske beslutninger sammen meget mere effektivt, hvis man har tingene (altså ønsker/klager/problemer) kortlagt så godt på den måde. Selvfølgelig kan man komme ud for, at folk smørre tykt på med, hvor store deres problemer er i forhold til andres, men så skal man jo bare lige sørge for, at det hele først bliver diskuteret og analyseret endnu mere (hvor man bl.a. kan tage stikprøver især fra folk der ligger lidt på kanten mellem to befolkningsgrupper (og/eller på anden måde har en position, hvor de har indsigt i, hvad sandheden egentligt er, men ikke har en personlig bias for selv at lyve/smøre tykt på)), inden man begynder at behandle det som fakta, at så og så mange borgere har et så og så stort problem med det og det. Så ja, det kan vi altså også se meget frem til --- det er desværre nok en af de ting, der kommer til at ligge meget sent i hele udviklingen, desværre, men vi skal nok nå dertil på et tidspunkt som (global) civilisation. (20:19)



[...]




## Web 3.0-bevægelse og forretningsidé

(07.08.22, 20:51) Hvis jeg skulle starte en virksomhed for at komme i gang med at opnå de drømme, jeg har om dette emne, så ville jeg fokusere på at starte med at lancere en web 2.0-side, det opfylder kravende fra min "forretningsidé" (om at kunder skal blive til medejere, og at det hele skal gå på omgang osv. osv. (se noterne i main.tex fra i januars og/eller februar, eller hvornår jeg helt præcist skrev dem)), og så vil jeg også virkeligt prøve på hurtigt at få indført, at betalende brugere for stemmemagt over en rigtig stor del af.. ja, af hvad jeg vist har kaldt skaber-aktierne, men nærmere bestemt, så vil jeg sørge for, at disse starter med at udgøre en rigtig stor andel af de samlede kunde/skaber-aktier, og så vil/ville jeg altså sørge for ret hurtigt at få indført, at det er brugerne/kunderne, der har høj stemmemagt over, hvordan diverse skaber-bidrag belønnes (med de skaber-aktier, som virksomheden uanset hvad alligevel er kontraktbunden til at udstede til nogen). (20:59) ..Og ja, så vil jeg bestemt også sørge for, at det hele er open source (og jeg ville bestemt prøve at få eksisterende open source programmører med på bølgen (som "skabere")). ..Og ja, derfra må man sige, at jo hurtigere brugerne kan begynde at føle, at der er flere muligheder på siden, bl.a. ved at der kommer flere og flere (open source) algoritme-muligheder, jo bedre, for det er jo så der, man kan begynde at tiltrække brugere/kundere på baggrund af selve indholdet/rammerne(/mulighederne).. (..og altså ikke bare på baggrund af hele forretnings- og open source-idéen ved det; pga.\ nutidige muligheder for brugerne, og ikke bare på baggrund af fremtidige visioner.) (21:04)

(17:42, 19.09.22) En god måde at starte et web 2.0-til-3.0-firma (med min forretningsidé), kunne bare være at starte et firma og en kickstarter, og så bare love, at alle donationer hurtigst muligt vil blive omdannet til kunde-aktier, så snart papirarbejdet er gjort. Angående iværksætterenes og arbejdernes egen aktie-gevinst, så kunne man jo bare sige, at der lige i starten gælder, at.. Tja, eller man kunne faktisk sørge for hurtigst muligt at brugere kan uploade og stemme om vedtægter på en hjemmeside over, hvordan lønnen skal fordeles. Og så kan der bare være en fast klausul fra starten om, at en vis andel af al denne løn i denne indledende fase skal gå som løn til iværksætterne, og/eller at disse så også for nogle kunde-aktier genereret herved, svarende til en lille procentdel af denne løn. Måske kan man endda også vedtage, at lønmodtagerne i denne indledende fase også skal have nogle procentdele af deres løn i form af kundeaktier. Og i starten vil alt dette så bare baseres på løfter (men hvor iværksætterne muligvis alligevel kan retsforfølges, hvis de bryder disse løfter, fordi de så har handlet falskt og har fået betaling for en vare, som de så har valgt ikke at levere --- hvilket kun er godt, hvis de kan det, også for iværksætterne selv, fordi dette så vil få flere kunder til at stole på opstarten). Men hurtigst muligt skal man altså have udarbejdet kontrakter osv., så man kan gå ind i en ny fase, bl.a. hvor folks stemmeret omkring løn m.m. er mere konkret og detaljeret udarbejdet og sikret.. (17:53) 
%..Og hvad skal firmaet så starte med at lave? Jo, det skal såmen, udover at få styr på kontrakterne til fase 2, planlægge og give løn for programmeringsbidrag til en open source web 2.0-side (gerne én der både kan fungere som YouTube, Twitter og Reddit (m.m.) på én gang (og også gerne Wikipedia, men det kan godt komme lidt senere), men hvor man altså bare kan starte ét sted (f.eks. som en Reddit- eller en Youtube-agtig side)), og så må man så regne med, at tingene bare kan rulle derfra --- det tror \emph{jeg} i hvert fald helt bestemt på, at de kan. ..(For man bevarer jo selvfølgelig bare et system, hvor aktionærerne (og dermed kunderne!) kan stemme på vedtægter omkring lønfordelingen.) :) ..Og så kunne man jo oplagt have endnu en faseovergang efter lidt tid, hvor man også får de sidste ting på plads, bl.a. om hvordan fissioner af firmaet (og måske fussioner med andre, hvis det virker realistisk) skal kunne foregå, plus hvad jeg ellers må have glemt her fra mine noter (i januars/februars)..:) (18:02)
%..(18:07) Hm, og selvom det godt må være open source lige i starten, så kan det godt være, at man hurtigst muligt vil lave et system, så det er rimeligt åbent at se, hvem har gjort hvad, og hvor det måske ikke kræver særligt meget at få adgang til selve kildekoden også, men hvor kildekoden alligevel er eget af firmaet og ikke må tages/stjæles af andre. (18:09)

(21.09.22, 11:09) Jeg kan ikke huske, om jeg har skrevet om dette før, men jeg kom i tanke om i går aftes, at det jo er ret vigtigt, at normale kunder ikke ligestilles med f.eks. andre firmaer. Et firma/"underfirma" skal altså gøre det klart, om dets services er til private kunder eller til andre firmaer (for hver service i det mindste). Man må f.eks. ikke komme ud i en situation, hvor en instans bare kan købe og videresælge produkter, og så få de samme kunde-aktier for det, som de kunder, der køber til eget forbrug. Så derfor skal man altså generelt kun sælge produkter beregnet til eget forbrug eksklusivt til eget forbrug (hvilket i øvrigt sikkert også er meget normalt for firmaer allerede her i nutiden). (11:13)

(27.09.22, 17:47) Okay, jeg har tænkt en hel del mere over forretningsidéen i dag (og også lidt i går aftes), og nu kan jeg se at: Never mind den der forestilling om at lave en kickstarter eller lignende og så forvente, at firmaet så kan brede sig videre og videre derfra. Det er jo for nemt for alle andre firmaer bare at konverterer over, hvilket jo i bund og grund er godt, men det gør jo altså, at der slet ikke bliver noget (BitCoin-agtigt)venture-hype omkring idéen, sådan som jeg ellers kom til at tænke det nu her, hvor jeg er begyndt at tænke over denne idé igen. Så never mind alt det med (som jeg sikkert har nævnt under "Planer" nedenfor) at idéen kan blive den nye "helt store ting" i den forstand. 
Tvært imod vil det nok ikke kunne betale sig at investere helt vildt i normale firmear, der begynder at konvertere til forretningsidéen, medmindre man på en eller anden måde kan mærke, at de konkurrerende firmaer ikke vil have evnen eller viljen til at hoppe med på bølgen, og derfor altså vil blive udkonkurreret (formentligt, hvis man tror på idéen) af det firma, man så investerer i. Så medmindre der kun er nogle gangske få firmaer, der formår at brande sig godt på at være med på den nye bølge, så vil det nok mere bare være en situation, hvor flere og flere firmaer langsomt vil konvertere til de nye forretningsprincipper. 
Der er så også lige den undtagelse, at nogen brancher jo netop kunne få rigtig meget god synergi med denne idé, hvor kunderne/forbrugerne/brugerne kommer til at bestemme meget, og her tænker jeg jo så særligt lige præcis på min idé til en ny web-forretning/bevægelse. Så lige akkurat her vil der altså muligvis være gode investeringsmuligheder, men bid så mærke i, at dette så ikke vil skyldes.. hvad der svarer lidt en pyramide eller boble, hvor de første investorer altså kan tjene kassen på baggrund af, at de kom lidt før de andre. I stedet vil det simpelthen bare skyldes, som jo er normen omkring investeringer, at den nye forretningsløsning har potentiale til at tilfredstille kunderne meget mere --- ikke bare fordi disse også er investorer (og i og med at de så får en pengesum i vente), men altså lige præcis bare i forhold til det produkt de bliver leveret som kunder! Fordi der altså er mulighed for at sådanne hjemmesider m.m. kan komme til at levere et meget bedre produkt (altså bedre funktionalitet, bedre tilpasningsmuligheder, større udvalg og bedre kvalitet af indhold osv.), så vil det altså være værd at investere i, og kun ligesom af den grund.. Tja altså, medmindre selvfølgelig at man også regner med et vist hype omkring det, men det er jeg nu slet ikke sikker på, vil komme, hvis man netop ikke har nogen grund til at tro, at virksomheden vil brede sig til andre brancher derfra (fordi dem med aktiver i forvejen der også bare selv kan joine den nye bevæglese til hver en tid). (18:07)

(02.10.22, 16:52) Det kan faktisk godt være, at der kan lægge en stor investerings/forretningsmulighed i, hvis nogen kan finde på et rigtigt godt brand og en tilhørende rigtig god (offentlig) plan for, hvordan upstarts-firmaet skal være forbrugernes falgskib for at få bragt liv i den nye forretningsbølge: Hvis man kan overbevise en stor gruppe kunder til, at "det er her det sker," og at firmaet er hvad, man bør "investere" i som kunde, hvis man gerne vil sikre sig, at bevægelsen bliver til noget.. Så ja, dette kunne altså potentielt være en mulighed, især hvis man altså kan finde på en godt sted at starte (måske med en supermarkedkæde, eller en eller anden stor og alsidig handels/salgs-forretning), som virkeligt har mulighed for at brede sig meget ud, og som dermed kan vokse sig kæmpe stor, hvis bare alle kunder pludselig begynder at priotere handler med denne i høj grad.
Men ja, dette er nu ikke ligefrem noget jeg forudsiger, bliver en mulighed; jeg siger bare, at der måske kunne være et potentiale. Og ellers så tror jeg altså på, at man hellere skal tænke i firmaer/brancher, hvor firmaet vil have direkte gavn af (ift. det produkt, de ender med at levere!), hvis kunderne kommer til at bestemme mere, og hvis det også er sikret, at det bliver de ved med. 
Og i den forbindelse, så tror jeg altså faktisk på, at dette kunne være tilfældet for nærmest alle brancher, der handler med noget digitalt på en eller anden måde, enten med indhold, film, spil, læsestof, nyheder, bruger-til-bruger-indhold.. you name it.. Alle sådanne brancher, hvor det enten er sådan, at brugere selv i høj grad til at bidrage til værdien af det digitale, man nu end snakker om, og/eller hvis bare vi snakker kreative ting som kan konsumeres digitalt, hvor brugerne samlet set vil drage gavn af, hvis der kommer bedre forhold for skabere/kunstere, og også ikke mindst at alting bliver mere åbent (uden at folk behøver at bekymre sig om, hvis andre stjæler). For ift. sidstnævnte, så tror jeg jo på, at man, ved at kunderne styrer, kan nå en situation, hvor skabere/kunstere kan "bagud-belønnes" for deres arbejde. Og derfor kan alt sådan noget blive meget mere open source. Hvis vi så f.eks. tager spilindustrien (som et rigtigt godt eksempel), så er det ret nemt at se, at den samlede brugerskare kunne drage kæmpe fordel, hvis skabere ikke var nødsaget til at gøre alting så lukket. 
Ok. :) (17:11)

(06.10.22, 9:29) Lad mig lige præcisere noget i den tekst i 21--22-noterne, som jeg skrev d. 12/02-22: Jeg skriver noget med at "opkræve penge" fra aktionærerne. Her mener jeg selvfølgelig ikke, at man sender dem en regning, men altså at man bare nedjusterer det afkast, de har i vente. (Det fremgår sikkert et sted, men nu synes jeg lige, jeg ville kommentere og rette det her.) ...Hm, vi kunne da sagtens snakke 40 år i stedet (20 virker da ikke vildt langt..), appropos samme tekst.. *(Ah, det var for ikke at gøre udsigten for lang til en ægte kd.v., så tja.. ..Ah, men i princippet kan den jo blive "ægte kundedrevet" efter en ret kort periode alligevel, for det handler jo bare om, hvor stor en stemmemagt aktionærerne giver til sig selv i starten og i hvor lang en periode den stemmemagt varer.! Så der er faktisk ingen grund rigtigt til at sætte en kortere kundeaktie-periode.!:) (10:12) ..Nå nej, det passer så ikke helt alligevel, for hvis perioden er for lang, så kan der blive et demografisk (m.m.) skel imellem gamle og nye kunder. Så ja, hvilken periode man skal vælge fra starten er lidt et åbent spørgsmål, men man skal så huske, at denne dog stadig skal justeres (langsomt) løbende, således at den kun er en vis faktor større end, hvad anlagsaktiv-størrelsen som minimum kræver..) ..Og appropos samme tekst, bemærk så at det der med at have en instans, der vurderer firmaets samlede værdi til forskellige tidspunkter (ved at se lidt tilbage i tiden, så måske et år eller to efter), det skal ikke forstås som et væsentligt krav. Det er bare godt at have, bl.a. fordi det altså så gør det mere fair overfor aktionærer, hvis aktier udløber over en periode, hvor firmaet gjorde mange nye investeringer, og at de samlede mere direkte penge-omregnelige aktiver faldt i perioden, på trods af at værdien steg. (Men måske kan sådan en instans også bare se på aktiernes værdi i handler som en god kilde, man lad mig lige genopfriske, hvad jeg endte med at beslutte omkring aktiehandler..) 
...(11:04) Det var måske ikke så tydeligt, da jeg skrev om at opdele virksomheden, sådan at IP(/IM)-skaberne lidt fik deres egen "undervirksomhed".. tja.. Tjo, tja, giver det ikke lidt sig selv, selvom jeg ikke lige fik formuleret det tydeligt? Tanken er bare, at de så kan komme til at tilhøre og sælge deres bidrag til en "undervirksomhed," som så kan have andre "undervirksomheder" som kunder, der så bruger IM-bidragene til at implementere f.eks. en Web 2.x/3.0-side. Ja, det var nok rimeligt selvsagt, men nu har jeg også sagt det her. 
(11:21) Det kan i øvrigt godt være, at jeg her ovenfor på et tidspunkt har glemt lidt igen, at der skal være klare sætninger for, hvem der er de primære kunder (som skal have kundeaktier), og hvem man ellers bare handler med. Men ja, dog kan man jo sagtens starte med at have "donorer" eller "investorer," som altså kun giver rene pengebeløb.. med som kunder.. tja, men det kommer ikke rigtigt til at fungere. ..Nej, i stedet skal sådanne investorer jo bare købe aktier med deres "pengebidrag," hvilke jo så i høj grad naturligvis vil være "start-aktierne," eller hvad jeg nu har kaldt dem (dem til de indledende iværksættere og investorer). 

(11:42, 06.10.22) Jeg bliver nødt til lige at slå følgende fast, for jeg har jo snakket lidt om, her for nyligt, at der "ikke er den helt store investeringsdrøm." Men det passer ikke, eller rettere: sætningen skal i hvert i så fald bare forstås relativt til, hvis nu situationen var, at en enkelt kundedrevet virksomhed ville kunne udbrede sig til det meste af markedet. Jeg siger som sagt ikke, at der ikke er en vis sandsynlighed for, at ikke-så-web-baserede kd.v.'er kan udbrede sig rigtigt meget, men det er nok lidt for stor en drøm at forvente, at en sådan kan udbrede til stor mængde af markede, for som sagt kan andre firmaer jo altid bare følge trop.. Tjo tja.. Whatever, det giver ikke mening for mig at sidde her og prøve at forudsige den ene eller den anden vej på det punkt. Det jeg i stedet ville nævne var bare, at man jo (og det havde jeg måske kortvarigt glemt ovenfor, det ved jeg ikke..) skal huske, at de web-relaterede bracher jo også er \emph{kæmpe} store i sig selv. Så never mind, "at der ikke er en stor investeringsdrøm i det," for hvis jeg har ret, og at mine idéer omkring en stor web 2.0--3.0-virksomhed virkeligt vil kunne udkonkurrere gængse web-virksomheder, jamen så vil der jo potentielt set være en kæmpe investeringsdrøm i det, det er klart. Selvfølgelig er intet sikkert, hvorfor det er vigtigt at understrege 'potentielt' i den sætning. Men ja, følte bare, det var ret vigtigt lige at pointere. :) Det ville være lidt ærgerligt, hvis jeg unødvendigt kom til at ende på, at "der ikke er så stor en investeringsdrøm i idéen." Det kan der jo nemlig selvsagt meget vel gå hen og blive. :)
Lad mig også bare lige gentage, at hvis nu det var mig, der skulle starte sådan en "kundedrevet" web 2.0--3.0-virksomhed, så ville jeg altså virkeligt prøve at gøre virksomheden tiltrækkende for ""open source"-programmører" og andre skabere som muligt, nemlig ved meget hurtigt at prøve at implementere et "bagud-belønning"-system, rigtigt gerne hvor kunderne (og måske også gerne tidligere skabere/programmører) hurtigt for stemmemagt ift. bagud-belønningen også (og i øvrigt gerne hvor man også prøver at opstætte retningslinjer omkring, hvem fortjener hvad for hvad). Og her skal "open source" altså forstås meget i gåseøjne: Vi snakker nemlig slet ikke open source bidrag, for IP-rettighederne skal meget gerne gå til en fælles pulje som eges af virksomheden (eller endnu bedre: en mere uafhængig instans/"undervirksomhed" som virksomeheden så er kunde hos..). Men når jeg alligevel kalder dem ""open-soruce"-programmører," så er det altså bare for at pointere/hentyde til, at deres arbejde i høj grad så kommer til at minde om open source-arbejde, fordi tanken altså netop er, at bidragsyderne bare kan bidrage rimeligt frit og altså uden at være ansat og/eller have underskrevet en masse kontrakter, men hvor de så alligevel kan få løn for arbejdet via bagud-belønnings-systemet. Så ja, det ville jeg sandsynligvis nok prøve at sigte efter, hvis det var mig, der skulle starte sådan en ("kundedreven") virksomhed. (12:07) ..Ah, og vigtigt: Jeg ville også bestemt sørge for, at denne "bagud-belønning" også i høj grad (i starten især) ville komme i form af "IM-skaber-aktier," det er klart, for så kan man jo dermed belønne dem (hvis alt går godt) meget mere fra starten, også selvom man ikke har de store indtægter (fra kunder) endnu, og samtidigt så også gøre alle disse programmører/skabere mere investerede (også altså i overført betydning) i projektet. Det kan godt være, man lige skal se denne sidstnævnte ting efter i sømmene og regne efter på det hele først.. men ja, det ville jeg jo så gøre, hvis det var mig, der skulle være med til at opstarte en kd.v., for umiddelbart ser det ud til, at denne sidstnævnte ting også kunne gå hen og blive rigtigt smart at gøre. Nå. :) Følte lige for at gentage/understrege disse ting. :) (12:17, 06.10.22)


(19.10.22, 10:29) Jeg har fået tænkt noget mere over min forretnings(bevægelse)idé. Jeg har skrevet lidt ny brainstorm i et andet dokument, hvor jeg i går overvejde igen at tage mere udgangspunkt i omsætningen, men det går ikke. Og nu er jeg faktisk kommet frem til, at jeg nok bør ændre idéen til en mere simpel udgave (overordnet set, for jeg har også nogle nye tilføjelser, som jeg fandt på i går, om frit at kunne købe en vis størrelse kundeaktie oven i sin egen som kunde).
I bund og grund tror jeg nu på (bl.a. fordi jeg har indset, at mange af mine tidligere bekymringer skyldtes en tanke om, at virksomheden skulle være den eneste af sin slags, men det skal den slet ikke.. hm, ikke på nær måske hvis man tænker en web 3.0-virksomhed.. det må jeg lige tænke over, men lad mig her bare skrive om idéen med tankerne rettet mod normal industri og handel).. Jeg tror nu på, at idéen faktisk er bedst, hvis bare man simpelthen har start-aktierne og kunde-aktierne som før beskrevet, begge med en vis fast udløbskurve, således at aktiernes "størrelse" starter på et punkt og efterfølgende aftager efter hver lille salgsperiode. Nu mener jeg så, at det så bare skal være frit op til den samlede mængde aktionære (via deres stemmemagt) at beslutte løbende, hvor stort et afkast skal betales pr. aktiestørrelse efter disse salgsperioder. Virksomheden skal så bare have åbne regnskaber, så alle kan følge med i, inklusiv fremtidige kunder, hvad virksomheden har af reelle omkostninger, og dermed hvad pris-markup'en er for hvert produkt over tid (hvor man så selv kan vælge som iagttager, hvordan man vil regne udviklingsomkostninger ind sammen med "produktionsomkostningerne"). Det er så fornuftigt at forvente som kunde, og fornuftigt at drive virksomheden som aktionær, således at markup'en er rimelig konstant, når man midler over en periode, f.eks. over et år eller to. Og den skal i hvert fald helst gøres så stor, at kunderne er mere investerede til hver en tid, end hvad de samlede aktiver er hver, hvis man skulle sælge dem. Og desuden er det også smart at have en højere markup, hvis man gerne vil give større encitament for aktionærerne til at træffe gode beslutninger frem for dårlige --- plus dette giver også en vis investeringsbuffer, så virksomheden ikke hele tiden teknisk set er på randen af konkurs, altså fordi den "kun lige løber rundt," kan man sige. Men en al for høj markup er dog heller ikke at fortrække, for det kan skræmme nye kunder væk, som ikke rigtigt har nogen kundeaktier i forvejen. Dette vil så gøre virksomheden sårbar over for, at en konkurrent kan melde sig på banen og tiltrække alle disse kunder. Så disse tanker bør man altså gøre sig, når markup'en og det løbende afkast skal udregnes (i forhold til produktions og udviklingsomkostningerne). Men det er selvfølgelig rart lige at huske, at i sidte ende så kommer virksomheden jo meget hurtigt til at være styret af en stor (og i mange tilfælde almen) gruppe mennesker, som dermed ikke vil have meget ud af at prøve at presse citronen over for nye kunder, da disse jo ofte i høj grad vil være dem selv. Og i de få tilfælde, hvor der kan være en anseelig forskel på gamle og nye kunder, jamen så må man også forvente, at hvis en (\emph{stor}) gruppe mennesker vil presse en anden (\emph{stor}) gruppe mennesker som forbrugere, så er der jo stor chance for, at den anden gruppe vil gøre gengæld. Hvis der altså vil ende med at være visse store grupperinger af forbrugere, så er det altså naturligt at forudse, at disse bare vil indgå aftaler med hinanden i stedet for at prøve at presse citronen og skrabe til sig.. Ja, og al denne snak er jo stort set ligegyldig, for det vil jo være meget sjældent, at der vil være stor forskel på gamle og nye forbrugere af en virksomhed, og hvis der er, jamen så vil det jo kun lige være midlertidigt, må man regne med. Ja, så never mind al denne snak i bund og grund. (11:00) 
Så ja, afkaststørrelsen pr aktiestørrelse, eller rettere aktiestørrelse der udløber, hvis nu kurverne ikke er lineære, skal altså bare bestemmes rimeligt frit af aktionærerne via deres stemmemagt, og det samme gælder alle priserne. "proportionalitetsfaktoren," som jeg har snakket om, hvad der også svarer ret meget til "markup'en," den er altså nu bare en implicit størrelse, som folk selv kan regne ud hver især (idet alle regnskaber skal være offentlige (samt i øvrigt også alt muligt andet i virksomheden, f.eks. også hele beslutningsprocessen, når det kommer til den overordnede ledelse af virksomheden)). Så nu skal kunderne altså bare foholde sig til en enkelt pris, og så kan de selv regne ud, hvad markup'en er på denne. I denne version af idéen tænker jeg så også bare, at aktierne udstedes i slutningen af enhver lille salgsperiode, således at den altså er propertionel med prisen divideres med det samlede salg i den pågældende periode. Det vil sige, at kunderne ikke ved eksakt hvor stor en aktie de får ved købet ned til hvert decimal, men de kan stadig regne det ud tilnærmelsesvist eksakt i de fleste tilfælde, for man må jo formode, at salget vil være rimeligt konstant. Og hvis det lige tager et hop op på et tidspunkt, så vil det jo ikke gøre det vildt store. Det virker altså ikke som om, at det vil være værd at indføre et buffersystem eller tilsvarende, bare for den mikro lille generelle usikkerhed omkring aktiestørrelsen, man får med i købet, slet ikke.. (11:09)
Så det er altså den store nye ændring. Jeg har i øvrigt så også lige nævnt, at virksomhedens regnskaber og ledelses-beslutningsprocess gerne skal være offentlige (for det vil helt klart være det værd fra kundernes synspunkt, frem for den lille makedsfordel det vil kunne være, at holde visse ting hemmelige). Så det vil jeg også fremføre som en del af idéen. Og så har jeg altså nogle nye tanker om, hvordan kundeaktionærerne skal kunne sælge og ikke sælge deres aktier.
Jeg har som sagt fundet på, at kunder jo gerne må kunne sælge deres kundeaktier til andre kunder, i hvert fald så længe en specifik kunde bare ikke kan købe mere en en vis aktiestørrelse i forhold til dennes originale kundeaktie(størrelse), altså den mængde aktier, vedkomne har tilegnet sig via eget forbrug. Hm, jeg skal egentligt lige tænke over, hvad man gør, hvis.. Hm.. (11:14)
(11:24) Hm, der er ingen grund til at give så mange restriktioner. Man kan bare sige, at man altså kun må sælge sine kundeaktier til andre kundeaktionærer, og at enhver kunde ikke må købe en aktie(størrelse), således at deres samlede aktie bliver større end, vi kunne jo sige det dobbelte, af deres nuværende del af deres aktie, som de har tilegnet sig via forbrug. Fordi nykøbte aktier dog godt kan aftage langsommere i størrelsen (f.eks. hvis man bruger en lineær forskrift), så kan man godt komme ud for, at forholdet overstiger det dobbelte i den efterfølgende fremtid, men det er også fint nok; man kan bare sige, at det ikke må overskride det dobbelte i selve handelstidspunktet. Denne forordning bør være tilstrækkelig til at sikre, at folk kan finde købere, hvis de nu gerne vil sælge deres kundeaktier (måske til en anesle lavere end, hvad de er værd), men systemet forhindrer stadigvæk tredjeparter i at komme udefra og opkøbe en majoritet i virksomheden, så denne ryger væk fra kundernes hænder. (Det gør også, at afdøde kunder, kan få "solgt" deres aktier videre (og dette bør man altså sikre sig, at de kan, selvfølgelig ved at deres arvtagere får lov at styre handlen). 
Jeg har vist i øvrigt også en lille note om fissioner, lad mig lige se, og var der så ellers andet, jeg lige skulle nævne?.. (11:34) ..Hm, nogle ændringer i mine planer, men var der ellers andet?.. ..Nej, det var der vist ikke, og ellers kommer jeg i tanke om dem. ..Ah jo, lad mig lige berøre amnet omkring, hvem der er kunder kort.. Virksomheden skal stadig have en beskrivelse og, hvad der er dets "servicer og produkter," og hvad der ikke er, og dette skal det kunne ændre i løbende, men gerne hvor der så er en vetoret, således at enhver stor nok mængde aktionærer (også alt efter hvilken type) kan vetoe enhver ændring. ..Ja. Mere er der sådan set ikke at sige om dette..
Min tanke angående fission var bare, (hvilket jeg også har tænkt på før) at man jo også potentielt set kunne forstille sig virksomhedssplittelser, hvor virksomheden bare dels i flere afdelinger, som administrerer (og har magt over) forskellige ting. Og så kan det så altså stadig være sådan, at kunder, der er mere kunde det ene sted, så vil få mere aktie og stemmemagt hos denne afdeling. Vi kunne altså forestille os en form for "blød fission," hvor den "hårde fission" så altså vil være, at virksomheden skilte sig totalt ad i to (eller flere). Dette er helt klart værd at have med i tankerne, når man skal overveje fissioner. Hm, i øvrigt tror jeg da egentligt sagtens, man bare kan planlægge fissionsreglerne efter at virksomheden er i sving; man behøver vel ikke nødvendigvis at planlægge disse ting fra starten af? Nej, det må man jo ikke behøve, for hvis kunderne har træng og lyst til, at sådanne fissionsregler skal være en realitet, så kan de jo også indføre dem.. Hm tjo, men det kunne dog måske være smart nok, hvis man som startaktionær lovede sine kunder, måske med en kontrakt indblandet, hvis det giver mening.. at man rimeligt hurtigt vil forsøge at udforme gode fissionsregler for virksomheden.. Hm.. ..Tja, det allerførste kd.v.'er på markedet kan jo nok sagtens bare give dette som et løfte (hvis de vil), og når så teknologien bliver mere udviklet, så kan eventuelle nye kd.v.'er jo bare adoptere andres fissionsregler. Og det vil betale sig for en virksomhed (der har udviklingspotentiale) på et tidspunkt at lave sådanne regler, da dette sandsynligvis vil behage den brede kundebase, og dermed vil det sænke risikoen for, at en konkurrent med gode fissionsregler melder sig på banen (hvis vi altså tænker på en virksomhed, hvor startaktionærerne stadig har magt og så overvejer, om det kan betale sig at indføre fissionsregler på et halvtidligt tidspunkt).. Ok. Lad mig bare lade det være det for nu. Og i det dokument, jeg har tænkt mig at skrive nu her i de kommende dage, regner jeg så bare med, at jeg nævner "virksomhedsfission" kun ultrakort, måske bare i en liste over ting, jeg kan uddybe på et senere tidspunkt. Ok. Nu vil jeg så lige opdatere mine "planer" nedenfor.. (11:58)
...(12:13) Nå ja, jeg vil også lige nævne, at der jo er stor forretnings- og investeringsmulighed (for start-aktionærerne), for den positive kunde-feedback, nemlig i form af at kunderne i højere grad vil fravælgekonkurrenter, jo mere de selv er investerede, gør at kd.v.'er vill udkonkurrere ikke-kd.v.'er i samme branche, hvorved der jo vil være et stort vækst- og fortjeneste-potentiale. ..Og denne indsigt, hvis folk kan forstå den, gør også, at idéen vil sælge lidt sig selv, når først bare nogen vil have forstået den (tror jeg, 7, 9, 13). (12:17)

(25.10.22, 16:17) Okay, jeg har nogle nyheder. Nej jeg tror ikke længere, at idéen vil sælge sig selv helt så meget, som jeg ellers sluttede den sidste paragraf af med at sige. Min forretningsbevægelse vil nok i høj grad skulles gennemføres på baggrund af en stor politisk vilje i folk til at opnå de fremtidsudsigter, den handler om at opnå. ..Hm, jeg kan forresten også lige nævne, at jeg nu mener, at det nok kunne være en god idé at gøre, så at kunder i starten er frie til at handle med hvem som helst, og at det så bare er efter et vist tidspunkt, at nye kundeaktier kun må sælges til andre kunder. For det kræver jo, at mængden af kunder (for heri tæller vi jo ikke start-aktionærerne) er stor nok, før at det kan fungere, at de kun må sælge til andre kunder. Jeg mener så stadig, at man kunne gøre det sådan, at ingen kunde må få mere end dobbelt ved tidspunktet af sælget, end hvad denne ville have haft, hvis denne aldrig havde solgt eller købt kundeaktier. Og det skal så i øvrigt siges, at disse restriktioner måske er ret skrappe ift., hvad der måske er nødvendigt, men ja, om ikke andet så tror jeg altså, at de er gode nok, hvis ikke man kan finde på et mildere system, der også stadig klarer ærterne. Nå, det var lidt et sidespring: Det var egentligt ikke med i de 'nyheder,' jeg tænkte på.. ..Ja, for at fortsætte de egentlige nyheder, jeg tænkte på, så tror jeg altså, at forretningsidéen nu skal sælges meget mere på baggrund af den 'gode bevægelse' ligesom, mere end at 'de første kunder også bliver belønnede'.. Og nu tænker jeg så at skrive mine "bright future"-noter om, så at afsnittet om denne idé så mere bare kommer til at sige: Jamen hvis nu det går endnu mere ned ad bakke, jamen så må man jo på et tidspunkt nå et punkt, hvor der bliver grobund for sådan en bevægelse, hvis altså den ikke bare finder god nok grobund med det samme. Så afsnittet bliver så ikke så meget: "Nu skal I se denne sikre plan, som vi kan gå i gang med med det samme," men mere: "Okay, selvom kapitalismen for nogen kan se lidt sort ud pt., så kan det altså ikke gå helt galt; det skal nok gå den rigtige vej overordnet set." Og jeg kan se på, om jeg så vil prøve at vinkle afsnittet mere som et "fix af" kapitalismen eller som et "forsvar for" kapitalismen, det kan godt være, at jeg så vælger det sidste i stedet.. ..Nå, men på en god eftermiddagsgåtur her kom jeg så også på en masse nye ting. Jeg skal faktisk udbygge mit e-demokrati-afsnit en del: Det er ikke nok bare at lægge op til det majoritetsenevælde, som den gør; der kan være meget mere komplicerede og spændende forhold, som gør, at folk gerne vil kunne oprette grupper, hvor de så kan handle med de andre grupper med deres stemmer til diverse ting ("hvis vi stemmer for det og det, så stemmer I for det og det"), og også i øvrigt handle omkring, om gruppen overhovedet vil bakke op om den samlede enhed (hvilket f.eks. kunne være et parti eller et firma), hvis ikke de får sådan og sådan. Så det skal jeg altså også lige skrive om, inkl. at skrive om, at det digitale system skal indrettes, så brugere altså kan oprette disse grupper og sådan. Og noget helt andet er så, at.. Ja, eller der er faktisk tre ting mere, og måske skulle jeg starte med denne i stedet: Jeg har tænkt på, at.. (16:42) ..Hm nå, der gik jeg lige lidt død, så lad mig i stedet starte med: At jeg fik tænkt over, hvad der egentligt lidt er en gammel version af en anden idé (nemlig min "donationskæde-idé"), nemlig at man også kunne igangsætte en bevægelse, hvor folk simpelthen lover dusører for, hvis virksomheder eller organisationer m.m. opnår et eller andet specifikt i fremtiden. ..Nå, nu gik jeg også lidt død i det igen.. Hm, og den tredje ting var så omkring.. Hm, den første/anden ting var bare noget omkring at give de nuværende kunder mere magt, hvilket jeg faktisk tror kan være ret vigtigt især for en Web 2.0--3.0-virksomhed. Hm, men mere er der vel ikke nødvendigvis at sige/nævne om den ting her, så det var én ud af tre.. Og den tredje ting var så.. ..Nå jo, det var at jeg også nu har tænkt mig at hive "forbrugerforeninger" m.m. (også samt hvad jeg lidt har kaldt "civilforeninger" på et tidspunkt) mere i forfronten nu. Jeg kan så skrive om dette, efter jeg har skrevet om e-demokrati og det. Og så kan jeg altså kort gøre rede for, hvad man kan opnå med sådanne "forbruger"-/"civil"-foreninger, eller hvad vi skal kalde dem, og så også at man jo kan gøre brug af et e-demokrati her. I øvrigt kan jeg lige gentage/præcisere, at min idé om "civilforeninger" i høj grad bare handlede om, at man går sammen i grupper til at representere sig som borger og/eller forbruger i et samfund (gerne altså sammen med nogen i en lignende båd), og hvor man så begynder at samarbejde og opføre sig meget som en slags forrestning, både idet man så begynder at lave handler samlet og betale nogen for at stå for at indgå (handels- etc.)aftaler med andre grupper/instanser, og også idet man så begynder at betale folk der kommer med nye smarte idéer til, hvad man kan gøre som gruppe. Hm, det lyder godt nok løst, når jeg skriver det her (og det er lidt blandet sammen med "forbrugerforeninger"), men på den anden side er jeg også ret træt nu, kan jeg mærke, og ja, det er lang tid siden, jeg har tænkt så meget over det, så det skal nok give god mening, når lige jeg får støvet det af.. ..Men ja, så nummer tre ting er altså bare, at jeg også vil skrive om disse forbrugerforeninger, og altså sikkert også "m.m.".. Hm, og lad mig så vende tilbage og skrive om donations-/dusør-idéen på at senere tidspunkt, når jeg er mere frisk i hovedet igen.. (17:06, 25.10.22)
%(26.10.22, 10:58) Okay, jeg har tænkt lidt mere over min nye version af denne dusør-idé, og jo, det er lidt kød på den, men ikke nok til, at jeg vil skrive om det, og måske ingen gang nok til, at jeg vil bruge tid nu på at forklare den. ..Nej, lad mig bare forklare den på et senere tidspunkt ved lejlighed, for idéen er altså ikke så vigtig (den handler bare om, at investorer, der øjner nogle gode fremtidsmuligheder i en idé, men ikke synes investeringsmulighederne/afkastudsigterne er helt gode nok, de kan så bede folk om at søtte idéen ved at udlove dusører, betinget af at idéen går godt og at disse folk vil drage nytte af den; så kan de så love at give lidt retur for dette. (Så ja, ikke den vildt store idé, når man tænker over det..)).. Hm, nu fik jeg faktisk næsten forklaret den her i den parentes.. Og ellers tilføjer idéen bare lidt om, hvordan disse dusørløfter kunne gøres.. men det gider jeg ikke skrive om nu.. Ok. Jeg har så også tænkt mere over nogle af de andre ting, jeg vil skrive om, og nu tror jeg planen er, at skrive en lidt mere simpel version af min business movement-idé, at skrive e-demokrati-sektionen færdig (med de nye tilføjelser), og så også bare lige inkludere et afsnit om forbruger-/civilforeninger, inden jeg når til Web 2.0--3.0-sektionerne (og så videre derfra). Og jeg kan så forklare, at denne idé altså i bund og grund handler om, at gå sammen i store grupper og ansætte agenter til at finde gode tilbud til gruppen og til at finde frem til mulige gavnlige aftaler for gruppen i det hele taget.. (11:08) Ok.. Mere er der vist ikke at sige her. Nu tror jeg så derfor, jeg vil fortsætte skriveriet, hvor jeg så nok lige begynder på en version 2 af "bright future"-dokumentet i øvrigt. (11:09)
(12:34) Okay, nu har jeg godt nok lige tænkt lidt mere, og.. Ja, det korte af det lange er bare, at jeg faktisk virkeligt tror, det kunne blive en stor bevægelse. Nærmere bestemt så skal jeg vist bare lige fokusere på, at det især er detail- (retail på engelsk) forretninger, man nok bør starte med virkeligt at fokusere bevægelsen på. Alle kd.v.'er kan så have et bestemt mærke, som folk kan gå efter. Fordelen er her, at denne branche indeholder mange små virksomheder, og generelt kræver det ikke så meget kapital at starte en butik m.m., og hvis vi tænker butiks\emph{kæder}, så er der også alligevel så mange, så man kunne godt forestille sig at mindst én vil konvertere. Nå ja, og webshops / web stores (det første er muligvis et begreb vi mest bruger Danmark, men det ved jeg ikke helt) er selvfølgig også en rigtigt vigtig mulighed at satse på også. Så herved kan bevægelsen altså starte og begynde at få mere og mere opmærksomhed og kapital til rådighed. Og når man så har fået samlet kapital nok, så kan man så også begynde at brede sig ud til andre brancher, hvis altså ikke der ellers er nogen virksomheder, der har konverteret endnu her. :) Okay, så min idé kan altså muligvis stadig rykke en hel del, selv her i nutiden; det er nok alligevel ikke bare en idé, der ligesom kan være et sikkerhedsnet i fremtiden, den er mere end det.. :) (12:44) 




## Energi, ressourcer, klima

(18.08.22, 13:30) Ja, vi bør helt klart fremelske en slags tang- og/eller vandplante, der kan fungere som en slags hvede (eller lignende), men på havoverfladen. Og hvis vi så kunne opdyrke f.eks. dele af stillehavet, så kunne man jo i teorien få en KÆMPE ny ressource, som kan bruges til alt muligt (energi, føde, og potentielt set også til at grave ned for at indkapsle CO_2, hvis man altså \emph{[virkeligt} får overskud..). Man kunne jo så evt. gøde planterne bare ved at suge næringsholdningt vand op fra dybet. Kunne have potentiale til at blive kæmpe stort, hvis det kan fungere.. 

(08.09.22, 9:57) Kom lige til at tænke på: Gad vide, hvad der ville ske, hvis man bare konstruerede en masse rev i havet, måske bare ved at lægge et netværk a flydende slanger/rør ud eller lignende. Tanken er lidt, at der måske så ville dannes en hel masse tang/alger og måske andet fiskeliv. Og en follow-up-tanke er jo så lige, om man eventuelt så skulle pumpe næringsholdigt vand op og igennem slangerne/rørene.. Jeg tror, jeg vil give dette emne nogle flere tanker i min fritid, og så vende tilbage hertil, hvis jeg skulle få yderligere idéer omkring det, der er værd at nævne. ..Men ja, den helt simple version er disse tanker er bare: Gad vide, om man kunne skabe en masse plantevækst, og vækst i betanden af diverse havdyr, hvis man bare lagde flydende rev ud. Jeg ved jo godt, at alt bliver vildt dyrt at udføre, hvis det skal skaleres meget op, før det får en effekt, men idéen er da nogle flere tanker værd.. (10:05)

(26.12.22, 11:06) Idéen med at køle planeten med "aske" eller lignende er selvfølgelig fundet på, og det er faktisk en rigtig populær idé endda. Den går under navnet "stratospheric aerosol injection," og den virker nemlig faktisk rigtigt lovende (rent teoretisk altså)..:)



## Andre ting, der relaterer sig lidt web- og forretnings-idéer m.m., men som er lidt uden for kategori

(18:28, 27.09.22) Jeg kom til at tænke lidt over, i går eller i forgårs, retsystemer i fremtiden. Jeg vil så bare lige nævne, at jeg jo allerede har skrevet om (i mine 21--22-noter), hvordan man kan få klare fælles etiske retningslinjer i fremtiden i diverse samfund. Og hertil kunne man jo så lige tilføje, at man så også kunne forestille sig et retssystem, som baserer jeg lidt mindre på lovskrifter og deres fortolkninger, men lidt mere på en.. ontologi/model/.. mængde.. af retningslinjer for, hvordan man skal dømme diverse forseelser og andre sager/konflikter osv. Og så kunne man jo naturligvis have et rangsystem af en slags dommere, men hvor hele befolkningen i det samlede samfund ligesom sidder i toppen (i praksis) og som fællesskab så har ret til.. måske hvor forskellige mennesker får ret til forskelligt data.. ret til at udtage stikprøver af under-dommerinstansernes bedømmelser, og hvor man derfor kan rette op på.. Ja og/eller man kan selvfølgelig også bare have et anke-system, som jeg også tænkte på, hvor parterne --- og måske også vedrørende til disse (eller måske bare folk, der har fulgt med i retsagen) --- jo så kan få lov at anke til en højere instans.. Så ja, men den primære pointe er altså, at man kunne forestille sig et alternativ til et lovsystem baseret på en masse paragraffer, hvor man måske i stedet bare havde en stor mængde (hierarkisk ordnede) retningslinjer og eksempler at gå ud fra.. (18:40) 




## Evolutionsspykologi

(18.08.22, 13:35) Jeg tror ikke, jeg har understreget dette i mine 2021-22-noter, men ét punkt, hvor jeg virkeligt tror man kan komme langt med evo.-spyk., er til at analysere og prøve at forstå, hvad der bringer os lykke som mennesker. Jeg tror virkeligt man kan skubbe meget til den analyse, hvis man tager evo.-psyk. godt i betragtning.


## Lykke

(18.08.22, 13:38) Angående lykke, så har jeg også lige nogle flere ting, jeg vil skrive om det emne, men det kan være, at jeg lige udsætter det en gang.. 

(31.08.22, 20:02) Okay, lad mig lige prøve at forklare de her tanker lidt. Jeg havde/har for det første lyst til lige at kommentere noget omkring, hvorfor man bl.a. har så mange glæder som barn (hvis man altså er heldig nok). Ja, der er vel mange mange grunde, men jeg har bare lige lyst til at fremhæve, at man (måske især som drengebarn) kan have en virkeligt stor lyst til eventyr --- i hvert fald til tanken om dem, men man kan alligevel godt på en måde få udlevet sin eventyr lyst (også selvom man ikke rent faktisk tager på et eventyr) via lege og via bøger, tegneserier og film m.m. Jeg kan personligt huske at tanken om et eventyr (og nu tænker jeg faktisk selv meget på One Piece som et eksempel på et "eventyr") bare var "helt oppe og ringe" dengang. Samtidigt havde man også en anden ting som barn, som virkeligt gjorde mange glæder mere tilgængelige: Man blev så nemt awestruck af ting. Jeg kan huske at en af de helt store øjeblikke i mit liv, var da vi fik vores første pokemon (blå) gameboy-spil. Vi var bare Helt oppe at køre, og det var bare sådan en lykke; det var sådan en fed følelse; så spændende. Nå, hvorfor har jeg så lyst til at nævne disse ting (for de er jo rimeligt velkendte)?. Jo, min pointe er så, at vi jo altså lidt kan tabe nogen af disse lykke-givende faktorer --- eller "parametre" kunne man kalde det --- i vores sind, når vi bliver ældre. Eller de dæmpes i hvert fald lidt. Men man kan dog have disse ting for øje, når man skal overveje, hvordan man konstruerer et lykkeligt liv sammen i et fælleskab. Særligt den der awe: man kan gøre mange til for at give ting mere mening.. Ah, jeg kunne også nævne det her med, at gamle ting, man er så nostalgisk omkring (hvis man altså har sådan nogle ting), der er ofte en god grund til, at man synes de var/er så store, og faktisk en ting som ikke nødvendigvis har rent med alder at gøre: Når mange mennesker går og er hypet omkring det samme, f.eks. når et nyt spil (eller en ny bog, eller hvad har vi) udkommer, jamen så ligger der bare så meget mere.. underbevist prestige omkring at klare sig godt i det, og selv for middelmådige spillere vil der stadig bare være.. en følelse af at ting "vejer" meget mere (end hvis man går i gang med samme spil mange år efter, eller nu hvor der er sådan et kæmpe udvalg så ingen spiller det samme (medmindre det er et rigtigt populært og nyt spil)). Det er f.eks. også bl.a. derfor Pokemon GO var så stort, fordi så mange interesserede sig for det, og fordi man derfor blev grebet meget af, at avancere i spillet. Og igen, hvorfor er det her så interessant og nævne i forbindelse med emnet om "lykkelige fælleskaber?" jo, fordi man jo så kan prøve at begrænse adgangen til hobbyer og/eller gøre tiltag for at skabe nogle store "diller" (som det hedder; det er bare ikke vildt tit man bruger det ord mere) løbende, som folk i høj grad kan blive grebet af. Og ja, jeg synes det er interessant, for det er jo en helt anden dimension end bare at sørge for, at man har nogle gode muligheder og nogle gode traditioner (og kreative mennesker til at finde på events): Der er også hele den dimension omkring, at folk også tit skal \emph{gribes} af en lyst til at deltage i en ny dille(/sport / kunstnerrisk/udfoldningsmæssig/literær strømning osv.). Så man skal altså ikke nødvendigvis se på bare at lave gode events som folk kan deltage i; man må også gerne overveje, hvordan man får folk grebet af ting i fælleskab. 
Nå, så det var så ligesom tankerne omkring awe og generel begejstring (som også relaterer sig til, hvad man går og har nostalgifølelse omkring). ..Nå ja, og jeg kan forresten også lige hurtigt nævne, at disse betragtninger altsammen er noget, der kan forklares med evolutionspsykologi (hvis man lige udvikler den gren, så den handler meget mere om at se på, hvordan evolutionsprincippet spiller aktivt og dynamisk ind i vores nutid *(og i vores helt nære fortid osv.), og hvordan vi i høj grad stadig fungerer som skabninger af evolutionen, hvad vi er, *(således at vores følelser og handlinger stadig i høj (men dog ikke fuldstændig) grad kan.. "forklares," eller man kan i hvert fald gøre sig meget mere vis på emnet, ved at se det i lyset af, at vi er formet af evolutionen) i stedet for (*himler og tager hånden op til panden* (for effekt)) at fokusere på, hvordan vi er "fortidsmennesker" med basale instinkter, der ikke længere passer til den morderne verden (*himler*)). Nå, det var et lille sidespring..
Og lad mig så lige prøve at vende tilbage til det med eventyrlyst (hvis der altså er noget her jeg mangler at sige..?).. ..Nej, her er der faktisk ikke så meget at sige, andet end at det er lidt ærgerligt, at vi mister nogle af disse barnlige trænge/lyster.. hvad jeg en gang ofte ville kalde "behov" (og måske også "værdier" nogen gange, det kan jeg ikke lige huske (og gider ikke lige prøve)) (hvilket handlede om, at vi jo fra naturens side har behov, og mange af disse behov er vi så så heldige, at naturen har udviklet en "gulerod" til, således at vi føler lykke, når det lykkes os at opfylde behovet.). Det kan være at vi i fremtiden kan opfinde en terapi eller andet, så man kan forstærke disse "behov," i.e. disse lyster, men indtil da må vi jo bare nyde det imens vi er børn, og så ellers prøve i nogen grad ikke at slippe de "behov"/lyster, når vi bliver ældre. Så ja, det var vist rimeligt meget de tanker (plus lidt sidespring), jeg havde lyst til at nævne/notere. :) (20:50)

(05.09.22, 19:55) Mon ikke jeg har været lidt inde på dette, men lad mig lige nævne, at et rigtigt godt råd, og en rigtig god ting at bestræbe sig på, er: Vær gavmild med komplimenter! Giv dem ofte! Igen: Vi har jo i bund og grund mest bare den glæde, vi får fra andre. Og det giver bare SÅ meget mere lykke, hvis man hele tiden sørger for at sætte udtrykkeligt pris på sine nærmeste --- og komplimenter til alle de ikke-nærmeste er også en god idé at bestræbe sig på at være gavmild med: det bringer alt sammen lykke. Selvfølgelig er det mere og mere vigtigt, jo nærmere folk kommer på en, at give komplimenter (og ros og andre tegn på værdsættelse), men ja, så længe rosen/komplimenterne kan gives oprigtigt, så er der ingen grund til at holde igen med dem. Bare en lille ting, jeg lige ville nævne (måske igen), og som virkeligt kan være værd at leve efter --- det må jeg også selv gøre meget for at huske på. 




## Eksistens

(06.09.22, 10:31) Jeg har i øvrigt også tænkt nogle små tanker omkring, hvis man antager at der er en overordnet skaber-gud. Hvis man bare antager, at han ikke er \emph{uendeligt} potents, men måske bare potent langt ud over, hvad vi kan forestille os, og hvis man også antager at han faktisk er god --- ikke bare sådan god som i: "Åh, hvor er du god, gud! ..Vær nådig ikke at sende mig i helvede og brænde..!" eller som i: "Definition af god er hvad gud er, for gud er den største og bedste,"  men som i at han rent faktisk ønsker så mange sjæle som muligt at opleve så meget lykke (og så lidt ulykke, selvfølgelig, der trækker fra af den samlede lykke) som muligt --- jamen så ville det jo egentligt give god mening, at han ville vælge at bruge sine skaberkræfter på at skabe en afsindig stor mængde af universer, som er nemme at opstille lovene for, og som ikke kræver nogen kræfter at styre, således at han ikke hele tiden skal gå tilbage og passe sine gamle kreationer men hele tiden bare kan fokusere sine kræfter på at skabe flere. Jeg synes dette er ret oplagt at forestille sig, at det ville kræve mere energi, hvis man hele tiden skulle overvåge og indgibe i alle de universer, man har lavet --- især hvis man sætter sig selv den umilge opgave for at sørge for at "intet ondt sker imod gode mennesker"---ja, for så skal han jo endda gå ind og forudsige, hvad der sker, og så skal han jo basalt set "køre simulationen flere gange alligvel, indtil han får det resultat, han ønsker, og det må jo tage mange mange kræfter og meget fokus, relativt til bare at fokusere på at skabe det næste uhyrligt store batch af universer, som skal sættes i gang. ..Hm, der er egentligt også andre antagelser, man kunne tænke over, selvfølgelig er der det, men lad os bare begrænse os til dette her.. Og ja, det skal så understreges, at denne analyse ikke handler om, at konkludere på, hvad der må gælde for sådan en gud. I stedet handler det bare om at forklare, hvorfor det ikke er langt ude i hampen, hvis man gør disse antagelser, at nå til en teori/hypotese, hvor guds ikke-indgriben og tings tilsyneladende tilfældighed faktisk kan forklares ret godt, på trods af guds godhed (for en af antagelserne er jo faktisk, at han \emph{rent faktisk} er \emph{god}, altså i en forstand der passer meget bedre til, hvad vi almindeligvis vil betegne (på trods af at der kan være mindre variationer af, hvad folk ser som 'godhed') som 'godt,' når vi snakker om \emph{menneskers} handlinger). (10:52) ..(10:56) Hm, jeg kom lige i tanke om, at gud pr. den kristne (og jødiske --- og sikkert også den muslimske) tro jo skabte verden på seks/syv dage, så en antagelse om at "uendeligt potent" bare skal ses lidt metaforisk må jo egentligt være ret oplagt.. 

(05.10.22, 15:10) Jeg havde tænkt mig at forberede en lille teaser udgivelse, som jeg vill lægge ud på GitHub her som noget af det første, men nu har jeg lidt fortrudt. Her er mine tanker (noter) nu her fra et andet dokument, jeg skrev i:
\# Existence theory

%Let me begin this introduction/teaser on a small personal note.
%
%When I was younger, in my teens, I was quite interested in 

"How and why was the universe created?" "What constitutes consciousness?" and "how does matter, in particular brains, gain consciousness?" 
These are questions that many people have asked themselves, probably often with the same open-ended conclusion: These questions are perhaps just to big for us "mortals" to answer. 

Indeed this seems to be the case: Even if we found a good answer, how would we ever know whether it is actually correct or not? And furthermore, we might not even be able to understand the correct answer if a god/oracle could tell it to us; it might be too complicated, and it even might include some otherworldly logic that we can never comprehend.

So if we look at the %...(10:44, 05.10.22) Jeg søgte lige lidt på filosofi (har bare læst wiki-artikler), og jeg har lige set, at hypotesen/antagelsen om at "alle mulige verdener eksisterer" også hedder "modal realism." Spændende. Jeg har vist hidtil kun læst om "mathematical universe theory." Der står på wiki-siden, at nogle modstandere mener, at hypotesen er i konflikt med Occam's razor, hvilket jo er rigtigt interessant, for det er den nemlig ikke; det kræver bare lige lidt omtanke og analyse for at komme frem til det resultat. (Og altså også en antagelse om at verdener/universer har en naturlig, fundamental ordning i multiverset, nemlig ud fra, hvilken information de indeholder/bygger på.) ..Lad mig lige læse videre om det, og også omkring de andre emner, der relaterer sig til spørgsmålene ovenfor.. (10:51) ...(11:13) Hm, det virker alligevel til at associationerne omkring "modal realism," inklusiv hovedproponentens egne holdninger, alligevel er for forskellige fra, hvad jeg tænker på. Og nu læser jeg lidt om MUH, og det virker helt klart til, at det passer mine tanker. Jeg skal lige finde ud af, og der er forskel på CUH og så det, der også er nævnt i wiki-artiklen, nemlig MUH eksklusivt med kontruerbare universer, det må jeg lige finde ud af. Men ja, ovedparten af mine tanker lægger sig altså rigtig meget op ad MUH/CUH, og så er det altså muligvis bare mine tanker omkring bevisthed (som muligvis lægger sig op ad "idealism" og/eller "Platonism," *(nej Platon var vist "realist," ser det ud til, som så er det modsatte..) men det skal jeg lige have læst op på igen), og så måske også bare min tilgang med ikke at lede efter \emph{den} rigtige eksistensteori, men i stedet bare forgrene analysen, hver gang man støder på et spørgsmål, der med fornuft både kan antages, i nogen grad, at være sandt eller at være falsk. (11:21) ..Wow! "Virtually all  historically successful theories of physics violate the CUH"!! Helt ærligt. Så svært kan det altså heller ikke være at forstå Gödels ufuldstændighedsprincip..! Det ser ud til, at den originale (hvilket jeg på en måde også kan siges at være, mener jeg, men der var altså andre, der kom først..) opfinder/opdager af teorien tror (som i øvrigt ser ud til selv at foreslå CUH som modsvar på en vis kritik (om så end det ham, der fandt på det først, det fangede jeg ikke lige)), at f.eks. mængdelære og andre matematiske teorier, hvor ufuldstændighedsprincippet gælder, at det ikke har fuldstændige modeller.. Suk suk. Jeg ville ønske at Gödels fuldstandighedsprincip ikke var kommet så meget i skyggen af det andet princip; der virker til at være meget forvirring omkring det.. ..Hm, folk burde bare blive undervist mere i Gödels kontruerbare univers, og i hvad det betyder: at al matematik kan deles op i to mængder: matematik over objekter kontrueret af en endelig mængde information, og matematik over (filosofisk questionable) objekter dannet af uendeligt meget information.. (11:34)
%...(12:11) Hm, jeg tror hellere lige, jeg må summe lidt over, om det overhovedet kan betale sig for mig at tease mine idéer på dette område nu; MUH er jo ret gammel --- den går faktisk mindst helt tilbage til 1998, kan jeg se (jeg troede den var lidt nyere, selvom den jo stadig er ret ny overordnet set).. Og hvis man tager MUH, eller rettere CUH, som afsæt, så vil det måske blive svært for mig, at få mine idéer teaset, så de lyder interessante (altså hvis man kender til CUH i forvejen).. Ja, lad mig summe lidt over det.. (12:15)
%... (14:56) Ja, jeg er bange for, at der ikke bliver nok kød på det til ligesom at tease det.. Det er for inviklet at forklare, hvad jeg tror, jeg kan bidrage med til emnet. ..Det kan nok ikke rigtigt gøres kortfattet særligt godt. Og ift. at jeg jo havde tænkt mig primært bare at fortælle den lille hurtige redegørelse for, hvorfor man med meget normale antagelser (udbredt blandt ateister og lignende især) hurtigt kommer frem til, at vi så i bund og grund lever uendeligt og i alle afskygninger, nemlig i og med at der så også vil findes alle (og man kan i princippet blive ved med at zoome ind) mellemtrin imellem to forskellige personligheder og tilhørende liv ("oplevelse," i.e.), så vi dermed i praksis alle er den samme, bare i forskellige udgaver. Og jo, mange af mellemtrinene er vildt usandsynlige, men selv "vildt usandsynlig" er forekommer stadig groteskt "ofte" sat op i mod "uendelighed." Og budskabet er jo så, at vi alle er den samme i praksis, og at alt vi gør mod andre mennesker, det bliver så gjort mod os selv i et fremtidigt liv, i praksis, altså.. Men ja, selvom dette resultat ikke behøver så meget teori i forvejen, så føler jeg stadig, at det er altså bart i sig selv til, at det giver mening at tease/forklare det. Hvis det kunne være en del af en teaser til "eksistensteori" generelt, så ville det give god mening, men jeg tror ikke, det vil blive modtaget med meget begejstring, hvis det bare står helt alene.. ..Også fordi, hvis man skal være lidt streng ved sig selv, så kan det jo i bund og grund reduceres bare til at sige: "hey, har I tænkt over, at multiversets uendelighed vil medføre, at alle afskygninger af "liv"/"oplevelser" vil førekomme?" Og det vil sgu nok ikke skabe særligt mange bølger i sig selv.. (15:07) ..Så ja, jeg venter med at udgive (og brygge videre på) mine eksistenstanker (som jeg dog stadig tror virkeligt kan noget, \emph{selvom} meget af det jo dog er tæt på noget kendt)..
slut. 
Så ja, som sagt, jeg føler altså, at jeg virkeligt har noget at byde på, men jeg tror ikke jeg kan finde på noget kortfattet, der kan skabe meget interesse i sig selv.. (15:13)
... (17:13) Åh, jeg kan også lige nævne, at min intension var efter "Indeed this seems to be the case ..." paragrafen (i readme-filen) så at lægge op til: Men hvad med at droppe målet om at finde \emph{det} korrekte svar, men i stedet bare prøve at overveje/analysere (gerne i fællesskab, btw), det samlede træ, ligesom, over de mulige svar der kan være til de grundlæggende spørgsmål. (Og her jeg jeg i øvrigt lige nævne, at der dog ikke bare vil være ét træ, for man kan godt stille spørgsmålene i forskellige rækkefølger, hvor analysen godt kan have karakter heraf. Særligt kan det vist være betydende hvilket spørgsmål (eller hvilke få spørgsmål), man starter med. Jeg mener dog stadig, at det ikke er sådan, at vi så skal analysere en hel skov af træer på en gang; jeg tror på, at det nok bare bliver en lille gruppe af træer, der vil være interessante for de fleste..) (17:19) 



(05.01.23, 11:33) Jeg har nogle tilføjelser til dette emne, og så har jeg også nogle idéer til, hvordan jeg nok vil strukturere en artikel om det. Lad mig se.. I virkeligheden har jeg nok skrevet meget af det før. ..Hm, jeg tror lidt, at jeg har en ny måde at tænke på det mulige fænomen med at Oplevelser bliver vagt til live, hvor jeg altså nu tænker meget, at man nærmest kan sige, at det er [...] At man nærmest kan sige, at det er "universet" --- og her snakker vi altså om det idealistiske univers: et univers der beskriver Oplevelser (hvilket også bare kan være hele multiverset selv) --- der oplever Oplevelserne. Og ja, mine yndlingsteorier har multiverset som selve dét (eneste) idealistiske univers, så jeg fortolker det altså nu meget som at "multiverset oplever Oplevelserne." (Og da en af mine klart yndlingsteorier nu er den hvor multiverset "udregner" al logik, og at Eksistens dermed ligesom er den fundamentale logik om alt, der så at sige udleder sig selv (eller rettere alle "sætninger" i logikken), så ser jeg det altså meget sådan at "multiverset udleder alle Oplevelser, og dermed også udlever/oplever dem"). [...] Nå, men dette var jo en lidt mindre ting. 

En større ting er så, at jeg er gået lid væk fra at forestille mig den fundamentale logik som et sprog; altså som noget med en syntaks eller tilsvarende. Nu tænker jeg altså mere, at den fundamentale Logik ligesom er "rent semantisk." ..Vi kunne snakke om "Pure Reasoning".. "Pure ..." Hm.. ..Ja, "pure and fundamental logical reasoning." Og denne opfattelse betyder faktisk rigtigt meget, for det gør det nemlig pludselig meget nemmere at forestille sig, at der bare er én "fundamental logik for alt." ..Så ja, det er altså derfor, at jeg nu hælder rigtigt meget til, at multiverset ligesom bare er en fundamental og "ren" "logik," der forstår mere og mere "af sig selv," så at sige, og dermed også forstår, hvordan diverse forskellige sammenhængende Oplevelser må føles, og idet "den" forstår dette, så vil den også opleve disse Oplevelser, enten netop idet den forstår det, eller for alt tid igen og igen efter den ("den") har forstået det. (Så altså med andre ord: Enten sker udlevnigen af Oplevelserne på kanten af den forståelses-kulge/-mængde, der udvider sig mere og mere, eller også sker udlevningen konstant indenfor kuglen/mængden (hvor der så bare kommer flere og flere Oplevelser til denne mængde).) 

...Nå, og nu kunne jeg så fortsætte med at sammenligne "den fundamentale logik" i denne teori med, hvad man næsten kunne kalde en gud (og den sammenligning er hurtig at lave), men nu vil jeg i stedet prøve at følge den struktur, jeg har i tankerne for en artikel om det. Jeg forestiller mig nemlig at starte med at liste nogle gode kandidater til Eksistens-teorier, hvoraf den ene så skal være den, jeg lige har beskrevet (som egentligt er to teorier, alt efter om Oplevelserne udleves på kanten eller inden i kuglen/mængden af Oplevelser).

Hm, så lad mig prøve at skrive, hvad den første sektion i den artikel kunne indeholde.. (12:56)

Nå ja, jeg skal jo starte med at redegøre for hypotesen om, at "alt hvad der kan eksistere, eksisterer." Med andre ord er hypotesen, at der er en komplet symmetri ift. hvad der kan eksistere, og hvad der rent faktisk eksisterer af denne mængde; hvis ét univers eller ét delmultivers indeholder nogle specifikke "valg," jamen så må der bare eksisterer modsvarende universer/delmultiverser i lige mængde. Men denne tanke leder så til at spørge: Hvordan defineres den underliggende teori for det samlede multivers så, for for at man kan afgøre, at multiverset er symmetrisk eller ej, så må man jo have et udgangspunkt for at definere, hvad der er symmetrisk og ikke symmetrik --- hvad vil det sige, at et univers er "modsvarende" til et andet univers for eksempel? Ja, og svaret på det er... Ej, det var bare for sjov; det kan vi selvfølgelig ikke svare på. Men vi kan hypotisere, at der kunne finde en fundamental "teori" for multiverset, selvom "teori" dog i så fald vil være et dårligt ord at bruge for det, fordi det indebærer, at der findes andre teorier. Så lad os hellere kalde det "en fundamental logik for alt," og her skal "logik" altså ikke forstås som en formel logik, men i en meget mere løs forstand, nemlig som det fundamentale koncept om, at visse ting kan være sande og visse ting kan være falske, og ting kan følge logisk af andre ting. Så lad os hypotisere, at der eksistere en fundamental logik under den samlede eksistens, for hvad er alternativet? At der ikke er en samlet eksistens? At der ikke er nogen "logik" (i ordets meget løse forstand) bag? Nej, det går ikke rigtigt, så det virker som et fornuftigt aksiom. Og lad os forresten også bare benævne den samlede eksistens for 'multiverset,' da det er lettere at sige. (Og så må man bare endeligt ikke antage, at 'multiverset' består af en mængde af universer i den forstand som 'universer' ofte betegner, nemlig en samling love, noget rum og noget tid. Lad os endeligt ikke antage, at det er den eneste form for selvstændige eksistenser, der findes.) ..Hm, faktisk tror jeg, jeg vil bruge et nyt begreb om.. Hm, eller..?.. ..Nå, det vender jeg tilbage til. Men, hvad dælen beskriver denne fundamentale logik så? Ja, det er så her det store spørgsmål virkeligt ligger. Hvad beskriver den fundamentale logik? Den kunne f.eks. beskrive objekter, så som strygejern, computere, jordkloder og hele universer, og så ville vores princip om, at "alt hvad der kan eksistere, må eksistere (i symmetrisk forhold)," føre til at alle "ting" eksisterer. Så det vil særligt sige, at alle mulige universer må eksistere. Jamen det lyder da meget godt, især hvis man er materialist i forhold til spørgsmålet om bevidsthed. Men selv da kunne man så også spørge, eksistere strygejern virkeligt side om side og på lige fod med hele universer? Er multiverset ikke bare en samling af universer? I så fald må man jo hypotisere, at den samlede fundametale logik om alt har en særlig "klausul" om, at det der kan eksisterer, er "universer," hvordan man så lige skal definere det begreb helt præcist. Nå, men det er detaljer: Overordnet set har vi altså bare en mulighed for, at genstandene for eksistens i multiverset --- med andre ord de ting, der kan eksistere --- er, ja, "ting." "Genstande." Fysiske objekter med andre ord, og muligvis altså yderligere begrænset til kun at indebære, hvad vi kan tænke på som 'universer' i en ret gængs forstand af ordet (altså samlinger af love, materie, rum og tid). Der er dog også en anden vigtig mulighed, især hvis man mere er såkaldt idealist frem for materialist, og der er, at genstandene for eksistens i multiverset er: Bevidste (sammenhængende) oplevelser. I denne hypotese er universerne i multiverset altså ikke en samling af rum, tid, materie og tilhørende love for, hvordan dette forløber, men af bevidste oplevelser (jeg vil skrive Oplevelser med stort fra nu af), der så også har nogle love for, hvordan de forløber. Da sanseinput er en stor del af Oplevelser, så vil sådanne Oplevelser jo også skulle indeholde beskrivelser/"love" for, hvad der sanses i Oplevelsen, og disse "love" kan jo så indebære de fysiske love i vores universer. Så hvis vi tænker på vores eget univers, så er der tydeligvis nogle love for, hvordan objekter bevæger sig og udvikler sig i tid. Med den.. objektorienterede hypotese om genstandene for eksistens, så vil disse love være "indskrevet" i multiverset direkte om objekterne, hvorimod i den bevidstheds-/Oplevelse-orienterede hypotese, der findes de samme love også for et univers, men de hører så i stedet bare ind under, hvor jeg-personen/erne oplever. To sider af samme sag. Ingen af os kan empirisk afgøre, om vores univers har love, der tager udgangspunkt i objekter eller om det tager udgangspunkt i de bevidste oplevelser i det. Hardcore materialister vil nok hælde mest til, at universer er orienteret omkring love, da disse per definition ikke ser noget problem i, at bevidste oplevelser bare opstår af sig selv, men alle os andre, der synes, at der ligger noget mærkeligt i tanken om, at bevidste oplevelser bare kan opstå af genstandes bevægelser (og denne forundring bliver kun forstærket, når man dykker grundigt ned i kvantemekanikkens verden, skulle jeg hilse og sige) *(Det skal så dog siges, at jeg hele mit liv selv har hældt mest til materialisme, selvom jeg kunne se nogle store spørgsmål ved det, som er svært at svare på, og at jeg først opdagede det elegante ved idealismen (som jeg slet ikke vidste, det hed på det tidspunkt) der i sommeren 2019, hvor jeg pludselig fik en række åbenbaringer om dette emne, bl.a. også omkring hvad der så svarer til den eksisterende teori om CUH.), jamen vi vil naturligvis være så meget mere desto åbne over for et oplevelse-/bevidsthedsorienteret multivers, fordi dette løser hele den problematik automatisk: I et oplevelseorienteret multivers skal oplevelserne ikke opstå fra noget andet, men i stedet er de der fra starten af, og man kan så nærmere sige, at objekter "opstår" ud fra dem (fordi objekterne altså så kun eksisterer i det omfang, at de bliver oplevet af en bevidsthed). (13:42)

Man kan så selvfølgelig ogå hypotisere, at multiverset har begge ting som genstand for eksistens, fysiske objekter og bevidste oplevelser, og så vil det så bare være spørgsmålet, om man er materialist eller ej, der afgør om førstnævnte så også fører bevidste oplevelser med sig indirekte eller ej.

Okay, så det var en ret vigtig og grundlæggende opdeling i mulighederne ved, hvad den fundamentale logik om alt har som genstand for, hvad kan eksistere i det samlede multivers. Et andet vigtigt spørgsmål omhandler så subjekterne for Oplevelserne. (13:46) ...(14:10) Hvem (eller hvad) er jeg-personerne i fortællingen med andre ord. En hardcore materialist tror som bekendt ikke på, at der er en sjæl som er genstand for de oplevelser, som objekterne producerer, og en sådan vil derfor nok sige, at det er simpelthen ikke er nogen.. hm, vi kunne sige, at der ikke er noget "modul," der indgår i oplevelsesskabelsen, som oplever Oplevelsen; en oplevelse oplever bare sig selv. Hm, jeg vil meget nødigt kalde disse moduler for "sjæle," for vi har desværre nogle associatioer til dette begreb, der er uhensigtsmæssige i visse sammenhænge (altså med visse hypoteseantagelser).. ..Hm, lad os kalde det en "oplevergenstand" her.. ..Nej, et "oplevelsessubjekt." Ok. Nå, og det er så slet ikke kun materialister, der kan have denne opfattelse. I hypotesen, hvor den fundamentale logik beskriver Oplevelser som genstande for eksistens er det jo også ret unødvendigt at have eksistensen af en helt trejde ting, nemlig et oplevelsessubjekt, for at Oplevelserne kan udleves. En måske mere naturlig opfattelse (det synes jeg i hvert fald) er nok, at det bare ligesom er multiverset selv, der udgør det samlede "opevelsessubjekt," og at alle eksisterende oplevelser derfor bare opleves af.. ja, af muliverset, eller af den "fundamentale logik for alt," kunne man også tolke det som. Men der findes altså også en mellemvej, hvor at multiverset indeholder mere end ét oplevelsessubjekt, som vi altså ofte kan tænke på som "sjæle" (men ikke i alle henseender). Og hvis multiverset indeholder en samling (meget vel en uendelig samling) OS'er.. Hm, lad mig bare kalde det Subjekter med stort S fra nu af.. Hvis multiverset indeholder sådan en samling Subjekter, så afhænger det så af, hvorvidt multiverset er oplevelses- eller objektorienteret (eller en blanding), om hvert Subjekt så tilknytter sig en Oplevelse i multiverset, eller om de tilknytter sig en "hjerne" (lad mig skrive Hjerne fra nu af), som altså skal forstås i en meget bred forstand af ordet (maskiner kan f.eks. også være Hjerner, og det kan alt muligt andet også (medmindre man specifikt begrænser sin hypotese for multiverset herom)), i et specifikt fysisk univers af objekter. I denne todelte hypotese vil vores oplevelser altså hver især være et produkt af, at der "sidder" et specifik Subjekt et eller andet "sted" i multiverset --- enten i et abstrakt rum eller i et fysisk rum, muligvis lige oven i din Hjerne, som den oplever fra (i hvilket tilfælde "Subjekt" netop bliver helt ækvivalent med vores normale forståelse af begrebet "sjæl") --- og er så i færd med at opleve, de vi oplever ligenu, og nærmere bestemt er vi hver især det Subjekt og den er os. Vi skal senere diskutere noget mere om, hvad dette betyder for os. Men lad os bare her påpege, at det sjove ved denne hypotese, hvis vi kan sige det sådan, er at hvis man har to universer, der indeholder samme Hjerne med same tidsudvikling, eller hvis man i et oplevelsesorienteret multivers har to oplevelser, der i en vis periode er helt identiske med hinanden, så vil helt den samme personlighed, med de samme tanker og helt den samme selvforståelse, opleves af to forskellige entiteter.. Tja, det kan man jo sige alligevel.. Nå, men.. Ja, lad mig bare lige nævne her i stedet, at lige netop denne (todelte) hypotese kan føre til, at man kan bekymre sig om døden, for hvad skal der så ske med ens "sjæl" (ens Subjekt) bagefter? Dette problem har materialisterne og idealisterne der tror på at alle Oplevelser udleves af et stort samlet Subjekt (som man meget vel kunne tænke på som multiverset selv); for dem er to identiske (del-)oplevelser, der udleves i multiverset, også identiske i forhold til, hvad de betyder for den samlede mængde af oplevelser, nemlig fordi der ikke i disse hypoteser vil findes nogen skjult variabel, så at sige, der afgør om Oplevelsen i givet fald opleves af det ene eller det andet Subjekt i multiverset. For dem, eller rettere for os, for jeg er selv idealist med tro på ét samlet Subjekt) er alle oplevelser ligeværdige, for vi vil så mene, at vores egen nuværende oplevelse er en del af ét samlet hele, og at vi altså ligeså meget er en del af alle andre Oplevelser i multiverset, som den Oplevelse, vi selv føler at vi lever lige nu (hvad "nu" så end betyder helt præcist, men det kommer vi til).

Nå, og nu fik jeg så lige akkurat teaset den næste store opdeling, man kan have i hypotesen for multiverset, og det er nemlig i forhold til, om der findes en form for en global tid eller ej. (14:47) ...For materialister er det meget naturligt at antage, hvis ikke en global tid for det samlede multivers, så i det mindste lokal tid for hvert enkle univers --- ja, det er nærmest uundgåeligt. Og herfra er der så ikke megt i vejen for videre at antage, at der også er en global tid. For de idealistiske hypoteser er spørgsmålet derimod en smule mere indviklet. Her er der nemlig ikke et behov for "tid" andet end som noget, der er subjektivt for hver Oplevelse. En oplevelsesorienteret multivershypotese kan således godt bare antage, at "tid" er et rent subjektivt begreb, og at alle Oplvelser (hvad end de bliver oplevet af individuelle, adskildte Subjekter eller af et stort samlet Subjekt) bare er, og at de via deres væren (som altså så er konstant så at sige, i og med der ikke findes nogen egentlig tid (eller måske bare ikke en "tid" som svarer til, hvad vi normalt forstår ved begrebet)) bare resulterer i at de ligesom konstant udlever sig selv, så at sige, eller bliver udlevet af de Subjekter, der har knyttet sig til dem. Dette er faktisk ret dejligt, for konceptet om Tid kan også i sig selv godt virke lidt mærkeligt, lidt ligesom da vi snakkede om, at konceptet om, at bevidsthed skulle opstå af genstandes bevægelser, også kan virke mærkeligt. Så det er dejligt, at der findes Eksistens-hypoteser, hvor tid er et rent "subjektivt" fænomen, så at sige. Nå, men idealistiske hypoteser kan nu også godt indeholde koncepter om tid. For eksempel kunne man have en specifik idealistisk hypotese, der sagde at alle oplevelser udleves samtidigt i multiverset --- enten af et stort samlet Subjekt eller af adskilte, individuelle Subjekter hver især --- og at den globale Tid i multiverset så dermed bare måles i den subjektive tid som hver Oplevelse har. Med andre ord kunne man fortolke dette som, at multiverset indeholder en (uendelig) række af Oplevelser, som multiverset, hvis vi personificerer dette i denne metafor, så ligesom "trykker play på" i Tidens begyndelse, og så kører de ellers hver især samtidigt i henhold til deres subjektive tidsopfattelse. En anden mulighed kan være, at der til hver Oplevelse også er tilknyttet en vis "regnekraft" så at sige, og at man metaforisk set så kan sammenligne hver af de "afspilne" Oplevelser som en slags computer, der regner på, hvordan Oplevelsen forløber. En oplevelse, der foregår i et stort univers med meget materie i og med "regnetunge" fysiske love, vil så "afspilles" langsommere end en oplevelse, der foregår i et mindre "regnetungt" univers. (I øvrigt kunne man også have den hypotese, når det kommer til at objektorienteret univers: Her kunne man også stille alle universerne på række og så sige, at den globale tid ikke svarer til de lokale tider, men i stedet afhænger af, hvor "mange udregner skal klares," så at sige.) Disse tanker svarer altså til en antagelse om at den fundamentale logik ligesom skal bruge tid på at udregne sig selv, og så at sige opdage flere og flere sandheder om sig selv, hvilket, når vi siger det på den måde (og ikke snakker om det som om hvert univers/Oplevelse udregnes af en computer), så lyder det jo faktisk pludselig slet ikke helt så dumt. Og med den grundlæggende fortolkning, så hører dette faktisk også med til min egen yndlingshypotese, nemlig at den fundamentale logik om alt ligesom fra Tidens begyndelse opdager flere og flere sætninger om sig selv, og at det er i takt med, at den opdager (og nu tillader vi os altså lige at personificere den her) disse sætninger, så udlever den så også de Oplevelser, som sætningerne omhandler. *(Det skal dog siges, at jeg også synes rigtigt godt om flere andre hypoteser.) Nå, men det vender vi tilbage til. Ellers skulle man ikke tro, at denne forskel gør så meget, nemlig om den globale Tid, hvis der er en, afhænger af "udrengernes" kompleksitet eller ej, men faktisk så giver det meget muligt en forskel i den samlede 'prior'-sandsynlighed, som det hedder. Som et sovt lille eksempel på dette, så kan man faktisk teoretisere omkring, om det faktum, at vores univers er relativistisk, måske ligesom kunne skyldes, at det hermed så faktisk kan have uendelig størrelse, uden at det er uendeligt komplekst at "regne på"/"simulere," fordi man i et relativistisk univers kan tillade sig at regne på/simulere alt ved at starte i et enkelt punkt og så regne på alt med udgangspunkt i en lyskegle derfra. Hermed bliver et uendeligt tung simulering faktisk til en endelig tung simulering. Man kunne også nævne mange andre sjove ting i denne sammenhæng, men lad os bare stoppe her, for det er lidt et sidespor ift. det overordnede tema her. (15:37)

Nå, det næste man så kunne tage fat på, det er så sprøgsmålet, hvis vi specifikt snakker de oplevelsesorienterede multivershypoteser, og det er hvordan.. hm, hvordan de beskrives, men det bør næsten komme i et helt nyt afsnit, for nu bevæger vi os så videre til noget helt nyt, og det er at fundere over, ..ja, over "strukturen" af den fundamentale logik så at sige.. Hm, det er lidt en stor mundfuld, men jeg tror muligvis, der er en god, hurtig vej igennem det, så lad mig lige tænke mig om først... (15:42) ..(Okay, men hjerne skal også lige bruge en god pause, tror jeg...) (15:47) ...(16:02) Ah jo, jeg tror godt nogenlunde, jeg ved, hvad jeg vil sige.. ..Ja.. Men jeg synes næsten, emnet fortjener, at jeg skriver det færdigt i morgen, når jeg er mere frisk igen --- hvilket jeg i øvrigt godt tror, jeg kan; de næste "afsnit" behøver nok ikke at blive så lange..:) 

(06.01.23, 11:45) Okay, det var rigtigt godt, at jeg lige tog aftnen til at tænke mere over emnet, for nu kom jeg i tanke om nogle andre vigtige ting. I forbindelse med de objektorienterede multivershypoteser, så fik jeg kun snakket om genstande/objekter og universer, og fik så også snakket om Subjekter, altså hvad vi nærmest kan tænke på som en slags "sjæle," selvom der dog følger flere antagelser med, hvis vi kalder det 'sjæle,' som vi ikke ønsker at antage om Subjekter. Men ja, jeg fik jo så udeladt den mulighed, at det kun er Subjekter, der eksisterer, og at fysiske universer og objekter bare er noget som de ligesom "tænker frem," så at sige. Sådanne hypoteser indeholder jo også de hypoteser, der siger at multiverset består af en mængde guder, som hver især står for at skabe fysiske universer, samt udleve de tilhørende oplevelser i de universer. Så altså også en rigtig vigtig gruppe af hypoteser at få med. Og i sidste ende bør det også nævnes, at når vi kommer til at antage, at "alt hvad der kan eksistere i den fundamentale logik, gør det," så får vi jo faktisk et samlet multivers hvor en "logik" ligesom skaber ting spontant. Aha, men kunne man så ikke også i stedet forestille sig et multivers af flere end én "logik," hvor hver "logik" hver især så skaber universer/delmultiverser og skaber og udlever de tilhørende Oplevelser (..eller skaber Subjekter, som så udlever dem)? Jo, det kunne man selvfølgelig godt, men i så fald så falder disse hypoteser jo faktisk sammen med de hypoteser, der siger at Subjekter er de fundamentale genstande for eksistens.. Nå nej, ikke helt, vent lidt.. ..Hm, hvis vi ser på hypoteserne, hvor en fundamental logik skaber alt og også udelver alt selv, og hvor der altså ikke er individuelle adskildte Subjekter, men hvor alt opleves af multiverset/"logikken" selv.. Hvis vi tager de hypoteser og omdanner dem, så der nu er flere "logikker" i stedet, så svarer denne mængde af hypoteser ret meget til.. Nej, den indgår i mængden af hypoteser, hvor kun Subjekter er genstande for eksistens, nemlig hvis man tillader sig at bruge en bred definition af, hvad Subjekter kan være (og hvorfor ikke, for det er et super abstrakt begreb i forvejen), således at det også inkluderer "logikker." Hm, og hvad så med de hypoteser, hvor den fundamentale logik også skaber individuelle Subjekter, der er adskildt fra alt andet..? ..Tjo, men her kunne man udvide.. Tja, never mind, det er også lige meget; lad os bare inkludere den mulighed som en selvstændig ting, og lad os bare notere os, at det under visse antagelser også kan svare til en gruppe af hypoteser som hører til den mængde, hvor Subjekter er de fundamentale genstande for eksistens (altså de hypoteser, hvor "guder" (hvis man fortolker dem sådan) er de fundamentale genstande for eksistens, nemlig fordi man her kan omfortolke "logikkerne" til at være det samme som "guder;" at en "gud" er en "logik"). 

Okay, så det var rigtig godt lige at få de former for mulige hypoteser med. 

Nu kommer vi så til at tale om, hvad man så får ud af at antage at "alt hvad der kan eksistere, eksisterer" ovenpå de beskrevne hypoteser om, hvad kan eksistere. Og dette bliver så faktisk et relativt kort afsnit, for det korte af det lange er, at det ville kræve en forståelse af, hvordan den fundamentale logik om alt er.. "struktureret"/"ordnet," før man ville kunne sige noget præcist om vores univers-/Oplevelse-prior-sandsynligheder (og her må man altså lige læse lidt sandsynlighedsregning og statistik for at forstå, hvad menes med 'prior-sandsynligheder'). Og det kan vi jo aldrig komme til. Men! Vi kan teoretisere os frem til nogle ting, bl.a., og dette er rigtigt vigtigt, at man med nogle ordninger vil opnå det, der (desværre allerede er opfundet af en anden;)) kaldes 'Mathematical Universe Hypothesis' (MUH), eller hvis man skal være mere præcis (for selv ham, der postulerede idéen er vist gået over til at fokusere på følgende også): 'Computable Universe Hypothesis' (CUH). Begge teorier (altså teorierne omhandlende hypoteserne) handler så om, at man ved at antage, at alt hvad der kan beskrives i en (ordnet) matematisk teori eksisterer, faktisk nok for en høj frekvens af ikke-kaotiske universer som vores eget i multiverset. (Mere specifikt en høj frekvens af universer, som kan beskrives med relativt lidt information.) Og CUH præciserer så bare og siger: Vi er ligeglade med ikke-konstruerbare matematiske objekter (og hvorfor skulle nogen også kære sig om dem, andet som en filosofisk beskæftigelse? (jeg er matematisk konstruktivist, kan man høre)). 

Men for at nå CUH, så kræver det altså, at der er en vis ordning i den fundamentale logik, samt en ordning i.. ja, i hvilken rækkefølge at Oplevelserne bliver udlevet (men hvor man dog godt i princippet kan have, at et endeligt antal Oplevelser kan udleves på én gang). I mange hypoteser kan dette skabe nogle store problemer. Men det gode er så, at man altid kan sige, at, jamen, bare fordi vi med vores jordlige (er det et ord?..) matematik ikke har mulighed for at definere et fornuftigt sandsynlighedsrum, hvis ikke alle Oplevelser i multiverset er ordnet på en vis måde, så er det jo ikke ensbetydende med, at multiverset ikke selv kan.. ja.. se ud på en fornuftig måde. ..Bare fordi vi vil opnå logiske paradokser, hvis vi prøver at tildele sandsynligheder til noget, der er udvalgt fra en uendelig mænge, så betyder det ikke at multiverset behøver at indeholde paradokser, hvis det nu f.eks. indebærer, at uendeligt mange Oplevelser udleves "samtidigt" --- eller hvis de udleves i rækkefølge, forresten, med hvor prior-sandsynlighederne bare aldrig konvergerer.. Så ja, selv hvis man ikke lige kan finde, eller ikke lige synes om de hypoteser, hvor alle Oplevelser er ordnet pænt, så betyder det ikke at multiverset ikke godt kan følge de hypoteser, uden at det bryder sammen. Vi kan så selv pålægge nogle antagelser til de hypoteser, der får prior-sandsynlighederne til at konvergere alligevel, og her er det så bestemt værd at nævne, at man herved alt andet end lige sikkert også vil komme frem til CUH i sidste ende. 

Der er dog også et problem til den anden side, og det er, at nogle hypoteser fører til et komplet kaotisk univers. Disse problemer er dog helt anderledes, for der kan man bare sige, at fordi vores eget univers/vores egen Oplevelse ikke er komplet kaotisk, så må man forkaste de hypoteser, der siger, at det/den/de bør være det. Dette forklares nemmest, hvis vi ser på et eksempel. ...(13:04) Hvis vi ser på et idealistisk multivers, hvor det er (bevidste) Oplevelser, der er genstand for eksistens, så er det betydende for Oplevelsernes prior-sandsynlighed, hvordan Oplevelserne er "beskrevet" i den fundamentale logik, så at sige. Hvis en signifikant delmængde af alle Oplevelser er beskrevet med udgangspunkt i en Hjerne, hvor man altså ser på de fysiske bevægesler i en Hjerne (som dog med idealistiske antagelser kun eksisterer i kraft af den bevidste Oplevelse og ikke omvendt), og hvor Oplevelsens forløb så afhænger af disse bevægelser.. Hvis en signifikant delmænge af Oplevelserne i multiverset er beskrevet på den måde, så vil ikke opnå komplet kaos i multiverset, og hypotesen kan således ikke forkastes. Men hvis vi i stedet antager til vores hypotese, at alle Oplevelser i multiverset er beskrevet lidt som et slags computerprogram, hvor hver linje beskriver en ny følelse i rækken, som Oplevelserne følger, så vil der jo herved blive komplet koas, når man så antager (\emph{hvis} man altså antager), at "alt hvad kan eksistere, eksisterer." Så ville der være 0 orden i alle Oplevelser og alt ville være koas og tilfældigt. Enhvert udsnit af en Oplevelse, hvor denne indebærer en følelse af orden, vil så med al sandsynlighed hurtigt erstattes af nget komplet kaotisk igen. Og selv hvis man prøver at pålægge, at kun Oplevelser, hvor der er en sammenhængende selvforståelse, der gennemgår Oplevelsen, er gyldige, så vil dette stadig ikke kunne forklare, hvorfor vores omgivelser ikke går amok omkring os. Så alle sådanne hypoteser kan vi altså udelukke.

Dette er i øvrigt også interessant i en anden henseende, for nogle af modargumenterne mod materialisme går bl.a. ud på, hvis vi forestiller os.. Ja, der findes en vis xkcd, hvor en mand går i en ørken og flytter sten for at simulere vores univers. ..To sek.. ..Nummer 505, A Bunch of Rocks, hedder den. Så kan man så spørge, hvad ville der ske, hvis han gjorde det to gange? Hvad ville der ske, hvis han gjorde det to steder samtidigt, måske forskudt med en lille tidsforskel eller ej? Og slutteligt, hvad hvis han bare havde to sten hvert sted, som han mere eller mindre flyttede samtidigt? Nå ja, og helt slutteligt, hvad hvis det i stedet var bunker af sand, han flyttede rundt på, måske endda hvor nogle sandkorn faldt fra og nogle kom til i bunkerne, når han flyttede dem. Disse spørgsmål klarer de Oplevelses-orienterede multivershypoteser jo nemt, hvor der definerer hver Oplevelse jo bare selv, som en del af dens "naturlove," hvordan dens Hjerne defineres, samt hvordan denne bevæger sig og udvikler sig i tid. Hm, jeg kan dog nævne, at jeg lige her i går kom til at tænke på, at man måske kunne slippe af sted hvs man prøvede at definere en materialistisk hypotese, hvor man gør brug af entropi og koncepter om, hvad definere information, hvrnår information er unikt, og.. ja, og ting i den stil, men hvem ved? måske løber man bare ind i andre paradokser/svære spørgsmål herved.. Anyway, det jeg egentligt ville hen til, det var at jeg kan huske, at vi på et tidspunkt snakkede om dette i forbindelse med VT (videnskabsteori og etik (for fysikere)) på fysik, hvor en af mine venner fra fysik sagde, at han så (vist nok; sådan husker jeg det i hvert fald) troede på, at to identiske.. ja, "Hjerner" med identisk udvikling bare producerer netop én bevidst Oplevelse i multiverset. Elegant svar. Men nu kan jeg jo så se, at der faktisk er et stort problem med dette svar, for medmindre vi begrænser multiverset til noget meget endeligt, så vil alle mulige Hjerner jo forekomme, hvilket vil sige at alle mulige Oplevelser, der afviger fra hinanden vil forekomme netop én gang i multiverset/den samlede Eksistens. Men dette vil jo derfor medføre en komplet kaotisk prior-sadsynlighed for alle Oplevelser, og denne hypotese går derfor faktisk ikke, interessant nok. (13:46)

(15:10) Hov, jeg har også helt haft glemt noget andet virkeligt vigtigt. Når jeg har skrevet om de objektorienterede/materialistiske hypoteser ovenfor, så har det måske lydt som om, at materialismen har nogle ting, den ikke kan forklare, som Oplevelsesorienterede hypoteser kan forklare, men sådan er det nu slet ikke. Jeg synes personligt, at de Oplevelses-orienterede hypoteser gør det en anelse mere elegant, men det er bare en personlig holdning. For hvis vi nu starter med at se på den her hypotese, som jeg beskrev, med at hver Oplevelse har i/med sig en beskrivelse af/nogle love for, hvordan Oplevelsen starter og udviler sig i tid, eksempelvis ved at definere en Hjerne (muligvis sammen med en større samling af objekter, som Hjernen er en del af, nemlig et fysisk univers) samt nogle love for, hvordan bevægelsen af information i den Hjerne (hvor 'Hjerne' altså er et fuldstændigt abstrakt begreb, og kan endda indebære et helt univers f.eks.) fører til en (eller flere) bevidst(e) oplevelser. Men i de objektorienterede hypoteser kan man jo i stedet bare have nogle love, ved siden af lovene der beskriver, hvordan materie i universet bevæger og udvikler sig, som så beskriver, hvordan bevidste oplevelser kan opstå ud fra disse fysiske genstande. Hvis vi så tænker på xkcd-eksemplet (A Bunch of Rocks), så kunne der altså bare være nogle universer, hvor to af hver sten vil føre til to adskildte Oplevelser, nogle hvor de kun vil føre til én, osv (men hvor alle de fysiske love måske er de samme, og hvor startkonfigurationen af universet også er det samme; bare hvor lovene for de resulterende Oplevelser produceret af den fysiske materie er forskellige). Så ja, det kan sagtens lade sig gøre at give et klart svar på, hvorfor fysiske objekter kan føre til bevidsthed i et ellers overvejende objektorienteret multivers, og som altså ikke bare antager hardcore materialisme og siger: "jamen det sker bare helt automatisk, nemlig at når man har en fysisk Hjerne et sted, der kan have en bevidst oplevelse, så har den det også." Men ja, jeg synes så dog, at de Oplevelses-orienterede hypoteser klarer denne del mere elegant, end de overvejende objektorienterede multiversehypoteser, hvor man så indfører Oplevelses-love oveni, ved siden af de "fysiske love" i de indeholdte universer. (15:29)

Okay, nu når vi så til et afsnit, hvor jeg bare lige siger et par ting om, hvilke af de hypoteser, vi har set på, som jeg selv synes er ret nice, og som jeg tror mange sikkert vil kunne finde fornuftige i større eller mindre grad, og derefter kommer så det sidste afsnit, hvor vi ser på konsekvenserne ift. multiversets Subjekter (du og jeg og vi), og også på nogle pointer omkring moral.

Lad mig starte med at pointere, at en hypotese, hvor der er én grundlæggende (og "ren") logik om alt, og hvor alt så forekommer i takt med at denne logik ligesom "opdager flere og flere sætninger om sig selv," så at sige, og dermed også forstår hvordan flere og flere samlede oplevelser må føles.. At denne hypotese faktisk muligvis kunne give et matematisk regnestykke for prioren, ikke som vi kan finde frem til nøjagtigt, selvfølgelig, men hvor vi kan sige, at dette regnestykke faktisk på fornuftig vis godt kunne indeholde en ordning af alle Oplevelser, således at vi faktisk (med vores "jordlige" matematik) ville kunne tillægge en prior-sandsynlighed til hver Oplevelse i princippet. Lad mig prøve at omformulere dette.. ..Det er ikke ufornuftigt med en sådan hypotese, at teoretisere, at multiverset i princippet kunne indeholde en orden, en rækkefølge, kunne vi også sige, hvor alle Oplevelser (og jeg kan som man måske kan gætte sig til godt lide at antage, at hver Oplevelse er endelig --- det er i hvert fald en god antagelse, hvis man gerne vil nå frem til, at der må være en i princippet udregnelig prior i multiverset på denne måde..).. hvor alle Oplevelser udleves mere eller mindre én efter hinanden. Okay, kan jeg sige dette endnu mere klart..? ..Whatever, måske er dette underemne bare for komplekst, således at vi må gemme det til endnu senere (og at jeg altså ikke vil tale så meget om det i min første artikel om emnet). Men det korte af det lange er bare, at jeg altså tror, at der findes hypoteser, der (i hver fald for mig, og sikkert for mange) lyder ret fornuftige, og som kan føre til en fornuftig antagelse om, at multiverset har en pæn ordning af alle dets Oplevelser (også selvom rækken er uendelig), som gør at man matematisk (vores vores "jordlige" matematik) kan tilægge en prior-sandsynlighed til hvert univers/hver Oplevelse. Ok. Jeg vil ikke sige meget mere om, hvorfor jeg tror dette, men jeg bliver dog nødt til lige at nævne her, at jeg i går kom lidt i tvivl om fornuften ved dette, for hvordan skal en stor, samlet, "Ren" logik om alt lige vægte forholdet imellem, hvor lang tid sætninger "tager" at udlede (altså hvor mange logiske skridt, der går til udledningen), og hvor meget information sætningerne indeholder, når de skal ordnes. Nu ved jeg godt jeg "vrøvler" igen, så lad mig lige se på, om ikke jeg kan omformulere dette mere klart..  ..Hm, jo: Hvis vi ordner alle matematiske sætninger i en teori, f.eks. mængdelære, ud fra, hvor mange logiske skridt det tager for at udlede dem, så går det ikke ift., hvad vi ønsker a opnå, for så vil der (så vidt jeg lige kan se) blive uendeligt mange sætninger i hver (skridtantal-)kasse. Men hvis man så tilgengæld indfører, at det også koster nogle skridt at læse lange sætninger, f.eks. hvis man har en lang antecedent som skal sammenlignes med en vist sætning i et modus ponens-skridt --- ret meget som om det foregik på en Turing-maskine (eller anden maskine), jamen så vil der pludselig blive endeligt mange sætninger i hver (skridtantal-)kasse. Problemet bliver så, at dette giver noget arbitrært til den "Rene" fundamentale logik, men ja.. Hm.. ..Okay, lad mig bare stoppe her, for det bliver hurtigt vildt kompliceret.. ..Det næste man så kunne tage hul på, det er at sige: "jamen hvad så hvis der så bare er en undelig mængde af fundamentale logikker," hvorved hver "logik" så kan tildeles en matematisk veldefineret prior for dets Oplevelserne, men så render man så bare ind i spørgsmålet: "Hvad hjælper det at der er lokale eldefinerede prior-sandsynligheder for hver 'logik,' hvis de samlede Oplvelser i multiverset, når man sætter det hele sammen, så stadig giver en svært-definerbar samlet prior for hver Oplevelse".. ..Hm.. ..Hm, måske skal man bare give op på, at få en matematisk veldefineret prior (ikke at vi nogensinde ville kunne regne den ud alligevel). I så fald kan man dog godt måske sige, at hypotesen, som jeg beskrev her lige ovenfor, nemlig med en helt fundamental "logik," der ligesom udleder (og "forstår"/"føler"/"oplever") sætninger omkring sig selv, "kommer tæt på," hvis det giver mening.. Det synes jeg i hvert fald lidt det gør..

Nå, men ellers har vi altå ogå bare rigtigt mange andre gode kandidater, må man sige. Jeg kan personligt faktisk også rigtigt godt lide den, hvor det ligesom er en masse "logikker"/"guder," der ikke har et mål med deres tankevirksomhed, og nok ikke har en selvbevidsthed på samme måde som, hvad vi forstår ved selvbevidsthed, som bare fremtænker universer, nærmest som en konsekvens af, hvad man kunne kalde en simpel nysgærrighed --- eller hvis man tænker mere "logikker" frem for "guder," så bare fordi at, jamen det er bare det fundamentale logikker gør; udvikler sig selv og "opdager" (og oplever) sætninger i sig selv. ..Og i sidste ende, så kan jeg egentligt også ok godt lide den hypotese, der bare siger: Alle mulige "objekter" eksisterer, og så forholder det sig i øvrigt bare sådan, at 'objekter' ikke bare indebærer dumme genstande, der flyver rundt og passer sig selv, men at 'objekter' i vores multivers også kan indeholde nogle definitioner af Oplevelser, som så bliver udlevet, enten ved at multiverset indeholder Subjekter samt nogle love for, hvordan disse Subjekter kan opleve ting, eller fordi der i hvert univers simpelthen bare er plads til, at der ved siden af de fysiske love også står nogle love, der simpelthen bare definere, hvornår og hvordan diverse Oplevelser bliver udlevet i universet (altså et sæt love, som vi nærmest kunne kalde "sjæle-love"). En todelt objektorienteret multivershypotese, som også bestemt lyder ret fornuftig i mine ører, selvom jeg dog selv hælder mere til de første, jeg har nævnt her, som ikke er ligeså "objektorienterede." 

Ok. Jeg synes, vi slap nogenlunde godt igennem det. Så når vi til det sidste afsit, som i høj grad handler om spørgsmålet: Hvordan skal vi forholde os til "døden?" (Ikke at vi vil besvare dette spørgsmål eksakt, men det er altså i høj grad temaet for afsnittet.) Og et andet spørgsmål for afsnittet er også: Hvordan skal vi forholde os til moralspørgsmålet. (16:16)

Det korte af det lange, hvis vi snakker omkring "døden," det er at langt de fleste af de hypoteser, inklusiv alle dem, der garanteret er/vil være mest udbredte hos folk, er at, hvad vi normalt betegner som "døden" ikke rigtigt har nogen betydning. For hvis multiverset er uendeligt, så vil vi leve igen og igen og igen i alle mulige afskygninger af os selv, og dette gælder så f.eks. både hvis man er hardcore eller semi-materialist, eller hvis man tror at ens eget Subjekt, som jeg har kaldt det ovenfor, ikke er fundamentalt adskildt fra andre Subjekter i universet, men bare er en del af det store hele, hvad end "det store hele" så er en "gud" eller en "logik" for ens univers. Så hvis man altså ikke er tilbøjelig til at tro, at vi har en "sjæl," jamen så når man den konklusion (at vi skal lave alle afskygninger af vores liv --- og af alle andres liv, men det kommer jeg til om lidt), og hvis man tror på, at vi har en slags sjæl, men at den sjæl enten bare er en del af Gud (eller en del af noget andet meget grundlæggende i universet og/eller multiverset som helhed), eller vi returnere og blive en del af Gud efter døden, og således også blandes sammen med alle andre nuæevende sjæle, jamen så når man også samme konklusion. I sidstnævnte tilfælde (især hvis man også forestiller sig at der er en global Tid i multiverset --- eller bare en lokal til, som den lokale "gud" eller "logik" også følger) så il det jo nemlig være naturligt videre at antage, at når en ny person fødes i universet, jamen så tages der lidt af "Gud" igen til at danne en sjæl igen, og således vil ens nuværende sjæl altså fordeles ud på alle andre personer/Oplevelser, som leves efter en selv. Og medmindre Gud/"logikken" der foresager det univers, vi lever i, er utroligt begrænset, så vil der være mange universer, som dette væsen foresager, og der vil altså aldrig ophøre med at være liv. Jeg tror allerede disse antagelser dækker, hvad rigtig mange mennesker ville synes giver god mening. En anden antagelse, som ikke har så meget med "alt hvad der kan eksistere, eksisterer"-hypotesen at gøre, men som måske også ville være udbredt hos folk, det er at sige: Multiverset er faktisk ret begrænset, men jeg tror på en Gud, og at man sjæl når op til ham og bliver passet på ham efter døden. Og ja, denne antagelse gør jo selvsagt også frygten for døden ret irrelevant --- ja, medmindre man i stedet tror på, at man skal i helvede, men det er nu nok de færreste, der slås for alvor med den tanke, for hvis man er en person, der tager den tanke seriøst om sit eget "efterliv," så vil man nok bare prøve at leve mere fromt og så håbe på det bedste. Så ja, det virker virkeligt ikke som om, at nogle af de hypoteser, der nok vil være mest udbredte hos folk, vil føre til andet end, at man ikke behøver at frygte dødens kommen. Nu mangler jeg så bare lige at forklare mere om, hvorfor at vi ikke bare skal lave alle afskygninger af vores egne liv men også alle andres med de første hypoteser nævnt i denne paragraf, og så mangler jeg også at dykke ned i de få hypoteser, der antager "alt hvad der kan eksistere, eksisterer"-hypotesen, men hvor man stadig kan være urolig for "efterlivet".. Nå ja, og så mangler jeg også lige at sige: Der vil måske også være nogle få helt- eller sei-materialister, som af en eller anden grund tror på, at multiverset er ret begrænset. Men hvis bare multiverset indeholder universer som vores, og hvis det nu indeholder bare ét unvers, som bliver ved med at Big Crunch'e og udvide sig igen i en uendelighed, jamen så er det også med al sandsynlighed det vi lever i *(og alle andre døende universer vil med al sandsynlighed allerede være døde for en fantasilliard år siden), og så vil vi dermed også leve igen i alle afskygninger. (16:42)

Ok, nu til den der pointe om, at vi ikke bare skal lave "vores eget liv" i alle afskygninger, men også alle andre, og det er simpelthen fordi, at hvis vi antager at multiverset er uendeligt, så vil hver eneste mulige liv man kan forestille sig (og også sikkert vildt mange, som vi ikke kan forestille os;)) forekomme, nogle bare med virkelig lav frekvens i forhold til andre. Så hvis vi opstiller to liv overfor hinanden, så vil vi kunne finde udgaver af liv midt imellem de to liv på en kontinuer linje, hvor hvert liv vi plotter ind på linjen har en vis, større eller mindre frekvens ift. hvor ofter det forekommer i multiverset. Og vi kan sågar finde undeligt mange forbindelser på denne måde mellem to liv. Hvis man så spørger, hvad hvis den ene er en mand og den anden en kvinde, hvad hvis de bor på to forskellige planeter, hvad hvis de er af to helt forskellige arter (f.eks. hvis den ene eller de begge er en eller anden alien)? Jamen selv i alle disse tilfælde vil man kunne finde en glidende overgang, hvis altså man bare dykker dybt nok ned og tager fra de lavfrekvente livsforekomster i multiverset. Og målt op mod uendeligt vil selv ufatteligt lav frekvente livsforekomster forekomme, ja, uendeligt mange gange. Så på den måde indebærer "alle afskygninger af vores eget liv" simpelthen bare "alle afskygninger af mulige liv." 

Og hvis man altså dermed tror på, at multiverset ikke er begrænset, men er uendeligt ift. dets muligheder og dets forekomster, så når man altså ret nemt til, at "vi skal leve alle afskygninger af vores eget og alle andres liv igen og igen." Dette er dog medmindre man altså antager nogle ret specifikke ting, nemlig: At vi har en sjæl hver især, som er adskilte fra hinanden, og som aldrig smelter sammen igen på noget tidspunkt. At disse sjæle af en eller anden mærkelig grund også oveni købet er dødelige --- eller at de bare ligesom lever den samme meget begrænsede mængde liv igen og igen. Hvis man mener nogle af disse ting, så når man altså ikke nødvendigvis den konklusion. Jeg tror dog, at førstnævnte udgave, nemlig at vi alle har for altid adskildte sjæle, som dog er dødelige, vil være vildt sjælden at finde hos folk. At ens sjæl f.eks. lever det samme liv igen og igen, enten helt uden eller måske med nogle få variationer, den vil måske være lidt mere udbredt, men på den anden side kun slem, hvis man har haft et ligefremt dårligt liv. Men selv da, så vil mange nok hælde til den version, hvor der dog sker nogle få variationer gang på gang, og hvis man så dykker ned hypotesen herfra, så vil mange nok ende med at erkende, at hvis dette sker i al uendelighed, så vil variantionerne også ende med at blive uendeligt store, og så når man samme konklusion igen. Det kan man om ikke andet håbe.

Men ja, så det korte af det lange er altså, at hvis man tænker i dybden over multiverset afgrænsning, eller nærmere bestemt mangel på afgrænsning, så vil de fleste nok kunne blive ret afklaret med døden heraf. (Ikke at de fleste ikke allerede er afklaret med døden, men der findes dog alligevel også mange der frygter den på nuværende tidspunkt). 

Og så kan vi så slutte af med spørgsmålet om moral *(hov, jeg mener 'etik,' rettere), for det smukke ved disse teorier, er at når man når omtalte konklusion med at vi skal leve alle afskygninger af alle liv igen og igen, jamen så når man dermed også frem til en utrolig bogstavelig udgave af "what comes around goes around." Man udvisker altså herved helt forskellen på, hvad en filosofisk egoist vil mene er korrekt etik, og hvad en utilitarist (eller andre etikker, der fremhæver altruisme og "godhed") vil mene! Jeg vil altså påstå, at man, uanset hvordan man vender og drejer det (stort set), når frem til en etik der siger: Lev dit liv som om, at alt hvad du foresager af godt og ondt med andre (inklusiv andre arter og livsformer), det vil du selv opleve (med rollerne byttet om) i et efterfølgende liv, og bak om om at andre i dit samfund bør følge den samme etik. Og svaret på, hvorfor denne etik bør følges, er så både, at, jamen, dette er en god etik at følge for et samfund, men også at, jamen, antagelsen i den etiksætningen er sikkert også sand for all intends and purposes. ..Kortere sagt kan man sige: Lev dit liv ud fra en antagelse om, at du og alle andre skal leve hinandens liv i lige forhold i alle jeres efterliv.

Det skal så lige siges, at nogen vil pege på fri vijle og sige, at jamen, bare fordi jeg med min frie vilje gør skade/ondt på en anden person, jamen så betyder det ikke, at den/en anden person vil gøre skade/ondt på mig i efterlivet i de udgaver, hvor rollerne er byttet om. Denne opfattelse fordrer, at man tror at ens handlinger ikke kan forklares alene ved hjernens fysiske bevægelser, men at sjælen på en måde også sidder der med en slags joystick i overført betydning og påvirker, hvad hjernen gør. Hm, tja, det har jeg vel egentligt ikke så meget at sige til, når det kommer til stykket, for det er muligvis så langt væk fra min egen opfattelse af, hvordan virkeligheden fungerer, at jeg nok aldrig har tænkt så meget på at argumentere omkring de antagelser.. Hm.. Oh well, lad mig så bare slutte for nu, og så lade den diskussion stå åbent.. ..Hm, ah øv, den holdning kunne godt gå hen at blive problematisk, når det kommer til at enes om etiksprørgsmålet.. ..Hm, og måske også når det kommer til at trøste folk, der har levet et direkte dårligt liv, og er bekymret for, hvis de skal gøre det igen og igen i al uendelighed.. Hm.. Nå, men jeg lader det være for nu. I de ovenstående noter har jeg bare antaget, at alle Subjekter bare oplever deres Oplevelser, og at Subjekterne altså ikke selv går ind og påvirker de Hjerner, de har tilknyttet sig.. Hm, men det er da egentligt en hypotese-mængde, der er værd at have med også. Ja, ok. Så når jeg på et tidspunkt skriver dette som en artikel, så må jeg lige huske, at inkludere sådanne hypoteser også, og så må jeg også lige sørger for inden da at tænke lidt mere over, hvad man så kan sige om døden og om etik, hvis man antager sådanne hypoteser.

Men slut for nu.:) Det var rigtigt dejligt lige at få gennemgået det hele i en nogenlunde sammenhængende tekst, for jeg tror at alle mine tidligere noter omkring emnet alt i alt har været ret usammenhængede. Så rigtigt dejligt lige at få gået det hele (eller rettere det meste af det) igennem igen i store træk --- og dejligt også at få tænkt nogle nye tanker omkring emnet!:) Og jeg tror altså umiddelbart godt, jeg kunne skrive denne gannemgang her om, så det kunne blive en god lille (eller knap så lille, alt efter hvor kortfattet jeg kan gøre det..) artikel. Det vil jeg se frem til.:) (17:36, 06.01.23)

(07.01.23, 9:13) Okay, der er lige nogle få ting, jeg skal huske at nævne også, og så mangler jeg også at diskutere den mulighed, at vi har dødelige "sjæle"/Subjekter noget mere. Lad mig lige starte med at uddybe, at i de Oplevelse-orienterede hypoteser er hvert "univers" i multiverset ikke et fysisk univers, men et idealistisk univers, der indeholder én eller flere Subjekter, der udlever en eller flere Oplevelser. Jeg mener endda, at det mest oplagte for sådanne hypoteser bare er at have ét univers pr. Oplevelse. Men så skal det altså ikke forstås sådan, at vi er.. alene om at være bevidst i vores fysiske univers --- jo, det er vi på en måde, for det fysiske univers findes kun i kraft af vores egen Oplevelses eksistens i så fald, og ikke af personer okring os, men man skal så huske på (og dette er så selvfølgelig antaget, at "sjælen"/Subjektet ikke har indflydelse på Oplevelsen, og at hver Oplevelse der fastlagt ud fra nogle love, ligesom også jeg har antaget i resten af denne tekst), at alt hvad man gør i ens "eget" univers så bare bliver spejlet i et tilsvarende univers, hvor "sjælen"/Subjektet har tilknyttet sig en anden hjerne (med lille 'h,' fordi vi her snakker om vores egne "kød-hjerner"). Det var det første, jeg lige ville uddybe.

Det næste, jeg vil uddybe handler om hvordan Oplevelser mon kan defineres i hypotesen. Her kan vi starte med at se på en sjov lille idé om, at alle Oplevelser kunne være meget korte, altså i subjektiv tid, og at vores egen opfattelse af en lang, sammenhængende Oplevelse bare er.. ja.. er subjektivt skabt.. Men nej, vi kan faktisk forkaste sådanne hypoteser, eller i hvert fald givet den indledende antagelse om at "alt hvad kan eksistere, eksisterer," for så vil en vilkårlig sammenhængende Oplevelse jo ikke behøve at følge nogen lovsætninger rigtigt. For eksempel kunne vi have en sammenhængende oplevelse af, at en person træder ud af en dør og med det samme kommer ind ad en helt anden dør; oplevelser der isoleret set måske følger nogle lovmæssigheder, men ikke når man sætter dem sammen. Alt ville derfor blive totalt kaotisk (sandsynlighedsmæssigt), mere eller mindre, og derfor kan vi altså forkaste det. Ok. Så Oplevelser i sådanne Oplevelse-orienterede multivershypoteser skal altså være længerevarende. Men hvor lang tid skal de så vare? Jo, det ville jo være underligt, hvis de alle var en meget specifik længde; hvorfor skulle det store samlede multivers være så specifik? Så de tre eneste fornuftige muligheder er nok bare, at de enten er uendelige alle sammen, at de er endelige men med vilkårlige længder, og som den tredje mulighed at der både findes uendelige og endelige Oplevelser. Nu er det så oplagt at spørge: Jamen slutter en Oplevelse ikke bare ved døden, mere specifikt altså når den pågældende Hjerne ikke længere opfylder de krav der skal gælde for den (pr. de love som "universet" (som i dette specifikke tilfælde er defineret som led i definitionen af Oplevelsen) påskriver)? Tjo, det kunne de sikkert sagtens gøre i mange tilfælde. Men man kunne også sagtens forestille sig, at Oplevelsens/"universets" definition af den iboende Hjerne tillader, at Oplevelsen kan hoppe fra fysisk hjerne/Hjerne til en anden fysisk hjerne/Hjerne. Desuden kunne man også have Oplevelser, der bare simpelthen er sammensatte, i den forstand at de er defineret med en "lovtekst" noget i retning af: "Først skal du hoppe sådan og sådan fra fysisk hjerne/Hjerne i det her fysiske univers for så og så lang tid (eller indtil det og det sker), og efter det skal du så leve i det og det fysiske univers (med de og de fysiske love) og hoppe sådan og sådan fra Hjerne til Hjerne indtil sådan og sådan, og efter det..." På den måde kan man altså definere nogle oplevelser, der er virkeligt lange. Og både hvad angår endelige og uendelige Oplevelser kunne man endda have definitioner, der definerede hver del-Oplevelse i sekvensen ud fra en mere abtrakt formel (eller (Turing-)maskine-starttilstand, eller hvad man nu kan tænke sig), og så iterere over alle de individuelle udgaver som følger den formel (eller hvad man tænker sig). Og hermed kan man så nå nogle virkeligt lange Oplevelser. Og hvis vi videre tillader, at Oplevelsernes definitioner også kan sige noget så som: "Gentag disse iterationer et antal gange lig Grahams tal" (eller TREE(3) eller TREE(TREE(3)) og hvad vi ellers kan finde på), jamen så kan man have (bogstavelig talt) helt ufatteligt lange Oplevelser. Ok. Hertil skal det så pointeres, at hvis vi antager at, "alt hvad kan eksistere, eksisterer," og hvis vi kan danne vilkårligt lange af sådan nogle Oplevelses-beskrivelser, jamen hvis vi så prøver at spå om, hvor lang vores egen Oplevelse-beskrivelse må være, så vil vi jo ikke kunne sige andet end: Dens længde må i gennemsnit være uendeligt. Det virker vildt med det er faktisk konsekvensen.. Nå, og nu er det så her teorien om CUH kommer ind, for hvis man så overvejer hvilke nogle del-Oplevelser (altså dem med en nogenlunde konstant selvforståelse), der må være flest af i sådan en uendelig mængde af Oplevelser med vilkårligt lange beskrivelser, jamen så kommer man vist rimeligt nemt frem til (tror jeg/mener jeg), at de del-Oplevelser med tilsyneladende relativt lav information vil være mere frekvente.. Tja, eller det giver faktisk lidt sig selv: Der vil være en høj frekvens af del-Oplevelser, hvor det iagtagende univers tilsyneladende følger ret simple principper. Så det er altså sådan, at CUH kommer ind i billedet, når vi snakker Oplevelses-orienterede multiverser. *(Der er kan være lidt forskel på, hvad (prior-)sandsynlighedsfordelingen er i forskellige Oplevelse-orienterede hypoteser, eksempelvis afhængende af hvorvidt multiverset ordner Oplevelser og udlever dem ud fra, hvad der svarer til en "regnekraft" og sådan. Men det er nu ikke fordi, vi alligevel kan regne os frem til den faktiske sandsynlighedsfordeling overhovedet. Så for os behøver vi bare at vide, at vi nok får noget, der svarer til CUH, hvilket det vil gøre hvis det opfører sig pænt, og altså ikke giver os komplet kaos, som vi har set på.)

Nå, og nu kan jeg så slutteligt vende tilbage til den mulighed, at multiverset indeholder endelige ("dødelige") Oplevelser, og hvad det bør betyde for vores tilgang til døden. Og det korte af det lange her er så bare, at hvis der både er endelige og uendelige Oplevelser, jamen så vil der med al sandsynlighed, for dig der læser dette, allerede været gået en fantasiliard år (altså TREE(TREE(TREE(...))) år; find selv på hvor mange TREE der skal stå i rækken) og størstedelen af alle endelige Oplevelser vil allerede være døde, og du vil med al sandsynlighed være en af de uendelige. Og faktisk kan man næsten sige noget tilsavarende, når det kommer til de hypoteser, der kun indeholder endeligt varende Oplevelser. ..Tja, eller det afhænger godt nok af den specifikke hypotese, men under antagelse af at Oplevelsers levetider godt kan være defineret ud fra koncepter så som TREE(TREE(...)) osv., så vil de afsindigt lange Oplevelser også lynhurtigt.. tja, det var et forkert ord at bruge, men set i forhold til uendeligt, så jo, så vil det "lynhurtigt" blive kun dem, der er tilbage. Og ikke nok med det, de vil også veje utroligt meget mere end alle de knap så lange Oplevelser. Hvis man f.eks. ser specifikt på antallet af gange, hvor et Subjekt spørger sig selv (eller rettere har oplevelsen af at spørge sig selv): "Hvor lang mon min Subjektive levetid er endnu?" Hvis vi ser på statestikken omkring det antal for hver Oplevelse, så stort set alle.. nej, basalt set alle (som i: alle for all intents and purposes) forekomster af disse spørgsmål findes i Oplevelser, der er længere end T=TREE(TREE(...)) tid, efter vi den globale tid er lig TREE(TREE(...)). Altså når det globale ur slår TREE(TREE(...)), så vil forekomsten af alle sådanne spørgsmål til sig selv have fundet sted på ligeledes langvarige (eller længere) Oplevelser. Ok, dette er sikkert vrøvl for mange, og desuden så har jeg også her antaget, at der findes en global tid, eller i det mindste at der findes noget der svarer til en sådan. ..Okay, lad mig bare sige det sådan her: Hver gang et Subjekt spørger sig selv (eller rettere har oplevelsen af at spørge sig selv): "hvor lang mon min Subjektive tid har varet indtil nu?" så vil denne tid i gennemsnit være uendeligt (så i praksis, når vi spørger os selv, så vil svaret være: så ufattelig stor at du ikke vil kunne skelne det fra uendeligt). Og når man så spørger: "hvor lang mon min Subjektive levetid er endnu?" så vil svaret være det samme. Forklaringen på, at jeg mener at dette er tilfældet, den er lidt indviklet, og den kræver også lige nogle antagelser, må jeg indrømme, men under rimelige antagelser når man altså ret nemt hen til det samme svar: Hvis alle Oplevelser er dødelige, så vil alle korte og mellemlange Oplevelser "lynhurtigt" dø ud, når man opvejer dem mod uendeligt, og kun de ektremt lange (ja faktisk ufatteligt lange) Oplevelser vil være tilbage, hvilket vil sige at du selv, i det øjeblik du læser dette, i så fald så med al sandsynlighed vil tilhøre mængden af de ufatteligt lange Oplevelser. Okay, så er det vist godt med det for nu.xD^^ 

Det var, hvad jeg havde at tilføje om dette emne.:) (10:31, 07.01.23)







## Blockchain

(31.08.22, 10:25) Jeg havde egentligt lidt tænkt mig alligevel at udgive min angrebsvektor (med tilhørende forsvar) hurtigt på min GitHub, men jeg tænkte lidt over det i går, og det er lige før, at jeg faktisk ikke gør det alligvel; ikke i nogen stor fart.. Jeg kan lige tænke lidt mere over det, men jeg tror faktisk ikke helt, der er kød nok på.. tjo, tja, jeg ved det ikke; jeg skal nok lige tænke lidt mere over det. Men umiddelbart tænker jeg altså ikke at bekymre mig om at skynde mig at få det ud.. 
(02.09.22, 11:01) Nej, jeg tror simpelthen ikke der rigtigt er noget guf på denne idé. Så ja, alle mine blockchain-tanker er nu ret meget ude af vinduet.. Selvfølgelig vil jeg dele idéen om angrebet på et tidspunkt, men jeg tror altså ikke rigtigt, den kan få nogen til at spærre øjnene op (og være vildt interesseret, i.e.).. 


*((18.09.22, 11:51) Okay, glem stort set alt, hvad jeg har skrevet her i går:)
(17.09.22, 12:27) Jeg havde en ret vild (som i 'ude af normen'; ikke vild som i ' fest-vild') nat, hvor jeg gik i seng lidt efter tolv og så var vågen helt til omkring seks. I lang tid kunne jeg bare ikke sove (selvom jeg ikke synes, jeg gjorde noget som helt galt, eller havde det for varmt/koldt; det eneste jeg kan tænke på var, at jeg måske var en lille smule sulten, men kun en meget lille smule og ikke noget, jeg synes forstyrrede mig!..), men på et tidspunkt begyndte jeg også at tænke lidt over fysik, så lidt om mine "planer" (jævnfør sektionen nedenfor), og så fik jeg så også tænkt på blockchain, hvor jeg mellem fem og seks synes jeg fik et hel væld af gode idéer.. Så ja, dem vel jeg så skrive om her (og genoverveje dem)..:).. 
..Hm, lad mig bare prøve at forklare, og så kan jeg overveje imens:
..Hm, eller lad mig egentligt lige starte med at skrive om, at jeg i første omgang fik nogle tanker, hvor jeg bare tænkte: ah, måske kan jeg alligevel godt skrive om blockchain (og mit angreb) i min GitHub-mappe, hvis jeg bare indleder den grundlæggende del af det/hovedparten med at sige, at det altså bare er et argument, hvorfor en kryptovaluta (KV) ikke kan overtage og blive en meget almindelig valuta på lige fod med normale penge. Så fik jeg tænkt på, at pointen i sig selv om, at et angreb kan være mere attraktivt at udføre, fordi man nemt kan ende med ikke at skulle betale alle de mønter, man satte på højkant i replay-angrebet, jo er en vigtig pointe i sig selv, som er værd at dele.. Nå, og der ved femtiden (måske lidt efter; måske tyve over) kom jeg så til at tænke på, at man jo faktisk ingen gang behøver at sætte penge på højkant, i bund og grund, fordi man bare kan forke og lave en gren, hvor man har (brugte!) penge, og så sætte dem på "højkant." I løbet af den næste halve time derfra (omkring tyve over fem altså) fik jeg så en masse gode supplerende idéer (nogen af dem, som jeg har fået før i andre sammenhænge), nemlig om at man jo kan snyde med tiden og dermed sørge for at booste mining-farten en smule til at starte med (ved at skrue tiden frem en smule, nemlig til.. tja, eller også kunne man egentligt bare starte fra nutidspunktet, det ville måske være det nemmeste..), hvilket både gør at man hurtigere kan danne blokke trods stærkt formindsket minin-kapacitet (og man bør i øvrigt helst forke lige inden den blok, hvor "målet" (the target) bliver sat ned) og altså hrutigere kan få sine malicious kontrakter ud, og også gør at det bliver noget nemmere at tiltrække minere (for lønnen er jo den samme pr. blok). Derfra tænkte jeg så også, at man bare kon offentliggøre kontrakterne forud for at blokkende bliver minet. Og så tænkte jeg på, at man jo så også kan rekrutere alle, der har solgt KV siden fork-punktet, for man kan jo gentage alle de kontrakter, medmindre de har betingelser, der nævner tidligere blokke i kæden (som jo nu vil blive erstattet). At sælge sine mønter med smartkontrakter, der nævner, hvad tidligere blokke er, kan så i øvrigt være en måde at sikre sig mod sådanne angreb *(nå nej, man sikrer sig jo ikke herved, så never mind den del (her til venstre for denne kommentar)), hvilket på en måde faktisk bare er godt for angrebet, for det gør det jo bare nemmere at rekrutere folk, nemlig som har sikret sig, at de ikke selv kan udnyttes. Nå, men man kan også gøre endnu mere, tænkte jeg på: man kan også starte med at gøre så den ekstra miningløn, som angreberne sætter minerne i vente på angrebsforken, ligesom bliver administreret på en sidechain af angrebsforken. Her skal folk så kunne melde sig til rimeligt frit, og man kan så implementere, at man kan udlove dusører på betingelse af, at den endelige blok overholder nogle ting, hvorved man altså kan gøre så dusører bliver betinget af, hvilke kontrakter der kommer med, og hvilke ikke kommer med (når man arbejder sig hen imod nutidsdatoen). I øvrigt (tænkte jeg på lige nu her) kan man jo potentielt set, hvis man skal gøre det virkeligt sofistikeret, sørge for at angriberne har en vis frihed til at lave bestemmelser over de dusører, de allerede har udlovet, hvor de altså så får en vis frihed til at vælge, hvilke eksisterende kontrakter skal med og ikke med, men hvor det så skal sikres, at de ikke har mulighed for at skabe nogen modstride og gøre deres dusør-kontrakter ugyldige herved. Og ja, ellers kan man jo også bare have det sådan, at de betingede dusører udstedes, når der er behov for dem, for angriberne kan jo sagtens give flere og flere dusører løbende (og hvor "angriberne" altså også med tiden kan indbefatte flere og flere).. (13:04) ..Nå ja, og prikken over i'et, som jeg kom til at tænke på omkring ti i seks (ja, det gik ret hurtigt, kan man sige ..tja, men mange af "idéerne" minder jo trodsalt rigtig meget om tidligere idéer..), er jo så denne pointe, nemlig at angrebskæden faktisk har en fordel over for den "uskyldige kæde"/den originale ("rigtige") kæde, fordi angriberne jo, i modsætning til folk på den "rigtige" kæde, der skal prøve at gå til modangreb, ikke "mister" noget som sådan, når de udlover dusører, for de penge har de jo allerede brugt!. (13:08)
..(13:09) Okay, nu ser min angrebsidé jo faktisk ud til virkeligt at virke, hvilket ville være kæmpe stort, for det kommer bare til at booste interessen for mit andet arbejde \emph{så} meget mere, kan jeg forestille mig. Så hvis jeg ikke tager fejl i det her, så har jeg altså nu potentialet til, med en rimelig kortfattet tekst (jeg vil bare forklare det simpelt; i kortfattede, måske to-tre-sætnings-lange, paragrafer), at opnå, hvad der ift. mit fysik projekt vil svare til vildt meget arbejde (hvis vi tænker på sådan noget som at arbejde på "future work"-emnerne, og også bare sådan noget som at rette min artikel godt igennem, så den kommer til at fremstå skarp --- det bliver pludesligt ikke nær så vigtigt, hvis jeg også har denne kæmpe nyhed om blockchain (der endeligt kan bane vejen for mere "grønne" blockchains(/soft forks)))..! :D:) (13:15) ..Hm, det er i øvrigt sikkert også det \emph{helt} rigtige tidspunkt at komme med sådan en idé som denne..!.. 
...(13:33) Hm, og det er jo klart, at dette stadig mere er et argument for, hvorfor nuværende KV'er ikke kan fungere som konventionelle betalingsmilder, for det vil aldrig være attraktivt at lave et konspirationelt replay-angreb på en kæde, der så bare mister al dens værdi (hvilket jo er rigtig godt; det gør jeg jeg ikke behøver at have skrubler over at udgive det). Men en etableret kæde, hvor store dele af samfundet pludselig har stake i kæden, det er en anden snak, for så kan værdien nemlig holdes oppe af denne stake. 

(18.09.22, 11:53) Glem alt hvad jeg har skrevet her ovenfor. Nu hvor jeg har fået mere hjerne igen (har sovet godt i nat, men kom allerede frem til i går aftes, at mine tanker her ovenfor ikke kunne bruges), kan jeg se, at det ikke holder. Især ikke den del med, at angriberne "har en fordel".. ..Og de andre ting holder bare heller ikke rigtig; gider næsten ingen gang forklare hvorfor.. Nej.. ..Nej, lad mig bare strege hele den.. hvad skal vi kalde det? undersektion?.. fra i går.. Ok. 




## Planer

(02.09.22) Nu hvor blockchain-idéen lidt er ude af vinduet, så tror jeg let det kan blive en langtrukken proces om at slå igennem. Jeg er så ved lige at planlægge, hvad jeg skal gøre efter min udgivelse. Jeg er lidt kommet frem til, at jeg skal starte med at give nogle korte ("less is more"-agtige) udgivelser (i min GitHub-mappe) omkring mine web 3.0-idéer --- hvor jeg i øvrigt måske kan fokusere lidt på "ratings" i det semantiske web, som en af de gannemgående idéer, men så ellers også bare det at starte det ud fra web 2.0-sider og wiki-sider; ting som virker allerede, men som kan forbedret af brugerdrevet semantik.. ..Nå, men jeg har lige nogle få ting, jeg lige skal planlægge færdigt omkring det.. (11:10) ...Ah, jeg tror måske, jeg ved hvad jeg gør.. Måske lader jeg bare være med sige, at der er mere ved idéerne end hvad jeg skriver --- ah, og måske også skriver om dem som om de er meget nye idéer (hvad de på én led også er, kan man sige, selvom jeg dog har gennemarbejdet dem lidt (men ja, jeg kan så lidt lægge skjul på, at jeg har gennemarbejdet dem overhovedet, og præsentere dem bare som nogle idéer, jeg gerne vil arbejde videre på.:))) --- og så tænker jeg nemlig særligt også at præsentere min forretningsidé som en helt ny idé, jeg godt kunne tænke mig at arbejde videre på.. ..Hm, men jeg tænker nu dog lige at vente en ekstra omgang stadigvæk med at udgive denne idé-skitse af min forretningsidé. ..Ah, eller endnu bedre ift. at kalde dem nye idéer: Jeg kan bare sige, at de alle er ret nye; at jeg har haft lidt tid til at overvejet dem hver især, men at de dog stadigvæk er på design-stadiet (og på et stadie, hvor de bør overvejes endnu mere for at finde fejl og mangler i dem)..:) Nice.!.. For så har jeg dækket ryggen på en fin måde, og så kan jeg bare lade pitchene tale for sig selv. ..Nå ja, og jeg skal så heller ikke reklamere med, at jeg har en forretningsidé til at starte med: Pointen er lidt \emph{ikke} at give folk opfattelsen af, at jeg muligvis sidder og gemmer på en guldgrube --- ikke før det kan betale sig. (11:42) (For på den måde tror jeg, jeg vil få meget mere positiv energi, og nemlig forhindret en masse potentiel negativ energi (bl.a. fordi folk kan få en negativ reaktion til en, de mener, har munden for fuld).) (11:43)
(03.09.22, 11:09) Tror faktisk endda lige jeg venter en omgang med at nævne min wiki-idé og dabatside-idé også. Til gengæld kan jeg måske nævne ret hurtigt, at der er en/nogle ekistensteori(er) på vej også. Men det kan jeg lige se på. Det er forresten lige før, at jeg vil arbejde på at skrive en uddybende tekst (på engelsk) omkring min forretningsidé først, før jeg begynder at skrive om mit selvadjungeretheds-bevis.. 

(04.09.22, 10:11) Jeg skal faktisk lige overveje noget mere, om det nu også er klogt at offentliggøre sine idéer, så de gængse sider faktisk nærmest får et forspring. Jeg tror muligvis, det går, men det må jeg altså lige tænke mere over i mine pauser.
(14:58) Ja, det går; jeg skal helt sikkert bare offentliggøre de idéer med det samme. Jeg skal i øvrigt også huske at snakke om brugergrupper og anonymitet i disse, men ja, jeg vil jo bare gå igennem alle mine punkter, npr jeg når dertil, og så udgive om alle dem, der kan forklares rimeligt kortfattet, nok på nær dem der har med vidensdeling og debat lige i første omgang, dog. (15:00)


(13:17, 17.09.22) Jeg tror muligvis jeg snart vil begynde at oplaude ting til min note-mappe, og måske vil jeg faktisk også allerde begynde at lave nogle små tidslåse (bare over et par måneder), f.eks. til min blockchain-angrebs-idé (hvor jeg så på den anden side vil udgive løsningsforslaget med det samme). 

(11:58, 18.09.22) Hm, nu overvejer jeg faktisk bare at give hints til koden, og så måske lave en kode (også) bare af nogle danske ord sat sammen (og måske gentaget tre gange, hvis det ellers er kort). ..Det virker på en måde sjovere, end det andet, og jeg tænker alligevel faktisk at åbne det hele før snarere end senere.. ..Og jeg overvejer også faktisk lidt at pitche min forretningsidé i samme omgang, som jeg pitcher mine idéer til en ny web 3.0-bølge.. ..Men det må jeg jo lige tænke nærmere over i de kommende dage (i min "fritid").. (12:03)

(19.09.22, 16:11) Okay, nu har jeg nogle meget bedre planer..! Jeg er kommet frem til her i dag (i en pause fra at tænke på fysik her på min seneste gåtur). Det korte af det lange er, at jeg bare skal forklare om min forrestningsidé med det samme, og så ikke være bleg for at sige: jamen, jeg er sikker på, at dette bliver den næste vildt store ting; langt langt større end BitCoin osv. endda (hvad jeg jo helt klart også tror på selv). ..Jeg tror lige, jeg vil vende tilb.. nå nej, det hører jo alligevel til ovenfor. Ok, så jeg har også nogle idéer til, hvordan man kunne opfordre til at starte det meget simpelt, nemlig med en kickstarter og med nogle løfter (som man så hurtigst muligt skal underbygge med kontrakter). Og dette vil jeg så altså også bare lægge op til, når jeg (meget hurtigt efter min fysik-udgivelse) skriver om idéen i min GitHub-mappe.. ..Ok, lad mig vende tilbage hertil, og uddybe lidt mere, hvis jeg synes, jeg bør sige noget mere...
%..(16:21) Nå jo, lad mig også lige nævne, at jeg nu også har tænkt mig at give en kort note om, hvordan der ikke skal "særligt meget uendelighed til" i multiverset for at man når frem til, at vi basalt set genfødes som alle levende væsner igen og igen --- og at man derfor kan tage "what comes around goes around" fuldstændig bogstaveligt, for vi skal på den måde alle opleve de samme glæder og smerter. 
%(17:38) Jeg tænker så også bare at lægge alle mine noter ud med det samme, bare til \emph{hvis nu}, nogen skulle være interesseret i at sneak-peak'e (og ofre noget tid), inden jeg får skrevet mere sammenhængende noter over emnerne (de vigtige af dem i hvert fald). Og så tænker jeg i øvrigt også bare ikke at lægge skjul på, at jeg tror min forretningsidé, hvis den udbreder sig til andre områder, kan blive lidt en kur til de negative sider af kapitalismen (men endda uden at bryde med kapitalismen; forretnings baserer sig helt på et frit markede osv.!). (17:41) 


(18:07, 27.09.22) Jeg tænker nu lidt faktisk at starte med at skrive om min forretningsidé i sig selv (og som jeg lige har skrevet ovenfor, bliver dette altså uden at hype den som den næste store investeringsdrøm, for det er jeg kommet i tanke om, at den jo nok ikke vil være som sådan..). Jeg vil så forklare om, hvordan den fungerer, hvad den lover for fremtiden, hvad hver part vil få ud af den, hvorfor forbrugere i det hele taget burde have magten i et kapitalistisk ("forbruger-")samfund, hvem det kan komme til at gå ud over, og, også rimeligt vigtigt, hvorfor det egentligt ikke behøver at gå ud over de rige som sådan (og hertil hører også en lille idé om, at man kan slå to fluer med ét smæk og gøre idéen mere attraktiv for de rige, og forhindre, at boligmarkedet eksploderer i fremtiden, når bevægelsen har slået rod og de velhavende naturligvis vil begynde at lede efter steder, hvor de alligevel kan få deres penge til at yngle..) (For hele pointen med idéen er jo netop, at man når til et ret lige samfund, hvor der stadig er masser af plads til rigdom, men hvor rigdom i sig selv ikke bare kan yngle; hvis rige mennesker skal gøre sig selv rigere, skal de gøre dette ved at bruge deres (eventuelle) talenter til at forbedre samfundet, i.e. de skal gøre et stykke arbejde (medmindre selvfølgelig, de har haft held med at lave og opretholde en privat virksomhed, der ikke følger princippet i min forretningsidé, for sådanne skal bestemt ikke forbydes --- der skal jo ikke laves nogen som helst nye regler/love i samfundet i princippet i forbindelse med min forretningsidé).) (18:18) ..Og så vil jeg så forklare, hvad jeg lige har nævnt ovenfor under den relevante sektion, nemlig at man kun i visse tilfælde kan forvente, at der kan blive en stor investeringsdrøm i det, og at én (vigtig) mulighed her lige præcis er (nogen af) mine web-idéer, fordi sådanne virkeligt kan have gavn af, at brugerne har magten (og ved at brugerne af systemerne altid vil have dette). Og det vil jeg altså så nok slutte af med at referere til, og så må jeg jo se, hvornår jeg så får skrevet om de emner efterfølgende.. (18:26) 


(11:59, 19.10.22) Okay, der er ændringer i mine planer. Min fysikartikel ser faktisk næsten ud til at være lidt et flop muligvis.. Jeg skal lige have overvejet alle tingene og læst godt op på literaturen, men ja, der er altså muligvis ikke så meget nyt i den. Mine nye planer er så, at jeg skriver et samlet dokument om lidt af hvert: min forretningsidé, e-demokrati, og nogle af mine web-idéer/forudsigelser, samt også om eksistensteori, og så regner jeg med at gøre den samlede overskrift til noget a la "idéer og forudsigelser..".. Hm, eller jeg har også tænkt bare "En lys fremtid".. Men ja, pointen er: Jeg har nogle forudsigelser om den nære og fjerne fremtid, som fortæller, at denne er rigtig lys (og så har jeg også lige nogle eksistensbetragtninger, som gør det hele endnu mere lyst at tænke på). Så altså en meget positiv tekst om den nære fremtid (og idéer til at gøre denne bedre), om den fjerne fremtid også (og det vil jeg i øvrigt også skrive lidt om (at vi nok får et godt post-scarcity-samfund på et tidspunkt)), samt også hvorfor vi sagtens kan gå at glæde os over en lys fjern fremtid, også selvom vi ikke selv kommer til at leve iden (hvor vi altså snakker mine eksistensteori-betragtninger). Det vil jeg altså gå i gang med nu, og så vil jeg prøve også at arbejde lidt på at reparere min fysik-artikel om aftenerne. (12:07)   




## Diverse tilføjelser her i starten af 2023 (muligvis afrundig på dokument)

(04.01.23, 16:36) Okay, jeg har lige fået ig en ny computer, efter at min forhenværende gik i stykker i julen, og har lige fået installeret Linux. Her under nytårsræset (nærmere bestemt natten til d. 31.) fik jeg tænkt lidt over Eksistens igen, og det har jeg også brugt de følgende dage på indtil nu, imens jeg ikke havde en funktionel computer. Jeg har et par nye tanker om emnet, og har også en nogenlunde idé til, hvordan jeg ville strukturere en lille artikel om det. Det vil jeg derfor lige skrive om ovenfor i 'Eksistens'-sektionen, efter (..uh ha, det er et lidt drilsk tastatur, fordi man godt kan trykke ned på siden af en tast uden at den registrerer et ryk.. he, meget passende: jeg skulle have skrevet 'tryk'..) at jeg har skrevet denne sektion. 

Jeg kan starte med at skrive, at selvom jeg tror på at mine idéer omkring tag-ratings, kommentarkategorier, fanetræer og alt det omkring en web 2.0--3.0-side virkeligt indeholder et kæmpe potentiale (også rent forretningsmæssigt), så er det altså ikke sikkert, at de idéer bare sådan lige vil føre et semantisk web med sig med det samme.. Måske vil det hjælpe, at folk bliver vent til at diskutere i træer/diskussionsgrafer, når det kommer til kommentarer, men det kommer nok ikke til at accelerere overgangen til en videns-/diskussionsgraf (med sandsynligheder på, altså en "p-ontologi," som jeg før har kaldt det) super meget. Det tror jeg til gengæld nu mere mine debatside-idéer vil. Og også mine e-demokrati-applikations-idéer. For e-demokrati-idéerne kommer jo måske nok til at give incitamenter til folk om at deltage i (graf-)diskussionerne, og mine debatside idéer vil i øvrigt også åbne op for, at folk vil være meget mere villige til at lytte til hinanden på tværs af grupper, nemlig fordi man gør det til en udfordring: "Hvis du tror denne anden gruppe er helt henne i vejret, hvorfor så ikke bevise det via en grundig diskussion med en saglig og upartisk 'dommer' (eller flere) for diskussionerne?" Så det tror jeg altså mere kunne være dét, der kan sætte skub i den udvikling. 

Nå ja, og så kan jeg også sige, at jeg i går eller i forgårs kom til at tænke på, at (semantisk struktureret) NLP måske kan gå hen at blive en virkelig vigtig kilde til, at vi en dag får videnskabet semantisk struktureret (hvad ville være SÅ (SSSÅÅ!!) godt), hvis nu ikke semantic web-løsninger når at komme inden om først og tilveje bringe den realitet på en mere direkte måde (altså mere via menneskeligt arbejde frem for hjælp fra en A.I.). For når først den tekologi bliver udviklet tilstrækkeligt, hvad den sikkert gør i en ikke alt for fjern fremtid, så vil man ret nemt kunne opnå et kæmpe spring fremad simpelthen ved at få en A.I. til at lave et stort forarbejde med at analysere al eksisterende videnskabeligt arbejde og fylde det ind i en stor semantisk struktureret graf. Og når først man så har sådan en graf, så kan man derfra ved hjælp af menneskearbejde og v.h.a. yderligere forbedriger af A.I.'en (og altså ikke mindst ved en synergisk sammenblanding af disse) relativt nemt få udbygget denne store graf, så man får struktureret al viden semantisk. Så NLP-teknologien kan altså sagtens komme til, hvis ikke sem-web-løsninger kommer indenom først, at blive utroligt vigtige for den samlede videnskab (og dermed hele vores teknologiske udvikling). Og det virker som om, at den allerede er godt på rette spor, for der tænkes vist allerede godt i, hvordan man kan gøre NLP-algoritmer mere semantisk funderet (hvilket jo er klart, for så må man jo i sidste ende ende ud med et mere "intelligent" produkt). Så lad mig da endeligt huske mig selv på at følge noget mere med i den (spændende) udvikling. 

Udover min e-demokrati- og debatside-idéer, som jeg på et tidspunkt vil prøve at iværksætte (er planen), så har jeg altså nogle forskellige politiske og økonomiske idéer. Nå ja, og én af de politiske idéer handler jo så om at bruge en e-demokrati-app, nemlig til at lave et e-demokrati-parti.. ..Hm, jeg skal forresten nævne, at man også kunne forestille sig e-demokrati-appen mere som et socialt medie, hvorved altså alle personer og grupper altså enten direkte eller inddirekte har adgang til hinanden, og mere eller mindre deltager i det samme "digitale rum," kan man kalde det, når de bruger appen. Og så kan alle mulige befolkningsgrupper, og alle andre typer grupper (hvis ikke befolkningsgrupper er et altomfattende udtryk), altså finde sammen på appen og diskutere og forhadle med alle andre grupper på mediet. ..Og ja, så tænker jeg altså også, at man kunne blande appen sammen med debat-appen, således at e-demokrati-grupperne også kan oprette diskussioner/debatter med hindanden ovre i det andet ben af appen, der så altså er "debatside"-delen. (17:18)

Men ja, og udover disse ting, så har jeg altså mine idéer til nye typer af forbrugerforeninger (som forklaret i mit nye dokument ('consumer unions.pdf')) og mine idéer omkring "share-redistributing companies," hvilke jeg begge tror kan blive virkeligt store, ja, og sådan set ligefrem neutralisere alt (ikke ALT, men ~alt) hvad der er skidt i den nuværende udgave af kapitalismen, således at vi kan få en meget bedre form for det. Så planen er altså, at jeg vil prøve at diskutere SRC, CU, og.. EDP (e-demokrati-parti) med andre politik-interesserede, og så prøve at fokusere mit ejet faglige arbejde i retning mod at kunne iværksætte ED- og debatside-app --- bl.a. faktisk også ved simpelthen at læse open source kodebaser i min fritid (når jeg får sådan en igen). (17:25)

Hm, var der mere jeg skulle sige her (som ikke omhandler Eksistens)..? Der ver lige en lille ting, og det er bare, at jeg lige tænkte på igen, at hvis man som befolkning skulle åbne mere op for overvågning, så skulle det være via personlige kamerare, som alle folk helt selv er frie (ikke bare lovligt men også fysisk) til at slå fra og til, hvornår det skal være, og hvor det alt sammen sendes til datacentre, som er offentligt overvåget via kamerarer og lydoptagelser --- og digitale diagnosticeringer --- hvor man så bare altså har alt indholdet krypteret hele vejen, også hvis nu det skal bruges af ejeren og altså dermed sendes tilbage igen.. Men ja, jeg har sikkert sagt (skrevet) noget tilsvarende en gang.. ..Men jeg kom fra: var der andet, jeg skulle skrive her..? 

..Det tror jeg ikke, og ellers vil jeg bare lige vende tilbage. Så nu vil jeg altså skrive nogle tilføjelser til Eksistens ovenfor. Derefter vil jeg gå i gang med at se på at skrive (/ om jeg skal skrive) min SRC-artikel om og/eller skrive en ny version. Samtidigt med dette vil jeg så også tage kontakt til flere omkring det, og her tænker jeg så særligt på en vis tænketank, jeg er blevet fortalt om: Demokratisk Erhverv. Og når jeg så endelig når til et holdt/en pause i det, så vil jeg faktisk se på at få skrevet en Eksistens-artikel. Og ja, derefter snakker vi så at arbejde mig frem mod at kunne realisere mine e-demokrati- og debatside-idéer. Nå jo, jeg skulle forresten også nævne, at min wiki-idé nok heller ikke i sig selv bliver sindsygt accelererende for sem-web-udviklingen, mener jeg nu, selvom den nok dog vil kunne blive en rigtig god ting med tiden. Og lad mig så lige påpege, hvis det giver mening, at det nok er for meget at håbe på, at en videnskabelig semantisk ontologi vil kunne indeholde de helt grundlæggende argumenter til at starte med. I stedet bør man nok huske kun at sigte efter i starten, at den bare kan indeholde samlede videnskabelige værker som dens grundsten, hvilke så godt nok dog kan få noget semantisk data omkring sig om, hvad de siger, men som man dog altså ikke kan forvente skal splittes ad til atomer i den semantiske struktur. Jeg håber, det giver mening.. ..Man må altså forvente, at videnskabelige værker, der belyser et emne, må indgå so mere eller mindre atomare grundsten i den videnskabelige ontologi/semantiske graf til at starte med, og så er det nok først på meget længere sigt --- medmindre NLP-/AI-teknologien virkeligt kommer til at sparke røv på det punkt ---  at man kan forvente at disse værker i sig selv bliver splittet ad til semantiske atomer. ..Og når vi f.eks. særligt snakker e-demokrati-og-debatside-diskussioner, så må man også i høj grad forvente, at folk skal gøre brug af tredjepartsinstanser (altså forskere/forskergrupper, tænketanker, fagfolk, eksperter, ordfører, osv.) og hvad de siger/skriver om en ting, mere end at man skal forvente at hver del af alle disse analyser også bliver uploadet til mediet og behandlet online af brugerne (og analyseret semantisk). Selvfølgelig er det sundt jo mere der kommer online og bliver behandlet der, nemlig således at alle kan få indblik i detaljerne, og så der altså er høj gennemsigtighed, og så det er nemt at finde og udpege fejlslutninger m.m., men i sidste ende må man altså nok regne med, at dette ikke er muligt, og at man er nødt til bare at inkludere værker og udsagn fra andre som grundlæggende, atomart materiale for diskussionerne, uden at dette materiale bliver taget ind og splittet ad (og analyseret) på selve siden/i selve appen. (17:52)

..Men ja, nu fik jeg vist skrevet alt, hvad jeg ville sige, lige inden jeg kom med den sidste tangent her, så jeg kan vist bare afslutte her.:) Det er altså planerne, som de er nu, og for det første vil jeg altså så lige prøve at tilføje nogle ting om Eksistens ovenfor. (17:54, 04.01.23)


\end{comment}




































































%
%
%
%\chapter{E-democracies} \label{E_democracies}
%
%The concept of a so-called `e-democracy' is not a new one. Wikipedia thus has (in the moment of writing) a whole article about the overall concept that one can read. (That article, in its current form, defines the concept perhaps a bit more abstractly than what we need for our purposes here, but it might still be helpful to glance at.) In this section, I will therefore not introduce the overall concept, but simply give some short notes on how one might implement such an e-democracy, which can for instance be used to govern a company like the ones described in the previous chapter (as its shareholders), or a political party, etc. 
%
%
%\section{A basic digital application where voters can build proposition graphs}
%
%Imagine a digital application where all voters in a given democracy (concerning e.g.\ a company, a union, a political party, etc.) can log on and build a proposition graph together, which can then define the policies of the body governed by the given democracy. We are here talking about the `graphs' of mathematical graph theory. (One can make a brief search the internet for `graph theory' to see what this is about, and one might then also want to search for `directed graphs' and `connected graphs' at the same time.) 
%
%Each node of the graph holds a proposition, which is simply expressed in plain text of whatever natural language (such as English) is appropriate for the case. 
%
%When adding a new node to the graph, one can add it by itself (i.e.\ not connected to any other nodes) or add it with at least one of two kinds of (directed) edges to an existing node. The two types of edges then represents whether the node's proposition is an elaboration on the parent node, or if it is a self-contained proposition that should, however, only apply conditioned on the parent node being active.
%
%A node becomes active if it has enough votes and if a majority of those votes are positive rather than negative. Whether `votes' are counted simply by number (such that all voters have equal power) or if the votes are weighted (meaning that some voters have more power than others) of course depends on the case. 
%
%The point of being able to `elaborate' the proposition nodes rather than having to replace it with a more detailed note instead is simply to make the work easier for everyone, and also to make the graphs easier to read. It means that the policies can be defined somewhat loosely at first (and therefore much more quickly and easily), and whenever some vagueness of the propositions is discovered subsequently, either by people studying them or because of a relevant case that reveals it, the voters can then work to specify and mend the propositions. 
%
%The point of being able to add proposition nodes that are conditional on their parent nodes being active is of course some propositions might only be beneficial to implement given that certain other ones are already in place. If a somewhat fundamental proposition node is voted inactive again, it is thus convenient that such `conditional child nodes' follow along. If the parent node is then voted active once again (or perhaps for the first time) at a later time, all the child notes that has retained a positive voting score in the meantime will then become active one again, as well as any child node whose score has become positive in that time. 
%
%The application might also allow these `conditional child notes' to have several parent nodes for convenience. 
%And the same could also apply for the `elaboration child nodes' since there might be case where it could be beneficial to be able to elaborate the interpretation when two propositions nodes are active at the same time, for instance if these two proposition have a slight conflict with each other, or if the create some other issue that needs to be handled when they are both active together.
%
%`Elaboration child nodes' should of course also depend on their parents being active. The difference between a `conditional child node' and an `elaboration child node' is therefore essentially only in the interpretation: The propositions of `conditional child notes' and those of their parents are meant to be independent of each other as statements, whereas `elaboration child nodes' are free to correct and override parts of the statements contained in their parent nodes, thus allowing these to not necessarily be absolutely precise and self-contained. 
%
%Every user should be able to add new proposition nodes and each node should also have a separate `interest score' that users can rate (with the same weights on the votes as for the first score in the case where these are weighted). A proposition node whose `interest score' exceeds a certain threshold becomes visible to all users in the main graph, and people will then have to give their votes to it, if they want to influence whether it is applied or not. 
%
%Users should thus also be able to view nodes in the graph that has not yet exceeded said threshold, perhaps by being able to select various ranges of interest scores to look at. It might also be beneficial to let such nodes expire after a certain time if their `interest score' has been low enough for too long. 
%
%Users should also preferably have their own `workbench' with enough storage capacity to hold a number of propositions nodes. If a proposition node expires, they can thus make sure that the work is not lost as long as they keep said proposition on their own `workbench.' It would probably be beneficial also if users could then have shared `workbenches' as well, where they can collaborate on making new proposition nodes. 
%
%Anonymity is of course generally very important for democracies. So it is naturally very important that no one can see which user has added what nodes, unless of course they have collaborated on it from the same `workbench.' Users should also not (for most cases of democracies) be able to see which users has voted for what. 
%%
%For cases with weighted voting, either with very few voters or with very precise weights, this might be helped further by making sure that the exact voting scores are not visible to the users, and that the can thus only see a number that is rounded to a less precise floating point number. One might also implement intervals such that new votes are always declared together in groups, some time after they have been cast individually. 
%
%
%\section{How the proposition graphs are used to govern a body}
%
%The point of building these proposition graphs is then that the leadership of the body you are governing should to some extent be required to follow the active propositions, at least within some basic limitations of they can be required to do. 
%
%When the proposition graph changes, they leadership should be required to implement these, but here it might of course be a good idea to implement some delays on when new changes are supposed to be carried out. One might thus rule that a change should only be implemented after a certain period from when happened, and only if that change has remained active in the proposition graph during that period. 
%
%If the proposition graph gets some contradictions and/or ambiguities that makes it hard for the leadership to know what to do, they should also be allowed to postpone implementing the relevant changes until the voters sorts out the issues (by which they make some new changes which restarts the acceptance process). 
%
%How to make sure that the leadership follows the democratic decisions of the proposition graph? Well, by making sure that the voters also have enough direct power over the governed body to enforce their will. This might typically be ensured by the group of voters having the power to fire leaderships and/or decrease or increase salaries, thus giving this group ``sticks'' and potentially ``carrots'' that they can use to make sure that the hired leadership does what it is supposed to.
%
%
%It has to be mentioned that a high level of transparency is an all-important part of an effective e-democracy when it comes to the body that is being governed. Luckily, one can say that as long as the voters have the aforementioned ``carrots'' and ``sticks'' at their disposal, they should at least be able to make the body more and more transparent, even if it not very much so from the beginning.
%
%
%
%%Husk:
%	% Fortolkningspolitikker (inkl. hvad man gør, hvis der er modstride) og delays. (tjek)
%	% The point with having conditional propositions.. (tjek)
%
%
%\section{A more advanced application}
%
%A basic system like the one described above is good enough for very simple cases where it is okay to just have a majority rule. But for more complex cases where there are a lot of groupings of voters with different interests when it comes to various topics, to have such a majority rule is not really sufficient. If we for instance think of the policies of a whole country, this is a good example of such a complex case, where most people probably have \emph{some} special interests that are only shared with a fraction of the society. In such a system, it is important for people to be able to \emph{negotiate} with their voting power in order to get what they want, not just to always vote for exactly what they want as individuals. One group might thus want to meet with another to make a deal where they say: ``If you vote for this and this, even though you might not be particularly interested in that, we will vote for this and this which you do have a particular interest in (even though we might not).'' 
%
%In order to accommodate these realistic needs of its users, the digital application in question should therefore also first of all make it possible for users to form groups in the system. On a technical note, having such groups can of course be implemented in a lot of ways, but I can suggest an implementation where the creators of a group start out with some divisible `moderation tokens' that give them power to decide who can join the group and who gets kicked out, and where they are then free to transfer parts of (or all of) these tokens to other users within the group at any time. This moderation system is open enough that the users can implement any other moderation system on top of this, if only they trust some central party (which can then control a user profile in the system) to enforce the results of this external moderation system.
%
%And in order for such groups to be able to start negotiating with their voting power, it is then first of all important that some overall statistics (perhaps where numbers are rounded to ones with less precision for the sake of anonymity) of how a group votes on average is made public at all times. Otherwise a group who has made a deal with another group would not be able to check that this other group holds its promises. 
%
%With this addition to the application, groups can now in principle make all the deals in private that they want to. But of course, a good implementation of an `e-democracy application' would also afford its users with ways to make these deals within the digital application, online. 
%A way to do this might be to add what we could call `conditional votes' to the system. A `conditional vote' is then a vote on a proposition, whose sign can depend on other factors. In particular, a conditional vote should be able to depend on parameters regarding the voting statistics of groups. A group that want to make a deal with another one can then decide to make a conditional vote for something the other group wants and then make the condition such that this latter group has to vote for something the former group wants to unlock the conditional votes. 
%
%On another technical note: Depending on how the system is implemented, such conditional votes might be able to cause deadlocks, where two or more conditional votes all wait for the other to turn the other way in order to turn themselves. But a way to mitigate this is to give a direction to all conditional votes which denotes the sign expected from a successful deal. The system can then continuously refresh the proposition graph by turning all conditional votes in their positive direction and then see if they fall back to the same state or if they settle on a new state, which will mean that some deadlock has been conquered. 
%
%And on a design-related note: The conditional votes can be visualized/rendered as leaves in the graphs, each one attached to a certain proposition node. The users can then create and add these `conditional vote nodes' to the system in the same way as proposition nodes are added, and then all users can decide to cast their vote for the given proposition either by casting it (unconditionally) on the proposition itself or casting it instead on one of the conditional vote nodes (meaning that their vote will now be automatically conditioned on some parameters of the voting statistics in the system, continuously, until they change their vote again). 
%
%%Another thing that a more advanced system might account for, is the fact that the power of the voters might not just have different weights but might also be dependent the area that a given proposition deals with. This could for instance be in a company or a government where there are many departments/ministries in charge of different areas. If such a company/government decided to go for a more democratic leadership, it might still want to keep some division of power within the democracy. This example is perhaps not the most realistic one so here is one that is more so:  
%
%
%Another very important thing that an advanced application should afford its users is to make sure that the voters can choose representatives. It might seem odd to want to implement a direct democracy only for people to end up choosing representatives once again, but is indeed exactly what a direct democracy should aim for. It is nowhere near feasible if the system requires all users to engage in all discussions and decision making in order for the democracy to work, not unless we have a simple case with relatively few voters who are all quite engaged. But in most cases that one could think of, being able to choose representatives and trust these with looking into specific and/or complicated matters and vote on the person's behalf is all-important. The problem with representative democracies that a direct democracy aims to fix should thus not be to get rid of representatives, but simply to ensure that people can change these much more rapidly should they want to, and also that any voter can always choose to look into specific matters themselves and choose to vote differently on those than how their representative has voted.
%
%An advanced e-democracy application should therefore also allow users to choose representatives. With `conditional votes' implemented, users can of course in principle just cast conditional votes on all propositions that they want representatives to decide for them, but this is too cumbersome and we can do much better than that. The application could thus first of all allow the voters to give their votes to others. But it is likely that some users will only trust a certain representative to decide for them in a certain area of concern. And in general, users will therefore probably want to be able to have multiple representatives at a time, each responsible for making decisions for the voters in specific areas. 
%
%I therefore propose that an advanced e-democracy application also implement what we could simply call `areas of concern,' which are then essentially groupings of propositions regarding a certain subject. Whenever a new proposition is added by itself (as what we could think of as the ``root'' of a connected graph) it should thus be given an area of concern such that the application can group it with proposition graphs with the same area of concern. And whenever a child node is added, it should of course get the same area of concern as its parent. With this implemented, users should then be able to give their votes to another user (i.e.\ representatives) when it comes to any specific area, which should then effectively mean that the user will automatically cast the same vote as their representative, at least when it comes to propositions that the user has not voted on themselves. (And one might then implement different settings to this, such that a user for instance might be able to even let a representative override the user's own previous votes.) 
%%(If someone creates an otherwise relevant proposition node but adds the wrong area of concern, one can expect that it will then simply not be voted forth (over the aforementioned threshold), not until the author gives it the right area of concern.)
%
%It now almost goes without saying that each `group' in the system can then potentially choose to have their own specific representatives that the members are recommended (or perhaps required in some special instances) to use. 
%
%The system might also implement `subareas,' such that any user can try to add one such to any proposition node. Users should then be able to vote such subareas in and out, and if one is voted in, the proposition node and all its children will then get this extra area, that users can then also choose to sign representatives to. With these `subareas' implemented, this then allows users to delegate different representatives to these, even if they are also part of the same overall area of concern. **(This paragraph will probably need some rephrasing.)
%%(One could also implement `subareas' simply by requiring that these are also added from the beginning when the relevant proposition nodes are created, but it might make it easier for the users if they can just change the subareas by vote at any later time (instead of having to recreate and substitute the whole subgraph, with similar nodes but with updated subareas).)
%
%
%%These `areas of concern' also allows for something else that might be very useful, namely that the same community of users/voters can govern a variety of bodies with the same overall (potentially disconnected) proposition graph. 
%%Because when the contracts and/or agreements concerning the various bodies' commitment to follow the proposition graph are external to the digital system, a body might as well agree to only be ruled ...
%%The usefulness in this, apart from maybe having everything gathered in one place, is that this will mean that voters
%%(16:06, 06.11.22) Hm, dette kan nok godt blive mere kompliceret, fordi flere foreninger så skal blive enige om, hvordan stemmerne fordeler sig, og så skal det lige pludseligt topstyres på en helt anden måde.. Så lad mig lige se en gang... ..Hm, men handler det så ikke bare om, at forskellige grupper skal kunne "genbruge" de samme propositionsgrafer, og også om at de andre gruppers aktivitet så også godt må kunne gøres synlig i samme propositionsgraf (altså for en vis gruppe, der bruger denne)..? (16:09) ...(16:29) Jo, men så har det så ikke rigtigt så meget med subareas at gøre.. ..Nej, for så skal man også simpelthen gøre, så at graferne.. er helt adskilte, ja, så måske giver det altså slet ikke mening.. hm, andet end at man stadigvæk kunne have graferne side om side, og mere vigtigt, at conditional votes så også kan komme til at afhænge af eksterne grafer.. Ja, er det ikke bare det..?:) 
%
%
%%Furthermore, each group should have its own page and/or own `area of concern,' where the members of that group have all the voting power. This is useful since it means that each group can then build their own proposition graph over its policies and opinions. A group might then also signal external actions via this proposition graph. For instance, a group the represents a workers union might create conditional votes in their own proposition graph that depends on some statistics regarding the main proposition graph.. 
%%say that: ``On these
%
%
%And lastly, it might be beneficial for various groups in society who govern different bodies, e.g.\ political parties, unions, organizations, companies, to also be able to negotiate with each other online and to view each other's policies. For instance, a company might want to say to a governing political party that: ``if you implement certain laws, we will move out company elsewhere.'' And a trade union might then say to a company: ``if you do not give us higher salaries, we will go on strike.'' These are thus examples where a group in society can use their power over one body (including simply themselves as a group) to negotiate concessions from another group with power over a different body. 
%So if the advanced e-democracy application really wants to afford its users with all that they could want for negotiating effectively with each others, it should allow different voter groups to come together in the same space. First of all, each `group' in a e-democracy should have their own proposition graph that only they have voting power over. This local proposition graph can then be used to signal the groups policies, opinions and potential actions. And when it comes to the `conditional votes' in this local proposition graph, these should also be allowed to depend on statistical parameters in the main proposition graph, outside of the local one. 
%And furthermore, different e-democracies (governing over different bodies) that uses this same digital application should also be able to invite another group to join together, such that the two e-democracies can have their proposition graphs shown side by side (but with the same distribution of voting power in each of these graphs), and more importantly, that one e-democracy can then start making conditional votes that depends on statistics regarding the other e-democracy and vice versa. 
%
%
%\section{An e-democracy party}
%
%There are of course a lot of different examples where an e-democracy such as this could be useful; political parties, companies, unions, organizations and other communities. %In this section, I will, however, only give some points when it comes to political parties and companies of the type that I described in the previous chapter.
%%If we ..
%And when it comes to political parties, there is the natural option that these are run only by their members. We could thus imagine two or more parties competing for power, each being run e-democratically by its members. But since politically parties are typically inclusive anyway, why not just strive to have one political party where every person in the society gets an equal vote? 
%
%I believe that such a party could gain massive support over time. It might start out as a small party, especially in the early days where people are still getting used to working with the proposition graphs, and when the technology is perhaps at an early stage. And then as the technology to matures and the userbase grows, more and more people would trust the new system enough that they would want to give their vote to this `e-democracy party.' That party might then, at least in multi-party systems, get some representatives in government and by that point, the interest in the party would grow further, since all registered users would then be able to have a say in the policies of those representatives. And if the technology works, more and more people would then see the potential in an e-democracy. This is especially true in countries where the people in general do not always feel heard by there politicians: When they then see that the resulting proposition graph for most people will fit their interest better than what the traditional political parties offer, they will want for the e-democracy party to be voted in as a ruling party. 
%
%Now, if the party thus lets member of society have an equal vote in it, this might then be problematic at this stage when the party want to take over from the traditional parties, since people might then be tempted to make their vote count twice, essentially, by voting for their favorite traditional party and then also using their vote in the e-democracy party. And since voting is anonymous, there is no real way that the e-democracy party stop this. That is, apart from taking steps to balance out this effect. The e-democracy party might thus choose to temporarily break its commitment to giving all people an equal vote in this phase, and instead promise that it will commit itself to try to counter all representatives in government that are not part of the e-democracy party by giving more votes internally to a group of representatives whom it deems are exactly at the mirrored end of the spectrum than the group of non-e-democracy representatives in government. (The chosen counter-group can, however, be much larger then the group of representatives it is supposed to counter.) This way, a voter who wants the e-democracy party to take over would not be tempted to cast their votes to a traditional party instead. It also means that once the party sits on a majority of the power in government, other representatives are more likely to join it while in power (if the relevant constitution permits such migrations of representatives while in power) if they see that the e-democracy is more practical since the e-democracy party can then just remove the appropriate amount of counterbalance as these former outsiders join. 
%
%
%E-democracies as governments of countries might thus be a much closer reality in the near future, than a lot of literature on the topic seems to suggest, at least in countries governed by a multi-party system. In two- or one-party systems, the development might of course be much slower. But then again, once some multi-party governments successfully switches to an e-democracy, the two- and one-party governments would then be able to analyze and copy the technology, at least giving them a much easier route to an e-democracy, should their voters want one. 
%
%
%
%\section{Anonymity}
%
%
%As mentioned, anonymity is often very important for a democracy, especially if we think about the case of governing a country. Therefore, the digital application should allow the users to vote anonymously. This can be achieved letting each user control an anonymous profile, but if information about which user has which anonymous profile is stored on a server, that server might be hacked. 
%
%So the question is, can an e-democracy system be as safe and anonymous as going to a box, drawing a cross in a field on a piece of paper and putting that paper into a box? Yes, actually: there are ways to ensure complete anonymity of the users where the anonymity is preserved even if the servers of the application is hacked.
%
%The following protocol allows a set of clients to each provide a server with a set of public keys such that each client knows the private key of exactly one of the keys in the set (and no one else but them knows this private key) but where no one knows which public key belongs to which client apart from the clients knowing their own key. The protocol is furthermore resistant to DoS attacks. 
%
%It works by having the clients take turns building blocks in a block chain, which we can think of as a `block spiral,' where the clients form a circle and where the turn to provide a new block to the spiral goes around in the circle. 
%
%\ldots\ \textit{Okay, jeg tror lige, jeg venter med at forklare om min idé her, for det kan godt være, at der findes en lidt nemmere måde. Det vil jeg lige tænke over. Men ellers er det en god idé, altså den hvor hver klient sender nogle nøgler videre til en tilfældig anden klient i kredsen (hvor hver blok krypteres med den næste klients offentlige nøgle (fra begyndelsen) og sendes til denne), og hvor klienter, der modtager nøgler gerne skal sende dem videre og slette dem fra hukommelsen. Herved vil man meget sjællendt kunne se, hvem var den oprindelige sender af en nøgle (medmindre både modtagerklienten og klienterne for og bag brugeren er ondsindede), og selv hvis den bliver sporet tilbage kan pågældende klient bare sige, at ``den nøgle kom fra en tidligere omgang og altså fra en helt anden bruger, men jeg har altså slettet data om, hvor den kom fra, som jeg burde.'' Men ja, jeg tænker nu lige lidt mere over det, inden jeg skriver denne sektion færdig. .\,.\,Jeg har i øvrigt også tænkt mig at sige, at man efter at have brugt denne protokol så bare kan bruge et VPN herfra, men hvis man vil være endnu mere sikker, så kan man endda bruge helt den samme protokol til at indsende data om, hvordan man vil stemme med sin profil, hvor man så altså bare erstatter de (tilfældigt) genererede nøgler i protokollen med tilfældigt genereret data samt det faktiske data, man vil indsende, og til sidst så offentliggør man så bare, hvilket skrald, man har sendt ind, men ikke den faktisk data, man så lader serveren beholde. (.\,.\,Så kan det dog godt være, at man skal ændre protokollen lidt, så man lige sørger for, at hver mængde data også vil nå slutningen af protokollen, så at ingen data-klumper bliver tabt i protokollen --- medmindre der altså er sket en synlig fejl i protokollen.)}
%
%\ldots\ \textit{Nej, der er vist en nemmere protokol, hvor man vist nok også kan finde frem til en DoS attacker. Man kan vist bare have et VPN, hvor klienterne sender beskeder frem og tilbage, og hvor de så kan pakke en nøgle ind i flere krypteringer med forskellige nøgler, hvor beskeden så skal sendes til alle de klienter i rækkefølge, som kan dekryptere beskeden en efter en. Og hvis så man gør det tilfældigt, hvor mange krypteringer, der skal til, så kan ingen igen vide, om en nøgle kom fra en person, bare fordi de får opsnappet, at beskeden på et tidspunkt blev sendt fra denne, for vedkommende kunne jo sagtens have fået den fra andre og så bare have sendt den videre. Og hvis man så har nogle få DoS'ere i netværket, så kan brugere der har sendt en nøgle der aldrig nåede frem jo pege på, hvem der kan have været de skyldige (af den række af brugere).\,. Hm, ja, men hvis man nu vil bevise det også, så kunne disse brugere.\,. Nå nej, man kan ikke bevise det på et VPN, men det gør vist heller ikke noget. For brugere skal jo stadig gerne sende flere nøgler pr.\ protokol, og hvor de så bare opsiger alle på nær én til sidst. Og hvis der så er en DoS'er i netværket, jamen da det ikke vil være fatalt, så må det være fint nok, at brugerne bare kan page dem ud nogenlunde. (Og hvis det så bliver et større problem, så kan man altid bare bruge den mere krævende blok-spiral-protokol, jeg har haft tænkt på.)} %(08.11.22, 10:27)
%
%
%
%
%
%
%
%%Hm, jeg har fået tænkt lidt over anonymitet, men det kan godt være, at jeg lige skal tænke lidt mere. Men jeg har altså fundet på nogen fine systemer til at skjule stemmeres identitet, og jeg tænker, at stemmere generelt skal kunne vælge enhver tredjepartsbruger til at videreformidle deres stemme anonymt. Sådanne kunne så med fordel få lov at give floating point værdier (i stedet for bare 0 eller 1) med deres stemmer, eller de kunne bare råde over et antal stemmer, således at de både kan give et antal positive og et antal negative stemmer til hver proposition (men jeg tænker at det første næsten er nemmest..). Og ja, så kunne én form for sådan en tredjepart så være en organisation med fysiske lokationer, hvor medlemmerne så kan møde op personligt og ændre deres stemmer og/eller repræsentanter, og hvorved organisationen kan opdatere deres stemmer herefter med en frekvans, der kan afhænge af, hvor mange ændrer deres stemmer ad gangen over en gennemsnitlig periode. Og en anden, meget smartere;), måde at have en videreformidlingsrepræsentant på, kunne så også være.. hm, lad mig lige se.. (13:50) ...(14:30) (ordner også vask) Jo, man kan også have en videreformidlingsrepresentant, der fungerer via mindst to tredjeparter, som klienten selv kan vælge. Først er der en trejdpart, eller instans bør vi nok hellere kalde det, bare.. som via asymmetrisk krypering får en nøgle fra hver bruger, som kun denne instans og hver enkelt relevant bruger må kende. ..Ja, eller på nær at de også så skal sende alle disse nøgler til en anden instans, der heller ikke må offentliggøre dem, og som så i øvrigt ikke ved hvor hver enkelt nøgle stammer fra (og må ikke få dette af vide af første instans). ..Hm, vent, giver dette mening..? ..Ah, jo, jeg kan få det til at give mening, men lad mig nu lige se.. (14:36) ..Hm jo, denne instans nr. 2 kan så også få en offentlig nøgle med fra brugeren til hver enkelt nøgle af første instans, sådan at denne altså bare får et sæt af nøgle par, hvor den ene er en offentlig nøgle. Denne instans kan så kryptere.. Hov, nej, så behøver vi faktisk ikke den første nøgle; instans nr. 1 sender altså bare et sæt af offentlige nøgler videre (gennem en krypteret kanal) til instans nr. 2. Denne offentliggør aldrig disse, men bruger dem hver især til at kryptere en besked med en ny nøgle i, og offentliggør alle disse krypterede beskeder. Brugerne prøver så at dekryptere dem hver især, indtil de finder deres egen.. Hm, er dette får ressourcekrævende, eller skal denne instans også lige tilknytte et meget lille hash a hver offentlig nøgle med beskeden, så hver bruger ikke skal igennem så mange..? ..Det kunne man sige.. ..although.. ..Tjo, men brugerne kan så stadig downloade alle beskeder i rækkefølge og så bare nøjes med at beholde dem, de skal tjekke.. Hm, lad mig lige tænke, om ikke der er en smartere løsning.. ..Hm, men ellers var pointen så, at enhver bruger, som ikke får en passende besked, bør så anråbe dette, hvorefter alle nøgler så skal indgives, sådan så man kan finde ud af, hvilken part var synderen (inkl. anråberen, hvis dette var en fejl), hvorefter man så kan starte forfra, muligvis uden synderen. Men når hver bruger så har fået en ny nøgle, som kan kan spores hen til dem, hvis alle de involverede instanser (for man kan godt have flere nr.-2-instanser her) bryder deres løfter og offentliggør deres data (og ikke bare sletter det kort tid efter). Nu kan man så være sikker på, at alle brugere i gruppen har netop én anonym nøgle, som nu kan bruges til at oprette en anonym bruger profil for hver bruger, selvfølgelig med VPNs involveret, hvormed denne frit kan afgive sine stemmer og ændre dem, hvornår det skal være, uden at det kan spores tilbage til dem. (14:52) .. ..Og disse anonyme brugere kan så udløbe således at de skal opdateres en gang imellem, således at hvis nu nogen for lækket deres bruger, så vil det allerhøjest kun være den seneste aktivitet, der bliver lækket (og derudover kan man selvfølgelig også dele brugeren op i flere (der ikke kan kædes sammen af andre), hvis man synes, der er besværet værd, men ja, og sådan vil der selvfølgelig altid være ting, man kan tilføje, hvis man finder frem til, at det giver mening..). Nå, men selv hvis der findes et bedre system end dette, så kan jeg jo bare skrive, at det f.eks. ikke er svært at finde på systemer, hvor man via flere instanser, der hver især holder på sin del af en samlet hemmelighed (hvor alle stykker skal bruges, hvis man vil spore tilbage), kan opnå at hver bruger i en gruppe får netop én anonym bruger. Og ja, hvis man så sørger for at de udløber med jævne mellemrum.. Og at brugerne skifter.. Hm.. ..Hm, men det er nu ikke perfekt anonymitet, hvis man sammenligner med valg, hvor ingen data bliver gemt til at starte med, således at ingen nogensinde kan spore det tilbage.. Hm.. ..Hm, men kunne man ikke bare bruge en teknik, som jeg vist også har tænkt på før, hvor en instans bare offentliggør en mængde af.. Hm.. ..Hm jo, en mængde af dens egne offentlige nøgler, nemlig med et antal svarende til antallet af klient-deltagere i øvelsen, og hvor hver klient så vælger et hemmeligt ID, krypterer.. Hm, nej, lad mig lige se... ..Hm, hvad med at alle klienter bare opretter et VPN kun med demselv som noder, og så begynder at sende data rundt. På et tilfældigt tidspunkt sender hver bruger så et ID videre til en naboknude, som modtager, sender ID'et videre til én naboknude, og noterer også ID'et og modtagelsestidspunktet.. nej.. Hm, dette virker vist næsten, men ikke helt.. ..(15:21) Ah, nu har jeg det måske. Man kunne lave en kæde af krypterede blokke, hvor hver blok offentligt hører til en klient, og hver blok rummer data, som brugeren fik tilsendt af ejeren af den tidligere blok, og data som brugeren har sendt videre til næste klient. Denne blokkæde kører så på omgang i en ring, således at den tager flere runder. Og på et tilfældigt tidspunkt tilføjer hver bruger så et offentlig nøgle, som de sender videre. ..Hm, nej det er endnu ikke helt vandtæt.. ..Ah, men måske hvis man tilføjer sin nøgle i krypteret tilstand, så den først kan lukkes op, når den når til en (tilfældigt udvalgt) anden bruger.. Hm, spændende idé.. (15:27) ..Ja, man må næsten kunne lave sådan et system, hvor klienterne billedligt talt danner sådan en rundkreds, hvisker data videre til hinanden én ad gangen i rundkredsen, og hvor klienter i kredsen så kan kryptere en hemmelighed, som en klient et andet sted så kan forstå. Denne bør så med det samme kryptere en ny besked, hvormed denne hemmelighed kastes videre til en anden person. Hemmeligheden er så en offentlig krypteringsnøgle. Man slutter så, efter et vist tidspunkt, når man er næsten 100 \% sikker på, at alle brugere for længst vil have kastet deres nøgle ind i rundkredsen, og at denne er læst af modtageren. Alle brugere offentliggør så de nøgler, der har været sendt frem til dem. Herefter skal alle brugere/klienter (jeg kan ikke lade være med at skrive "brugere" i stedet for klienter, men det er vel også næsten ligeså godt..) så sige, om deres nøgle er iblandt de offentliggjorte (men selvfølgelig ikke udpege dem). Hvis antallet af nøgler passer og alle brugere/klienter siger, at deres er med i mængden, så stopper "legen" succesfuldt. ..Eller rettere, det gør den, efter at man så beder alle brugere om at slette de nøgler, de ville have brugt til at dekryptere deres egne blokke med. Og sikkerheden i systemet handler så om, at man har tillid til, at størstedelen af klienter vil gøre dette (selvfølgelig fordi de bare bruger det udleverede software til det, og ikke har bygget eller tilegnet sig en malicious kopi af denne software).. Men hvis der er for mange nølger, eller at en klient mangler en nøgle, jamen så må man så bede alle brugere om at dekryptere alle deres blokke. Og så er pointen, at man kun ved at have alle disse blokke dekrypteret, kan finde frem til, hvem der er synderen, fordi man så både vil kunne se, hvis de ikke har opfundet netop én nøgle selv, og fordi man kan se, hvis de ikke har videresendt den rette nøgle hver gang.. Nå ja, og hver bruger skal så også bare i det hele taget indsende deres private nølger, så man kan finde frem til synderen. Og hvis enten en klient nægter at indsende den private nølge i dette tilfælde, eller hvis man finder synderen ved at dekryptere alle nøglerne, så må man så udelukke denne bruger i næste tur (altså give denne karantæne). Men ja, som sagt, hvis legen derimod ender succesfuldt, så skal brugerne endeligt ikke offentliggøre deres private nøgler, nej faktisk skal de slette alle deres nøgler, der blev brugt under selve legen og kun beholde den private nøgle, som passer til den offentlige nøgle, de herved fik indsendt anonymt via legen. Og ja, så længe de fleste brugere bare gør dette, så er man ikke i fare for, at det bliver afsløret, hvilke nøgler i slutmængden hører til hvilke klienter. :) (15:53) ..(16:02) Hm, der er faktisk en lille smule hangman's paradox tilstede i denne løsning, men det kan man vist gøre bod på ved bare at sige, at hver knude.. Hm.. ..Hm, eller hvad i stedet med bare at gøre sådan, at klienter i kredsen generelt skal vente et tilfældigt antal omgange, inden.. hm, men det løser dog ikke problemet eksakt.. (16:07) ..Hm, men jo, man kunne vel også bare sørge for, at sandsynligheden for at ens software sender en nøgle starter virkeligt lille og kun vokser over mange runder, og så kunne man gøre sådan, at hvis en bruger bagefter kan se, at deres software har sendt.. Hov, vent, dette er da slet ikke et problem, netop fordi man kaster hemmeligheden frem i rækken.. hm.. ..Hm, der skal kun tre (specifikke) andre brugere til at afsløre en i denne løsning, men de kan det kun hvis man har været uheldig at softwaren har sendt ens nøgle tidligt.. ..Hm, man kunne også bare give hver bruger mulighed for at afbryde legen, hvis deres sofware har sendt deres nøgle tilstrækkeligt tidligt.. Hm.. ..Hm, i øvrigt kan man hurtiggøre processen, hvis kredesn har mange kæder i gang på én gang, så alle klienter kan bygge en blok i hver runde (nemlig hvis der er ligeså mange kæder i gang, osm der er klienter i kredsen).. ..Hm, men kan man ikke bare generere flere nølger, end der er behov for..? (16:18) ..Jo, og så kan brugerne/klienterne til sidst bare vælge, hvilken nøgle af dem, de har fået genereret i legen, de vil beholde, ved at.. Hm.. ..Ah, ved selvfølgelig bare at bekende offentligt bagefter, at "disse nølger var mine, men jeg skal ikke bruge dem alligevel."!:) Og hvis så der lige præcis bliver det samme antal efterfølgende, som der er klienter, og hvis alle meddeler, at de har en nøgle iblandt de endelige, så når man i mål, og ellers må man så bare til at optrævle kæden, for at finde DoS-synderen, hvis ikke legen ender som den burde. :) (16:25) Og ja, det skal så bare anbefales, at hver bruger ikke vælger en nøgle, der blev genereret helt i starten af systemet, men ved at det stadig er brugerens beslutning at udvælge den ønskede nøgle, så eliminerer man altså hangman's paradox.:) (16:26) ..Nå ja, og lad mig lige præcisere, at hver blok så skal indeholde en liste af krypterede nøgler (som hver er kryperet med en tilfældig andens offentlige nøgle), og denne lister vokser altså bare.. tja, eller man kan måske begynde at fjerne ting fra bunden af listen efter et vist stykke tid, når det er sikkert, at samme nøgle er blevet indsat igen i ny version (nemlig ved at en knude har dekrypteret og re-krypteret nøglen og sat den på). Og man kan så kræve, at hver knude tilføjer netop én ting til listen i hver runde.. how, "runde" er et dårligt term at bruge for hvert enkelt lille step, når vi har en rundkreds, så lad os kalde.. tja, lad os bare kalde det enten hver 'step'/'skridt' eller hver tur.. nej, lad os udelukkende kalde det 'skridt'/'step.' Og hvis en bruger så modtager flere beskeder på én gang i et step.. ..Hm, nej vi kan også godt kalde det turn i stedet (for så tænker man jo bare på et lille turn af hjulet).. Så må denne bruger så altså gerne vente en omgang med at sende nummer 2 besked (osv., hvis der modtages flere end to), og altså så kun videresende én af nøglerne i den første tur, hvor nøglerne modtages. Ok, så det var vist bare det, jeg lige skulle præcisere.. (16:40)
%
%%(16:42) Nå, men der er også et andet issue, jeg skal tænke over, og det handler om: Vil det ikke være for fristende for folk at stemme på deres vante repræsentanter i et regeringsvalg, hvor et e-demokrati kæmper, og ser ud til at kunne vinde? For hvis man gør dette, så vil man vel kunne få dobbelt magt, medmindre e-demokratiet kan se, hvem der ikke stemte på det.. hvad de jo ikke vil kunne.. Hm, måske er dette et ret stort problem, men ja, nu vil jeg altså give mig til at tænke godt over det... (16:44)
%
%%(31.10.22, 9:21) Kort efter, jeg klappede i i går kom jeg frem til, hvad vist også havde været oppe at vende i periferien af mine tanker tidligere på dagen, at den simple og måske eneste løsning nok bare er, at sørge for, at e-demokrati-partiet i starten også har til opgave at booste stemmevægte inde i systemet (på en helt transparant måde selvfølgelig), således, at alle repræsentanter, der ellers har fået mandater udover partiet, de får en modvægt til sig inde i partiet. På den måde kan det ikke betale sig at stemme uden for partiet for at pågældende mening skal få mere magt, for så vil den pågældende mening bare blive countered. Og ja, det er så partiets opgave at finde frem til og være ærlig omkring, hvad der er midten af det politiske spektrum i henhold til forskellige punkter, således at man kan counter'e et vist mandat ved at give mere magt til en (eller flere) fra den modsatte (i.e. spejlede) ende af spektrummet. Når partiet så er i regering, så kan man så også bede de repræsentanter, der ikke er med, om at joine, for så vil e-demokratiet bare fjerne magt igen fra dem, der står for at counter'e/udbalancere magtbalancen.. (9:29)
%
%
%%\section{A note on transparency}
%
%
%\section{E-democracies in companies}
%
%%To finish this chapter, let me just make a small point about how an e-democracy application like this might also be incredibly useful when it comes to democratically run companies, or indeed the almost-democratically run `Economically Sustainable' Companies (ESCs.\,. hm, that looks a lot like `Escape(s)'.\,.) that was described in Chapter \ref{MSE}. 
%%
%%If the company in question has a goal of expansion, such as should be the case for the 
%
%To finish this chapter, let me just make a small point about how an e-democracy application like this might also be incredibly useful when it comes to democratically run companies, or indeed the almost-democratically run `economically sustainable' companies that was described in Chapter \ref{MSE}. 
%
%If the company in question has a goal of expansion, such as should be the case in general for the `economically sustainable' companies as described, I envision that this venture will be all the more exciting for the participants if there is a vibrant online community that engages in discussing and finding what strategies to go ahead and try in order to expand the company. 
%
%And if this e-democracy application can be as useful a tool for this as I believe it can, it could thus accelerate the interest in taking part and supporting such a company immensely. 
%
%%Hm, skal jeg så bare stoppe her for nu? (Eller skal jeg skrive videre på denne sektion, og var der i øvrigt andet, jeg har glemt at nævne..?) (15:21) ..Hm, jeg har glemt at nævne min pointe omkring gennemsigtighed ved at sørge for, at folk med jævne mellemrum bliver udtaget til at sætte sig ind i detaljerne og så rapportere tilbage til den interessegruppe, der udvalgte vedkommende, men måske jeg bare skal gemme denne pointe til en anden gang.. (15:23)
%%...(16:01) Nej, jeg tror ikke, jeg behøver at tilføje mere nu. Når jeg så lige får tænkt lidt mere over spiral-protokollen, så kan jeg skrive om den, og ellers er det nok bare lige at redigere teksten. (16:02)
%
%
%
%%Husk:
%	%Jeg havde tænkt mig her at nævne det med, at det kan være smart at udvælge nogen (som så jo kan vælges til at være upartisk og/eller repræsentativ (men måske smart/intelligent nok)) fra en gruppe til at studere og gennemgå systemet i nærmere detaljer og så rapportere tilbage..
%	%Transparancy.
%	%
%	%You never have to waste a vote (and never have to fear wasting a vote). And never have to be fearful, that who you voted for does something you didn't expect (since this system requires no trust in representatives, at least not except in cases why you don't feel like you have the time (or interest) to go through the details of a matter).
%	%..(16:47, 29.10.22) Hm, og husk det her med at man kan have flere områder, hvor forskellige bestemmer, og at dette så også gør, at andre grupper kan logge sig på i systemet, hvor vi snakker om at styre et land. I et sådant e-demokrati kan grupper altså også tilføje områder. På den måde kan de gøre det offentligt for alle, hvad de har tænkt sig. Hermed kan vi altså få en stor markedsplads, der handler om at lave aftaler og bestemmelser, både i regeringen, men også i andre instanser (det kunne f.eks. være såsom fagforeninger, hvilket jo vil være meget relevant i den sammenhæng). ..Og ja, det kan også være grupper, der egentligt ikke har nogen anden magt over noget, men som alligevel vil oprette et område, der hedder "vi mener sådan og sådan, og vil vil gøre sådan og sådan," altså et område, hvor de kan signalerer til omverdnen, hvad deres interesser er, og hvad de gerne vil / er parate til at gøre. (16:54) ..Og ja, det kan så nævnes, at dette så også kan være sådan noget som at trække sig fra den overordnede gruppe (f.eks. e-regeringspartiet eller trække sig som kunde og/eller investor i et firma). 
%	%Jeg kunne godt nævne muligheden i "forbrugerforeninger" kort også (som et eksempel på anden form for magt), men så tilføje, at min kd.v.-idé så netop nok ville være endnu bedre her, for så kan man undgå sådanne reprimanter (eller hvordan det staves). Men om ikke andet kunne det så blive en måde at tvinge gang i en kd.v., hvis nu virksomhederne indenfor en branche er tøvende med det. 
%	%Jeg skal forresten huske at have område-repræsentanter med under avancerede punkter, sammen selvfølgelig med områderne selv. Jeg kan således nok godt nævne "områderne" først, også selvom det egentligt er vigtigere, det med at kunne vælge repræsentanter.. 
%	%"Det handler om at det bliver: meget lettere at samle sig i små grupper, og meget lettere at sætte i gang i en proces, hvor man overvejer, om ikke der kan gøres noget ved et forhold, netop fordi man bare kan starte denne diskussion i nogle små grupper (som så kan kontakte andre grupper, små eller store, når de har fået samlet en oversigt over, hvad problemet er, og hvad man kunne gøre for at løse det m.m.). Så altså langt større tilgængelighed for den enkelte og dermed mange mange flere mennesker aktiveret ad gangen (som så overvejer og finder på løsningsforslag til problemer i samfundet (ofte særlige problemer for nogen specifikke i samfundet, men det kan jo også være mere almene)). Og så vil der så derefter også kunne være meget kortere tid til, fra løsningsforslag til løsning i sådan et direkte demokrati, der er klart. Og ikke mindst vil folk (i grupper) få langt nemmere mulighed for at indgå selv komplicerede politiske aftaler med andre folk (i grupper (ikke nødvendigvis disjunkte med de første, btw)), således at man får et meget bedre og hurtigere kan få handlet sig til at få opfyldt sine behov som en gruppe af mennesker, og således at smafundet derfor vil blivet meget bedre fintunet, så at sige, til at opfylde så mange menneskers forskellige behov som muligt på en gang."
%	%(15:01, 01.11.22) Jeg skal huske noget, jeg lige fik tænkt på, og det er, at et sådant demokrati kan få en meget meget fladere struktur, hvor at man, når man har en ny idé til forandring, lad os sige som lille gruppe, i stedet for så at skulle indsende og ansøge om idéen til en central, så kunne man i første omgang dele den, med den/de mest relevante nabogruppe(r). Hvis de så også er med på den, så kunne man så brede det til endnu flere. Og når idéen så har samlet nok opbakning, så kan man melde det til det brede fællesskab, hvor idéen så allerede har opbakning, når den ansøges om. Jeg ved godt, at sådanne måder at fremføre idéer på allerede finder sted mange steder, men jeg tror, at man i et e-demokrati kunne gøre den fremgangsmåde endnu nemmere og endnu mere hyppig.. Hm, måske vil jeg skrive om dette, men om ikke andet er det da bare rart at tænke på, at der kunne blive sådan en rigtig flad struktur, hvor relaterede grupper selvstændigt kan diskutere og handle om, hvilke idéer og forslag, man vil gå videre med..:).. (15:08)
%	%Man kan også bruge min blok-spral-idé til når stemmerne skal kastes..!
%
%
%
%
%
%
%%(09.11.22, 9:46):
%\section{(I'm considering adding something like:) A similar application for scientific discussion}
%
%\textit{I have now realized that this application could also be used for scientific discussion graphs, which goes hand in hand with decision making since facts are of course important when deciding policies. In a discussion graph, on would just not really need the `conditional node' edges, but would instead just use the `conditional votes' instead --- which could then be drawn as edges between notes for this type of application. %(This all of a sudden make this idea quite a bit more interesting for me in terms of what I would like to work on myself.\,. .\,.\,Hm, hvilket er relevant for mig at have i tankerne i denne stund, for jeg skal nemlig snart til jobsøgeningsmøde med A-kassen. Og ja, med denne indsigt, så må det da næsten være denne idé, jeg vil prøve at gå videre med (og sige jeg vil iværksætte), det tænker jeg.. (..Altså i stedet for Web 2.0--3.0-idéen/erne.))
%*And it should then be very much recommended (as a key part of the idea), that users try to commit themselves to continuously update their votes for propositions as conditional ones, once more fundamental propositions are added to the system. A scientist might for instance be an expert on drugs and say (or actually ``vote'') that: ``this drug is so and so addictive,'' but then once propositions are added about the existence of relevant studies are added, as well as propositions about trust, then that scientist (along with everyone) are then strongly recommended to change the vote into a conditional vote such that the vote now depends on the study existence proposition and the trust proposition. This way (if the community follows this (strong) recommendation), every proposition can slowly become more and more founded in the basis empirical propositions/data, plus trust propositions (which are essentially propositions about how the users want to apply epistemology, i.e.\ when these propositions are also boiled down to their roots). This both has the advantage of the system being more flexible, when new studies turn up or if old ones come into question at some point, and also, importantly, it makes it easier to browse and find out what fundamental facts our more abstract facts in society are built on, i.e.\ to find the sources, and it also gives a better and easier understanding of what is interesting to research, since it shows were the ``gaps'' are, so to speak, or more precisely: where the research is thin and could use bolstering. 
%}
%
%
%
%
%
%
%
%
%
%
%
%
%
%
%
%
%
%
%
%
%%\chapter{A possible road towards Web 3.0}
%%
%%
%%**(Lad mig bare lige skrive, hvad jeg tænker nu at skrive i dette afsnit/kapitel bare ud i én køre, og så kan jeg altid redigere bagefter..) %(06.11.22, 9:46) (Jeg fik nemlig lige tænkt en del over emnet igen i går aftes, og nu synes jeg alligevel, at jeg bør kunne forklare meget af det ret kortfattet..:))
%%*(Okay, jeg har alligevel ikke tænkt mig at beholde dette kapitel, men lad mig bare lige skrive denne køre færdig, også fordi jeg har nogle små nye gode tilføjelser, mener jeg..)
%%
%%
%%\section{Everything section}
%%
%%
%%My idea for how we can reach the promises of Web 3.0, and specifically the Semantic Web, is to first implement a Web 2.0 site with an underlying semantic structure and then really try to give the users a lot of power to redesign things on the site and to program algorithms themselves. This implementation of the Semantic Web then does \emph{not} rely on XML/HTML. Instead, all semantic sentences should be recorded in relational database. 
%%
%%This is radically different from the first implementations of the Semantic Web, where metadata is simply added to various sites and resources on the web, and then algorithms in the Semantic Web would simply work by querying the web (i.e.\ the World Wide Web) and finding the necessary information online. But when all the semantic sentences (what is also called triplets in the current conventional implementations) are stored in one database, the algorithms can run way faster. 
%%
%%Essentially, one can say that the idea is to start out with a Web 2.0 site as we know them, e.g.\ such as YouTube, Reddit, Twitter, etc., and then implement the Semantic Web there. But hold on, you might say, it can hardly be called the Semantic \emph{Web}, if it is controlled by a private company. No, but if it is instead controlled by a open source organization (similar to how the web is run today, e.g.) it is another matter. Hereby it can be ensured that no one owns all the user contribution, save perhaps for the relevant user, and that any other organization can always come and take up the mantle at any time, should it be needed (just how it also is with Wikipedia).
%%
%%Alternatively, if starting this idea as a non-profit organization is slow and lacks investment, one could also start it as the type of company as described in Chapter \ref{MSE}, such that the ``organization'' can start out as a private, commercial company, but where it is guaranteed that the users will slowly become the owners. 
%%
%%But let us move on from this topic for now and assume that the organization will have plenty of funding (just like Wikipedia has). 
%%
%%
%%Let me now try to explain the overall design of a Web 2.0 site that I envision, which has an underlying semantic structure. Some of the details here are more important than others (and some are less), but it is nice to see a good example that could work (and attract many users), and then from there, I can explain why the underlying semantic structure becomes important. 
%%
%%If I were to design such a Web 2.0 site, where the intention is that it can grow into a Web 3.0 site over time, I would probably give it this following initial design:
%%
%%A main feature of the site should be a page with a category tree, which I would implement basically as a structure of tabs, i.e.\ the kind of tab system we see everywhere in the interfaces of Windows and Mac applications and so on. Whenever a new tab is selected, it will then potentially open a new list of sub-tabs. The user might thus have selected the category `movies' as a tab, and then a submenu of movie categories should open. Thus, we get a category tree (which hopefully should be pretty quick and easy to navigate as a user). Whenever a new list of (sub-)tabs is in focus, it should be expanded as a whole box of selection, in fact one might even implement it as a whole HTML page at some point (instead of just a box containing a lot of tabs). But when a tab/subcategory is selected, the box/page should nevertheless collapse into just a single bar of a horizontally adjacent tabs, where one tab is then selected. And underneath, the new subcategory selection should then automatically expand. Also, if a user has navigated down into a category tree, but want to go to a different super-category, the user can either just click on some of the visible tabs in the one of the above tab bars (which are aligned vertically adjacent, each with tabs aligned horizontally adjacent), or he/she can expand that given tab box once again.
%%
%%Okay, that was a lot of details to explain a very simple design, but it is nice to have an example to hold on to, and one that does the job. But of course, there could be many other types of design that this Web 2.0 site might start with.
%%
%%Moving on, now that we have a category tree, we should also have some resources in it, of course. So at the same time as the user selects these categories and subcategories, there should be a list of resources at the bottom that is updated in principle whenever the user chooses a new category (although the user might want to click a button manually to make the resource list refresh such that it doesn't refresh al the time while the user is navigating the category tree).
%%
%%When the user then selects a resource from the list, the user is led to a the page of that resource. That resource is then displayed pretty much at the top of that page. And how the resource is displayed then depends on what kind of resource it is, i.e.\ whether it a video or a HTML page, and so on. And each resource should then also have a list of comments below, but similarly to the all the main resources of the site, if we can call them that, these commant should also be ordered in a similar category tree. Examples of different categories of comment could be `related resources,' `user reactions,' `related discussions,' `links to source material,' and so on and so forth. 
%%
%%And to finish up this description of this basic design, there should also be a homepage where each user can see one or more lists of the user's favorite categories and resources, such that the user can quickly navigate to some of their favorite spots in the category tree (without having to start from the root and navigate down).
%%
%%Okay, that was a quick sketch of a quite basic site design. 
%%%
%%%...Jeg har fået tænkt lidt mere. Nu ved jeg faktisk bedre, hvad man skal sige gælder for de rating-tal, der skal følge med sætningerne/tripletterne. Jeg tror ikke, det vil tage mig lang tid at færdiggøre denne hurtige udredning herfra, men lad mig lige se, om jeg lige vil skifte emne lidt og skrive på noget andet, eller om jeg vil holde en lille pause.. 
%%%...Okay, jeg prøver at skrive færdigt..
%%%
%%Now I can get on to some of the stuff that is actually interesting.
%%
%%Assuming that the reader knows about the Semantic Web (and about triplets and so on), the reader might have already guessed that the categorization of the resources should then of course be user-driven. The users should thus be able to say for instance: ``this resource belongs to this category,'' and thereby be able to vote resources into various categories. Note that ``this resource belongs to this category'' can be implemented as a triplet. The users should also be able to say ``this category is a relevant subcategory to show under this other category.'' The users should thus also be in control of the category tree --- and of the category trees under each resource (where one can reuse resource category trees for similar types of resources).
%%
%%So far so good: One thus get a Web 2.0 site where the categorization structure is semantic and user-driven. And because it is semantic, all the user data can easily be reused in for other similar sites, and specifically also for other implementations of the site in question.
%%
%%If such a Web 2.0 site can become more and more popular, and if it is run by an open source organization (and/or community) as mentioned, this site might thus effectively become all that people hope for in terms of what Web 3.0 might bring.
%%
%%Okay, at this point I have explained the overall idea, and also explained an overall type of implementation that could be the starting point for a Web 2.0 site that thus aims to become, what we could call af Web 3.0 site (bringing forth the features that people hope for in Web 3.0). Now I will move on to the \emph{really} intersting stuff, because I actually have a few idea that i believe can make such a ``Web 2.0--3.0 site,'' as I like to call it, really take off! 
%%
%%I actually believe that \emph{triplet} system will not be enough to carry forth a really useful Semantic Web (which is a big part of people associate with Web 3.0)! %(12:44)...
%%
%%First of all, it is important that the ``triplets,'' but let us actually just call them `relations' or `sentences' instead, should contain the user ID of who uploaded it, as well as a timestamp for the upload. So they should not just have the three entries. Second of all, I think it is \emph{so} important for the usability of the system, that each relation/sentence can also include a number (with whatever precision is appropriate for the case) that signifies a rating of \emph{how much} the user believes the sentance to be true. 
%%
%%This makes it possible to \ldots \textit{Okay, jeg har skrevet så meget af det her allerede, så lad mig ikke gentage alle pointerne her, nemlig da jeg nu igen har besluttet mig, at jeg alligevel bare børe vente med at fokusere på dette emne. Så lad mig i stedet bare lige ridse mine nye tilføjelser op.\,.}
%%
%%Okay, let me make this short and just mention the new thoughts that a had about this idea. The rest of the ideas, as well as the explanation of why they will be so good, can be found in my 21--22 notes (in \texttt{main.tex}, as the document is still called in the moment of writing).
%%
%%My big idea for making it easy and attractive to rate the resources on their lists, is that they can simply drag them up and down on the lists to rate them (according to the proposition that they are viewing). So when the user moves the resources in the list around, it should generate sentences/relation to the database, where the rating number in these relations are determined by where the user drops the resource in the list (and where only the most recent adjustment applies (which is what the timestamp is useful for determining)). The number might run from 0 to 1, or from -1 to 1, or whatever; that does not matter much (and the site can always change the conventions and then simply convert the previous data to such that it is scaled to the new convention). And the scale should only be very vaguely defined. The real precision that the user should worry about is how the number related to the neighboring resources on the list. So if the user believes, say, a movie to be incredible, and it has a low rating, the user might want to pull it up closer to 1. But if the question is, should it have a 0.6 rating or a 0.9 rating (assuming 1 is the highest score), that should actually only depend on the existing scores of what movies have received scores in about that interval. So the underlying rating should thus be primarily defined in relation to what resources are already rated. And then! If one wants to turn the resulting rating into one where points on the rating axis is more precisely defined, one can then just (and should be able to), upload a translation of that rating, which is basically a conversion function that takes the primary rating and converts all the numbers to the new rating. This `translation' function can then simply be defined by taking a bunch of resources, plotting them in on the list, and then use statistics after that to plot in all the other resources on that new axis (with an updated metric). And by putting a Gaussian ``error'' on all the ``fixed'' resources on this new axis, one can make sure that this process does not run into contradictions. So in short: The normal rating axis that is used when users drag and drop resources to rate them should have very vaguely defined semantics to begin with, and then the users can always translate the resulting axis from all the user activity into something with a more precise meaning, simply by defining a new metric for the axis that moves the resources into new positions on the axis.
%%
%%I also want to mention that when users drag and drop resources, they should be able to dial up and down the number of resources shown on the list as the drag and drop. Here, a setting to show few items in the list could thus only show the most `popular' items, i.e.\ such that an external predicate can be used for defining this setting, other than the predicate that orders the list. The lists should thus also have `filters,' and these filters should have different settings. And if the user can change these settings by hitting some keys, they can basically ``zoom in and out'' in the lists, namely by changing the filter dial to show fewer or more `popular' items, e.g. The user can then be ``zoomed in'' and chose a resource to rate. The user would then start draging it up or down, but since the list is long when ``zoomed in,'' the users might then hit the key to ``zoom out'' while dragging the resource. And when the user find the desired spot to drop it, the user might even not drop it right away, but ``zoom in'' a bit first to find a more precise spot for the resource. 
%%
%%Okay, I think that was it.\,. no, wait, maybe I also want to quickly reiterate something about the subcategories actually being implemented via `compound predicates,' and also that when rating such predicates, the user actually have to rate each atomic predicate individually (such that the `compound predicates' are not meant for rating, but only showing resources in a list.\,. oh, and then if the user wants to filter the list such that only resources of the one category/predicate is shown while rating resources in terms of another predicate, the user then just has to use the `filter' that I just mentioned.\,.).\,. Hm, no, I think that I have already covered these point in my 21--22 notes (in \texttt{main.tex}). So let me just stop again with this subject for now.\,. 
%%
%%\ldots\ Well, let me just mention another thing quickly, namely that the site might also use some automatically generated meta-sentences/relation, such as: ``this resource was uploaded by this user'' or ``this resource was uploaded as a comment to this resource.'' These are thus automatic sentences/relations that the site itself is responsible for applying to all uploads.
%%
%%.\,.\,Oh, and I also intended to mention something else, by the way: I wanted to mention that certain resources, e.g.\ HTML resources (or other markup), might actually get access to the database themselves. For instance, a HTML document might say ``insert a list of the top resources in this category here'' or something like that. More generally these resources might thus be able to query the database when they are viewed and change their appearance after what is contained in the database. (And this way, the site can thus also implement my so-called ``wiki-idea'' from the 21--22 notes.) 
%%
%%\ldots And yeah, all that jazz about `user groups' to distribute trust, and about the user-driven filter algorithms, about rating tags, and about so on and so forth, all that is written about in my 21--22 note collection, I don't want to try to repeat these things now.\,:) %(14:29)
%%
%%%(07.11.22, 10:32):
%%\ldots\ Oh, and I of course also need to mention an important point, and that is: Sure the idea could work as an organization, but if we think about e.g.\ YouTube and Twitch, the commercial part is a big part of what makes those sites work. And by using my ``Economically Sustainable'' (ES) companies instead, the Web 2.0--3.0 site would still be able to give big rewards to the users who create popular content. And on that topic, if I were in charge of such a company, I would try to bring the average user on as soon as possible, given them some vote and some say in how the creators should get paid (how much in total, perhaps, and more importantly: how it should be distributed). I thus envision an e-democracy using the system described above for all the users, where their decisions in this e-democracy will be heard by the company, at least if it is reasonable (and in fact, the company could also be a part of the e-democracy, giving it self a significant weight on its vote, which would then make it easier for the company to keep to its promise of listening to that e-democracy). And of course, since it is an ES company, this e-democracy will be more a representation of the true power over the company, not just power that is ``lend out'' to the users, as long as their decisions does not stray to much from the company's wishes. *(This last sentence does not seem to make a whole lot of sense, but I guess I just needed to point out that if the company is what I am currently calling an 'SRC,' the users will also eventually get that power \emph{within} the company..)
%
%
%%*(26.11.22, 9:01) I mentioned movies at some point above as an example of a category of resources. That does not mean that the site then has to contain all the movies themselves; the resources can simply be reference-type resources (such as kind of movie ID, etc.). And since users should be able to control how resources in different categories are generally viewed, meaning that they can add HTML-wrappers to the resources, they can thus make it so that all movie refernce resources are viewed with potential links to site where they can be viewed or what not (the HTML code can fetch anything that is desired from the database (and/or from the web)). 
%
%
%*(20.12.22) I should also mention, that I imagine that all sentences/relations (formerly known as triplets) should be signed by a private key of the user, which is publicly associated with the (or \emph{is} the) user ID. This way you don't have to trust the particular server when it comes to who uploaded what and when. 






















\end{document}