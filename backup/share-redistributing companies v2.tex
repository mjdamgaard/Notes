\documentclass{article}
\usepackage[utf8]{inputenc}


%\usepackage{amsmath}

%\usepackage[toc, page]{appendix}
%\usepackage[nottoc, numbib]{tocbibind}
\usepackage[nottoc]{tocbibind}
%\usepackage[bookmarks=true]{hyperref}
\usepackage[numbered]{bookmark}

\hypersetup{
	colorlinks	= true,
	urlcolor	= blue,
	linkcolor	= black,
	citecolor	= black
}


\title{
%	Share-redistributing companies
	A new type of company in which shares are slowly and continuously redistributed to the customers
	\author{Mads J.\ Damgaard%
		%\footnote{
		%	See https://www.github.com/mjdamgaard/notes for potential updates, additional points, and other work.
		%}
		%\footnote{
		%	B.Sc.\ at the Niels Bohr Institute, University of Copenhagen.
		%	B.Sc.\ at the Department of Computer Science, University of Copenhagen.
		%	E-mail: fxn318@alumni.ku.dk.
		%	GitHub folder: https://www.github.com/mjdamgaard/notes.
		%}
	}
}

\usepackage[margin=1.8in]{geometry}

\begin{document}
\maketitle

\section{Introduction}
{\centering\noindent
	\vspace{-\baselineskip}
	\hspace{-0.7em}
	{\hspace{-4.em}$|$\hspace{\linewidth}\hspace{8em}$|$}
}
This paper introduces an idea for a new type of company in which the ownership shares of the company are continuously redistributed to the customer base. The aim is thus that the customers will end up being the owners of the company after a time. The idea is supposed to provide an alternative type of company that functions much like a limited liability company (LLC) initially, but then slowly turns into something similar to a consumer cooperative over time as the ownership shifts hands. 
%
This is assuming that the customers of the company are consumers rather than other companies. We will stick to this assumption in the following text and then get back to the question of how to define an SRC that has other companies as its customers in Section \ref{sect_B2B}. 

Part of the idea is that the amount of shares that each customer receives should be proportional to the money they have spend on the company's goods and services. We will argue below that the company should then be able to raise their prices accordingly, such that the customers pay a fair price for the shares that they get along with the goods and services. The company can then give this money to its shareholders as a dividend and hereby reimburse them fairly for the shares that they have parted with. 
%
Another way of describing the company is thus as one in which the costumers slowly buy out the initial shareholders over time.
%Hm, jeg kom lige til at tænke på: Hvad hvis nu aktiverne er langt større end, hvad kunderne til sammen kan købe på et liv..? (Er det realistisk..?) Hm.. Det skal jeg faktisk lige tænke over.. (11:33, 12.01.23) ...Jamen der må man jo bare sige, at så kan man håbe på, at folk vil være villige til at spare aktier nok op til, at de kan give en god portion af dem videre i arv til deres børn (eller tilsvarende arvtagere). Og derudover kan man så også bare sige, at idéen jo virker allerbedst for de brancher for S/P er høj, nemlig således at det ikke behøver at tage vildt langt tid for kunderne at købe de indledende aktionærer ud. 

For the remainder of this text, we will be referring to this new type of company as a `share-redistributing company,' or an `SRC' for brevity. 



%While it might sound as if the concept of an SRC is similar to simply putting a tax on all the shareholders, we will argue a thesis below that an SRC will generally be able to reimburse the shareholders for the shares that the part with in terms of an added dividend. We will thus argue that investing in SRCs might be almost exactly similar to investing in an LLC, %from the investors point of view, with the only difference that the shareholders of an SRC is required to essentially sell a small portion of their shares to the company at frequent intervals. 




\section{Motivation}

%At der er mange der ønsker mere demokratisk erhverv. Det gør nemlig, at virksomhederne ikke er evige skyldnere til ejerne, men at de kan bruge de penge i stedet til gvn for kunderne og/eller arbejdere. Ydermere kan en større udbredelse af demokratisk erhverv føre til, at virksomheder bliver drevet mindre af pengegrådighed, og mere bare af kundernes og/eller arbejdernes interesser i virksomheden. Har kan man så nævne nogle forskellige punkter, bl.a. også omkring miljø og klima. (Og er dette fint nok for den indledende motivation? Så kan jeg jo altid tilføje ting senere, hvis det er..? ..Nå nej, så skal jeg jo i det mindste også fortsætte og sige:) Men cooperativer kan naturligvis være sværere at finde investorer til.. Hm.. (..Søger lige på, om der er gængse måder/metoder til, at folk kan investere i kooperativer, og i så fald hvordan... (12:43)) ... (17:56) Okay, det har åbentbart taget mig en krig (gik dog også lige en god tur tidligere), men jeg har fået læst nogle gode artikler om kooperativer: To artikler af A. Lingane og A. McShiras fra project-equity.org/about-us/publications/coop-investment og en artikel af M. Lund fra resources.uwcc.wisc.edu/Finance/Cooperative Equity and Ownership.pdf. Og nu kan jeg heldigvis mærke, at det har givet mig noget much-needed blod på tanden igen ift. denne idé. Jeg når nu nok ikke meget mere i dag (hvis jeg kender mig selv ret her om aftenen), men nu glæder jeg mig faktisk til at få udarbejdet denne artikel. Og jeg kan nu se, at jeg virkeligt så skal huske at fiksere på, at min idé gør, at man kan konvertere firmaer til næsten-kooperativer på en måde, hvor firmaet virkeligt kommer til at køre i høj grad som almindelige aktieselskaber. Særligt kan de indledende investorer købe og sælge aktier frit helt som i et normalt aktieselskab, hvilket altså bare er super faktisk, for de fleste aktionærer tænker nemlig mest i de her day trading-agtige baner, hvor det handler om at købe og sælge (og tænker nok ikke helt så meget på dividenden/afkastet)..! Så det skal jeg altså bare virkeligt sørge for at lægge god vægt på. (18:04)
%(18:12) Og jeg skal altså så fortsætte og sige, at kooperativer kan være sværere at finde investorer til. Kooperativer er nemlig meget anderledes at investere i end aktieselskaber... indtil nu!... 

%Jeg bør måske specifikt motivere en virksomhed, hvor denne "køber" aktier fra aktionærer og videregiver dem til kunderne, og så kan jeg så slutte af med at sige, at SRC essentielt set gør dette (måske bare på en mere elegant måde...)..


In conventional companies, the set of owners is often much different from the set of consumers that buy the goods and services that the company helps produce, and the owners' stake in the company is often predominantly to make money. The stakeholders of a company are not just the owners, however. The workers and the consumers are also stakeholders as well. In theory, one could claim that if a worker or a consumer is dissatisfied with a decision of the owners, they can simply move to another company instead, but in practice this is rarely that easy. In reality, workers and consumers are often very much influenced by the decisions of the companies that they deal with. 

..We therefore often times see companies which follows a very consumer-friendly direction in their early stages as the grow, but then when their growth stagnated, they change directions to become more exploitative of their customers who has grown somewhat reliant on their goods and services in the meantime.. 

Worker cooperatives, consumer cooperatives and multi-stakeholder cooperatives seek to change that dynamic by having either the workers, the consumers or a mix of both being the owners of the companies as well. Furthermore, when the workers and/or the consumers are the owners of companies, it means that the profits of the companies are quite widely distributed. Such cooperatives are therefore believed to not contribute to an increased wealth inequality in the relevant society, as opposed to the more conventional types of companies such as LLCs (see Wright \cite{Wright}). 

As Lingane and McShiras \cite{Lingane and McShiras} %\cite{Lingane and McShiras 1}--\cite{Lingane and McShiras 2} 
points out, however, cooperatives often have a hard time attracting investors. %Since the fundamental idea behind cooperatives, which can be boiled down to: ``one member, one vote,'' is 
One of the big issues that the point to is simply the unfamiliarity of what to expect when investing in cooperatives for people who are used to investing in LLCs. 

This is why SRCs might provide a useful option in between a cooperative, specifically a consumer cooperative, and an LLC. We will thus argue below that being an investor in an SRC will be almost exactly like being an investor in an LLC in the initial phase of the company, but over time, the consumers will take over an increasing portion of the ownership, making the company function much like a consumer cooperative in the end. 




%\section{The core concept} %of an SRC}
\section{A company that slowly and continuously redistributes ownership}
\label{sect_def}

%*(Jeg skal skrive denne sektion om, så jeg også inkluderer muligheden, at virksomheden bare er tvunget til at vedlægge aktier til varene sådan at en vis kvote bliver overholdt over en vis udvidet salgsperiode, lad os sige et kvartal..)

%*(Hm, måske skal jeg også bare gøre det klart her, at vi også kan tale om virksomheder med andre virksomheder som kunder, og måske skulle jeg så gå over til kun at skrive consumers, hvor jeg før tit har skrevet customers.. *Eller også skal jeg bare inkludere den version af idéen, hvor virksomheder også kan være en del af kunderne..% ...Hm, det her er faktisk lige et ret vigtigt spørgsmål, som jeg bliver nødt til at finde ud af, men derfor skal jeg lige tænke mig lidt om: Vil det kunne være fordelagtigt med SRC'er mere som i min gamle version, hvor virksomheder også bare kan være kunder..? Eller er det altid bedre at se på forbrugerne i sidste ende, som i min nyere version her..? ..Hm, det første kan jo muligvis være nemmere at få op at køre i visse tilfælde.. ..Hm, jeg kunne måske også bare tilføje i introen, at "THis is at least true if the customers are the consumers and not other businesses. In Section whatevr, we will discuss the options for when.. B2B or a mix.. (But for now, let us consider businesses whose customers are.. people..).." ..Ja, det må være det, jeg gør. ...Gjort for nu.
%)

The core concept of an SRC is a company where all shareholders are continuously required to give a tiny portion of their shares back to the company at frequent intervals, which is then itself required to distribute these shares out among all its recent customers. The amount of shares that a shareholder owes the company after each interval should be proportional to the amount of shares that they own, and the amount of shares that each customer receives should be proportional to the money that they have spent on the company's goods and services in that interval. 

These rules are permanent and thus continue to apply for all shareholders, including those who have obtained their shares via this redistribution process. In other words, when the customers receive shares in this redistribution process, this now makes them shareholders as well, meaning that they will also owe a small amount of these shares after each interval. 
This means that if the customer base of the company shifts at some point, the new customers will then simply become the new recipients of the redistributed shares.

To give an example, let us suppose that an SRC has chosen the interval between when each share redistribution occurs to be a week, and let us also suppose that the amount of shares that each shareholders owes after each interval is chosen to be $0.1\, \%$ of the amount of shares that they own. A shareholder who currently owns 1 million shares will then have to part with roughly 1\,k shares each week, whereas another shareholder who owns 2 million shares will have to part with 2\,k shares. And if for instance the combined sales in that week totals 1 million dollars and one customer has bought goods and services for a sum of 100 dollars, that customer will then receive $0.01\,\%$ of all the shares that are redistributed at the end of that week. 
If for instance the total number of shares of the company is $10^{12}=1$ trillion, that customer would then receive $10^{12} \times 0.1\,\% \times 0.01\,\% = 10^{5}=100$\,k shares along with the goods and services bought for 100 dollars. 

This example assumes that the SRC redistributes all the shares received from the shareholders right away at the end of an interval. This would mean that customers will not know exactly how many shares they get per dollar spent. However, if the intervals are relatively short, they and the SRC would generally still be able to predict this amount with good precision. 

Alternatively, the SRC can implement a system where it is allowed to keep a small buffer of shares that are waiting to be redistributed. The SRC then has to be obligated to continuously adjust the exact number of shares that the customers get per dollar spent such that the buffer is kept approximately constant. If the sales then suddenly goes up unexpectedly, the buffer is then intended to give the company time to adjust the number of shares per price down such that the customers get fewer shares per dollar spent, which means that the buffer can grow back to its intended size. Note that such an adjustment still means that the customers will pay the same price for each share that they receive, but they will simply receive fewer shares per good or service that they buy. With this system, they customers will thus know the exact amount of shares that they get/buy along with each good and service, rather than simply knowing the approximate amount. 

%This requirement means that if the customer base of the company shifts significantly at some point, the targets of the redistribution process will also shift accordingly, namely such that the new customer base will be the one that slowly acquires ownership of the company, not the old one. *(rewrite)





\section{Ensuring that the customers hold on to the redistributed shares}


Since the aim of an SRC is that customers should end up being the majority owners, it is natural to limit how the redistributed shares can be sold by the customer-shareholders. Otherwise any third party with enough money could come and buy up a majority of the shares, perhaps with the intent of then changing the company direction to a more exploitative one.

A way to prevent this is to first of all divide the company shares up into classes. *(rewrite) Let us thus take the initial shares of the SRC to be the Class A shares of the company, and take the Class B shares to be the shares that have undergone the redistribution process (once or several times). This means that more and more Class A shares will be converted into Class B shares as time progresses. Both of these classes of shares should include voting rights, but the SRC can now put limits on how the Class B shares can be sold by the customer-shareholders. 

There are different options when it comes to limiting the sales of the Class B shares, but the solution that we will propose in this paper, is to rule that the Class B shares can generally only be sold to other customer-shareholders, and only as long as these shareholders do not obtain more Class B shares than a certain factor above what they would have had if they had 
%hold on to all the class B shares that they have received from the redistribution process and never traded any of these Class B shares away. 
only ever received and parted with Class B shares through the SRC's redistribution process and thus never traded Class B shares with other customer-shareholders. 

This factor of how much the customer-shareholders are able to increase their stock of Class B shares can potentially be set as quite large, perhaps as large as 5 or 10, even, without running the risk of allowing a third party to come in and buy up a majority of the Class B shares. *(Rewrite. The factor depends on the largest expected customer.\,.) For in order to do so, that party would have to make up $51\,\% / x$ of the demand of the company's goods and services, where $x$ is the factor in question, in order to be able to buy a majority of the Class B shares.

The exception to this rule could be to allow Class B share owners to sell these shares freely in an initial period of the SRC. This way it can be ensured that there are large enough amount of potential buyers when the rules goes into effect so that the customer-shareholders can still get a fair price if they want to sell their Class B shares right away.

And of course, the customer-shareholders should also be able to give their Class B shares away as inheritance to their heirs at the end of life. 

With these restrictions on the Class B shares, the company is ensured to be increasingly owned by the customers, but it still allows the customer-shareholders to have an internal market for the shares. This first of all means that any person in need of money have the ability to sell their shares and probably get a fair price for them, just as if they were shareholders of a more conventional company such as an LLC. And perhaps more importantly, customer-shareholders at an old age who want to use part of their fortune as their pension can do the same. 

However, in order to be absolutely sure that the customer-shareholders have the freedom to sell their Class B shares whenever they want and get a reasonable price for them, an SRC might also implement another class of shares, call them Class C shares, which are similar to the Class A shares in that they can be traded freely, but which are non-voting shares. So it for some reason happens that there will not be enough potential buyers of Class B shares at some point to give the sellers fair prices, these can then convert them and try to sell them (freely) as Class C (non-voting) shares. Since many investors do not care much about the voting rights that follow with a share, it is reasonable to expect that the Class C shares will sell for almost as much as the Class B shares (in the case when there are also plenty enough buyers of the latter). 


%\section*{navi.}


%Another option, which could also be combined with the previous option, is to allow customers to trade B type shares to some extent, but to then tax all sales of such shares. This alone would not prevent the customers from selling their B type shares to other parties, but it would give them an incentive not to do so.

%If this last option is combined with the second one, it might help ensure that there are always enough potential buyers... Hm.. %..(15:21)
%... Hm, jeg tror faktisk, jeg vil droppe at nævne det med skatten, for jeg synes faktisk alligevel ikke særligt godt om den mulighed.. ...(17:21).. ..Ja, lad mig bare droppe at nævne det for nu..




\section{Why investing in an SRC should not be much different from investing in an LLC in the initial phase}
\label{sect_thesis}

If system described in the previous section works to ensure that the customer-shareholders can always sell their shares and get a fair price, that should mean that the customers will also be willing to pay a fair price for the shares that they get along with the goods and services in the first place. For in that case, the customers can simply treat the shares that they get/buy from the SRC in this process as part of their pension plan. 

This thesis of course has to be tested empirically before we can claim it with any more certainty, but it is certainly not unreasonable to expect that it will hold true.

And if it does hold true, it means that the initial shareholders can be fairly reimbursed by the SRC for the shares that they part with in the redistribution process. This means that the value of the shares should be equal to that of the shares had the company been an LLC instead (assuming that the customer interest is the same). 

And just like in an LLC, the Class A shares contain voting right and can be traded freely. The only difference between being a Class A shareholders in an SRC and a shareholder in an LLC should thus be that in the SRC, the company automatically buys a small portion of the shareholders shares at frequent intervals. It is therefore reasonable to expect that investors familiar with investing in LLC will also be interested in investing in SRCs. 



\section{The time it takes for the customer base to take over the ownership}

The time it will take for the customers to buy out the previous owners will depend on the portion of shares that the shareholders owe after each interval, which will again depend the how much money the SRC generally wants its customers to pay for the attached shares compared to the actual price of the goods and services that they are buying, as well as on the price-to-sales ratio (P/S) of the company. Suppose for instance that a certain SRC predicts that its customers would prefer to pay about a $15\, \%$ increased price for each good and service due to the attached shares. Suppose also that the price-to-sales ratio (P/S) is at 2 and that the interval before each redistribution is chosen to be half a month. The company then wants to redistribute shares to the customers at each interval worth about $15\, \% / 2 / 24 \approx 0.3\, \%$ of the company's total price (P).
%\footnote{
%	This means that company then chooses that the shareholders owe $0.1\, \%$ of their shares after each interval, and will continue to do so regardless of whether the price of the company changes. 
%	If this price increases, the prices on all goods and services can then go up slightly, and vice versa.
%} 
Since
\[
	(1 - 0.003)^{24 t} = \frac{1}{2} 
	\;\Rightarrow\; 
	\log 0.997 \times 24 t = -\log 2 
	\;\Rightarrow\; 
	t = \frac{-\log 2}{24 \log 0.997} \approx 9.6,
\]
it would then in this case take about 10 years for the customers to obtain half of the shares of the company. (And it would take them about 20 years to obtain three quarters, and so on.)

The fact that these numbers are not unrealistic is a big motivator for considering the prospects of SRCs when compared to consumer cooperatives, especially in industries with relatively small price-to-sales ratios. We have argued above that SRCs might better at attracting investments in the initial phase. And the fact that SRCs can potentially have quite short periods before the customers will have bought a majority of the shares in the company, it means that it will potentially not take long for the companies to start functioning more and more like consumer cooperatives.




\section{B2B...}
\label{sect_B2B}



%\section{A hypothetical future where SRCs have become very prevalent..}


%\section{Additional points...} %Perhaps about having many shares...


\section{Comparing concept of SRCs to other ideas}

%CuSOPs, being an individual investor, investment funds..


\ 

\newpage
%\section*{navi.}

%When requiring the shareholders to frequently give out part of their shares to the customer base, it sounds similar to putting a tax on the shareholders, and one might think that this would then reduce the value of these shares. We will now give an argument for why this should not be the case. 
%
%
%The reason for this lies in the fact that the number of shares given to each customer is proportional to the amount of money that the customer has paid to the company in the relevant interval. Assuming that the customers are able to follow the sales numbers and calculate how many shares they will get from any dollar spent at the SRC's goods and services with reasonable precision, one might very well expect that they will be willing to pay that much extra for the goods and services. This is certainly true if the company had no sales restrictions on the B type shares as well. In that case, the customers will certainly be able to sell the shares that they get along with the trades whenever they want.
%
%If we include sales restrictions on the B type shares, however, the model gets a bit more complicated. Specifically, if we look at the type of SRC that allows no sales of the B type shares, it is reasonable to expect that this would lower the value of these shares in the eyes of the customers since the assets then would not be liquid. 
%
%On the other hand, if we look at the types of SRCs as described in the previous section which aim to ensure that the customers can always sell their B type shares to a large number of potential buyers, it then reasonable to again expect that the value of the shares will see no such drop in the eyes of the customers.
%
%If this is true, it means that investing in an SRC can look almost the same for investors as investing in more conventional companies, specifically such as LLCs. If customers are generally willing to pay the extra price for goods and services that corresponds with the value of the shares that are then redistributed to them, the previous shareholders will be able to reimbursed for the shares that they part with, and the shares will thus not lose any value when compared to the shares of an LLC. Furthermore, the initial investors in an SRC would be able to trade the A type share completely freely, just like they are when it comes to the shares of an LLC. 
%The only difference would thus be that the initial shareholders in an SRC would continuously have to part with a small portion of their shares at frequent intervals in return for an increased dividend that reimburses them for these losses.




%\section*{navi.}



%\section{Possible counterpoints}
%Now, if we try to search for counterpoints to the thesis of the previous section, we have to consider if and how it could happen that there a number of B type shareholders would want to sell their shares, but not be able to find enough buyers to yield them a fair price. 
%..Hm, men det skal jo være en vedvarende situation, hvis det skal kunne propagere tilbage til "nutiden".. Med andre ord, medmindre der er et systematisk forhold, der gør det forventeligt, at der ikke vil være nok købere.. Hm, og (uden at færdiggøre sidste sætning, for vi ved, hvad jeg taler om) det ville kun ske, hvis restriktionerne sammen med et forhold, hvor det kun er nogle få, der har lyst til at købe, og mange har lyst til at sælge.. Hm, så jeg behøver faktisk ikke tænke så meget på.. vent, og kunne man egentligt forresten ikke bare gøre sådan, at man godt kan hæve faktoren.. tja, og dog.. Lad mig lige tænke.. ..(Skulle til at sige: behøver faktisk ikke tænke så meget på tilfælde, hvor få personer pludselig kommer i pengemangel, btw..) ...Måske skulle man overveje at have en pensionsalder, hvor kundrne ikke længere behøver at "købe"/købe aktier med, når de køber varer og servicer.. ..Ja.. ..Lad mig lige tænke noget mere over et hele, og prøve at finde ud af, hvad jeg så kan sige klart, men muligvis ret vigtig tilføjelse (har overvejet det i glimt førhen, men jeg har aldrig taget tanken vildt seriøs før nu)..
%(12.01.23) Okay, i går blev en ret afslappet dag, men jeg kom til gengæld på noge super vigtige tanker. For det første var der muligheden for at undsige ældre fra at købe aktier, men nu har jeg fået en bedre idé, mener jeg. Hvis folk ikke kan sælge deres B-aktier til en fair pris, jamen så skal de altid bare kunne konvertere og sælge dem som C-aktier, hvilket så er ligesom A-aktierne, bare uden voting rights hørrende til sig. Nu vil jeg så skrive det hele om, også hvor jeg faktisk ændrer kernekonceptet til bare at være 'en virksomhed, som skal købe dens aktier tilbage fra investorer kontinuert, lidt ad gangen, og så skal disse aktier fordeles til kunderne.' Det vil jeg gå i gang med nu.






%\section{SRCs with other companies as customers...}


%\section{Other potential types of SRCs..}

%\section{Additional points...} %Perhaps about having many shares...
%%\label{sect_additional}






\ 

\newpage



\section{Introduction} \label{MSE_intro}

{\itshape
	
	Capitalism, in its current form, is the perfect system from the average consumer's point of view.
	
	\vspace{0.5em}
	
	\ldots right? 
	
	\vspace{0.5em}
	
	\noindent The competition between companies to fulfill the needs of their costumers means that all companies are forced to behave completely in line with the interests of those costumers.
	
	\vspace{0.5em}
	
	\ldots right?
	
	\vspace{0.5em}
	
	\noindent Together the consumers hold all the power in a capitalistic society. They are free to decide who gets to stay in business and who does not, free to enforce whatever desirable changes to a company that will benefit its costumer base at large.
	
	\vspace{0.5em}
	
	\ldots are they not?
	
}


\vspace{0.8em}

\noindent Well, this might be the case in theory, at least.  And indeed, if there is a lot of healthy competition within a certain industry, the companies should be required put their costumers first to a large degree.
But in reality, we do see many instances in our present societies where companies decide to take actions that negatively affect their customer bases at large, prioritizing the interests of their shareholders instead. 


The following are some overall examples of how companies might prioritize their shareholders over their costumers: 
\begin{enumerate}
	\item Companies might let their prices be decided much more by what the customers are willing to pay for certain goods or services rather than by what their production, maintenance and development costs are.
	(They might also inflate their prices in cases where the exact prices are calculated and disclosed only after the services have been given.)

	\item They might pollute the environment near their production or cause greenhouse gas emissions, perhaps degrading the life quality of their average customer. 
	
	\item They might move their production away from where their products are generally being sold, taking away jobs and job opportunities from their average customers.
	
	\item They might offer bad salaries and/or bad working conditions for their workers, which might degrade the jobs and job opportunities of their average customers.

	\item They might try to avoid taxes to pay their shareholders greater returns, even when that tax could have otherwise gone to benefit a large portion of their customers.

	\item They might choose to go for a reduced quality of some goods or services in order to cut costs and save money for their shareholders, for instance by choosing a reduced durability of some products, which can often be hard to spot as a customer when buying them. %(They could also put up additional paywalls or force users to watch more comercials when it comes to digital products and services, but this can also be seen as part of the first point above.)
	
	\item They might halt investments into developing new products or features on the grounds that the shareholders do not stand to gain enough from this investment, even when these developments could benefit their customers significantly. 

	\item They might keep secrets about their business operations, their technology, and/or their development projects, or they might patent/copyright their technology and keep the licenses to themselves. These are of course completely natural things to do in our current capitalistic societies, but from the average consumer's point of view, it might be much better if the companies would commit themselves to more open source sharing of ideas.
\end{enumerate}
%
%It might therefore be beneficial for the consumers to start exerting their power over companies in order to make them prioritize the interests of their costumers to a greater degree. 


And although capitalism has been an incredible force for driving technology forward, and has given us much prosperity in the world, it also seems to lead to an increasing inequality in our societies in its current form. This is a natural consequence when most companies are ultimately run by their owners' desire to make their money multiply. 
It means that the yearly earnings per stock price have to be significant in order to have engaged directors of the businesses in our societies. %to make these function well,  
%since being an active investor takes up personal time that could have otherwise been spent enjoying one's current wealth. 
But when the yearly earnings are significant, it unfortunately implies that the rich generally get richer, %and as a consequence (unless society at the time undergoes rapid technological development), the poor gets poorer.
meaning that the inequality of the relevant societies generally increases. 

Additionally, since wealth and power goes hand in hand, rising inequality also creates a feedback loop where it gets increasingly easy for the wealthy and powerful to cheat the average consumer to benefit their own interests, which again gives rise to more inequality. 
So even in countries with relatively low inequality and good working conditions, it might pay off for the consumers to do something sooner rather than later, as things will probably only get more difficult as time progresses.

%
%But what can they do? Well, the average consumers in a society actually hold an enormous amount of power as a group, when you think about it. They could therefore in theory easily extort shares and concessions from companies gradually over time if they could only band together in order to make demands from certain companies, and then threaten to boycott these if they do not give in. 
%In theory, they could also even band together to invest in certain companies and then subsequently start demanding goods and services exclusively from those companies, making their own stock increase in value and thereby gradually conquering more and more assets and wealth in a society.
%But this is first of all a difficult thing to do; 
%banding together and agreeing to limit their own consumption in a certain way.
%And even if they succeeded, it would still be a very costly process for all parties involved, including the companies. 
%%namely since boycotting a certain brand of products will be a big inconvenience for many consumers. And it would of course also be costly for the operation of the companies, which is not desirable either, not for any party. 


%Luckily, there is another solution that seems to be much easier to implement. 
Luckily, there seems to be a solution to all these problems, and even one that is relatively easy to implement. 
It involves a new type of company where the shares are slowly but surely redistributed over time so that they end up on the hands of the customers. 
The idea thus aims to solve the problems mentioned above 
by making sure that the customers of the companies actually \emph{become} the shareholders over time. %, which will of course ensure that the interests of the shareholders will start to align with those of the customers. 
And when the consumers start owning more and more of the industries in which they are costumers, the relevant companies can then simply be run by the engagement due to the owners' interests as customers rather than their interests in making their invested money multiply. 


%And unlike many attempt to solve these problems before it, this solution does not require any political change! In fact, it functions fully \emph{within} capitalism as we know it. It is thus simply a suggestion of a change in how we \emph{utilize} capitalism.





\section[A new type of company]{A new type of company where the customers slowly become the shareholders over time}


This idea for solving the current problems with capitalism revolves around a new type of company, which will be referred to as a `share-redistributing company' in the remainder of this text, or an `SRC' for brevity.
An SRC is then defined as a company where all shareholders are continuously required to give a tiny portion of their shares back to the company at frequent intervals, which is then itself required to distribute these shares out among all its recent customers. The amount of shares that a shareholder owes the company after each interval should be proportional to the amount of shares that they own, and the amount of shares that each customer receives should be proportional to the money that they have spent on the company's goods and services in that interval. These rules are permanent and thus continue to apply for all shareholders, including those who have obtained their shares via this redistribution process, i.e.\ as customers.

At the same time as shares are redistributed in this process, each shareholder is also paid a return proportional their amount of shares. How much is paid in returns after each interval is a company decision, which means that each shareholder ultimately has a vote in the matter proportional to the amount of shares that they own. 
In practice though, the returns of an SRC will first of all generally depend on the average surplus of the company, just like a more traditional company. But in an SRC, the returns will also generally include an additional amount of money equal to the market value of all the shares that are distributed to the customers.
The simple reason why this should be the case is that, within reasonable parameters, one can assume that the customers will be willing to pay an additional price for each good and service equal to the market price of the shares that they get along with them. And while the exact amount of shares that each trade will yield the customer is only calculated at the end of the current interval, each customer will still know roughly how many shares they will get from each trade, assuming that the sales numbers are public and also frequently updated. 
So while the shareholders are in principle bound to ``give'' some of their shares to the company, which is then bound to ``give out'' these to the customers, the company can still be expected to get a fair amount money in return for these shares, which will then generally be paid back to the shareholders as part of the returns, thus reimbursing them for the shares that they have parted with. 


It almost goes without saying that an SRC ought to make sure that its shares are very divisible, such that even if the company grows a lot, the atomic units of the shares will still only have a value significantly below e.g.\ a cent, preferably. 
This way, even a customer who only spends a small amount of money at an SRC will still get a fair amount of shares along with it, instead of these being rounded down to zero.
%(And investors who spend a small amount of money will still have to give them back at a slow, exponentially decreasing rate.)


%The only potential difference, given the assumption that the market does not predict a difference in the customer interest, should be due to the fact that the customers will at some point have bought out a majority of the shareholders, which means that they might take a less exploitative stance towards their customers at that point (which is of course the whole point of this new type of company). But this might take some time, and until then, the shareholders are free to run the company in a completely traditional way. 


Since the point of the idea is to ensure that the customers buy out the initial shareholders over time, it is also important that the company implements further restrictions on the shares, limiting who can buy the shares to some extent. Otherwise, any interested party could simply buy up all the shares from the customers and perhaps make the company more exploitative once again, thus gaining a profit at the expense of the customers (who might have become somewhat reliant on the company's goods and services). 

A solution to this problem could be for any new SRC to first of all declare a certain date and then restrict all sales of shares that have undergone the redistribution process described above after said date such that these can only be sold to other registered customers. 
%Since all the company's customers will already have to register as customers in order for the company to be able to send them their shares, the company will have to keep account of its customers anyway, which makes this solution all the more easy to implement. 
%Furthermore, since any party who wants to buy the company out from the customer base can just buy a small number of goods or services to become costumers, the solution should therefore also limit how many shares each customer is able to buy. 
Furthermore, each customer should also be limited in how many of these shares (redistributed after said date) they can buy, depending on how much money they have spent as customers at the SRC. 
Otherwise, any party who wanted to buy the company out from under the customer base could simply just buy a small amount of goods or services to become costumers themselves.
 
As an example, the SRC could choose to make a rule that no customer is allowed buy shares that have undergone redistribution after the relevant date if they will then end up with more than double (e.g.) of what they would have had if they had never bought or sold any shares, but only received and held onto the shares they have gotten through the redistribution process as costumers. 

The reason why it might be a good idea to postpone this restriction until after a certain date is to ensure that there will be a large number of registered customers (who will have had to register to get their shares as customers) at the time when the rule starts going into effect, thus ensuring that there is a large number potential buyers for these shares right away. Should a customer want to sell their attached shares immediately after obtaining them, perhaps if the person is in need of money at that moment and cannot afford the extra investments (but still wants to support SRCs), they should then be still able to do so easily, and get a fair price for them, despite these sales restrictions.
 
%This solution ensures that no outside party can come in and buy the company out from the customers, and thus that after a certain point in time, the company will forever be governed by its customers. 

With these requirements of an `SRC,' such a company will then be destined to become increasingly owned by its customer base as time progresses, slowly perhaps, but surely. And even if its customer base changes rapidly at some point, the continuous redistribution process will in that case still mean that the new customer base will slowly but surely buy out the previous customer base as owners. 

The time it will take for the customers to buy out the previous owners will depend on the portion of shares that the shareholders owe after each interval, which will again depend the how much money the SRC generally wants its customers to pay for the attached shares compared to the actual price of the goods and services that they are buying, as well as on the price-to-sales ratio (P/S) of the company. Suppose for instance that a certain SRC predicts that its customers would prefer to pay about a $5\, \%$ increased price for each good and service due to the attached shares. Suppose also that the price-to-sales ratio (P/S) is at 2 and that the interval before each redistribution is chosen to be half a month. The company then wants to redistribute shares to the customers at each interval worth about $5\, \% / 2 / 24 \approx 0.1\, \%$ of the company's total price (P).\footnote{
	This means that company then chooses that the shareholders owe $0.1\, \%$ of their shares after each interval, and will continue to do so regardless of whether the price of the company changes. 
	If this price increases, the prices on all goods and services can then go up slightly, and vice versa.
} 
Since
\[
	(1 - 0.001)^{24 t} = \frac{1}{2} 
	\;\Rightarrow\; 
	\log 0.999 \times 24 t = -\log 2 
	\;\Rightarrow\; 
	t = \frac{-\log 2}{24 \log 0.999} \approx 29,
\]
it would then in this case take about thirty years for the customers to obtain half of the shares of the company. (And it would take them about sixty years to obtain three quarters, and so on.)

%\section{Comparing...}

%[...] Jeg skal så også tilføje (noget som jeg har tænkt på her for en time eller to siden), at hvis start-investorerne nu ikke er helt overbeviste om, at konverteringen virkeligt vil bringe den succes med sig, og på den anden side også er bekymret over det mulige fald i aktierne, fordi man nu får udsigt til at firmaet på et tidspunkt vil stoppe med at udnytte dets kunder efter bedste evne, jamen så kan de faktisk give sig selv den sikkerhed, at de gør konverteringen betinget af, at virksomheden vil vokse til en vis grad inden for en vis periode, før at konverteringen bliver permanentgjort. Og indtil da vil aktierne altså kun skulle omdeles på den omtalte måde i en vis periode. Men hvis virksomheden (og aktiernes værdi) så er vokset nok i denne periode, så vil konverteringen så gøres permanent, således at omtalte omfordeling så vil blive ved og ved med at ske (efter hvert "lille interval") derefter. Så herved kan de første aktionærer ligesom lægge hele risikoen over på kunderne (bortset fra at disse kunder så i øvrigt heller ikke rigtigt vil miste så meget, hvis det skulle slå fejl, og firmaet ikke når at vokse det, den burde i den periode). Okay. Og efter at jeg har skrevet om alt dette, så kan jeg så begynde at sammenligne med de mere basale metoder (nemlig at gruppere sig i kundeforeninger for at presse firmaerne til at overgive magt (og muligvis endda aktier) til kunderne), eller den endnu mere simple "løsning" (og som slet ikke vil du i praksis (hvorimod den idé, jeg lige har nævnt godt kan du, især hvis man får et godt e-demokrati-system (se nedenfor;) )), som bare er at gå sammen for at presse firmaerne til at sænke deres priser osv.). Og ja, så bør jeg så også lige tilknytte nogle flere overordnede betragtninger om, hvad konceptet handler om (omkring bl.a. at kunderne ligesom "investerer i sig selv," og i denne nye form for kapitalisme (hvor ting bliver besluttet mere demokratisk, bl.a.)), men så er det vist muligvis også bare det.:) 



\section{Additional points about these new companies}


All sectors of industry can be a focus of the movement. When it comes to SRCs in the primary sector, they just have to make sure that all their goods come with a code to redeem the shares that are owed to the customers, or something to that effect.\footnote{
	This is of course a crude solution. At a later stage in the future, all purchases from an SRC will probably be able to be registered right away in the stores. % where they take place.
} 
If certain businesses in the secondary sector then add a price to the final one that the private customer sees, this added price will then simply not count in terms of how many shares the customer gets from the company in question of the primary sector. But if these companies of the secondary sector are also SRCs, this added price will then of course go towards buying the customer shares in these companies as well. %(And this will also be the case regardless of whether the company of the primary sector is an SRC or not). 


An SRC can either have its beginning as a newly upstarted company or as a more traditional company that decides to convert to an SRC. A company that converts to an SRC will simply choose a date from which it will be obligated to follow the redistribution process described above (as well as an interval length, a percentage of shares owed after each interval, and also a later date after which the sales of the redistributed shares are restricted, as explained in the previous section). Once the new SRC enters its first interval, it can then raise its prices according to the expected added value to each of its goods and services due to the shares that now come along with them, such that it can pay back its shareholders for the shares that they part with immediately after this interval.

A functioning SRC will of course require a system where the customers can register themselves and their purchases. Since this will generally require some cost to develop, it might be beneficial for the movement initially to focus only on convincing large companies to convert to SRCs, such that they can easily afford this cost. They same thing can also be said about the cost of formulating and signing the contracts that goes into the conversion. And of course, convincing a company to convert might also take some effort from the interested consumers. Therefore, due to all these overhead costs, the movement should probably focus primarily on convincing large companies to convert at first. This also has the advantage that once the customer bases start to take over the power of these companies, it will generally requires less effort per costumer to govern them, as opposed to when the costumer bases of the companies are small. (For instance, if the company has a board of directors, it will require fewer directors per customer if the customer base is large.)



For a company whose directors sees that a large part of its customer base likely wants it to convert to an SRC, but who are not necessarily convinced that it will be beneficial for the company and its (initial) shareholders, the following approach might be beneficial: The company can undergo a temporary conversion to an SRC, which is then only made permanent if the company sees a certain amount of growth in the relevant period. This way the consumers who want the company to convert thus get a chance to prove that this conversion will be beneficial for the company, and if not, the redistribution process simply halts automatically once again, making the company revert back to its initial type. 
%(Apart from such temporary conversions that gets reverted, the conversions to SRCs are otherwise supposed to be permanent: once the redistribution process is initiated, the shareholders should not generally be able to stop it again (at least not for as long as it would take for the customers to become a great majority of the shareholders..).)

Apart from this, consumers can of course also try to convince companies to convert simply by threatening to boycott them if they do not. They can then go to the competitors of a certain company instead to put pressure on it, even if these competitors are not SRCs either. When one company finally gives into the pressure and converts, the consumers can then start supporting it once again. Additionally, if no company in an industry wants to convert to an SRC initially, the interested consumers might also simply support a startup company of the SRC type and then try to make it grow into a big competitor in the relevant industry.


%If the movement grows big, however, companies that have been slow to convert might run the risk of turning all the customers who support the movement against them, and especially if they wait long enough that these customers will have become shareholders the competing SRCs in the meantime. At that point, the hesitating company might risk that these customers stays opposed to the company, even after it finally decides to convert, since said costumers might then gain more from seeing the company reduced so that they can conquer its assets, namely as shareholders of a competing company. 

%"The fact that any company in principle can convert at all times also be beneficial for the consumers in another way if the movement really takes off in a big way. In the beginning, when the movement is relatively small yet, the question for each company will simply be whether to convert and join the movement or not. But if the movement really grows a lot and start taking over big parts of the various industries, the non-SRCs (or SRCs that lost out to other SRC-type competitors) might even consider trying to win back customers by converting to an SRC that has an accelerated rate at which the customers take over the power. This might be achieved by simply declaring that the shares obtained by the customers will yield an increased amount of voting power compared to the initial shares. So if the movement really gains a massive following, the customer bases might even be able to convince companies to accelerate the customer takeover thus in order to remain the favorites of the movement. And knowing that this accelerated takeover might potentially become a reality, this might boost the interest for consumers to join the movement even further."









 
%Some examples include big retail industries, banking, housing (when it comes to both construction, retail, and renting out housing), insurance, private hospitals and dentists, pharmaceutical companies, general web stores, streaming platforms, and social media sites as well as other Web 2.0 sites. 

%And when it comes to smaller industries, if they deal with hobbies or other spare time activities where the customer base in general is very interested and engaged in the industry to begin with, the movement might also be successful early on in such cases.

%Some primary industries such as the food industry, and so on, might also principle benefit by being run more democratically, but in these cases, it is probably much easier to focus simply on the retail industries to then gain indirectly power over the primary industries. For instance, if a certain part of the customer base wants some primary production to become friendlier towards the environment, and the customer base at large controls the relevant retail industry, these customers could instead just use their votes to make the retail companies prioritize such qualities in the primary industry more. *(Or maybe it is at least as good an idea to simply just go for primary industry companies that .\,.  hm.\,. \ldots I should instead just write, that this would require some effort by the retail stores as well, but that if such stores are converted as well, they will be willing to make this effort, and will of course even try to prioritize converted primary industry companies..)






%When trying to start up the movement in a certain industry, it is also relevant to look at the general price-to-sales ratio (P/S) in the industry. If this is low, it means that the time it will take the customers to take over will be shorter than in cases where the P/S is generally higher. And since it might be more attractive for customers to join the movement in a certain industry when the redistribution time is relatively short, this number is thus also worth considering when deciding where the movement should put its focus initially. 
%
%
%And on top of all this, the movement will of course want to focus as much as possible on the industries where it is most believed that having the customers own more assets and have more power can lead to a positive change in how the industry benefits the consumers of the society. 








\section{The future that this movement might bring}


Suppose that SRCs become widespread in the various industries of society. Then, when the customers start taking over the power to govern these companies, they will be able to start eliminating the problems listed in Section \ref{MSE_intro}. And when they do, this will perhaps also make it harder for their competitors not to make similar changes in order to keep up with them. So even in a case where SRCs do not take over an entire industry, they might still be able to change that industry for the better even beyond the portion of it that they control directly. 



To give some examples (that repeats some of the point listed in Section \ref{MSE_intro}), the customers of the now almost democratically governed SRCs within a given industry will be able to reduce the carbon footprint of that industry, as well as limit other types of pollution by it. 
They will be able to make sure that the industry offers good job opportunities for the customer base, with good salaries and good working conditions. 
They will be able to make the operations of the companies more transparent, and specifically make sure that they do not try to cut the quality of their products in secret to earn more money. (In terms of the environment and our natural resources, prioritizing quality and durability of products to a greater extent will also result in a better use of those resources.) They will also be able to make the technological development of the industry more open source, so to speak, since they will not need to fear as much that the ideas will be stolen and used by competitors. They can also make sure that the industry does not spare expenses when it comes to technological development whenever there is a good change that this will benefit the customer base. They might thus, of course depending on the relevant industry, be able to accelerate the technological development related to the industry, benefiting future generations of customers as well as themselves.



If SRCs end up controlling a big part of the various industries, it also means that the consumers will own more of the assets on the market. Since nobody can invest in an SRC when it is well beyond its initial stage except for the investment that the costumers give through their consumption (plus the limited amount of shares that they are able to buy on top of that), the general finance market will then have to relegate itself to dealing only with the non-SRC companies that are left. This means that there will then be fewer assets to buy in that market, from which it follows that the price-to-earnings ratio (P/E) will generally go up for the stocks. Wealthy individuals will thus still be able to keep all their money in theory, if they trade cleverly, but the yearly earnings on their stocks will generally decrease. This can of course be seen as bad news for these individuals, who will now have to actually \emph{use} their money to a greater extent (on consumption, i.e.) rather than trying to make them multiply. 
%(Although, being collectively forced as a group to focus more on their own life and happiness, and on using the wealth that they have acquired instead of trying to make it multiply, might actually turn out to even be a good thing overall for many. And since all other rich people are forced to do the same, there will be no loss of pride by it.) 
But for society in general, this should lead to less inequality. 

In terms of the P/E of the SRCs themselves, the customers can in principle also start reducing the earnings (thus increasing the P/E) when they take over the power. But on the other hand, when the customers starts taking over the shares of the companies, they also get a proportional share of those earnings. So even if the P/E is continuously kept constant for an SRC, the customer base will still eventually be liberated from having to pay the extra money that only goes into generating a surplus for somebody else, as they will instead start paying that money simply to themselves. 

A society where SRCs become very widespread and popular will thus tend towards a state where the wealth of a person is simply equal to the wealth that they have inherited (minus taxes on that, of course) and/or received through social programs, plus the wealth that they have acquired through their own work (not counting the ``work'' of owning stocks, i.e.) and minus the wealth that they have spent on consumption. And the power of each person over the various industries will simply be proportional to their consumption within each industry. The society will therefore tend towards a state with great equality and a very flat power distribution. And the businesses in society can still be run efficiently, only where the engagement to govern them comes more from the owners' interests as customers rather than from their desire to make their money multiply. 
%In fact, this might even accelerate our technological advancement as society due to the fact that it might lead to much more open source technological development. 














%
%
%%\section{Transparency and direct democracy}
%
%%Mention also some details about a healthy e-democracy, specifically that voters should be able to choose representatives, but of course have the power to change their mind and reallocate their votes at any time (and probably also that the 'areas of concern' are important..). 
%
%%An SRC should ideally strive to be as transparent as possible, and to bring its customers into the decision making process as much as possible early on. 
%
%\section{Encouraging customer engagement}
%
%Since the point of SRCs is to let the customers gain more and more power over time, it will probably be a good idea for an SRC to engage their customer base in the decision making processes early on, even when they do not yet have much actual power over the company, since this might boost the customer interest for the SRC all the more. 
%
%An SRC could thus aim to afford its customers with a an online application where the customers can express their desires, and perhaps even given the power to vote ... %hm.. ..Hm, eller kunne dette mon bare gøres til en kommentar eller to i den næste sektione..? ..Ja, det er da lige før (selvom så er det ikke sikkert, jeg "får plads" til at nævne det med, at et online demokrati er nice bl.a. fordi det også giver mulighed for et mere direkte demokrati.. (..hm, men det er måske også okay...)) ..(16:08) Ah ja, det er faktisk helt fint bare at inkludere det som en lille kommentar i næste sektion, og hvis bare jeg lige får nævnt det her med, at det måske kunne stemme på forslag(/propositioner) (i stedet for bare at kunne stemme på bestyrelse), så ligger det jo faktisk allerede i det, at jeg ligger op til et direkte (e-)demokrati.:) Nice.:) (16:10)




%\section{Online ...}
%
%%It is important that an SRC commits itself to providing an online environment where its customers/shareholders can log on
%
%Once the customer base start to take over the power in an SRC, ...
%(16.11.22, 10:35) Hm, jeg bliver ved med at være på vippen til, om jeg skal tilføje en sektion som denne eller ej... 

%%..Hm, lad mig prøve at skrive:
%\section{Online ...}
%
%When the customer base start to take over the power in an SRC, they can govern the company in various ways. One option is to use representative democracy similar to how consumer cooperatives are governed. But an alternative option, which might spark more customer interest, is to use an e-democracy, i.e.\ a direct democracy functioning via online applications. SRCs might thus commit themselves to providing an online community ...
%...Nej, jeg tror ikke, det rigtigt er det værd.. (11:26)





\section{The market potential of these new companies}


The main reason that SRCs might see an increased customer interest is first of all due to the future that it promises. And even for people who do not expect to live to see much of this change, these people will generally still care for the future of their children, grandchildren, and so on. 
%(But if this turns out to be a significant factor, the SRCs can always do the thing, mentioned somewhere else out here in the comments, where they make the votes on the customer shares count for more.)

If the customer interest increases due to a conversion to an SRC, the initial shareholders are expected to see their shares increase in value, just the way that they would for a regular company. And while the shareholders are of course bound to give out a portion of their shares at frequent intervals in an SRC, they will generally be able to be reimbursed fairly for these shares, as explained above. Therefore, if the costumer interest goes up significantly after the conversion to an SRC (when compared to the overhead costs of the conversion), the initial shareholders will thus stand to gain from it.

From the customers' point of view, these will generally have to pay a small extra amount of money each time they buy goods or services from an SRC. But this money will, on the other hand, not be lost to them at all since it will buy them some shares in the company. If the company remains stable, the costumers will get the same money back that they have invested, but if it increases or decreases in value, they will of course either gain or lose some money. 
Customers who believe that SRCs will generally grow and become prevalent in the future can thus in fact simply see the extra expenses as solid investment to secure part of their pension. And if SRCs become widespread over all the various industries of society, these investments will most likely become spread out over many companies for the consumers who support the movement, which means that their resulting financial portfolio will have a low risk. It is thus reasonable to believe that the consumers will not generally be dissuaded much from joining the movement due to these small investments that a continuous support for the movement will require. 

%Consumers who support the movement will likely want it to grow as big as possible in a short period of time since this will secure the sustainable economic future as desired. But on the other hand, if the movement instead simply grows slowly but steadily, and the SRCs thus grows slowly with it, it will at least mean that these supporting consumers will gain some money in return from their support. For if the SRCs grow slowly enough that some of this growth happens after customers have taken over some of the shares of the company, they will %..Hm, lad mig lige tænke lidt mere over dette; og om det er værd at nævne.. (13:34, 22.11.22)
%%... (15:11) Hm, lige før, at jeg pt. bare mangler at / bør nævne punkterne omkring "positive feedback" og "if movement becomes big, SRCs might compete by increasing voting power of custoemers" (og hvor nr. 2 punkt så måske burde nævnes i forrige sektion), og at det så bare skal være det, udover at jeg selvfølgelig lige skal omskrive og renskrive lidt.. ... Jeg kunne måske også tilføje "increasing voting power"-paragrafen til "additional points"-sektionen.. (16:45) ...Ja, det tror jeg, jeg gør. Og så kan jeg bare indlede paragrafen med en pointe om, at SRC'er kan gøre sig endnu mere attraktive over for kunderne, hvis... Og så kan jeg så pointere, at dette kan blive relevant, hvis bevægelsen bliver stor nok.:) Nice nok. 





Furthermore, SRCs are likely to get a lot of free advertisement from their customers, especially once these start to become financially invested in the companies as well. This financial interest will then be added on top of the initial interests (e.g.\ due to the future promises of the movement) and make the customers even more likely to recommend and advertise the SRCs to their friends and family. This might thus very well create somewhat of a positive feedback loop where more and more people will become interested in the movement.

As the customers of SRCs becomes increasingly invested in them as time progresses, they will probably also be less likely to buy goods and services from competitors. The customer bases of SRCs are thus likely to become more and more loyal to these companies over time.
And even though a non-SRC competitor can in principle convert at any time (in general), any such competitor who waits too long to convert might not be able to attract the same support from the customers at that point. 
For when these potential customers already own part of the shares in a competing SRC, they might simply prefer to conquer the latecomer's assets by having the first SRC defeat it in the industry as a competitor, rather than joining the new SRC as customers and obtaining the same assets slowly through the newly initiated redistribution process. 
If this will turn out to be the case, namely that the customers will stay loyal to the first SRCs in the industries even after their competitors finally choose to convert as well, the initial shareholders of the first SRCs might thus stand to gain a lot of wealth from their early commitment to the movement. 


%The only drawbacks of an SRC thus seems to be the fact that will require some overhead costs to get the new system going, and then the fact that the company is now destined to become owned by the %..Nå nej, for kunderne kan jo netop sagtens beholde en konstant P/E, så den pointe virker ingen gang rigtigt.. ..Tja, men der er jo dog stadigvæk de andre punkter fra Sektion 1, ud over bare punkt 1.. ..Hm, men det er nu stadig næsten ikke værd at nævne.. ..Især ikke fordi det alligevel hurtigt bliver countered af "The main reason that SRCs might see ...".. ..Ja, jeg tror det er for lidt at sige til at det giver mening at nævne --- og jeg har jo allerede nævnt overhead costs'ne, så lad mig bare undlade at nævne det igen her.. ..Nå, nu har jeg lige omskrevet anden paragraf ovenfor, så nu er det faktisk nævnt.:)





%[...] SRCs might compete by increasing voting power of customers, [...]




%\ 
%
%\ 
%
%
%The drawbacks of SRCs are of course first of all that it takes some extra work to make sure that the customers can register their purchases and redeem their shares, and, more importantly, that the customers on average will have to be able to afford the extra cost of buying the shares attached to the goods and services. But despite these drawbacks, SRCs might have a huge potential for growth due to the likelihood of a greatly increased customer interest. 
%The main reason that customers might prefer SRCs to a great extent is of course because of the future society it promises. And on an individual economic level, the future prospect of breaking free from a state of feeling under the thumb of various exploitative industries as an individual consumer might of course greatly increase that consumer's interest in SRCs.

%Additionally, the first generations of customers might even stand to make economic gains from supporting SRCs. The money that these individuals spend on the attached shares of SRCs are not lost to them: They get shares in return for all this money, which has the potential to increase in value if the relevant SRCs grow afterwards. If the movement for supporting SRCs thus grows gradually (and not just explodes right from the get-go), the value of the SRCs will also grow gradually, meaning that the customers will get more money in return than what they have invested. And while the amount of money that a single consumer has invested into a single SRC in general might be limited, if SRCs become spread out over a wide range of industries, the consumers of these first generations will have the opportunity to invest a lot of money into the overall concept/movement via their consumption. 

%When the customers of SRCs thus also becomes economically invested in the concept, even if the average customer is only invested by a little amount compared to their combined consumption, this will likely still have a big overall effect of accelerating the interest for the SRCs. When the customers of SRCs become more and more invested in the companies, it will likely create a positive feedback loop since each customer will then be economically incentivised (on top of being incentivized due to the societal future prospects) to get others, friends and relatives, to join the movement as well. And even though the general economic incentives might be relatively small in many cases, and that not every customer will therefore try to persuade others to join for that reason, many customers likely will, which will accelerate the customer interest in SRCs.

%Furthermore, when the customer bases start to take over the power in the SRCs, the expected changes that will happen, i.e.\ the company becoming less exploitative, might also mean that the competitors of the SRC will have to make similar changes to keep up. Since this might cause a decreased earnings-to-sales ratio, it might cause investors to take their money elsewhere. If this happens, this will likely reduce the price of the shares of the competitors, thus enhancing the possibility for the SRC to grow and take over more assets in the relevant industry. 

%In terms of getting SRCs started in the various industries, there are two options to go for. Either someone starts up a new company as an SRC, hoping to attract an increased amount of investment and customer interest to then grow to a big player in that industry. Or a existing company might simply convert to an SRC. 
%The first option might have the potential to generate more interest from the first generations of customers if it succeeds since having a small SRC to begin with means that the company thus might have more potential for growth. But the fact that any company can in principle convert to an SRC at any time (unless of course it has signed some contracts that is in conflict with this conversion) is also a big advantage for the movement, since it means that consumers might be able to convince their favorite companies to convert, continuing with their successful policies and experienced work force in exactly the same way as they were before the conversion. (Of course, if the customers are \emph{completely} satisfied with the current company, they might not be as interested in supporting an SRC conversion, except perhaps if they want to make sure that the company sticks to its good ways and do not become corrupted in the future.)


%At the early stage of the movement, it might happen that part of the customer base of a company tries to convince it to convert to an SRC, but the company's current directors are not convinced that this will be beneficial for the company. But in such cases, the company then has the option to make a temporary conversion instead, such that it only becomes permanent if the market price of the company really grows by a certain amount before a deadline. If the interested customers then does not convince enough other customers to join and the company thus does not grow enough before the deadline, the company will simply revert back to its previous form after that deadline, meaning that it is no longer required to continuously redistribute its shares to its customers. But if the customers does manage to get enough others to join in supporting the company, it will continue as an SRC from that point on. This option thus has the potential to help convince companies to convert to SRCs. 


%The fact that any company in principle can convert at all times also be beneficial for the consumers in another way if the movement really takes off in a big way. In the beginning, when the movement is relatively small yet, the question for each company will simply be whether to convert and join the movement or not. But if the movement really grows a lot and start taking over big parts of the various industries, the non-SRCs (or SRCs that lost out to other SRC-type competitors) might even consider trying to win back customers by converting to an SRC that has an accelerated rate at which the customers take over the power. This might be achieved by simply declaring that the shares obtained by the customers will yield an increased amount of voting power compared to the initial shares. So if the movement really gains a massive following, the customer bases might even be able to convince companies to accelerate the customer takeover thus in order to remain the favorites of the movement. And knowing that this accelerated takeover might potentially become a reality, this might boost the interest for consumers to join the movement even further.


%\textit{I would maybe like to make another paragraph here where I talk a bit about an online community for the movement...}

















%\ 
%
%\ 
%
%%Potential (big) interest due to the future prospects, ...
%
%As mentioned, the market value of the company might in principle be slightly decreased due to the future prospect that the company will no longer be able to take an exploitative stance towards its costumers at some point. Additionally, the market value also has to account for the fact that the customers are now required to invest some money in the company. Given that there are always enough potential buyers of the shares, the customers that hold on to their shares will not loose any money on this, unless of course the market value changes (and for the worse), but it still means that they cannot invest that money into other things such as buying housing to eliminate rent, which for many would be the first natural choice for ``investment'' (i.e.\ in themselves). %(..And speaking of (renting out) housing, this might by the way be a business where a company of this new type might be really successful.. **) 
%So investing into into this new type of company, and the future that it promises, might therefore thus essentially require a sacrifice for the first generation of customers, especially if the company is not projected to grow much from there.
%A company that considers converting to this new type --- or a startup company that want to start out as such a company --- will thus have to consider these potential factors when predicting the worth of the company. 
%
%But these minuses might very well be countered by the increased customer interest that such a company might receive. Since the spread and growth (\textit{again: Let me just focus on the semantics (i.e.\ the meaning) for now, and then rewrite the sentences later on.\,!})
%of these new companies might very well bring about a better future --- and, equally important, prevent a boring and sad one --- one would naturally except a much increased customer interest, thus increasing the worth of the companies by a lot.
%
%Furthermore, it might very well turn out that there will be a positive feedback loop of customer interest as more and more customers join and essentially invest in the companies as customers. 
%Presumably, these customers will believe (at least somewhat) in the bright future that this consumer/business movement can bring about, and they would want others to join in and be part of the movement. 
%And since it is part of the idea that these customers will each be financially invested in the companies by a small amount, this will probably also increase their will to get others to join in. Even if each customer is only invested by a small amount, this fact might still have a big effect due the sheer number of customers, %.\,.\,if we are talking about a big company.\,. %(17:17) Hm, holder lige en pause og ser om jeg får friske øjne, og ellers må jeg jo bare lige næsten-færdiggøre (bare lige til husbehov) de resterende sektioner her i morgen (minus "technical additions"-afsnittet, som jeg tænker at lade være for denne omgang).. 
%at least when it comes to the big businesses such as those mentioned in Section \ref{MSE_good_company_types}. 
%
%Chapter \ref{E_democracies} below, describes an online application for a so-called e-democracy, where people can engage very efficiently in democratic discussions and decision-making processes. If such an application becomes a reality, or if the customers find another type of good internet forum where they can discuss and decide how the company should be run, this would also increase customer interest all the more since it would make the customers feel able to engage themselves and be a big part of the movement. And given that the companies of this new type will often want to expand as much as possible, and swallow up competitors, it might be a very exciting project to take part in planning. 


%And of course, as was already mentioned **(Is no longer mentioned; mention it here first.), since the companies of this new type also might make it hard for competitors to keep up without loosing interests from their investors, this might also be a reason why such companies have the potential to expand and take over more and more of the business from their more traditional competitors. 


%The fact that the interest in these new companies might be spread a lot due to the ..engagement.. actions.. commitment.. of the first generation of customers, this might actually mean that these can make their own shares grow this way. This way, the first generation of customers might actually gain a reward for being part of spreading the interest% 
%--- making up for ...

%So a start-up company of this new type, or even a traditional company that converts to this new type, might potentially become very successful. The initial owners of such a company, whether being a new upstart or a converted company, might therefore stand to gain a lot. And even the first generation of customers, who took the risk of investing in the company rather than investing that money elsewhere, might stand to gain money in the end, if the company will have grown a lot since they started being customers (which is only fair; that these customers, who was a big part of helping the company grow, also sees their fair share in return for their investments). 
%
%
%\textit{I also have to mention here how the initial shareholders of converting companies might give themselves a security by making it part of the conversion contracts, that the company has to grow a certain amount during a period after the conversion, for this conversion to become permanent; otherwise the company will simply revert back to a traditional one. *Hm, but do I mention that here, or does it belong better somewhere else.\,.\,?}
%
%
%
%%(13:59):
%\ldots\ \textit{Hm, I think I might actually just end the chapter here for now.\,.\,! I can then just add the points that I miss in following sections, that I have now out-commented, at in another version. Maybe I'll even leave out this part about converting companies giving themselves the mentioned security. I think I might then just rewrite the final parentheses about costumers' investments being fairly treated as such and just end the chapter like that.\,.}

%\section{Comparing the idea with some other possible solutions..}
%
%
%(\textit{Here I might then also point out some general things about the idea, such as the customers investing in themselves, and some more things... *I should for example also explian some more what it will mean that (at some point) the shareholders will only (pretty much) be able to get their shares through consumption..})
%
%\section{Some possible technical additions to the idea..}
%
%(\textit{Keeping a clear definition about what is the goods and services, veto right, automatic splitting of companies are some examples of what I could mention..}
%
%
%
%\section{Final remarks}
%
%\textit{Here I will include whatever remarks I don't get to at the end of the ``Comparing \ldots'' section, and particularly I can mention something about the fairness of this new system (i.e.\ that customers' investments are actually also treated as such in this type of company), compared to the system of traditional companies. 
%I should probably also mention something about the excitement of running an company that tries to expand and expand e-democratically.. perhaps.. (done)
%}







\section{Comparing the idea to other solutions}

It is first of all relevant to compare the idea of SRCs to the existing concept of consumer cooperatives, which are companies governed democratically and owned equally by all its members. This is similar to what SRCs are supposed to achieve over time, with the only difference that in a hypothetical society where almost all companies are SRCs, the distribution of the power of each consumer will naturally depend on the distribution of their consumption, whereas it would simply depend on where they choose to apply for memberships if these companies were all consumer cooperatives instead.
However, the advantage of SRCs over consumer cooperatives is that they do not require the consumers to have/get ownership over the company to begin with. 
Instead, the consumers can in principle start out with zero capital when it comes to an SRC. %, and then just slowly gain more and more of the capital in the company as they automatically use part of their consumption to buy shares when they buy goods and services from it. 
And since the initial shareholders thus generally do not have to part with any wealth when a company is converted to an SRC, it should thus be much easier for consumers to persuade the companies, old ones or new ones, to become SRCs, as opposed to becoming consumer cooperatives. Furthermore, SRCs might also generally reward financial growth more than consumer cooperatives since only those who were shareholders during a certain period of growth will see a correlated increase in their wealth, and not the consumers who only joined the party after/in the upswing, so to speak. 


Comparing the idea to another potential solution, one could also ask: ``If the concept of SRCs is just that consumers gradually buy the shares of the companies at which they are customers, why can they not instead just agree to collectively start putting some money aside each month and invest them in the companies they care about?'' But this is first of all not as efficient in the long run since some of the initial shareholders might then decide to hold on to their shares and thus raise the price of the company's shares artificially. The consumers would thus have to pay for more than what they get in this case. And furthermore, this would also take a great deal more organization and commitment from the consumers, as opposed to when they just need to buy more goods and services from SRCs from time to time rather than from the more traditional types of companies. 

%
%%As mentioned in Section \ref{MSE_intro}, t
%The consumers could also, at least in theory, organize themselves to invest in certain companies and then afterwards help these grow by prioritizing them as customers, namely with the goal of conquering more and more assets and wealth in the given society. However, this idea would of course also require a great deal of organization and commitment, and unless these consumers can find a way of disguising where they invest, the initial shareholders would then also be able to see what was going on beforehand, which would mean that they could simply raise the price of the shares before the relevant consumers change their habits as customers. 
%

And lastly, one could also compare the idea of supporting SRCs with the idea of socialism/communism. The critiques of the latter are, however, rather well-known, for instance due to the centralized administration of the society that such a system requires, combined with the corrupting effects that such a centralized and powerful administration is generally exposed to. 
A society where most companies are SRCs, on the other hand, works completely within the market system, and will thus still have the same relatively decentralized power structure (on a fundamental level) as before. And the fact that the movement requires no governmental changes to a capitalistic society, but only requires consumers to start prioritizing a certain type of company over others, also makes it a great deal easier to get it going. 




\section{Conclusion}

Share-redistributing companies, combined with a consumer movement supporting them, have the potential to bring about a future of great equality in our societies, and one where companies work more efficiently in the favor of the consumers. And since they function completely within the market system, they thus generally require no changes to the laws of our current governments. 

Who knew that capitalism might not necessarily lead to an increasing inequality in a society? We might have simply been utilizing it wrong all along.


%(Even if this movement does not gain momentum right away, I am still quite sure it will change the game in terms of how we use capitalism in the long run..)











\begin{thebibliography}{20}
	
	
	\bibitem{Wright}
	E.\ O.\ Wright, 
	\textit{ How to Be an Anticapitalist in the Twenty-First Century} 
	(Verso, La Vergne, 2019).
	
	
	
	
\end{thebibliography}



















%
%
%
%\chapter{E-democracies} \label{E_democracies}
%
%The concept of a so-called `e-democracy' is not a new one. Wikipedia thus has (in the moment of writing) a whole article about the overall concept that one can read. (That article, in its current form, defines the concept perhaps a bit more abstractly than what we need for our purposes here, but it might still be helpful to glance at.) In this section, I will therefore not introduce the overall concept, but simply give some short notes on how one might implement such an e-democracy, which can for instance be used to govern a company like the ones described in the previous chapter (as its shareholders), or a political party, etc. 
%
%
%\section{A basic digital application where voters can build proposition graphs}
%
%Imagine a digital application where all voters in a given democracy (concerning e.g.\ a company, a union, a political party, etc.) can log on and build a proposition graph together, which can then define the policies of the body governed by the given democracy. We are here talking about the `graphs' of mathematical graph theory. (One can make a brief search the internet for `graph theory' to see what this is about, and one might then also want to search for `directed graphs' and `connected graphs' at the same time.) 
%
%Each node of the graph holds a proposition, which is simply expressed in plain text of whatever natural language (such as English) is appropriate for the case. 
%
%When adding a new node to the graph, one can add it by itself (i.e.\ not connected to any other nodes) or add it with at least one of two kinds of (directed) edges to an existing node. The two types of edges then represents whether the node's proposition is an elaboration on the parent node, or if it is a self-contained proposition that should, however, only apply conditioned on the parent node being active.
%
%A node becomes active if it has enough votes and if a majority of those votes are positive rather than negative. Whether `votes' are counted simply by number (such that all voters have equal power) or if the votes are weighted (meaning that some voters have more power than others) of course depends on the case. 
%
%The point of being able to `elaborate' the proposition nodes rather than having to replace it with a more detailed note instead is simply to make the work easier for everyone, and also to make the graphs easier to read. It means that the policies can be defined somewhat loosely at first (and therefore much more quickly and easily), and whenever some vagueness of the propositions is discovered subsequently, either by people studying them or because of a relevant case that reveals it, the voters can then work to specify and mend the propositions. 
%
%The point of being able to add proposition nodes that are conditional on their parent nodes being active is of course some propositions might only be beneficial to implement given that certain other ones are already in place. If a somewhat fundamental proposition node is voted inactive again, it is thus convenient that such `conditional child nodes' follow along. If the parent node is then voted active once again (or perhaps for the first time) at a later time, all the child notes that has retained a positive voting score in the meantime will then become active one again, as well as any child node whose score has become positive in that time. 
%
%The application might also allow these `conditional child notes' to have several parent nodes for convenience. 
%And the same could also apply for the `elaboration child nodes' since there might be case where it could be beneficial to be able to elaborate the interpretation when two propositions nodes are active at the same time, for instance if these two proposition have a slight conflict with each other, or if the create some other issue that needs to be handled when they are both active together.
%
%`Elaboration child nodes' should of course also depend on their parents being active. The difference between a `conditional child node' and an `elaboration child node' is therefore essentially only in the interpretation: The propositions of `conditional child notes' and those of their parents are meant to be independent of each other as statements, whereas `elaboration child nodes' are free to correct and override parts of the statements contained in their parent nodes, thus allowing these to not necessarily be absolutely precise and self-contained. 
%
%Every user should be able to add new proposition nodes and each node should also have a separate `interest score' that users can rate (with the same weights on the votes as for the first score in the case where these are weighted). A proposition node whose `interest score' exceeds a certain threshold becomes visible to all users in the main graph, and people will then have to give their votes to it, if they want to influence whether it is applied or not. 
%
%Users should thus also be able to view nodes in the graph that has not yet exceeded said threshold, perhaps by being able to select various ranges of interest scores to look at. It might also be beneficial to let such nodes expire after a certain time if their `interest score' has been low enough for too long. 
%
%Users should also preferably have their own `workbench' with enough storage capacity to hold a number of propositions nodes. If a proposition node expires, they can thus make sure that the work is not lost as long as they keep said proposition on their own `workbench.' It would probably be beneficial also if users could then have shared `workbenches' as well, where they can collaborate on making new proposition nodes. 
%
%Anonymity is of course generally very important for democracies. So it is naturally very important that no one can see which user has added what nodes, unless of course they have collaborated on it from the same `workbench.' Users should also not (for most cases of democracies) be able to see which users has voted for what. 
%%
%For cases with weighted voting, either with very few voters or with very precise weights, this might be helped further by making sure that the exact voting scores are not visible to the users, and that the can thus only see a number that is rounded to a less precise floating point number. One might also implement intervals such that new votes are always declared together in groups, some time after they have been cast individually. 
%
%
%\section{How the proposition graphs are used to govern a body}
%
%The point of building these proposition graphs is then that the leadership of the body you are governing should to some extent be required to follow the active propositions, at least within some basic limitations of they can be required to do. 
%
%When the proposition graph changes, they leadership should be required to implement these, but here it might of course be a good idea to implement some delays on when new changes are supposed to be carried out. One might thus rule that a change should only be implemented after a certain period from when happened, and only if that change has remained active in the proposition graph during that period. 
%
%If the proposition graph gets some contradictions and/or ambiguities that makes it hard for the leadership to know what to do, they should also be allowed to postpone implementing the relevant changes until the voters sorts out the issues (by which they make some new changes which restarts the acceptance process). 
%
%How to make sure that the leadership follows the democratic decisions of the proposition graph? Well, by making sure that the voters also have enough direct power over the governed body to enforce their will. This might typically be ensured by the group of voters having the power to fire leaderships and/or decrease or increase salaries, thus giving this group ``sticks'' and potentially ``carrots'' that they can use to make sure that the hired leadership does what it is supposed to.
%
%
%It has to be mentioned that a high level of transparency is an all-important part of an effective e-democracy when it comes to the body that is being governed. Luckily, one can say that as long as the voters have the aforementioned ``carrots'' and ``sticks'' at their disposal, they should at least be able to make the body more and more transparent, even if it not very much so from the beginning.
%
%
%
%%Husk:
%	% Fortolkningspolitikker (inkl. hvad man gør, hvis der er modstride) og delays. (tjek)
%	% The point with having conditional propositions.. (tjek)
%
%
%\section{A more advanced application}
%
%A basic system like the one described above is good enough for very simple cases where it is okay to just have a majority rule. But for more complex cases where there are a lot of groupings of voters with different interests when it comes to various topics, to have such a majority rule is not really sufficient. If we for instance think of the policies of a whole country, this is a good example of such a complex case, where most people probably have \emph{some} special interests that are only shared with a fraction of the society. In such a system, it is important for people to be able to \emph{negotiate} with their voting power in order to get what they want, not just to always vote for exactly what they want as individuals. One group might thus want to meet with another to make a deal where they say: ``If you vote for this and this, even though you might not be particularly interested in that, we will vote for this and this which you do have a particular interest in (even though we might not).'' 
%
%In order to accommodate these realistic needs of its users, the digital application in question should therefore also first of all make it possible for users to form groups in the system. On a technical note, having such groups can of course be implemented in a lot of ways, but I can suggest an implementation where the creators of a group start out with some divisible `moderation tokens' that give them power to decide who can join the group and who gets kicked out, and where they are then free to transfer parts of (or all of) these tokens to other users within the group at any time. This moderation system is open enough that the users can implement any other moderation system on top of this, if only they trust some central party (which can then control a user profile in the system) to enforce the results of this external moderation system.
%
%And in order for such groups to be able to start negotiating with their voting power, it is then first of all important that some overall statistics (perhaps where numbers are rounded to ones with less precision for the sake of anonymity) of how a group votes on average is made public at all times. Otherwise a group who has made a deal with another group would not be able to check that this other group holds its promises. 
%
%With this addition to the application, groups can now in principle make all the deals in private that they want to. But of course, a good implementation of an `e-democracy application' would also afford its users with ways to make these deals within the digital application, online. 
%A way to do this might be to add what we could call `conditional votes' to the system. A `conditional vote' is then a vote on a proposition, whose sign can depend on other factors. In particular, a conditional vote should be able to depend on parameters regarding the voting statistics of groups. A group that want to make a deal with another one can then decide to make a conditional vote for something the other group wants and then make the condition such that this latter group has to vote for something the former group wants to unlock the conditional votes. 
%
%On another technical note: Depending on how the system is implemented, such conditional votes might be able to cause deadlocks, where two or more conditional votes all wait for the other to turn the other way in order to turn themselves. But a way to mitigate this is to give a direction to all conditional votes which denotes the sign expected from a successful deal. The system can then continuously refresh the proposition graph by turning all conditional votes in their positive direction and then see if they fall back to the same state or if they settle on a new state, which will mean that some deadlock has been conquered. 
%
%And on a design-related note: The conditional votes can be visualized/rendered as leaves in the graphs, each one attached to a certain proposition node. The users can then create and add these `conditional vote nodes' to the system in the same way as proposition nodes are added, and then all users can decide to cast their vote for the given proposition either by casting it (unconditionally) on the proposition itself or casting it instead on one of the conditional vote nodes (meaning that their vote will now be automatically conditioned on some parameters of the voting statistics in the system, continuously, until they change their vote again). 
%
%%Another thing that a more advanced system might account for, is the fact that the power of the voters might not just have different weights but might also be dependent the area that a given proposition deals with. This could for instance be in a company or a government where there are many departments/ministries in charge of different areas. If such a company/government decided to go for a more democratic leadership, it might still want to keep some division of power within the democracy. This example is perhaps not the most realistic one so here is one that is more so:  
%
%
%Another very important thing that an advanced application should afford its users is to make sure that the voters can choose representatives. It might seem odd to want to implement a direct democracy only for people to end up choosing representatives once again, but is indeed exactly what a direct democracy should aim for. It is nowhere near feasible if the system requires all users to engage in all discussions and decision making in order for the democracy to work, not unless we have a simple case with relatively few voters who are all quite engaged. But in most cases that one could think of, being able to choose representatives and trust these with looking into specific and/or complicated matters and vote on the person's behalf is all-important. The problem with representative democracies that a direct democracy aims to fix should thus not be to get rid of representatives, but simply to ensure that people can change these much more rapidly should they want to, and also that any voter can always choose to look into specific matters themselves and choose to vote differently on those than how their representative has voted.
%
%An advanced e-democracy application should therefore also allow users to choose representatives. With `conditional votes' implemented, users can of course in principle just cast conditional votes on all propositions that they want representatives to decide for them, but this is too cumbersome and we can do much better than that. The application could thus first of all allow the voters to give their votes to others. But it is likely that some users will only trust a certain representative to decide for them in a certain area of concern. And in general, users will therefore probably want to be able to have multiple representatives at a time, each responsible for making decisions for the voters in specific areas. 
%
%I therefore propose that an advanced e-democracy application also implement what we could simply call `areas of concern,' which are then essentially groupings of propositions regarding a certain subject. Whenever a new proposition is added by itself (as what we could think of as the ``root'' of a connected graph) it should thus be given an area of concern such that the application can group it with proposition graphs with the same area of concern. And whenever a child node is added, it should of course get the same area of concern as its parent. With this implemented, users should then be able to give their votes to another user (i.e.\ representatives) when it comes to any specific area, which should then effectively mean that the user will automatically cast the same vote as their representative, at least when it comes to propositions that the user has not voted on themselves. (And one might then implement different settings to this, such that a user for instance might be able to even let a representative override the user's own previous votes.) 
%%(If someone creates an otherwise relevant proposition node but adds the wrong area of concern, one can expect that it will then simply not be voted forth (over the aforementioned threshold), not until the author gives it the right area of concern.)
%
%It now almost goes without saying that each `group' in the system can then potentially choose to have their own specific representatives that the members are recommended (or perhaps required in some special instances) to use. 
%
%The system might also implement `subareas,' such that any user can try to add one such to any proposition node. Users should then be able to vote such subareas in and out, and if one is voted in, the proposition node and all its children will then get this extra area, that users can then also choose to sign representatives to. With these `subareas' implemented, this then allows users to delegate different representatives to these, even if they are also part of the same overall area of concern. **(This paragraph will probably need some rephrasing.)
%%(One could also implement `subareas' simply by requiring that these are also added from the beginning when the relevant proposition nodes are created, but it might make it easier for the users if they can just change the subareas by vote at any later time (instead of having to recreate and substitute the whole subgraph, with similar nodes but with updated subareas).)
%
%
%%These `areas of concern' also allows for something else that might be very useful, namely that the same community of users/voters can govern a variety of bodies with the same overall (potentially disconnected) proposition graph. 
%%Because when the contracts and/or agreements concerning the various bodies' commitment to follow the proposition graph are external to the digital system, a body might as well agree to only be ruled ...
%%The usefulness in this, apart from maybe having everything gathered in one place, is that this will mean that voters
%%(16:06, 06.11.22) Hm, dette kan nok godt blive mere kompliceret, fordi flere foreninger så skal blive enige om, hvordan stemmerne fordeler sig, og så skal det lige pludseligt topstyres på en helt anden måde.. Så lad mig lige se en gang... ..Hm, men handler det så ikke bare om, at forskellige grupper skal kunne "genbruge" de samme propositionsgrafer, og også om at de andre gruppers aktivitet så også godt må kunne gøres synlig i samme propositionsgraf (altså for en vis gruppe, der bruger denne)..? (16:09) ...(16:29) Jo, men så har det så ikke rigtigt så meget med subareas at gøre.. ..Nej, for så skal man også simpelthen gøre, så at graferne.. er helt adskilte, ja, så måske giver det altså slet ikke mening.. hm, andet end at man stadigvæk kunne have graferne side om side, og mere vigtigt, at conditional votes så også kan komme til at afhænge af eksterne grafer.. Ja, er det ikke bare det..?:) 
%
%
%%Furthermore, each group should have its own page and/or own `area of concern,' where the members of that group have all the voting power. This is useful since it means that each group can then build their own proposition graph over its policies and opinions. A group might then also signal external actions via this proposition graph. For instance, a group the represents a workers union might create conditional votes in their own proposition graph that depends on some statistics regarding the main proposition graph.. 
%%say that: ``On these
%
%
%And lastly, it might be beneficial for various groups in society who govern different bodies, e.g.\ political parties, unions, organizations, companies, to also be able to negotiate with each other online and to view each other's policies. For instance, a company might want to say to a governing political party that: ``if you implement certain laws, we will move out company elsewhere.'' And a trade union might then say to a company: ``if you do not give us higher salaries, we will go on strike.'' These are thus examples where a group in society can use their power over one body (including simply themselves as a group) to negotiate concessions from another group with power over a different body. 
%So if the advanced e-democracy application really wants to afford its users with all that they could want for negotiating effectively with each others, it should allow different voter groups to come together in the same space. First of all, each `group' in a e-democracy should have their own proposition graph that only they have voting power over. This local proposition graph can then be used to signal the groups policies, opinions and potential actions. And when it comes to the `conditional votes' in this local proposition graph, these should also be allowed to depend on statistical parameters in the main proposition graph, outside of the local one. 
%And furthermore, different e-democracies (governing over different bodies) that uses this same digital application should also be able to invite another group to join together, such that the two e-democracies can have their proposition graphs shown side by side (but with the same distribution of voting power in each of these graphs), and more importantly, that one e-democracy can then start making conditional votes that depends on statistics regarding the other e-democracy and vice versa. 
%
%
%\section{An e-democracy party}
%
%There are of course a lot of different examples where an e-democracy such as this could be useful; political parties, companies, unions, organizations and other communities. %In this section, I will, however, only give some points when it comes to political parties and companies of the type that I described in the previous chapter.
%%If we ..
%And when it comes to political parties, there is the natural option that these are run only by their members. We could thus imagine two or more parties competing for power, each being run e-democratically by its members. But since politically parties are typically inclusive anyway, why not just strive to have one political party where every person in the society gets an equal vote? 
%
%I believe that such a party could gain massive support over time. It might start out as a small party, especially in the early days where people are still getting used to working with the proposition graphs, and when the technology is perhaps at an early stage. And then as the technology to matures and the userbase grows, more and more people would trust the new system enough that they would want to give their vote to this `e-democracy party.' That party might then, at least in multi-party systems, get some representatives in government and by that point, the interest in the party would grow further, since all registered users would then be able to have a say in the policies of those representatives. And if the technology works, more and more people would then see the potential in an e-democracy. This is especially true in countries where the people in general do not always feel heard by there politicians: When they then see that the resulting proposition graph for most people will fit their interest better than what the traditional political parties offer, they will want for the e-democracy party to be voted in as a ruling party. 
%
%Now, if the party thus lets member of society have an equal vote in it, this might then be problematic at this stage when the party want to take over from the traditional parties, since people might then be tempted to make their vote count twice, essentially, by voting for their favorite traditional party and then also using their vote in the e-democracy party. And since voting is anonymous, there is no real way that the e-democracy party stop this. That is, apart from taking steps to balance out this effect. The e-democracy party might thus choose to temporarily break its commitment to giving all people an equal vote in this phase, and instead promise that it will commit itself to try to counter all representatives in government that are not part of the e-democracy party by giving more votes internally to a group of representatives whom it deems are exactly at the mirrored end of the spectrum than the group of non-e-democracy representatives in government. (The chosen counter-group can, however, be much larger then the group of representatives it is supposed to counter.) This way, a voter who wants the e-democracy party to take over would not be tempted to cast their votes to a traditional party instead. It also means that once the party sits on a majority of the power in government, other representatives are more likely to join it while in power (if the relevant constitution permits such migrations of representatives while in power) if they see that the e-democracy is more practical since the e-democracy party can then just remove the appropriate amount of counterbalance as these former outsiders join. 
%
%
%E-democracies as governments of countries might thus be a much closer reality in the near future, than a lot of literature on the topic seems to suggest, at least in countries governed by a multi-party system. In two- or one-party systems, the development might of course be much slower. But then again, once some multi-party governments successfully switches to an e-democracy, the two- and one-party governments would then be able to analyze and copy the technology, at least giving them a much easier route to an e-democracy, should their voters want one. 
%
%
%
%\section{Anonymity}
%
%
%As mentioned, anonymity is often very important for a democracy, especially if we think about the case of governing a country. Therefore, the digital application should allow the users to vote anonymously. This can be achieved letting each user control an anonymous profile, but if information about which user has which anonymous profile is stored on a server, that server might be hacked. 
%
%So the question is, can an e-democracy system be as safe and anonymous as going to a box, drawing a cross in a field on a piece of paper and putting that paper into a box? Yes, actually: there are ways to ensure complete anonymity of the users where the anonymity is preserved even if the servers of the application is hacked.
%
%The following protocol allows a set of clients to each provide a server with a set of public keys such that each client knows the private key of exactly one of the keys in the set (and no one else but them knows this private key) but where no one knows which public key belongs to which client apart from the clients knowing their own key. The protocol is furthermore resistant to DoS attacks. 
%
%It works by having the clients take turns building blocks in a block chain, which we can think of as a `block spiral,' where the clients form a circle and where the turn to provide a new block to the spiral goes around in the circle. 
%
%\ldots\ \textit{Okay, jeg tror lige, jeg venter med at forklare om min idé her, for det kan godt være, at der findes en lidt nemmere måde. Det vil jeg lige tænke over. Men ellers er det en god idé, altså den hvor hver klient sender nogle nøgler videre til en tilfældig anden klient i kredsen (hvor hver blok krypteres med den næste klients offentlige nøgle (fra begyndelsen) og sendes til denne), og hvor klienter, der modtager nøgler gerne skal sende dem videre og slette dem fra hukommelsen. Herved vil man meget sjællendt kunne se, hvem var den oprindelige sender af en nøgle (medmindre både modtagerklienten og klienterne for og bag brugeren er ondsindede), og selv hvis den bliver sporet tilbage kan pågældende klient bare sige, at ``den nøgle kom fra en tidligere omgang og altså fra en helt anden bruger, men jeg har altså slettet data om, hvor den kom fra, som jeg burde.'' Men ja, jeg tænker nu lige lidt mere over det, inden jeg skriver denne sektion færdig. .\,.\,Jeg har i øvrigt også tænkt mig at sige, at man efter at have brugt denne protokol så bare kan bruge et VPN herfra, men hvis man vil være endnu mere sikker, så kan man endda bruge helt den samme protokol til at indsende data om, hvordan man vil stemme med sin profil, hvor man så altså bare erstatter de (tilfældigt) genererede nøgler i protokollen med tilfældigt genereret data samt det faktiske data, man vil indsende, og til sidst så offentliggør man så bare, hvilket skrald, man har sendt ind, men ikke den faktisk data, man så lader serveren beholde. (.\,.\,Så kan det dog godt være, at man skal ændre protokollen lidt, så man lige sørger for, at hver mængde data også vil nå slutningen af protokollen, så at ingen data-klumper bliver tabt i protokollen --- medmindre der altså er sket en synlig fejl i protokollen.)}
%
%\ldots\ \textit{Nej, der er vist en nemmere protokol, hvor man vist nok også kan finde frem til en DoS attacker. Man kan vist bare have et VPN, hvor klienterne sender beskeder frem og tilbage, og hvor de så kan pakke en nøgle ind i flere krypteringer med forskellige nøgler, hvor beskeden så skal sendes til alle de klienter i rækkefølge, som kan dekryptere beskeden en efter en. Og hvis så man gør det tilfældigt, hvor mange krypteringer, der skal til, så kan ingen igen vide, om en nøgle kom fra en person, bare fordi de får opsnappet, at beskeden på et tidspunkt blev sendt fra denne, for vedkommende kunne jo sagtens have fået den fra andre og så bare have sendt den videre. Og hvis man så har nogle få DoS'ere i netværket, så kan brugere der har sendt en nøgle der aldrig nåede frem jo pege på, hvem der kan have været de skyldige (af den række af brugere).\,. Hm, ja, men hvis man nu vil bevise det også, så kunne disse brugere.\,. Nå nej, man kan ikke bevise det på et VPN, men det gør vist heller ikke noget. For brugere skal jo stadig gerne sende flere nøgler pr.\ protokol, og hvor de så bare opsiger alle på nær én til sidst. Og hvis der så er en DoS'er i netværket, jamen da det ikke vil være fatalt, så må det være fint nok, at brugerne bare kan page dem ud nogenlunde. (Og hvis det så bliver et større problem, så kan man altid bare bruge den mere krævende blok-spiral-protokol, jeg har haft tænkt på.)} %(08.11.22, 10:27)
%
%
%
%
%
%
%
%%Hm, jeg har fået tænkt lidt over anonymitet, men det kan godt være, at jeg lige skal tænke lidt mere. Men jeg har altså fundet på nogen fine systemer til at skjule stemmeres identitet, og jeg tænker, at stemmere generelt skal kunne vælge enhver tredjepartsbruger til at videreformidle deres stemme anonymt. Sådanne kunne så med fordel få lov at give floating point værdier (i stedet for bare 0 eller 1) med deres stemmer, eller de kunne bare råde over et antal stemmer, således at de både kan give et antal positive og et antal negative stemmer til hver proposition (men jeg tænker at det første næsten er nemmest..). Og ja, så kunne én form for sådan en tredjepart så være en organisation med fysiske lokationer, hvor medlemmerne så kan møde op personligt og ændre deres stemmer og/eller repræsentanter, og hvorved organisationen kan opdatere deres stemmer herefter med en frekvans, der kan afhænge af, hvor mange ændrer deres stemmer ad gangen over en gennemsnitlig periode. Og en anden, meget smartere;), måde at have en videreformidlingsrepræsentant på, kunne så også være.. hm, lad mig lige se.. (13:50) ...(14:30) (ordner også vask) Jo, man kan også have en videreformidlingsrepresentant, der fungerer via mindst to tredjeparter, som klienten selv kan vælge. Først er der en trejdpart, eller instans bør vi nok hellere kalde det, bare.. som via asymmetrisk krypering får en nøgle fra hver bruger, som kun denne instans og hver enkelt relevant bruger må kende. ..Ja, eller på nær at de også så skal sende alle disse nøgler til en anden instans, der heller ikke må offentliggøre dem, og som så i øvrigt ikke ved hvor hver enkelt nøgle stammer fra (og må ikke få dette af vide af første instans). ..Hm, vent, giver dette mening..? ..Ah, jo, jeg kan få det til at give mening, men lad mig nu lige se.. (14:36) ..Hm jo, denne instans nr. 2 kan så også få en offentlig nøgle med fra brugeren til hver enkelt nøgle af første instans, sådan at denne altså bare får et sæt af nøgle par, hvor den ene er en offentlig nøgle. Denne instans kan så kryptere.. Hov, nej, så behøver vi faktisk ikke den første nøgle; instans nr. 1 sender altså bare et sæt af offentlige nøgler videre (gennem en krypteret kanal) til instans nr. 2. Denne offentliggør aldrig disse, men bruger dem hver især til at kryptere en besked med en ny nøgle i, og offentliggør alle disse krypterede beskeder. Brugerne prøver så at dekryptere dem hver især, indtil de finder deres egen.. Hm, er dette får ressourcekrævende, eller skal denne instans også lige tilknytte et meget lille hash a hver offentlig nøgle med beskeden, så hver bruger ikke skal igennem så mange..? ..Det kunne man sige.. ..although.. ..Tjo, men brugerne kan så stadig downloade alle beskeder i rækkefølge og så bare nøjes med at beholde dem, de skal tjekke.. Hm, lad mig lige tænke, om ikke der er en smartere løsning.. ..Hm, men ellers var pointen så, at enhver bruger, som ikke får en passende besked, bør så anråbe dette, hvorefter alle nøgler så skal indgives, sådan så man kan finde ud af, hvilken part var synderen (inkl. anråberen, hvis dette var en fejl), hvorefter man så kan starte forfra, muligvis uden synderen. Men når hver bruger så har fået en ny nøgle, som kan kan spores hen til dem, hvis alle de involverede instanser (for man kan godt have flere nr.-2-instanser her) bryder deres løfter og offentliggør deres data (og ikke bare sletter det kort tid efter). Nu kan man så være sikker på, at alle brugere i gruppen har netop én anonym nøgle, som nu kan bruges til at oprette en anonym bruger profil for hver bruger, selvfølgelig med VPNs involveret, hvormed denne frit kan afgive sine stemmer og ændre dem, hvornår det skal være, uden at det kan spores tilbage til dem. (14:52) .. ..Og disse anonyme brugere kan så udløbe således at de skal opdateres en gang imellem, således at hvis nu nogen for lækket deres bruger, så vil det allerhøjest kun være den seneste aktivitet, der bliver lækket (og derudover kan man selvfølgelig også dele brugeren op i flere (der ikke kan kædes sammen af andre), hvis man synes, der er besværet værd, men ja, og sådan vil der selvfølgelig altid være ting, man kan tilføje, hvis man finder frem til, at det giver mening..). Nå, men selv hvis der findes et bedre system end dette, så kan jeg jo bare skrive, at det f.eks. ikke er svært at finde på systemer, hvor man via flere instanser, der hver især holder på sin del af en samlet hemmelighed (hvor alle stykker skal bruges, hvis man vil spore tilbage), kan opnå at hver bruger i en gruppe får netop én anonym bruger. Og ja, hvis man så sørger for at de udløber med jævne mellemrum.. Og at brugerne skifter.. Hm.. ..Hm, men det er nu ikke perfekt anonymitet, hvis man sammenligner med valg, hvor ingen data bliver gemt til at starte med, således at ingen nogensinde kan spore det tilbage.. Hm.. ..Hm, men kunne man ikke bare bruge en teknik, som jeg vist også har tænkt på før, hvor en instans bare offentliggør en mængde af.. Hm.. ..Hm jo, en mængde af dens egne offentlige nøgler, nemlig med et antal svarende til antallet af klient-deltagere i øvelsen, og hvor hver klient så vælger et hemmeligt ID, krypterer.. Hm, nej, lad mig lige se... ..Hm, hvad med at alle klienter bare opretter et VPN kun med demselv som noder, og så begynder at sende data rundt. På et tilfældigt tidspunkt sender hver bruger så et ID videre til en naboknude, som modtager, sender ID'et videre til én naboknude, og noterer også ID'et og modtagelsestidspunktet.. nej.. Hm, dette virker vist næsten, men ikke helt.. ..(15:21) Ah, nu har jeg det måske. Man kunne lave en kæde af krypterede blokke, hvor hver blok offentligt hører til en klient, og hver blok rummer data, som brugeren fik tilsendt af ejeren af den tidligere blok, og data som brugeren har sendt videre til næste klient. Denne blokkæde kører så på omgang i en ring, således at den tager flere runder. Og på et tilfældigt tidspunkt tilføjer hver bruger så et offentlig nøgle, som de sender videre. ..Hm, nej det er endnu ikke helt vandtæt.. ..Ah, men måske hvis man tilføjer sin nøgle i krypteret tilstand, så den først kan lukkes op, når den når til en (tilfældigt udvalgt) anden bruger.. Hm, spændende idé.. (15:27) ..Ja, man må næsten kunne lave sådan et system, hvor klienterne billedligt talt danner sådan en rundkreds, hvisker data videre til hinanden én ad gangen i rundkredsen, og hvor klienter i kredsen så kan kryptere en hemmelighed, som en klient et andet sted så kan forstå. Denne bør så med det samme kryptere en ny besked, hvormed denne hemmelighed kastes videre til en anden person. Hemmeligheden er så en offentlig krypteringsnøgle. Man slutter så, efter et vist tidspunkt, når man er næsten 100 \% sikker på, at alle brugere for længst vil have kastet deres nøgle ind i rundkredsen, og at denne er læst af modtageren. Alle brugere offentliggør så de nøgler, der har været sendt frem til dem. Herefter skal alle brugere/klienter (jeg kan ikke lade være med at skrive "brugere" i stedet for klienter, men det er vel også næsten ligeså godt..) så sige, om deres nøgle er iblandt de offentliggjorte (men selvfølgelig ikke udpege dem). Hvis antallet af nøgler passer og alle brugere/klienter siger, at deres er med i mængden, så stopper "legen" succesfuldt. ..Eller rettere, det gør den, efter at man så beder alle brugere om at slette de nøgler, de ville have brugt til at dekryptere deres egne blokke med. Og sikkerheden i systemet handler så om, at man har tillid til, at størstedelen af klienter vil gøre dette (selvfølgelig fordi de bare bruger det udleverede software til det, og ikke har bygget eller tilegnet sig en malicious kopi af denne software).. Men hvis der er for mange nølger, eller at en klient mangler en nøgle, jamen så må man så bede alle brugere om at dekryptere alle deres blokke. Og så er pointen, at man kun ved at have alle disse blokke dekrypteret, kan finde frem til, hvem der er synderen, fordi man så både vil kunne se, hvis de ikke har opfundet netop én nøgle selv, og fordi man kan se, hvis de ikke har videresendt den rette nøgle hver gang.. Nå ja, og hver bruger skal så også bare i det hele taget indsende deres private nølger, så man kan finde frem til synderen. Og hvis enten en klient nægter at indsende den private nølge i dette tilfælde, eller hvis man finder synderen ved at dekryptere alle nøglerne, så må man så udelukke denne bruger i næste tur (altså give denne karantæne). Men ja, som sagt, hvis legen derimod ender succesfuldt, så skal brugerne endeligt ikke offentliggøre deres private nøgler, nej faktisk skal de slette alle deres nøgler, der blev brugt under selve legen og kun beholde den private nøgle, som passer til den offentlige nøgle, de herved fik indsendt anonymt via legen. Og ja, så længe de fleste brugere bare gør dette, så er man ikke i fare for, at det bliver afsløret, hvilke nøgler i slutmængden hører til hvilke klienter. :) (15:53) ..(16:02) Hm, der er faktisk en lille smule hangman's paradox tilstede i denne løsning, men det kan man vist gøre bod på ved bare at sige, at hver knude.. Hm.. ..Hm, eller hvad i stedet med bare at gøre sådan, at klienter i kredsen generelt skal vente et tilfældigt antal omgange, inden.. hm, men det løser dog ikke problemet eksakt.. (16:07) ..Hm, men jo, man kunne vel også bare sørge for, at sandsynligheden for at ens software sender en nøgle starter virkeligt lille og kun vokser over mange runder, og så kunne man gøre sådan, at hvis en bruger bagefter kan se, at deres software har sendt.. Hov, vent, dette er da slet ikke et problem, netop fordi man kaster hemmeligheden frem i rækken.. hm.. ..Hm, der skal kun tre (specifikke) andre brugere til at afsløre en i denne løsning, men de kan det kun hvis man har været uheldig at softwaren har sendt ens nøgle tidligt.. ..Hm, man kunne også bare give hver bruger mulighed for at afbryde legen, hvis deres sofware har sendt deres nøgle tilstrækkeligt tidligt.. Hm.. ..Hm, i øvrigt kan man hurtiggøre processen, hvis kredesn har mange kæder i gang på én gang, så alle klienter kan bygge en blok i hver runde (nemlig hvis der er ligeså mange kæder i gang, osm der er klienter i kredsen).. ..Hm, men kan man ikke bare generere flere nølger, end der er behov for..? (16:18) ..Jo, og så kan brugerne/klienterne til sidst bare vælge, hvilken nøgle af dem, de har fået genereret i legen, de vil beholde, ved at.. Hm.. ..Ah, ved selvfølgelig bare at bekende offentligt bagefter, at "disse nølger var mine, men jeg skal ikke bruge dem alligevel."!:) Og hvis så der lige præcis bliver det samme antal efterfølgende, som der er klienter, og hvis alle meddeler, at de har en nøgle iblandt de endelige, så når man i mål, og ellers må man så bare til at optrævle kæden, for at finde DoS-synderen, hvis ikke legen ender som den burde. :) (16:25) Og ja, det skal så bare anbefales, at hver bruger ikke vælger en nøgle, der blev genereret helt i starten af systemet, men ved at det stadig er brugerens beslutning at udvælge den ønskede nøgle, så eliminerer man altså hangman's paradox.:) (16:26) ..Nå ja, og lad mig lige præcisere, at hver blok så skal indeholde en liste af krypterede nøgler (som hver er kryperet med en tilfældig andens offentlige nøgle), og denne lister vokser altså bare.. tja, eller man kan måske begynde at fjerne ting fra bunden af listen efter et vist stykke tid, når det er sikkert, at samme nøgle er blevet indsat igen i ny version (nemlig ved at en knude har dekrypteret og re-krypteret nøglen og sat den på). Og man kan så kræve, at hver knude tilføjer netop én ting til listen i hver runde.. how, "runde" er et dårligt term at bruge for hvert enkelt lille step, når vi har en rundkreds, så lad os kalde.. tja, lad os bare kalde det enten hver 'step'/'skridt' eller hver tur.. nej, lad os udelukkende kalde det 'skridt'/'step.' Og hvis en bruger så modtager flere beskeder på én gang i et step.. ..Hm, nej vi kan også godt kalde det turn i stedet (for så tænker man jo bare på et lille turn af hjulet).. Så må denne bruger så altså gerne vente en omgang med at sende nummer 2 besked (osv., hvis der modtages flere end to), og altså så kun videresende én af nøglerne i den første tur, hvor nøglerne modtages. Ok, så det var vist bare det, jeg lige skulle præcisere.. (16:40)
%
%%(16:42) Nå, men der er også et andet issue, jeg skal tænke over, og det handler om: Vil det ikke være for fristende for folk at stemme på deres vante repræsentanter i et regeringsvalg, hvor et e-demokrati kæmper, og ser ud til at kunne vinde? For hvis man gør dette, så vil man vel kunne få dobbelt magt, medmindre e-demokratiet kan se, hvem der ikke stemte på det.. hvad de jo ikke vil kunne.. Hm, måske er dette et ret stort problem, men ja, nu vil jeg altså give mig til at tænke godt over det... (16:44)
%
%%(31.10.22, 9:21) Kort efter, jeg klappede i i går kom jeg frem til, hvad vist også havde været oppe at vende i periferien af mine tanker tidligere på dagen, at den simple og måske eneste løsning nok bare er, at sørge for, at e-demokrati-partiet i starten også har til opgave at booste stemmevægte inde i systemet (på en helt transparant måde selvfølgelig), således, at alle repræsentanter, der ellers har fået mandater udover partiet, de får en modvægt til sig inde i partiet. På den måde kan det ikke betale sig at stemme uden for partiet for at pågældende mening skal få mere magt, for så vil den pågældende mening bare blive countered. Og ja, det er så partiets opgave at finde frem til og være ærlig omkring, hvad der er midten af det politiske spektrum i henhold til forskellige punkter, således at man kan counter'e et vist mandat ved at give mere magt til en (eller flere) fra den modsatte (i.e. spejlede) ende af spektrummet. Når partiet så er i regering, så kan man så også bede de repræsentanter, der ikke er med, om at joine, for så vil e-demokratiet bare fjerne magt igen fra dem, der står for at counter'e/udbalancere magtbalancen.. (9:29)
%
%
%%\section{A note on transparency}
%
%
%\section{E-democracies in companies}
%
%%To finish this chapter, let me just make a small point about how an e-democracy application like this might also be incredibly useful when it comes to democratically run companies, or indeed the almost-democratically run `Economically Sustainable' Companies (ESCs.\,. hm, that looks a lot like `Escape(s)'.\,.) that was described in Chapter \ref{MSE}. 
%%
%%If the company in question has a goal of expansion, such as should be the case for the 
%
%To finish this chapter, let me just make a small point about how an e-democracy application like this might also be incredibly useful when it comes to democratically run companies, or indeed the almost-democratically run `economically sustainable' companies that was described in Chapter \ref{MSE}. 
%
%If the company in question has a goal of expansion, such as should be the case in general for the `economically sustainable' companies as described, I envision that this venture will be all the more exciting for the participants if there is a vibrant online community that engages in discussing and finding what strategies to go ahead and try in order to expand the company. 
%
%And if this e-democracy application can be as useful a tool for this as I believe it can, it could thus accelerate the interest in taking part and supporting such a company immensely. 
%
%%Hm, skal jeg så bare stoppe her for nu? (Eller skal jeg skrive videre på denne sektion, og var der i øvrigt andet, jeg har glemt at nævne..?) (15:21) ..Hm, jeg har glemt at nævne min pointe omkring gennemsigtighed ved at sørge for, at folk med jævne mellemrum bliver udtaget til at sætte sig ind i detaljerne og så rapportere tilbage til den interessegruppe, der udvalgte vedkommende, men måske jeg bare skal gemme denne pointe til en anden gang.. (15:23)
%%...(16:01) Nej, jeg tror ikke, jeg behøver at tilføje mere nu. Når jeg så lige får tænkt lidt mere over spiral-protokollen, så kan jeg skrive om den, og ellers er det nok bare lige at redigere teksten. (16:02)
%
%
%
%%Husk:
%	%Jeg havde tænkt mig her at nævne det med, at det kan være smart at udvælge nogen (som så jo kan vælges til at være upartisk og/eller repræsentativ (men måske smart/intelligent nok)) fra en gruppe til at studere og gennemgå systemet i nærmere detaljer og så rapportere tilbage..
%	%Transparancy.
%	%
%	%You never have to waste a vote (and never have to fear wasting a vote). And never have to be fearful, that who you voted for does something you didn't expect (since this system requires no trust in representatives, at least not except in cases why you don't feel like you have the time (or interest) to go through the details of a matter).
%	%..(16:47, 29.10.22) Hm, og husk det her med at man kan have flere områder, hvor forskellige bestemmer, og at dette så også gør, at andre grupper kan logge sig på i systemet, hvor vi snakker om at styre et land. I et sådant e-demokrati kan grupper altså også tilføje områder. På den måde kan de gøre det offentligt for alle, hvad de har tænkt sig. Hermed kan vi altså få en stor markedsplads, der handler om at lave aftaler og bestemmelser, både i regeringen, men også i andre instanser (det kunne f.eks. være såsom fagforeninger, hvilket jo vil være meget relevant i den sammenhæng). ..Og ja, det kan også være grupper, der egentligt ikke har nogen anden magt over noget, men som alligevel vil oprette et område, der hedder "vi mener sådan og sådan, og vil vil gøre sådan og sådan," altså et område, hvor de kan signalerer til omverdnen, hvad deres interesser er, og hvad de gerne vil / er parate til at gøre. (16:54) ..Og ja, det kan så nævnes, at dette så også kan være sådan noget som at trække sig fra den overordnede gruppe (f.eks. e-regeringspartiet eller trække sig som kunde og/eller investor i et firma). 
%	%Jeg kunne godt nævne muligheden i "forbrugerforeninger" kort også (som et eksempel på anden form for magt), men så tilføje, at min kd.v.-idé så netop nok ville være endnu bedre her, for så kan man undgå sådanne reprimanter (eller hvordan det staves). Men om ikke andet kunne det så blive en måde at tvinge gang i en kd.v., hvis nu virksomhederne indenfor en branche er tøvende med det. 
%	%Jeg skal forresten huske at have område-repræsentanter med under avancerede punkter, sammen selvfølgelig med områderne selv. Jeg kan således nok godt nævne "områderne" først, også selvom det egentligt er vigtigere, det med at kunne vælge repræsentanter.. 
%	%"Det handler om at det bliver: meget lettere at samle sig i små grupper, og meget lettere at sætte i gang i en proces, hvor man overvejer, om ikke der kan gøres noget ved et forhold, netop fordi man bare kan starte denne diskussion i nogle små grupper (som så kan kontakte andre grupper, små eller store, når de har fået samlet en oversigt over, hvad problemet er, og hvad man kunne gøre for at løse det m.m.). Så altså langt større tilgængelighed for den enkelte og dermed mange mange flere mennesker aktiveret ad gangen (som så overvejer og finder på løsningsforslag til problemer i samfundet (ofte særlige problemer for nogen specifikke i samfundet, men det kan jo også være mere almene)). Og så vil der så derefter også kunne være meget kortere tid til, fra løsningsforslag til løsning i sådan et direkte demokrati, der er klart. Og ikke mindst vil folk (i grupper) få langt nemmere mulighed for at indgå selv komplicerede politiske aftaler med andre folk (i grupper (ikke nødvendigvis disjunkte med de første, btw)), således at man får et meget bedre og hurtigere kan få handlet sig til at få opfyldt sine behov som en gruppe af mennesker, og således at smafundet derfor vil blivet meget bedre fintunet, så at sige, til at opfylde så mange menneskers forskellige behov som muligt på en gang."
%	%(15:01, 01.11.22) Jeg skal huske noget, jeg lige fik tænkt på, og det er, at et sådant demokrati kan få en meget meget fladere struktur, hvor at man, når man har en ny idé til forandring, lad os sige som lille gruppe, i stedet for så at skulle indsende og ansøge om idéen til en central, så kunne man i første omgang dele den, med den/de mest relevante nabogruppe(r). Hvis de så også er med på den, så kunne man så brede det til endnu flere. Og når idéen så har samlet nok opbakning, så kan man melde det til det brede fællesskab, hvor idéen så allerede har opbakning, når den ansøges om. Jeg ved godt, at sådanne måder at fremføre idéer på allerede finder sted mange steder, men jeg tror, at man i et e-demokrati kunne gøre den fremgangsmåde endnu nemmere og endnu mere hyppig.. Hm, måske vil jeg skrive om dette, men om ikke andet er det da bare rart at tænke på, at der kunne blive sådan en rigtig flad struktur, hvor relaterede grupper selvstændigt kan diskutere og handle om, hvilke idéer og forslag, man vil gå videre med..:).. (15:08)
%	%Man kan også bruge min blok-spral-idé til når stemmerne skal kastes..!
%
%
%
%
%
%
%%(09.11.22, 9:46):
%\section{(I'm considering adding something like:) A similar application for scientific discussion}
%
%\textit{I have now realized that this application could also be used for scientific discussion graphs, which goes hand in hand with decision making since facts are of course important when deciding policies. In a discussion graph, on would just not really need the `conditional node' edges, but would instead just use the `conditional votes' instead --- which could then be drawn as edges between notes for this type of application. %(This all of a sudden make this idea quite a bit more interesting for me in terms of what I would like to work on myself.\,. .\,.\,Hm, hvilket er relevant for mig at have i tankerne i denne stund, for jeg skal nemlig snart til jobsøgeningsmøde med A-kassen. Og ja, med denne indsigt, så må det da næsten være denne idé, jeg vil prøve at gå videre med (og sige jeg vil iværksætte), det tænker jeg.. (..Altså i stedet for Web 2.0--3.0-idéen/erne.))
%*And it should then be very much recommended (as a key part of the idea), that users try to commit themselves to continuously update their votes for propositions as conditional ones, once more fundamental propositions are added to the system. A scientist might for instance be an expert on drugs and say (or actually ``vote'') that: ``this drug is so and so addictive,'' but then once propositions are added about the existence of relevant studies are added, as well as propositions about trust, then that scientist (along with everyone) are then strongly recommended to change the vote into a conditional vote such that the vote now depends on the study existence proposition and the trust proposition. This way (if the community follows this (strong) recommendation), every proposition can slowly become more and more founded in the basis empirical propositions/data, plus trust propositions (which are essentially propositions about how the users want to apply epistemology, i.e.\ when these propositions are also boiled down to their roots). This both has the advantage of the system being more flexible, when new studies turn up or if old ones come into question at some point, and also, importantly, it makes it easier to browse and find out what fundamental facts our more abstract facts in society are built on, i.e.\ to find the sources, and it also gives a better and easier understanding of what is interesting to research, since it shows were the ``gaps'' are, so to speak, or more precisely: where the research is thin and could use bolstering. 
%}
%
%
















\end{document}