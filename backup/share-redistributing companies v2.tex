\documentclass{article}
\usepackage[utf8]{inputenc}


%\usepackage{amsmath}

%\usepackage[toc, page]{appendix}
%\usepackage[nottoc, numbib]{tocbibind}
\usepackage[nottoc]{tocbibind}
\usepackage[bookmarks=true]{hyperref}
\usepackage[numbered]{bookmark}

\hypersetup{
	colorlinks	= true,
	urlcolor	= blue,
	linkcolor	= black,
	citecolor	= black
}


\title{
%	Share-redistributing companies
	A new type of company in which shares are slowly and continuously redistributed to the customers
	\author{Mads J.\ Damgaard%
		%\footnote{
		%	See https://www.github.com/mjdamgaard/notes for potential updates, additional points, and other work.
		%}
		%\footnote{
		%	B.Sc.\ at the Niels Bohr Institute, University of Copenhagen.
		%	B.Sc.\ at the Department of Computer Science, University of Copenhagen.
		%	E-mail: fxn318@alumni.ku.dk.
		%	GitHub folder: https://www.github.com/mjdamgaard/notes.
		%}
	}
}

\usepackage[margin=1.8in]{geometry}

\begin{document}
\maketitle

\begin{abstract}
	We introduce the concept of a new type of company where the shares are slowly and continuously redistributed to the customers. 
	We argue that such companies will be able to reimburse the shareholders fairly for the shares that they part with in this redistribution process by raising the prices for their goods and services accordingly, meaning that the customers will thus essentially buy out the initial shareholders over time. 
	The goal of this new type of company is to provide an alternative in between a regular company and a consumer cooperative, namely one that will turn into something similar to a consumer cooperative over time, but will nevertheless be as effective at attracting initial investments as a more conventional company such as a limited liability company.
\end{abstract}


\section{Introduction}
%{\centering\noindent
%	\vspace{-\baselineskip}
%	\hspace{-0.7em}
%	{\hspace{-4.em}$|$\hspace{\linewidth}\hspace{8em}$|$}
%}
This paper introduces an idea for a new type of company in which the shares of the company are continuously redistributed to the customers with the aim that the customers will end up being the majority owners of the company after a time. A company of this kind is thus supposed to function much like a limited liability company (LLC) initially, but then slowly turn into something similar to a consumer cooperative over time as the ownership shifts hands. 
%
This is assuming that the customers of the company are private consumers rather than other companies. We will stick to this assumption in the following text and then get back to the question of how to define an SRC that has other companies as its customers in Section \ref{sect_B2B}. 

Part of the idea is that the amount of shares that each customer receives should be proportional to the money that they have %(hm, jeg burde i princippet skrve 'he/she has' her i stedet, men det tror jeg ikke folk opdager i dette tilfælde..)
spend on the company's goods and services. We will argue below that the company should then be able to raise its prices accordingly, such that the customers pay a fair price for the shares that they get along with each good and service. The company can then give this money to its shareholders as a dividend and hereby reimburse them fairly for the shares that they have parted with. 
%
Another way of describing the company is thus as one in which the costumers slowly buy out the initial shareholders over time.
%Hm, jeg kom lige til at tænke på: Hvad hvis nu aktiverne er langt større end, hvad kunderne til sammen kan købe på et liv..? (Er det realistisk..?) Hm.. Det skal jeg faktisk lige tænke over.. (11:33, 12.01.23) ...Jamen der må man jo bare sige, at så kan man håbe på, at folk vil være villige til at spare aktier nok op til, at de kan give en god portion af dem videre i arv til deres børn (eller tilsvarende arvtagere). Og derudover kan man så også bare sige, at idéen jo virker allerbedst for de brancher for S/P er høj, nemlig således at det ikke behøver at tage vildt langt tid for kunderne at købe de indledende aktionærer ud. 

For the remainder of this text, we will be referring to this new type of company as a `share-redistributing company,' or an `SRC' for brevity. 



%While it might sound as if the concept of an SRC is similar to simply putting a tax on all the shareholders, we will argue a thesis below that an SRC will generally be able to reimburse the shareholders for the shares that the part with in terms of an added dividend. We will thus argue that investing in SRCs might be almost exactly similar to investing in an LLC, %from the investors point of view, with the only difference that the shareholders of an SRC is required to essentially sell a small portion of their shares to the company at frequent intervals. 




\section{Motivation}

%At der er mange der ønsker mere demokratisk erhverv. Det gør nemlig, at virksomhederne ikke er evige skyldnere til ejerne, men at de kan bruge de penge i stedet til gvn for kunderne og/eller arbejdere. Ydermere kan en større udbredelse af demokratisk erhverv føre til, at virksomheder bliver drevet mindre af pengegrådighed, og mere bare af kundernes og/eller arbejdernes interesser i virksomheden. Har kan man så nævne nogle forskellige punkter, bl.a. også omkring miljø og klima. (Og er dette fint nok for den indledende motivation? Så kan jeg jo altid tilføje ting senere, hvis det er..? ..Nå nej, så skal jeg jo i det mindste også fortsætte og sige:) Men cooperativer kan naturligvis være sværere at finde investorer til.. Hm.. (..Søger lige på, om der er gængse måder/metoder til, at folk kan investere i kooperativer, og i så fald hvordan... (12:43)) ... (17:56) Okay, det har åbentbart taget mig en krig (gik dog også lige en god tur tidligere), men jeg har fået læst nogle gode artikler om kooperativer: To artikler af A. Lingane og A. McShiras fra project-equity.org/about-us/publications/coop-investment og en artikel af M. Lund fra resources.uwcc.wisc.edu/Finance/Cooperative Equity and Ownership.pdf. Og nu kan jeg heldigvis mærke, at det har givet mig noget much-needed blod på tanden igen ift. denne idé. Jeg når nu nok ikke meget mere i dag (hvis jeg kender mig selv ret her om aftenen), men nu glæder jeg mig faktisk til at få udarbejdet denne artikel. Og jeg kan nu se, at jeg virkeligt så skal huske at fiksere på, at min idé gør, at man kan konvertere firmaer til næsten-kooperativer på en måde, hvor firmaet virkeligt kommer til at køre i høj grad som almindelige aktieselskaber. Særligt kan de indledende investorer købe og sælge aktier frit helt som i et normalt aktieselskab, hvilket altså bare er super faktisk, for de fleste aktionærer tænker nemlig mest i de her day trading-agtige baner, hvor det handler om at købe og sælge (og tænker nok ikke helt så meget på dividenden/afkastet)..! Så det skal jeg altså bare virkeligt sørge for at lægge god vægt på. (18:04)
%(18:12) Og jeg skal altså så fortsætte og sige, at kooperativer kan være sværere at finde investorer til. Kooperativer er nemlig meget anderledes at investere i end aktieselskaber... indtil nu!... 

%Jeg bør måske specifikt motivere en virksomhed, hvor denne "køber" aktier fra aktionærer og videregiver dem til kunderne, og så kan jeg så slutte af med at sige, at SRC essentielt set gør dette (måske bare på en mere elegant måde...)..


In conventional companies, the set of owners is often much different from the set of consumers that buy the goods and services that the company helps produce, and the owners' stake in the company is often predominantly to make money. However, the stakeholders of a company are generally not just the owners. The workers and the consumers are also stakeholders as well in their own rights.
In theory, one could claim that if a worker or a consumer is dissatisfied with a decision of the owners, they can simply move to another company instead, but in practice, this is rarely that easy. In reality, workers and consumers are often very much influenced by the decisions of the companies that they deal with. 
%
%..We therefore often times see companies which follows a very consumer-friendly direction in their early stages as the grow, but then when their growth stagnated, they change directions to become more exploitative of their customers who has grown somewhat reliant on their goods and services in the meantime.. 

%It can thus happen that a company can take on a quite consumer and worker-friendly direction in an early stage, but once the growth of the company comes to a halt, the owners might start to look at new options for increasing their profits. This sometimes means trying to find ways to squeeze more money.. and more work.. out of the customers and workers, who might have grown somewhat reliant on the company's goods and services.. and jobs.. in the meantime. And even in cases where the original owners do not intend for the company to ever become exploitative towards its customers, other investors might simply come in and buy up the company if they see the opportunity that the company can be milked for more profits than the current owners allow it to.
%Hm, paragrafen gik ikke som den var, bl.a. fordi den sagde nogle ikke så underbyggede ting i sin tidligere tilstand, men nu føler jeg også, at jeg bliver nødt til at flette arbejdere ind i det, og det kan jeg ikke.. Hm, jeg kunne dog gøre det til to paragrafer, hvi det endeligt er, og hvis jeg kan finde noget relevant at sige om arbejderne.. ..Hm, eller lad mig lige prøve at lave den om til:..
It can thus happen that a company takes on a direction that is quite friendly towards the consumers and workers at an early stage, but later on starts to look for ways to exploit these in order to squeeze out more profits. Even in cases where the original owners do not intend for the company ever to become exploitative towards its customers and workers, other investors might still simply come in and buy up the company if they sense the opportunity that it can be milked for more profits than the current owners allow it to. %Bum, det virker da meget bedre..:)

Worker cooperatives, consumer cooperatives, and multi-stakeholder cooperatives seek to change this dynamic by having respectively the workers, the consumers, and a mix of both being the owners of the companies. This means that the owners can no longer exploit the workers and/or the consumers since this would mean exploiting themselves as a group. 
%Furthermore, when the workers and/or the consumers are the owners of companies, it means that the profits of the companies are quite widely distributed. Such cooperatives are therefore believed to not contribute to an increased wealth inequality in the relevant society, as opposed to the more conventional types of companies such as LLCs (see Wright \cite{Wright}). %Hm, denne sidste del giver vel egentligt sig selv.. ..Hm, og hvor det ikke giver sig selv, nemlig når vi snakker worker cooperatives, jamen det er jo ikke særligt relavant for denne artikel.. ..Hm, jeg har lyst til at nævne, at selv for virksomheder, hvor kunderne ikke ligefrem bliver udnyttet groft, men hvor der stadigvæk genereres en avance til ejerne, de bidrager stadigvæk til en større og større økonomisk ulighed i samfundet på lang sigt. Derfor bør man altså muligvis også være urolig over for sådanne virksomheder.. Men jeg tror nu måske bare, at jeg lader den pointe være i denne omgang..

Having workers and/or consumers as the owners also generally means that the profits of the company go to a very broad class of people. Cooperatives are therefore also believed to lead to less wealth inequality in a society compared to more conventional companies (see e.g.\ Wright \cite{Wright}). 

As Lingane and McShiras \cite{Lingane and McShiras_2} %\cite{Lingane and McShiras 1}--\cite{Lingane and McShiras 2} 
point out, however, cooperatives often have a hard time attracting investors, which is often quite crucial for companies, especially for start-ups. %Since the fundamental idea behind cooperatives, which can be boiled down to: ``one member, one vote,'' is 
One of the big issues that they point to is simply the unfamiliarity of what investing in a cooperative entails. Investors thus do not know what to expect from such investments when they are mainly used to investing in more conventional companies such as LLCs. 

This is why SRCs might provide a useful option in between a cooperative, specifically a consumer cooperative, and an LLC. We will thus argue below that being an investor in an SRC might turn out to be almost exactly like being an investor in an LLC in the initial phase of the company. But over time, the consumers will take over an increasing portion of the ownership, making the company function much like a consumer cooperative in the end. 




%\section{The core concept} %of an SRC}
\section{A company that slowly and continuously redistributes ownership}
\label{sect_def}


The core concept of an SRC is a company where all shareholders are continuously required to give a tiny portion of their shares back to the company at frequent intervals, which is then itself required to distribute these shares out among all its recent customers. The amount of shares that a shareholder owes the company after each interval should be proportional to the amount of shares that they own, and the amount of shares that each customer receives should be proportional to the money that they have spent on the company's goods and services in that interval. 

%These rules are permanent and thus continue to apply for all shareholders, including those who have obtained their shares via this redistribution process. In other words, when the customers receive shares in the redistribution process, this now makes them shareholders as well, meaning that they will also owe a small amount of these shares after each interval. 
%Therefore, if the customer base of the company shifts significantly at some point, the new customer base will then simply become the new recipients of the redistributed shares. Any inactive customers will thus slowly have their shares redistributed to the active customers.

To give an example, suppose that an SRC has chosen an interval of a week between when each share redistribution occurs, and suppose that the amount of shares that each shareholders owes after each interval is chosen to be $0.1\, \%$ of the amount that they own. A shareholder who currently owns 1 million shares will then have to part with roughly 1 thousand shares each week, whereas another shareholder who owns 2 million shares will have to part with 2 thousand shares. And if for instance the combined sales in a given week totals 1 million dollars and one customer has bought goods and services for a sum of 100 dollars, that customer will then receive $0.01\,\%$ of all the shares that are redistributed at the end of that week. 
Suppose now that the total number of shares of the company is 1 trillion. That customer would then receive $10^{12} \times 0.1\,\% \times 0.01\,\% = 100.000$ shares along with the goods and services bought for 100 dollars. 

This example assumes that the SRC redistributes all the shares received from the shareholders right away at the end of an interval. This would mean that the customers will not know the exact amount of shares that they get per dollar spent after each interval, as this amount would depend on the sales for that interval. But if the intervals are relatively short, the customers and the SRC would generally still be able to predict this amount with good precision, which should then allow them to agree on a fair price added to the goods and services due to the shares that come along with them. 

Alternatively, the SRC can implement a system where it is allowed to keep a small buffer of shares that are waiting to be redistributed. The SRC then has to be obligated to continuously adjust the exact number of shares that all the customers get per dollar spent in order to keep the buffer approximately constant. If the sales then suddenly goes up unexpectedly, the buffer is then intended to give the company time to adjust the number of shares per price down such that the customers get fewer shares per dollar spent, which means that the buffer can grow back to its intended size. Note that such an adjustment still means that the customers will pay the same price for each share that they receive, but they will simply receive fewer shares per good or service that they buy. With this system, the customers will thus know the exact amount of shares that they get/buy along with each good and service, rather than simply knowing the approximate amount. 

%This requirement means that if the customer base of the company shifts significantly at some point, the targets of the redistribution process will also shift accordingly, namely such that the new customer base will be the one that slowly acquires ownership of the company, not the old one. *(rewrite)


The rules of an SRC are permanent and thus continue to apply for all shareholders, including those who have obtained their shares via the redistribution process. In other words, when the customers receive shares in said process, this now makes them shareholders as well, meaning that they will also owe a small amount of these shares after each interval. 
Therefore, if the customer base of the company shifts significantly at some point, the new customer base will then simply become the new recipients of the redistributed shares. Any inactive customers will thus slowly have their shares redistributed to the active customers.



\section{Ensuring that the customers hold on to the redistributed shares}


Since the aim of an SRC is that customers should end up being the majority owners, it is natural to limit how the redistributed shares can be sold by the customer-shareholders. Otherwise any party of opportunists with enough money could come and buy up a majority of the shares, perhaps with the intent of changing the direction of the company to a more exploitative one. On the other hand, if the customer-shareholders are not allowed to trade the redistributed shares at all, the value of the shares might become less in the eyes of the consumers than what it would have been if the shares could be traded freely. 

We will now look at a solution that seeks to prevent the takeover from an outside party while still giving some freedom for the customer-shareholders to trade their shares. Let us first of all divide the company shares up into at least two classes. Let us thus take the initial shares of the SRC to be the Class A shares of the company, and take the Class B shares to be the shares that have undergone the redistribution process, once or several times. This means that more and more Class A shares will be converted into Class B shares as time progresses. Both of these classes of shares should include voting rights, but the SRC can now put limits on how the Class B shares can be sold by the customer-shareholders. 

There are different options when it comes to limiting the sales of the Class B shares, but the solution that we will propose in this paper is to rule that the Class B shares can %generally%\footnote{
	%An exception could perhaps be in an initial phase.. %Hm, jeg synes faktisk ikke, dette er værd at nævne..
%} 
only be sold to other customer-shareholders, and only as long as these shareholders do not obtain more Class B shares than a certain factor above what they would have had if they had never traded any Class B shares, meaning that they would have only received and parted with Class B shares through the redistribution process. 

To put this in other words, let us think of the amount of Class B shares that each customer-shareholder owns as a curve. If the shareholders were not able to trade their Class B shares with others, this curve would have an underlying exponential decay due to the redistribution process, but would go up whenever the customer in question buys new goods and services from the SRC. If we then from there relax the sales restrictions on the Class B shares according to the previous paragraph, we can add another curve alongside the previous one to show the actual amount of Class B shares that the customer owns now that he/she is able to trade these as well. The latter curve will then also go up and down whenever the customer make such trades. But according to the restrictions of the last paragraph, the customer is then not able to buy Class B shares if it would mean that the latter curve would become higher than the mentioned factor times the height of the former curve.

This factor of how much the customer-shareholders are able to increase their stock of Class B shares can potentially be chosen as quite large without running the risk of allowing an opportunist party to come in and buy up a majority of the Class B shares. 
For in order to do so, that party would have to make up $51\,\% / x$ of the demand on the company's goods and services, where $x$ is the factor in question. And given the assumption that the customers are private consumers, the company might very well expect that even their biggest customer will only ever make up a small part of their total demand.

%The exception to this rule could be to allow Class B share owners to sell these shares freely in an initial period of the SRC. This way it can be ensured that there are large enough amount of potential buyers when the rules goes into effect so that the customer-shareholders can still get a fair price if they want to sell their Class B shares right away.
%Hm, jeg tror faktisk, jeg vil udelade denne pointe i denne version af artiklen. Jeg kunne jo så i stedet bare lige sørge for at fremhæve den i min GitHub-mappe --- og ellers står den i første verion af artiklen..


%And of course, the customer-shareholders should also be able to give their Class B shares away as inheritance to their heirs at the end of life. 

With these restrictions on the Class B shares, the company is ensured to be increasingly owned by the customers, but it still allows the customer-shareholders to have an internal market for the Class B shares. This first of all means that any customer-shareholders in need of money will have the ability to sell their shares and probably get a fair price for them, just as if they were shareholders of an LLC instead. And perhaps more importantly, customer-shareholders at old ages who want to use their stock of Class B shares as part of their pension can do the same.\footnote{
	Speaking of old age, it almost goes without saying that the shareholders of the various classes of shares should be free to hand these over as inheritance to their heirs at the end of life, in the same way as for conventional types of stock.
}

However, in order to be absolutely sure that the customer-shareholders have the freedom to sell their Class B shares whenever they want and get a reasonable price for them, an SRC might also implement another class of shares, call them Class C shares, which are then similar to the Class A shares in that they can be traded freely, but which are non-voting shares. So if it for some reason happens that there will not be enough potential buyers of Class B shares at some point to give the sellers fair prices, the sellers can then convert their shares and try to sell them (freely) as Class C shares instead. Since many investors do not care much about the voting rights that follow with the shares that they buy, it is reasonable to expect that these Class C shares will sell for almost as much as what the Class B shares would normally sell for at the given time, i.e.\ when there are not a shortage of buyers. 

%\section*{navi.}




%Another option, which could also be combined with the previous option, is to allow customers to trade B type shares to some extent, but to then tax all sales of such shares. This alone would not prevent the customers from selling their B type shares to other parties, but it would give them an incentive not to do so.

%If this last option is combined with the second one, it might help ensure that there are always enough potential buyers... Hm.. %..(15:21)
%... Hm, jeg tror faktisk, jeg vil droppe at nævne det med skatten, for jeg synes faktisk alligevel ikke særligt godt om den mulighed.. ...(17:21).. ..Ja, lad mig bare droppe at nævne det for nu..




%\section[Investing in an SRC compared to an LLC]{Why investing in an SRC should not be much different from investing in an LLC in the initial phase}
\section{Investing in SRCs compared to LLCs}
\label{sect_thesis}

If the system described in the previous section works to ensure that the customer-shareholders can always sell their shares and get a fair price, it should mean that the customers will also be willing to pay a fair price for the shares that they get along with the goods and services in the first place. For in that case, the customers can simply treat the shares that they get/buy from the SRC in this process as part of their pension plan. 

This thesis of course has to be tested empirically before we can claim it with any more certainty, but it is certainly not unreasonable to expect that it will hold true.
%
And if it does hold true, it means that the initial shareholders can be fairly reimbursed for the shares that they part with in the redistribution process. It follows that the value of the shares of an SRC should be equal to what it would have been had the same company been an LLC instead, at least if we assume that the customer interest is the same either way. And if anything, the customer interest might actually increase if, say, an LLC converts to an SRC, namely due to the fact that a lot of people might feel positive towards a company that promises more consumer stock ownership in the future. It is therefore even possible that some LLCs will be able to increase the value of their shares by converting to SRCs.

And just like in an LLC, the Class A shares contain voting right and can be traded freely. The only difference between being a Class A shareholders in an SRC and a shareholder in an LLC should thus be that in the SRC, the company automatically buys a small portion of the any given shareholder's shares at frequent intervals, which should not be a problem for most investors. It is therefore reasonable to expect that investors familiar with investing in LLCs will also be interested in investing in SRCs. 



\section{The change of ownership over time}

The time it will take for the customers to buy out the previous owners will depend on the portion of shares that the shareholders owe after each interval, which will again first of all depend on how much money the SRC generally wants its customers to pay for the attached shares compared to the actual price of the goods and services that they are buying. It will also depend on the price-to-sales ratio (P/S) of the company. 

Suppose for instance that a certain SRC predicts that its customers would prefer to pay about a $10\, \%$ increased price for each good and service due to the attached shares. Suppose also that the price-to-sales ratio (P/S) is at 2, and that the interval before each redistribution is chosen to be half a month. The company then wants to redistribute shares to the customers at each interval worth about $10\, \% / 2 / 24 \approx 0.2\, \%$ of the company's total price (P).
%\footnote{
%	This means that company then chooses that the shareholders owe $0.1\, \%$ of their shares after each interval, and will continue to do so regardless of whether the price of the company changes. 
%	If this price increases, the prices on all goods and services can then go up slightly, and vice versa.
%} 
Since
\[
	(1 - 0.002)^{24 t} = \frac{1}{2} 
	\;\Rightarrow\; 
	\log 0.998 \times 24 t = -\log 2 
	\;\Rightarrow\; 
	t = \frac{-\log 2}{24 \log 0.998} \approx 14.4,
\]
it would then in this case take roughly fifteen years for the customers to obtain half of the shares of the company. (And it would take them about thirty years to obtain three quarters of the shares, and so on.)

The fact that these numbers are not unrealistic is a big motivator for considering the prospects of SRCs when compared to consumer cooperatives, especially in industries with relatively small price-to-sales ratios. We have argued above that SRCs might be better at attracting investments in their initial phase compared to consumer cooperatives. And given the fact that SRCs can potentially have quite short periods before the customers will have bought a majority of the shares in the company, it will potentially not take very long before an SRC can start to function much like a consumer cooperative. 

For although an SRC does not turn exactly into a consumer cooperative over time, it does turn into something quite similar to that. Only, instead of following the principle of `one member, one vote' (and one share) known from cooperatives, an SRC will tend towards a state where the consumers have voting power proportional to how much money they tend to spend on the company. %Some might see this as a slight minus compared to consumer cooperatives, but others might see it as a slight plus since it means that in a hypothetical future where SRCs have become prevalent in all kinds of industries, each consumer will have more power over the companies that they use the most. For instance, if we consider a local company with a local set of primary customers in this hypothetical future, the ownership will then mainly be on the hands of these local customers, rather than %..Tja, men folk kan jo selv vælge, hvor de vil være medlemmer i en hypotetisk forbrugerkooperativ-fremtid, så måske er dette ikke værd at skrive..
%
%
Thus, in a hypothetical future where SRCs have become prevalent in all kinds of industries, the distribution of the power of each consumer will naturally depend on the distribution of their consumption, whereas it would simply depend on where they choose to apply for %(and buy into) 
memberships if these companies were all consumer cooperatives instead.


\section{When SRCs have other companies as their customers}
\label{sect_B2B}

The concept of an SRC does not have to be limited to companies that sell their goods and services directly to the consumers. When it comes to SRCs that have other companies as their customers, they have at least two options. First of all, they can follow exactly the same rules, meaning that their customer companies will slowly buy up and take over the ownership of these SRCs over time. But unless their customer companies are already owned by the consumers, or perhaps will be in the future if these companies are also SRCs, this option will not directly lead to more consumer stock ownership, as is otherwise the goal behind SRCs. 

However, if an SRC has customer companies which are not SRCs themselves (and also not consumer cooperatives), the SRC can follow a second option when dealing with these, %which is to make sure that all their goods come with a code to redeem the shares that are owed to the customers, or something to that effect.. Hm, det kan faktisk godt være, jeg bliver nødt til at omtænke dette en smule.. Lad mig se.. ..Ah, det går lige op for mig, at jeg med det der code-redeem-halløj jo har tænkt specifikt på virksomheder, der sælger færdige produkter til detailforretninger. Og ja, her kunne den løsning være meget god, men i alle mulige andre tilfælde går dette jo ikke. ..
which is to require them to sign an agreement that they will distribute the shares that they receive from the SRC further out to their customers, such that the shares ultimately get distributed out to the consumers. Whether the shares should then be distributed in proportion to how much money the consumers spend on the customer companies, or if they should be distributed only to consumers who buy specific goods and services that the SRC has helped produce, that is then simply up to the SRC in question, as well as to what its customer companies will agree.



%\section{A hypothetical future where SRCs have become very prevalent..}


%\section{Additional points...} %Perhaps about having many shares...


%\section{Comparing concept of SRCs to other ideas for achieving consumer ownership}
\section{Comparing the idea of SRCs to the existing concept of consumer stock ownership plans}

%CSOPs, being an individual investor, investment funds..

%Jeg tror faktisk bare, jeg vil nøjes med at gøre denne sektion til sammenligningen med CuSOPs. For de andre ting kommer bare til at have lidt for meget med hele idéen om en stor forbrugerbevægelse at gøre, og det siger jeg jo ikke rigtigt noget om i denne version af artiklen.
%I øvrigt kan jeg lige nævne her, at idéen om alternativt at oprette en demokratisk investeringsfond, som folk kan bruge i stedet for deres normale investeringsfonde, faktisk kunne være en rigtig god mulighed også. Så skal fonden bare have (gerne direkte) (e-)demokrati og overordnet set sigte mod at investere i ting, der kommer kunderne til gavn (og hvor man bestemt bruger stemmerettighederne, der følger med aktierne). Samtidigt er det også meget vigtigt, at folk kan give deres anparter i fonden videre til arvtagere. På denne måde kan almindelige forbrugere også overtage mere og mere ejerskab i industrien. Herved sker det så bare ikke så meget 'en virksomhed ad gangen,' men i stedet har vi bare en gruppe af forbrugere, der får mere og mere ejerskab, mere og mere magt (følgelig) og flere og flere penge som gruppe. Og når andre forbrugere ser at denne gruppe er på vej frem i samfundet, så vil de jo også kunne blive tiltrukket til at joine. Når det så er sagt, så synes jeg dog ikke at denne idé er ligeså god; jeg tror ikke, den har ligeså kraftigt et potentiale. For der ligger bare noget virkeligt godt i SRC-idéen i, at forbrugerne bare skal gøre, hvad de gør bedst, nemlig forbruge (og nærmere bestemt være selektive som forbrugere, hvilket de fleste allerede er!), og i det hele taget at idéen nærmest ingen organisation kræver fra forbrugernes side i princippet for at virke. :) 
%Nå jo, og en lille ekstra, hurtig tilføjelse til fond-idéen, så kan det også nævnes, at hvis målet er at forbrugergruppen skal samle mere og mere ejerskab og flere og flere penge (eller værdier, rettere) sammen, så kan de i princippet gøre dette, selv hvis fonden fik profitter på dens samlede investeringer. For så længe at fonden som minimum bare kan holde på værdierne, så kan idéen faktisk også bare gå ud på, at medlemmerne basalt set 'lægger flere og flere penge til side over tid (som går i arv til deres arvtagere, som så også kunne være med på idéen) til at købe mere og mere af samfundets aktiver tilbage fra de eksisterende kapitalister. Denne version af idéen er ikke nær så sjov for medlemmerne, og der findes meget bedre idéer, også som ikke kræver at medlemmerne er meget bekyrede for fremtidens økonomi, hvad denne sidstnævnte version af idéen kræver. Men det er dog en idé, der på et teoretisk niveau er interessant, faktisk eventuelt som et \emph{modargument} for, hvorfor man behøver socialisme; "folk kan da bare lægge penge til side, hvis de er så bekymrede for fremtidens økonomi." Anyway, det var bare en lille ekstra-idé, der er lidt interessant på en eller anden led. Men alt i alt, så er det altså SRC-idéen, hvor jeg virkeligt ville sætte mine penge, hvis jeg skulle væde om, hvilken idé kunne have størst potentiale. 


An existing idea that relates a lot to the idea of an SRC is that of consumer stock ownership plans (CSOPs). 
The concept behind a CSOP is that a company with a relatively constant and reliant customer base of consumers, e.g.\ a water or electricity supplier, can take on a loan on behalf of the consumers in order to finance a certain expansion. The consumers will then gradually pay back this loan as they pay for the company's goods and services, and in return, they will acquire shares in the company and thus become co-owners. 

The difference between a CSOP and an SRC is thus that a CSOP covers financing for a specific expansion, and the plan will therefore end once the consumers have repaid the loan. This is opposed to the redistribution plan of an SRC, which is a perpetual plan, and where the recipients of the shares can change freely over time as the customer base changes. And in an SRC, the plan is always meant to involves all the shares of the company, meaning that the ownership will tend towards being $100\,\%$ on the hands of the consumers as time progresses. 



\section{Conclusion} 

We have introduced the concept of a `share-redistributing company,' and argued why such companies might be as good at attracting investors as an LLC. We have thus argued that owning Class A shares of an SRC will be almost equivalent of owning shares of an LLC, except for the fact that the company will automatically buy back a small portion of these shares at frequent intervals in the former case. We have also argued that the value of the Class A (and Class B) shares of an SRC should not be different from what they would have been if the company had been an LLC instead. 
%
This is of course assuming that the customer interest is unaltered whether the company is an SRC or an LLC, although if anything, the SRCs might attract a greater customer interest due to the fact that a lot of consumers might feel positive towards companies that will lead to more consumer stock ownership in the near future. 

%While an SRC does not turn exactly into a consumer cooperative over time, it does turn into something quite similar to that. Only, instead of following the principle of `one member, one vote' (and one share), SRCs will tend towards a state where the consumers have voting power in the company proportional to how much money they tend to spend on the company. %Some might see this as a slight minus compared to consumer cooperatives, but others might see it as a slight plus since it means that in a hypothetical future where SRCs have become prevalent in all kinds of industries, each consumer will have more power over the companies that they use the most. For instance, if we consider a local company with a local set of primary customers in this hypothetical future, the ownership will then mainly be on the hands of these local customers, rather than %..Tja, men folk kan jo selv vælge, hvor de vil være medlemmer i en hypotetisk forbrugerkooperativ-fremtid, så måske er dette ikke værd at skrive..
%%
%Thus, in a hypothetical future where SRCs have become prevalent in all kinds of industries, the distribution of the power of each consumer will naturally depend on the distribution of their consumption, whereas it would simply depend on where they choose to apply for %(and buy into) 
%memberships if these companies were all consumer cooperatives instead.










%\ 
%
%\newpage
%\section*{navi.}

%When requiring the shareholders to frequently give out part of their shares to the customer base, it sounds similar to putting a tax on the shareholders, and one might think that this would then reduce the value of these shares. We will now give an argument for why this should not be the case. 
%
%
%The reason for this lies in the fact that the number of shares given to each customer is proportional to the amount of money that the customer has paid to the company in the relevant interval. Assuming that the customers are able to follow the sales numbers and calculate how many shares they will get from any dollar spent at the SRC's goods and services with reasonable precision, one might very well expect that they will be willing to pay that much extra for the goods and services. This is certainly true if the company had no sales restrictions on the B type shares as well. In that case, the customers will certainly be able to sell the shares that they get along with the trades whenever they want.
%
%If we include sales restrictions on the B type shares, however, the model gets a bit more complicated. Specifically, if we look at the type of SRC that allows no sales of the B type shares, it is reasonable to expect that this would lower the value of these shares in the eyes of the customers since the assets then would not be liquid. 
%
%On the other hand, if we look at the types of SRCs as described in the previous section which aim to ensure that the customers can always sell their B type shares to a large number of potential buyers, it then reasonable to again expect that the value of the shares will see no such drop in the eyes of the customers.
%
%If this is true, it means that investing in an SRC can look almost the same for investors as investing in more conventional companies, specifically such as LLCs. If customers are generally willing to pay the extra price for goods and services that corresponds with the value of the shares that are then redistributed to them, the previous shareholders will be able to reimbursed for the shares that they part with, and the shares will thus not lose any value when compared to the shares of an LLC. Furthermore, the initial investors in an SRC would be able to trade the A type share completely freely, just like they are when it comes to the shares of an LLC. 
%The only difference would thus be that the initial shareholders in an SRC would continuously have to part with a small portion of their shares at frequent intervals in return for an increased dividend that reimburses them for these losses.




%\section*{navi.}



%\section{Possible counterpoints}
%Now, if we try to search for counterpoints to the thesis of the previous section, we have to consider if and how it could happen that there a number of B type shareholders would want to sell their shares, but not be able to find enough buyers to yield them a fair price. 
%..Hm, men det skal jo være en vedvarende situation, hvis det skal kunne propagere tilbage til "nutiden".. Med andre ord, medmindre der er et systematisk forhold, der gør det forventeligt, at der ikke vil være nok købere.. Hm, og (uden at færdiggøre sidste sætning, for vi ved, hvad jeg taler om) det ville kun ske, hvis restriktionerne sammen med et forhold, hvor det kun er nogle få, der har lyst til at købe, og mange har lyst til at sælge.. Hm, så jeg behøver faktisk ikke tænke så meget på.. vent, og kunne man egentligt forresten ikke bare gøre sådan, at man godt kan hæve faktoren.. tja, og dog.. Lad mig lige tænke.. ..(Skulle til at sige: behøver faktisk ikke tænke så meget på tilfælde, hvor få personer pludselig kommer i pengemangel, btw..) ...Måske skulle man overveje at have en pensionsalder, hvor kundrne ikke længere behøver at "købe"/købe aktier med, når de køber varer og servicer.. ..Ja.. ..Lad mig lige tænke noget mere over et hele, og prøve at finde ud af, hvad jeg så kan sige klart, men muligvis ret vigtig tilføjelse (har overvejet det i glimt førhen, men jeg har aldrig taget tanken vildt seriøs før nu)..
%(12.01.23) Okay, i går blev en ret afslappet dag, men jeg kom til gengæld på noge super vigtige tanker. For det første var der muligheden for at undsige ældre fra at købe aktier, men nu har jeg fået en bedre idé, mener jeg. Hvis folk ikke kan sælge deres B-aktier til en fair pris, jamen så skal de altid bare kunne konvertere og sælge dem som C-aktier, hvilket så er ligesom A-aktierne, bare uden voting rights hørrende til sig. Nu vil jeg så skrive det hele om, også hvor jeg faktisk ændrer kernekonceptet til bare at være 'en virksomhed, som skal købe dens aktier tilbage fra investorer kontinuert, lidt ad gangen, og så skal disse aktier fordeles til kunderne.' Det vil jeg gå i gang med nu.








%\section*{navi.}

\begin{thebibliography}{20}
	
	
	\bibitem{Wright}
	E.\ O.\ Wright, 
	\textit{ How to Be an Anticapitalist in the Twenty-First Century} 
	(Verso, La Vergne, 2019).
	
	
%	\bibitem{Lingane and McShiras_1}
%	A.\ Lingane and A.\ McShiras. 
%	(2017). 
%	The original community investment: A guide to worker coop conversion investments.
%	Project Equity.\\
%	\url{https://project-equity.org/wp-content/uploads/2017/04/The-Original-Community-Investment_A-Guide-to-Worker-Coop-Conversion-Investments_Project-Equity.pdf}
%	
%	
%	\bibitem{Lingane and McShiras_2}
%	A.\ Lingane and A.\ McShiras. 
%	(2017). 
%	Addressing the risk capital gap for worker coop conversions: Strategies for the field to increase patient, risk capital.
%	Project Equity.\\
%	\url{https://project-equity.org/wp-content/uploads/2017/04/Addressing-the-Risk-Capital-Gap-for-Worker-Coop-Conversions\_Strategies-for-the-Field\_Project-Equity.pdf}
	
	
	\bibitem{Lingane and McShiras_1}
	A.\ Lingane and A.\ McShiras.
	(2017). 
	The Original Community Investment: A Guide to Worker Coop Conversion Investments.
	Project Equity.\\
	\url{https://project-equity.org/wp-content/uploads/2017/04/The-Original-Community-Investment_A-Guide-to-Worker-Coop-Conversion-Investments_Project-Equity.pdf}
	
	
	\bibitem{Lingane and McShiras_2}
	A.\ Lingane and A.\ McShiras.
	(2017). 
	Addressing the Risk Capital Gap for Worker Coop Conversions: Strategies for the Field to Increase Patient, Risk Capital.
	Project Equity.\\
	\url{https://project-equity.org/wp-content/uploads/2017/04/Addressing-the-Risk-Capital-Gap-for-Worker-Coop-Conversions\_Strategies-for-the-Field\_Project-Equity.pdf}
	
	
	\bibitem{Lund}
	M.\ Lund.
	(April, 2013). 
	Cooperative Equity and Ownership: An Introduction.
	UW Center for Cooperatives.\\
	\url{https://resources.uwcc.wisc.edu/Finance/Cooperative%20Equity%20and%20Ownership.pdf}
	
	
	\bibitem{Lowitzsch}
	J.\ Lowitzsch, Consumer Stock Ownership Plans (CSOPs): The Prototype Business Model for Renewable Energy Communities, 
	Energies \textbf{2020}, 13(1), 118; \url{https://doi.org/10.3390/en13010118}.
	
	\bibitem{Dragsted}
	P.\ Dragsted, 
	\textit{Nordisk Socialisme: På Vej mod en Demokratisk Økonomi} 
	(Gyldendal, Copenhagen, 2021).
	
	
	
\end{thebibliography}



















%
%
%
%\chapter{E-democracies} \label{E_democracies}
%
%The concept of a so-called `e-democracy' is not a new one. Wikipedia thus has (in the moment of writing) a whole article about the overall concept that one can read. (That article, in its current form, defines the concept perhaps a bit more abstractly than what we need for our purposes here, but it might still be helpful to glance at.) In this section, I will therefore not introduce the overall concept, but simply give some short notes on how one might implement such an e-democracy, which can for instance be used to govern a company like the ones described in the previous chapter (as its shareholders), or a political party, etc. 
%
%
%\section{A basic digital application where voters can build proposition graphs}
%
%Imagine a digital application where all voters in a given democracy (concerning e.g.\ a company, a union, a political party, etc.) can log on and build a proposition graph together, which can then define the policies of the body governed by the given democracy. We are here talking about the `graphs' of mathematical graph theory. (One can make a brief search the internet for `graph theory' to see what this is about, and one might then also want to search for `directed graphs' and `connected graphs' at the same time.) 
%
%Each node of the graph holds a proposition, which is simply expressed in plain text of whatever natural language (such as English) is appropriate for the case. 
%
%When adding a new node to the graph, one can add it by itself (i.e.\ not connected to any other nodes) or add it with at least one of two kinds of (directed) edges to an existing node. The two types of edges then represents whether the node's proposition is an elaboration on the parent node, or if it is a self-contained proposition that should, however, only apply conditioned on the parent node being active.
%
%A node becomes active if it has enough votes and if a majority of those votes are positive rather than negative. Whether `votes' are counted simply by number (such that all voters have equal power) or if the votes are weighted (meaning that some voters have more power than others) of course depends on the case. 
%
%The point of being able to `elaborate' the proposition nodes rather than having to replace it with a more detailed note instead is simply to make the work easier for everyone, and also to make the graphs easier to read. It means that the policies can be defined somewhat loosely at first (and therefore much more quickly and easily), and whenever some vagueness of the propositions is discovered subsequently, either by people studying them or because of a relevant case that reveals it, the voters can then work to specify and mend the propositions. 
%
%The point of being able to add proposition nodes that are conditional on their parent nodes being active is of course some propositions might only be beneficial to implement given that certain other ones are already in place. If a somewhat fundamental proposition node is voted inactive again, it is thus convenient that such `conditional child nodes' follow along. If the parent node is then voted active once again (or perhaps for the first time) at a later time, all the child notes that has retained a positive voting score in the meantime will then become active one again, as well as any child node whose score has become positive in that time. 
%
%The application might also allow these `conditional child notes' to have several parent nodes for convenience. 
%And the same could also apply for the `elaboration child nodes' since there might be case where it could be beneficial to be able to elaborate the interpretation when two propositions nodes are active at the same time, for instance if these two proposition have a slight conflict with each other, or if the create some other issue that needs to be handled when they are both active together.
%
%`Elaboration child nodes' should of course also depend on their parents being active. The difference between a `conditional child node' and an `elaboration child node' is therefore essentially only in the interpretation: The propositions of `conditional child notes' and those of their parents are meant to be independent of each other as statements, whereas `elaboration child nodes' are free to correct and override parts of the statements contained in their parent nodes, thus allowing these to not necessarily be absolutely precise and self-contained. 
%
%Every user should be able to add new proposition nodes and each node should also have a separate `interest score' that users can rate (with the same weights on the votes as for the first score in the case where these are weighted). A proposition node whose `interest score' exceeds a certain threshold becomes visible to all users in the main graph, and people will then have to give their votes to it, if they want to influence whether it is applied or not. 
%
%Users should thus also be able to view nodes in the graph that has not yet exceeded said threshold, perhaps by being able to select various ranges of interest scores to look at. It might also be beneficial to let such nodes expire after a certain time if their `interest score' has been low enough for too long. 
%
%Users should also preferably have their own `workbench' with enough storage capacity to hold a number of propositions nodes. If a proposition node expires, they can thus make sure that the work is not lost as long as they keep said proposition on their own `workbench.' It would probably be beneficial also if users could then have shared `workbenches' as well, where they can collaborate on making new proposition nodes. 
%
%Anonymity is of course generally very important for democracies. So it is naturally very important that no one can see which user has added what nodes, unless of course they have collaborated on it from the same `workbench.' Users should also not (for most cases of democracies) be able to see which users has voted for what. 
%%
%For cases with weighted voting, either with very few voters or with very precise weights, this might be helped further by making sure that the exact voting scores are not visible to the users, and that the can thus only see a number that is rounded to a less precise floating point number. One might also implement intervals such that new votes are always declared together in groups, some time after they have been cast individually. 
%
%
%\section{How the proposition graphs are used to govern a body}
%
%The point of building these proposition graphs is then that the leadership of the body you are governing should to some extent be required to follow the active propositions, at least within some basic limitations of they can be required to do. 
%
%When the proposition graph changes, they leadership should be required to implement these, but here it might of course be a good idea to implement some delays on when new changes are supposed to be carried out. One might thus rule that a change should only be implemented after a certain period from when happened, and only if that change has remained active in the proposition graph during that period. 
%
%If the proposition graph gets some contradictions and/or ambiguities that makes it hard for the leadership to know what to do, they should also be allowed to postpone implementing the relevant changes until the voters sorts out the issues (by which they make some new changes which restarts the acceptance process). 
%
%How to make sure that the leadership follows the democratic decisions of the proposition graph? Well, by making sure that the voters also have enough direct power over the governed body to enforce their will. This might typically be ensured by the group of voters having the power to fire leaderships and/or decrease or increase salaries, thus giving this group ``sticks'' and potentially ``carrots'' that they can use to make sure that the hired leadership does what it is supposed to.
%
%
%It has to be mentioned that a high level of transparency is an all-important part of an effective e-democracy when it comes to the body that is being governed. Luckily, one can say that as long as the voters have the aforementioned ``carrots'' and ``sticks'' at their disposal, they should at least be able to make the body more and more transparent, even if it not very much so from the beginning.
%
%
%
%%Husk:
%	% Fortolkningspolitikker (inkl. hvad man gør, hvis der er modstride) og delays. (tjek)
%	% The point with having conditional propositions.. (tjek)
%
%
%\section{A more advanced application}
%
%A basic system like the one described above is good enough for very simple cases where it is okay to just have a majority rule. But for more complex cases where there are a lot of groupings of voters with different interests when it comes to various topics, to have such a majority rule is not really sufficient. If we for instance think of the policies of a whole country, this is a good example of such a complex case, where most people probably have \emph{some} special interests that are only shared with a fraction of the society. In such a system, it is important for people to be able to \emph{negotiate} with their voting power in order to get what they want, not just to always vote for exactly what they want as individuals. One group might thus want to meet with another to make a deal where they say: ``If you vote for this and this, even though you might not be particularly interested in that, we will vote for this and this which you do have a particular interest in (even though we might not).'' 
%
%In order to accommodate these realistic needs of its users, the digital application in question should therefore also first of all make it possible for users to form groups in the system. On a technical note, having such groups can of course be implemented in a lot of ways, but I can suggest an implementation where the creators of a group start out with some divisible `moderation tokens' that give them power to decide who can join the group and who gets kicked out, and where they are then free to transfer parts of (or all of) these tokens to other users within the group at any time. This moderation system is open enough that the users can implement any other moderation system on top of this, if only they trust some central party (which can then control a user profile in the system) to enforce the results of this external moderation system.
%
%And in order for such groups to be able to start negotiating with their voting power, it is then first of all important that some overall statistics (perhaps where numbers are rounded to ones with less precision for the sake of anonymity) of how a group votes on average is made public at all times. Otherwise a group who has made a deal with another group would not be able to check that this other group holds its promises. 
%
%With this addition to the application, groups can now in principle make all the deals in private that they want to. But of course, a good implementation of an `e-democracy application' would also afford its users with ways to make these deals within the digital application, online. 
%A way to do this might be to add what we could call `conditional votes' to the system. A `conditional vote' is then a vote on a proposition, whose sign can depend on other factors. In particular, a conditional vote should be able to depend on parameters regarding the voting statistics of groups. A group that want to make a deal with another one can then decide to make a conditional vote for something the other group wants and then make the condition such that this latter group has to vote for something the former group wants to unlock the conditional votes. 
%
%On another technical note: Depending on how the system is implemented, such conditional votes might be able to cause deadlocks, where two or more conditional votes all wait for the other to turn the other way in order to turn themselves. But a way to mitigate this is to give a direction to all conditional votes which denotes the sign expected from a successful deal. The system can then continuously refresh the proposition graph by turning all conditional votes in their positive direction and then see if they fall back to the same state or if they settle on a new state, which will mean that some deadlock has been conquered. 
%
%And on a design-related note: The conditional votes can be visualized/rendered as leaves in the graphs, each one attached to a certain proposition node. The users can then create and add these `conditional vote nodes' to the system in the same way as proposition nodes are added, and then all users can decide to cast their vote for the given proposition either by casting it (unconditionally) on the proposition itself or casting it instead on one of the conditional vote nodes (meaning that their vote will now be automatically conditioned on some parameters of the voting statistics in the system, continuously, until they change their vote again). 
%
%%Another thing that a more advanced system might account for, is the fact that the power of the voters might not just have different weights but might also be dependent the area that a given proposition deals with. This could for instance be in a company or a government where there are many departments/ministries in charge of different areas. If such a company/government decided to go for a more democratic leadership, it might still want to keep some division of power within the democracy. This example is perhaps not the most realistic one so here is one that is more so:  
%
%
%Another very important thing that an advanced application should afford its users is to make sure that the voters can choose representatives. It might seem odd to want to implement a direct democracy only for people to end up choosing representatives once again, but is indeed exactly what a direct democracy should aim for. It is nowhere near feasible if the system requires all users to engage in all discussions and decision making in order for the democracy to work, not unless we have a simple case with relatively few voters who are all quite engaged. But in most cases that one could think of, being able to choose representatives and trust these with looking into specific and/or complicated matters and vote on the person's behalf is all-important. The problem with representative democracies that a direct democracy aims to fix should thus not be to get rid of representatives, but simply to ensure that people can change these much more rapidly should they want to, and also that any voter can always choose to look into specific matters themselves and choose to vote differently on those than how their representative has voted.
%
%An advanced e-democracy application should therefore also allow users to choose representatives. With `conditional votes' implemented, users can of course in principle just cast conditional votes on all propositions that they want representatives to decide for them, but this is too cumbersome and we can do much better than that. The application could thus first of all allow the voters to give their votes to others. But it is likely that some users will only trust a certain representative to decide for them in a certain area of concern. And in general, users will therefore probably want to be able to have multiple representatives at a time, each responsible for making decisions for the voters in specific areas. 
%
%I therefore propose that an advanced e-democracy application also implement what we could simply call `areas of concern,' which are then essentially groupings of propositions regarding a certain subject. Whenever a new proposition is added by itself (as what we could think of as the ``root'' of a connected graph) it should thus be given an area of concern such that the application can group it with proposition graphs with the same area of concern. And whenever a child node is added, it should of course get the same area of concern as its parent. With this implemented, users should then be able to give their votes to another user (i.e.\ representatives) when it comes to any specific area, which should then effectively mean that the user will automatically cast the same vote as their representative, at least when it comes to propositions that the user has not voted on themselves. (And one might then implement different settings to this, such that a user for instance might be able to even let a representative override the user's own previous votes.) 
%%(If someone creates an otherwise relevant proposition node but adds the wrong area of concern, one can expect that it will then simply not be voted forth (over the aforementioned threshold), not until the author gives it the right area of concern.)
%
%It now almost goes without saying that each `group' in the system can then potentially choose to have their own specific representatives that the members are recommended (or perhaps required in some special instances) to use. 
%
%The system might also implement `subareas,' such that any user can try to add one such to any proposition node. Users should then be able to vote such subareas in and out, and if one is voted in, the proposition node and all its children will then get this extra area, that users can then also choose to sign representatives to. With these `subareas' implemented, this then allows users to delegate different representatives to these, even if they are also part of the same overall area of concern. **(This paragraph will probably need some rephrasing.)
%%(One could also implement `subareas' simply by requiring that these are also added from the beginning when the relevant proposition nodes are created, but it might make it easier for the users if they can just change the subareas by vote at any later time (instead of having to recreate and substitute the whole subgraph, with similar nodes but with updated subareas).)
%
%
%%These `areas of concern' also allows for something else that might be very useful, namely that the same community of users/voters can govern a variety of bodies with the same overall (potentially disconnected) proposition graph. 
%%Because when the contracts and/or agreements concerning the various bodies' commitment to follow the proposition graph are external to the digital system, a body might as well agree to only be ruled ...
%%The usefulness in this, apart from maybe having everything gathered in one place, is that this will mean that voters
%%(16:06, 06.11.22) Hm, dette kan nok godt blive mere kompliceret, fordi flere foreninger så skal blive enige om, hvordan stemmerne fordeler sig, og så skal det lige pludseligt topstyres på en helt anden måde.. Så lad mig lige se en gang... ..Hm, men handler det så ikke bare om, at forskellige grupper skal kunne "genbruge" de samme propositionsgrafer, og også om at de andre gruppers aktivitet så også godt må kunne gøres synlig i samme propositionsgraf (altså for en vis gruppe, der bruger denne)..? (16:09) ...(16:29) Jo, men så har det så ikke rigtigt så meget med subareas at gøre.. ..Nej, for så skal man også simpelthen gøre, så at graferne.. er helt adskilte, ja, så måske giver det altså slet ikke mening.. hm, andet end at man stadigvæk kunne have graferne side om side, og mere vigtigt, at conditional votes så også kan komme til at afhænge af eksterne grafer.. Ja, er det ikke bare det..?:) 
%
%
%%Furthermore, each group should have its own page and/or own `area of concern,' where the members of that group have all the voting power. This is useful since it means that each group can then build their own proposition graph over its policies and opinions. A group might then also signal external actions via this proposition graph. For instance, a group the represents a workers union might create conditional votes in their own proposition graph that depends on some statistics regarding the main proposition graph.. 
%%say that: ``On these
%
%
%And lastly, it might be beneficial for various groups in society who govern different bodies, e.g.\ political parties, unions, organizations, companies, to also be able to negotiate with each other online and to view each other's policies. For instance, a company might want to say to a governing political party that: ``if you implement certain laws, we will move out company elsewhere.'' And a trade union might then say to a company: ``if you do not give us higher salaries, we will go on strike.'' These are thus examples where a group in society can use their power over one body (including simply themselves as a group) to negotiate concessions from another group with power over a different body. 
%So if the advanced e-democracy application really wants to afford its users with all that they could want for negotiating effectively with each others, it should allow different voter groups to come together in the same space. First of all, each `group' in a e-democracy should have their own proposition graph that only they have voting power over. This local proposition graph can then be used to signal the groups policies, opinions and potential actions. And when it comes to the `conditional votes' in this local proposition graph, these should also be allowed to depend on statistical parameters in the main proposition graph, outside of the local one. 
%And furthermore, different e-democracies (governing over different bodies) that uses this same digital application should also be able to invite another group to join together, such that the two e-democracies can have their proposition graphs shown side by side (but with the same distribution of voting power in each of these graphs), and more importantly, that one e-democracy can then start making conditional votes that depends on statistics regarding the other e-democracy and vice versa. 
%
%
%\section{An e-democracy party}
%
%There are of course a lot of different examples where an e-democracy such as this could be useful; political parties, companies, unions, organizations and other communities. %In this section, I will, however, only give some points when it comes to political parties and companies of the type that I described in the previous chapter.
%%If we ..
%And when it comes to political parties, there is the natural option that these are run only by their members. We could thus imagine two or more parties competing for power, each being run e-democratically by its members. But since politically parties are typically inclusive anyway, why not just strive to have one political party where every person in the society gets an equal vote? 
%
%I believe that such a party could gain massive support over time. It might start out as a small party, especially in the early days where people are still getting used to working with the proposition graphs, and when the technology is perhaps at an early stage. And then as the technology to matures and the userbase grows, more and more people would trust the new system enough that they would want to give their vote to this `e-democracy party.' That party might then, at least in multi-party systems, get some representatives in government and by that point, the interest in the party would grow further, since all registered users would then be able to have a say in the policies of those representatives. And if the technology works, more and more people would then see the potential in an e-democracy. This is especially true in countries where the people in general do not always feel heard by there politicians: When they then see that the resulting proposition graph for most people will fit their interest better than what the traditional political parties offer, they will want for the e-democracy party to be voted in as a ruling party. 
%
%Now, if the party thus lets member of society have an equal vote in it, this might then be problematic at this stage when the party want to take over from the traditional parties, since people might then be tempted to make their vote count twice, essentially, by voting for their favorite traditional party and then also using their vote in the e-democracy party. And since voting is anonymous, there is no real way that the e-democracy party stop this. That is, apart from taking steps to balance out this effect. The e-democracy party might thus choose to temporarily break its commitment to giving all people an equal vote in this phase, and instead promise that it will commit itself to try to counter all representatives in government that are not part of the e-democracy party by giving more votes internally to a group of representatives whom it deems are exactly at the mirrored end of the spectrum than the group of non-e-democracy representatives in government. (The chosen counter-group can, however, be much larger then the group of representatives it is supposed to counter.) This way, a voter who wants the e-democracy party to take over would not be tempted to cast their votes to a traditional party instead. It also means that once the party sits on a majority of the power in government, other representatives are more likely to join it while in power (if the relevant constitution permits such migrations of representatives while in power) if they see that the e-democracy is more practical since the e-democracy party can then just remove the appropriate amount of counterbalance as these former outsiders join. 
%
%
%E-democracies as governments of countries might thus be a much closer reality in the near future, than a lot of literature on the topic seems to suggest, at least in countries governed by a multi-party system. In two- or one-party systems, the development might of course be much slower. But then again, once some multi-party governments successfully switches to an e-democracy, the two- and one-party governments would then be able to analyze and copy the technology, at least giving them a much easier route to an e-democracy, should their voters want one. 
%
%
%
%\section{Anonymity}
%
%
%As mentioned, anonymity is often very important for a democracy, especially if we think about the case of governing a country. Therefore, the digital application should allow the users to vote anonymously. This can be achieved letting each user control an anonymous profile, but if information about which user has which anonymous profile is stored on a server, that server might be hacked. 
%
%So the question is, can an e-democracy system be as safe and anonymous as going to a box, drawing a cross in a field on a piece of paper and putting that paper into a box? Yes, actually: there are ways to ensure complete anonymity of the users where the anonymity is preserved even if the servers of the application is hacked.
%
%The following protocol allows a set of clients to each provide a server with a set of public keys such that each client knows the private key of exactly one of the keys in the set (and no one else but them knows this private key) but where no one knows which public key belongs to which client apart from the clients knowing their own key. The protocol is furthermore resistant to DoS attacks. 
%
%It works by having the clients take turns building blocks in a block chain, which we can think of as a `block spiral,' where the clients form a circle and where the turn to provide a new block to the spiral goes around in the circle. 
%
%\ldots\ \textit{Okay, jeg tror lige, jeg venter med at forklare om min idé her, for det kan godt være, at der findes en lidt nemmere måde. Det vil jeg lige tænke over. Men ellers er det en god idé, altså den hvor hver klient sender nogle nøgler videre til en tilfældig anden klient i kredsen (hvor hver blok krypteres med den næste klients offentlige nøgle (fra begyndelsen) og sendes til denne), og hvor klienter, der modtager nøgler gerne skal sende dem videre og slette dem fra hukommelsen. Herved vil man meget sjællendt kunne se, hvem var den oprindelige sender af en nøgle (medmindre både modtagerklienten og klienterne for og bag brugeren er ondsindede), og selv hvis den bliver sporet tilbage kan pågældende klient bare sige, at ``den nøgle kom fra en tidligere omgang og altså fra en helt anden bruger, men jeg har altså slettet data om, hvor den kom fra, som jeg burde.'' Men ja, jeg tænker nu lige lidt mere over det, inden jeg skriver denne sektion færdig. .\,.\,Jeg har i øvrigt også tænkt mig at sige, at man efter at have brugt denne protokol så bare kan bruge et VPN herfra, men hvis man vil være endnu mere sikker, så kan man endda bruge helt den samme protokol til at indsende data om, hvordan man vil stemme med sin profil, hvor man så altså bare erstatter de (tilfældigt) genererede nøgler i protokollen med tilfældigt genereret data samt det faktiske data, man vil indsende, og til sidst så offentliggør man så bare, hvilket skrald, man har sendt ind, men ikke den faktisk data, man så lader serveren beholde. (.\,.\,Så kan det dog godt være, at man skal ændre protokollen lidt, så man lige sørger for, at hver mængde data også vil nå slutningen af protokollen, så at ingen data-klumper bliver tabt i protokollen --- medmindre der altså er sket en synlig fejl i protokollen.)}
%
%\ldots\ \textit{Nej, der er vist en nemmere protokol, hvor man vist nok også kan finde frem til en DoS attacker. Man kan vist bare have et VPN, hvor klienterne sender beskeder frem og tilbage, og hvor de så kan pakke en nøgle ind i flere krypteringer med forskellige nøgler, hvor beskeden så skal sendes til alle de klienter i rækkefølge, som kan dekryptere beskeden en efter en. Og hvis så man gør det tilfældigt, hvor mange krypteringer, der skal til, så kan ingen igen vide, om en nøgle kom fra en person, bare fordi de får opsnappet, at beskeden på et tidspunkt blev sendt fra denne, for vedkommende kunne jo sagtens have fået den fra andre og så bare have sendt den videre. Og hvis man så har nogle få DoS'ere i netværket, så kan brugere der har sendt en nøgle der aldrig nåede frem jo pege på, hvem der kan have været de skyldige (af den række af brugere).\,. Hm, ja, men hvis man nu vil bevise det også, så kunne disse brugere.\,. Nå nej, man kan ikke bevise det på et VPN, men det gør vist heller ikke noget. For brugere skal jo stadig gerne sende flere nøgler pr.\ protokol, og hvor de så bare opsiger alle på nær én til sidst. Og hvis der så er en DoS'er i netværket, jamen da det ikke vil være fatalt, så må det være fint nok, at brugerne bare kan page dem ud nogenlunde. (Og hvis det så bliver et større problem, så kan man altid bare bruge den mere krævende blok-spiral-protokol, jeg har haft tænkt på.)} %(08.11.22, 10:27)
%
%
%
%
%
%
%
%%Hm, jeg har fået tænkt lidt over anonymitet, men det kan godt være, at jeg lige skal tænke lidt mere. Men jeg har altså fundet på nogen fine systemer til at skjule stemmeres identitet, og jeg tænker, at stemmere generelt skal kunne vælge enhver tredjepartsbruger til at videreformidle deres stemme anonymt. Sådanne kunne så med fordel få lov at give floating point værdier (i stedet for bare 0 eller 1) med deres stemmer, eller de kunne bare råde over et antal stemmer, således at de både kan give et antal positive og et antal negative stemmer til hver proposition (men jeg tænker at det første næsten er nemmest..). Og ja, så kunne én form for sådan en tredjepart så være en organisation med fysiske lokationer, hvor medlemmerne så kan møde op personligt og ændre deres stemmer og/eller repræsentanter, og hvorved organisationen kan opdatere deres stemmer herefter med en frekvans, der kan afhænge af, hvor mange ændrer deres stemmer ad gangen over en gennemsnitlig periode. Og en anden, meget smartere;), måde at have en videreformidlingsrepræsentant på, kunne så også være.. hm, lad mig lige se.. (13:50) ...(14:30) (ordner også vask) Jo, man kan også have en videreformidlingsrepresentant, der fungerer via mindst to tredjeparter, som klienten selv kan vælge. Først er der en trejdpart, eller instans bør vi nok hellere kalde det, bare.. som via asymmetrisk krypering får en nøgle fra hver bruger, som kun denne instans og hver enkelt relevant bruger må kende. ..Ja, eller på nær at de også så skal sende alle disse nøgler til en anden instans, der heller ikke må offentliggøre dem, og som så i øvrigt ikke ved hvor hver enkelt nøgle stammer fra (og må ikke få dette af vide af første instans). ..Hm, vent, giver dette mening..? ..Ah, jo, jeg kan få det til at give mening, men lad mig nu lige se.. (14:36) ..Hm jo, denne instans nr. 2 kan så også få en offentlig nøgle med fra brugeren til hver enkelt nøgle af første instans, sådan at denne altså bare får et sæt af nøgle par, hvor den ene er en offentlig nøgle. Denne instans kan så kryptere.. Hov, nej, så behøver vi faktisk ikke den første nøgle; instans nr. 1 sender altså bare et sæt af offentlige nøgler videre (gennem en krypteret kanal) til instans nr. 2. Denne offentliggør aldrig disse, men bruger dem hver især til at kryptere en besked med en ny nøgle i, og offentliggør alle disse krypterede beskeder. Brugerne prøver så at dekryptere dem hver især, indtil de finder deres egen.. Hm, er dette får ressourcekrævende, eller skal denne instans også lige tilknytte et meget lille hash a hver offentlig nøgle med beskeden, så hver bruger ikke skal igennem så mange..? ..Det kunne man sige.. ..although.. ..Tjo, men brugerne kan så stadig downloade alle beskeder i rækkefølge og så bare nøjes med at beholde dem, de skal tjekke.. Hm, lad mig lige tænke, om ikke der er en smartere løsning.. ..Hm, men ellers var pointen så, at enhver bruger, som ikke får en passende besked, bør så anråbe dette, hvorefter alle nøgler så skal indgives, sådan så man kan finde ud af, hvilken part var synderen (inkl. anråberen, hvis dette var en fejl), hvorefter man så kan starte forfra, muligvis uden synderen. Men når hver bruger så har fået en ny nøgle, som kan kan spores hen til dem, hvis alle de involverede instanser (for man kan godt have flere nr.-2-instanser her) bryder deres løfter og offentliggør deres data (og ikke bare sletter det kort tid efter). Nu kan man så være sikker på, at alle brugere i gruppen har netop én anonym nøgle, som nu kan bruges til at oprette en anonym bruger profil for hver bruger, selvfølgelig med VPNs involveret, hvormed denne frit kan afgive sine stemmer og ændre dem, hvornår det skal være, uden at det kan spores tilbage til dem. (14:52) .. ..Og disse anonyme brugere kan så udløbe således at de skal opdateres en gang imellem, således at hvis nu nogen for lækket deres bruger, så vil det allerhøjest kun være den seneste aktivitet, der bliver lækket (og derudover kan man selvfølgelig også dele brugeren op i flere (der ikke kan kædes sammen af andre), hvis man synes, der er besværet værd, men ja, og sådan vil der selvfølgelig altid være ting, man kan tilføje, hvis man finder frem til, at det giver mening..). Nå, men selv hvis der findes et bedre system end dette, så kan jeg jo bare skrive, at det f.eks. ikke er svært at finde på systemer, hvor man via flere instanser, der hver især holder på sin del af en samlet hemmelighed (hvor alle stykker skal bruges, hvis man vil spore tilbage), kan opnå at hver bruger i en gruppe får netop én anonym bruger. Og ja, hvis man så sørger for at de udløber med jævne mellemrum.. Og at brugerne skifter.. Hm.. ..Hm, men det er nu ikke perfekt anonymitet, hvis man sammenligner med valg, hvor ingen data bliver gemt til at starte med, således at ingen nogensinde kan spore det tilbage.. Hm.. ..Hm, men kunne man ikke bare bruge en teknik, som jeg vist også har tænkt på før, hvor en instans bare offentliggør en mængde af.. Hm.. ..Hm jo, en mængde af dens egne offentlige nøgler, nemlig med et antal svarende til antallet af klient-deltagere i øvelsen, og hvor hver klient så vælger et hemmeligt ID, krypterer.. Hm, nej, lad mig lige se... ..Hm, hvad med at alle klienter bare opretter et VPN kun med demselv som noder, og så begynder at sende data rundt. På et tilfældigt tidspunkt sender hver bruger så et ID videre til en naboknude, som modtager, sender ID'et videre til én naboknude, og noterer også ID'et og modtagelsestidspunktet.. nej.. Hm, dette virker vist næsten, men ikke helt.. ..(15:21) Ah, nu har jeg det måske. Man kunne lave en kæde af krypterede blokke, hvor hver blok offentligt hører til en klient, og hver blok rummer data, som brugeren fik tilsendt af ejeren af den tidligere blok, og data som brugeren har sendt videre til næste klient. Denne blokkæde kører så på omgang i en ring, således at den tager flere runder. Og på et tilfældigt tidspunkt tilføjer hver bruger så et offentlig nøgle, som de sender videre. ..Hm, nej det er endnu ikke helt vandtæt.. ..Ah, men måske hvis man tilføjer sin nøgle i krypteret tilstand, så den først kan lukkes op, når den når til en (tilfældigt udvalgt) anden bruger.. Hm, spændende idé.. (15:27) ..Ja, man må næsten kunne lave sådan et system, hvor klienterne billedligt talt danner sådan en rundkreds, hvisker data videre til hinanden én ad gangen i rundkredsen, og hvor klienter i kredsen så kan kryptere en hemmelighed, som en klient et andet sted så kan forstå. Denne bør så med det samme kryptere en ny besked, hvormed denne hemmelighed kastes videre til en anden person. Hemmeligheden er så en offentlig krypteringsnøgle. Man slutter så, efter et vist tidspunkt, når man er næsten 100 \% sikker på, at alle brugere for længst vil have kastet deres nøgle ind i rundkredsen, og at denne er læst af modtageren. Alle brugere offentliggør så de nøgler, der har været sendt frem til dem. Herefter skal alle brugere/klienter (jeg kan ikke lade være med at skrive "brugere" i stedet for klienter, men det er vel også næsten ligeså godt..) så sige, om deres nøgle er iblandt de offentliggjorte (men selvfølgelig ikke udpege dem). Hvis antallet af nøgler passer og alle brugere/klienter siger, at deres er med i mængden, så stopper "legen" succesfuldt. ..Eller rettere, det gør den, efter at man så beder alle brugere om at slette de nøgler, de ville have brugt til at dekryptere deres egne blokke med. Og sikkerheden i systemet handler så om, at man har tillid til, at størstedelen af klienter vil gøre dette (selvfølgelig fordi de bare bruger det udleverede software til det, og ikke har bygget eller tilegnet sig en malicious kopi af denne software).. Men hvis der er for mange nølger, eller at en klient mangler en nøgle, jamen så må man så bede alle brugere om at dekryptere alle deres blokke. Og så er pointen, at man kun ved at have alle disse blokke dekrypteret, kan finde frem til, hvem der er synderen, fordi man så både vil kunne se, hvis de ikke har opfundet netop én nøgle selv, og fordi man kan se, hvis de ikke har videresendt den rette nøgle hver gang.. Nå ja, og hver bruger skal så også bare i det hele taget indsende deres private nølger, så man kan finde frem til synderen. Og hvis enten en klient nægter at indsende den private nølge i dette tilfælde, eller hvis man finder synderen ved at dekryptere alle nøglerne, så må man så udelukke denne bruger i næste tur (altså give denne karantæne). Men ja, som sagt, hvis legen derimod ender succesfuldt, så skal brugerne endeligt ikke offentliggøre deres private nøgler, nej faktisk skal de slette alle deres nøgler, der blev brugt under selve legen og kun beholde den private nøgle, som passer til den offentlige nøgle, de herved fik indsendt anonymt via legen. Og ja, så længe de fleste brugere bare gør dette, så er man ikke i fare for, at det bliver afsløret, hvilke nøgler i slutmængden hører til hvilke klienter. :) (15:53) ..(16:02) Hm, der er faktisk en lille smule hangman's paradox tilstede i denne løsning, men det kan man vist gøre bod på ved bare at sige, at hver knude.. Hm.. ..Hm, eller hvad i stedet med bare at gøre sådan, at klienter i kredsen generelt skal vente et tilfældigt antal omgange, inden.. hm, men det løser dog ikke problemet eksakt.. (16:07) ..Hm, men jo, man kunne vel også bare sørge for, at sandsynligheden for at ens software sender en nøgle starter virkeligt lille og kun vokser over mange runder, og så kunne man gøre sådan, at hvis en bruger bagefter kan se, at deres software har sendt.. Hov, vent, dette er da slet ikke et problem, netop fordi man kaster hemmeligheden frem i rækken.. hm.. ..Hm, der skal kun tre (specifikke) andre brugere til at afsløre en i denne løsning, men de kan det kun hvis man har været uheldig at softwaren har sendt ens nøgle tidligt.. ..Hm, man kunne også bare give hver bruger mulighed for at afbryde legen, hvis deres sofware har sendt deres nøgle tilstrækkeligt tidligt.. Hm.. ..Hm, i øvrigt kan man hurtiggøre processen, hvis kredesn har mange kæder i gang på én gang, så alle klienter kan bygge en blok i hver runde (nemlig hvis der er ligeså mange kæder i gang, osm der er klienter i kredsen).. ..Hm, men kan man ikke bare generere flere nølger, end der er behov for..? (16:18) ..Jo, og så kan brugerne/klienterne til sidst bare vælge, hvilken nøgle af dem, de har fået genereret i legen, de vil beholde, ved at.. Hm.. ..Ah, ved selvfølgelig bare at bekende offentligt bagefter, at "disse nølger var mine, men jeg skal ikke bruge dem alligevel."!:) Og hvis så der lige præcis bliver det samme antal efterfølgende, som der er klienter, og hvis alle meddeler, at de har en nøgle iblandt de endelige, så når man i mål, og ellers må man så bare til at optrævle kæden, for at finde DoS-synderen, hvis ikke legen ender som den burde. :) (16:25) Og ja, det skal så bare anbefales, at hver bruger ikke vælger en nøgle, der blev genereret helt i starten af systemet, men ved at det stadig er brugerens beslutning at udvælge den ønskede nøgle, så eliminerer man altså hangman's paradox.:) (16:26) ..Nå ja, og lad mig lige præcisere, at hver blok så skal indeholde en liste af krypterede nøgler (som hver er kryperet med en tilfældig andens offentlige nøgle), og denne lister vokser altså bare.. tja, eller man kan måske begynde at fjerne ting fra bunden af listen efter et vist stykke tid, når det er sikkert, at samme nøgle er blevet indsat igen i ny version (nemlig ved at en knude har dekrypteret og re-krypteret nøglen og sat den på). Og man kan så kræve, at hver knude tilføjer netop én ting til listen i hver runde.. how, "runde" er et dårligt term at bruge for hvert enkelt lille step, når vi har en rundkreds, så lad os kalde.. tja, lad os bare kalde det enten hver 'step'/'skridt' eller hver tur.. nej, lad os udelukkende kalde det 'skridt'/'step.' Og hvis en bruger så modtager flere beskeder på én gang i et step.. ..Hm, nej vi kan også godt kalde det turn i stedet (for så tænker man jo bare på et lille turn af hjulet).. Så må denne bruger så altså gerne vente en omgang med at sende nummer 2 besked (osv., hvis der modtages flere end to), og altså så kun videresende én af nøglerne i den første tur, hvor nøglerne modtages. Ok, så det var vist bare det, jeg lige skulle præcisere.. (16:40)
%
%%(16:42) Nå, men der er også et andet issue, jeg skal tænke over, og det handler om: Vil det ikke være for fristende for folk at stemme på deres vante repræsentanter i et regeringsvalg, hvor et e-demokrati kæmper, og ser ud til at kunne vinde? For hvis man gør dette, så vil man vel kunne få dobbelt magt, medmindre e-demokratiet kan se, hvem der ikke stemte på det.. hvad de jo ikke vil kunne.. Hm, måske er dette et ret stort problem, men ja, nu vil jeg altså give mig til at tænke godt over det... (16:44)
%
%%(31.10.22, 9:21) Kort efter, jeg klappede i i går kom jeg frem til, hvad vist også havde været oppe at vende i periferien af mine tanker tidligere på dagen, at den simple og måske eneste løsning nok bare er, at sørge for, at e-demokrati-partiet i starten også har til opgave at booste stemmevægte inde i systemet (på en helt transparant måde selvfølgelig), således, at alle repræsentanter, der ellers har fået mandater udover partiet, de får en modvægt til sig inde i partiet. På den måde kan det ikke betale sig at stemme uden for partiet for at pågældende mening skal få mere magt, for så vil den pågældende mening bare blive countered. Og ja, det er så partiets opgave at finde frem til og være ærlig omkring, hvad der er midten af det politiske spektrum i henhold til forskellige punkter, således at man kan counter'e et vist mandat ved at give mere magt til en (eller flere) fra den modsatte (i.e. spejlede) ende af spektrummet. Når partiet så er i regering, så kan man så også bede de repræsentanter, der ikke er med, om at joine, for så vil e-demokratiet bare fjerne magt igen fra dem, der står for at counter'e/udbalancere magtbalancen.. (9:29)
%
%
%%\section{A note on transparency}
%
%
%\section{E-democracies in companies}
%
%%To finish this chapter, let me just make a small point about how an e-democracy application like this might also be incredibly useful when it comes to democratically run companies, or indeed the almost-democratically run `Economically Sustainable' Companies (ESCs.\,. hm, that looks a lot like `Escape(s)'.\,.) that was described in Chapter \ref{MSE}. 
%%
%%If the company in question has a goal of expansion, such as should be the case for the 
%
%To finish this chapter, let me just make a small point about how an e-democracy application like this might also be incredibly useful when it comes to democratically run companies, or indeed the almost-democratically run `economically sustainable' companies that was described in Chapter \ref{MSE}. 
%
%If the company in question has a goal of expansion, such as should be the case in general for the `economically sustainable' companies as described, I envision that this venture will be all the more exciting for the participants if there is a vibrant online community that engages in discussing and finding what strategies to go ahead and try in order to expand the company. 
%
%And if this e-democracy application can be as useful a tool for this as I believe it can, it could thus accelerate the interest in taking part and supporting such a company immensely. 
%
%%Hm, skal jeg så bare stoppe her for nu? (Eller skal jeg skrive videre på denne sektion, og var der i øvrigt andet, jeg har glemt at nævne..?) (15:21) ..Hm, jeg har glemt at nævne min pointe omkring gennemsigtighed ved at sørge for, at folk med jævne mellemrum bliver udtaget til at sætte sig ind i detaljerne og så rapportere tilbage til den interessegruppe, der udvalgte vedkommende, men måske jeg bare skal gemme denne pointe til en anden gang.. (15:23)
%%...(16:01) Nej, jeg tror ikke, jeg behøver at tilføje mere nu. Når jeg så lige får tænkt lidt mere over spiral-protokollen, så kan jeg skrive om den, og ellers er det nok bare lige at redigere teksten. (16:02)
%
%
%
%%Husk:
%	%Jeg havde tænkt mig her at nævne det med, at det kan være smart at udvælge nogen (som så jo kan vælges til at være upartisk og/eller repræsentativ (men måske smart/intelligent nok)) fra en gruppe til at studere og gennemgå systemet i nærmere detaljer og så rapportere tilbage..
%	%Transparancy.
%	%
%	%You never have to waste a vote (and never have to fear wasting a vote). And never have to be fearful, that who you voted for does something you didn't expect (since this system requires no trust in representatives, at least not except in cases why you don't feel like you have the time (or interest) to go through the details of a matter).
%	%..(16:47, 29.10.22) Hm, og husk det her med at man kan have flere områder, hvor forskellige bestemmer, og at dette så også gør, at andre grupper kan logge sig på i systemet, hvor vi snakker om at styre et land. I et sådant e-demokrati kan grupper altså også tilføje områder. På den måde kan de gøre det offentligt for alle, hvad de har tænkt sig. Hermed kan vi altså få en stor markedsplads, der handler om at lave aftaler og bestemmelser, både i regeringen, men også i andre instanser (det kunne f.eks. være såsom fagforeninger, hvilket jo vil være meget relevant i den sammenhæng). ..Og ja, det kan også være grupper, der egentligt ikke har nogen anden magt over noget, men som alligevel vil oprette et område, der hedder "vi mener sådan og sådan, og vil vil gøre sådan og sådan," altså et område, hvor de kan signalerer til omverdnen, hvad deres interesser er, og hvad de gerne vil / er parate til at gøre. (16:54) ..Og ja, det kan så nævnes, at dette så også kan være sådan noget som at trække sig fra den overordnede gruppe (f.eks. e-regeringspartiet eller trække sig som kunde og/eller investor i et firma). 
%	%Jeg kunne godt nævne muligheden i "forbrugerforeninger" kort også (som et eksempel på anden form for magt), men så tilføje, at min kd.v.-idé så netop nok ville være endnu bedre her, for så kan man undgå sådanne reprimanter (eller hvordan det staves). Men om ikke andet kunne det så blive en måde at tvinge gang i en kd.v., hvis nu virksomhederne indenfor en branche er tøvende med det. 
%	%Jeg skal forresten huske at have område-repræsentanter med under avancerede punkter, sammen selvfølgelig med områderne selv. Jeg kan således nok godt nævne "områderne" først, også selvom det egentligt er vigtigere, det med at kunne vælge repræsentanter.. 
%	%"Det handler om at det bliver: meget lettere at samle sig i små grupper, og meget lettere at sætte i gang i en proces, hvor man overvejer, om ikke der kan gøres noget ved et forhold, netop fordi man bare kan starte denne diskussion i nogle små grupper (som så kan kontakte andre grupper, små eller store, når de har fået samlet en oversigt over, hvad problemet er, og hvad man kunne gøre for at løse det m.m.). Så altså langt større tilgængelighed for den enkelte og dermed mange mange flere mennesker aktiveret ad gangen (som så overvejer og finder på løsningsforslag til problemer i samfundet (ofte særlige problemer for nogen specifikke i samfundet, men det kan jo også være mere almene)). Og så vil der så derefter også kunne være meget kortere tid til, fra løsningsforslag til løsning i sådan et direkte demokrati, der er klart. Og ikke mindst vil folk (i grupper) få langt nemmere mulighed for at indgå selv komplicerede politiske aftaler med andre folk (i grupper (ikke nødvendigvis disjunkte med de første, btw)), således at man får et meget bedre og hurtigere kan få handlet sig til at få opfyldt sine behov som en gruppe af mennesker, og således at smafundet derfor vil blivet meget bedre fintunet, så at sige, til at opfylde så mange menneskers forskellige behov som muligt på en gang."
%	%(15:01, 01.11.22) Jeg skal huske noget, jeg lige fik tænkt på, og det er, at et sådant demokrati kan få en meget meget fladere struktur, hvor at man, når man har en ny idé til forandring, lad os sige som lille gruppe, i stedet for så at skulle indsende og ansøge om idéen til en central, så kunne man i første omgang dele den, med den/de mest relevante nabogruppe(r). Hvis de så også er med på den, så kunne man så brede det til endnu flere. Og når idéen så har samlet nok opbakning, så kan man melde det til det brede fællesskab, hvor idéen så allerede har opbakning, når den ansøges om. Jeg ved godt, at sådanne måder at fremføre idéer på allerede finder sted mange steder, men jeg tror, at man i et e-demokrati kunne gøre den fremgangsmåde endnu nemmere og endnu mere hyppig.. Hm, måske vil jeg skrive om dette, men om ikke andet er det da bare rart at tænke på, at der kunne blive sådan en rigtig flad struktur, hvor relaterede grupper selvstændigt kan diskutere og handle om, hvilke idéer og forslag, man vil gå videre med..:).. (15:08)
%	%Man kan også bruge min blok-spral-idé til når stemmerne skal kastes..!
%
%
%
%
%
%
%%(09.11.22, 9:46):
%\section{(I'm considering adding something like:) A similar application for scientific discussion}
%
%\textit{I have now realized that this application could also be used for scientific discussion graphs, which goes hand in hand with decision making since facts are of course important when deciding policies. In a discussion graph, on would just not really need the `conditional node' edges, but would instead just use the `conditional votes' instead --- which could then be drawn as edges between notes for this type of application. %(This all of a sudden make this idea quite a bit more interesting for me in terms of what I would like to work on myself.\,. .\,.\,Hm, hvilket er relevant for mig at have i tankerne i denne stund, for jeg skal nemlig snart til jobsøgeningsmøde med A-kassen. Og ja, med denne indsigt, så må det da næsten være denne idé, jeg vil prøve at gå videre med (og sige jeg vil iværksætte), det tænker jeg.. (..Altså i stedet for Web 2.0--3.0-idéen/erne.))
%*And it should then be very much recommended (as a key part of the idea), that users try to commit themselves to continuously update their votes for propositions as conditional ones, once more fundamental propositions are added to the system. A scientist might for instance be an expert on drugs and say (or actually ``vote'') that: ``this drug is so and so addictive,'' but then once propositions are added about the existence of relevant studies are added, as well as propositions about trust, then that scientist (along with everyone) are then strongly recommended to change the vote into a conditional vote such that the vote now depends on the study existence proposition and the trust proposition. This way (if the community follows this (strong) recommendation), every proposition can slowly become more and more founded in the basis empirical propositions/data, plus trust propositions (which are essentially propositions about how the users want to apply epistemology, i.e.\ when these propositions are also boiled down to their roots). This both has the advantage of the system being more flexible, when new studies turn up or if old ones come into question at some point, and also, importantly, it makes it easier to browse and find out what fundamental facts our more abstract facts in society are built on, i.e.\ to find the sources, and it also gives a better and easier understanding of what is interesting to research, since it shows were the ``gaps'' are, so to speak, or more precisely: where the research is thin and could use bolstering. 
%}
%
%
















\end{document}