\documentclass{article}
\usepackage[utf8]{inputenc}


%\usepackage{amsmath}

%\usepackage[toc, page]{appendix}
%\usepackage[nottoc, numbib]{tocbibind}
\usepackage[nottoc]{tocbibind}
%\usepackage[bookmarks=true]{hyperref}
\usepackage[numbered]{bookmark}

\hypersetup{
	colorlinks	= true,
	urlcolor	= blue,
	linkcolor	= black,
	citecolor	= black
}


\title{
	Consumer unions as a way of combating capitalism
	\author{Mads J.\ Damgaard%
		%\footnote{
		%	See https://www.github.com/mjdamgaard/notes for potential updates, additional points, and other work.
		%}
		%\footnote{
		%	B.Sc.\ at the Niels Bohr Institute, University of Copenhagen.
		%	B.Sc.\ at the Department of Computer Science, University of Copenhagen.
		%	E-mail: fxn318@alumni.ku.dk.
		%	GitHub folder: https://www.github.com/mjdamgaard/notes.
		%}
	}
}

\usepackage[margin=1.8in]{geometry}

\begin{document}
\maketitle

%\begin{abstract}
%	
%\end{abstract}

\section{Introduction}
%{\centering\noindent
%	\vspace{-\baselineskip}
%	\hspace{-0.7em}
%	{\hspace{-4.em}$|$\hspace{\linewidth}\hspace{8em}$|$}
%}
This is a brief whitepaper introducing some consumer-oriented ideas for combating capitalism. More specifically, these ideas regard a new type of consumer union in which consumers use their plans for future consumption as leverage against companies in order to make them reduce their profits and instead spend that money e.g.\ on lowering prices, giving workers a higher salary or better working conditions, or on reducing pollution, etc. 

Here and in the following text, `capitalism' is loosely taken to mean a system which is `governed by the interests of money', following the same terminology as Wright \cite{Wright}. It is thus \emph{not} taken to be synonymous with just a `free market system,' and in fact, the ideas introduced in this paper are all supposed to function within a free market system. 

The scope of this whitepaper is quite limited and does not include any in-depth analysis of the ideas introduced. The paper instead simply aims to introduce the overall concept of the ideas and give just a few arguments for why they might merit further research. 

%The ideas introduced in this brief whitepaper include some ideas (rewrite)** regarding certain consumer unions which put pressure on companies to give them and/or the workers of the companies greater power within them and a greater share of the profits generated by the companies. The purpose of these unions is thus a lot similar to that of conventional labor unions, but instead of withholding their labor as a way of putting pressure on companies, the members of these consumer unions can withhold their consumption as a way to do the same. 



\section{Motivation}

One of the major concerns with capitalism as it functions today is that whenever workers generate value in the economy, the owners of the means of production will be able to take a part of this value as profit. The owners of the means of production will thus be able to increase their wealth on the back of other people's work. Apart from the fact that might seem unfair on the microscopic level, the big concern about this system is that it seems to lead to an increasing wealth inequality in society on a macroscopic level. 

There are several ways in which individuals in a society can try to combat capitalism. First of all, they can vote for political parties who want to enact anticapitalistic policies. Second, they can be selective about what job offers they take, which in practice often implies joining and agreeing to the terms of a labor union.
Third, they can be selective about where they buy their goods and services as consumers. 
And fourth, they can be selective about where they invest their money, e.g.\ in terms of what pension funds they choose. %, but this topic will not be expanded upon in this whitepaper.

In the past there has been a lot of focus on the first two ways combating capitalism, i.e.\ the political struggle and the struggle of workers to demand better conditions for themselves and their peers. But perhaps the last two ways mentioned here have been slightly overlooked so far, and in particular the third way of being selective as consumers, which is the topic of this whitepaper.

Indeed, if we think about the term `consumer union,' it is commonly associated with an organization that fights for transparency and consumer safety. These are important things to fight for, of course, but how about another type of consumer union that actually aims to be for consumers what a labor union is for workers? In other words, how about a union that uses its members' future consumer habits as leverage in order to put pressure on companies to lower their profits and spend that money on offering better deals for their workers and/or their costumers instead? 


%It is my view that such a strategy could turn out to have great potential in terms of combating the rising wealth inequality. %in the future to come. 
%For even if only a small portion of the workers in a society could come together in such a consumer union, say $10\, \%$ or so, they would still constitute a lot of potential revenue for the various companies, and would thus still have real negotiating power. 
%And why would the workers not want to come together this way, since it only increases their negotiating power in relation to their employers, even if just by a little? ...


%When Marx bla bla... This theory defines the value of a good or a service as the amount of money that the consumers are willing to pay for it. But ... bla bla bla, an equally valid model is to look it bla bla ... consumers are getting less than for what they pay. ...


It does seem that there could be some potential in this idea, and that consumers might be able to gain quite a lot of power by unionizing and using their future consumer habits as leverage in negotiations with companies. If we for instance consider the part of the tech industry that gains revenue from collecting and selling their users' data, the value of which is mainly due to how it can be used to predict future consumer habits, the fact that so much money is involved in that business paints a picture of how valuable it is for companies even to just \emph{predict} these habits. And if we then go on to consider the enormous amount of money and effort that companies spend on public relations and advertising, it shows how valuable it is for the companies to try to influence consumer habits. So if consumers are able to band together and somehow use their future consumer habits as a kind of trading commodity and/or bargaining tool, they might be able to gain a considerable amount of money and/or power thereby. 




\section[Possible goals]{Possible goals of these consumer unions}


Consumer unions who want to use their members' future consumer habits as leverage can do so for several possible reasons. 
One possible goal of such a consumer union could simply be to support certain labor unions, simply providing a different way that the workers of these labor unions can get leverage over their employers. One could thus imagine group of labor unions who are sympathetic to each other's causes coming together and founding a consumer union for all their combined workers to join (as well as any other individuals who are sympathetic to the same causes). This then makes it possible for these workers to start applying pressure on their employers from two sides in order to get the concessions that they want, both as employees threatening to go on strike and as consumers threatening to boycott the companies. 


Another aim of such a consumer union could be to negotiate lower prices of goods and services for consumers. So instead of persuading companies to reduce their profits in order to spend that money on better conditions for their workers, the union might instead aim to get them to spend that money on lowering the prices for their customers. The potential advantage of this is that it might make it easier to attract members to the union, since it would make the union rely less on solidarity between trades and more on the direct desires of the members to get a market of goods and services that is more favorable to the average consumer. 

Such consumer unions that focus more on lowering prices for consumers rather than raising the pay of workers also have to then consider if the aim should be to get the companies to lower their prices in general, or if they should instead strive to get desirable deals specifically for their own members when they can. The latter option might require a higher level of organization for the union, as we will briefly discuss in the next section, but the advantage might be that this would make it easier for the union to attract a large number of members, simply because these would miss the good deals otherwise. And the more members a consumer union has, the more power it has to get make companies change their directions. 


Consumer unions of this kind also do not have to focus solely on getting economic benefits. They can strive for other political goals as well, such as better working conditions, less pollution, and so on. 
Now, it might happen that some consumers come together to form a union with some specific goals in mind, e.g.\ such as not allowing companies to make too big profits on their behalves, but then later find themselves disagreeing on what other political goals the union should strive for. If the union is relatively small, a simple solution could then just be to make it a democratic decision what other goals to strive for. But when the union is large enough, it could be more beneficial for the members to agree to divide the union up into separate parts. These could then go on to strive for individual goals whenever they disagree, but still work together when it comes to the issues on which they agree.



%\section{Organizing members to be ready to change their consumer habits as leverage}
\section[Possible approaches]{Possible approaches of these consumer unions}


Similarly to a labor union, a consumer union of this kind could have a democratically elected leadership, whose task it is to first of all analyze what actions to take as a union in order best to strive for its goals. Such actions would often include negotiations with companies, and the leadership could also be responsible for these. As leverage in these negotiations, the leadership should then be able to mobilize the members of the union and persuade or require them to change their consumer habits. Thus, if the decision is to put pressure on a specific company in order to get certain concessions, the union leadership will likely ask or require all its members to reduce their demand for the goods and services of that company, possibly by changing their demand to other companies instead. %These other companies then do not have to be any better than then company in question.. nok ikke vigtigt nok at sige her..

In a simple version of such a consumer union, the union leadership can simply ask its members to do so, and not do anything further to monitor whether the individual members follow these instructions/recommendations or not, other than to monitor the general sales of the company in question to see how much they decrease as a consequence. If the boycott then seems to not be as strictly followed by the members as expected, the union leadership can then simply try to make further communications to the members in order to persuade them to live more up to the commitment that they agreed to when signing on to the union. 

This simple version might be preferable whenever the goals of the consumer union are somewhat altruistic in their nature, for instance if the aim is to support labor unions, or to make companies reduce their prices for all their customers, not just for the members of the union itself. In these cases, it would not make sense for the unions to try to make the agreements of the members specifically binding. For if the goals are altruistic, the members should only have to commit to the cause exactly as much as they want to. 

However, if the strategy of the consumer union includes striving for better deals specifically for the members of the union, 
it would then be natural to take steps to ensure that the members fulfill their parts of the agreement. Otherwise everyone would just join the union regardless of how committed they are to changing their consumer habits.


One potential solution for this is for the union to set up some sort of group purchasing organization (GPO), owned and run democratically by the members of the union. The clients of this GPO are then also the members of the union themselves, which means that each member can choose to spend part of their budget for consumption through this GPO, and if they do so, the GPO can use the collective buying power of the committed members as leverage in order to get discounts for them. 

This GPO could for instance function much like on online store, with a catalog over products and services from various companies. If any given company does not agree to give the certain discounts that the GPO requires of it, the GPO can then threaten to remove that company's products and services from its catalog. Assuming that the members of the union are generally committed to spending part of their consumption budget through the GPO, removing a company's products and services from its catalog would thus lead to a decreased demand for these overall, which gives the GPO leverage over the companies. And even though there might be an initial phase for each negotiation where the GPO has not yet obtained any discounts, as long as the members stays committed to spending money through the GPO, it will likely be able to eventually use the leverage that this commitment gives it in order to get some discounts for the members.

In order to be able to attract more commitment from the members in these initial phases of negotiation where the GPO has not yet achieved the desired discounts for a certain type of product or service, the GPO might implement some sort of system where the members can earn points whenever the spend money through the GPO. When the GPO then finally achieves the discounts that it seeks, the members might then get individual discounts based on how many points they have earned. This could serve to give an extra incentive for the members to commit themselves to the GPO even before it achieves the discounts that it seeks. 


%With this solution, the economic advantage for the members thus lies in the money they save on their consumption, which means that the gains for each member will be proportional to how much money they commit to the GPO as part of their monthly budget for consumption. ...






%\subsection{Prioritizing democratic companies}
%
%When a consumer union wants to put pressure on a certain company by instructing/recommending its members to limit their consumption at that company, it might need to point the members towards some competitors instead so that these do not have to reduce their consumption overall of the goods and services that the company offers. The union might then simply point the members towards similar competitors, not because these are better alternatives, but simply as part of the negotiating strategy. However, there exist ... ...Hm, måske bedre at gemme dette til en ny sektion efter SRC-sektionen..




\section[The potential of these unions]{The potential of these consumer unions}

We can obviously not conclude anything exact about the potential that these consumer unions might have in this short whitepaper. The ideas requires further research before any such conclusions can be made. But we can at least %conclude the whitepaper by looking 
look at a few arguments for why it is possible that they could have quite a big potential.

As mentioned above, changes in future consumer habits can have a big impact on companies, hence the large amount money spent on public relations and advertising. A significant part of a company's market value is thus often quite dependent on the (predicted) future consumer habits. This means that the consumers in principle have the power to make financial assets decrease or increase in value, and this power could in principle be used either for political or monetary ends. Of course, the consumers of a society include all people in it. One can therefore not expect that they will all agree to take the same actions. But if the consumers organize themselves in unions of peers who all share some of the same interests and agree on some of the same issues, the unions might then be able to utilize some of this potential power. 

If we then look at the simple (altruistic) versions of these consumer unions where the union leadership simply encourages its members to make certain changes in their consumer habits, but does nothing towards being able to monitor the habits of the individual members, the effectiveness of this approach will obviously depend on how committed the members can stay when there are no consequences for breaking their commitment. In other words, the union might be successful in attracting a lot of members, but if these members do not really follow the instructions/recommendations by the union, the effectiveness of the union will still be limited. However, it might turn out that the members of such a union \emph{will} be committed to follow its recommendations, perhaps due to the positive feeling of being part of a movement that achieves real political goals, and therefore this version of a consumer union might turn out to be quite effective still.

Regarding the version of a consumer union which includes a GPO, it will of course first of all take some more work to establish such a GPO in comparison to the simpler version. But this effort might be rewarded if the union can then offer economic advantages specifically for its members as this might increase the general interest of consumers to join the union. 
%Additionally, the negotiating power of this type of union might also be increased since it allows the members to make more binding commitments to spend part of their budget for consumption through the GPO. The GPO could thus allow its members to choose a minimum amount of money that they will spend through the GPO each month, ... which could/would strengthen the GPO's potition. 
It might also increase the union's negotiating power since it could allow the members of the union to make more binding commitments to spend part of their budget for consumption through the GPO. A consumer union which includes a GPO might thus get more leverage over
 the companies thereby.


A potential critique of these types of unions is that some consumers might not want to restrict their choices thus. And indeed, there are people who are very specific when it comes to the brands that they go for. But for many people, the brands themselves are often not that important; only the specifications of the products or services, the prices, and perhaps some political considerations as well, such as environmental impact or working conditions for the workers. So as long as there are competitive alternatives to a certain product or service that a company provides, which there usually is, most people would probably not complain if the union requires them to go for some of these alternatives instead. 

Furthermore, another potential benefit of a consumer union is the fact that it might be able to help the members analyze and test products and services on the market. The union could do this either by gathering and analyzing feedback from its own members or by employing researchers to analyze products and services on behalf of the members, or a combination of both. 
This could yield a more efficient way for the members of the union to analyze products and services as a group, rather than letting it be up to each individual to do their own market research. So in addition to perhaps being able to obtain better prices for goods and services, the union might also help ensure that the quality of the goods and services meets the expected standards. This might prevent a lot of frustration for the members, e.g.\ of experiencing buying a certain product only to see it break right after the warranty has expired. The union might also help the members gather information about companies in terms of their operations as well as that of their suppliers. This might save a lot of work for members who want to make good political choices as consumers, such as not supporting companies who have a bad environmental impact, who use sweatshops or child labor, etc.


\section{Final remarks}


Let us conclude this whitepaper with the following remarks. In regards to the idea where the consumers come together and leverage their collective buying power through a GPO, one can ask: If GPOs are successful for businesses in order to negotiate better deals from other businesses, why could the same concept not also be successful as a way for consumers to negotiate better deals for themselves?

And in regards to the overall idea for consumers to gather in unions in order to use their future consumption as a political bargaining tool, let us consider the vast amount of combined effort that people already put into other political organizations and unions. This effort for instance includes being politically active in these, arranging and taking part in demonstrations, communicating ideas and beliefs to others and engaging in political discussions, in real life and online. Even though most individuals probably do not expect that their own effort will lead to a big political change by itself, the idea is often that `every little thing counts.' Well, if indeed every little thing counts, then there is absolutely no reason not to give the subject that this whitepaper has discussed further consideration, and thus look further into whether unionizing as consumers could indeed lead to greater political power and influence for various groups of people.




%\section*{navi.}


%Disp:
%- Gentag, at der er meget på spil for firmaerne, og pointer at meget af samfundets værdier nok afhænger af forudsigelser omkring forbrugernes fremtidige forbrug.
%- Og når man så tager den simple version, hvor forbrugerforeningen bare opforder medlemmerne til aktion på baggrund af politiske grundlag, så kan man spørge, hvorfor ikke? Det bør ikke koste vildt meget, og som sagt er muligvis meget magt for forbrugerne at hente her.
%- Angående forbruger-GPO-versionen af idéen, så kan man sige det samme. Og her er der altså en mulig fordel i, at sådan en forbrugerforeningen vil være effektiv i at tiltrække en stor mængde medlemmer. Det kræver selvfølgelig lidt mere forarbejde, men på den anden side.. Hm, skal dette være dispositionen, eller?.. ..Nej, lad mig bare sige, at det kræver selvfølgelig lidt mere forarbejde, men måske er det dette forarbejde værd. 
%- En mulig kritik kunne være at pointere, at forbrugerne måske ikke vil være så glade for at skulle begrænse deres forbrug. Men.. ..Men på den anden side kan det være at nogle forbrugere vil finde det befriende, at de så skal træffe færre beslutninger. Og hvis GPO'en tillader forbrugerne selv at vælge.. Hm.. (Nå ja, måske skal jeg lige overveje, om GPO'en bare ville opdele det i områder, eller om forbrugerne skulle bruge det mere som en online store... Hm ja, det forklarer også lidt, at jeg har følt at der har været en ting i dag, som jeg manglede at overveje.. ..Ja..) ...(18:01) Jo, nu ved jeg, hvad jeg vil skrive. Jeg kan faktisk bare tilføje et par ret hurtige sætninger om, at man kan pointere, at det vil være begrænsende, at medlemmerne kun kan vælge specifikke leverandører, men i praksis er dette allerede ofte tilfældet, når forbrugere handler i detailforretninger; så er valget tit også truffet, da sådanne tit.. tja nej, det ville faktisk være en alt for stor (og lidt forkert) generalisering at sige.. ..Anyway, men man kan sige, at tit vil der være nok konkurrencedygtige muligheder på markedet. Og ydermere kan man så endda pointere til gengæld, at en sådan GPO sikkert vil have større ressourcer til at overveje og analysere produkterne, og dermed vil forbrugerne sikkert bare ofte kunne få større sikkerhed for, at det de køber har en høj standard til prisen, sammenlignet med når de ellers bare selv skal tage forbrugsvalgene individuelt. (18:07) ..(18:16) Jeg skal også lige pointere, at: "Hvis GPO'er kan være gavnlige for virksomheder, hvorfor skulle de så ikke kunne være gavnlige for forbrugere?"




\begin{thebibliography}{20}


\bibitem{Wright}
	E.\ O.\ Wright, 
	\textit{ How to Be an Anticapitalist in the Twenty-First Century} 
	(Verso, La Vergne, 2019).




\end{thebibliography}





























%
%
%
%\chapter{E-democracies} \label{E_democracies}
%
%The concept of a so-called `e-democracy' is not a new one. Wikipedia thus has (in the moment of writing) a whole article about the overall concept that one can read. (That article, in its current form, defines the concept perhaps a bit more abstractly than what we need for our purposes here, but it might still be helpful to glance at.) In this section, I will therefore not introduce the overall concept, but simply give some short notes on how one might implement such an e-democracy, which can for instance be used to govern a company like the ones described in the previous chapter (as its shareholders), or a political party, etc. 
%
%
%\section{A basic digital application where voters can build proposition graphs}
%
%Imagine a digital application where all voters in a given democracy (concerning e.g.\ a company, a union, a political party, etc.) can log on and build a proposition graph together, which can then define the policies of the body governed by the given democracy. We are here talking about the `graphs' of mathematical graph theory. (One can make a brief search the internet for `graph theory' to see what this is about, and one might then also want to search for `directed graphs' and `connected graphs' at the same time.) 
%
%Each node of the graph holds a proposition, which is simply expressed in plain text of whatever natural language (such as English) is appropriate for the case. 
%
%When adding a new node to the graph, one can add it by itself (i.e.\ not connected to any other nodes) or add it with at least one of two kinds of (directed) edges to an existing node. The two types of edges then represents whether the node's proposition is an elaboration on the parent node, or if it is a self-contained proposition that should, however, only apply conditioned on the parent node being active.
%
%A node becomes active if it has enough votes and if a majority of those votes are positive rather than negative. Whether `votes' are counted simply by number (such that all voters have equal power) or if the votes are weighted (meaning that some voters have more power than others) of course depends on the case. 
%
%The point of being able to `elaborate' the proposition nodes rather than having to replace it with a more detailed note instead is simply to make the work easier for everyone, and also to make the graphs easier to read. It means that the policies can be defined somewhat loosely at first (and therefore much more quickly and easily), and whenever some vagueness of the propositions is discovered subsequently, either by people studying them or because of a relevant case that reveals it, the voters can then work to specify and mend the propositions. 
%
%The point of being able to add proposition nodes that are conditional on their parent nodes being active is of course some propositions might only be beneficial to implement given that certain other ones are already in place. If a somewhat fundamental proposition node is voted inactive again, it is thus convenient that such `conditional child nodes' follow along. If the parent node is then voted active once again (or perhaps for the first time) at a later time, all the child notes that has retained a positive voting score in the meantime will then become active one again, as well as any child node whose score has become positive in that time. 
%
%The application might also allow these `conditional child notes' to have several parent nodes for convenience. 
%And the same could also apply for the `elaboration child nodes' since there might be case where it could be beneficial to be able to elaborate the interpretation when two propositions nodes are active at the same time, for instance if these two proposition have a slight conflict with each other, or if the create some other issue that needs to be handled when they are both active together.
%
%`Elaboration child nodes' should of course also depend on their parents being active. The difference between a `conditional child node' and an `elaboration child node' is therefore essentially only in the interpretation: The propositions of `conditional child notes' and those of their parents are meant to be independent of each other as statements, whereas `elaboration child nodes' are free to correct and override parts of the statements contained in their parent nodes, thus allowing these to not necessarily be absolutely precise and self-contained. 
%
%Every user should be able to add new proposition nodes and each node should also have a separate `interest score' that users can rate (with the same weights on the votes as for the first score in the case where these are weighted). A proposition node whose `interest score' exceeds a certain threshold becomes visible to all users in the main graph, and people will then have to give their votes to it, if they want to influence whether it is applied or not. 
%
%Users should thus also be able to view nodes in the graph that has not yet exceeded said threshold, perhaps by being able to select various ranges of interest scores to look at. It might also be beneficial to let such nodes expire after a certain time if their `interest score' has been low enough for too long. 
%
%Users should also preferably have their own `workbench' with enough storage capacity to hold a number of propositions nodes. If a proposition node expires, they can thus make sure that the work is not lost as long as they keep said proposition on their own `workbench.' It would probably be beneficial also if users could then have shared `workbenches' as well, where they can collaborate on making new proposition nodes. 
%
%Anonymity is of course generally very important for democracies. So it is naturally very important that no one can see which user has added what nodes, unless of course they have collaborated on it from the same `workbench.' Users should also not (for most cases of democracies) be able to see which users has voted for what. 
%%
%For cases with weighted voting, either with very few voters or with very precise weights, this might be helped further by making sure that the exact voting scores are not visible to the users, and that the can thus only see a number that is rounded to a less precise floating point number. One might also implement intervals such that new votes are always declared together in groups, some time after they have been cast individually. 
%
%
%\section{How the proposition graphs are used to govern a body}
%
%The point of building these proposition graphs is then that the leadership of the body you are governing should to some extent be required to follow the active propositions, at least within some basic limitations of they can be required to do. 
%
%When the proposition graph changes, they leadership should be required to implement these, but here it might of course be a good idea to implement some delays on when new changes are supposed to be carried out. One might thus rule that a change should only be implemented after a certain period from when happened, and only if that change has remained active in the proposition graph during that period. 
%
%If the proposition graph gets some contradictions and/or ambiguities that makes it hard for the leadership to know what to do, they should also be allowed to postpone implementing the relevant changes until the voters sorts out the issues (by which they make some new changes which restarts the acceptance process). 
%
%How to make sure that the leadership follows the democratic decisions of the proposition graph? Well, by making sure that the voters also have enough direct power over the governed body to enforce their will. This might typically be ensured by the group of voters having the power to fire leaderships and/or decrease or increase salaries, thus giving this group ``sticks'' and potentially ``carrots'' that they can use to make sure that the hired leadership does what it is supposed to.
%
%
%It has to be mentioned that a high level of transparency is an all-important part of an effective e-democracy when it comes to the body that is being governed. Luckily, one can say that as long as the voters have the aforementioned ``carrots'' and ``sticks'' at their disposal, they should at least be able to make the body more and more transparent, even if it not very much so from the beginning.
%
%
%
%%Husk:
%	% Fortolkningspolitikker (inkl. hvad man gør, hvis der er modstride) og delays. (tjek)
%	% The point with having conditional propositions.. (tjek)
%
%
%\section{A more advanced application}
%
%A basic system like the one described above is good enough for very simple cases where it is okay to just have a majority rule. But for more complex cases where there are a lot of groupings of voters with different interests when it comes to various topics, to have such a majority rule is not really sufficient. If we for instance think of the policies of a whole country, this is a good example of such a complex case, where most people probably have \emph{some} special interests that are only shared with a fraction of the society. In such a system, it is important for people to be able to \emph{negotiate} with their voting power in order to get what they want, not just to always vote for exactly what they want as individuals. One group might thus want to meet with another to make a deal where they say: ``If you vote for this and this, even though you might not be particularly interested in that, we will vote for this and this which you do have a particular interest in (even though we might not).'' 
%
%In order to accommodate these realistic needs of its users, the digital application in question should therefore also first of all make it possible for users to form groups in the system. On a technical note, having such groups can of course be implemented in a lot of ways, but I can suggest an implementation where the creators of a group start out with some divisible `moderation tokens' that give them power to decide who can join the group and who gets kicked out, and where they are then free to transfer parts of (or all of) these tokens to other users within the group at any time. This moderation system is open enough that the users can implement any other moderation system on top of this, if only they trust some central party (which can then control a user profile in the system) to enforce the results of this external moderation system.
%
%And in order for such groups to be able to start negotiating with their voting power, it is then first of all important that some overall statistics (perhaps where numbers are rounded to ones with less precision for the sake of anonymity) of how a group votes on average is made public at all times. Otherwise a group who has made a deal with another group would not be able to check that this other group holds its promises. 
%
%With this addition to the application, groups can now in principle make all the deals in private that they want to. But of course, a good implementation of an `e-democracy application' would also afford its users with ways to make these deals within the digital application, online. 
%A way to do this might be to add what we could call `conditional votes' to the system. A `conditional vote' is then a vote on a proposition, whose sign can depend on other factors. In particular, a conditional vote should be able to depend on parameters regarding the voting statistics of groups. A group that want to make a deal with another one can then decide to make a conditional vote for something the other group wants and then make the condition such that this latter group has to vote for something the former group wants to unlock the conditional votes. 
%
%On another technical note: Depending on how the system is implemented, such conditional votes might be able to cause deadlocks, where two or more conditional votes all wait for the other to turn the other way in order to turn themselves. But a way to mitigate this is to give a direction to all conditional votes which denotes the sign expected from a successful deal. The system can then continuously refresh the proposition graph by turning all conditional votes in their positive direction and then see if they fall back to the same state or if they settle on a new state, which will mean that some deadlock has been conquered. 
%
%And on a design-related note: The conditional votes can be visualized/rendered as leaves in the graphs, each one attached to a certain proposition node. The users can then create and add these `conditional vote nodes' to the system in the same way as proposition nodes are added, and then all users can decide to cast their vote for the given proposition either by casting it (unconditionally) on the proposition itself or casting it instead on one of the conditional vote nodes (meaning that their vote will now be automatically conditioned on some parameters of the voting statistics in the system, continuously, until they change their vote again). 
%
%%Another thing that a more advanced system might account for, is the fact that the power of the voters might not just have different weights but might also be dependent the area that a given proposition deals with. This could for instance be in a company or a government where there are many departments/ministries in charge of different areas. If such a company/government decided to go for a more democratic leadership, it might still want to keep some division of power within the democracy. This example is perhaps not the most realistic one so here is one that is more so:  
%
%
%Another very important thing that an advanced application should afford its users is to make sure that the voters can choose representatives. It might seem odd to want to implement a direct democracy only for people to end up choosing representatives once again, but is indeed exactly what a direct democracy should aim for. It is nowhere near feasible if the system requires all users to engage in all discussions and decision making in order for the democracy to work, not unless we have a simple case with relatively few voters who are all quite engaged. But in most cases that one could think of, being able to choose representatives and trust these with looking into specific and/or complicated matters and vote on the person's behalf is all-important. The problem with representative democracies that a direct democracy aims to fix should thus not be to get rid of representatives, but simply to ensure that people can change these much more rapidly should they want to, and also that any voter can always choose to look into specific matters themselves and choose to vote differently on those than how their representative has voted.
%
%An advanced e-democracy application should therefore also allow users to choose representatives. With `conditional votes' implemented, users can of course in principle just cast conditional votes on all propositions that they want representatives to decide for them, but this is too cumbersome and we can do much better than that. The application could thus first of all allow the voters to give their votes to others. But it is likely that some users will only trust a certain representative to decide for them in a certain area of concern. And in general, users will therefore probably want to be able to have multiple representatives at a time, each responsible for making decisions for the voters in specific areas. 
%
%I therefore propose that an advanced e-democracy application also implement what we could simply call `areas of concern,' which are then essentially groupings of propositions regarding a certain subject. Whenever a new proposition is added by itself (as what we could think of as the ``root'' of a connected graph) it should thus be given an area of concern such that the application can group it with proposition graphs with the same area of concern. And whenever a child node is added, it should of course get the same area of concern as its parent. With this implemented, users should then be able to give their votes to another user (i.e.\ representatives) when it comes to any specific area, which should then effectively mean that the user will automatically cast the same vote as their representative, at least when it comes to propositions that the user has not voted on themselves. (And one might then implement different settings to this, such that a user for instance might be able to even let a representative override the user's own previous votes.) 
%%(If someone creates an otherwise relevant proposition node but adds the wrong area of concern, one can expect that it will then simply not be voted forth (over the aforementioned threshold), not until the author gives it the right area of concern.)
%
%It now almost goes without saying that each `group' in the system can then potentially choose to have their own specific representatives that the members are recommended (or perhaps required in some special instances) to use. 
%
%The system might also implement `subareas,' such that any user can try to add one such to any proposition node. Users should then be able to vote such subareas in and out, and if one is voted in, the proposition node and all its children will then get this extra area, that users can then also choose to sign representatives to. With these `subareas' implemented, this then allows users to delegate different representatives to these, even if they are also part of the same overall area of concern. **(This paragraph will probably need some rephrasing.)
%%(One could also implement `subareas' simply by requiring that these are also added from the beginning when the relevant proposition nodes are created, but it might make it easier for the users if they can just change the subareas by vote at any later time (instead of having to recreate and substitute the whole subgraph, with similar nodes but with updated subareas).)
%
%
%%These `areas of concern' also allows for something else that might be very useful, namely that the same community of users/voters can govern a variety of bodies with the same overall (potentially disconnected) proposition graph. 
%%Because when the contracts and/or agreements concerning the various bodies' commitment to follow the proposition graph are external to the digital system, a body might as well agree to only be ruled ...
%%The usefulness in this, apart from maybe having everything gathered in one place, is that this will mean that voters
%%(16:06, 06.11.22) Hm, dette kan nok godt blive mere kompliceret, fordi flere foreninger så skal blive enige om, hvordan stemmerne fordeler sig, og så skal det lige pludseligt topstyres på en helt anden måde.. Så lad mig lige se en gang... ..Hm, men handler det så ikke bare om, at forskellige grupper skal kunne "genbruge" de samme propositionsgrafer, og også om at de andre gruppers aktivitet så også godt må kunne gøres synlig i samme propositionsgraf (altså for en vis gruppe, der bruger denne)..? (16:09) ...(16:29) Jo, men så har det så ikke rigtigt så meget med subareas at gøre.. ..Nej, for så skal man også simpelthen gøre, så at graferne.. er helt adskilte, ja, så måske giver det altså slet ikke mening.. hm, andet end at man stadigvæk kunne have graferne side om side, og mere vigtigt, at conditional votes så også kan komme til at afhænge af eksterne grafer.. Ja, er det ikke bare det..?:) 
%
%
%%Furthermore, each group should have its own page and/or own `area of concern,' where the members of that group have all the voting power. This is useful since it means that each group can then build their own proposition graph over its policies and opinions. A group might then also signal external actions via this proposition graph. For instance, a group the represents a workers union might create conditional votes in their own proposition graph that depends on some statistics regarding the main proposition graph.. 
%%say that: ``On these
%
%
%And lastly, it might be beneficial for various groups in society who govern different bodies, e.g.\ political parties, unions, organizations, companies, to also be able to negotiate with each other online and to view each other's policies. For instance, a company might want to say to a governing political party that: ``if you implement certain laws, we will move out company elsewhere.'' And a trade union might then say to a company: ``if you do not give us higher salaries, we will go on strike.'' These are thus examples where a group in society can use their power over one body (including simply themselves as a group) to negotiate concessions from another group with power over a different body. 
%So if the advanced e-democracy application really wants to afford its users with all that they could want for negotiating effectively with each others, it should allow different voter groups to come together in the same space. First of all, each `group' in a e-democracy should have their own proposition graph that only they have voting power over. This local proposition graph can then be used to signal the groups policies, opinions and potential actions. And when it comes to the `conditional votes' in this local proposition graph, these should also be allowed to depend on statistical parameters in the main proposition graph, outside of the local one. 
%And furthermore, different e-democracies (governing over different bodies) that uses this same digital application should also be able to invite another group to join together, such that the two e-democracies can have their proposition graphs shown side by side (but with the same distribution of voting power in each of these graphs), and more importantly, that one e-democracy can then start making conditional votes that depends on statistics regarding the other e-democracy and vice versa. 
%
%
%\section{An e-democracy party}
%
%There are of course a lot of different examples where an e-democracy such as this could be useful; political parties, companies, unions, organizations and other communities. %In this section, I will, however, only give some points when it comes to political parties and companies of the type that I described in the previous chapter.
%%If we ..
%And when it comes to political parties, there is the natural option that these are run only by their members. We could thus imagine two or more parties competing for power, each being run e-democratically by its members. But since politically parties are typically inclusive anyway, why not just strive to have one political party where every person in the society gets an equal vote? 
%
%I believe that such a party could gain massive support over time. It might start out as a small party, especially in the early days where people are still getting used to working with the proposition graphs, and when the technology is perhaps at an early stage. And then as the technology to matures and the userbase grows, more and more people would trust the new system enough that they would want to give their vote to this `e-democracy party.' That party might then, at least in multi-party systems, get some representatives in government and by that point, the interest in the party would grow further, since all registered users would then be able to have a say in the policies of those representatives. And if the technology works, more and more people would then see the potential in an e-democracy. This is especially true in countries where the people in general do not always feel heard by there politicians: When they then see that the resulting proposition graph for most people will fit their interest better than what the traditional political parties offer, they will want for the e-democracy party to be voted in as a ruling party. 
%
%Now, if the party thus lets member of society have an equal vote in it, this might then be problematic at this stage when the party want to take over from the traditional parties, since people might then be tempted to make their vote count twice, essentially, by voting for their favorite traditional party and then also using their vote in the e-democracy party. And since voting is anonymous, there is no real way that the e-democracy party stop this. That is, apart from taking steps to balance out this effect. The e-democracy party might thus choose to temporarily break its commitment to giving all people an equal vote in this phase, and instead promise that it will commit itself to try to counter all representatives in government that are not part of the e-democracy party by giving more votes internally to a group of representatives whom it deems are exactly at the mirrored end of the spectrum than the group of non-e-democracy representatives in government. (The chosen counter-group can, however, be much larger then the group of representatives it is supposed to counter.) This way, a voter who wants the e-democracy party to take over would not be tempted to cast their votes to a traditional party instead. It also means that once the party sits on a majority of the power in government, other representatives are more likely to join it while in power (if the relevant constitution permits such migrations of representatives while in power) if they see that the e-democracy is more practical since the e-democracy party can then just remove the appropriate amount of counterbalance as these former outsiders join. 
%
%
%E-democracies as governments of countries might thus be a much closer reality in the near future, than a lot of literature on the topic seems to suggest, at least in countries governed by a multi-party system. In two- or one-party systems, the development might of course be much slower. But then again, once some multi-party governments successfully switches to an e-democracy, the two- and one-party governments would then be able to analyze and copy the technology, at least giving them a much easier route to an e-democracy, should their voters want one. 
%
%
%
%\section{Anonymity}
%
%
%As mentioned, anonymity is often very important for a democracy, especially if we think about the case of governing a country. Therefore, the digital application should allow the users to vote anonymously. This can be achieved letting each user control an anonymous profile, but if information about which user has which anonymous profile is stored on a server, that server might be hacked. 
%
%So the question is, can an e-democracy system be as safe and anonymous as going to a box, drawing a cross in a field on a piece of paper and putting that paper into a box? Yes, actually: there are ways to ensure complete anonymity of the users where the anonymity is preserved even if the servers of the application is hacked.
%
%The following protocol allows a set of clients to each provide a server with a set of public keys such that each client knows the private key of exactly one of the keys in the set (and no one else but them knows this private key) but where no one knows which public key belongs to which client apart from the clients knowing their own key. The protocol is furthermore resistant to DoS attacks. 
%
%It works by having the clients take turns building blocks in a block chain, which we can think of as a `block spiral,' where the clients form a circle and where the turn to provide a new block to the spiral goes around in the circle. 
%
%\ldots\ \textit{Okay, jeg tror lige, jeg venter med at forklare om min idé her, for det kan godt være, at der findes en lidt nemmere måde. Det vil jeg lige tænke over. Men ellers er det en god idé, altså den hvor hver klient sender nogle nøgler videre til en tilfældig anden klient i kredsen (hvor hver blok krypteres med den næste klients offentlige nøgle (fra begyndelsen) og sendes til denne), og hvor klienter, der modtager nøgler gerne skal sende dem videre og slette dem fra hukommelsen. Herved vil man meget sjællendt kunne se, hvem var den oprindelige sender af en nøgle (medmindre både modtagerklienten og klienterne for og bag brugeren er ondsindede), og selv hvis den bliver sporet tilbage kan pågældende klient bare sige, at ``den nøgle kom fra en tidligere omgang og altså fra en helt anden bruger, men jeg har altså slettet data om, hvor den kom fra, som jeg burde.'' Men ja, jeg tænker nu lige lidt mere over det, inden jeg skriver denne sektion færdig. .\,.\,Jeg har i øvrigt også tænkt mig at sige, at man efter at have brugt denne protokol så bare kan bruge et VPN herfra, men hvis man vil være endnu mere sikker, så kan man endda bruge helt den samme protokol til at indsende data om, hvordan man vil stemme med sin profil, hvor man så altså bare erstatter de (tilfældigt) genererede nøgler i protokollen med tilfældigt genereret data samt det faktiske data, man vil indsende, og til sidst så offentliggør man så bare, hvilket skrald, man har sendt ind, men ikke den faktisk data, man så lader serveren beholde. (.\,.\,Så kan det dog godt være, at man skal ændre protokollen lidt, så man lige sørger for, at hver mængde data også vil nå slutningen af protokollen, så at ingen data-klumper bliver tabt i protokollen --- medmindre der altså er sket en synlig fejl i protokollen.)}
%
%\ldots\ \textit{Nej, der er vist en nemmere protokol, hvor man vist nok også kan finde frem til en DoS attacker. Man kan vist bare have et VPN, hvor klienterne sender beskeder frem og tilbage, og hvor de så kan pakke en nøgle ind i flere krypteringer med forskellige nøgler, hvor beskeden så skal sendes til alle de klienter i rækkefølge, som kan dekryptere beskeden en efter en. Og hvis så man gør det tilfældigt, hvor mange krypteringer, der skal til, så kan ingen igen vide, om en nøgle kom fra en person, bare fordi de får opsnappet, at beskeden på et tidspunkt blev sendt fra denne, for vedkommende kunne jo sagtens have fået den fra andre og så bare have sendt den videre. Og hvis man så har nogle få DoS'ere i netværket, så kan brugere der har sendt en nøgle der aldrig nåede frem jo pege på, hvem der kan have været de skyldige (af den række af brugere).\,. Hm, ja, men hvis man nu vil bevise det også, så kunne disse brugere.\,. Nå nej, man kan ikke bevise det på et VPN, men det gør vist heller ikke noget. For brugere skal jo stadig gerne sende flere nøgler pr.\ protokol, og hvor de så bare opsiger alle på nær én til sidst. Og hvis der så er en DoS'er i netværket, jamen da det ikke vil være fatalt, så må det være fint nok, at brugerne bare kan page dem ud nogenlunde. (Og hvis det så bliver et større problem, så kan man altid bare bruge den mere krævende blok-spiral-protokol, jeg har haft tænkt på.)} %(08.11.22, 10:27)
%
%
%
%
%
%
%
%%Hm, jeg har fået tænkt lidt over anonymitet, men det kan godt være, at jeg lige skal tænke lidt mere. Men jeg har altså fundet på nogen fine systemer til at skjule stemmeres identitet, og jeg tænker, at stemmere generelt skal kunne vælge enhver tredjepartsbruger til at videreformidle deres stemme anonymt. Sådanne kunne så med fordel få lov at give floating point værdier (i stedet for bare 0 eller 1) med deres stemmer, eller de kunne bare råde over et antal stemmer, således at de både kan give et antal positive og et antal negative stemmer til hver proposition (men jeg tænker at det første næsten er nemmest..). Og ja, så kunne én form for sådan en tredjepart så være en organisation med fysiske lokationer, hvor medlemmerne så kan møde op personligt og ændre deres stemmer og/eller repræsentanter, og hvorved organisationen kan opdatere deres stemmer herefter med en frekvans, der kan afhænge af, hvor mange ændrer deres stemmer ad gangen over en gennemsnitlig periode. Og en anden, meget smartere;), måde at have en videreformidlingsrepræsentant på, kunne så også være.. hm, lad mig lige se.. (13:50) ...(14:30) (ordner også vask) Jo, man kan også have en videreformidlingsrepresentant, der fungerer via mindst to tredjeparter, som klienten selv kan vælge. Først er der en trejdpart, eller instans bør vi nok hellere kalde det, bare.. som via asymmetrisk krypering får en nøgle fra hver bruger, som kun denne instans og hver enkelt relevant bruger må kende. ..Ja, eller på nær at de også så skal sende alle disse nøgler til en anden instans, der heller ikke må offentliggøre dem, og som så i øvrigt ikke ved hvor hver enkelt nøgle stammer fra (og må ikke få dette af vide af første instans). ..Hm, vent, giver dette mening..? ..Ah, jo, jeg kan få det til at give mening, men lad mig nu lige se.. (14:36) ..Hm jo, denne instans nr. 2 kan så også få en offentlig nøgle med fra brugeren til hver enkelt nøgle af første instans, sådan at denne altså bare får et sæt af nøgle par, hvor den ene er en offentlig nøgle. Denne instans kan så kryptere.. Hov, nej, så behøver vi faktisk ikke den første nøgle; instans nr. 1 sender altså bare et sæt af offentlige nøgler videre (gennem en krypteret kanal) til instans nr. 2. Denne offentliggør aldrig disse, men bruger dem hver især til at kryptere en besked med en ny nøgle i, og offentliggør alle disse krypterede beskeder. Brugerne prøver så at dekryptere dem hver især, indtil de finder deres egen.. Hm, er dette får ressourcekrævende, eller skal denne instans også lige tilknytte et meget lille hash a hver offentlig nøgle med beskeden, så hver bruger ikke skal igennem så mange..? ..Det kunne man sige.. ..although.. ..Tjo, men brugerne kan så stadig downloade alle beskeder i rækkefølge og så bare nøjes med at beholde dem, de skal tjekke.. Hm, lad mig lige tænke, om ikke der er en smartere løsning.. ..Hm, men ellers var pointen så, at enhver bruger, som ikke får en passende besked, bør så anråbe dette, hvorefter alle nøgler så skal indgives, sådan så man kan finde ud af, hvilken part var synderen (inkl. anråberen, hvis dette var en fejl), hvorefter man så kan starte forfra, muligvis uden synderen. Men når hver bruger så har fået en ny nøgle, som kan kan spores hen til dem, hvis alle de involverede instanser (for man kan godt have flere nr.-2-instanser her) bryder deres løfter og offentliggør deres data (og ikke bare sletter det kort tid efter). Nu kan man så være sikker på, at alle brugere i gruppen har netop én anonym nøgle, som nu kan bruges til at oprette en anonym bruger profil for hver bruger, selvfølgelig med VPNs involveret, hvormed denne frit kan afgive sine stemmer og ændre dem, hvornår det skal være, uden at det kan spores tilbage til dem. (14:52) .. ..Og disse anonyme brugere kan så udløbe således at de skal opdateres en gang imellem, således at hvis nu nogen for lækket deres bruger, så vil det allerhøjest kun være den seneste aktivitet, der bliver lækket (og derudover kan man selvfølgelig også dele brugeren op i flere (der ikke kan kædes sammen af andre), hvis man synes, der er besværet værd, men ja, og sådan vil der selvfølgelig altid være ting, man kan tilføje, hvis man finder frem til, at det giver mening..). Nå, men selv hvis der findes et bedre system end dette, så kan jeg jo bare skrive, at det f.eks. ikke er svært at finde på systemer, hvor man via flere instanser, der hver især holder på sin del af en samlet hemmelighed (hvor alle stykker skal bruges, hvis man vil spore tilbage), kan opnå at hver bruger i en gruppe får netop én anonym bruger. Og ja, hvis man så sørger for at de udløber med jævne mellemrum.. Og at brugerne skifter.. Hm.. ..Hm, men det er nu ikke perfekt anonymitet, hvis man sammenligner med valg, hvor ingen data bliver gemt til at starte med, således at ingen nogensinde kan spore det tilbage.. Hm.. ..Hm, men kunne man ikke bare bruge en teknik, som jeg vist også har tænkt på før, hvor en instans bare offentliggør en mængde af.. Hm.. ..Hm jo, en mængde af dens egne offentlige nøgler, nemlig med et antal svarende til antallet af klient-deltagere i øvelsen, og hvor hver klient så vælger et hemmeligt ID, krypterer.. Hm, nej, lad mig lige se... ..Hm, hvad med at alle klienter bare opretter et VPN kun med demselv som noder, og så begynder at sende data rundt. På et tilfældigt tidspunkt sender hver bruger så et ID videre til en naboknude, som modtager, sender ID'et videre til én naboknude, og noterer også ID'et og modtagelsestidspunktet.. nej.. Hm, dette virker vist næsten, men ikke helt.. ..(15:21) Ah, nu har jeg det måske. Man kunne lave en kæde af krypterede blokke, hvor hver blok offentligt hører til en klient, og hver blok rummer data, som brugeren fik tilsendt af ejeren af den tidligere blok, og data som brugeren har sendt videre til næste klient. Denne blokkæde kører så på omgang i en ring, således at den tager flere runder. Og på et tilfældigt tidspunkt tilføjer hver bruger så et offentlig nøgle, som de sender videre. ..Hm, nej det er endnu ikke helt vandtæt.. ..Ah, men måske hvis man tilføjer sin nøgle i krypteret tilstand, så den først kan lukkes op, når den når til en (tilfældigt udvalgt) anden bruger.. Hm, spændende idé.. (15:27) ..Ja, man må næsten kunne lave sådan et system, hvor klienterne billedligt talt danner sådan en rundkreds, hvisker data videre til hinanden én ad gangen i rundkredsen, og hvor klienter i kredsen så kan kryptere en hemmelighed, som en klient et andet sted så kan forstå. Denne bør så med det samme kryptere en ny besked, hvormed denne hemmelighed kastes videre til en anden person. Hemmeligheden er så en offentlig krypteringsnøgle. Man slutter så, efter et vist tidspunkt, når man er næsten 100 \% sikker på, at alle brugere for længst vil have kastet deres nøgle ind i rundkredsen, og at denne er læst af modtageren. Alle brugere offentliggør så de nøgler, der har været sendt frem til dem. Herefter skal alle brugere/klienter (jeg kan ikke lade være med at skrive "brugere" i stedet for klienter, men det er vel også næsten ligeså godt..) så sige, om deres nøgle er iblandt de offentliggjorte (men selvfølgelig ikke udpege dem). Hvis antallet af nøgler passer og alle brugere/klienter siger, at deres er med i mængden, så stopper "legen" succesfuldt. ..Eller rettere, det gør den, efter at man så beder alle brugere om at slette de nøgler, de ville have brugt til at dekryptere deres egne blokke med. Og sikkerheden i systemet handler så om, at man har tillid til, at størstedelen af klienter vil gøre dette (selvfølgelig fordi de bare bruger det udleverede software til det, og ikke har bygget eller tilegnet sig en malicious kopi af denne software).. Men hvis der er for mange nølger, eller at en klient mangler en nøgle, jamen så må man så bede alle brugere om at dekryptere alle deres blokke. Og så er pointen, at man kun ved at have alle disse blokke dekrypteret, kan finde frem til, hvem der er synderen, fordi man så både vil kunne se, hvis de ikke har opfundet netop én nøgle selv, og fordi man kan se, hvis de ikke har videresendt den rette nøgle hver gang.. Nå ja, og hver bruger skal så også bare i det hele taget indsende deres private nølger, så man kan finde frem til synderen. Og hvis enten en klient nægter at indsende den private nølge i dette tilfælde, eller hvis man finder synderen ved at dekryptere alle nøglerne, så må man så udelukke denne bruger i næste tur (altså give denne karantæne). Men ja, som sagt, hvis legen derimod ender succesfuldt, så skal brugerne endeligt ikke offentliggøre deres private nøgler, nej faktisk skal de slette alle deres nøgler, der blev brugt under selve legen og kun beholde den private nøgle, som passer til den offentlige nøgle, de herved fik indsendt anonymt via legen. Og ja, så længe de fleste brugere bare gør dette, så er man ikke i fare for, at det bliver afsløret, hvilke nøgler i slutmængden hører til hvilke klienter. :) (15:53) ..(16:02) Hm, der er faktisk en lille smule hangman's paradox tilstede i denne løsning, men det kan man vist gøre bod på ved bare at sige, at hver knude.. Hm.. ..Hm, eller hvad i stedet med bare at gøre sådan, at klienter i kredsen generelt skal vente et tilfældigt antal omgange, inden.. hm, men det løser dog ikke problemet eksakt.. (16:07) ..Hm, men jo, man kunne vel også bare sørge for, at sandsynligheden for at ens software sender en nøgle starter virkeligt lille og kun vokser over mange runder, og så kunne man gøre sådan, at hvis en bruger bagefter kan se, at deres software har sendt.. Hov, vent, dette er da slet ikke et problem, netop fordi man kaster hemmeligheden frem i rækken.. hm.. ..Hm, der skal kun tre (specifikke) andre brugere til at afsløre en i denne løsning, men de kan det kun hvis man har været uheldig at softwaren har sendt ens nøgle tidligt.. ..Hm, man kunne også bare give hver bruger mulighed for at afbryde legen, hvis deres sofware har sendt deres nøgle tilstrækkeligt tidligt.. Hm.. ..Hm, i øvrigt kan man hurtiggøre processen, hvis kredesn har mange kæder i gang på én gang, så alle klienter kan bygge en blok i hver runde (nemlig hvis der er ligeså mange kæder i gang, osm der er klienter i kredsen).. ..Hm, men kan man ikke bare generere flere nølger, end der er behov for..? (16:18) ..Jo, og så kan brugerne/klienterne til sidst bare vælge, hvilken nøgle af dem, de har fået genereret i legen, de vil beholde, ved at.. Hm.. ..Ah, ved selvfølgelig bare at bekende offentligt bagefter, at "disse nølger var mine, men jeg skal ikke bruge dem alligevel."!:) Og hvis så der lige præcis bliver det samme antal efterfølgende, som der er klienter, og hvis alle meddeler, at de har en nøgle iblandt de endelige, så når man i mål, og ellers må man så bare til at optrævle kæden, for at finde DoS-synderen, hvis ikke legen ender som den burde. :) (16:25) Og ja, det skal så bare anbefales, at hver bruger ikke vælger en nøgle, der blev genereret helt i starten af systemet, men ved at det stadig er brugerens beslutning at udvælge den ønskede nøgle, så eliminerer man altså hangman's paradox.:) (16:26) ..Nå ja, og lad mig lige præcisere, at hver blok så skal indeholde en liste af krypterede nøgler (som hver er kryperet med en tilfældig andens offentlige nøgle), og denne lister vokser altså bare.. tja, eller man kan måske begynde at fjerne ting fra bunden af listen efter et vist stykke tid, når det er sikkert, at samme nøgle er blevet indsat igen i ny version (nemlig ved at en knude har dekrypteret og re-krypteret nøglen og sat den på). Og man kan så kræve, at hver knude tilføjer netop én ting til listen i hver runde.. how, "runde" er et dårligt term at bruge for hvert enkelt lille step, når vi har en rundkreds, så lad os kalde.. tja, lad os bare kalde det enten hver 'step'/'skridt' eller hver tur.. nej, lad os udelukkende kalde det 'skridt'/'step.' Og hvis en bruger så modtager flere beskeder på én gang i et step.. ..Hm, nej vi kan også godt kalde det turn i stedet (for så tænker man jo bare på et lille turn af hjulet).. Så må denne bruger så altså gerne vente en omgang med at sende nummer 2 besked (osv., hvis der modtages flere end to), og altså så kun videresende én af nøglerne i den første tur, hvor nøglerne modtages. Ok, så det var vist bare det, jeg lige skulle præcisere.. (16:40)
%
%%(16:42) Nå, men der er også et andet issue, jeg skal tænke over, og det handler om: Vil det ikke være for fristende for folk at stemme på deres vante repræsentanter i et regeringsvalg, hvor et e-demokrati kæmper, og ser ud til at kunne vinde? For hvis man gør dette, så vil man vel kunne få dobbelt magt, medmindre e-demokratiet kan se, hvem der ikke stemte på det.. hvad de jo ikke vil kunne.. Hm, måske er dette et ret stort problem, men ja, nu vil jeg altså give mig til at tænke godt over det... (16:44)
%
%%(31.10.22, 9:21) Kort efter, jeg klappede i i går kom jeg frem til, hvad vist også havde været oppe at vende i periferien af mine tanker tidligere på dagen, at den simple og måske eneste løsning nok bare er, at sørge for, at e-demokrati-partiet i starten også har til opgave at booste stemmevægte inde i systemet (på en helt transparant måde selvfølgelig), således, at alle repræsentanter, der ellers har fået mandater udover partiet, de får en modvægt til sig inde i partiet. På den måde kan det ikke betale sig at stemme uden for partiet for at pågældende mening skal få mere magt, for så vil den pågældende mening bare blive countered. Og ja, det er så partiets opgave at finde frem til og være ærlig omkring, hvad der er midten af det politiske spektrum i henhold til forskellige punkter, således at man kan counter'e et vist mandat ved at give mere magt til en (eller flere) fra den modsatte (i.e. spejlede) ende af spektrummet. Når partiet så er i regering, så kan man så også bede de repræsentanter, der ikke er med, om at joine, for så vil e-demokratiet bare fjerne magt igen fra dem, der står for at counter'e/udbalancere magtbalancen.. (9:29)
%
%
%%\section{A note on transparency}
%
%
%\section{E-democracies in companies}
%
%%To finish this chapter, let me just make a small point about how an e-democracy application like this might also be incredibly useful when it comes to democratically run companies, or indeed the almost-democratically run `Economically Sustainable' Companies (ESCs.\,. hm, that looks a lot like `Escape(s)'.\,.) that was described in Chapter \ref{MSE}. 
%%
%%If the company in question has a goal of expansion, such as should be the case for the 
%
%To finish this chapter, let me just make a small point about how an e-democracy application like this might also be incredibly useful when it comes to democratically run companies, or indeed the almost-democratically run `economically sustainable' companies that was described in Chapter \ref{MSE}. 
%
%If the company in question has a goal of expansion, such as should be the case in general for the `economically sustainable' companies as described, I envision that this venture will be all the more exciting for the participants if there is a vibrant online community that engages in discussing and finding what strategies to go ahead and try in order to expand the company. 
%
%And if this e-democracy application can be as useful a tool for this as I believe it can, it could thus accelerate the interest in taking part and supporting such a company immensely. 
%
%%Hm, skal jeg så bare stoppe her for nu? (Eller skal jeg skrive videre på denne sektion, og var der i øvrigt andet, jeg har glemt at nævne..?) (15:21) ..Hm, jeg har glemt at nævne min pointe omkring gennemsigtighed ved at sørge for, at folk med jævne mellemrum bliver udtaget til at sætte sig ind i detaljerne og så rapportere tilbage til den interessegruppe, der udvalgte vedkommende, men måske jeg bare skal gemme denne pointe til en anden gang.. (15:23)
%%...(16:01) Nej, jeg tror ikke, jeg behøver at tilføje mere nu. Når jeg så lige får tænkt lidt mere over spiral-protokollen, så kan jeg skrive om den, og ellers er det nok bare lige at redigere teksten. (16:02)
%
%
%
%%Husk:
%	%Jeg havde tænkt mig her at nævne det med, at det kan være smart at udvælge nogen (som så jo kan vælges til at være upartisk og/eller repræsentativ (men måske smart/intelligent nok)) fra en gruppe til at studere og gennemgå systemet i nærmere detaljer og så rapportere tilbage..
%	%Transparancy.
%	%
%	%You never have to waste a vote (and never have to fear wasting a vote). And never have to be fearful, that who you voted for does something you didn't expect (since this system requires no trust in representatives, at least not except in cases why you don't feel like you have the time (or interest) to go through the details of a matter).
%	%..(16:47, 29.10.22) Hm, og husk det her med at man kan have flere områder, hvor forskellige bestemmer, og at dette så også gør, at andre grupper kan logge sig på i systemet, hvor vi snakker om at styre et land. I et sådant e-demokrati kan grupper altså også tilføje områder. På den måde kan de gøre det offentligt for alle, hvad de har tænkt sig. Hermed kan vi altså få en stor markedsplads, der handler om at lave aftaler og bestemmelser, både i regeringen, men også i andre instanser (det kunne f.eks. være såsom fagforeninger, hvilket jo vil være meget relevant i den sammenhæng). ..Og ja, det kan også være grupper, der egentligt ikke har nogen anden magt over noget, men som alligevel vil oprette et område, der hedder "vi mener sådan og sådan, og vil vil gøre sådan og sådan," altså et område, hvor de kan signalerer til omverdnen, hvad deres interesser er, og hvad de gerne vil / er parate til at gøre. (16:54) ..Og ja, det kan så nævnes, at dette så også kan være sådan noget som at trække sig fra den overordnede gruppe (f.eks. e-regeringspartiet eller trække sig som kunde og/eller investor i et firma). 
%	%Jeg kunne godt nævne muligheden i "forbrugerforeninger" kort også (som et eksempel på anden form for magt), men så tilføje, at min kd.v.-idé så netop nok ville være endnu bedre her, for så kan man undgå sådanne reprimanter (eller hvordan det staves). Men om ikke andet kunne det så blive en måde at tvinge gang i en kd.v., hvis nu virksomhederne indenfor en branche er tøvende med det. 
%	%Jeg skal forresten huske at have område-repræsentanter med under avancerede punkter, sammen selvfølgelig med områderne selv. Jeg kan således nok godt nævne "områderne" først, også selvom det egentligt er vigtigere, det med at kunne vælge repræsentanter.. 
%	%"Det handler om at det bliver: meget lettere at samle sig i små grupper, og meget lettere at sætte i gang i en proces, hvor man overvejer, om ikke der kan gøres noget ved et forhold, netop fordi man bare kan starte denne diskussion i nogle små grupper (som så kan kontakte andre grupper, små eller store, når de har fået samlet en oversigt over, hvad problemet er, og hvad man kunne gøre for at løse det m.m.). Så altså langt større tilgængelighed for den enkelte og dermed mange mange flere mennesker aktiveret ad gangen (som så overvejer og finder på løsningsforslag til problemer i samfundet (ofte særlige problemer for nogen specifikke i samfundet, men det kan jo også være mere almene)). Og så vil der så derefter også kunne være meget kortere tid til, fra løsningsforslag til løsning i sådan et direkte demokrati, der er klart. Og ikke mindst vil folk (i grupper) få langt nemmere mulighed for at indgå selv komplicerede politiske aftaler med andre folk (i grupper (ikke nødvendigvis disjunkte med de første, btw)), således at man får et meget bedre og hurtigere kan få handlet sig til at få opfyldt sine behov som en gruppe af mennesker, og således at smafundet derfor vil blivet meget bedre fintunet, så at sige, til at opfylde så mange menneskers forskellige behov som muligt på en gang."
%	%(15:01, 01.11.22) Jeg skal huske noget, jeg lige fik tænkt på, og det er, at et sådant demokrati kan få en meget meget fladere struktur, hvor at man, når man har en ny idé til forandring, lad os sige som lille gruppe, i stedet for så at skulle indsende og ansøge om idéen til en central, så kunne man i første omgang dele den, med den/de mest relevante nabogruppe(r). Hvis de så også er med på den, så kunne man så brede det til endnu flere. Og når idéen så har samlet nok opbakning, så kan man melde det til det brede fællesskab, hvor idéen så allerede har opbakning, når den ansøges om. Jeg ved godt, at sådanne måder at fremføre idéer på allerede finder sted mange steder, men jeg tror, at man i et e-demokrati kunne gøre den fremgangsmåde endnu nemmere og endnu mere hyppig.. Hm, måske vil jeg skrive om dette, men om ikke andet er det da bare rart at tænke på, at der kunne blive sådan en rigtig flad struktur, hvor relaterede grupper selvstændigt kan diskutere og handle om, hvilke idéer og forslag, man vil gå videre med..:).. (15:08)
%	%Man kan også bruge min blok-spral-idé til når stemmerne skal kastes..!
%
%
%
%
%
%
%%(09.11.22, 9:46):
%\section{(I'm considering adding something like:) A similar application for scientific discussion}
%
%\textit{I have now realized that this application could also be used for scientific discussion graphs, which goes hand in hand with decision making since facts are of course important when deciding policies. In a discussion graph, on would just not really need the `conditional node' edges, but would instead just use the `conditional votes' instead --- which could then be drawn as edges between notes for this type of application. %(This all of a sudden make this idea quite a bit more interesting for me in terms of what I would like to work on myself.\,. .\,.\,Hm, hvilket er relevant for mig at have i tankerne i denne stund, for jeg skal nemlig snart til jobsøgeningsmøde med A-kassen. Og ja, med denne indsigt, så må det da næsten være denne idé, jeg vil prøve at gå videre med (og sige jeg vil iværksætte), det tænker jeg.. (..Altså i stedet for Web 2.0--3.0-idéen/erne.))
%*And it should then be very much recommended (as a key part of the idea), that users try to commit themselves to continuously update their votes for propositions as conditional ones, once more fundamental propositions are added to the system. A scientist might for instance be an expert on drugs and say (or actually ``vote'') that: ``this drug is so and so addictive,'' but then once propositions are added about the existence of relevant studies are added, as well as propositions about trust, then that scientist (along with everyone) are then strongly recommended to change the vote into a conditional vote such that the vote now depends on the study existence proposition and the trust proposition. This way (if the community follows this (strong) recommendation), every proposition can slowly become more and more founded in the basis empirical propositions/data, plus trust propositions (which are essentially propositions about how the users want to apply epistemology, i.e.\ when these propositions are also boiled down to their roots). This both has the advantage of the system being more flexible, when new studies turn up or if old ones come into question at some point, and also, importantly, it makes it easier to browse and find out what fundamental facts our more abstract facts in society are built on, i.e.\ to find the sources, and it also gives a better and easier understanding of what is interesting to research, since it shows were the ``gaps'' are, so to speak, or more precisely: where the research is thin and could use bolstering. 
%}
%
%
%
%
%
%
%
%
%
%
%
%
%
%
%
%
%
%
%
%
%%\chapter{A possible road towards Web 3.0}
%%
%%
%%**(Lad mig bare lige skrive, hvad jeg tænker nu at skrive i dette afsnit/kapitel bare ud i én køre, og så kan jeg altid redigere bagefter..) %(06.11.22, 9:46) (Jeg fik nemlig lige tænkt en del over emnet igen i går aftes, og nu synes jeg alligevel, at jeg bør kunne forklare meget af det ret kortfattet..:))
%%*(Okay, jeg har alligevel ikke tænkt mig at beholde dette kapitel, men lad mig bare lige skrive denne køre færdig, også fordi jeg har nogle små nye gode tilføjelser, mener jeg..)
%%
%%
%%\section{Everything section}
%%
%%
%%My idea for how we can reach the promises of Web 3.0, and specifically the Semantic Web, is to first implement a Web 2.0 site with an underlying semantic structure and then really try to give the users a lot of power to redesign things on the site and to program algorithms themselves. This implementation of the Semantic Web then does \emph{not} rely on XML/HTML. Instead, all semantic sentences should be recorded in relational database. 
%%
%%This is radically different from the first implementations of the Semantic Web, where metadata is simply added to various sites and resources on the web, and then algorithms in the Semantic Web would simply work by querying the web (i.e.\ the World Wide Web) and finding the necessary information online. But when all the semantic sentences (what is also called triplets in the current conventional implementations) are stored in one database, the algorithms can run way faster. 
%%
%%Essentially, one can say that the idea is to start out with a Web 2.0 site as we know them, e.g.\ such as YouTube, Reddit, Twitter, etc., and then implement the Semantic Web there. But hold on, you might say, it can hardly be called the Semantic \emph{Web}, if it is controlled by a private company. No, but if it is instead controlled by a open source organization (similar to how the web is run today, e.g.) it is another matter. Hereby it can be ensured that no one owns all the user contribution, save perhaps for the relevant user, and that any other organization can always come and take up the mantle at any time, should it be needed (just how it also is with Wikipedia).
%%
%%Alternatively, if starting this idea as a non-profit organization is slow and lacks investment, one could also start it as the type of company as described in Chapter \ref{MSE}, such that the ``organization'' can start out as a private, commercial company, but where it is guaranteed that the users will slowly become the owners. 
%%
%%But let us move on from this topic for now and assume that the organization will have plenty of funding (just like Wikipedia has). 
%%
%%
%%Let me now try to explain the overall design of a Web 2.0 site that I envision, which has an underlying semantic structure. Some of the details here are more important than others (and some are less), but it is nice to see a good example that could work (and attract many users), and then from there, I can explain why the underlying semantic structure becomes important. 
%%
%%If I were to design such a Web 2.0 site, where the intention is that it can grow into a Web 3.0 site over time, I would probably give it this following initial design:
%%
%%A main feature of the site should be a page with a category tree, which I would implement basically as a structure of tabs, i.e.\ the kind of tab system we see everywhere in the interfaces of Windows and Mac applications and so on. Whenever a new tab is selected, it will then potentially open a new list of sub-tabs. The user might thus have selected the category `movies' as a tab, and then a submenu of movie categories should open. Thus, we get a category tree (which hopefully should be pretty quick and easy to navigate as a user). Whenever a new list of (sub-)tabs is in focus, it should be expanded as a whole box of selection, in fact one might even implement it as a whole HTML page at some point (instead of just a box containing a lot of tabs). But when a tab/subcategory is selected, the box/page should nevertheless collapse into just a single bar of a horizontally adjacent tabs, where one tab is then selected. And underneath, the new subcategory selection should then automatically expand. Also, if a user has navigated down into a category tree, but want to go to a different super-category, the user can either just click on some of the visible tabs in the one of the above tab bars (which are aligned vertically adjacent, each with tabs aligned horizontally adjacent), or he/she can expand that given tab box once again.
%%
%%Okay, that was a lot of details to explain a very simple design, but it is nice to have an example to hold on to, and one that does the job. But of course, there could be many other types of design that this Web 2.0 site might start with.
%%
%%Moving on, now that we have a category tree, we should also have some resources in it, of course. So at the same time as the user selects these categories and subcategories, there should be a list of resources at the bottom that is updated in principle whenever the user chooses a new category (although the user might want to click a button manually to make the resource list refresh such that it doesn't refresh al the time while the user is navigating the category tree).
%%
%%When the user then selects a resource from the list, the user is led to a the page of that resource. That resource is then displayed pretty much at the top of that page. And how the resource is displayed then depends on what kind of resource it is, i.e.\ whether it a video or a HTML page, and so on. And each resource should then also have a list of comments below, but similarly to the all the main resources of the site, if we can call them that, these commant should also be ordered in a similar category tree. Examples of different categories of comment could be `related resources,' `user reactions,' `related discussions,' `links to source material,' and so on and so forth. 
%%
%%And to finish up this description of this basic design, there should also be a homepage where each user can see one or more lists of the user's favorite categories and resources, such that the user can quickly navigate to some of their favorite spots in the category tree (without having to start from the root and navigate down).
%%
%%Okay, that was a quick sketch of a quite basic site design. 
%%%
%%%...Jeg har fået tænkt lidt mere. Nu ved jeg faktisk bedre, hvad man skal sige gælder for de rating-tal, der skal følge med sætningerne/tripletterne. Jeg tror ikke, det vil tage mig lang tid at færdiggøre denne hurtige udredning herfra, men lad mig lige se, om jeg lige vil skifte emne lidt og skrive på noget andet, eller om jeg vil holde en lille pause.. 
%%%...Okay, jeg prøver at skrive færdigt..
%%%
%%Now I can get on to some of the stuff that is actually interesting.
%%
%%Assuming that the reader knows about the Semantic Web (and about triplets and so on), the reader might have already guessed that the categorization of the resources should then of course be user-driven. The users should thus be able to say for instance: ``this resource belongs to this category,'' and thereby be able to vote resources into various categories. Note that ``this resource belongs to this category'' can be implemented as a triplet. The users should also be able to say ``this category is a relevant subcategory to show under this other category.'' The users should thus also be in control of the category tree --- and of the category trees under each resource (where one can reuse resource category trees for similar types of resources).
%%
%%So far so good: One thus get a Web 2.0 site where the categorization structure is semantic and user-driven. And because it is semantic, all the user data can easily be reused in for other similar sites, and specifically also for other implementations of the site in question.
%%
%%If such a Web 2.0 site can become more and more popular, and if it is run by an open source organization (and/or community) as mentioned, this site might thus effectively become all that people hope for in terms of what Web 3.0 might bring.
%%
%%Okay, at this point I have explained the overall idea, and also explained an overall type of implementation that could be the starting point for a Web 2.0 site that thus aims to become, what we could call af Web 3.0 site (bringing forth the features that people hope for in Web 3.0). Now I will move on to the \emph{really} intersting stuff, because I actually have a few idea that i believe can make such a ``Web 2.0--3.0 site,'' as I like to call it, really take off! 
%%
%%I actually believe that \emph{triplet} system will not be enough to carry forth a really useful Semantic Web (which is a big part of people associate with Web 3.0)! %(12:44)...
%%
%%First of all, it is important that the ``triplets,'' but let us actually just call them `relations' or `sentences' instead, should contain the user ID of who uploaded it, as well as a timestamp for the upload. So they should not just have the three entries. Second of all, I think it is \emph{so} important for the usability of the system, that each relation/sentence can also include a number (with whatever precision is appropriate for the case) that signifies a rating of \emph{how much} the user believes the sentance to be true. 
%%
%%This makes it possible to \ldots \textit{Okay, jeg har skrevet så meget af det her allerede, så lad mig ikke gentage alle pointerne her, nemlig da jeg nu igen har besluttet mig, at jeg alligevel bare børe vente med at fokusere på dette emne. Så lad mig i stedet bare lige ridse mine nye tilføjelser op.\,.}
%%
%%Okay, let me make this short and just mention the new thoughts that a had about this idea. The rest of the ideas, as well as the explanation of why they will be so good, can be found in my 21--22 notes (in \texttt{main.tex}, as the document is still called in the moment of writing).
%%
%%My big idea for making it easy and attractive to rate the resources on their lists, is that they can simply drag them up and down on the lists to rate them (according to the proposition that they are viewing). So when the user moves the resources in the list around, it should generate sentences/relation to the database, where the rating number in these relations are determined by where the user drops the resource in the list (and where only the most recent adjustment applies (which is what the timestamp is useful for determining)). The number might run from 0 to 1, or from -1 to 1, or whatever; that does not matter much (and the site can always change the conventions and then simply convert the previous data to such that it is scaled to the new convention). And the scale should only be very vaguely defined. The real precision that the user should worry about is how the number related to the neighboring resources on the list. So if the user believes, say, a movie to be incredible, and it has a low rating, the user might want to pull it up closer to 1. But if the question is, should it have a 0.6 rating or a 0.9 rating (assuming 1 is the highest score), that should actually only depend on the existing scores of what movies have received scores in about that interval. So the underlying rating should thus be primarily defined in relation to what resources are already rated. And then! If one wants to turn the resulting rating into one where points on the rating axis is more precisely defined, one can then just (and should be able to), upload a translation of that rating, which is basically a conversion function that takes the primary rating and converts all the numbers to the new rating. This `translation' function can then simply be defined by taking a bunch of resources, plotting them in on the list, and then use statistics after that to plot in all the other resources on that new axis (with an updated metric). And by putting a Gaussian ``error'' on all the ``fixed'' resources on this new axis, one can make sure that this process does not run into contradictions. So in short: The normal rating axis that is used when users drag and drop resources to rate them should have very vaguely defined semantics to begin with, and then the users can always translate the resulting axis from all the user activity into something with a more precise meaning, simply by defining a new metric for the axis that moves the resources into new positions on the axis.
%%
%%I also want to mention that when users drag and drop resources, they should be able to dial up and down the number of resources shown on the list as the drag and drop. Here, a setting to show few items in the list could thus only show the most `popular' items, i.e.\ such that an external predicate can be used for defining this setting, other than the predicate that orders the list. The lists should thus also have `filters,' and these filters should have different settings. And if the user can change these settings by hitting some keys, they can basically ``zoom in and out'' in the lists, namely by changing the filter dial to show fewer or more `popular' items, e.g. The user can then be ``zoomed in'' and chose a resource to rate. The user would then start draging it up or down, but since the list is long when ``zoomed in,'' the users might then hit the key to ``zoom out'' while dragging the resource. And when the user find the desired spot to drop it, the user might even not drop it right away, but ``zoom in'' a bit first to find a more precise spot for the resource. 
%%
%%Okay, I think that was it.\,. no, wait, maybe I also want to quickly reiterate something about the subcategories actually being implemented via `compound predicates,' and also that when rating such predicates, the user actually have to rate each atomic predicate individually (such that the `compound predicates' are not meant for rating, but only showing resources in a list.\,. oh, and then if the user wants to filter the list such that only resources of the one category/predicate is shown while rating resources in terms of another predicate, the user then just has to use the `filter' that I just mentioned.\,.).\,. Hm, no, I think that I have already covered these point in my 21--22 notes (in \texttt{main.tex}). So let me just stop again with this subject for now.\,. 
%%
%%\ldots\ Well, let me just mention another thing quickly, namely that the site might also use some automatically generated meta-sentences/relation, such as: ``this resource was uploaded by this user'' or ``this resource was uploaded as a comment to this resource.'' These are thus automatic sentences/relations that the site itself is responsible for applying to all uploads.
%%
%%.\,.\,Oh, and I also intended to mention something else, by the way: I wanted to mention that certain resources, e.g.\ HTML resources (or other markup), might actually get access to the database themselves. For instance, a HTML document might say ``insert a list of the top resources in this category here'' or something like that. More generally these resources might thus be able to query the database when they are viewed and change their appearance after what is contained in the database. (And this way, the site can thus also implement my so-called ``wiki idea'' from the 21--22 notes.) 
%%
%%\ldots And yeah, all that jazz about `user groups' to distribute trust, and about the user-driven filter algorithms, about rating tags, and about so on and so forth, all that is written about in my 21--22 note collection, I don't want to try to repeat these things now.\,:) %(14:29)
%%
%%%(07.11.22, 10:32):
%%\ldots\ Oh, and I of course also need to mention an important point, and that is: Sure the idea could work as an organization, but if we think about e.g.\ YouTube and Twitch, the commercial part is a big part of what makes those sites work. And by using my ``Economically Sustainable'' (ES) companies instead, the Web 2.0--3.0 site would still be able to give big rewards to the users who create popular content. And on that topic, if I were in charge of such a company, I would try to bring the average user on as soon as possible, given them some vote and some say in how the creators should get paid (how much in total, perhaps, and more importantly: how it should be distributed). I thus envision an e-democracy using the system described above for all the users, where their decisions in this e-democracy will be heard by the company, at least if it is reasonable (and in fact, the company could also be a part of the e-democracy, giving it self a significant weight on its vote, which would then make it easier for the company to keep to its promise of listening to that e-democracy). And of course, since it is an ES company, this e-democracy will be more a representation of the true power over the company, not just power that is ``lend out'' to the users, as long as their decisions does not stray to much from the company's wishes. *(This last sentence does not seem to make a whole lot of sense, but I guess I just needed to point out that if the company is what I am currently calling an 'SRC,' the users will also eventually get that power \emph{within} the company..)
%
%
%%*(26.11.22, 9:01) I mentioned movies at some point above as an example of a category of resources. That does not mean that the site then has to contain all the movies themselves; the resources can simply be reference-type resources (such as kind of movie ID, etc.). And since users should be able to control how resources in different categories are generally viewed, meaning that they can add HTML-wrappers to the resources, they can thus make it so that all movie refernce resources are viewed with potential links to site where they can be viewed or what not (the HTML code can fetch anything that is desired from the database (and/or from the web)). 
%
%
%*(20.12.22) I should also mention, that I imagine that all sentences/relations (formerly known as triplets) should be signed by a private key of the user, which is publicly associated with the (or \emph{is} the) user ID. This way you don't have to trust the particular server when it comes to who uploaded what and when. 























\end{document}