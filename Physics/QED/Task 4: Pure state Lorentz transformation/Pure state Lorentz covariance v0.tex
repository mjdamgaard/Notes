

\documentclass{article}
\usepackage[utf8]{inputenc}


\usepackage{amssymb}
%\usepackage{comment}
\usepackage{amsmath}
%\usepackage{graphicx}
\usepackage{braket}


\usepackage[margin=1.5in]{geometry}

\setlength{\skip\footins}{0.5cm}



\title{Task 4: Pure-state Lorentz covariance}
\author{Mads J.\ Damgaard}
\date{ October 8, 2022 }%\today}

\begin{document}
\maketitle




\section*{An unfinished draft for an earlier version of Section 6 in my QED paper}

In this section, we will show a heuristic argument for why the Lorentz covariance is preserved when we reduce the Hilbert space and the Hamiltonian, namely by throwing away the $\widetilde A_{0\mathbf{k}\sigma}$ and $\widetilde A_{3\mathbf{k}\sigma}$-dimensions along with $\hat H_{B03}$. 

Before we introduce the idea of the argument, let us first look at the Lorentz transformations and how we expect these to work. If we consider two inertial frames with coordinates $x^\mu = (t, \mathbf{x})$ and $(x^\mu)' = (t' \mathbf{x}')$ and want to make a Lorentz transformation of a quantum state from the hyperplane $P$ with $t=0$ to the hyperplane $P'$ with $t'=0$, then we expect to make a path integral of all paths between $P$ and $P'$. 
For the continuum limit of $\hat H_{init}$, we thus expect a Lorentz transformation from $P$ to $P'$ to be a path integral of the form
\begin{align}
\begin{aligned}
	\int \mathcal{D}A^\mu\,
		\exp\Big( i\int \mathcal{L}_{EM}(A^\mu, x^\mu) dx^\mu \Big)
		K_{IF}(P', P, A^\mu, \mathbf{\bar x}', \mathbf{\bar x}),
	\label{Lo_trans_form_01}
\end{aligned}
\end{align}
where $K_{IF}(P, P', A^\mu, \mathbf{\bar x}', \mathbf{\bar x})$ is the fermion propagator for fermions that start out with position coordinates $\mathbf{\bar x}$ in the initial inertial frame, and ends with coordinates $\mathbf{\bar x}'$ in the final inertial frame. 
The integral of $\mathcal{L}_{EM}$ is then carried out in the spacetime between $P$ and $P'$. For spacetime volumes where $P'$ is before $P$ in time, the action should just get a minus sign, and the fermions should be propagated backwards in time over that volume. 

Now, if we assume that $\hat H_{init}$ is indeed Lorentz-covariant, and that the Lorentz transformation has the form of Eq.\ (\ref{Lo_trans_form_01}), the task for this section is then to give a heuristic argument that this covariance is preserved for the reduced Hamiltonian. Since we just want to do a heuristic argument, let us forget the fact for now that the action diverges in each respective direction away from where $P$ and $P'$ intersect. Let us also simply assume that the $A^\mu$ field is parameterized by Fourier components in a continuous Fourier space of $(\omega, \mathbf{k})$-coordinates for the path integral, i.e.\ where the time coordinate $t$ is replaced with $\omega$ from a Fourier transformation. 
%
Let us take these Fourier components to resolve $A^\mu$ according to
\begin{align}
\begin{aligned}
	A^\mu(t, \mathbf{x}) = 
		\int \frac{d\omega\, d\mathbf{k}}{(2\pi)^4}\, 
		\sum_{\sigma, \varsigma}
		\sum_{\nu=0}^4
		\widetilde A_{\nu \mathbf{k} \sigma \omega \varsigma} 
		e_{\mathbf{k}}^{\nu \mu}
		f_{\mathbf{k} \sigma}(\mathbf{x})
		f_{\omega\varsigma}(t), 
	\label{A_mu_continuously_resolved}
\end{aligned}
\end{align}
where $\varsigma, \sigma \in \{1, -1\}$, where $f_{\omega\varsigma}$ is the one-dimensional version of $f_{\mathbf{k}\sigma}$, and where $e_{\mathbf{k}}^{0\mu} = (1,0,0,0)$ and $e_{\mathbf{k}}^{i\mu} = (0, \mathbf{e}_{i\mathbf{k}})$ for all $i\in\{1,2,3\}$ and all $\mathbf{k}\in\mathbb{R}^3$. 
In this continuous case, the previous normalization for $f_{\mathbf{k}\sigma}$ does not make sense, so let us simply assume that both $f_{\omega\varsigma}$ and $f_{\mathbf{k}\sigma}$ are just the plain, unnormalized sine and cosine functions for all indices for the remainder of this section. 
With this resolution of $A^\mu$, we thus assume that $\int \mathcal{D}A^\mu$ represents an integral over all the $\widetilde A_{\mu \mathbf{k} \sigma \omega \varsigma}$-amplitudes. Of course, we have infinitely many of those, but we will assume that the $\int \mathcal{D}A^\mu$ integral makes sense as some continuum limit of integrals over discrete and finite $\{\widetilde A_{\mu \mathbf{k} \sigma \omega \varsigma}\}$ sets, and that the path integral of Eq.\ (\ref{Lo_trans_form_01}) converges to a unitary transformation in this limit. It will turn out that we also need to assume that the transformation is continuous (at least at a certain point) for the following argument to work, but this is also a reasonable assumption to make for our purposes. 

The idea for the argument is then to consider the well-known gauge symmetry of the Dirac equation (which was also the original motivation for the $U$ factor in the previous section, as explained in Appendix \ref{App_motivation}). 
This gauge symmetry specifically means that if we add $\partial \lambda / \partial t$ to $\varphi$ and subtract $\nabla \lambda$ from $\mathbf{A}$, such that $(\varphi, \mathbf{A}) \to (\varphi + \partial \lambda / \partial t, \mathbf{A} - \nabla \lambda)$, where $\lambda$ is an arbitrary real function of $(t, \mathbf{x})$, then we can transform any previous solution $\psi_0$ by 
\begin{align}
\begin{aligned}
	\psi_0(t, \mathbf{x}) \to \psi(t, \mathbf{x}) = 
		e^{-iq_F\lambda(t, \mathbf{x})} 
		\psi_0(t, \mathbf{x})
	\label{gauge_sym_Dirac}
\end{aligned}
\end{align}
to get a solution $\psi$ for the transformed $V$ and $\mathbf{A}$.\footnote{
	See e.g.\ Shankar \cite{Shankar}. 
}
When we then consider $\mathcal{L}_{EM}$, which we can recall was given by
\begin{align}
	\mathcal{L}_{EM} = 
		\frac{1}{2}\big(\nabla \varphi + \frac{\partial}{\partial t} \mathbf A\big)^2 - 
		\frac{1}{2}\big(\nabla \times \mathbf A\big)^2 +
		\frac{1}{2\xi} (\frac{\partial}{\partial t} \varphi + \nabla \cdot \mathbf A)^2,
	\label{Lagrangian_EM_02}
\end{align}
the first two terms can easily be seen to be gauge-invariant. Furthermore, we can see that the last term 
is invariant as well if $\square^2\lambda = 0$, where $\square^2 = \nabla^2 - \partial^2 / \partial t^2$. If we therefore make a gauge transformation $(\varphi, \mathbf{A}) \to (\varphi + \partial \lambda / \partial t, \mathbf{A} - \nabla \lambda)$ with $\square^2\lambda = 0$ for any path in the path integral, the only difference will be that the fermion propagator gains a phase factor:
\begin{align}
\begin{aligned}
	K_F(
		P', P, 
		A^\mu,
		\mathbf{\bar x}', \mathbf{\bar x}
	)
	\to
	\exp\Big(
		iq_F \sum_{j=1}^{n}\big(
			\lambda(0, \mathbf{x}_j) - 
			\lambda'(0, \mathbf{x}_j')
		\big)
	\Big)
	K_F(
		P', P, 
		A^\mu,
		\mathbf{\bar x}', \mathbf{\bar x}
	),
	\label{gauge_transform_of_K_F_formal}
\end{aligned}
\end{align}
where $\lambda'(t',\mathbf{x}') = \lambda(t, \mathbf{x})$ whenever $(t', \mathbf{x}')$ and $(t, \mathbf{x})$ denotes the same point in spacetime. 



In order to analyze the symmetry that Eq.\ (\ref{gauge_transform_of_K_F_formal}) gives us, let us now consider a certain $\lambda$ given by 
\begin{align}
\begin{aligned}
	\lambda(t, \mathbf{x}) = 
		\sum_{\sigma, \varsigma} 
			\alpha_{\sigma \varsigma}
			f_{\mathbf{k}\sigma}(\mathbf{x}) f_{k\varsigma}(t)
	\label{particular_lambda}
\end{aligned}
\end{align}
for any particular $\mathbf{k}$, where $\alpha_{\sigma \varsigma} \in\mathbb{R}$ for all $\sigma, \varsigma \in\{1, -1\}$. It is easy to see that we have $\square^2\lambda = 0$ for all $\lambda$s of this form. 
The terms involved in the gauge transformation are then given by
\begin{align}
\begin{aligned}
	\frac{\partial}{\partial t} \lambda(t, \mathbf{x}) &= 
		-\sum_{\sigma, \varsigma} 
			\varsigma k \alpha_{\sigma \varsigma}
			f_{\mathbf{k}\sigma}(\mathbf{x}) f_{k(-\varsigma)}(t)
	\\
	-\nabla \lambda(t, \mathbf{x}) &= 
		\sum_{\sigma, \varsigma} 
			\sigma \mathbf{k} \alpha_{\sigma \varsigma}
			f_{\mathbf{k}-\sigma}(\mathbf{x}) f_{k\varsigma}(t)
	\label{lambda_derivaties_both_equations_01}
\end{aligned}
\end{align}
Let us then define four new parameters by 
\begin{align}
\begin{aligned}
	a_{\sigma \varsigma} &= 
		\frac{1}{\sqrt{2}} \big(
			-\varsigma \widetilde A_{0 \mathbf{k} \sigma k (-\varsigma)} 
			+ 
			\sigma \widetilde A_{3 \mathbf{k} (-\sigma) k \varsigma} 
		\big)
	\label{a_sigma_varsigma}
\end{aligned}
\end{align}
for all $\sigma, \varsigma \in\{1, -1\}$. 
Let also $\mathbf{a}\in\mathbb{R}^4$ be the coordinate vector of these four parameters: $\mathbf{a} = (a_{\sigma\varsigma})_{\sigma, \varsigma \in\{1, -1\}}$. It is easy to see that these four relations can be part of a unitary basis change, namely if we also define four other parameters 
$\{b_{\sigma\varsigma}\}$ 
with one of the signs in Eq.\ (\ref{a_sigma_varsigma}) flipped for their definitions. With this basis change for the path integral, we can now see that a change in the parameter space by 
\begin{align}
\begin{aligned}
	a_{\sigma \varsigma} \to 
		a_{\sigma \varsigma} + \Delta a_{\sigma\varsigma},
	\quad
	\Delta a_{\sigma \varsigma} = \sqrt{2} k \alpha_{\sigma\varsigma}
	\label{a_change}
\end{aligned}
\end{align}
for all $\sigma, \varsigma \in\{1, -1\}$ is exactly equivalent of a gauge transformation with the $\lambda$ of Eq.\ (\ref{particular_lambda}), 
namely because, according to Eq.\ (\ref{A_mu_continuously_resolved}), such a change will give us
\begin{align}
\begin{aligned}
	A^0(t, \mathbf{x}) &\to
		A^0(t, \mathbf{x}) 
		+ 
		\sum_{\sigma, \varsigma} 
			\frac{1}{\sqrt{2}} (
				-\varsigma \Delta a_{\sigma\varsigma}
			)
			f_{\mathbf{k}\sigma}(\mathbf{x}) f_{k(-\varsigma)}(t)
	\\&\,=
		A^0(t, \mathbf{x}) 
		-
		\sum_{\sigma, \varsigma} 
			\varsigma k \alpha_{\sigma \varsigma}
			f_{\mathbf{k}\sigma}(\mathbf{x}) f_{k(-\varsigma)}(t)
	\\&\,=
%		=
		A^0(t, \mathbf{x}) 
		+
		\frac{\partial}{\partial t} \lambda(t, \mathbf{x}),
	\\
	\mathbf{A}(t, \mathbf{x}) &\to
		\mathbf{A}(t, \mathbf{x})
		+ 
		\sum_{\sigma, \varsigma} 
			\frac{1}{\sqrt{2}} (
				\sigma \Delta a_{\sigma\varsigma}
			)
			\mathbf{e}_{3\mathbf{k}}
			f_{\mathbf{k}-\sigma}(\mathbf{x}) f_{k\varsigma}(t)
	\\&\,=
		\mathbf{A}(t, \mathbf{x})
		+
		\sum_{\sigma, \varsigma} 
			\sigma k \alpha_{\sigma \varsigma}
			\mathbf{e}_{3\mathbf{k}}
			f_{\mathbf{k}-\sigma}(\mathbf{x}) f_{k\varsigma}(t)
	\\&\,=
%		=
		\mathbf{A}(t, \mathbf{x})
		- 
		\nabla \lambda(t, \mathbf{x}).
\end{aligned}
\end{align}
Here we have used that $\mathbf{e}_{3\mathbf{k}} = \mathbf{k}/k$ to get the last equality. 


\section*{To be continued}
This was the first part of the argument. I have realized that there was an error in the second part of the argument that I wrote, unfortunately. so I have to correct this before I can include that. But it is basically about showing how changes in these $a_{\sigma \varsigma}$'s only leads to certain translations in both the initial and final state when viewed in the changed basis (by the $U$ factor of Sect.\ 5 in my QED paper). This means that certain translations of the initial state in the $\widetilde A_0$--$\widetilde A_3$-planes will only lead to other translations in these planes for the Lorentz-transformed state (when viewed in said basis). 

And the last part of the argument is then to use this symmetry to show that $\chi = \Psi \otimes \Phi$ states with a $\Phi$ that is a very narrow Gaussian function in Fourier space around 0 in the $\widetilde A_0$--$\widetilde A_3$-planes Lorentz-transform into states with the same property. 

Note, however, that this last part of the argument probably requires some heuristic assumptions in order for it to work. In particular, it might very well be required (for this simple, heuristic argument) to assume that we already know the property to be true for large enough $\mathbf{k}$. And it might also be a good idea to assume that the Lorentz transformation is continuous around $\widetilde \Phi = \delta(\mathbf{q}_{03})$ (in partial Fourier space, i.e.). 

But since I do not consider this task as important (at all) as my first three tasks (mentioned in Sect.\ 13 in my QED paper), I will instead focus on these first. 



\end{document}